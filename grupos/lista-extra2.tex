\documentclass{article}
\usepackage{graphicx}
\usepackage{indentfirst}
\usepackage[utf8]{inputenc}
\usepackage{amssymb}
\usepackage{enumitem}
\usepackage{color}
\usepackage[a4paper, margin=0.5in]{geometry}
\begin{document}
	
\title{Lista Extra 2 para Entregar}
\author{Vinicius Fernades}
	
\maketitle
	
\begin{enumerate}
	\item Mostre que se $(G,.)$ for um grupo, então o centro dele $Z(G)$ é um subgrupo normal.
	
	\textbf{Solução:} Sabemos que o centro de $G$ é definido por $Z(G)=\{g \in G: g.a=a.g \; \forall a \in G\}$, assim é evidente que o elemento neutro $e \in Z(G)$ pois comuta com todos os elementos de $G$. Devemos mostrar que $\forall a, b \in Z(G)$, então $a.b^{-1} \in Z(G)$. Com efeito: tomando $g \in G$ temos $a.b^{-1}.g = a.g.b^{-1}= g.a.b^{-1}$, portanto $a.b^{-1} \in Z(G)$. Portanto o centro de $G$ é um subgrupo. Por definição, um subgrupo é normal se, e somente se, suas classes laterias, esquerda e direita, são coincidentes, mas como os elementos de $Z(G)$ comutam com todos os elementos de $G$, então $\forall g \in G; \; g.Z(G) = Z(G).g \Rightarrow Z(G) \vartriangleleft G$, como desejávamos.
	
	\item Seja $G$ um grupo e $\mathcal{I}(G)$ o grupo de automorfismos de $G$. Mostre que $G/Z(G) \cong \mathcal{I}(G)$.
	
	\textbf{Solução:} Definindo $*_{g}:G \to G$ tal que $*_{g}(a):=g.a.g^{-1}$. Afirmo que esse operador é um homomorfismo, pois $\forall a, b \in G$ temos $*_{g}(a.b) = g.(a.b).g^{-1} = g.a.g^{-1}.g.b.g^{-1} = (g.a.g^{-1}).(g.b.g^{-1}) = *_{g}(a). *_{g}(b)$. Além disso é injetor pois $*_{g}(e) = g.e.g^{-1} = e$, e é sobrejetor, pois $\forall a \in G, \;   \exists b\in G; \; *_{g}(b) = a$. Tomando $b= g^{-1}.a.g$ temos $*_{g}(b) = g.b.g^{-1} = g.(g^{-1}.a.g).g^{-1} =a$, portanto é um sobrejetor, consequentemente é uma bijeção. Como é homomorfismo bijetor, é um isomorfismo.
	Sabemos de exercícios anteiores que o conjunto dos isomorfismos entre grupos forma um grupo com a operações de composição, portanto $(\mathcal{I}(G), \circ)$ é um grupo. Com isso, temos o operador $*:G \to \mathcal{I}(G)$ dado por $g \mapsto *_{g}$, definido anteriormente, é naturalmente sobrejetor, por construção. E por fim, é um homomorfismo entre grupos, pois $\forall a, b, c \in G$ temos $*(a.b):= *_{a.b}$, então: 
	$$
	*_{a.b}(c) = (a.b).c.(a.b)^{-1} = a.(b.c.b^{-1}).a^{-1} = a.(*_{b}(c)).a^{-1} = *_{a}(*_{b}(c)) = (*_{a}\circ*_{b})(c).
	$$
	Portanto, $*_{a.b}=*_{a}\circ*_{b}$, o que caracteriza um homomorfimo. Vamos mostrar que $Ker(*) = Z(G)$. Tome Suponha que $a \in Z(G)$, então $*_{a}(c) = a.c.a^{-1} = a.a^{-1}.c = c = Id(c)$, o que implica $*_{a} = Id$ e com isso $Z(G) \subseteq Ker(*)$. Por outro lado, tome $a \in Ker(*)$, assim $*_{a}(c) = a.c.a^{-1} = c = a.a^{-1}.c$, então pela lei do cancelamento $c.a^{-1} = a^{-1}.c$, e como $c$ é arbitrário, então a deve comutar com qualquer elemento de $G$, portanto $a \in Z(G)$, e por isso $Ker(*) \subseteq Z(G)$. Portanto $Z(G) = Ker(*)$. Pelo teorema isomorfismo de grupos teremos que $G/Ker(*) =  G/Z(G) \cong \mathcal{I}(G)$, como desejávamos.  
\end{enumerate}
		
\end{document}