\documentclass{article}
\usepackage{graphicx}
\usepackage{indentfirst}
\usepackage[utf8]{inputenc}
\usepackage{amssymb}
\usepackage{enumitem}
\usepackage{color}
\usepackage[fleqn]{amsmath}
\usepackage[a4paper, margin=0.5in]{geometry}
\begin{document}
	
	\title{Lista 8 para Entregar}
	\author{Vinicius Fernandes}
	
	\maketitle
	
	\begin{enumerate}
		
			\item Nos grupos abaixo encontre a equação de classes de conjugação e identifique o representante de cada classe.
			
			Vamos usar o fato de que dados $g,h \in S_{4}$ com $g=(g(1) \dots g(k))$ então podemos calcular $hgh^{-1}= (h(g(1)) \dots h(g(k)))$, e por definição $cl(g) = \{hgh^{-1}: h \in S_{4}\}$
			
			\begin{enumerate}
				\item
				$$
				\begin{aligned}
				cl(e) =& \{e\} 
				\\
				cl((12)) = 
				& \{e(12)e^{-1}, (12)(12)(12)^{-1}, (13)(12)(13)^{-1}, (14)(12)(14)^{-1}, (23)(12)(23)^{-1}, (24)(12)(24)^{-1}, (34)(12)(34)^{-1}, 
				\\ 
				& (123)(12)(123)^{-1}, (124)(12)(124)^{-1}, (132)(12)(132)^{-1}, (134)(12)(134)^{-1}, (142)(12)(142)^{-1}, 
				\\
				& (143)(12)(143)^{-1}, (1234)(12)(1234)^{-1}, (1324)(12)(1324)^{-1}, (1432)(12)(1432)^{-1}, (1243)(12)(1243)^{-1},
				\\
				& (4231)(12)(4231)^{-1}\} 
				\\
				= & \{(12), (12), (32), (42), (13), (14), (12), 
				\\ 
				& (23), (24), (31), (32), (41), 
				\\
				& (42), (23), (34), (41), (14),
				\\
				& (43)\} 
				\\
				= & \{(12), (24), (13), (14), (23), (43)\} 
				\end{aligned}
				$$
			\end{enumerate}
	\end{enumerate}
	
	
\end{document}