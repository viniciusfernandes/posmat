\documentclass{article}
\usepackage{graphicx}
\usepackage{indentfirst}
\usepackage[utf8]{inputenc}
\usepackage{amssymb}
\usepackage{enumitem}
\usepackage{color}
\usepackage[fleqn]{amsmath}
\usepackage[a4paper, margin=0.5in]{geometry}
\begin{document}
	
	\title{Lista Final para Entregar}
	\author{Vinicius Fernandes}
	
	\maketitle
	
	\begin{enumerate}
		
			\item Nos grupos abaixo encontre a equação de classes de conjugação e identifique o representante de cada classe.
			
			Vamos usar o fato de que dados $g,h \in S_{4}$ com $g=(g(1) \dots g(k))$ então podemos calcular $hgh^{-1}= (h(g(1)) \dots h(g(k)))$, e por definição $cl(g) = \{hgh^{-1}: h \in S_{4}\}$
			
			\begin{enumerate}
				\item
				$$
				\begin{aligned}
				S_{4} = & \{e, (12), (13), (14), (23), (24), (34), (123), (124), (132), (134), (142), (143), (234), (243), 
				\\
				&(1234), (1243), (1324), (1342), (1423), (1432), (12)(34), (13)(24), (14)(23)\}
				\\\\				
				cl(e) =& \{e\} 
				\\\\
				cl((12)) 
				= & \{e(12)e^{-1}, (12)(12)(12)^{-1}, (13)(12)(13)^{-1}, (14)(12)(14)^{-1}, (23)(12)(23)^{-1}, (24)(12)(24)^{-1}, 
				\\ 
				& (34)(12)(34)^{-1}, 
				 (123)(12)(123)^{-1}, (124)(12)(124)^{-1}, (132)(12)(132)^{-1}, (134)(12)(134)^{-1}, 
				\\
				& (142)(12)(142)^{-1}, 
				 (143)(12)(143)^{-1}, (234)(12)(234)^{-1}, (243)(12)(243)^{-1}, (1234)(12)(1234)^{-1}, 
				\\
				&(1243)(12)(1243)^{-1}, 
				 (1324)(12)(1324)^{-1}, (1342)(12)(1342)^{-1}, (1423)(12)(1423)^{-1}, (1432)(12)(1432)^{-1},
				\\
				&(12)(34)(12)((12)(34))^{-1}, (13)(24)(12)((13)(24))^{-1}, (14)(23)(12)((14)(23))^{-1}\} 
				\\
				= & \{(12), (12), (32), (42), (13), (14), (12), (23), (24), (31), (32), (41), (42), (13), (14), \\
				&(23), (24), (34), (31), (43), (41), (21), (34), (43)\}  
				\\
				= & \{(12), (13), (14), (23), (24), (34)\}
				\\\\
				cl((123)) 
				= & \{e(123)e^{-1}, (12)(123)(12)^{-1}, (13)(123)(13)^{-1}, (14)(123)(14)^{-1}, (23)(123)(23)^{-1}, (24)(123)(24)^{-1}, 
				\\ 
				& (34)(123)(34)^{-1}, 
				 (123)(123)(123)^{-1}, (124)(123)(124)^{-1}, (132)(123)(132)^{-1}, (134)(123)(134)^{-1}, 
				\\
				& (142)(123)(142)^{-1}, 
				 (143)(123)(143)^{-1}, (234)(123)(234)^{-1}, (243)(123)(243)^{-1}, (1234)(123)(1234)^{-1}, 
				 \\
				&(1243)(123)(1243)^{-1}, 
				 (1324)(123)(1324)^{-1}, (1342)(123)(1342)^{-1}, (1423)(123)(1423)^{-1}, \\
				&(1432)(123)(1432)^{-1},
				(12)(34)(123)((12)(34))^{-1}, (13)(24)(123)((13)(24))^{-1}, (14)(23)(123)((14)(23))^{-1}\} 
				\\
				= & \{(123), (213), (321), (423), (132), (143), (124), (123), (243), (312), (324), (413), (421), (134), (142), 
				\\
				&(234), (241), (341), (314), (431), (412), (214), (341), (432)\} 
				\\
				= & \{(123), (132), (143), (124), (243), (134), (142), (234)\}
				\\\\
				cl((1234)) 
				= & \{e(1234)e^{-1}, (12)(1234)(12)^{-1}, (13)(1234)(13)^{-1}, (14)(1234)(14)^{-1}, (23)(1234)(23)^{-1}, (24)(1234)(24)^{-1}, 
				\\ 
				& (34)(1234)(34)^{-1}, 
				(123)(1234)(123)^{-1}, (124)(1234)(124)^{-1}, (132)(1234)(132)^{-1}, (134)(1234)(134)^{-1}, 
				\\
				& (142)(1234)(142)^{-1}, 
				(143)(1234)(143)^{-1}, (234)(1234)(234)^{-1}, (243)(1234)(243)^{-1}, (1234)(1234)(1234)^{-1}, 
				\\
				&(1243)(1234)(1243)^{-1}, 
				(1324)(1234)(1324)^{-1}, (1342)(1234)(1342)^{-1}, (1423)(1234)(1423)^{-1}, 
				\\
				&(1432)(1234)(1432)^{-1},
				(12)(34)(1234)((12)(34))^{-1},
				\\
				&(13)(24)(1234)((13)(24))^{-1}, (14)(23)(1234)((14)(23))^{-1}\} 
				\\
				= & \{(1234), (2134), (3214), (4231), (1324), (1432), (1243), (2314), (2431), (3124), (3241), (4132), 
				\\
				& (4213), (1342), (1423), (1234), (2413), (3421), (3142), (4312), (4123),
				(2143), (3412), (4321)\} 
				\\
				= & \{(1234), (1324), (1432), (1243), (1342), (1423)\} 
				\end{aligned}
			$$
			$$
				\begin{aligned}
				cl((12)(34)) 
				= & \{e(12)(34)e^{-1}, (12)(12)(34)(12)^{-1}, (13)(12)(34)(13)^{-1}, (14)(12)(34)(14)^{-1}, (23)(12)(34)(23)^{-1}, 
				\\ 
				& (24)(12)(34)(24)^{-1}, (34)(12)(34)(34)^{-1}, (123)(12)(34)(123)^{-1}, (124)(12)(34)(124)^{-1}, 
				\\
				&(132)(12)(34)(132)^{-1}, 
				 (134)(12)(34)(134)^{-1}, (142)(12)(34)(142)^{-1}, 
				\\
				& (143)(12)(34)(143)^{-1}, (234)(12)(34)(234)^{-1}, (243)(12)(34)(243)^{-1}, (1234)(12)(34)(1234)^{-1}, \\
				&(1243)(12)(34)(1243)^{-1}, 
				 (1324)(12)(34)(1324)^{-1}, (1342)(12)(34)(1342)^{-1}, (1423)(12)(34)(1423)^{-1}, 
				\\
				&(1432)(12)(34)(1432)^{-1},
				 (12)(34)(12)(34)((12)(34))^{-1},
				\\
				&(13)(24)(12)(34)((13)(24))^{-1}, (14)(23)(12)(34)((14)(23))^{-1}\} 
				\\
				= & \{(12)(34), (21)(34), (32)(14), (42)(31), (13)(24), (14)(32), (12)(43), 
				\\ 
				& (23)(14), (24)(31), (31)(24), (32)(41), (41)(32), (42)(13), (13)(42), (14)(23), (23)(41), (24)(13), 
				\\
				&(34)(21), (31)(42), (43)(12), (41)(23), (12)(34), (34)(12), (43)(21)\}
				\\
				= & \{(12)(34), (13)(24), (14)(23)\}
				\end{aligned}
				$$
				Note que $cl((12))$ contém todos as transposições, $cl((123))$ contém todos os 3-ciclos, $cl((1234))$ contém todos os 4-ciclos e por fim $cl((12)(34))$ contém todos os produtos de 2-ciclos de $S_{4}$, com isso podemos concluir que para qualquer $g \in S_{4}$ teremos que $cl(g) = cl((12))$ ou $cl(g) = cl((123))$ ou $cl(g) = cl((1234))$ ou $cl(g) = cl((12)(34))$ pois as classes lateriais são disjuntos duas a duas ou são identicas.
				
				Já calculamos em exercícios anteriores que $Z(S_{4}) = \{e\}$, portanto temos a equação de conjugação:
				$$
				\begin{aligned}
				|S_{4}| = & |Z(S_{4})| + \sum_{x \notin Z(S_{4})} |S_{4}:Z(x)|
				\\
				=& |Z(S_{4})| + \sum_{x \notin Z(S_{4})} |cl(x)|
				\\
				=& |Z(S_{4})| + |cl((12))| + |cl((123))| + |cl((1234))| + |cl((12)(34))|
				\\
				=& 1 + 6 +8 +6+3
				\\
				=& 24.    
				\end{aligned}
				$$
			
			\item 
			$$
			\begin{aligned}
			A_{4} = & \{e, (123), (124), (132), (134), (142), (143), (234), (243), (12)(34), (13)(24), (14)(23)\}
			\\\\
			cl(e) = & \{e\}
			\\\\
			cl((123)) 
			= & \{e(123)e^{-1}, (123)(123)(123)^{-1}, (124)(123)(124)^{-1}, (132)(123)(132)^{-1}, (134)(123)(134)^{-1}, 
			\\
			&(142)(123)(142)^{-1}, (143)(123)(143)^{-1}, (234)(123)(234)^{-1}, (243)(123)(243)^{-1}, (12)(34)(123)(12)(34)^{-1}, 
			\\
			&(13)(24)(123)(13)(24)^{-1}, (14)(23)(123)(14)(23)^{-1} \}
			\\
			= & \{(123), (123), (243), (312), (324), (413), 
			(421), (134), (142), (214), (341), (432) \}
			\\
			= & \{(123), (134), (142), (243) \}
			\\\\
			cl((132)) 
			= & \{e(132)e^{-1}, (123)(132)(123)^{-1}, (124)(132)(124)^{-1}, (132)(132)(132)^{-1}, (134)(132)(134)^{-1},  
			\\
			&(142)(132)(142)^{-1}, (143)(132)(143)^{-1}, (234)(132)(234)^{-1}, (243)(132)(243)^{-1}, (12)(34)(132)(12)(34)^{-1}, 
			\\
			&(13)(24)(132)(13)(24)^{-1}, (14)(23)(132)(14)(23)^{-1} \}
			\\
			=& \{(132), (213), (234), (321), (342),  
			(431), (412), (143), (124), (241), (314), (423) \}
			\\
			=& \{(132), (234), (143), (124)\}
			\\\\
			cl((12)(34))
			=& \{e(12)(34)e^{-1}, (123)(12)(34)(123)^{-1}, (124)(12)(34)(124)^{-1}, (132)(12)(34)(132)^{-1}, (134)(12)(34)(134)^{-1}, 
			\\
			&(142)(12)(34)(142)^{-1}, (143)(12)(34)(143)^{-1}, (234)(12)(34)(234)^{-1}, (243)(12)(34)(243)^{-1}
			\\
			&(12)(34)(12)(34)(12)(34)^{-1}, (13)(24)(12)(34)(13)(24)^{-1}, (14)(23)(12)(34)(14)(23)^{-1} \}
			\\
			=& \{(12)(34), (23)(14), (24)(31), (31)(24), (32)(41), 
			(41)(32), (42)(13), (13)(42), (14)(23)
			\\
			&(12)(34), (34)(12), (43)(21) \}
			\\
			=& \{(12)(34), (13)(24), (14)(23)\}
			\end{aligned}
			$$
			\end{enumerate}
			Note que $cl((123)) \cap cl((132)) = \emptyset$ e $cl((123)) \cup cl((132))$ contém todas os 3-ciclos de $A_{4}$, portanto qualquer 3-ciclo $h = (h(1)h(2)h(3)) \in A_{4}$ teremos $cl(h)= cl((123))$ ou $cl(h)= cl((132))$, além disso, verificamos que $cl((12)(34))$ gera todos os produtos de 2-ciclos de $A_{4}$. Por fim, sabemos de exercícios anteriores que $Z(A_{4}) = \{e\}$, assim podemos escrever a equação de conjugação:
			$$
			\begin{aligned}
			|A_{4}| = & |Z(A_{4})| + \sum_{x \notin Z(A_{4})} |cl(x)|
			\\
			=& |Z(A_{4})| + |cl((123))| + |cl((132))| + |cl((12)(34))|
			\\
			=& 1 + 4 + 4 +3
			\\
			=& 12.    
			\end{aligned}
			$$
		\item Seja $g \in S_{5}$ tal que $g=(12)(34)$ e $H=<g>$. Encontre a ordem de $N_{S_{5}}(H)$.
		
		Notemos que $g^{2} = (12)(34)(12)(34) = e$, então $g^{n} = e$ se $n$ form par e $g^{n} = g$ se $n$ ímpar, portanto $H = \{e, g\}$. Por definição $ N_{S_{5}}(H)= \{a \in S_{5}: aHa^{-1} = H\}$, isto é, devemos satisfazer a equação $aga^{-1} \in H$ para $g=e$ e $g = (12)(34)$, então fazendo $a = (a(1) \dots a(5))$ teremos:
		$$
		\begin{aligned}
		(a(1) \dots a(5))(12)(34)(a(1) \dots a(5))^{-1} = & (a(1)a(2))(a(3)a(4)) 
		\\
		(a(1) \dots a(5))e(a(1) \dots a(5))^{-1} = & e
		\end{aligned}
		$$
		consequentemente temos o sistema $a(1) \rightleftarrows a(2)$, $a(3) \rightleftarrows a(4)$. Pois bem, como temos 5 fixado, então resta-nos 4 possibilidades para a escolha de $h(1)$ e escolhendo $a(1)$ devemos escolher $a(2)$ , deixando apenas 2 possibilidades para $a(3)$, consequentemente, teremos um total de 4.2 = 8 possibilidades para a permutação $a \in N_{S_{5}}(H)$, portanto $|N_{S_{5}}(H)| = 8$.
		
		\item Determine o centralizador e a ordem da classe de conjugação de $A \in GL_{2}(\mathbb{F}_{3})$ onde 
		$
		A = 
		\left[
		\begin{array}{cc}
		1 & 1 \\
		0 & 1
		\end{array}
		\right].
		$
		
		Sejam $B \in GL_{2}(\mathbb{F}_{3})$ tal que 
		$
		B = 
		\left[
		\begin{array}{cc}
		a & b \\
		c & d
		\end{array}
		\right]
		$
		e $\det(B) \neq 0$, ou seja, $ad-bc \neq 0$. Para que $B$ esteja no centralizador devemos satisfazer a equação $BA= AB$, resultando em:
		$$
		\left\{
		\begin{aligned}
		a =& a+c 
		\\
		a+b =& b + d
		\\
		c =& c
		\\
		c+d = d
		\end{aligned}
		\right.
		\Rightarrow
		a = d, \; c = 0 \; 
		\therefore Z(A) = 
		\left\{
		\left[
		\begin{array}{cc}
		a & b \\
		0 & a
		\end{array}
		\right]
		: 0 \neq a \in \mathbb{F}_{3}\right\}.
		$$
		Logo $a \in \{1,2\}$ e $b \in \{0,1,2\}$, portanto $|Z(A)| = 6$. Determinemos agora $|GL_{2}(\mathbb{F}_{3})|$. Seja $B \in GL_{2}(\mathbb{F}_{3})$, então podemos escrevê-la como
		$
		B = 
		\left[
		\begin{array}{cc}
		a & b \\
		c & d
		\end{array}
		\right]
		$
		onde $\det(B) = ad-cb \neq 0$. Suponha $a =0$, então $c$ não pode ser nulo pois $\det(B) \neq 0$, assim $c \in \{1,2\}$. Também não podemos ter $b=0$ pois do contrário teríamos a primeira columa como sendo linearmente independente da segunda, então $b \in \{1,2\}$, isso nos restringe as possibilidades
		$
		B = 
		\left[
		\begin{array}{cc}
		a=0 & b \in \{1,2\} \\
		c \in \{1,2\} & d \in \{0,1,2\}
		\end{array}
		\right],
		$
		portanto temos 12 possibilidades para escrevermos $B$. Aplicando o mesmo argumento para as entradas $a, b, c, d$ teremos 12.4 = 48 possibilidades para escrevermos $B$, sendo que essas são todas combinações possíveis, consequentemente $|GL_{2}(\mathbb{F}_{3})| = 48$. Por fim, podemos calcular a ordem da classe de conjugação $|cl(A)| = |GL_{2}(\mathbb{F}_{3})|/|Z(A)| = 48/6 = 8$.
		
		\item Um grupo $G$ de ordem 12 contém uma das classes de conjugação de ordem 4. Prove que a ordem do centro de $G$ é trivial.
		
		Seja $g \in G$ tal que $|cl(g)| = 4$, então temos $|G| = 12 = |cl(g)||Z(g)| \Rightarrow |Z(g)| = 3$. Sabemos que $Z(G) \subseteq Z(g)$ o que implica que $|Z(g)| = n|Z(G)|$ para algum $n \in \mathbb{N}$, isto é, $|Z(g)| = 3 = n|Z(G)| \Rightarrow |Z(G)| \in \{1,3\}$. Suponhamos que $|Z(G)|=3$, , isto é, $Z(G) = Z(g)$, então isso implicaria que $g \in Z(G)$ e teriamos que $Z(g) = \{h \in G: hgh^{-1}=g \} = \{h \in G: hh^{-1}g=g \} = \{g\}$, que é uma contradição pois por hipótese temos que $|Z(g)| = 3$, portanto devemos ter $|Z(G)| = 1$ (trivial).
		
		\item Seja $G$ um grupo de ordem ímpar e $N\triangleleft G$ tal que $|N| = 5$. Mostre que $N \subseteq Z(G)$.
		
		Como $N$ é normal em $G$ então temos que $\forall g \in G$ e $x \in N$ teremos $gxg^{-1} \in N$, portanto $cl(x) = \{gxg^{-1}: g \in G \} \subseteq N$, consequentemente podemos escrever $N = \bigcup_{x \in N} cl(x)$. Como $N$ é normal, então $\forall x \neq y \in N$ temos $cl(x) \cap cl(y) = \{e\}$, o que implica que
		$$
		|N| = \sum_{x \in N} |cl(x)| = 1+\sum_{x \neq e}|cl(x)|,
		$$
		onde removemos da somatória $|cl(e)| = 1$. Teremos as possibilidades $|cl(x)| \in \{1, 2 , 3 , 4\}$, porém, como a ordem de $cl(x)$ deve dividir a ordem de $G$, que é ímpar, então podemos descartar os valores pares, com isso teremos $|cl(x)| \in \{1, 3\}$. Caso exista um elemento tal que $|cl(x)| = 3$, então existe um $x \neq y \in N$ tal que $|cl(y)| = 1$, portanto $y \in Z(G) \neq \emptyset$, além disso, como $N$ tem ordem prima então pode ser gerado por qualquer um de seus elementos, isto é, $N = <y>$. Sejam $y^{k} \in N$ e $g \in G$, então
		$$
		gy^{k}g^{-1} = g \underbrace{y...y}_{k-vezes}g^{-1} = \underbrace{ yg }_{y \in Z(G)} y...y g^{-1} = \underbrace{y...y}_{k-vezes}gg^{-1} = y^{k},
		$$
		logo $y^{k} \in Z(G)$, e como é um elemento arbitrário, então $N \subseteq Z(G)$.
		Por outro lado, se não existe elemento $x \in N$ tal que $|cl(x)| = 3$, então teremos $|cl(x)| = 1 \; \forall x \in N$, o que implica que $cl(x) = \{x\}$, portanto os elementos de $N$ comutam com todos os elementos de $G$, com isso $N \subseteq Z(G)$.
		
		\item Mostre que um grupo $G$ com 105 elementos tem ou um 5-Sylow subgrupo normal ou um 7-Sylow subgrupo normal. Deduza que  $G$ tem um subgrupo normal cíclico $N$ tal que $[G:N] = 3$.
		
		Podemos escrever $|G| = 105 = 3.5.7$ então pelo primeiro teorema de Sylow temos ao menos um 5-Sylow subgrupo e ao menos um 7-Sylow subgrupo. Pelo terceiro teorema de Sylow, temos que o número de 5-Sylow subgrupo deve satisfazer $n_{5} \equiv_{5} 1$ e $n_{5} | 21$ sendo que as possibilidades são $n_{5} \in \{1, 3, 7, 21\}$, contudo, os valores satisfazendo essas condições são $n_{5} = \{1, 21\}$. Analogamente, temos que o número de 7-Sylow subgrupo deve satisfazer $n_{7} \equiv_{7} 1$ e $n_{7} | 15$ sendo que as possibilidades são $n_{7} \in \{1, 3, 5, 15\}$, contudo, os valores satisfazendo essas condições são $n_{7} = \{1, 21\}$. Suponha que tenhamos $n_{5} = 21$, então teremos $H_{i} < G$ onde $1 \leq i \leq 21$ subgrupos de ordem $5$ com $H_{i} \cap H_{j} = \{e\}$, consequentemente, teremos $(21-1).5 = 100$ elementos, restando apenas 5 elementos no grupo $G$, e como temos ao menos um 7-Sylow subgrupo, isso implica que teremos um total de $100 + 1 + 7 = 106$ elementos de $G$, o que é um absurdo pois $|G| = 105$, portanto $n_{5} = 1$. Analogamente, supondo $n_{7} = 15$ teremos $H_{i} <G$ onde $1 \leq i \leq 15$ subgrupos de ordem $7$ com $H_{i} \cap H_{j} = \{e\}$, consequentemente, teremos $(15-1).7 = 98$ elementos de G, restando apenas $7$ elementos no grupo $G$, e como temos ao menos um 5-Sylow e 3-Sylow subgrupos, isso implica que poderemos ter $98+1+5+3 = 107$ elementos de $G$, o que é um absurdo pois $|G| = 105$, portanto $n_{7} = 1$.
		
		Sejam $A$ e $B$ os únicos 5-Sylow e 7-Sylow subgrupos,  respectivamente, então pelo segundo teorema de Sylow podemos afirmar que ambos são subgrupos normais de $G$, isto é, $A, B \vartriangleleft G$ e $A \cap B = \{e\}$. Nessas condições, já vimos que o produto $AB < G$, além disso $|AB| = |A||B|/|A\cap B| = 5.7/1 = 35$. Definindo $N = AB$ teremos $[G:N] = |G|/|N| = 105/35 = 3$, como desejávamos.
	\end{enumerate}
	
	
\end{document}