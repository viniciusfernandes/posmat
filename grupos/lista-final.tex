\documentclass{article}
\usepackage{graphicx}
\usepackage{indentfirst}
\usepackage[utf8]{inputenc}
\usepackage{amssymb}
\usepackage{enumitem}
\usepackage{color}
\usepackage[fleqn]{amsmath}
\usepackage[a4paper, margin=0.5in]{geometry}
\begin{document}
	
	\title{Lista Final para Entregar}
	\author{Vinicius Fernandes}
	
	\maketitle
	
	\begin{enumerate}
		
			\item Nos grupos abaixo encontre a equação de classes de conjugação e identifique o representante de cada classe.
			
			Vamos usar o fato de que dados $g,h \in S_{4}$ com $g=(g(1) \dots g(k))$ então podemos calcular $hgh^{-1}= (h(g(1)) \dots h(g(k)))$, e por definição $cl(g) = \{hgh^{-1}: h \in S_{4}\}$
			
			\begin{enumerate}
				\item
				$$
				\begin{aligned}
				S_{4} = & \{e, (12), (13), (14), (23), (24), (34), (123), (124), (132), (134), (142), (143), (234), (243), 
				\\
				&(1234), (1243), (1324), (1342), (1423), (1432), (12)(34), (13)(24), (14)(23)\}
				\\\\				
				cl(e) =& \{e\} 
				\\\\
				cl((12)) 
				= & \{e(12)e^{-1}, (12)(12)(12)^{-1}, (13)(12)(13)^{-1}, (14)(12)(14)^{-1}, (23)(12)(23)^{-1}, (24)(12)(24)^{-1}, 
				\\ 
				& (34)(12)(34)^{-1}, 
				 (123)(12)(123)^{-1}, (124)(12)(124)^{-1}, (132)(12)(132)^{-1}, (134)(12)(134)^{-1}, 
				\\
				& (142)(12)(142)^{-1}, 
				 (143)(12)(143)^{-1}, (234)(12)(234)^{-1}, (243)(12)(243)^{-1}, (1234)(12)(1234)^{-1}, 
				\\
				&(1243)(12)(1243)^{-1}, 
				 (1324)(12)(1324)^{-1}, (1342)(12)(1342)^{-1}, (1423)(12)(1423)^{-1}, (1432)(12)(1432)^{-1},
				\\
				&(12)(34)(12)((12)(34))^{-1}, (13)(24)(12)((13)(24))^{-1}, (14)(23)(12)((14)(23))^{-1}\} 
				\\
				= & \{(12), (12), (32), (42), (13), (14), (12), (23), (24), (31), (32), (41), (42), (13), (14), \\
				&(23), (24), (34), (31), (43), (41), (21), (34), (43)\}  
				\\
				= & \{(12), (13), (14), (23), (24), (34)\}
				\\\\
				cl((123)) 
				= & \{e(123)e^{-1}, (12)(123)(12)^{-1}, (13)(123)(13)^{-1}, (14)(123)(14)^{-1}, (23)(123)(23)^{-1}, (24)(123)(24)^{-1}, 
				\\ 
				& (34)(123)(34)^{-1}, 
				 (123)(123)(123)^{-1}, (124)(123)(124)^{-1}, (132)(123)(132)^{-1}, (134)(123)(134)^{-1}, 
				\\
				& (142)(123)(142)^{-1}, 
				 (143)(123)(143)^{-1}, (234)(123)(234)^{-1}, (243)(123)(243)^{-1}, (1234)(123)(1234)^{-1}, 
				 \\
				&(1243)(123)(1243)^{-1}, 
				 (1324)(123)(1324)^{-1}, (1342)(123)(1342)^{-1}, (1423)(123)(1423)^{-1}, \\
				&(1432)(123)(1432)^{-1},
				(12)(34)(123)((12)(34))^{-1}, (13)(24)(123)((13)(24))^{-1}, (14)(23)(123)((14)(23))^{-1}\} 
				\\
				= & \{(123), (213), (321), (423), (132), (143), (124), (123), (243), (312), (324), (413), (421), (134), (142), 
				\\
				&(234), (241), (341), (314), (431), (412), (214), (341), (432)\} 
				\\
				= & \{(123), (132), (143), (124), (243), (134), (142), (234)\} 
				\\\\
				cl((1234)) 
				= & \{e(1234)e^{-1}, (12)(1234)(12)^{-1}, (13)(1234)(13)^{-1}, (14)(1234)(14)^{-1}, (23)(1234)(23)^{-1}, (24)(1234)(24)^{-1}, (34)(1234)(34)^{-1}, 
				\\ 
				& (123)(1234)(123)^{-1}, (124)(1234)(124)^{-1}, (132)(1234)(132)^{-1}, (134)(1234)(134)^{-1}, (142)(1234)(142)^{-1}, 
				\\
				& (143)(1234)(143)^{-1}, (234)(1234)(234)^{-1}, (243)(1234)(243)^{-1}, (1234)(1234)(1234)^{-1}, 
				\\
				&(1243)(1234)(1243)^{-1}, 
				 (1324)(1234)(1324)^{-1}, (1342)(1234)(1342)^{-1}, (1423)(1234)(1423)^{-1}, 
				 \\
				&(1432)(1234)(1432)^{-1},
				 (12)(34)(1234)((12)(34))^{-1},
				\\
				&(13)(24)(1234)((13)(24))^{-1}, (14)(23)(1234)((14)(23))^{-1}\} 
				\\
				= & \{(1234), (2134), (3214), (4231), (1324), (1432), (1243), (2314), (2431), (3124), (3241), (4132), 
				\\
				& (4213), (1342), (1423), (1234), (2413), (3421), (3142), (4312), (4123),
				(2143), (3412), (4321)\} 
				\\
				= & \{(1234), (1324), (1432), (1243), (1342), (1423)\} 
				\\\\
				cl((12)(34)) 
				= & \{e(12)(34)e^{-1}, (12)(12)(34)(12)^{-1}, (13)(12)(34)(13)^{-1}, (14)(12)(34)(14)^{-1}, (23)(12)(34)(23)^{-1}, 
				\\ 
				& (24)(12)(34)(24)^{-1}, (34)(12)(34)(34)^{-1}, (123)(12)(34)(123)^{-1}, (124)(12)(34)(124)^{-1}, 
				\\
				&(132)(12)(34)(132)^{-1}, 
				 (134)(12)(34)(134)^{-1}, (142)(12)(34)(142)^{-1}, 
				\\
				& (143)(12)(34)(143)^{-1}, (234)(12)(34)(234)^{-1}, (243)(12)(34)(243)^{-1}, (1234)(12)(34)(1234)^{-1}, \\
				&(1243)(12)(34)(1243)^{-1}, 
				 (1324)(12)(34)(1324)^{-1}, (1342)(12)(34)(1342)^{-1}, (1423)(12)(34)(1423)^{-1}, 
				\\
				&(1432)(12)(34)(1432)^{-1},
				 (12)(34)(12)(34)((12)(34))^{-1},
				\\
				&(13)(24)(12)(34)((13)(24))^{-1}, (14)(23)(12)(34)((14)(23))^{-1}\} 
				\\
				= & \{(12)(34), (21)(34), (32)(14), (42)(31), (13)(24), (14)(32), (12)(43), 
				\\ 
				& (23)(14), (24)(31), (31)(24), (32)(41), (41)(32), (42)(13), (13)(42), (14)(23), (23)(41), (24)(13), 
				\\
				&(34)(21), (31)(42), (43)(12), (41)(23), (12)(34), (34)(12), (43)(21)\}
				\\
				= & \{(12)(34), (13)(24), (14)(23)\}
				\end{aligned}
				$$
				Note que $cl((12))$ contém todos as transposições, $cl((123))$ contém todos os 3-ciclos, $cl((1234))$ contém todos os 4-ciclos e por fim $cl((12)(34))$ contém todos os produtos de 2-ciclos de $S_{4}$, com isso podemos concluir que para qualquer $g \in S_{4}$ teremos que $cl(g) = cl((12))$ ou $cl(g) = cl((123))$ ou $cl(g) = cl((1234))$ ou $cl(g) = cl((12)(34))$ pois as classes lateriais são disjuntos duas a duas ou são identicas.
				
				Já calculamos em exercícios anteriores que $Z(S_{4}) = \{e\}$, portanto temos a equação de conjugação:
				$$
				\begin{aligned}
				|S_{4}| = & |Z(S_{4})| + \sum_{x \notin Z(S_{4})} |S_{4}:Z(x)|
				\\
				=& |Z(S_{4})| + \sum_{x \notin Z(S_{4})} |cl(x)|
				\\
				=& |Z(S_{4})| + |cl((12))| + |cl((123))| + |cl((1234))| + |cl((12)(34))|
				\\
				=& 1 + 6 +8 +6+3
				\\
				=& 24.    
				\end{aligned}
				$$
			
			\item 
			$$
			\begin{aligned}
			A_{4} = & \{e, (123), (124), (132), (134), (142), (143), (234), (243), (12)(34), (13)(24), (14)(23)\}
			\\\\
			cl(e) = & \{e\}
			\\\\
			cl((123)) 
			= & \{e(123)e^{-1}, (123)(123)(123)^{-1}, (124)(123)(124)^{-1}, (132)(123)(132)^{-1}, (134)(123)(134)^{-1}, 
			\\
			&(142)(123)(142)^{-1}, (143)(123)(143)^{-1}, (234)(123)(234)^{-1}, (243)(123)(243)^{-1}, (12)(34)(123)(12)(34)^{-1}, 
			\\
			&(13)(24)(123)(13)(24)^{-1}, (14)(23)(123)(14)(23)^{-1} \}
			\\
			= & \{(123), (123), (243), (312), (324), (413), 
			(421), (134), (142), (214), (341), (432) \}
			\\
			= & \{(123), (134), (142), (243) \}
			\\\\
			cl((132)) 
			= & \{e(132)e^{-1}, (123)(132)(123)^{-1}, (124)(132)(124)^{-1}, (132)(132)(132)^{-1}, (134)(132)(134)^{-1},  
			\\
			&(142)(132)(142)^{-1}, (143)(132)(143)^{-1}, (234)(132)(234)^{-1}, (243)(132)(243)^{-1}, (12)(34)(132)(12)(34)^{-1}, 
			\\
			&(13)(24)(132)(13)(24)^{-1}, (14)(23)(132)(14)(23)^{-1} \}
			\\
			=& \{(132), (213), (234), (321), (342),  
			(431), (412), (143), (124), (241), (314), (423) \}
			\\
			=& \{(132), (234), (143), (124)\}
			\\\\
			cl((12)(34))
			=& \{e(12)(34)e^{-1}, (123)(12)(34)(123)^{-1}, (124)(12)(34)(124)^{-1}, (132)(12)(34)(132)^{-1}, (134)(12)(34)(134)^{-1}, 
			\\
			&(142)(12)(34)(142)^{-1}, (143)(12)(34)(143)^{-1}, (234)(12)(34)(234)^{-1}, (243)(12)(34)(243)^{-1}
			\\
			&(12)(34)(12)(34)(12)(34)^{-1}, (13)(24)(12)(34)(13)(24)^{-1}, (14)(23)(12)(34)(14)(23)^{-1} \}
			\\
			=& \{(12)(34), (23)(14), (24)(31), (31)(24), (32)(41), 
			(41)(32), (42)(13), (13)(42), (14)(23)
			\\
			&(12)(34), (34)(12), (43)(21) \}
			\\
			=& \{(12)(34), (13)(24), (14)(23)\}
			\end{aligned}
			$$
			\end{enumerate}
			Note que $cl((123)) \cap cl((132)) = \emptyset$ e $cl((123)) \cup cl((132))$ contém todas os 3-ciclos de $A_{4}$, portanto qualquer 3-ciclo $h = (h(1)h(2)h(3)) \in A_{4}$ teremos $cl(h)= cl((123))$ ou $cl(h)= cl((132))$, além disso, verificamos que $cl((12)(34))$ gera todos os produtos de 2-ciclos de $A_{4}$. Por fim, sabemos de exercícios anteriores que $Z(A_{4}) = \{e\}$, assim podemos escrever a equação de conjugação:
			$$
			\begin{aligned}
			|A_{4}| = & |Z(A_{4})| + \sum_{x \notin Z(A_{4})} |cl(x)|
			\\
			=& |Z(A_{4})| + |cl((123))| + |cl((132))| + |cl((12)(34))|
			\\
			=& 1 + 4 + 4 +3
			\\
			=& 12.    
			\end{aligned}
			$$
		\item Seja $g \in S_{5}$ tal que $g=(12)(34)$ e $H=<g>$. Encontre a ordem de $N_{S_{5}}(H)$.
		
		Notemos que $g^{2} = (12)(34)(12)(34) = e$, então $g^{n} = e$ se $n$ form par e $g^{n} = g$ se $n$ ímpar, portanto $H = \{e, g\}$. Por definição $ N_{S_{5}}(H)= \{a \in S_{5}: aHa^{-1} = H\}$, isto é, devemos satisfazer a equação $aga^{-1} \in H$ para $g=e$ e $g = (12)(34)$, então fazendo $a = (a(1) \dots a(5))$ teremos:
		$$
		\begin{aligned}
		(a(1) \dots a(5))(12)(34)(a(1) \dots a(5))^{-1} = & (12)(34) = (a(1)a(2))(a(3)a(4)) \\
		\text{ou}
		\\
		(a(1) \dots a(5))(12)(34)(a(1) \dots a(5))^{-1} = & e =(a(1)a(2))(a(3)a(4))
		\end{aligned}
		$$
		no primeiro caso temos o sistema $a(1)=2, \; a(2)=1, \; a(3)=4, \; a(4)=3$, já no segundo caso temos 
		
	\end{enumerate}
	
	
\end{document}