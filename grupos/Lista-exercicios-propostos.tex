\documentclass{article}
\usepackage{graphicx}
\usepackage{indentfirst}
\usepackage[utf8]{inputenc}
\usepackage{amssymb}
\usepackage{enumitem}
\usepackage{color}
\usepackage[fleqn]{amsmath}
\usepackage[a4paper, margin=0.5in]{geometry}
\begin{document}
	
	\title{Lista exercícios propostos para Entregar}
	\author{Vinicius Fernandes}
	
	\maketitle
	
	\begin{enumerate}
		
		\item Mostre que um grupo de ordem 105 possui um subgrupo normal não trivial.
		
		Seja $G$ um grupo tal que $|G| = 105 = 3.5.7$ então pelo primeiro teorema de Sylow temos ao menos um 5-Sylow subgrupo. Pelo terceiro teorema de Sylow, temos que o número de 5-Sylow subgrupo deve satisfazer $n_{5} \equiv_{5} 1$ e $n_{5} | 21$ sendo que as possibilidades são $n_{5} \in \{1, 3, 7, 21\}$, contudo, o único natural satisfazendo essas condições é $n_{5} = 1$. Pelo segundo teorema de Sylow podemos afirmar que o 5-Sylow $A$ é subgrupo normal de $G$, isto é, $A \vartriangleleft G$, e como $|A| = 5 \Rightarrow A \neq \{e\}$ não-trivial.
		
		\item Mostre que um grupo de ordem 56 possui um subgrupo normal não trivial.
		
		Seja $G$ um grupo tal que $|G| = 56 = 2^{3}.7$ então pelo primeiro teorema de Sylow temos ao menos um 2-Sylow subgrupo. Pelo terceiro teorema de Sylow, temos que o número de 2-Sylow subgrupo deve satisfazer $n_{2} \equiv_{2} 1$ e $n_{2} | 7$ sendo que as possibilidades são $n_{7} \in \{1, 7\}$, contudo, o único natural satisfazendo essas condições é $n_{2} = 1$. Pelo segundo teorema de Sylow podemos afirmar que o 2-Sylow $A$ é subgrupo normal de $G$, isto é, $A \vartriangleleft G$, e como $|A| = 8 \Rightarrow A \neq \{e\}$ não-trivial.
		
		\item Sejam $p,q$ números primos tal que $p>q$ e seja $G$ um grupo de ordem $|G| = pq$. Mostre que $G$ possui um p-Sylow subgrupo normal de ordem $p$.
		
		Seja $G$ um grupo tal que $|G| = pq$ com $p>q$, então pelo primeiro teorema de Sylow temos ao menos um p-Sylow subgrupo. Pelo terceiro teorema de Sylow, temos que o número de p-Sylow subgrupo deve satisfazer $n_{p} \equiv_{p} 1$ e $n_{p} | q$ temos as possibilidades $n_{p} \in \{1, q\}$, contudo, o único elemento satisfazendo essas condições é $n_{p} = 1$ pois $q$ é primo. Pelo segundo teorema de Sylow podemos afirmar que o p-Sylow $A$ é subgrupo normal de $G$, isto é, $A \vartriangleleft G$, e como $|A| = p \Rightarrow A \neq \{e\}$ não-trivial.
		
		\item Sejam $p,q$ números primos e $G$ um grupo de ordem $|G| = p^{2}q$. Mostre que $G$ possui um p-subgrupo não-trivial.
		
		Seja $G$ um grupo tal que $|G| = p^{2}q$, então pelo primeiro teorema de Sylow temos ao menos um p-Sylow subgrupo e subgrupos de ordem $p^{k}, \; 1 \leq k \leq 2$, com isso, podemos definir $H$ como sendo um dos subgrupos de ordem $p$. Sabemos que todo grupo de ordem prima é cíclico, então suponha que $H = <g>$ para algum $g \in H$. Sabemos que a ordem $o(g)$ divide a ordem $H$, isto é, $|H| = ko(g), \; k \in \mathbb{N}$, mas como $|H|$ é um número primo, isso implica que $o(g) = 1$ ou $o(g) = |H|$, mas não podemos ter $o(g) = 1$ pois isso implicaria que $g = e$, o que é um absurdo pois $H$ é não-trivial, então $o(g) = p$. Portanto, todo elemento de $H$ tem como ordem um número multiplo de $p$, assim $H$ é um p-subgrupo não trivial.
		
		\item Mostre que todo grupo de ordem não-prima até 59 não é simples.
		
		Suponha que $G$ é um grupo com ordem par, isto é, $|G| = 2^{k}b, \; b \in \mathbb{N}$, então pelo primeiro teorema de Sylow existe um grupo existe um subgrupo $H < G$ de ordem $|H| = 2$, que é normal em $G$, portanto grupos de ordem par não são simples. 
		
		Eliminando os grupos de ordem par e prima teremos os casos em que $|G| \in \{9, 15, 21, 25, 27, 33, 35, 39, 45, 49, 51, 55, 57\}$. Vejamos os casos:
		\begin{enumerate}
			\item $|G| \in \{9, 25, 27, 49\}= \{3^{2}, 3^{3}, 5^{2}, 7^{2}\}$ teremos a ordem de $G$ de dois tipos $|G| =3^{k}$ ou $|G| =5^{m}$ ou $|G| =7^{n}$ para algum $k,m ,n \in \mathbb{N}$. Em ambos os casos o terceiro teorema de Sylow afirma que temos apenas um único 3-Sylow subgrupo, um único 5-Sylow subgrupo e um único 7-Sylow subgrupo, respectivamente. Com isso, o segundo teorema de Sylow afirma que se são únicos então são normais, portanto $G$ não é simples.
			
			\item $|G| \in \{15, 21, 33, 35, 39, 51, 55, 57\}= \{3.5, 3.7, 3.11, 5.7, 3.13, 3.17, 5.11, 3.19\}$ que são multiplos de números primos. Sabemos de um exercício anterior que se a ordem $G$ for um multiplo de números primos dada por $|G| = pq$ com $p>q$, então $G$ possui em p-Sylow subgrupo normal, portanto $G$ não é simples.
			
			\item $|G| \in \{45\} = \{3^{2}.5 \}$, nesse caso o primeiro teorema de Sylow afirma que existe um 5-Sylow subgrupo e o terceiro teorema de Sylow diz que o número de 5-Sylow $n_{5}$ deve satisfazer as condições $n_{5} \equiv_{5} = 1$ e $n_{5}|9$, com isso temos as possibilidades $n_{5} \in \{1, 3, 9\}$, mas o único natural satisfazendo essas condições é $n_{5} = 1$, portanto temos um único 5-Sylow subgrupo. Nessas condições o segundo teorema de Sylow garante que esse subgrupo é normal, portanto $G$ não é simples.
		\end{enumerate}
		
		Com isso cobrimos todas as possibilidades de grupos de ordem não-prima de até 59 mostrando que cada uma delas trata-se um grupo que não é simples, como desejávamos.
		
	\end{enumerate}
	
	
\end{document}