\documentclass{article}
\usepackage{graphicx}
\usepackage{indentfirst}
\usepackage[utf8]{inputenc}
\usepackage{amssymb}
\usepackage{enumitem}
\usepackage{color}
\usepackage[a4paper, margin=0.5in]{geometry}
\begin{document}
	
\title{Lista 1 para Entregar}
\author{Vinicius Fernades}
	
\maketitle
	
\begin{enumerate}
	\item \textbf{Solução:} Sejam $K, H \subseteq G \neq \emptyset$. Como $K,H$ são subgrupos, então o elemento neutro $e \in K, e\in H \Rightarrow e \in K \cap H$. Tomando $a, b \in K \cap H$, assim $ab^{-1} \in K$ e $ab^{-1} \in H$ pois ambos são subgrupos, portanto $ab^{-1} \in K \cap H$ e $K \cap H$ é um subgrupo, como desejávamos.
	
	\item \textbf{Solução:} O elemento neutro $0 \in a\mathbb{Z}$, pois $0 = a.0 \in a\mathbb{Z}$. Tomando $x,y \in a\mathbb{Z}$, veremos que $x-y \in a\mathbb{Z}$. Com efeito: $\exists p,q \in \mathbb{Z}; \; x=ap, y=aq$, consequentemente, $x-y = ap - aq = a(p-q) \in a \mathbb{Z}$, portanto $a \mathbb{Z} \subset \mathbb{Z}$ é um subgrupo.
	
	O elemento neutro $0 \in a\mathbb{Z} + b\mathbb{Z}$, pois $0 = a.0 +b.0 \in a\mathbb{Z} + b\mathbb{Z}$. Tomando $x,y \in a\mathbb{Z} + b\mathbb{Z}$, veremos que  $x-y \in a\mathbb{Z} + b\mathbb{Z}$. Com efeito: $\exists p,q,r,s \in \mathbb{Z}; \; x=ap +bq, y=ar+bs$, então $x-y = ap +bq - (ar+bs) = a(p-r) + b(q-s) \in a\mathbb{Z} + b\mathbb{Z}$, portanto $a\mathbb{Z} + b\mathbb{Z} \subset \mathbb{Z}$ é um subgrupo. 
	
	Para exibir o gerador de $a\mathbb{Z} + b\mathbb{Z}$ tome $d = mdc(a, b)$, isto é, $\exists \delta, \lambda \in \mathbb{Z}; \; a = \delta d, b = \lambda d$, então $\forall x \in a\mathbb{Z} + b\mathbb{Z}$ teremos $x= ap+bq = \delta dp +\lambda dq = d(\delta p +\lambda q)$, portanto $x = dn$, para algum $n \in \mathbb{Z}$, consequentemente, $a\mathbb{Z} + b\mathbb{Z} = <d>$ e $d$ é um gerador desse conjunto.
	
	\item \textbf{Solução:} Devemos verificar os axiomas de grupos para $(G^{\circ},*)$, isto é:
		\begin{enumerate}
			\item $G^{\circ}$ é fechado pela operação $*: G^{\circ} \times G^{\circ} \to G^{\circ}$, pois $\forall a, b \in G^{\circ}$ temos $a*b = b.a \in G \Rightarrow b.a \in G^{\circ}$, pois $G^{\circ}$ contêm os mesmos elementos de $G$.
			
			\item $*: G^{\circ} \times G^{\circ} \to G^{\circ}$ é associativa pois $a*(b*c) = a*(c.b) = (c.b)a = c.b.a = c.(b.a) = (b.a)*c = (a*b)*c$.
			
			\item O elemento neutro $e \in G^{\circ}$ pois $\forall a \in G^{\circ}; \; e*a = a.e = e.a = a*e =a$.
			
			\item $\forall x \in G^{\circ} \; \exists y \in G^{\circ}$ tal que $x*y=e$ pois $x*y= y.x = e= x^{-1}.x \Rightarrow y = x^{-1} \in G = G^{\circ}$.
		\end{enumerate}
		Portanto $(G^{\circ}, *)$ é um grupo.
		
		Definindo $*_{g}:G\to G^{\circ}$ tal que $*_{g}(a) = g*a*g^{-1} = g^{-1}.a.g$, e vejamos que $*_{g}$ é um homomorfismo, pois 
		$$
		*_{g}(a.b) = g*(a.b)*g^{-1} = ((a.b).g)*g^{-1} = g^{-1}.(a.b).g = g^{-1}.a.g.g^{-1}.b.g=
		$$ 
		$$
		= (g^{-1}.a.g).(g^{-1}.b.g) = *_{g}(a)*_{g}(b).
		$$
		
		Além disso, $*_{g}$ é uma bijeção, pois tomando $a, b \in G$ temos $*_{g}(a) = g*(b) \Rightarrow g^{-1}.a.g = g^{-1}.b.g \Rightarrow a=b$ pela regra do cancelamento, portanto é injetora. É uma sobrejeção pois a imagem $*_{g}(G) = G= G^{\circ}$, com isso temos uma bijeção. Conclusão: $*_{g}:G\to G^{\circ}$ é um isomomorfismo, como desejávamos.
		
		\item \textbf{Solução:} Devemos verificar os axiomas de grupos para $(G,*)$, isto é, $\forall a, b, d \in G; $ temos:
		\begin{enumerate}
			\item A operação $*: G \times G \to G$ é fechada pois $a*b = a.c^{-1}.b \in G$, portanto $a*b \in G$
			
			\item Também é associativa pois tal que 
			$$
			a*(b*d) = a.c^{-1}.(b*d) =  a.c^{-1}.(b.c^{-1}.d) = (a.c^{-1}.b).c^{-1}.d =
			$$
			$$
			= (a.c^{-1}.b).c^{-1}.d = (a*b).c^{-1}.d = (a*b)*d.
			$$
			
			\item Uma condição para que o elementro neutro $e' \in (G,*)$, devemos a equação $a*e' = a = a.e = a.c^{-1}.e'$ que pela regra do cancelamento teremos $e = c^{-1}.e'$, e portanto $ e'=c$, onde utilizamos a hipótese de que $e \in (G,.)$ é o elemento neutro desse grupo. De fato, $\forall a \in (G,*)$ teremos $a*c = a.c^{-1}.c = a = c.c^{-1}.a = c*a$.
			
			\item Afirmo que $\forall a \in (G,*) \; \exists y \in (G,*); \; a*y = y*a = c$ (elemento neutro nesse grupo). Pois bem, temos a equação $a.c^{-1}.y = y.c^{-1}.a = c \Rightarrow y = c.a^{-1}.c$. De fato: 
			$$
			a*y = a*(c.a^{-1}.c) = a.c^{-1}.c.a^{-1}.c = c,
			$$
			$$
			y*a = (c.a^{-1}.c)*.a = c.a^{-1}.c.c^{-1}.a = c.
			$$
			Portanto $(G,*)$ é um grupo, como desejávamos.
		\end{enumerate}
		
		\item \textbf{Solução:} Sejam $G \neq \emptyset$ um grupo e $K,H \subseteq G$ subgrupos normais. Sabemos de um exercício anterior que $H \cap K \neq \emptyset$ é um subgrupo, então resta-nos mostrar que $H \cap K$ é normal, isto é, $\forall g \in G$ devemos ter $g(H \cap K)g^{-1} = H \cap K$. Com efeito: a inclusão $H \cap K \subseteq g.(H \cap K ).g^{-1}$ é imediata pois basta tomar $g = e$, agora vamos mostrar $g.(H \cap K ).g^{-1}\subseteq H \cap K$.  Sejam $g.a.g^{-1} \in g.(H \cap K ).g^{-1}$ e $a \in H \cap K$, então temos $g.a.g^{-1} \in H, K$ pois ambos são normais, então $g.a.g^{-1} \in H \cap K \Rightarrow g.(H \cap K ).g^{-1} \subseteq H \cap K $. Portanto $g.(H \cap K ).g^{-1}= H \cap K$ e esse subgrupo é normal.
		
		\item \textbf{Solução:} Fazendo $ab = aba^{7} = a(ba^{3})a^{4} = a(a^{3}b)a^{4} = a^{4}ba^{4} = a^{4}(ba^{3})a = a^{4}(a^{3}b)a = a^{7}ba = ba$, como desejávamos.
		
		\item \textbf{Solução:} Como foi definido, temos que $\varphi: U \to \mathbb{R}$ tal que $\varphi(A) = A_{11}^{2}$ onde $A_{11} \neq 0$, então tomando $A, B \in U$ temos $\varphi(A.B) = (A_{11}B_{11})^{2} = A_{11}^{2}B_{11}^{2} = \varphi(A)\varphi(B)$, portanto é um homomorfismo. Note que se $\varphi(A)=1 \Rightarrow A \in Ker(\varphi)$, portanto $Ker(\varphi) =\{A \in U: A_{11} = 1\}$. Além disso, $\varphi(A)>0 \Rightarrow Im(\varphi) = \mathbb{R^{+}}$.
		
		\item \textbf{Solução:} Sabemos que $\mathbb{Z} = <1>$, então suponha que $f:\mathbb{Z} \to \mathbb{Z}$ seja um homomorfismo, tal que $f(1)=m$. Então,
		$$
		f(k) = f(\underbrace{1+\dots+1}_{k-vezes})=\underbrace{f(1)+\dots+f(1)}_{k-vezes}= mk.
		$$
		Assim, para cada $m \in \mathbb{Z}$ temos um homomorfismo, portanto existe uma bijeção entre o conjunto de homomorfismos de $G$ e $\mathbb{Z}$.
		
		\item \textbf{Solução:} Supondo que $G_{1}, G_{2}$ sejam grupos não-vazios e $Z_{1}, Z_{2}$ seus respectivos centros. Então $(G_{1}\times G_{2}, *)$ é um grupo tal que $(a, b), (c, d) \in G_{1}\times G_{2}$ temos $(a,b)*(c,d) := (a.c, b.d)$, assim $Z(G_{1}\times G_{2}) = \{(a, b) \in G_{1}\times G_{2}: (a, b)*(c, d) = (c, d)*(a, b), \forall (c, d) \in G_{1}\times G_{2}\}$, isto é, $(a, b)*(c,d)=(c,d)*(a, b) \iff (a.c, b.d) =(c.a, d.b)$, então $a \in Z_{1}, b \in Z_{2}$, portanto $(a,b) \in Z_{1} \times Z_{2}$, ou seja, $Z(G_{1}\times G_{2}) = Z_{1} \times Z_{2}$, como desejávamos.
		
		\item \textbf{Solução:} As classes laterais de $H = 0 \times \mathbb{Z}_{2}$ são dadas por $g+H, \; g\in G =\mathbb{Z} \times \mathbb{Z}_{2}$, então tomando $g=(a, [b])$ teremos: $g+H = \{(a,[c+b]): (0,[c]) \in H\} = \{a\} \times \mathbb{Z}_{2}$. Como $G$ é abeliano pois $\mathbb{Z}$ é abeliano, consequentemente $H$ também é abeliano, portanto suas classes laterias são identicas, isto é, $g+H = H+g$.  
		
		\item \textbf{Solução:} Sejam $h_{1}, h_{2} \in H$, devemos mostrar que $h_{1}h_{2}^{-1} \in H$, assim:
		$$
		h_{2} = \left(
		\begin{array}{ccc}
		1 & b_{2} & c_{2} \\
		0 & 1 & d_{2} \\
		0 & 0 & 1 \\
		\end{array}
		\right), \; 
		h_{2}^{-1} = \left(
		\begin{array}{ccc}
		1 & -b_{2} & b_{2}d_{2} - c_{2}\\
		0 & 1 & -d_{2} \\
		0 & 0 & 1 \\
		\end{array}
		\right) \in H,
		$$
		consequentemente,
		$$
		h_{1}h_{2}^{-1} = \left(
		\begin{array}{ccc}
		1 & b_{1} & c_{1} \\
		0 & 1 & d_{1} \\
		0 & 0 & 1 \\
		\end{array}
		\right).
		\left(
		\begin{array}{ccc}
		1 & -b_{2} & b_{2}d_{2} - c_{2}\\
		0 & 1 & -d_{2} \\
		0 & 0 & 1 \\
		\end{array}
		\right) = 
		$$
		$$
		= 		\left(
		\begin{array}{ccc}
		1 & -b_{2}+b_{1} & (b_{2}- b_{1})d_{2} - c_{2} +c_{1}\\
		0 & 1 & -d_{2} + d_{1}\\
		0 & 0 & 1 \\
		\end{array}
		\right) \in H.
		$$
		Portanto esse conjunto é um subgrupo de $GL_{3}$.
		A demonstração anterior vale $\forall b_{1}, c_{1}, d_{1} \in \mathbb{R}$, assim, vale para $b_{1}=d_{1}=0$, o que caracteriza $K \subset H$, portanto $K \subset H$ também é um subgrupo de $GL_{3}$, e evidentemente de $H$. Resta-nos mostrar que $K$ é normal em $H$, isto é $\forall h \in H; \; hKh^{-1} = K$, assim, tomando $k \in K$:
		$$
		hkh^{-1} = \left(
		\begin{array}{ccc}
		1 & b & c\\
		0 & 1 & d \\
		0 & 0 & 1 \\
		\end{array}
		\right).
		\left(
		\begin{array}{ccc}
		1 & 0 & a\\
		0 & 1 & 0 \\
		0 & 0 & 1 \\
		\end{array}
		\right).
		\left(
		\begin{array}{ccc}
		1 & -b & bd - c\\
		0 & 1 & -d \\
		0 & 0 & 1 \\
		\end{array}
		\right) = 
		$$
		$$
		= \left(
		\begin{array}{ccc}
		1 & b & a\\
		0 & 1 & d \\
		0 & 0 & 1 \\
		\end{array}
		\right).
		\left(
		\begin{array}{ccc}
		1 & -b & bd - c\\
		0 & 1 & -d \\
		0 & 0 & 1 \\
		\end{array}
		\right) =
		\left(
		\begin{array}{ccc}
		1 & 0 & -c+a\\
		0 & 1 & 0 \\
		0 & 0 & 1 \\
		\end{array}
		\right) \in K.
		$$
		Note que o resultado da operação anterior $	hkh^{-1}$ é uma relação $hKh^{-1} \leftrightarrow K$ (1-para-1), portanto $hKh^{-1} = K$, e esse conjunto é um subgrupo normal de $H$. Por definição $H/K = \{h.K: h \in H\}$, assim, tomando $h \in H, k \in K$ como antes:
		$$
		hk = \left(
		\begin{array}{ccc}
		1 & b & c\\
		0 & 1 & d \\
		0 & 0 & 1 \\
		\end{array}
		\right).
		\left(
		\begin{array}{ccc}
		1 & 0 & a\\
		0 & 1 & 0 \\
		0 & 0 & 1 \\
		\end{array}
		\right) =  
		\left(
		\begin{array}{ccc}
		1 & b & a\\
		0 & 1 & d \\
		0 & 0 & 1 \\
		\end{array}
		\right),
		$$
		$$
			H/K = \left\{
				\left(
				\begin{array}{ccc}
				1 & b & a\\
				0 & 1 & d \\
				0 & 0 & 1 \\
				\end{array}
				\right): a,b,d \in \mathbb{R}
				\right\} \cong H.
		$$
		O centro de $H$ é definido por $Z(H) = \{h_{1} \in H: h_{1}.h_{2} = h_{2}.h_{1}, \forall h_{2}\in H \}$, assim, tomando $h_{1}, h_{2} \in H$ como antes:
		$$
		h_{1}.h_{2} = h_{2}.h_{1},
		$$
		$$
		\left(
		\begin{array}{ccc}
		1 & b_{1} & c_{1} \\
		0 & 1 & d_{1} \\
		0 & 0 & 1 \\
		\end{array}
		\right).
		\left(
		\begin{array}{ccc}
		1 & b_{2} & c_{2} \\
		0 & 1 & d_{2} \\
		0 & 0 & 1 \\
		\end{array}
		\right)=
		\left(
		\begin{array}{ccc}
		1 & b_{2} & c_{2} \\
		0 & 1 & d_{2} \\
		0 & 0 & 1 \\
		\end{array}
		\right).
		\left(
		\begin{array}{ccc}
		1 & b_{1} & c_{1} \\
		0 & 1 & d_{1} \\
		0 & 0 & 1 \\
		\end{array}
		\right)
		$$
		$$
		\left(
		\begin{array}{ccc}
		1 & b_{1}+b_{2} & c_{2}+c_{1} +b_{1}d_{2}\\
		0 & 1 & d_{1}+d_{2} \\
		0 & 0 & 1 \\
		\end{array}
		\right)=
		\left(
		\begin{array}{ccc}
		1 & b_{1}+b_{2} & c_{2}+c_{1} +b_{2}d_{1}\\
		0 & 1 & d_{1}+d_{2} \\
		0 & 0 & 1 \\
		\end{array}
		\right) \Rightarrow
		$$
		$$
		\Rightarrow b_{1}d_{2} = b_{2}d_{1}.
		$$
		Como essa relação entre $b_{1}, d_{1}$ deve valer para todos os $b_{2}, d_{2}$, então $b_{1}= d_{1}=0$, ou seja, $Z(H) \cong K$.
		
		\item \textbf{Solução:} Por hipótese $H \vartriangleleft G$, então o quociente $G/H$ é um grupo quociente. Sabemos que $G$ é cíclico, então $\exists x \in G; \; G=<x>$. Como todos os elementos de $G$ podem ser escritos como uma potência de $x$, então as classes lateriais serão da forma $[x^{p}]:=x^{p}H$ e $G/H = \{[x^{p}]: p\in \mathbb{Z}\}$. Vamos mostrar que $G/H = <[x]>$. Com efeito, seja $[x^{p}] \in G/H$, e como o quociente é um grupo e $G$ é cíclico, vale a regra $[x^{p}] = [x]^{p}$, então $G/H$ é gerado por $[x]$, portanto é cíclico, como desejávamos.
		
\end{enumerate}
		
\end{document}