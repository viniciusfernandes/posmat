\documentclass{article}
\usepackage{graphicx}
\usepackage{indentfirst}
\usepackage[utf8]{inputenc}
\usepackage{amssymb}
\usepackage{enumitem}
\usepackage{color}
\usepackage[fleqn]{amsmath}
\usepackage[a4paper, margin=0.5in]{geometry}
\begin{document}
	
	\title{Cohomologia - Algumas demostrações}
	\author{Vinicius Fernades}
	
	\maketitle
	Seja $V$ um espaço vetorial, dizemos que uma função p-linear é $\omega : V \times \dots \times V \to \mathbb{R}$, tal que $a_{i}, b_{i} \in V$ onde $1\leq i \leq p$ e $\lambda \in \mathbb{R}$ teremos $\omega(a_{1} + \lambda b_{1}, ..., a_{p} + \lambda b_{p}) = \omega(a_{1} , ..., a_{p} )+ \lambda \omega(b_{1}, ..., b_{p})$. Afirmamos que $\Lambda^{p}(V)$ como sendo o conjunto das funções p-lineares forma um espço vetorial. Tomemos $\Lambda^{1}(V)$ e $\{e_{1}, ..., e_{n}\}$ com sendo uma base de $V$, então $\Lambda^{1}(V) = V^{*}$ (espaço dual de V), consequentemente, escreveremos $\{dx_{1}, ..., dx_{n}\}$ como sendo uma base de $\Lambda^{1}(V)$, isto é, $dx_{j}(e_{i}) = \delta_{ij}$, desse modo, uma função 1-linear pode ser escrita como $\omega = \sum w_{i}dx_{i}$. Definindo o produto tensorial $\otimes: \Lambda^{1}(V) \times \Lambda^{1}(V) \to \Lambda^{2}(V)$ tal que $\otimes(dx_{i}, dx_{j})(e_{k}, e_{l}) = dx_{i}(e_{k}) \otimes dx_{j}(e_{l}) = \delta_{ik}\delta_{jl}$. Afirmamos que uma função 2-linear pode ser escrita como sendo $\omega = \sum \omega_{ij}dx_{i} \otimes dx_{j}$. Analogamente, teremos uma função p-linear como sendo  $\omega = \sum \omega_{i_{1}...i_{p}}dx_{i_{1}} \otimes ... \otimes dx_{i_{p}}$.
	
	Seja $\omega$ uma função p-linear tal que $\omega(a_{1}, ..., a_{p}) = 0 $ sempre que $a_{i} = a_{j}$, para algum $i, j$. Diremos então que $\omega$ é uma função alternada p-linear. Notemos que a seguinte função 2-linear dada por $\omega = dx_{i} \otimes dx_{j} - dx_{j} \otimes dx_{i}$, então $\omega(e_{k},e_{l}) = dx_{i}(e_{k}) \otimes dx_{j}(e_{l}) - dx_{j}(e_{k}) \otimes dx_{i}(e_{l}) = \delta_{ik}\delta_{jl} - \delta_{jk}\delta_{il} =  -\delta_{il}\delta_{jk} + \delta_{jl}\delta_{ik}= -(\delta_{il}\delta_{jk} - \delta_{jl}\delta_{ik}) = dx_{i}(e_{l}) \otimes dx_{j}(e_{k}) - dx_{j}(e_{l}) \otimes dx_{i}(e_{k}) = -\omega(e_{l}, e_{k})$, que é uma função alternada 2-linear. Definiremos o produto exterior como sendo $\wedge: \Lambda^{1}(V) \times \Lambda^{1}(V) \to \Lambda^{2}(V)$ por $ dx_{i} \wedge dx_{j}(a, b) = dx_{i}(a) \otimes dx_{j}(b) - dx_{j}(a) \otimes dx_{i}(b) = \sum_{\sigma} sgn(\sigma) dx_{\sigma(i)}(a) \otimes dx_{\sigma(j)}(b)$. Analogamente, o produto $\wedge: \Lambda^{p}(V) \times \Lambda^{q}(V) \to \Lambda^{p+q}(V)$ será definido como $dx_{i_{1}} \wedge ... \wedge dx_{i_{p}} = \sum_{\sigma}  sgn(\sigma) dx_{\sigma (i_{1})} \otimes ... \otimes dx_{\sigma (i_{p})}$. Afirmamos que qualquer função alternada p-linear poderá ser escrita como a combinação linear do produto exterior $\omega = \sum \omega_{i_{1}...i_{p}} dx_{i_{1}} \wedge... \wedge dx_{i_{p}}$.
	
	\textbf{Lema de Poincaré:} seja $U \subseteq \mathbb{R}^{n}$ um conjunto estrelado, então $H^{p}(U) = 0$ se $p>0$ e $H^{0}(U) = \mathbb{R}^{n}$.
	
	$\square$ Sem perda de generalidade, podemos assumir que $U$ é estrelado com relação a origem (pois a operaçao de translação de um conjunto estrelado gera um comjunto estrelado). Seja o mapa $S^{p}: \Omega^{p}(U \times \mathbb{R}) \to \Omega^{p-1}(U)$, e podemos escrever $\omega \in \Omega^{p}(U \times \mathbb{R}) $ como sendo $\omega = \sum g_{J}(x,t)dt \wedge dx_{J} + \sum f_{I}(x,t)dx_{I}$, onde $I=(i_{1}, \dots ,i_{p})$ e $J=(i_{1}, \dots ,i_{p-1})$. Agora definindo
	$$
	S^{p}(\omega) = \sum \Big( \int_{0}^{1} g_{J}(x,t)dt \Big) dx_{J},
	$$
	podemos ver que
	$$
	\begin{aligned}
	dS^{p}(\omega) =& \sum_{J,i} \Big( \int_{0}^{1} \partial_{i} g_{J}(x,t)dt \Big) dx_{i} \wedge dx_{J},
	\\
	d(\omega) =& \sum_{J,k} \partial_{k} g_{J}(x,t) dx_{k} \wedge dt \wedge dx_{J} + \sum_{I,k} \partial_{k} f_{I}(x,t) dx_{k} \wedge dx_{I}
	\\
	 =& \sum_{J,k} -\partial_{k} g_{J}(x,t) dt \wedge dx_{k} \wedge dx_{J} + \sum_{I,k} \partial_{k} f_{I}(x,t) dx_{k} \wedge dx_{I},
	 \\
	 S^{p+1}d(\omega) =&  -\sum_{J,k} \Big( \int_{0}^{1}  \partial_{k} g_{J}(x,t)dt \Big) dx_{k} \wedge dx_{J} + \sum_{I} \Big( \int_{0}^{1}  \partial_{t} f_{I}(x,t)dt \Big) dx_{J},
	 \\
	  d S^{p}(\omega) + S^{p+1}d(\omega) =& \sum_{J,i} \Big( \int_{0}^{1} \partial_{i} g_{J}(x,t)dt \Big) dx_{i} \wedge dx_{J} -\sum_{J,k} \Big( \int_{0}^{1}  \partial_{k} g_{J}(x,t)dt \Big) dx_{k} \wedge dx_{J} 
	  \\
	  +& \sum_{I} \Big( \int_{0}^{1}  \partial_{t} f_{I}(x,t)dt \Big) dx_{J}
	  \\
	  =& \sum_{I} \Big( \int_{0}^{1}  \partial_{t} f_{I}(x,t)dt \Big) dx_{J} = \sum_{I} \Big( f_{I}(x,1)  - f_{I}(x,0)\Big) dx_{I}
	  \\
	  =& \sum_{I} f_{I}(x,1) dx_{I} - \sum_{I} f_{I}(x,0) dx_{I}.
	\end{aligned}
	$$
	Como $U$ é um conjunto estrelado, podemos conectar todo ponto com a origem através de uma reta, assim, a função $\phi: U \times [0,1] \to U$ tal que $\phi(x,t) = tx$ esta bem definida. Vejamos então que $\phi^{*} : \Omega(U) \to \Omega(U \times \mathbb{R})$ será dado por $\phi^{*}(\alpha)_{x}(a_{1},\dots,a_{p}) = \alpha_{tx}(D_{x}\phi(a_{1}),\dots, D_{x}\phi(a_{p}))$, consequentemente, se tivermos $\alpha_{x} = \sum \alpha_{I}(x)dx_{I}$, então 
	$$
	\begin{aligned}
	d\phi_{i}(x) =& d(tx_{i}) = dtx_{i} + tdx_{i} 
	\\
	(\phi^{*}\alpha_{x})(a_{1},\dots,a_{p}) =& \sum \alpha_{I}(\phi(x))d\phi_{I}(D_{x}\phi(a_{1}),\dots, D_{x}\phi(a_{p}))
	\\
	=& \sum \alpha_{I}(tx)(dtx_{I} + tdx_{I})(D_{x}\phi(a_{1}),\dots, D_{x}\phi(a_{p})).
	\end{aligned}
	$$
	Notemos que podemos simplificar os seguintes termos: 
	$$
	\begin{aligned}
	(dtx_{i} + tdx_{i}) \wedge (dtx_{j} + tdx_{j}) =& x_{i}x_{j} \overbrace{ dt \wedge dt}^{=0} + tdt \wedge dx_{j} 
	\\
	+& tx_{j}dx_{i}\wedge dt + t^{2} dx_{i} \wedge dx_{j}
	\\
	=& \underbrace{ tdt \wedge dx_{j} - tx_{j} dt \wedge dx_{i} }_{cancela}+ t^{2} dx_{i} \wedge dx_{j},
	\end{aligned}
	$$
	sendo que o penultimo termo se cancelará sob o sinal da somatória, com isso, podemos escrever: 
	$$
	\phi^{*}\alpha_{x} = \sum \alpha_{I}(tx) t^{p} dx_{I}.
	$$
	Fazendo $\omega_{x} = \phi^{*}\alpha_{x}$ e voltando na relação:
	$$
	\begin{aligned}
   	dS^{p}(\phi^{*}\alpha_{x}) + S^{p+1}d(\phi^{*}\alpha_{x}) 
	 =& \sum_{I} \alpha_{I}(1.x) 1^{p} dx_{I} - \sum_{I} \alpha_{I}(0.x) 0^{p} dx_{I}
	 \\
	 =& \sum_{I} \alpha_{I}(x) dx_{I}
	 \\
	 (dS^{p}\phi^{*} + S^{p+1}d\phi^{*})(\alpha_{x}) =& \alpha_{x},
	\end{aligned}$$
	portanto, $dS^{p}\phi^{*} + S^{p+1}d\phi^{*} = Id$. Como $d\phi^{*} = \phi^{*}d$, podemos definir $B^{p} = S^{p}\circ\phi^{*}$, assim vamos reescrever
	$$
	dS^{p}\phi^{*} + S^{p+1}d\phi^{*} = dS^{p}\phi^{*} + S^{p+1}\phi^{*}d = dB^{p} + B^{p+1}d = Id.
	$$
	Sabemos que um p-forma pode ser fechada ou exata, suponhamos que $\omega \in \Omega^{p}(U)$ seja fechada, então teremos $d\omega = 0$ o que implica que $\omega = dB^{p}\omega + B^{p+1}d\omega = dB^{p}\omega \Rightarrow [\omega] = [dB^{p}\omega] = [0]$. Já no caso em que $\omega$ seja exata, isto é, $\omega = d\alpha$ para algum $\alpha \in \Omega^{p}(U)$, teremos imediatamente $[\omega] =[d\alpha] = [0]$, portanto $H^{p}(U) = 0$ para $p>0$. Para o caso em que $p=0$ teremos..........

	\textbf{Lema:} Sejam $A^{*}, B^{*}, C^{*}$ complexos de cadeia e $f:A^{*} \to B^{*}$, $g:B^{*} \to C^{*}$ mapa de cadeias, além disso, se a sequência $0 \to A^{*} \xrightarrow{f} B^{*} \xrightarrow{g} C^{*} \to 0$ for uma sequência exata curta, então a sequência $H^{p}(A^{*}) \xrightarrow{f} H^{p}(B^{*}) \xrightarrow{g} H^{p}(C^{*}) $ é exata.
	
	$\square$ Por hipótese temos $Im(f^{p}) = Ker(g^{p})$, isto é, dado $a \in A^{p}$ teremos $(g^{p}\circ f^{p})(a) = 0$, consequentemente, $(g\circ f)^{*}([a]) = g^{*}(f^{*}([a])) = [(g^{p}\circ f^{p})(a)] = [0]$, portanto $Im(f^{*}) \subset Ker(g^{*})$. Vamos mostrar a inclusão $Ker(g^{*}) \subset Im(f^{*})$. Tomemos $[b] \in Ker(g^{*})$, assim $g^{*}([b]) = [0] \Rightarrow [g^{p}(b)] = [0]$, isto é, $g^{p}(b) = d^{p-1}c$, para algum $c \in H^{p}(C)$. Como $g^{p-1}$ é sobrejtor, então existe $b' \in H^{p}(B)$ tal que $g^{p-1}(b') = c$, assim podemos afirmar que $g^{p}(b) = d^{p-1}g^{p-1}(b') = g^{p}d^{p}(b')$, portanto $g^{p}(b - d^{p}(b')) = 0$, isto é, $b - d^{p}(b') \in Ker(g^{p})$. Como temos uma sequência exata curta, isto é $Im(f^{p}) = Ker(g^{p})$, podemos afirmar que existe $a \in A^{p}$ tal que $f^{p}(a) = b - d^{p-1}(b') \in Ker(g^{p})$ e como $f^{p}$ é injetora (por hipótese) e também sobrejetora (pois $Im(f^{p}) = Ker(g^{p})$) , então podemos escrever $f^{*}[a] = [f^{p}(a)] = [b - d^{p-1}(b')] =  [b] - [d^{p-1}(b')] = [b]$, portanto $[b] = f^{*}[a] \in Im(f^{*})$, o que nos permite afirmar $Ker(g^{*}) \subset Im(f^{*})$. Conclusão, $Ker(g^{*}) = Im(f^{*})$, e temos uma sequência exata. $\blacksquare$

	\textbf{Proposição:} O mapa $\partial^{*}: H^{p}(C^{*}) \to H^{p+1}(A^{*})$ dado por $\partial^{*}([c]) = [(f^{p+1})^{-1}(d^{p}((g^{p})^{-1}(c)))] $ esta bem definido.
	
	$\square$ Seja $c \in C^{p}$ tal que $d^{p}c=0$, e como $g^{p}$ é sobrejetora, então existe $b \in B^{p}$ tal que $g^{p}(b) = c$ e $d^{p}g^{p}(b) = d^{p}c = 0$, portanto $g^{p+1}(d^{p}b) = d^{p}g^{p}(b) = 0 \Rightarrow d^{p}b \in Ker(g^{p+1}) = Im(f^{p+1})$, pois temos uma sequência exata curta, além disso, $f^{p}$ é injetora, então existe um único $a \in A^{p+1}$ tal que $f^{p+1}(a) = d^{p}b$, consequentemente, $d^{p+1}f^{p+1}(a) = f^{p+2}(d^{p+1}(a))=d^{p+1}d^{p}b=0 \Rightarrow f^{p+2}(d^{p+1}(a)) =0$ e como $f^{p+2}$ é injetora, então devemos ter $d^{p+1}(a)=0$. Novamente, como $g^{p}$ é sobrejetora, então existem $b_{1} \neq b_{2} \in B^{p}$ tais que $g^{p}(b_{1}) = g^{p}(b_{2}) = c$, ou ainda, $g^{p}(b_{1}) - g^{p}(b_{2}) = g^{p}(b_{1}-b_{2}) = 0 \Rightarrow b_{1}-b_{2} \in Ker(g^{p}) = Im(f^{p})$, então existe $a' \in Im(f^{p+1})$ tal que $f^{p}(a') = b_{1}-b_{2}$, consequentemente, $d^{p}b_{1} - d^{p}b_{2} = d^{p}f^{p}(a') = f^{p+1}(d^{p}a')$, e pela injetividade de $f^{p+1}$ podemos definir $a_{1} = (f^{p+1})^{-1}(d^{p}b_{1})$, $ a_{2} = (f^{p+1})^{-1}(d^{p}b_{2}) \in A^{p+1}$, então $(f^{p+1})^{-1}(d^{p}b_{1} - d^{p}b_{2}) =(f^{p+1})^{-1}(d^{p}b_{1}) - (f^{p+1})^{-1}(d^{p}b_{2})  = (f^{p+1})^{-1}f^{p+1}(d^{p}a')$, ou seja, $a_{1} - a_{2} = d^{p}a'$, portanto, $[a_{1}] - [a_{2}] = [d^{p}a'] = [0] \Rightarrow [a_{1}] = [a_{2}] \in H^{p}(A^{*})$. Conclusão: o mapa $\partial^{*}: H^{p}(C^{*}) \to H^{p+1}(A^{*})$, dado por $\partial^{*}[c] = [a_{1}]$, leva uma classe de equivalência em apenas uma classe de equivalência, como desejávamos. $\blacksquare$
	
	\textbf{Lemma 1 (sequência exata):} A sequência $H^{p}(B^{*}) \xrightarrow{g^{*}} H^{p}(C^{*}) \xrightarrow{\partial^{*}} H^{p+1}(A^{*})$ é exata.
	
	$\square$ Vamos mostrar a inclusão $Im(g^{*}) \subset Ker(\partial^{*})$. Seja $b \in H^{p}(B^{*})$, então $\partial^{*}g^{*}([b]) = \partial^{*}([g^{p}(b)]) = [(f^{p+1})^{-1}(d^{p}(g^{p})^{-1}(g^{p}(b)))] = [(f^{p+1})^{-1}(d^{p}(b))] = [d^{p}(f^{p})^{-1}(b)] = [0]$, portanto $g^{*}([b]) \in Ker(\partial^{*})$. Agora vamos mostrar a inclusão inversa $Ker(\partial^{*}) \subset Im(g^{*})$, para isso tomemos $[c] \in Ker(\partial^{*})$, isto é, $\partial^{*}([c]) = [(f^{p+1})^{-1}(d^{p}(g^{p})^{-1}(c))] = [0]$, e como $g^{p}$ é sobrejetor, existe $b \in B^{p}$ tal que $g^{p}(b) = c$. Tomando $a \in A^{p}$, onde $b = f^{p}(a)$, teremos 
	$$
	\begin{aligned}
	d^{p}b =& d^{p}f^{p}(a)
	\\
	g^{p+1}(d^{p}b) =&  g^{p+1}(d^{p}f^{p}(a)) 
	\\
	d^{p}g^{p}(b) =& d^{p}g^{p}(f^{p}(a)),
	\end{aligned}
	$$
	ou ainda, $d^{p}(g^{p}(b) - g^{p}(f^{p}(a)))=d^{p}(c - g^{p}(f^{p}(a)))=0 \Rightarrow c = g^{p}(f^{p}(a)) + dq$, para algum $q \in C^{p-1}$, com isso, 
	$$
	\begin{aligned}
	[c][c] =& [g^{p}(f^{p}(a)) + dq] 
	\\
	=& [g^{p}(f^{p}(a))] + [dq] 
	\\
	=& g^{*}([f^{p}(a)]) +[0] 
	\\
	=& g^{*}([a']) \in  Im(g^{*}).
	\end{aligned}
	$$
	Como temos $[c] \in Ker(\partial^{*})$ e $[c] = g^{*}[a'] \in Im(g^{*})$, o que implica que $Ker(\partial^{*}) \subset Im(g^{*})$, consequentemente, $Ker(\partial^{*}) = Im(g^{*})$, e temos uma sequência exata, como desejávamos. $\blacksquare$
	
	\textbf{Lemma 2 (sequência exata):} A sequência $H^{p}(C^{*}) \xrightarrow{\partial^{*}} H^{p+1}(A^{*}) \xrightarrow{f^{*}}  H^{p}(B^{*})$ é exata.
	
	$\square$ Vamos mostrar a inclusão  $Im(\partial^{*}) \subset Ker(f^{*})$. Seja $c \in H^{p}(C^{*})$, então $f^{*}\partial^{*}([c]) = f^{*}([(f^{p+1})^{-1}(d^{p}(g^{p})^{-1}(c))]) = [f^{p+1}((f^{p+1})^{-1}(d^{p}(g^{p})^{-1}(c)))] = [d^{p}(g^{p})^{-1}(c)] = [0]$, logo $\partial^{*}([c]) \in Ker(f^{*})$. Vamos mostrar a inclusão inversa $Ker(f^{*}) \subset Im(\partial^{*}) $. Tomando $[a] \in Ker(f^{*})$ teremos $f^{*}([a]) = [f^{p+1}(a)] = [0]$, assim podemos afirmar que $f^{p+1}(a) = d^{p}b$, para algum $b \in B^{p}$, e como por hipótese $f^{p}$ é injetor, então podemos escrever $a = (f^{p+1})^{-1}d^{p}b \Rightarrow [a] = [(f^{p+1})^{-1}d^{p}(g^{p})^{-1}g^{p}(b)] = \partial^{*}[g^{p}(b)] \in Im(\partial^{*})$, portanto temos $Ker(f^{*}) \subset Im(\partial^{*})$, logo, $Ker(f^{*}) = Im(\partial^{*})$, e temos uma sequência exata, como desejávamos. $\blacksquare$
	
	\textbf{Teorema (sequência longa exata da homologia):} Seja $0 \to A^{*} \xrightarrow{f} B^{*} \xrightarrow{g} C^{*} \to 0$ uma sequência curta exata de complexos de cadeias, então a sequência $$
	\dots \to H^{p}(A^{*}) \xrightarrow{f^{*}} H^{p}(B^{*}) \xrightarrow{g^{*}} H^{p}(C^{*}) \xrightarrow{\partial^{*}} H^{p+1}(A^{*}) \xrightarrow{f^{*}} H^{p+1}(B^{*}) \to \dots
	$$
	é exata.
	
	$\square$ É uma aplicação dos dois lemas anteriores. $\blacksquare$
	
	\textbf{Definição (mapas homotopicos entre cadeias):} Dois mapas de cadeiras $f,g : A^{*} \to B^{*}$ são ditos mapas homotopicos de cadeias se existe um mapa linear $s^{p}: A^{p} \to B^{p-1}$ tal que
	$$
	s^{p}d^{p} + d^{p}s^{p} = f-g : A^{p} \to B^{p}
	$$
	para todo $p$.
	
	\textbf{Lema (invariância homotópica):} Sejam $f,g : A^{*} \to B^{*}$ mapas homotópicos de cadeias, então $f^{*} = g^{*}: H^{p}(A^{*}) \to H^{p}(B^{*})$.
	
	$\square$ Por hipótese temos $s^{p}: A^{p} \to B^{p-1}$, então tomanto $[a] \in H^{p}(A^{*})$:
	$$
	\begin{aligned}
	(f-g)^{*}([a]) =& [(f-g)(a)]
	\\
	=& [(s^{p}d^{p} + d^{p}s^{p})(a)]
	\\
	=& [(s^{p}d^{p})(a)] + [(d^{p}s^{p})(a)]
	\\
	=& [(s^{p} \underbrace{d^{p})(a))}_{
		=0}] +[0]
	\\
	=& [0]+[0] = [0],
	\end{aligned}
	$$ 
	onde utilizamos o fato de que $a\in Z^{p}(A^{*})$, logo $d^{p}a = 0$. Portanto, $f^{*} = g^{*}$, como desejávamos $\blacksquare$
	
	Vamos demonstrar agora um teorema suja aplicação terá como um dos resultados mais importântes desse capítulo. O teorema de sequências de Mayer-vietoris. Ferramentas muito importante utilizada na determinação dos grupos de cohomologia.
	
	\textbf{Teorema:} Sejam $U_{1}, U_{2} \subseteq \mathbb{R}^{n}$ subconjuntos abertos e defina $U=U_{1} \cup U_{2}$ e as seguintes inclusões $i_{k}: U_{k} \to U$ e $j_{k}: U{1}\cap U{2} \to U_{k}$, onde $k \in \{1,2\}$. Então a sequência
	$$
	0 \to \Omega(U) \xrightarrow{I^{p}} \Omega^{p}(U_{1}) \oplus \Omega^{p}(U_{2}) \xrightarrow{J^{p}} \Omega^{p}(U_{1} \cap U_{2}) \to 0 
	$$
	é exata, onde $I^{p}(\omega) = (i^{*}_{1}(\omega), i^{*}_{2}(\omega))$ e $J^{p}(\omega_{1}, \omega_{2}) = j^{*}_{1}(\omega_{1})- j^{*}_{2}(\omega_{2})$.
	
	$\square$ Vamos mostrar a inclusão $Im(I^{p}) \subset Ker(J^{p})$. Tomando $\omega = \sum f_{I}dx_{I} \in \Omega^{p}(U)$ e avaliando $J^{p}(I^{p}(\omega)) = J^{p}(i^{*}_{1}(\omega), i^{*}_{2}(\omega)) = j^{*}_{1}(i^{*}_{1}(\omega))- j^{*}_{2}(i^{*}_{2}(\omega)) = \omega|_{U_{1} \cap U_{2}} - \omega|_{U_{1} \cap U_{2}} = 0$, logo $I^{p}(\omega) \in Ker(J^{p})$. Além disso $I^{p}$ é injetora pois $I^{p}(\omega) = (i^{*}_{1}(\omega), i^{*}_{2}(\omega)) = 0 \iff i^{*}_{1}(\omega)=\sum f_{I}\circ i_{1}(x) dx_{I} = 0$ e $i^{*}_{2}(\omega) = \sum f_{I}\circ i_{2}(x) dx_{I} = 0$, implicando em $f_{I}(x) = 0$ para $x \in U$.  Agora vamos demonstrar a inclusão inversa $Ker(J^{p})\subset Im(I^{p})$. Tomemos as p-formas $\omega_{k} = \sum f_{k,I} dx_{I} \in \Omega^{p}(U_{k})$ e considere $\omega_{k} \in Ker(J^{p})$, então $J^{p}(\omega_{1}, \omega_{2}) = 0 \Rightarrow j^{*}_{1}(\omega_{1}) = j^{*}_{2}(\omega_{2})$, isto é, $\sum (f_{1,I}\circ j_{1}) dx_{I} = \sum (f_{2,I}\circ j_{2}) dx_{I}$, o que nos leva a duas condições: $f_{1,I}\circ j_{1}(x) = f_{2,I}\circ j_{2}(x)$ caso $x \notin U_{1} \cap U_{2}$ ou $f_{1,I}(x) = f_{2,I}(x)$ caso $x \in U_{1} \cap U_{2}$, assim, com essas condições podemos definir a função 
	$$
	h_{I}(x) = \left\{
	\begin{array}{cc}
	f_{1,I}(x), & x \in U_{1} \\
	f_{2,I}(x), & x \in U_{2} \\
	\end{array},
	\right.
	$$
	consequentemente, $I^{p}(\sum h_{I}dx_{I}) = (\sum h_{I}\circ i_{1}(x)dx_{I}, \sum h_{I}\circ i_{2}(x)dx_{I}) = (\sum f_{1,I}(x)dx_{I}, \sum f_{2,I}(x)dx_{I}) = (\omega_{1}, \omega_{2}) \in Im(I^{p})$. Resta-nos mostrar que $J^{p}$ é sobrejetora. Sejam $k \in \{1,2\}$ e $p_{k}: U \to [0,1]$ uma partição da unidade de $U$ tal que $supp(p_{k}) \subset U_{k}$. Seja $F: U_{1} \cap U_{2} \to \mathbb{R}$ uma função suave, e definamos as funções
	$$
	F_{1}(x) = \left\{
	\begin{array}{cc}
	-F(x)p_{1}(x), &, x \in U_{1} \cap U_{2} \\
	0, & x \in U_{2}\backslash supp(p_{1})\\
	\end{array}
	\right.
	\; ,F_{2}(x) = \left\{
	\begin{array}{cc}
	F(x)p_{2}(x), & x \in U_{1} \cap U_{2} \\
	0, & x \in U_{1}\backslash supp(p_{2}) \\
	\end{array}.
	\right.
	$$ 
	Através da construção anterior, podemos afirmar que qualquer p-forma $\omega = \sum F_{I}dx_{I} \in \Omega^{p}( U_{1} \cap U_{2})$ pode ser escrita aplicando as tais condições para $F_{I}$, e assim teremos $\omega(x) = \omega_{k}(x)$ para $x \in U_{k}$, portanto, dada a p-forma $\omega \in \Omega^{p}( U_{1} \cap U_{2})$, existem $\omega_{k} \in \Omega^{p}(U)$ tal que $J^{p}(\omega_{1}, \omega_{2}) = \omega$, indicando a sobrejeção do mapa $J^{p}$, como desejávamos. $\blacksquare$
	
	\textbf{Lema:} Os mapas $I: \Omega^{*}(U) \to \Omega^{*}(U_{1}) \oplus \Omega^{*}(U_{2})$ e $J: \Omega^{*}(U_{1}) \oplus \Omega^{*}(U_{2}) \to \Omega^{*}(U_{1} \cap U_{2})$, definidos anteriormente, são mapas de cadeias.
	
	$\square$ Tomando $\omega \in \Omega^{p}(U)$, então 
	$$
	I^{p+1}(d^{p}\omega) = (i^{*}_{1}(d^{p}\omega), i^{*}_{2}(d^{p}\omega)) = (d^{p}i^{*}_{1}(\omega_{1}), d^{p}i^{*}_{2}(\omega_{2})) = d^{p}(i^{*}_{1}(\omega_{1}), i^{*}_{2}(\omega_{2})) = d^{p}I^{p}(\omega),$$
	logo $I^{p+1}d^{p} =  d^{p}I^{p}$. Analogamente:
	$$
	J^{p+1}(d^{p}\omega_{1}, d^{p}\omega_{2}) = j^{*}_{1}(d^{p}\omega_{1}) - j^{*}_{2}(d^{p}\omega_{2}) = d^{p}j^{*}_{1}(\omega_{1}) - d^{p}j^{*}_{2}(\omega_{2}) = d^{p}(j^{*}_{1}(\omega_{1}) - j^{*}_{2}(\omega_{2})) = d^{p}J^{p+1}(\omega_{1}, \omega_{2}),
	$$
	logo $J^{p+1}d^{p} =  d^{p}J^{p}$, como desejávamos. $\blacksquare$
	
	\textbf{Teorema (Sequência de Mayer-Vietoris):} Sejam $U_{1}, U_{2} \subseteq \mathbb{R}^{n}$ conjuntos abertos e $U = U_{1} \cup U_{2}$, então a sequência
	$$
	\dots \to H^{p}(U) \xrightarrow{I^{*}} H^{p}(U_{1}) \oplus H^{p}(U_{2}) \xrightarrow{J^{*}} H^{p}(U_{1} \cap U_{2}) \xrightarrow{\partial^{*}} H^{p+1}(U) \to \dots
	$$
	é exata.
	
	$\square$ Sabemos que os mapas $I: \Omega^{*}(U) \to \Omega^{*}(U_{1}) \oplus \Omega^{*}(U_{2})$ e $J: \Omega^{*}(U_{1}) \oplus \Omega^{*}(U_{2}) \to \Omega^{*}(U_{1} \cap U_{2})$ são mapas de cadeias, além disso, a sequência 
	$$
	0 \to \Omega^{p}(U) \xrightarrow{I^{p}} \Omega^{p}(U_{1}) \oplus \Omega^{p}(U_{2}) \xrightarrow{J^{p}} \Omega^{p}(U_{1} \cap U_{2}) \to 0 
	$$
	é exata curta, consequentemente, a sequência de homologia
	$$
	\dots \to H^{p}(U) \xrightarrow{I^{*}} H^{p}(U_{1}) \oplus H^{p}(U_{2}) \xrightarrow{J^{*}} H^{p}(U_{1} \cap U_{2}) \xrightarrow{\partial^{*}} H^{p+1}(U) \to \dots
	$$
	é exata. $\blacksquare$
	
	\textbf{Definições (homotopia):} Dizemos que $f, g : A \to B$ são homotópicamente equivalentes se existe uma função contínua $h: A \times [0,1] \to B$ tal que $h(x,0) = f(x)$ e $h(x,1) = g(x)$, além disso, escrevemos $f \sim g$. Dizemos que $A, B$ são homotópicamente equivalentes se existem mapas contínuos $f: A \to B$ e $g: B \to A$ tais que $f\circ g \sim Id_{B}$ e $g\circ f \sim Id_{A}$.
	
	
	\textbf{Exemplo $S^{1} \sim \mathbb{R}^{2} \backslash \{0\}$:} Sabemos que se dois conjuntos $A, B$ forem homotópicamente equivalentes, então seus grupos de cohomologia serão iddenticos. Vamos utilizar esse fato para calcular os grupos de cohomologia de $S^{1}$. Façamos $f: \mathbb{R}^{2} \backslash \{0\} \to S^{1}$ e $g: S^{1} \to \mathbb{R}^{2} \backslash \{0\}$, tal que $f(x) = x/||x||$ e $g(x) = x$. Temos imediatamente que $f \circ g \sim Id_{S^{1}}$, agora tomemos $h: S^{1} \times [0,1] \to \mathbb{R}^{2} \backslash \{0\}$ tal que 
	$h(x,t) = x/||x||^{t}$, de modo que, $h(x, 0) = x = Id_{\mathbb{R}^{2} \backslash \{0\}}(x)$ e também $h(x, 1) = (g \circ f)(x)$, portanto $g \circ f \sim Id_{\mathbb{R}^{2} \backslash \{0\}}$, como desejávamos. Conclusão, $S^{1} \sim \mathbb{R}^{2} \backslash \{0\}$ e com isso $H^{p}(S^{1}) = H^{p}(\mathbb{R}^{2} \backslash \{0\})$ onde $0 \leq p \leq 2$.
	Definindo $U_{1} = \mathbb{R}^{2} \backslash \{(x_{1}, x_{2}) \in \mathbb{R}^{2} : x_{1} \geq 0 \}$, $U_{2} = \mathbb{R}^{2} \backslash \{(x_{1}, x_{2}) \in \mathbb{R}^{2} : x_{2} \leq 0 \}$ e por construção $U_{1}, U_{2}$ são subconjuntos estrelados e abertos do plano, e por fim definiremos $\mathbb{R}^{2} \backslash \{0\} = U_{1} \cup U_{2}$ e $U_{1}\cap U_{2} = \mathbb{R}^{2}_{+} \cup \mathbb{R}^{2}_{-}$. Temos a sequência:
	$$
	\begin{aligned}
	& 0 \to H^{0}(\mathbb{R}^{2} \backslash \{0\} ) \xrightarrow{I^{0}} H^{0}(U_{1}) \oplus H^{0}(U_{2}) \xrightarrow{J^{0}} H^{0}(\mathbb{R}^{2}_{+} \cup \mathbb{R}^{2}_{-}) \xrightarrow{\partial^{*}} 
	\\
	&\xrightarrow{\partial^{*}} H^{1}(\mathbb{R}^{2} \backslash \{0\} ) \xrightarrow{I^{1}} H^{1}(U_{1}) \oplus H^{1}(U_{2}) \xrightarrow{J^{1}} H^{1}(\mathbb{R}^{2}_{+} \cup \mathbb{R}^{2}_{-}) \xrightarrow{\partial^{*}} 
	\\
	&\xrightarrow{\partial^{*}} H^{2}(\mathbb{R}^{2} \backslash \{0\} ) \xrightarrow{I^{2}} H^{2}(U_{1}) \oplus H^{2}(U_{2}) \xrightarrow{J^{2}} H^{2}(\mathbb{R}^{2}_{+} \cup \mathbb{R}^{2}_{-}) \to 0.
	\end{aligned}
	$$
	Pelo lema de Poincaré sabemos calcular a homologia de conjuntos abertos e estrelados do plano, e com isso $H^{0}(U_{k}) \cong \mathbb{R}$ e $H^{p}(U_{k}) = 0$ com $k \in \{1,2\}$ e $p\geq 1$, e a partir desse resultado vamos calcular toda a sequência. Podemos afirmar que $H^{0}(\mathbb{R}^{2} \backslash \{0\}) \cong \mathbb{R}$ pois já vimos que a dimensão do 0-ésimo grupo de homologia é identico ao número componentes conexas do conjunto, e como $H^{0}(\mathbb{R}^{2} \backslash \{0\} )$ é conexo, isto é, contém apenas uma componente conexa, logo $H^{0}(\mathbb{R}^{2} \backslash \{0\}) \cong \mathbb{R}$. Sabemos que $\mathbb{R}^{2}_{+} \cup \mathbb{R}^{2}_{-}$ é uma união desconexa de conjuntos abertos e estrelados, então $H^{0}(\mathbb{R}^{2}_{+} \cup \mathbb{R}^{2}_{-}) =  H^{0}(\mathbb{R}^{2}_{+}) \oplus H^{0}(\mathbb{R}^{2}_{-})\cong \mathbb{R} \oplus \mathbb{R}$, e consequentemente $H^{p}(\mathbb{R}^{2}_{+} \cup \mathbb{R}^{2}_{-}) \cong 0\oplus 0$ para $p\geq 1$, pelo Lema de Poincaré. Assim temos a segunda parte da sequência $H^{0}(U_{1}) \oplus H^{0}(U_{2})  \xrightarrow{\partial^{*}} H^{1}(\mathbb{R}^{2} \backslash \{0\} ) \xrightarrow{I^{1}} 0 \oplus 0 $, e como $\partial^{*}$ é injetor e $Im(\partial^{*}) = Ker(I^{1}) = H^{1}(\mathbb{R}^{2} \backslash \{0\})$, logo, $\partial^{*}$ é sobrejetor, portanto, um isomorfismo. Consequentemente, $H^{0}(U_{1}) \oplus H^{0}(U_{2}) \cong \mathbb{R} \oplus \mathbb{R}$ e $ H^{0}(U_{1}) \oplus H^{0}(U_{2}) \cong H^{1}(\mathbb{R}^{2} \backslash \{0\} ) $, logo, $H^{1}(\mathbb{R}^{2} \backslash \{0\} ) \cong \mathbb{R} \oplus \mathbb{R}$. Por fim, temos a sequência $0 \cong H^{1}(\mathbb{R}^{2}_{+} \cup \mathbb{R}^{2}_{-})\xrightarrow{\partial^{*}} H^{2}(\mathbb{R}^{2} \backslash \{0\} ) \xrightarrow{I^{2}} H^{2}(U_{1}) \oplus H^{2}(U_{2}) \cong 0 \oplus 0$. Como $\partial^{*}$ é injetor e $Im(\partial^{*}) = Ker(I^{2}) = H^{2}(\mathbb{R}^{2} \backslash \{0\} )$, logo, $\partial^{*}$ é sobrejetor, portanto é um isomorfismo. Assim, $ H^{1}(\mathbb{R}^{2}_{+} \cup \mathbb{R}^{2}_{-}) \cong 0$ e $H^{2}(\mathbb{R}^{2} \backslash \{0\} ) \cong H^{1}(\mathbb{R}^{2}_{+} \cup \mathbb{R}^{2}_{-})$, portanto $H^{2}(\mathbb{R}^{2} \backslash \{0\} ) \cong 0$.
	
	Conclusão, $H^{0}(\mathbb{R}^{2} \backslash \{0\}) \cong \mathbb{R}$, $H^{1}(\mathbb{R}^{2} \backslash \{0\}) \cong \mathbb{R} \oplus \mathbb{R}$ e  $H^{p}(\mathbb{R}^{2} \backslash \{0\}) \cong 0$ para $p \geq 2$.
\end{document}