\documentclass[twoside,openright,titlepage,numbers=noenddot,headinclude,  lineheaders footinclude=true,cleardoublepage=empty,
                                BCOR=5mm,paper=a4,fontsize=12pt ]{scrbook} 
\usepackage[usenames,dvipsnames,svgnames,table]{xcolor}
\usepackage[utf8]{inputenc}
\usepackage[eulerchapternumbers,beramono]{classicthesis} 
\usepackage{estilo}
\usepackage[T1]{fontenc}                       
\usepackage{graphicx}
\usepackage[brazil,portuguese, english]{babel}
\usepackage{hyperref}
\usepackage{amsmath}
\usepackage{amsfonts}
\usepackage{amssymb}
\usepackage{amsthm}
\usepackage{breqn}
\usepackage{pdfpages}
\usepackage{makeidx}
\usepackage{blindtext}

\usepackage{indentfirst}
\usepackage{tikz}
\usetikzlibrary{calc,shapes}
% \usepackage{enumitem}
\usepackage{mathtools}
\usepackage{mathrsfs}
\usepackage{tikz-cd}
\usepackage[all,cmtip]{xy}

\addtolength{\headsep}{0.6cm}
\setlength{\textheight}{22.7cm}
\setlength{\textwidth}{16.2cm}
\setlength{\oddsidemargin}{0cm}
\setlength{\evensidemargin}{0cm}

\linespread{1.3}
%%%%%%%%%%%%%IMPORTANTE%%%%%%%%%%%%%%%%%%%
%%%%%%%%%%%%%%%%%%%%%%%%%%%%%%%%%%%%%%%%%%
%Para acentos nos títulos use o comando latex%%%
%%% \'a, \~a \^e, etc%%%%%%%%%%%%%%%%%%%%%%

%%%%%%%%%%%%%%%%%%%%%%%%%%%%%%%%%%%%%%%%%%%%%%%%%%%%%%%%%%%%%%%%%%%%%%%%%%%%%%%%%%%%%%
%%%%%%%%%%%%%%%%%%%%%%%%%%%%%%%%%%%%%%%%%%%%%%%%%%%%%%%%%%%%%%%%%%%%%%%%%%%%%%%%%%%%%%
%%%%%%%%%%%%%%%%%%%%%%%%%%%%%%%%%%%%%%%%%%%%%%%%%%%%%%%%%%%%%%%%%%%%%%%%%%%%%%%%%%%%%%
%%%%%%%%%%%%%%%%%%%%%%%%%%%%%%%%%%%%%%%%%%%%%%%%%%%%%%%%%%%%%%%%%%%%%%%%%%%%%%%%%%%%%%

\newtheorem{teo}{Teorema}[chapter]
\newtheorem{lema}[teo]{Lema}
\newtheorem{prop}[teo]{Proposição}
\newtheorem{cor}[teo]{Corolário}
\newtheorem{defn}[teo]{Definição}



%%%%%%%%%%%%%%%%%%%%%%%%%%%%%%%%   CONFIGURAÇÔES        %%%%%%%%%%%%%%%%%%%%%%%%%%%%%%%%%%%%
%%%%%%%%%%%%%%%%%%%%%%%%%%%%%%%%%%%%%%%%%%%%%%%%%%%%%%%%%%%%%%%%%%%%%%%%%%%%%%%%%%%%%%%%%%%%


\newcommand{\autor}{Nome completo do aluno}
%  Se for do sexo feminino, descomente a linha a seguir.
% \def\femaleAuthor{}
%  Substituir 'Título da defesa' pelo título da defesa.
\newcommand{\titulo}{Título da defesa}
%  Se estiver no programa de mestrado, descomente a linha a seguir.
 \def\mestrado{}
%  Substituir 'Nome completo do orientador' pelo nome completo do seu
% orientador.
\newcommand{\orientador}{Nome completo do orientador}
%  Se for orientado por uma mulher, descomente a linha a seguir.
% \def\femaleOrientador{}
%  Substituir 'Nome completo do coorientador' pelo nome completo do seu
% coorientador. Caso não tenha coorientador, comente a linha a seguir.
\newcommand{\coorientador}{Nome completo do coorientador}
%  Se for coorientado por uma mulher, descomente a linha a seguir.
% \def\femaleCoorientador{}
%  Substituir 'Ano' pelo ano em que ocorreu sua defesa.
\newcommand{\fomento}{da ``Nome da Agência de fomento''}

%%%%%%%%%%%%%%%%%%%%%%%%%%%%%%%Versão
%%Se for a versão final, com as correções sugeridas pela banca descomente a linha abaixo
 \def\versaofinal{}

\newcommand{\ano}{2013}
\newcommand{\centro}{Centro de Matemática, Computação e Cognição \xspace}
\newcommand{\titulacao}{Mestre em Matemática \xspace}
\newcommand{\palavraschaves}{ TCC, Trabalho, Modelo}
\newcommand{\keywords}{aubergine,carrot, radish} 

%%% COMANDOS CUSTOMIZADOS%%%%%%
%%%%%%%%%%%%%%%%%%%%%%%%%%%%%%%%%%%%%%%%%%%%%%%%%%%%%%
%%%%%%%%%%%%%%%%%%%%%%%%%%%%%%%%%%%%%%%%%%%%%%%%%%%%%%
\newcommand{\celula}[2]{D^{#1}_{#2}}
\newcommand{\celulabordo}[2]{\partial D^{#1}_{#2}}
\newcommand{\classe}[1]{[#1]}
\newcommand{\cohomologia}[2]{H^{#1}(#2)}
\newcommand{\cohomologiadual}[2]{H^{#1}(#2)^{*}}
\newcommand{\cohomologiacompac}[2]{H^{#1}_{c}(#2)}
\newcommand{\cohomologiacompacdual}[2]{H^{#1}_{c}(#2)^{*}}
\newcommand{\dparcial}[2]{\frac{\partial #1}{\partial #2}}
\newcommand{\funcaocond}[5]{
	#1 = 
	\left\{
	\begin{array}{cc}
		#2, & #3\\
		#4, & #5\\
	\end{array}
	\right.
}
\newcommand{\skeleton}[1]{X^{(#1)}}
\newcommand{\homologia}[2]{H_{#1}(#2)}
\newcommand{\homologiarel}[3]{H_{#1}(#2,#3)}
\newcommand{\homologiarelcel}[3]{H_{#1}(D^{#2}_{#3}, \partial D^{#2}_{#3})}
\newcommand{\homologiarelskele}[3]{H_{#1}(X^{(#2)}, X^{(#3)})}
\newcommand{\homologiarelskelesimpl}[2]{H_{#1}(X^{(#2)}, X^{(#2-1)})}
\newcommand{\iprod}[2]{\langle #1, #2 \rangle}
\newcommand{\liederivada}[1]{\mathcal{L}_{#1}}
\newcommand{\real}[1]{\mathbb{R}^{#1}}
\newcommand{\realprojetivo}[1]{\mathbb{R}P^{#1}}
\newcommand{\somadir}[1]{\bigoplus \limits_{#1}}

\newcommand{\tese}[3]{\vspace{2mm} \textit{\textbf{#1}}: \textit{#2} \par $\square$ #3 \par $\blacksquare$}
\newcommand{\innerprod}[2]{\langle #1, #2 \rangle}
\newcommand{\morsefunc}[1]{\mathcal{M}o(#1)}
\newcommand{\pontocritico}[1]{\textit{Crit}(#1)}
\newcommand{\suavefunc}[1]{C^{\infty}(#1, \real{})}
\newcommand{\vermelho}[1]{{\color{red}#1}}
\newcommand{\bigmodulo}[1]{\Bigm\lvert #1 \Bigm\lvert }
\newcommand{\bigparenteses}[1]{\big( #1 \big) }
%%% COMANDOS CUSTOMIZADOS%%%%%%
%%%%%%%%%%%%%%%%%%%%%%%%%%%%%%%%%%%%%%%%%%%%%%%
%%%%%%%%%%%%%%%%%%%%%%%%%%%%%%%%%%%%%%%%%%%%%%%

\makeindex

%%%%%%%%%%%%%%%%%%%%%%%%%%%%%%%%%%%%%%%%%%%%%%%%%%%%%%%%%%%%%%%%%%%%%%%%%%%%%%%%%%%%%%%%%%%%
%%%%%%%%%%%%%%%%%%%%%%%%%%%%%%%%%%%%%%%%%%%%%%%%%%%%%%%%%%%%%%%%%%%%%%%%%%%%%%%%%%%%%%%%%%%%
%%%%%%%%%%%%%%%%%%%%%%%%%%%%%%%%%%%%%%%%%%%%%%%%%%%%%%%%%%%%%%%%%%%%%%%%%%%%%%%%%%%%%%%%%%%%
%%%%%%%%%%%%%%%%%%%%%%%%%%%%%%%%%%%%%%%%%%%%%%%%%%%%%%%%%%%%%%%%%%%%%%%%%%%%%%%%%%%%%%%%%%%%




%%%%%%%%%%%%%%%%%%%%%%%%%% Início do texto da defesa %%%%%%%%%%%%%%%%%%%%%%%%%%%
\begin{document}

\thispagestyle{empty}
\includegraphics[height=2.7cm, keepaspectratio=true]{figuras/logo-ufabc} \hfill

\vspace*{2.8cm}
\begin{center}
  % O tamanho da fonte deve ser 16pt.
  % Deve-se utilizar caixa alta.
  {\Large \scshape \autor}
\end{center}
\vspace{5cm}
\begin{center}
  % O tamanho da fonte deve ser 16pt em negrito.
  % Deve-se utilizar caixa alta.
  {\huge \scshape \bfseries \titulo}
\end{center}
\vfill
\begin{center}
  % O tamanho da fonte deve ser 12pt em negrito.
  % Deve-se utilizar caixa alta.
  { Durante o desenvolvimento deste trabalho o autor
recebeu auxílio financeiro \fomento \\} \vspace{0.8cm}
  {\bfseries Santo  André, \ano}
\end{center}
  % Não edite esse arquivo.
\thispagestyle{empty}\newpage\mbox{}\thispagestyle{empty}\newpage\thispagestyle{empty} % Pagina em branco.
\frontmatter

\thispagestyle{empty}
\includegraphics[height=2.7cm, keepaspectratio=true]{figuras/logo-ufabc} \hfill

\vspace{1.37cm}
\begin{center}
  {\large  Universidade Federal do ABC
  \vspace{.5cm}}
\end{center}
\vspace{.7cm}
\begin{center}
  {\large \autor}
\end{center}
\vspace{.7cm}
\begin{center}
  {\LARGE \bfseries \titulo}
\end{center}
\vspace{1cm}
{\bfseries
\noindent
Orientador\ifx\femaleOrientador\undefined
\else
a\fi: Prof\ifx\femaleOrientador\undefined
\else
a\fi. Dr\ifx\femaleOrientador\undefined
\else
a\fi. \orientador
\vspace{.25cm}

\ifx\coorientador\undefined
\else
\noindent
Coorientador\ifx\femaleCoorientador\undefined
\else
a\fi: Prof\ifx\femaleCoorientador\undefined
\else
a\fi. Dr\ifx\femaleCoorientador\undefined
\else
a\fi. \coorientador
\fi
}

\vspace{1.4cm}
\begin{flushright}
  \begin{minipage}[c]{.6\textwidth}
    \begin{flushleft}
      \ifx\mestrado\undefined
      \noindent Tese de doutorado
      \else
    \noindent   Dissertação de mestrado
      \fi
      apresentada ao Centro de Matemática, Computação e Cognição para \\ \noindent  obtenção do título de
      Mestre em Matemática 
    \end{flushleft}
  \end{minipage}
\end{flushright}
\vspace{1.3cm}
  \ifx\versaofinal\undefined
\noindent 
{\footnotesize \scshape
Esta é a versão original da tese, tal como\\
submetida à Comissão Julgadora.\\
}
\else
\noindent 
{\footnotesize \scshape
Este exemplar corresponde à versão final da 
\ifx\mestrado\undefined
tese
\else
dissertação
\fi \\
defendida 
\ifx\femaleAuthor\undefined
pelo aluno 
\else
pela aluna
\fi
\autor,\\
e orientada pel\ifx\femaleOrientador\undefined
o\else
a\fi{} Prof\ifx\femaleOrientador\undefined
\else
a\fi. Dr\ifx\femaleOrientador\undefined
\else
a\fi. \orientador.
}
\fi



\vspace{1.2cm}


\vfill
\begin{center}
  {\small \scshape \bfseries Santo André, \ano}
\end{center}




%%  A ficha catalográfica deve estar no verso da folha de rosto.
%  O arquivo ficha-catalografica.pdf deve ser sobrescrito com uma cópia
% do arquivo pdf que a biblioteca lhe enviar.
\includepdf{ficha-catalografica}
%
%  A folha de aprovação deve ser assinada pelos membros da banca apos a
% defesa.
%  Substitua o arquivo folha-de-aprovacao.pdf por uma copia escaneada.
%\includepdf{folha-aprovacao}
%
%  Se não for incluir a dedicatória, comentar a linha abaixo.
\chapter*{Dedicat\'oria}


Dedico essa tese ....

 


%
%  Se não for incluir os agradecimentos, comentar a linha abaixo.
\chapter*{Agradecimentos}

Agradeço a 

%
%  Se não for incluir a epigrafe, comentar a linha abaixo.
%%TODO Escrever o modelo da epigrafe.
\chapter*{}
\vfill
\begin{flushright}
  Ohana means family.\\
Family means nobody gets left behind, or forgotten.\\
— Lilo \& Stitch

\end{flushright}

%
\chapter*{Resumo}
Insira o resumo e as palavras chaves.

\vspace{.5cm}
\textbf{Palavras-chave}:\palavraschaves




\chapter*{ Abstract}
\selectlanguage{english}

Insira o Abstract e as Keywords


\vspace{.5cm}
\textbf{Keywords}:\palavraschaves

\selectlanguage{portuguese}

%
\tableofcontents
%
\mainmatter


%%%%%%%%%%%%%%Capitulos


%%%%%%%%%%%%%%%%%%%%%%%%%%%%%%%%%%%%%%%%%%%%%%%%%%%%%%%%%%%%%%%%%%%%%
%%%%%%%%%%%%%%%%%%%%%%%%%%%%%%%%%%%%%%%%%%%%%%%%%%%%%%%%%%%%%%%%%%%%%
%%%%%%%%%%%%%%%%%%%%%%%%%%%%%%%%%%%%%%%%%%%%%%%%%%%%%%%%%%%%%%%%%%%%%
%%%%%%%%%%%%%%%%%%%%%%%%%%%%%%%%%%%%%%%%%%%%%%%%%%%%%%%%%%%%%%%%%%%%%
%%%%%%%%%%%%%%%%%%%%%%%%%%%%%%%%%%%%%%%%%%%%%%%%%%%%%%%%%%%%%%%%%%%%%
%%%%%%%%%%%%%%%%%%%%%%%%%%%%%%%%%%%%%%%%%%%%%%%%%%%%%%%%%%%%%%%%%%%%%
%%%%%%%%%%%%%%%%%%%%%%%%%%%%%%%%%%%%%%%%%%%%%%%%%%%%%%%%%%%%%%%%%%%%%
%%%%%%%%%%%%%%%%%%%%%%%%%%%%%%%%%%%%%%%%%%%%%%%%%%%%%%%%%%%%%%%%%%%%%
%%%%%%%%%%%%%%%%   Apague as linhas abaixo... 
%%%%%%%%%%%%%%%%    Apenas para visualização inicial

%%%%%% Início do modelo de Texto
\chapter{CW-Homologia}
\section{Homologia Formal}
Sejam A um anel comutativo com unidade, $C_{k}$ com $k \in \mathbb{Z}$ $A$-módulos e $\partial_{k}: C_{k} \to C_{k-1} $ homomorfismos tais que $\partial_{k-1}\circ\partial_{k} = 0$ (o homomorfismo trivial), então denotamos por complexo de cadeias a sequência $\mathcal{C} = (C_{k}, \partial_{k})$. Cada elemento $\alpha \in C_{k}$ é chamado k-cadeia.

\begin{defn}
	(A-módulo de homologia) Chamamos o homomorfismo $\partial_{k}: C_{k} \to C_{k-1} $ de operador de bordo. Dados $\alpha \in C_{k}$, $\beta \in C_{k-1}$  é um k-ciclo se $\partial_{k }\alpha=0$ e se $\beta =  \partial_{k }\alpha$ dizemos que $\beta \in C_{k-1}$ é um (k-1)-bordo da k-cadeia $\alpha$. O A-módulo de todos os k-cíclos denotamos por $Z_{p} \subseteq C_{p}$ e o  A-módulo de todos os k-bordos por $B_{p} \subseteq C_{p}$. O A-módulo quociente $H_{k} = Z_{k} / B_{k}$ é chamado grupo de homologia k-dimensional, onde $[\alpha] \in H_{k}$ é tal que $[\alpha] = \{\alpha+\partial_{k+1}\gamma :\gamma \in  C_{k+1}\}$ = $\alpha + B_{k}$. A homologia do complexo de cadeias $\mathcal{C}$ é denotado pela soma formal $H_{*}(\mathcal{C}) = \bigoplus_{k \in \mathbb{Z}} H_{k}(\mathcal{C})$.
\end{defn}

A partir desse momento omitiremos a indexação dimensional do operador bordo dos complexos de cadeia, sendo que a dimensão estará implícita no contexto. Sejam $\mathcal{C}, \mathcal{D}$ dois complexos de cadeias, um morfismo $f: \mathcal{C} \to \mathcal{D}$ é uma sequência de homomorfismos $f_{k}: C_{k} \to D_{k}$ tais que $f_{k-1}\circ\partial = \partial\circ f_{k}$, ou seja, comutam com o operador bordo. Uma consequência imediata dessa definição é que $f_{k}(Z_{k}(\mathcal{C})) \subseteq Z_{k}(\mathcal{D})$ e $f_{k}(B_{k}(\mathcal{C})) \subseteq B_{k}(\mathcal{D})$, ou seja, leva ciclos em ciclos e bordos em bordos. Podemos definir na passagem ao quociente o homomorfismo $(f_{k})_{*}:H_{k}(\mathcal{C}) \to H_{k}(\mathcal{D})$ tal que $(f_{k})_{*}([\alpha]) = [f_{k}(\alpha)]$, que por brevidade denotaremos $f_{*}:H_{k}(\mathcal{C}) \to H_{k}(\mathcal{D})$ o homomorfismo induzido pelos morfismos de cadeias.

\begin{defn}
	(Sequências exatas) Uma sequência de homomorfismos de $A$-módulos
	$$
	\xymatrix{
		\dots \ar[r] & \homologiarel{k}{Y}{Z} \ar[r]_{i_{*}} & \homologiarel{k}{X}{Z} \ar[r]^{j_{*}} & \homologiarel{k}{X}{Y} \ar[r]^{\delta_{*}} & \homologiarel{k-1}{Y}{Z} \ar[r] & 1\dots
	}
	$$
	
	
\end{defn}

\section{Homologia singular}
Sejam $X$ um espaço topológico e $A$ um anel, então:
\begin{defn}
	(Simplexo singular) Denotaremos por $\Delta_{k}$ o simplexo k-dimesional cujos vertices $e_{0}, \dots, e_{k}$ formam um base canônica de $\real{k+1}$ de modo que $\Delta_{k} = \{(x_{0}, \dots, x_{k}) \in \real{k+1}: x_{j}\geq 0, \;\sum_{j}x_{j}=1\}$. Um k-simplexo singular no espaço topológico $X$ é uma aplicação contínua $\sigma:\Delta_{k} \to X$. Denotaremos por $S_{k}(X)$ o A-módulo livre gerado pelos k-cadeias singulares de $X$, consequentemente, seus elementos são as combinações lineares $\alpha = \sum_{\sigma} \alpha_{\sigma}.\sigma $ de k-simplexos singulares $\sigma$ e $\alpha_{\sigma} \in A$.
\end{defn}

\begin{defn}
	(Operador bordo) A i-ésima face do k-simplexo singular $\sigma: \Delta_{k} \to X$ é o (k-1)-simplexo singular $\partial_{i}\sigma:\Delta_{k-1} \to X$ onde $\partial_{i}\sigma = \sigma(t_{0}, \dots, t_{i-1},0,t_{i+1}, \dots, t_{k-1})$. O operador bordo é o homomorfismo denotado por $\partial : S_{k} \to S_{k-1}$ onde $\partial\sigma = \sum_{i} (-1)^{i}\partial_{i}\sigma$.
\end{defn}
Pode-se mostrar que o operador bordo satisfaz $\partial_{k-1}\circ\partial_{k} = 0$, ou seja, é o homomorfismo trivial. Com isso, podemos definir o complexo singular o espaço $X$ como sendo a sequência $\mathcal{S} = (S_{k}(X), \partial_{k})$. Desse modo teremos os k-clíclos e os (k-1)-bordos tal que, dado $\alpha \in S_{k}$ e $\beta \in S_{k-1}$, se $\partial\alpha = 0$, então $\alpha \in Z_{k}(X)$ (o A-módulo dos k-cíclos) e se $\beta = \partial\alpha$, então $\beta\in B_{k-1}(X)$ (o A-módulo dos (k-1)-bordos).

\begin{defn}
	(Homologia singular) O grupo de homologia k-dimensional de X é definido por $H_{k}(S(X)) = Z_{k}(S(X))/B_{k}(S(X))$, que por abreviação denotaremos $H_{k}(X)=H_{k}(S(X))$. A homologia do complexo singular de $X$ é definida pela soma formal $H_{*}(X) = \bigoplus_{k \in \mathbb{Z}}H_{k}(X)$.
\end{defn}

\begin{defn}
	(Sequência da triplas) Sejam $X, Y, Z$ espaços topológicos tais que $Z \subseteq Y \subseteq X$. Então a sequência da tripla $(X,Y,Z)$ é dada por:

$$\xymatrix{\ldots \ar[r] & H\Delta(A) \ar[r]^{i} \ar[d]^{id}& H\Delta(M) \ar[r]^{p} \ar[d]^{\Phi}&
	H\Delta(A^*)\ar[r]^{{\partial}}\ar[d]^{id} & H\Delta(A)\ar[d]^{id} \ar[r] & \ldots\\
	\ldots \ar[r] & H(c(A)) \ar[r]^{i} & H(c(M)) \ar[r]^{p} & H(c(A^*)) & H(c(A)) \ar[r] & \ldots }$$

	onde $i_{*},\;j_{*}$ são inclusões e $\delta_{*}$ o homomorfismo de conexão.
\end{defn}


\subsection{title}
\begin{teo}
 teste teo \blindtext 
\end{teo}
\begin{lema}
 Lema Lesma Lemur
\end{lema}

\begin{teo}
 testeshaskskdhs fhsdhfdhf fsdhfhdklfhsd
\end{teo}
\begin{teo}
 testeshaskskdhs fhsdhfdhf fsdhfhdklfhsd
\end{teo}\begin{teo}
 testeshaskskdhs fhsdhfdhf fsdhfhdklfhsd
\end{teo}

%%%% Fim do modelo de texto 


%%%%%%%%%%%%%%%%%%%%%%%%%%%%%%%%%%%%%%%%%%%%%%%%%%%%%%%%%%%%%%%%%%%%%
%%%%%%%%%%%%%%%%%%%%%%%%%%%%%%%%%%%%%%%%%%%%%%%%%%%%%%%%%%%%%%%%%%%%%
%%%%%%%%%%%%%%%%%%%%%%%%%%%%%%%%%%%%%%%%%%%%%%%%%%%%%%%%%%%%%%%%%%%%%
%%%%%%%%%%%%%%%%%%%%%%%%%%%%%%%%%%%%%%%%%%%%%%%%%%%%%%%%%%%%%%%%%%%%%
%%%%%%%%%%%%%%%%%%%%%%%%%%%%%%%%%%%%%%%%%%%%%%%%%%%%%%%%%%%%%%%%%%%%%
%%%%%%%%%%%%%%%%%%%%%%%%%%%%%%%%%%%%%%%%%%%%%%%%%%%%%%%%%%%%%%%%%%%%%
%%%%%%%%%%%%%%%%%%%%%%%%%%%%%%%%%%%%%%%%%%%%%%%%%%%%%%%%%%%%%%%%%%%%%
%%%%%%%%%%%%%%%%%%%%%%%%%%%%%%%%%%%%%%%%%%%%%%%%%%%%%%%%%%%%%%%%%%%%%
%  Se não for utilizar apêndices, comentar a linha abaixo.
\backmatter
\appendix
\chapter{Ap\^endice}




\addcontentsline{toc}{chapter}{Referência Bibliográfica}

%%%%%%%%%%%%%%%%%%%BIbliografia
\begin{thebibliography}{9999}  
\bibitem[GJT]{Guivarc'h}GUIVARC'H, Y.; JI, L.; TAYLOR, J. C.;
\emph{Compactifications of Symmetric Spaces, }Birkh\"{a}user, 1998.

\bibitem[Hu]{Hu}HUMPHREYS, J.E., \emph{Linear Algebraic \ Groups, }Springer
Verlag, New York, 1975.

\bibitem[Ma]{Margulis}MARGULIS, M. A.; \emph{Discrete Subgroups of Semisimples
Lie Groups, }Springer Verlag 1990.

\bibitem[Ron1]{Ronan}RONAN, MARK; \emph{Lecture on Buildings, }Perspectives in
Mathematics, Academic Press, 1989.
\end{thebibliography}



\clearpage
\phantomsection
\addcontentsline{toc}{chapter}{Índice Remissivo}
\printindex
\end{document}
