\documentclass[12pt]{book}
\usepackage[portuguese]{babel}
\usepackage{graphicx}
\usepackage{indentfirst}
\usepackage[utf8]{inputenc}
\usepackage{amssymb}
\usepackage{enumitem}
\usepackage{color}
\usepackage[fleqn]{amsmath}
\usepackage[a4paper, margin=1.0in]{geometry}
\usepackage{verbatim}

\usepackage{fancyhdr}
\pagestyle{fancy}
\renewcommand{\sectionmark}[1]{\markboth{#1}{}} % set the \leftmark
\fancyhf{}
\fancyhead{}
\fancyhead[RO, LE]{\leftmark} % 1. sectionname
\fancyfoot[C]{\thepage}
\fancypagestyle{plain}{%
	\fancyhead{}%
	\renewcommand{\headrulewidth}{0pt}
}

% INICIO CONFIGURACAO DO HYPER LINK %
\usepackage{hyperref}
\usepackage[dvipsnames]{xcolor}
\newcommand\myshade{85}
\colorlet{mylinkcolor}{violet}

\hypersetup{
	linkcolor  = mylinkcolor!\myshade!black,
	citecolor  = mylinkcolor!\myshade!black,
	urlcolor   = mylinkcolor!\myshade!black,
	colorlinks = true,
}
% FIM CONFIGURACAO DO HYPER LINK %

\usepackage{amsthm, amssymb, amsfonts, amsmath}
\usepackage{graphicx}
\usepackage{tikz}
\usetikzlibrary{calc,shapes}
\usepackage{mathtools}
\usepackage{mathrsfs}
\usepackage{tikz-cd}

% PACOTES PARA DIAGRAMAS %
\usepackage[all,cmtip]{xy}
% PACOTES PARA DIAGRAMAS %

% QUALIFICADORES DOS RESULTADOS: TEOREMAS, LEMAS, COROLARIOS E PROVA %
\newtheorem{teorema}{Teorema}[section]
\newtheorem{corolario}[teorema]{Corolário}
\newtheorem{convensao}[teorema]{Convensão}
\newtheorem{lema}[teorema]{Lema}
\newtheorem{definicao}[teorema]{Definição}
\newtheorem{exemplo}[teorema]{Exemplo}
\newtheorem{suposicao}[teorema]{Suposição}
\newtheorem{observacao}[teorema]{Observação}
\newtheorem{proposicao}[teorema]{Proposição}
\newenvironment{prova}[1]{$\square$ #1}{\hfill$\blacksquare$}
% QUALIFICADORES DOS RESULTADOS: TEOREMAS, LEMAS, COROLARIOS E PROVA %

% AQUI ESTAO OS COMANDOS %
\newcommand{\aplicacaoexponencial}[2]{\exp_{#1}(#2)}
\newcommand{\aplicacaoexponencialgeral}[1]{exp_{#1}}
\newcommand{\aplicaoessuaves}[2]{C^{\infty}(#1, #2)}
\newcommand{\aplicaoessuavesloc}[2]{C^{\infty}_{\text{loc}}(#1, #2)}
\newcommand{\aplicaoessuavesreatacirculo}{C^{\infty}(\retacartesianocirculo; M)}
\newcommand{\autoespaco}[1]{E_{#1}}
\newcommand{\autovalorprimeirotipo}[1]{\sigma^{1}(#1)}
\newcommand{\bigmodulo}[1]{\Bigm\lvert #1 \Bigm\lvert }
\newcommand{\bigparenteses}[1]{\Big( #1 \Big) }
\newcommand{\bordo}[1]{\partial_{#1}}
\newcommand{\bordorel}[1]{\overline{\partial}_{#1}}
\newcommand{\cadeia}[2]{C_{#1}(#2; A)}
\newcommand{\caminhosfechadoscirculo}[2]{L([#1,#2], S^{1})}
\newcommand{\caminhosfechadospontobase}[2]{\mathcal{L}^{o}_{#1}(#2)}
\newcommand{\caminhosfechadosSp}[2]{L([#1,#2], \gruposimpletico{2n})}
\newcommand{\caminhosespeciais}[1]{\mathcal{L}^{*}(#1)}
\newcommand{\caminhosdecaimentoexponencial}[2]{C^{\infty}_{\searrow}(#1, #2)}
\newcommand{\caminhosdecaimentoexponencialpadrao}{\caminhosdecaimentoexponencial{x^{-}}{x^{+}}}
\newcommand{\caminhosexponenciaisconectantesabrev}{\mathcal{P}(x^{-},x^{+})}
\newcommand{\caminhosexponenciaisconectantes}[2]{\mathcal{P}^{1,p}(#1, #2)}
\newcommand{\caminhosexponenciaisSobolev}{\mathcal{L}^{1,p}M}
\newcommand{\caminhosexponenciaisconectantespadrao}{\caminhosexponenciaisconectantes{x^{-}}{x^{+}}}
\newcommand{\caminhos}{\mathcal{L}}
\newcommand{\caminhosfechados}[1]{\caminhos^{o}(#1)}
\newcommand{\caminhoslagrangianos}[3]{\caminhos_{#3}(#1,#2)}
\newcommand{\caminhoslagrangianosV}[2]{\caminhoslagrangianos{#1}{#2}{V}}
\newcommand{\caminhospontobase}[1]{\caminhos_{#1}}
\newcommand{\caminhossempontobase}[1]{\caminhos(#1)}
\newcommand{\caminhosNaoDegeneradosSp}{\caminhos^{*}(\gruposimpletico{2n})}
\newcommand{\caminhospontobasegeral}[2]{\caminhos_{#1}(#2)}
\newcommand{\caminhossuavesconectantes}[2]{\caminhos(#1, #2)}
\newcommand{\caminhossubespacoslagrangianos}[2]{L[#1,#2]}
\newcommand{\campogradiente}{\mathcal{X}}
\newcommand{\campogradientefuncional}{\mathcal{X}_{\mathcal{A}}}
\newcommand{\campohamiltoniano}[1]{X_{H}(#1)}
\newcommand{\campohamiltonianoabrev}{X_{H}}
\newcommand{\campossuaves}[1]{\mathfrak{X}(#1)}
\newcommand{\celula}[2]{D^{#1}_{#2}}
\newcommand{\celulabordo}[2]{\partial D^{#1}_{#2}}
\newcommand{\circulo}{S^{1}}
\newcommand{\circulovariedade}{\circulo\times M}
\newcommand{\cktopologia}[1]{\mathcal{C}^{#1}\text{-topologia}}
\newcommand{\classe}[1]{[#1]}
\newcommand{\cohomologia}[2]{H^{#1}(#2)}
\newcommand{\cohomologiadual}[2]{H^{#1}(#2)^{*}}
\newcommand{\cohomologiacompac}[2]{H^{#1}_{c}(#2)}
\newcommand{\cohomologiacompacdual}[2]{H^{#1}_{c}(#2)^{*}}
\newcommand{\colecao}[1]{\{#1_{k} \}_{k\in \inteiros}}
\newcommand{\colecaoabrev}[1]{\{#1 \}_{k\in \inteiros}}
\newcommand{\colecaofinita}[2]{\{#1_{j} \}_{j=1}^{#2}}
\newcommand{\colecaofinitaabrev}[2]{\{#1 \}_{j=1}^{#2}}
\newcommand{\complementar}[2]{#1 \backslash #2}
\newcommand{\complexificacao}[1]{#1_{\complexo{}}}
\newcommand{\complexificacaotensorial}[1]{\complexo{}\otimes_{\reta} #1}
\newcommand{\complexificado}[1]{\mathcal{#1}}
\newcommand{\complexificacaoelemento}[2]{#1\otimes_{\reta} #2}
\newcommand{\complexo}[1]{\mathbb{C}^{#1}}
\newcommand{\coordenada}[2]{#1_{(#2)}}
\newcommand{\diferencialfloer}{D\operadorFloer}
\newcommand{\diferencialfloeradj}{\diferencialfloer^{*}}
\newcommand{\diferencialfloerponto}[1]{D_{#1}\operadorFloer}
\newcommand{\diferencialfloeradjponto}[1]{D_{#1}\operadorFloer^{*}}
\newcommand{\diferencialfloerabrev}{\mathcal{D}}
\newcommand{\derivada}[2]{\frac{d #1}{d #2}}
\newcommand{\derivadaparcial}[2]{\frac{\partial #1}{\partial #2}}
\newcommand{\derivadaparcialdois}[2]{\frac{\partial^{2} #1}{\partial #2^{2}}}
\newcommand{\derivadaparcialdoisdois}[3]{\frac{\partial^{2} #1}{\partial #2 \partial#3}}
\newcommand{\derivadaparcialabrev}[1]{\partial_{#1}}
\newcommand{\diag}{\text{diag}}
\newcommand{\diferencialhamiltoniano}[1]{(dX_{H})_{#1}}
\newcommand{\distribuicoes}{\distribuicoesgeral{\Omega}}
\newcommand{\distribuicoesgeral}[1]{\mathcal{D'}(#1)}
\newcommand{\dominioMaslov}{\caminhos^{*}(\gruposimpletico{2n})}
\newcommand{\energiafinitaM}{\mathcal{M}}
\newcommand{\energiafinitaMHamiltoniana}[1]{\energiafinitaM(#1, J)}
\newcommand{\energiafinitaMconectante}{\energiafinitaM(x^{-}, x^{+})}
\newcommand{\energiafinitaMconectanteHamiltoniana}{\energiafinitaM(x^{-}, x^{+},H+h,J)}
\newcommand{\espacoLdois}[1]{L^{2}(#1)}
\newcommand{\espacoLdoiscontradom}[2]{L^{2}(#1, #2)}
\newcommand{\espacoLp}[1]{L^{p}(#1)}
\newcommand{\espacoLpcomp}[1]{L^{p}_{loc}(#1)}
\newcommand{\espacoLpcontradominio}[2]{L^{p}(#1;#2)}
\newcommand{\espacoLpGeral}[2]{L^{#1}(#2)}
\newcommand{\espacoLpretacirculo}{\espacoLpcontradominio{\retacartesianocirculo}{\real{2n}}}
\newcommand{\espacoLpadjuntoretacirculo}{L^{q}(\retacartesianocirculo;\real{m})}
\newcommand{\espacoLpdual}{(L^{p})^{*}(\retacartesianocirculo;\real{m})}
\newcommand{\espacoLpdualdois}{(L^{2})^{*}(\retacartesianocirculo;\real{m})}
\newcommand{\espacomoduli}[2]{\mathcal{M}_{#1#2}}
\newcommand{\espacoSimpleticoOrtogonal}[1]{#1^{\omega}}
\newcommand{\espacosobolev}[1]{W^{1,p}(#1)}
\newcommand{\espacosobolevcontradominio}[2]{W^{1,p}(#1;#2)}
\newcommand{\espacosobolevadjuntoretacirculo}{W^{1,q}(\retacartesianocirculo;\real{2n})}
\newcommand{\espacosobolevdual}{(W^{1,p})^{*}(\retacartesianocirculo;\real{2n})}
\newcommand{\espacosobolevgeneralizadocontra}[2]{W^{k,p}(#1; #2)}
\newcommand{\espacosobolevgeneralizado}[1]{W^{k,p}(#1)}
\newcommand{\espacosobolevretacirculo}{\espacosobolevcontradominio{\retacartesianocirculo}{\real{2n}}}
\newcommand{\espacosobolevdois}[2]{\espacosobolevgeral{2}{#1, #2}}
\newcommand{\espacosobolevgeral}[2]{W^{1,#1}(#2)}
\newcommand{\espacotangente}[1]{\espacotangenteponto{p}{#1}}
\newcommand{\espacotangentevariedadeestavel}{T^{s}_{p}M}
\newcommand{\espacotangentevariedadeinstavel}{T^{u}_{p}M}
\newcommand{\espacotangenteponto}[2]{T_{#1}#2}
\newcommand{\espacotangentevariedade}{\espacotangenteponto{p}{M}}
\newcommand{\espectrooperador}[1]{\sigma(#1)}
\newcommand{\estruturacomplexa}{J_{0}}
\newcommand{\estruturascomplexas}[2]{\mathcal{J}(#1, #2)}
\newcommand{\estruturascomplexasM}{\estruturascomplexas{M}{\omega}}
\newcommand{\estruturascomplexaspadrao}{\mathcal{J}(V, \omega)}
\newcommand{\Exp}{\text{Exp}}
\newcommand{\fibradocaminhosexponenciais}{\mathcal{E}(x^{-}, x^{+})}
\newcommand{\fibradocaminhosexponenciaisabrev}{\mathcal{E}}
\newcommand{\formaSimpletica}[2]{\omega(#1, #2)}
\newcommand{\formaSimpleticaabrev}{\omega_{0}}
\newcommand{\formaSimpleticaExtendida}[2]{\Omega(#1, #2)}
\newcommand{\formaSimpleticaPadrao}[2]{\omega_{0}(#1, #2)}
\newcommand{\funcaocond}[5]{
	#1 = 
	\left\{
	\begin{array}{cc}
		#2, & #3\\
		#4, & #5\\
	\end{array}
	\right.
}
\newcommand{\funcionalH}{\mathcal{A}_{H}}
\newcommand{\funcionalHponto}[1]{\mathcal{A}_{H}(#1)}
\newcommand{\funcoesdiferenciaveis}[2]{C^{#1}(#2)}
\newcommand{\funcoesdiferenciaveissupp}[2]{C^{#1}_{c}(#2)}
\newcommand{\funcoesmorse}[1]{\mathcal{M}_{o}(#1)}
\newcommand{\funcoesmorsesmale}[1]{\mathcal{M}^{S}_{o}(#1)}
\newcommand{\funcoessuaves}[1]{C^{\infty}(#1, \real{})}
\newcommand{\funcoessupcompact}[1]{C^{\infty}_{c}(#1)}
\newcommand{\funcoessupcompactcontradom}[2]{C^{\infty}_{c}(#1, #2)}
\newcommand{\funcoesteste}{\funcoestestegeral{\Omega}}
\newcommand{\funcoestestegeral}[1]{\mathcal{D}(#1)}
\newcommand{\generalgroup}[2]{GL(#1, #2)}
\newcommand{\generalgroupreal}[1]{\generalgroup{#1}{\real{}}}
\newcommand{\generalgroupcomplexo}[1]{\generalgroup{#1}{\complexo{}}}
\newcommand{\gerador}[1]{\langle #1\rangle}
\newcommand{\gradiente}{\nabla f}
\newcommand{\gradientefuncional}{\nabla \funcionalH}
\newcommand{\grupofundamental}[1]{\pi_{1}(#1)}
\newcommand{\grupofundamentalpontobase}[2]{\pi_{1}(#1; #2)}
\newcommand{\gruposimpletico}[1]{Sp(#1)}
\newcommand{\gruposimpleticocomplexo}[1]{Sp(#1; \complexo{})}
\newcommand{\gruposimpleticoreal}[1]{Sp(#1;\reta)}
\newcommand{\gruposimpleticoespecial}[1]{Sp^{1}(#1)}
\newcommand{\gruposimpleticonaodegenerado}[1]{Sp^{#1}(2n)}
\newcommand{\gruposimpleticopositivo}[1]{Sp_{+}(#1)}
\newcommand{\hamiltonianasRegulares}{\mathcal{H}_{reg}}
\newcommand{\hessianaponto}[2]{\text{Hess}_{#1}(#2)}
\newcommand{\homologia}[2]{H_{#1}(#2;A)}
\newcommand{\homologiaabrev}[2]{H_{#1}(#2)}
\newcommand{\homologiarel}[3]{H_{#1}(#2,#3)}
\newcommand{\homologiarelcel}[3]{H_{#1}(D^{#2}_{#3}, \partial D^{#2}_{#3})}
\newcommand{\homologiarelskele}[3]{H_{#1}(X^{(#2)}, X^{(#3)})}
\newcommand{\homologiarelskelesimpl}[2]{H_{#1}(X^{(#2)}, X^{(#2-1)})}
\newcommand{\iconley}[1]{\iconleyabrev(#1)}
\newcommand{\iconleyabrev}{\mu_{CZ}}
\newcommand{\imagembordo}[2]{B_{#1}(#2;A)}
\newcommand{\imagembordoabrev}[2]{B_{#1}(#2)}
\newcommand{\ind}{\text{Ind}}
\newcommand{\induzida}[1]{#1_{\#}}
\newcommand{\inteiros}{\mathbb{Z}}
\newcommand{\inteirospos}{\inteiros_{+}}
\newcommand{\iprod}[2]{\langle #1, #2 \rangle}
\newcommand{\intervalo}{[0,1]}
\newcommand{\intervalofechado}[1]{[-#1,#1]}
\newcommand{\kernelbordo}[2]{Z_{#1}(#2;A)}
\newcommand{\kernelbordoabrev}[2]{Z_{#1}(#2)}
\newcommand{\liederivada}[1]{\mathcal{L}_{#1}}
\newcommand{\operadoreslimitadosauto}[1]{\mathcal{B}(#1)}
\newcommand{\operadorFloer}{\mathcal{F}}
\newcommand{\operadorFloerDefinicao}[1]{\derivadaparcial{#1}{s} + J_{#1}\derivadaparcial{#1}{t} - J_{#1}X_{H}(#1)}
\newcommand{\operadorFloerDefParametros}[1]{\derivadaparcial{#1}{s}(s,t) + J_{#1}\derivadaparcial{#1}{t}(s,t) - J_{#1}X_{H}(#1(s,t))}
\newcommand{\operadorFloerParametro}[1]{\mathcal{F}(#1)}
\newcommand{\operadorFloerPadrao}{\operadorFloerParametro{u}}
\newcommand{\matrizantisimetrica}[1]{Asym(#1)}
\newcommand{\matrizortogonal}[1]{O(#1)}
\newcommand{\matrizquadcomplexa}[1]{M_{#1 \times #1}(\complexo{})}
\newcommand{\matrizquadreal}[1]{M_{#1 \times #1}(\real{})}
\newcommand{\matrizsimetrica}[1]{Sym(#1)}
\newcommand{\matrizSimpleticaOrtogonal}{\mathcal{U}}
\newcommand{\matrizsimetricapositiva}[1]{Sym_{+}(#1)}
\newcommand{\matrizsimpleticapositiva}[1]{Sp_{+}(#1)}
\newcommand{\matrizunitaria}[1]{U(#1)}
\newcommand{\norma}[1]{||#1||}
\newcommand{\normagrande}[1]{\Big|\Big|#1\Big|\Big|}
\newcommand{\normaLdois}[1]{||#1||_{L^{2}}}
\newcommand{\normaLp}[1]{||#1||_{L^{p}}}
\newcommand{\normaLpdefinicao}[2]{ \Big(\int_{#2}#1^{p}\Big)^{1/p}}
\newcommand{\normaLpDominio}[2]{||#1||_{L^{p}(#2)}}
\newcommand{\normaLgGeral}[3]{\norma{#1}_{\espacoLpGeral{#2}{#3}}}
\newcommand{\normapequenaLpdefinicao}[2]{ \normaLpdefinicao{\norma{#1}}{#2}}
\newcommand{\normagrandeLpdefinicao}[2]{ \normaLpdefinicao{\normagrande{#1}}{#2}}
\newcommand{\normasubscrito}[2]{\norma{#1}_{#2}}
\newcommand{\normaWdois}[1]{||#1||_{W^{1,2}}}
\newcommand{\normaWp}[1]{||#1||_{W^{1,p}}}
\newcommand{\normaWpgeneralizado}[1]{||#1||_{W^{k,p}}}
\newcommand{\normaWpgeral}[2]{||#1||_{W^{1,#2}}}
\newcommand{\normaWpGeralDominio}[3]{\norma{#1}_{W^{1,#2}(#3)}}
\newcommand{\normaWpDominio}[2]{||#1||_{W^{1,p}(#2)}}
\newcommand{\operadorcauchyabrev}[1]{\overline{\partial}_{#1}}
\newcommand{\operadoresfredholm}[2]{\mathcal{F}r(#1, #2)}
\newcommand{\operadoreslimitados}[2]{\mathcal{B}(#1, #2)}
\newcommand{\orbitaponto}[1]{\mathcal{O}(#1)}
\newcommand{\orbitasconectantes}[2]{\mathcal{M}(#1, #2)}
\newcommand{\orbitasConectantesZ}{\mathcal{Z}(x^{-}, x^{+},H, J)}
\newcommand{\orbitasconectantespadrao}{\mathcal{M}(x^{-}, x^{+})}
\newcommand{\paresregulares}{\mathcal{H}_{reg}}
\newcommand{\parteImaginaria}[1]{\Im(#1)}
\newcommand{\parteReal}[1]{\Re(#1)}
\newcommand{\perturbacaoHamiltoniana}[1]{C^{\infty}_{\epsilon}(#1)}
\newcommand{\pontoscriticos}[1]{\textit{Cr}(#1)}
\newcommand{\pontoscriticosordem}[2]{\textit{Cr}^{(#1)}(#2)}
\newcommand{\produtointerno}[2]{\langle #1, #2 \rangle}
\newcommand{\produtointernoLdois}[2]{\langle #1, #2 \rangle_{L^{2}}}
\newcommand{\produtointernoWdois}[2]{\langle #1, #2 \rangle_{W^{1,2}}}
\newcommand{\produtointernoabrev}{\langle ., .\rangle}
\newcommand{\produtosinternos}[1]{Riem(#1)}
\newcommand{\produtotensorial}[2]{ #1_{1} \otimes_{\mathbb{K}} \dots \otimes_{\mathbb{K}} #1_{#2}}
\newcommand{\produtotensorialabrev}[2]{#1\otimes #2}
\newcommand{\produtotensorialdual}{\produtotensorialabrev{\complexificado{V}^{*}}{\complexificado{V}^{*}}}
\newcommand{\produtotensorialreal}[2]{\bigotimes_{j=1}^{#1} #2_{j}}
\newcommand{\pullbackfibradotangente}[2]{#1^{*}T#2}
\newcommand{\pullbackfibradotangenteM}[1]{\pullbackfibradotangente{#1}{M}}
\newcommand{\pullbackfibradotangenteMpadrao}{\pullbackfibradotangente{u}{M}}
\newcommand{\quocientetrajetorias}{\widehat{\energiafinitaM}(x^{-}, x^{+}, H, J)}
\newcommand{\retacartesianocirculo}{\real{} \times \circulo}
\newcommand{\retacartesianovariedade}{\real{} \times M}
\newcommand{\real}[1]{\mathbb{R}^{#1}}
\newcommand{\realprojetivo}[1]{\mathbb{R}P^{#1}}
\newcommand{\reta}{\real{}}
\newcommand{\subespacoslagrangianos}{L(V)}
\newcommand{\lacocontrateis}{\mathcal{L}^{o}M}
\newcommand{\somadir}[1]{\bigoplus \limits_{#1}}
\newcommand{\cilindrosLM}{\mathcal{C}M}
\newcommand{\morsefunc}[1]{\mathcal{M}o(#1)}
\newcommand{\skeleton}[1]{X^{(#1)}}
\newcommand{\variedadeconectante}{\variedadeconectantepontos{p}{q}}
\newcommand{\variedadeconectantepontos}[2]{W_{#1#2}}
\newcommand{\variedadeestavel}[1]{W^{s}(#1)}
\newcommand{\variedadeinstavel}[1]{W^{u}(#1)}
\newcommand{\aviso}[1]{{\color{violet}(#1)}}
\newcommand{\alerta}[1]{{\color{red}#1}}
% AQUI ESTAO OS COMANDOS %

\begin{document}
	
	\title{O Índice de Maslov e suas Aplicações em Topologia Simplética: a homologia de Floer e a Conjectura de Arnold.}
	
	\author{Vinicius Fernandes}
	
	\maketitle
	
	\tableofcontents
	\chapter*{Agradecimentos}
	Seja $\mathcal{A}$ o conjunto dos objetos passíveis de meus agradecimentos. Um \textit{agradecimento direto por esse trabalho} é uma sequência $(a_{n})$ em $\mathcal{A}$ tal que
	$$
	\lim\limits_{n\to \infty} a_{n} \in \{\text{Mariana R. da Silveira}\} \cup \{http://gen.lib.rus.ec/ \} \subset \mathcal{A}.
	$$
	
	
	\chapter*{Resumo}
	Nesse trabalho são apresentadas uma construção do Índice de Maslov e uma aplicação desse índice na demonstração da conjectura de Arnold, que dá um limitante inferior para o número de soluções 1-períodicas de um sistema  Hamiltoniano dependente do tempo em uma variedade simplética compacta M. A estratégia para a demonstração, semelhante a demonstração das desigualdades de Morse, é construir um complexo de cadeias, chamado complexo de Floer, cuja homologia recupera a homologia singular da variedade. O complexo de Floer é gerado pelos pontos críticos de um funcional de ação definido no espaço dos laços em M e graduado pelo índice de Maslov. O operador bordo conta as soluções da Equação de Floer que conectam pontos críticos de índice de Maslov consecutivos.
	\\
	\\	
	\textbf{Palavras-chave:} Conjectura de Arnold, Equação de Floer, Espaços Vetoriais Simpléticos, Grupo Simplético, Homologia de Floer, Índice de Maslov, Índice de Conley-Zehnder, Sistemas Hamiltonianos, Teoria de Morse-Witten, Topologia Simplética, Variedade Simplética.
	
	\chapter*{Abstract}
	In this work, a construction of the Maslov Index, as well as its application on the proof of Arnold's conjecture, which gives a lower bound to the number of 1-periodic solutions of a time-dependent Hamiltonian  system in a compact symplectic manifold M, are presented. The idea of the proof, analogously to the proof of the Morse Inequalities, is to build a chain complex, namely the Floer complex, whose homology recovers the singular homology of M. The Floer complex is generated by the critical points of  an action function defined on the loop space of M and graded by the Maslov Index. The boundary operator counts the solutions of the Floer equation which connect critical points with consecutive Maslov indices.
	\\
	\\
	\textbf{Keywords:}  Arnold Conjecture, Floer Equation, Symplectic Vector Spaces, Symplectic Group, Floer Homology, Maslov Index, Conley-Zehnder Index, Hamiltonian Systems, Morse-Witten Theory, Symplectic Topology, Symplectic  Manifold.
	
	\chapter*{Introdução}\label{capitulo_introducao}
	Como muitas das construções desenvolvidas na matemática a Homologia de Floer teve como motivação um problema de origem na Física, mais especificamente, no estudo de sistemas Hamiltonianos. A descrição Newtoniana da mecânica clássica tem como um de seus equivalentes a descrição Hamiltoniana. Nesse contexto, grande parte dos sistemas dinâmicos Hamiltonianos podem ser desenvolvidos em uma 2n-variedade diferenciável $M$ munida de uma 2-forma fechada e não degenerada $\omega$. O sistema dinâmico é modelado por uma função Hamiltoniana dependente do tempo $H: M\times \reta\to \reta$ satisfazendo a equação geométrica $\omega_{x}(\campohamiltonianoabrev(x,t), Y) = d_{x}H(Y)$ para todo $Y \in \espacotangenteponto{x}{M}$ e todo $x \in M$, onde o campo vetorial $\campohamiltonianoabrev \in \campossuaves{M}$ é chamado de campo Hamiltoniano associado a $H$. O sistema de equações diferenciais $\dot{x} = \campohamiltoniano{(x,t)}$ é chamado sistema Hamiltoniano e suas soluções dão origem a uma família de simplectomorfismos $\psi_{t}:M\to M$, onde $t\in \reta$, que satisfazem a equação diferencial
	$$
	\dot{\psi}_{t}(x) = \campohamiltoniano{\psi_{t}(x),t}.
	$$
	
	Foi estudando sistemas dinâmicos Hamiltonianos que V. I. Arnold formulou a primeira versão da conjectura que atualmente leva seu nome. De acordo com a Teoria de Lefschetz toda aplicação Hamiltoniana possui ao menos um ponto fixo, quando a característica de Euler de $M$ é não-nula. A ideia de Arnold era generalizar esse resultado, o que levou a seguinte versão da conjectura:
	
	\textbf{Conjectura de Arnold (V1):} \textit{
		Sejam $(M, \omega)$ uma 2n-variedade simplética compacta, $H:M\times \reta\to \reta$ uma função Hamiltoniana e $\psi:M\times \reta \to M$ o fluxo Hamiltoniano associado a $H$. Então, o número de pontos fixos do simplectomorfismo $\psi_{1}:M\to M$ é, no mínimo, o número de pontos críticos de $H$.}
	
	Posteriormente, trabalhando em particularizações do problema, tais como $T^{2}$ e $S^{2}$, Arnold observou que, se a função Hamiltoniana $H$ depende periodicamente no tempo, isto é, $H:\circulovariedade\to \reta$ tem período 1, então as soluções 1-periódicas do campo Hamiltoniano estão em bijeção com os pontos fixos da função Hamiltoniana. Com isso, ao invés de determinar diretamente os pontos fixos da aplicação Hamiltoniana $\psi$, basta determinar as soluções 1-periódicas do sistema Hamiltoniano, o que resultou na seguinte reformulação da conjectura.
	
	\textbf{Conjectura de Arnold (V2):} \textit	{Seja $(M,\omega)$ uma 2n-variedade compacta e simplética. Defina $H:M\times \real{} \to \reta$ uma função Hamiltoniana 1-periódica e suponha que as soluções 1-periódicas do sistema Hamiltoniano sejam não-degeneradas. Então o número de soluções $\mathcal{N}$ desse sistema será limitado inferiormente pela soma dos números de Betti de M, isto é:
		$$
		\mathcal{N}\geq \sum_{i=0}^{2n}\beta_{i}(M),
		$$
		onde $\beta_{i}(M)$ é a dimensão do i-ésimo grupo de homologia singular de $M$.}
	
	Motivado pela demonstração dessa conjectura, nasce um dos principais formalismos no estudo da topologia simplética: a Homologia de Floer.
	
	A semelhança dessa conjectura com as desigualdades de Morse motivou a ideia de se construir um complexo análogo ao complexo de Morse cujo diferencial conta as soluções das equações de Floer.
	
	A construção de Witten, que teve como origem em estudos dos aspéctos topológicos de sistema quânticos super simétricos (veja em \cite{witten_supersymmetry_morse}), permitiu um entendimento geométrico do operador bordo do complexo de cadeias de Morse. Munindo a variedade $M$ com uma métrica Riemanniana, determina-se os pontos críticos de função de Morse $f: M \to \reta$ e admitindo a transversalidade das variedades estáveis e instáveis mostra-se que, para cada par de pontos críticos com índices consecutivos, existe um número finito de linhas do fluxo gradiente conectando-os. O conjunto de tais linhas são subvariedades de $M$ chamadas variedades de conexão. Escolhe-se uma orientação para a variedade de conexão. Witten definiu o grupo das $k$-cadeias como sendo o grupo abeliano gerado pelos pontos críticos com índice de Morse igual a $k$. Já o operador bordo, quando avaliado em uma k-cadeia, é a soma sob todas as $(k-1)$-cadeias, do número de órbitas conectantes contados com orientação. Essa é uma metodologia simples e elegante de cálculo da homologia da variedade que permite o estudo dos aspectos topológicos das variedades via sistemas dinâmicos, sendo que a dinâmica é materializada no gradiente da função de Morse $f$.
	
	É tentador enveredar-se na demonstração da conjectura de Arnold via construção de Witten. Contudo, dificuldades técnicas fizeram com que uma abordagem diferente fosse realizada por Floer. Em vez de estudar o gradiente de uma função Hamiltoniana definida na variedade $M$ a estratégia foi analisar o comportamento do funcional $\funcionalH$, chamado funcional de ação, definido no espaço de curvas 1-periódicas e contráteis em $M$. Aplicando o princípio variacional clássico, mostra-se que as soluções 1-periódicas e contráteis do sistema Hamiltoniano são os pontos críticos do funcional de ação $\pontoscriticos{\funcionalH}$. Essa técnica é conhecido como abdordagem variacional. Uma das grandes dificuldades nesse cenário é que o conjunto de curvas 1-periódicas e contráteis forma uma variedade de dimensão infinita, com isso, encontrar um invariante análogo ao índice de Morse para os pontos críticos do funcional é uma tarefa árdua.
	
	O princípio variacional aplicado no funcional de ação resulta em um sistema de equações diferenciais parciais de primeira ordem, cujas soluções são as curvas integrais que conectam os pontos críticos do funcional de ação, situação análoga a determinação das linhas de fluxo do gradiente negativo de uma função de Morse. A esse sistema de equações diferenciais é associado o operador $\operadorFloer$, chamado operador de Floer. O operador $\operadorFloer$ é um operador linear e limitado 
	definido no espaço de todas as curvas que conectam um par de pontos críticos do funcional de ação. Mostra-se que esse espaço é uma variedade de dimensão infinita (uma variedade de Banach) e que o diferencial de tal operador é um operador de Fredholm. A todo operador de Fredholm associa-se um número inteiro, chamado índice de Fredholm, que no caso de interesse, coincide com a diferença dos índices de Maslov de cada um dos pontos críticos. Feito isso, pode-se mostrar que, no caso de uma função Hamiltoniana independente do tempo e cuja Hessiana é limitada, tal índice coincide com o índice de Morse, a menos de uma constante. Tem-se aqui uma generalização da Homologia de Morse-Witten.
	
	Essa abordagem é uma consequência da definição do complexo de Floer e sua homologia. Analogamente ao complexo de Morse-Witten, as k-cadeias do complexo de Floer são geradas pelos pontos críticos do funcional de ação $\funcionalH$ de índice $k \in \inteiros_{+}$. A graduação do grupo gerado pelos pontos críticos do funcional de ação é dada pelo índice de Maslov $\mu$.
	
	\chapter{Preliminares de Topologia Algébrica}
	A topologia algébrica é o ramo da matemática que utiliza ferramentas algébricas para analisar a topologia de um conjunto (um espaço topológico) associando a ele invariantes topológicos que são estruturas algébricas.
	
	Neste capítulo, são apresentados dois destes invariantes: o grupo fundamental e os grupos de homologia.
		
	\section{Grupo Fundamental e a Aplicação Grau}\label{capitulo_grupo_fundamental}
	Nessa seção será abordada um caso específico de estrutura algébrica associada a um espaço topológico $X$, chamado grupo fundamental de $X$. Mais detalhes dos resultados aqui enunciados podem ser encontrados em \cite{elon_grupo_fundamental} e \cite{massey}.
	
	\subsection{Grupo Fundamental}
	
	Sejam $X$ um espaço topológico e $\caminhossempontobase{X}$ o conjunto dos caminhos contínuos $\gamma:[0,1]\to X$. Tome $\gamma, \beta \in \caminhossempontobase{X}$ tais que $\gamma(1) = \beta(0)$. A justaposição $*:\caminhossempontobase{X}\times \caminhossempontobase{X} \to \caminhossempontobase{X}$ entre $\gamma$ e $\beta$ é definida por
	$$
	\funcaocond{(\gamma*\beta)(t)}{\gamma(2t)}{0\leq t \leq 1/2}{\beta(2t-1)}{1/2 \leq t \leq 1}.
	$$
	
	Dado um caminho $\gamma\in \caminhossempontobase{X}$, seu caminho inverso $\hat{\gamma} \in \caminhossempontobase{X}$ é definido por $\hat{\gamma} (t) = \gamma(1-t)$.
	
	
	\begin{definicao}\label{definicao_caminhos_homotopicos}
		(Homotopia entre caminhos) Sejam $\gamma, \gamma' \in \caminhossempontobase{X}$ tais que $\gamma(0)=\gamma'(0)$ e $\gamma(1)=\gamma'(1)$. Uma aplicação contínua $F:\intervalo \times \intervalo \to X$ é chamada homotopia entre $\gamma$ e $\gamma'$ se $F(t, 0) = \gamma(t)$ e $F(t, 1) = \gamma'(t)$ para qualquer $t\in \intervalo$. Diz-se que  $\gamma$ e $\gamma'$ são homotópicos se existe uma homotopia entre eles.
	\end{definicao}
	
	\begin{definicao}\label{definicao_homotopia_extremos_fixos}
		(Homotopia de extremos fixos entre caminhos) Suponha as condições da definição anterior. Diz-se que a homotopia entre os caminhos $\gamma$ e $\gamma'$ é uma homotopia de extremos fixos se $F(0,s) = \gamma(0) = \gamma'(0)$ e $F(1,s) = \gamma(1) = \gamma'(1)$ para todo $s\in \intervalo$.
	\end{definicao}
	
	Diz-se que dois caminhos são homotópicos com extremos fixos (ou apenas homotópicos quando não houver perigo de confusão) quando existir uma homotopia de extremos fixos entre ambos. Nesse utilizaremos a notação $\gamma \sim \gamma'$.
	
	\begin{observacao}
		A definição de caminhos homotópicos não depende da homotopia escolhida, pois, dadas $F,G$ homotopias entre $\gamma$ e $\gamma'$, pode-se definir uma aplicação contínua $H: [0,1] \times [0,1] \times [0,1] \to X$ tal que $H(t,s ,0) = F(t,s)$ e $H(t,s, 1) = G(t,s)$ que é uma homotopia entre as homotopias, logo será uma homotopia entre ambas as famílias de caminhos.
	\end{observacao}
	
	O próximo lema demonstra a compatibilidade entre a justaposição de caminhos e a relação de homotopia.
	
	\begin{lema}\label{lema_compatibilidade_produto_caminhos}
		Sejam $\gamma, \gamma', \alpha, \alpha' \in \caminhossempontobase{X}$, onde $\gamma(1) = \alpha(0), \gamma'(1) = \alpha'(0) $, tais que $\gamma \sim \gamma'$ e $\alpha \sim \alpha'$, então $\gamma * \alpha \sim \gamma' * \alpha'$.
	\end{lema}
	\begin{prova}
		Sejam  $F, G:[0,1] \times [0,1] \to X$ homotopias com extremos fixos entre caminhos tais que $F(t,0)=\gamma(t)$, $F(t,1)=\gamma'(t)$, $G(t,0)=\alpha(t)$, $G(t,1)=\alpha'(t)$. Então, $F_{s}, G_{s}:\intervalo \to X$, definidas por $F_{s}(t) = F(s,t)$ e $G_{s}(t) = G(s,t)$, são curvas contínuas para todo $s \in \intervalo$ tais que $F_{s}(1) = G_{s}(0)$. Com isso, $H :\intervalo\times \intervalo\to X$ tal que $H(t, s)=(F_{s}*G_{s})(t)$ é uma homotopia de extremos fixos entre $\gamma*\alpha$ e $\gamma'*\alpha'$. De fato, a continuidade de $H$ é imediata. Além disso, $H(t, 0) = (\gamma*\alpha)(t)$ e $H(t, 1) = (\gamma'*\alpha')(t)$. Como F e G mantém os extremos fixos, então H também é homotopia de extremos fixos. Logo, $\gamma*\alpha \sim \gamma'*\alpha'$.
	\end{prova}
	
	A justaposição de caminhos em $\caminhossempontobase{X}$ não é associativo. Contudo, o seguinte lema afirma que tal associatividade vale a menos de uma homotopia. Sua demonstração pode ser encontrada em $\cite{massey}$.
	
	\begin{lema}\label{lema_associatividade_produto_caminhos}
		Sejam $\alpha, \beta, \gamma\in \caminhossempontobase{X}$ tais que $\alpha(1)=\beta(0)$ e $\beta(1)=\gamma(0)$. Então $(\alpha*\beta)*\gamma \sim \alpha*(\beta*\gamma)$.
	\end{lema}
	
	Note que em geral $\gamma*\hat{\gamma}\neq \hat{\gamma}*\gamma$ e $\gamma*c\neq \gamma$, mas o lema seguinte mostra tais caminhos são homotópicos.
	
	\begin{lema}\label{lema_caminho_inverso}
		Sejam $\gamma, \hat{\gamma} , c,c' \in \caminhossempontobase{X}$ tais que $c(t) = \gamma(0)$, $c'(t) = \gamma(1)$ para todo $t\in \intervalo$ e $\hat{\gamma} (t) = \gamma(1-t)$. Então $\gamma*\hat{\gamma}  \sim c$ e $\hat{\gamma}  *\gamma\sim c'$. Além disso, $\gamma * c' \sim\gamma$ e $c * \gamma \sim \gamma$.
	\end{lema} 	
	
	Seja $\caminhospontobasegeral{p}{X} = \{\gamma\in \caminhossempontobase{X}: \gamma(0)=\gamma(1)=p \}$ o conjunto dos caminhos contínuos e fechados com o ponto base $p$. Denote o quociente $\{ \classe{\gamma} : \gamma' \in \caminhospontobasegeral{p}{X},\;\;\gamma \sim \gamma'\}$ por $\caminhospontobasegeral{p}{X}/\sim $. Então, o produto definido por $\classe{\gamma}.\classe{\alpha} = \classe{\gamma*\alpha}$ esta bem-definido. De fato, tomando $\gamma, \gamma',\alpha, \alpha' \in \caminhospontobasegeral{p}{X}$ tal que $\gamma \sim \gamma'$ e $\alpha \sim \alpha'$, então $\classe{\gamma}.\classe{\alpha} = \classe{\gamma*\alpha} = \classe{\gamma'*\alpha'} = \classe{\gamma'}.\classe{\alpha'}$.
	
	
	\begin{definicao}
		(Grupo Fundamental) Sejam $X$ um espaço topológico $X$ e $p\in X$. O conjunto
		$$
		\grupofundamentalpontobase{X}{p} = (\caminhospontobase{p}(X)/\sim, .)
		$$
		é chamado grupo fundamental de $X$.
	\end{definicao}
	
	\begin{teorema}
		O grupo fundamental $\grupofundamentalpontobase{X}{p}$ é de fato um grupo.
	\end{teorema}
	\begin{prova}
	A associatividade segue do Lema \ref{lema_associatividade_produto_caminhos} e a existência do elemento neutro e do elemento inverso  seguem do Lema \ref{lema_caminho_inverso}. Portanto, $\grupofundamentalpontobase{X}{p}$ é um grupo. 
	\end{prova}
	
	Na definição de grupo fundamental se mantém a escolha do ponto base, contudo, pode-se mostrar que, para espaços topológicos conexos por caminhos, a definição de grupo fundamental independe da escolha do ponto base. A demonstração do teorema a seguir pode ser encontrado em $\cite{massey}$.
	
	\begin{teorema}
		Se $X$ é um espaço topológico conexo por caminhos, então $\grupofundamentalpontobase{X}{p} \cong \grupofundamentalpontobase{X}{q}$ para quaisquer $p,q \in X$.
	\end{teorema}
	
	No caso em que $X$ é um espaço espaços topológico conexo por caminhos $\grupofundamentalpontobase{X}{p}$ é denotado por $\grupofundamental{X}$ e $\caminhospontobasegeral{p}{X}$ é denotado por $\caminhossempontobase{X}$.
	
	
	Seja $f:X\to Y$ uma aplicação contínua entre espaços topológicos e $\gamma \in \caminhospontobasegeral{p}{X}$, então tem-se a composição $f\circ \gamma \in \caminhospontobasegeral{f(p)}{Y}$. Suponha que $\gamma' \in \caminhospontobasegeral{p}{X}$ seja tal que $\gamma \sim \gamma'$. Se $F$ é a homotopia entre ambas, então $(f\circ F)(t,0) =  f(\gamma(t))$ e $(f\circ F)(t,1) =  f(\gamma'(t)) $, logo $f\circ F$ é uma homotopia entre $f\circ \gamma$ e $f\circ \gamma'$ em $\caminhospontobasegeral{f(p)}{Y}$.
	
	Dada $f:X\to Y$ uma aplicação contínua entre espaços topológicos, é imediato que a aplicação $f_{*}:\grupofundamentalpontobase{X}{p} \to \grupofundamentalpontobase{Y}{f(p)}$ dada por $f_{*}\classe{\gamma} = \classe{f\circ\gamma}$ é um homomorfismo. Como consequência, o seguinte resultado mostra que o grupo fundamental é um invariante topológico.
	
	\begin{teorema}
		Seja $f:X\to Y$ um homeomorfismo entre espaços topológicos. Então $\grupofundamentalpontobase{X}{p}$ e $\grupofundamentalpontobase{Y}{f(p)}$ são isomorfos.
	\end{teorema}
	
	Dados $X, Y$ espaços topológicos, $p\in X$ e $q\in Y$ pontos base, então pode-se construir o espaço topológico $X\times Y$ com o ponto base $(p,q)$.
	
	\begin{proposicao}\label{proposicao_produto_grupo_fundamental}
		Se $X, Y$ são espaços topológicos, $p\in X$ e $q\in Y$ são os respectivos pontos base, então $\grupofundamentalpontobase{X\times Y}{(p,q)}$ é isomorfo ao produto direto $\grupofundamentalpontobase{X}{p}\otimes \grupofundamentalpontobase{Y}{q}$.
	\end{proposicao}
	
	\begin{definicao}
		Seja $X$ um espaço topológico conxo por caminhos e
		$c \in \caminhospontobasegeral{p}{X}$ o caminho constante tal que $c(t) = p$ para todo $t \in \intervalo$. Se para todo $\gamma \in \caminhospontobasegeral{p}{X}$ tem-se que $\gamma \sim c$, então é dito que $X$ é simplesmente conexo.
	\end{definicao}
	
	\begin{proposicao}\label{proposicao_grupo_fundamental_simplesmente_conexo}
		Se $X$ é um espaço topológico simplesmente conexo, então $\grupofundamental{X}$ é trivial.
	\end{proposicao}
	
	\subsection{Aplicação Grau e o Grupo Fundamental de $\circulo$}
	Nesta seção será mostrado que o grupo fundamental de $S^{1}$ é um grupo cíclico infinito pois é isomorfo a $\inteiros$. Para demonstrar este fato, a cada caminho fechado $\gamma:[0,1] \to S^{1}$, associa-se um número $\deg(\gamma) \in \inteiros$, chamado de \textit{grau de $\gamma$}, de modo que dois caminhos $\gamma, \beta$ em $S^{1}$ são homotópicos se, e somente se, $\deg(\gamma) = \deg(\beta)$ (possuem o mesmo grau). Finalmente, será demonstrado que a aplicação grau induz um isomorfismo entre $\pi_{1}(S^{1})$ e $\inteiros$.
	
	A proposição a seguir é necessária para a definição e também para o estudo das propriedades da aplicação grau e sua demonstração pode ser encontrada em $\cite{elon_grupo_fundamental}$.
	
	Defina a aplicação exponencial $\Exp: \reta\to \circulo$ por $\Exp(t) = e^{it}$.
	
	\begin{proposicao}\label{proposicao_levantamento_curvas}
		(Levantamento de caminhos) Seja $\gamma:\intervalo \to S^{1}$ uma aplicação contínua e $t_{0}\in \real{}$ tal que $\gamma(0) = e^{it_{0}}$. Então existe uma única aplicação contínua $\alpha:\intervalo\to \real{}$ tal que $\gamma(t) = e^{i\alpha(t)}$ para todo $t\in \intervalo$ e $\alpha(0) = t_{0}$. A aplicação $\alpha$ é chamada de levantamento do caminho $\gamma$ e faz com que o diagrama abaixo comute:
		$$
		\xymatrix{
			& \real{}\ar[d]\ar[d]^{\Exp}
			\\
			\intervalo\ar[ur]^{\alpha} \ar[r]_{\gamma} & S^{1}
		}
		$$
	\end{proposicao}
	
	Seja $\gamma \in \caminhossempontobase{\circulo}$. Então pode-se escrever $\gamma(t) = e^{i\alpha(t)}$, onde $\alpha:\intervalo \to \reta$ é uma aplicação contínua. Como $\hat{\gamma} (\{1\})$ é um subconjunto fechado do compacto $\intervalo$, então é um compacto, e com isso, finito. Denote $t_{\gamma} = \max\{t \in \hat{\gamma} (\{0\})\}$ e defina os caminhos $\gamma_{0}(t) = \gamma(t_{\gamma}t)$ e $\gamma_{1}(t) = \gamma((1-t_{\gamma})t+t_{\gamma})$. Com isso, pode-se escrever $\gamma = \gamma_{0}*\gamma_{1}$. Tem-se que $\gamma_{0}$ é um caminho fechado pois $\gamma_{0}(0) = \gamma_{0}(1)$ e $\gamma_{1}$ um caminho aberto. Além disso, $\gamma_{1}$ é contrátil e $\gamma = \gamma_{0}*\gamma_{1}$. Logo gamma é homotópico a $\gamma_{0}$ (homotopia sem extremo fixos). Portanto, qualquer $\gamma \in \caminhos(\circulo)$ é homotópico a um caminho fechado $\gamma_{0} \in \caminhos(\circulo)$.
	
	O grau de um caminho será definido como uma aplicação que associa um dado caminho em $\circulo$ a um número inteiro. Na construção anterior foi visto que todo caminho em $\circulo$ é homotópico a um caminho fechado. Assim, as hipóteses serão restritas aos caminhos fechados.

	\begin{definicao}\label{definicao_aplicacao_grau}
		(Aplicação grau) Sejam $p \in \circulo$ um ponto base e $deg: \caminhospontobasegeral{p}{\circulo} \to \inteiros$ a aplicação dada por $\deg(\gamma) = (\alpha(1)-\alpha(0))/2\pi$, onde $\alpha:[0,1] \to \real{}$ é o levantamento de $\gamma$ dado pela Proposição $\ref{proposicao_levantamento_curvas}$. A aplicação $deg$ é chamada de aplicação grau e o valor $\deg(\gamma)$ é chamado grau do caminho $\gamma$.
	\end{definicao}
	
	Tem-se que $\gamma(0) = \gamma(1)$, o que é equivalente a $e^{i\alpha(0)} = e^{i\alpha(1)}$ pela Proposição $\ref{proposicao_levantamento_curvas}$. Com isso, $e^{i(\alpha(1)-\alpha(0))} = 1$ e se tem  $\alpha(1)-\alpha(0) = 2\pi k$ para algum $k \in \inteiros$. Portanto, $\deg(\gamma) = (\alpha(1)-\alpha(0))/2\pi \in \inteiros$ e $deg$ está bem-definida. Além disso, a aplicação grau possui as seguintes propriedades.
	
	\begin{proposicao}\label{proposicao_grau_aplicacao}
		(Propriedades da aplicação grau) Sejam $\gamma, \beta \in \caminhospontobasegeral{p}{\circulo}$, então
		\begin{enumerate}
			\item $\deg(\caminhospontobasegeral{p}{\circulo}) = \inteiros$ e, se $\gamma(t) = e^{i2\pi t}$, então $\deg(\gamma) = 1$.
			
			\item $\deg(\gamma*\beta)=\deg(\gamma)+\deg(\beta)$.
			
			\item $\gamma\sim \beta$ se, e somente se, $\deg(\gamma)=\deg(\beta)$
			
			
			\item $\deg(\hat{\gamma} ) = -\deg(\gamma)$ e $\deg(\gamma*\hat{\gamma} ) = 0$.
			
		\end{enumerate}
	\end{proposicao}
	\begin{prova}
		\begin{enumerate}
			\item Supondo que $\gamma(t) = e^{i2\pi t}$, então $\deg(\gamma) = (2\pi -0)/2\pi =1$. Dado $k \in \inteiros$ e supondo que $\beta(t) = e^{i2\pi kt}$ se tem $\deg(\beta) = k$, logo $deg$ é sobrejetora e $\deg(\caminhospontobasegeral{p}{\circulo}) = \inteiros$.
			
			\item Suponha que $\alpha$ e $\lambda$ sejam os respectivos levatamentos de $\gamma$ e $\beta$. Pela justaposição de caminhos tem-se $(\gamma*\beta)(t) = e^{i\alpha(2t)}$ para $0\leq t\leq 1/2$ e  $(\gamma*\beta)(t) = e^{i\lambda(2t - 1)}$ para $1/2\leq t\leq 1$. Logo 
			$$
			\begin{aligned}
			\deg(\gamma*\beta) &= (\lambda(1)- \alpha(0))/2\pi 
			\\
			&= (\lambda(1) -\lambda(0)+ \alpha(1)- \alpha(0))/2\pi
			\\
			&= \deg(\gamma)+\deg(\beta).
			\end{aligned}
			$$  
						
			\item Suponha que $h:[0,1]\times [0,1]\to S^{1}$ seja uma homotopia de extremos fixos entre $\gamma$ e $\beta$ tal que $h(t,0) = \gamma(t)$ e $h(t,1) = \beta(t)$. Então, para cada $s \in [0,1]$ fixo tem-se $h_{s}(0) = h_{s}(1)$, portanto $h_{s} \in \caminhospontobasegeral{p}{\circulo}$. Pela Proposição $\ref{proposicao_levantamento_curvas}$ é possível escrever $h_{s}(t) = e^{i\alpha_{s}t}$. Com isso, pode-se escolher $s, s_{0 }\in [0,1]$ tais que $\norma{h_{s}(t)-h_{s_{0}}(t)} =\norma{e^{i\alpha_{s}(t)} - e^{i\alpha_{s_{0}}(t)}} <2$, ou seja, $h_{s}(t), h_{s_{0}}(t) \in \circulo$ não são anti-podais, logo $|\alpha_{s}(t)-\alpha_{s_{0}}(t)| <\pi$ para todo $t\in [0,1]$. Seja $0=s_{0}<s_{1}<s_{2}\dots s_{m-1}<s_{m} = 1$ uma partição do intervalo $[0,1]$, tal que $\norma{h_{s_{j+1}}(t)-h_{s_{j}}(t)}<2$ para $0\leq j \leq m-1$. A partição escolhida implica em $|\alpha_{s_{j+1}}(t)-\alpha_{s_{j}}(t)| <\pi$. Então
			$$
			\begin{aligned}
			2\pi|\deg(h_{s_{j+1}})-\deg(h_{s_{j}})| 
			&= |\alpha_{s_{j+1}}(1)-\alpha_{s_{j+1}}(0) - \alpha_{s_{j}}(1)+\alpha_{s_{j}}(0)|
			\\
			&\leq |\alpha_{s_{j+1}}(1)-\alpha_{s_{j}}(1)| + |\alpha_{s_{j}}(0)+\alpha_{s_{j}}(0)|
			\\
			&<2\pi,
			\end{aligned} 
			$$
			logo $|\deg(h_{s_{j+1}})-\deg(h_{s_{j}})| <1$. Portanto $\deg(h_{s_{j+1}})=\deg(h_{s_{j}})$ para todo $0\leq j \leq m-1$. Logo, 
			$$
			\deg(\beta) = \deg(h_{s_{m}})=\deg(h_{s_{m-1}})=\dots=\deg(h_{s_{0}}) = \deg(\gamma).
			$$
			Por outro lado, suponha que $n = \deg(\gamma)=\deg(\beta)$. Considere os levantamentos $\alpha$ e $\lambda$ dos caminhos $\gamma$ e $\beta$, respectivamente. Seja $H:[0,1]\times [0,1] \to \real{}$ definida por $H(t,s) = (1-s)\alpha(t) + s\lambda(t)$. Então $H(t,0)=\alpha(t)$ e $H(t,1)=\lambda(t)$, logo $H$ é uma homotopia entre $\alpha$ e $\lambda$. Além disso, 
			$$
			\begin{aligned}
			H(1,s) - H(0,s) 
			&= (1-s)\alpha(1) + s\lambda(1) - (1-s)\alpha(0) + s\lambda(0) 
			\\
			&= (1-s)(\alpha(1)-\alpha(0)) + s(\lambda(1)-\lambda(0))
			\\
			&=\big( (1-s)\deg(\gamma) +s\deg(\beta)\big)2\pi
			\\
			&= 2\pi n.
			\end{aligned}
			$$ 
			
			Seja a aplicação contínua $G(t, s) = e^{iH(t,s)}$. Então, para $s \in [0,1]$ fixo tem-se que $G_{s} \in \caminhospontobasegeral{p}{\circulo}$ com $\deg(G_{s}) = n$. Além disso, $G(t,0) = e^{i\alpha(t)} = \gamma(t)$ e $G(t,1) = e^{i\lambda(t)} = \beta(t)$, logo $G$ é uma homotopia entre $\gamma$ e $\beta$, portanto $\gamma \sim \beta$.
			
			\item Sejam $p=\gamma(0)$ e $c\in \caminhossempontobase{X}$ o caminho constante tal que $c(t)=p$ para todo $t\in \intervalo$. Como $\gamma*\hat{\gamma}  \sim c$. Logo, então $0=\deg(c) = \deg(\gamma*\hat{\gamma} ) = \deg(\gamma) + \deg(\hat{\gamma} )$. Portanto, $\deg(\hat{\gamma} ) = -\deg(\gamma) $.
		\end{enumerate}
	\end{prova}
	
	É fácil mostrar que a aplicação $\phi:\grupofundamental{\circulo}\to  \inteiros$ dada por $\phi (\classe{\gamma})=\deg(\gamma)$ é um isomorfismo de grupos. Com isso, segue-se que
	
	\begin{teorema}\label{teorema_grupo_fundamental_circulo}
		(Grupo fundamental de $\circulo$) $\grupofundamental{\circulo} \cong \inteiros$.
	\end{teorema}
	
	Dados $\gamma \in \caminhospontobasegeral{p}{X}$ e $k \in \mathbb{N}$, a k-justaposição $\gamma*\dots * \gamma$ é denotada por $\gamma^{k}$. O seguinte resultado é uma consequência da aplicação grau.
	
	\begin{proposicao}\label{proposicao_gerador_grupo_fundamental_ciruclo}
		O grupo $\grupofundamental{\circulo}$ é gerado por $\gamma(t) = e^{i2\pi t}$, onde $t \in [0,1]$, isto é, $\grupofundamental{\circulo} = \gerador{\classe{\gamma}} $.
	\end{proposicao}
	\begin{prova}
		Supondo $\beta \in \caminhospontobasegeral{p}{\circulo}$, pela Proposição $\ref{proposicao_grau_aplicacao}$ é possível escrever $\beta(t)=e^{i2\pi \alpha(t)}$ e $\deg(\gamma) = 1$, logo 
		$$
		\deg(\beta) = k = \underbrace{1+\dots+1}_{k-vezes} = \deg(\gamma)+\dots+\deg(\gamma) = \deg(\gamma *\dots *\gamma) = \deg(\gamma^{k}).
		$$
		
		Portanto $\beta \sim \gamma^{n}$ e $\classe{\beta} = \classe{\gamma^{n}}=\classe{\gamma}^{n}$. Logo $\grupofundamental{\circulo} = \gerador{\classe{\gamma}}$.
	\end{prova}
	
	\begin{exemplo}
		(Grupo fundamental de $T^{2}$) Seja $T^{2}=\circulo \times \circulo$ o 2-toro. Como $T^{2}$ é conexo por caminhos, então o grupo fundamental não depende do ponto base escolhido, logo tem-se $\grupofundamental{T^{2}} \cong \grupofundamental{\circulo} \times \grupofundamental{\circulo} \cong \inteiros \times \inteiros$, pelo Lema $\ref{proposicao_produto_grupo_fundamental}$ e o Teorema $\ref{teorema_grupo_fundamental_circulo}$. 
	\end{exemplo}
	
	\begin{exemplo}\label{exemplo_grupo_fundamental_plano_furo}
		(Grupo fundamental $\grupofundamental{\real{2}\backslash\{p\}}$) Sejam $X = \real{2}\backslash \{(0,0)\}$, $p=(1,0) \in X$ o ponto base e $\gamma,c \in \caminhospontobase{p}$ onde $\gamma$ é o círculo envolvendo a origem e $c$ a curva constante. Se $\alpha \in \caminhospontobase{p}$ é um caminho fechado que não envolve a origem, então $\alpha$ pode ser deformada contínuamente para $p$, logo $\alpha \sim c$ e $\classe{\alpha} = Id$. Por definição $\deg(\gamma) = 1$ e $\deg(c) =0$, o que implica que $0= \deg(c)=\deg(\gamma*\hat{\gamma} ) = 1 +\deg(\hat{\gamma} )$, portanto $\deg(\hat{\gamma} )=-1$. Com isso, dado $k \in \inteiros$ tem-se que $\deg(\gamma^{k}) = \deg(\gamma)+\dots +\deg(\gamma) = k$. Seja $\beta \in \caminhospontobasegeral{p}{X}$ tal que $\deg(\beta)=n$. Então $\deg(\beta) = \deg(\gamma^{n})$, e $\beta \sim \gamma^{n}$, pela Proposição $\ref{proposicao_grau_aplicacao}$. Portanto $\classe{\beta} =  \classe{\gamma^{n}}=\classe{\gamma}^{n}$. Logo, o grupo fundamental de $X$ é um grupo infinito cíclico gerado por $\classe{\gamma}$, isto é, $\grupofundamental{X} = \gerador{\classe{\gamma}}$. A aplicação $\grupofundamental{X} \ni \classe{\gamma}^{n} \mapsto n \in \inteiros$ é um isomorfismo, portanto $\grupofundamental{X} \cong \inteiros$.
	\end{exemplo}
		
	\section{Homologia}
	Nesta seção serão apresentados os grupos de homologia singular de um espaço topológico e depois os grupos de homologia celular de um CW-complexo, e as referências utilizadas são \cite{banyaga_morse_homology}, \cite{elon_homologia}, \cite{massey} e \cite{vick_homology}. 
	
	\subsection{Homologia Singular}
	Para as próximas definições, considere $X$ um espaço topológico e $A$ um anel.
	\begin{definicao}
		(Simplexo singular) Seja $\Delta_{k}$ o simplexo k-dimesional cujos vértices $e_{0}, \dots, e_{k}$ formam um base canônica de $\real{k+1}$ de modo que $\Delta_{k} = \{(x_{0}, \dots, x_{k}) \in \real{k+1}: x_{j}\geq 0, \;\sum_{j}x_{j}=1\}$. Um k-simplexo singular no espaço topológico $X$ é uma aplicação contínua $\sigma:\Delta_{k} \to X$.
	\end{definicao}
	
	O A-módulo livre gerado pelas k-cadeias singulares de $X$ é denotado por $\cadeia{k}{X}$. Consequentemente, cada um dos seus elementos é uma soma formal finita $\alpha = \sum_{\sigma} \alpha_{\sigma}.\sigma $ de k-simplexos singulares $\sigma$, onde $\alpha_{\sigma} \in A$.
	
	\begin{definicao}
		(Operador bordo) A i-ésima face do k-simplexo singular $\sigma: \Delta_{k} \to X$ é o (k-1)-simplexo singular $\partial^{i}\sigma:\Delta_{k-1} \to X$ onde $\partial^{i}\sigma = \sigma(t_{0}, \dots, t_{i-1},0,t_{i+1}, \dots, t_{k-1})$. O k-operador bordo é o homomorfismo denotado por $\bordo{k}: \cadeia{k}{X} \to \cadeia{k-1}{X}$ onde $\bordo{k}\sigma = \sum_{i} (-1)^{i}\partial^{i}\sigma$.
	\end{definicao}
	
	Como o $k$-operador bordo é um homomorfismo, então $ker(\bordo{k})$ e $Im(\bordo{k+1})$ são subgrupos abelianos de $\cadeia{k}{X}$, os quais são denotados por $\kernelbordo{k}{X}$ e $\imagembordo{k}{X}$, respectivamente.
	
	\begin{teorema}
		A composição $\bordo{k-1}\circ\bordo{k}: \cadeia{k}{X} \to \cadeia{k-2}{X}$ é o homomorfismo trivial para todo $k>0$.
	\end{teorema}
	
	Uma consequência imediata do teorema anterior é que $\imagembordo{k}{X} \subseteq \kernelbordo{k}{X}$, logo $\imagembordo{k}{X}$ é um submódulo normal de $\kernelbordo{k}{X}$ e o quociente $\kernelbordo{k}{X}/\imagembordo{k}{X}$ define um módulo quociente.
	
	\begin{definicao}
		(k-módulo de homologia singular) Seja $X$ um espaço topológico. Então para cada $k \in \inteirospos$ o módulo quociente
		$$
		\homologia{k}{X} = \frac{\kernelbordo{k}{X}}{\imagembordo{k}{X}}
		$$
		é o $k$-ésimo módulo de homologia singular de $X$ com coeficientes em $A$.
	\end{definicao}
	
	Um módulo graduado $G$ é uma coleção de módulos $\colecao{G}$ tal que as operações de $G$ induzidas em cada $G_{k}$ são fechadas para todo $k \in \inteiros$. Se $G$ e $G'$ são módulos graduados, então um homomorfismo $f:G\to G'$ de grau $r$, onde $r \in \inteiros$ é uma coleção de homomorfismos $\colecao{f}$, onde 
	$$
	f_{k}:G_{k}\to G'_{k+r}.
	$$
	
	Um subgrupo $H\subseteq G$ de um módulo graduado é um módulo graduado $H=\colecao{H}$, onde $H_{k} \subseteq G_{k}$ é um submódulo para $k \in \inteiros$. Nesse caso, o módulo quociente $G/H$ é o módulo graduado $\{G_{k}/H_{k} \}_{k\in \inteiros}$.
	
	Note que, dado um espaço topológico $X$, as coleções $\cadeia{*}{X} = \colecaoabrev{\cadeia{k}{X}}$, $\kernelbordo{*}{X}=\colecaoabrev{\kernelbordo{k}{X}}$, $\imagembordo{*}{X}=\colecaoabrev{\imagembordo{k}{X}}$ e $\homologia{*}{X}=\colecaoabrev{\homologia{k}{X}}$ são A-módulos graduados.
	
	O módulo graduado $\homologia{*}{X}$ é chamado de homologia singular do espaço $X$ e é denotado por
	$$
	\bigoplus_{k\in \inteiros}\homologia{k}{X}.
	$$
	
	\begin{definicao}
		(Complexo de cadeia) Um complexo de cadeia com coeficientes em $A$ é um A-módulo graduado $C_{*} = \colecao{C}$ juntamente com um homomorfismo $\bordo{}:C_{*} \to C_{*}$ (de grau -1)
		$$
		\xymatrix{
			\dots \ar[r]^{\bordo{k+1}}  & C_{k} \ar[r]^{\bordo{k}} & C_{k-1}\ar[r]^{\bordo{k-1}} &\dots
		}
		$$
		tal que a composição $\bordo{k-1}\circ\bordo{k} = 0$ para $k \in \inteiros$. Denota-se o complexo de cadeia por $(C_{*}, \bordo{*})$ e o homomorfismo $\bordo{*}$ é chamado operador bordo do complexo de cadeia.
	\end{definicao}
	
	Note que o par $(\cadeia{k}{X}, \bordo{*})$, onde $\bordo{*}=\colecaoabrev{\bordo{k}}$, forma um complexo de cadeia.
	
	\begin{definicao}
		(Homologia do complexo de cadeia) Seja $C=(C_{*}, \bordo{*})$ um complexo de cadeia. O $k$-ésimo módulo de homologia do complexo de cadeia $C$ é o módulo quociente
		$\homologiaabrev{k}{C} = \kernelbordoabrev{k}{C}/\imagembordoabrev{k}{C}$. O módulo graduado $\homologiaabrev{*}{C} = \colecaoabrev{\homologiaabrev{k}{C}}$ é chamado homologia do complexo de cadeia $C$ e é denotado por
		$$
		\bigoplus_{k\in \inteiros}\homologiaabrev{k}{C}.
		$$				
	\end{definicao}

	\begin{definicao}
		(Aplicação de cadeias) Sejam $C$ e $C'$ dois complexos de cadeias. Uma aplicação de cadeias é um homomorfismo $f:C\to C'$ de grau zero tal que o diagrama abaixo comuta para todo $k \in \inteiros$
		$$
		\xymatrix{
			C_{k}\ar[d]_{\bordo{k}}\ar[r]^{f_{k}}  & C'_{k}\ar[d]^{\bordo{k}'} 
			\\
			C_{k-1}\ar[r]_{f_{k-1}} & C'_{k-1},
		}
		$$
		isto é, $\bordo{k}'\circ f_{k} = f_{k-1}\circ \bordo{k}$.
	\end{definicao}
	
	Dada uma aplicação de cadeias $f:C\to C'$, tem-se que $f(\imagembordoabrev{k}{C})\subseteq \imagembordoabrev{k}{C'}$ e $f(\kernelbordoabrev{k}{C})\subseteq \kernelbordoabrev{k}{C'}$. Logo, $f$ induz um homomorfismo $f_{*}:\homologiaabrev{*}{C}  \to \homologiaabrev{*}{C'}$ de grau 0 definido por $f_{*}(\classe{a}) = \classe{f(a)}$.
	
	\begin{observacao}
		Note que, dadas as aplicações de cadeias $f:C\to C'$ e $f':C'\to C''$, a composição de aplicações de cadeias $f'\circ f:C\to C''$ é uma aplicação de cadeias.
	\end{observacao}
	
	Suponha que $f:X\to Y$ seja uma aplicação contínua entre espaços topológicos. Define-se o homomorfismo induzido por $f$ como sendo a aplicação $\induzida{f}:\cadeia{*}{X}\to \cadeia{*}{Y}$ tal que, dada $\phi \in \cadeia{k}{X}$, tem-se $\induzida{f}(\phi) = f\circ \phi$, para todo $k \in \inteiros$. Pode-se mostra que $\bordo{}'\induzida{f}=\induzida{f}\bordo{}$, ou seja, $\induzida{f}$ é uma aplicação de cadeias.
	
	O teorema a seguir mostra que a homologia singular é um invariante topológico.
	
	\begin{teorema}
		(Isomorfimo induzido) Sejam $f:X \to Y$ um homeomorfismo entre espaços topológicos e $C = \cadeia{*}{X}$ e $C'=\cadeia{*}{Y}$ os complexos de cadeias singulares de $X$ e $Y$, respectivamente. Então $f_{*}: \homologiaabrev{*}{C} \to \homologiaabrev{*}{C'}$ é um isomorfismo.
	\end{teorema}
	
	Analogamente a homotopia de caminhos definida na seção anterior, é possível definir homotopia entre quaisquer duas funções contínuas. Sejam $f, g: X\to Y$ duas aplicações contínuas entre espaços topológicos. Diz-se que $f$ e $g$ são homotópicas se existe uma aplicação contínua $h:\intervalo\times X \to Y$, tal que $h(0, .) = f$ e $h(1, .) = g$. A aplicação $h$ é chamada homotopia entre $f$ e $g$ e denota-se essa propriedade por $f\sim g$.
	
	\begin{teorema}
		Sejam $f, g: X\to Y$ duas aplicações contínuas e homotópicas entre espaços topológicos. Então $f_{*}=g_{*}$.
	\end{teorema}
	
	\begin{definicao}
		(Equivalência homotópica) Seja $f:X\to Y$ uma aplicação contínua entre espaços topológicos. Diz-se que $f$ é uma equivalência homotópica se existe uma aplicação contínua $g:Y \to X$ tal que $f\circ g \sim Id_{Y}$ e $g\circ f \sim Id_{X}$. Neste caso, é dito que $X$ e $Y$ são homotopicamente equivalentes e essa propriedade é denotada por $X\sim Y$.
	\end{definicao}
	
	O resultado seguinte mostra que a equivalência de homotopia é condição suficiente para que os módulos de homologia sejam isomorfos.
	
	\begin{teorema}
		Se $f: X \to Y$ é uma equivalência homotópica entre espaços topológicos, então $f_{*}:\homologiaabrev{*}{C} \to \homologiaabrev{*}{C'}$ é um isomorfismo, onde $C$ e $C'$ são os complexos de cadeias singulares de $X$ e $Y$, respectivamente.
	\end{teorema}
	
	\begin{definicao}
		Uma tripla $\xymatrix{C\ar[r]^{f} & D\ar[r]^{g} & E }$ de módulos e homomorfismos é chamada exata se $Im(f) = Ker(g)$. Uma sequência de módulos e homomorfismos
		$$
		\xymatrix{\dots \ar[r]& G_{1}\ar[r]^{f_{1}} & G_{2}\ar[r]^{f_{2}} & G_{3}\ar[r]^{f_{3}} & \dots \ar[r]\ar[r]^{f_{k-1}} & G_{k}\ar[r]^{f_{k}} &\dots}
		$$
		é chamada de sequência exata longa se cada tripla é exata, isto é, $Im(f_{k}) = Ker(f_{k+1})$ para todo $k\in \inteiros$.
	\end{definicao}
	
	Uma sequência do tipo $ \xymatrix{0 \ar[r] & C\ar[r]^{f} & D\ar[r]^{g} & E \ar[r]& 0}$ é chamada sequência exata curta. Nesse caso, $f$ é um monomorfismo e $g$ é um epimorfismo. Sejam $C, D, E$ complexos de cadeias. Uma sequência exata curta de complexos de cadeias é uma coleção de sequências exatas
	$$ 
	\xymatrix{0 \ar[r] & C_{k}\ar[r]^{f_{k}} & D_{k}\ar[r]^{g_{k}} & E_{k} \ar[r]& 0}
	$$
	para todo $k \in \inteiros$.
	
	Utilizando a sequência exata curta de complexos de cadeias, um homomorfismo entre as sequências exatas curtas, chamado homomorfismo conectante, será utilizado na construção da sequência exata longa. Considere o diagrama abaixo
	$$
	\xymatrix{
		 & \vdots \ar[d]^{\bordo{k+1}} & \vdots \ar[d]^{\bordo{k+1}'} & \vdots \ar[d]^{\bordo{k+1}''}& 
		\\
		0 \ar[r] & C_{k}\ar[d]^{\bordo{k}}\ar[r]^{f_{k}} & D_{k}\ar[d]^{\bordo{k}'}\ar[r]^{g_{k}} & E_{k} \ar[d]^{\bordo{k}''}\ar[r]& 0
		\\
		0 \ar[r] & C_{k-1}\ar[d]^{\bordo{k-1}}\ar[r]^{f_{k-1}} & D_{k-1}\ar[d]^{\bordo{k-1}'}\ar[r]^{g_{k-1}} & E_{k-1}\ar[d]^{\bordo{k-1}''} \ar[r]& 0
		\\
		 & \vdots & \vdots  & \vdots & 
	}
	$$
	Como $g$ é epimorfismo, dado $e \in \kernelbordoabrev{k}{E}$ existe $d \in D_{k}$ tal que $g_{k}(d) = e$. Pela comutatividade do diagrama, tem-se que $g_{k-1}(\bordo{k}'(d)) = \bordo{k}''(g_{k}(d)) = \bordo{k}''(e) = 0$. Portanto, $\bordo{k}'(d) \in Ker(g_{k-1}) = Im(f_{k-1})$, pela exatidão. Com isso, pode-se tomar $c \in C_{k-1}$ tal que $f_{k-1}(c)=\bordo{k}'(d)$. Então $f_{k-2}(\bordo{k-1}(c)) = \bordo{k-1}'(f_{k-1}(c)) = \bordo{k-1}'(\bordo{k}'(d)) = 0$. Como $f$ é monomorfismo, então $\bordo{k-1}(c) \in \kernelbordoabrev{k-1}{C}$. Logo, tem-se uma correpondência $\kernelbordoabrev{k}{E} \ni e \mapsto c \in \kernelbordoabrev{k-1 }{C}$, a qual é denotada por $\delta_{k}(e) = c$. 
	
	\begin{teorema}\label{teorema_homomorfismo_conectante}
		(Homomorfismo conectante) Se $ \xymatrix{0 \ar[r] & C\ar[r]^{f} & D\ar[r]^{g} & E \ar[r]& 0}$ é uma sequência exata curta de complexos de cadeias, então a sequência longa 
		$$
		\xymatrix{\dots \ar[r]^{f_{*}} & \homologiaabrev{k}{D}\ar[r]^{g_{*}} & \homologiaabrev{k}{E}\ar[r]^{\delta_{k}}& \homologiaabrev{k-1}{C}\ar[r]^{f_{*}} & \homologiaabrev{k-1}{D} \ar[r]^{g_{*}}& \dots} 
		$$
		é exata.
	\end{teorema}
	
	Como consequência do Teorema $\ref{teorema_homomorfismo_conectante}$, obtêm-se um importante método de cálculo da homologia de espaços topológicos. Dados $U, V$ subconjuntos abertos de um espaço topológico $X$ tal que $X = U\cup V$, considere a sequência exata curta de complexos de cadeias
	$$ 
	\xymatrix{0 \ar[r] & C_{k}(U\cap V)\ar[r]^{f} & C_{k}(U)\oplus C_{k}(V)\ar[r]^{g} & C_{k}(X) \ar[r]& 0},
	$$
	onde $f(x) = (x,-x)$ e $g(x) = x+y$.
	
	A sequência exata longa induzida como no Teorema $\ref{teorema_homomorfismo_conectante}$ é chamada sequência de Mayer-Vietoris.
	
	\begin{teorema}
		(Sequência de Mayer-Vietoris) Sejam $U, V$ subcojuntos abertos do espaço topológico $X$ tais que $X = U \cup V$. Então existe uma sequência exata longa 
		\[
		\xymatrix{\dots \ar[r]^{\delta_{k+1}} & \homologiaabrev{k}{U\cap V}\ar[r]^{f_{*}} & \homologiaabrev{k}{U}\oplus \homologiaabrev{k}{V}\ar[r]^{g_{*}}& \homologiaabrev{k}{X}\ar[r]^{\delta_{k}} & \homologiaabrev{k-1}{U \cap V} \ar[r]^{f_{*}}& \dots} 
		\]
	\end{teorema}
	
	Seja $U \subseteq X$ um subespaço do espaço topológico $X$ e considere o par topológico $(X, U)$. Caso $U = \emptyset$, identifica-se $(X, \emptyset)$ com $X$. Desse modo, a aplicação de inclusão $U \hookrightarrow X$ induz a aplicação de cadeias $\induzida{i}: \cadeia{*}{U} \to \cadeia{*}{X}$ tal que, para todo $k \in \inteirospos$, $i_{k}$ é um monomorfismo. Assim, é possível identificar $\cadeia{*}{U}$ com um subcomplexo de $\cadeia{*}{X}$ e se pode definir o quociente entre ambos.
	
	\begin{definicao}
		O complexo de cadeia singular do par topológico $(X,U)$ é o complexo de cadeia $(\cadeia{*}{X,U},\bordorel{*} )$ onde 
		$$
		\cadeia{*}{X,U} = \frac{\cadeia{*}{X}}{\cadeia{*}{U}}
		$$
		e $\bordorel{k}: \cadeia{k}{X} \to \cadeia{k-1}{U}$ é definido por $\bordorel{k}(\classe{c}) = \classe{\bordo{k}(c)}$.
	\end{definicao}
	
	\begin{definicao}
		(Homologia relativa) Seja $(X, U)$ um par topológico. O $k$-ésimo grupo de homologia relativa de $X$ mod $U$ é o grupo quociente
		$$
		\homologia{k}{X,U} = \frac{\kernelbordo{k}{X,U}}{\imagembordo{k}{X, U}}.
		$$
	\end{definicao}
	
	\begin{proposicao}
		Dado um par topológico $(X, U)$ tal que $U$ seja um retrato de deformação de $X$, então $\homologia{*}{X,U} = 0$.
	\end{proposicao}
	
	\begin{teorema}
		(Sequência exata do par) Seja $(X, U)$ um par topológico. Então a sequência exata longa desse par é a sequência
		$$
		\xymatrix{\dots \ar[r]^{\delta_{k+1}} & \homologiaabrev{k}{U}\ar[r]^{i_{*}} & \homologiaabrev{k}{X}\ar[r]^{\pi_{*}}& \homologiaabrev{k}{X, U}\ar[r]^{\delta_{k}} & \homologiaabrev{k-1}{U} \ar[r]^{i_{*}}& \dots},
		$$
		onde $\delta_{k}$ é o homomorfismo conectante para $k \in \inteirospos$.
	\end{teorema}
	
	\begin{teorema}
		(Excisão) Seja $(X, U)$ um par topológico. Se $U' \subset U$ com $\overline{U'} \subset int(U)$, então a inclusão $i : (\complementar{X}{U'}, \complementar{U}{U'}) \hookrightarrow (X, U)$ induz um isomorfismo 
		$$
		i_{*}: \homologiaabrev{*}{\complementar{X}{U'}, \complementar{U}{U'}} \to \homologiaabrev{*}{X, U}.
		$$
	\end{teorema}
	
	
	Sejam $U'\subseteq U \subseteq X$ subespaços do espaço topológico $X$ e considere a tripla topológica $(X, U, U')$. Analogamente ao caso do par, o teorema seguinte garante a existência de uma sequência exata longa da tripla.
	
	\begin{teorema}
		(Sequência exata da tripla) Seja $(X, U, U')$ uma tripla topológica. Então a sequência exata longa da tripla é a sequência
		$$
		\xymatrix{\dots \ar[r]^{\delta_{k+1}} & \homologiaabrev{k}{U, U'}\ar[r]^{i_{*}} & \homologiaabrev{k}{X, U'}\ar[r]^{\pi_{*}}& \homologiaabrev{k}{X, U}\ar[r]^{\delta_{k}} & \homologiaabrev{k-1}{U, U'} \ar[r]^{i_{*}}& \dots},
		$$
		onde $\delta_{k}$ é o homomorfismo conectante para $k \in \inteirospos$.
	\end{teorema}
	
	\subsection{CW-Homologia}\label{secao_cw_complexo}
	Um CW-complexo é um tipo de espaço topológico introduzido por J. H. C. Whitehead, que teve como objetivo inicial facilitar alguns cálculos em teoria de homotopia. A ideia é que um CW-complexo é construído com sucessivas colagens (identificações) de outros espaços mais simples (células), de modo que, para se determinar a homologia do todo basta determinar a homologia das partes mais simples, as células.
	\begin{definicao}
		(Colagem de célula) Sejam $X$ um espaço topológico, $D^{n}=\{x\in \mathbb{R}^{n} : ||x|| \leq 1\}$ e $S^{n-1} = \partial D^{n}=\{x\in \mathbb{R}^{n} : ||x|| = 1\}$. Se $f_{\partial}:S^{n-1} \to X$ é uma função contínua, denota-se por $X\cup_{f_{\partial}}D^{n}$ o espaço quociente da união disjunta $X \coprod D^{n}$ pela identificação de $x \in \partial D^{n} = S^{n-1}$ com $f_{\partial}(x) \in X$. Diz-se que $X\cup_{f_{\partial}}D^{n}$ é obtido a partir de $X$ colando uma $n-$célula e $f_{\partial}$ é chamado de mapa de colagem.
	\end{definicao}
	
	\begin{definicao}
		(CW-complexo) Diz-se que um espaço topológico $X$ tem uma CW-estrutura se existem uma sequência de espaços
		$$
		\skeleton{0} \subseteq \skeleton{1} \subseteq \dots \subseteq X = \bigcup \limits_{n\in \mathbb{N}} \skeleton{n}
		$$ 
		tais que:
		\begin{enumerate}
			\item $\skeleton{0}$ é um conjunto discreto de pontos.
			
			\item $\skeleton{n+1}$ é obtido anexando $(n+1)-$células a $\skeleton{n}$.
			
			\item $X$ tem uma topologia fraca, ou seja, um dado $A \subseteq X$ é dito um aberto se, e somente se, $A \cap \skeleton{n}$ for um aberto em $\skeleton{n}$ para todo $n \in \mathbb{N}$.
		\end{enumerate}
	\end{definicao}
	
	Um espaço $X$ com uma CW-estrutura é chamado de CW-complexo e cada subespaço $\skeleton{n}$ é chamado $n-$esqueleto do CW-complexo $X$. Uma aplicação $f_{\partial}:S^{n-1} \to \skeleton{n-1}$ estende a uma aplicação $f:D^{n} \to \skeleton{n}$ chamada aplicação característica. A imagem de $D^{n}$ por $f$ é chamada célula fechada em $X$ e a imagem de $\complementar{D^{n}}{\partial D^{n}} $ por $f$ é chamada célula aberta em $X$.
	
	\begin{exemplo}
		(n-esfera) Será exibida uma estrutura de CW-complexo para $S^{n}$. Fixe um ponto-base $p \in S^{n}$ e defina o $0-$esqueleto $\skeleton{0}=\{p\}$. Anexando uma $(n+1)-$célula a $\skeleton{0}$ tem-se $f_{\partial}: \partial D^{n+1} \to \skeleton{0}$, isto é, $S^{n} \approx \skeleton{n} = \{p\}\cup_{f_{\partial}} \celula{n}{}$.
		\begin{figure}[!h]
			\centering
			\includegraphics[width=5cm,height=3cm]{imagem/cw_esfera.pdf}
			\caption{Estrutura de CW-complexo de $S^{2 }$}
		\end{figure}
	\end{exemplo}

	
	\begin{exemplo}
		(Disco com alça) Sejam $p=(1,0), q=(-1,0) \in D^{2}$ e $I=[a,b] \subset \reta$. Tem-se $\partial I=\{a,b\}$. Definindo $f_{\partial_{0}}: \partial I \to D^{2}$ tal que $f_{\partial_{0}}(a)=p$ e $f_{\partial_{0}}(b)=q$ se obtêm o disco com alça $X=D^{2}\cup_{f_{\partial_{0}}}I$.
		
		\begin{figure}[!h]
			\centering
			\includegraphics[width=5cm,height=3cm]{imagem/disco_alca.pdf}
			\caption{Estrutura de CW-complexo do disco com alça.}
		\end{figure}   
	\end{exemplo}
	
	\begin{exemplo}
		(2-toro) Será exibida uma estrutura de CW-complexo para $T^{2}$. Representando o toro como o quadrado cujos os lados opostos estão identificados preservando a orientação, todos os vértices do quadrado serão identificados a um único ponto $p \in T^{2}$. Defina o $0-$esqueleto como sendo $\skeleton{0} = \{p\}$. As arestas horizontais representam o mesmo $S^{1}$ no toro. Isso equivale a colar uma 1-célula ao 0-esqueleto, ou seja, $\skeleton{0}\cup_{f_{1\partial}}\celula{1}{1}$. Analogamente, as faces verticais também representam o mesmo $S^{1}$ no toro, o que indica que deves-se anexar uma outra 1-célula o espaço anexado anteriormente, isto é, $\skeleton{0}\cup_{f_{1\partial}}\celula{1}{1}\cup_{f_{2\partial}}\celula{1}{2}$. Por fim, tem-se que anexar um 2-célula para cobrir o interior do quadrado. Então $T^{2} =\skeleton{2} = \skeleton{0}\cup_{f_{1\partial}}\celula{1}{1}\cup_{f_{2\partial}}\celula{1}{2}\cup_{f_{3\partial}}\celula{2}{3}$.
	\end{exemplo}
	
	\begin{exemplo}
		(n-espaço projetivo) Será exibida uma estrutura de CW-complexo para $\realprojetivo{n}$. Para $n=0$ tem-se que $\realprojetivo{0} = \{\classe{p}\}$, para um determinado $p \in \real{}$. Já para $n=1$, sabe-se que existe um homeomorfismo $\realprojetivo{1} \approx S^{1}_{\sim} = \{\classe{p}: p \in S^{1},\; p \sim -p\}$. Note esse espaço quociente já possui, naturalmente, uma CW-estrutura pois, na passagem ao quociente, identifica-se todos os pontos do equador com um único ponto desse conjunto. Seja este ponto $p_{0} = (1,0)$, sem perda de generalidade. Isso quer dizer que ao se colar uma 1-célula no ponto $p_{0}$ obtêm-se $S^{1}_{\sim} \approx \{[p_{0}]\} \cup_{f}D^{1} \approx \realprojetivo{0}\cup_{f}D^{1} $ pois $\realprojetivo{0} \approx \{[p_{0}]\}$, portanto $ \realprojetivo{1} \approx \realprojetivo{0}\cup_{f}D^{1}$. Repetindo o procedimento anterior para $\realprojetivo{n} \approx S^{n}_{\sim} $ onde $p_{0} = (1,0,\dots, 0)$, tem-se $\realprojetivo{n} \approx \realprojetivo{n-1} \cup_{f_{\partial}}D^{n}$. Assim, tem-se a CW-estrutura $\realprojetivo{j-1} \subseteq \realprojetivo{j}$ e $\realprojetivo{j} = \skeleton{j} \approx \skeleton{j-1}\cup_{f_{j\partial}}D^{j}$ para $1\leq j \leq n$.
	\end{exemplo}
	
	\begin{lema}\label{homologiacelular}
		(Homologia celular relativa) Sejam $A$ um anel comutativo com unidade e $X$ um CW-complexo, então
		$$
		\homologiarelskelesimpl{k}{n} \cong 
		\left\{
		\begin{array}{cc}
		\mathcal{C}_{n}(X), & k = n\\
		0, & k\neq n\\
		\end{array}
		\right.,
		$$
		onde $\mathcal{C}_{n}(X)$ é um $A-$módulo livre e finitamente gerado pelas $n-$células de $X$. Além disso,
		$$
		\mathcal{C}_{n}(X) \cong \somadir{\sigma} \homologiarelcel{n}{n}{\sigma} \cong \somadir{\sigma} A
		$$
		em que 
		$$
		\somadir{\sigma}f_{\sigma*}: \somadir{\sigma} \homologiarelcel{n}{n}{\sigma} \to \homologiarelskelesimpl{n}{n}
		$$
		denota o isomorfismo descrito.
	\end{lema}
	\begin{prova}
		Por definição, tem-se  $\skeleton{n} = \skeleton{n-1} \bigcup_{f_{\partial \sigma} } \celula{n}{\sigma}$ e também $\celula{n}{\sigma} \subset \skeleton{n}$, onde $f_{\sigma}:\celula{n}{\sigma} \to \skeleton{n}$ é a aplicação característica. Sejam $C_{\sigma}$ e $A_{\sigma}$ discos fechado e abertos, respectivamente, contendo o hemisfério norte de $f_{\sigma}(\celula{n}{\sigma})$. Definindo $N_{\sigma} = f_{\sigma}(\celula{n}{\sigma}) - C_{\sigma}$, $M_{\sigma} = f_{\sigma}(\celula{n}{\sigma}) - A_{\sigma}$ se tem $\overline{N_{\sigma}} \subset M_{\sigma}$, e considerando $U = \skeleton{n} - \bigcup C_{\sigma}$, $Y = \skeleton{n} - \bigcup A_{\sigma}$ se tem $U \subseteq Y$. Note que $\skeleton{n-1}$ é um retrato de deformação de $Y$, logo, pela invariância homotópica tem-se $\homologiarel{k}{\skeleton{n}}{\skeleton{n-1}} \cong  \homologiarel{k}{\skeleton{n}}{Y}$. Como $\skeleton{n} - U = \bigcup C_{\sigma}$ e $Y - U $ é homotópico a $\bigcup S^{n}_{\sigma}$, pelo teorema da excisão $\homologiarel{k}{\skeleton{n} - U}{Y- U} \cong \homologiarel{k}{\skeleton{n}}{Y}$. Portanto $\homologiarel{k}{\skeleton{n} - U}{Y- U} = \homologiarel{k}{\bigcup C_{\sigma}}{\bigcup S^{n}_{\sigma}} \cong \homologia{k}{\bigcup (\celula{n}{\sigma}, \celulabordo{n}{\sigma})} \cong \somadir{\sigma} \homologiarelcel{k}{n}{\sigma}$, pois $C_{\sigma} \approx \celula{n}{\sigma}$ e $S^{n}_{\sigma} = \celulabordo{n}{\sigma}$. Enfim, tem-se o diagrama comutativo:
		\[
		\xymatrix{
			\somadir{\sigma} \homologiarelcel{k}{n}{\sigma} \ar[r]^{id_{*}} \ar[d]^{\cong} & 
			\somadir{\sigma} \homologiarelcel{k}{n}{\sigma} \ar[r]^{id_{*}} \ar[d]^{\somadir{\sigma}f_{\sigma*}} & 
			\somadir{\sigma} \homologiarelcel{k}{n}{\sigma} \ar[d]^{\somadir{\sigma}f_{\sigma*}} 
			\\
			\homologiarel{k}{\skeleton{n} - U}{Y- U} \ar[r]^{\cong} & \homologiarel{k}{\skeleton{n}}{Y} \ar[r]^{\cong} & 
			\homologiarelskelesimpl{k}{n}.
		}
		\]
		Por fim, sabe-se que $\homologiarelcel{k}{n}{\sigma} \cong A$ para $k=n$ e é trivial para $k\neq n$, então pela sequência anterior tem-se $\homologiarelskelesimpl{k}{n} \cong \somadir{\sigma}\homologiarelcel{k}{n}{\sigma} \cong \somadir{\sigma} A$ se $k=n$ e é trivial caso $k\neq n$.
	\end{prova} 
	
	\begin{definicao}
		(Aplicação de Pares) Seja $X = \skeleton{n}$ um CW-complexo e tome $p \in \skeleton{n-1}$ como um ponto-base. Será identificado um dado $q \in \skeleton{n}$ com o ponto-base se $q \in \skeleton{n-1}$, e diz-se que $q \sim p$. Defina $\skeleton{n}/\skeleton{n-1} = \{[q]: q \in \skeleton{n}, \; q \nsim p\}$ e a aplicação quociente $\pi : \skeleton{n} \to \skeleton{n}/\skeleton{n-1}$ por:
		$$
		\pi(q) = 
		\left\{
		\begin{array}{cc}
		\classe{p}, & q \in \skeleton{n-1}\\
		\classe{q}, & q \notin \skeleton{n-1}\\
		\end{array}
		\right..
		$$
		Seja $\bigvee_{\sigma} S^{n}_{\sigma}$ o buquet de $n-$esferas com o ponto-base $p$, então $\skeleton{n}/\skeleton{n-1} \approx \bigvee_{\sigma} S^{n}_{\sigma}$. Agora, definina $s_{\sigma} : \skeleton{n}/\skeleton{n-1} \to S^{n}_{\sigma}$ por 
		$$
		\funcaocond{s_{\sigma}([q])}{q}{q \in \celula{n}{\sigma}}{p}{q \notin \celula{n}{\sigma}}.
		$$
		
		A aplicação $p_{\sigma} = s_{\sigma} \circ \pi : (\skeleton{n}, \skeleton{n-1}) \to (S^{n}_{\sigma}, \{p\})$ é chamada $\sigma-$aplicação de pares.
	\end{definicao}
	
	Seja $\Psi_{n}:\mathcal{C}_{k}(X) \to \homologiarelskelesimpl{k}{n}$ o isomorfismo definido no lema anterior e dado por 
	$$
	\Psi(\sum_{\sigma} n_{\sigma} \sigma) = \sum_{\sigma} n_{\sigma} f_{\sigma *}[D^{n}],
	$$
	onde $[D^{n}]$ é um gerador do módulo $\homologiarelcel{n}{n}{}$.
	
	\begin{lema}
		(Inversa de $\Psi_{n}$) A aplicação inversa $\Phi_{n} : \homologiarelskelesimpl{n}{n} \to \mathcal{C}_{n}(X)$ do isomorfismo definido anteriormente é dada por
		$$
		\Phi_{n}(\alpha) = \sum_{\sigma} \phi_{n}(p_{\sigma *}\alpha)\sigma,
		$$
		onde $\phi_{n}: \homologiarel{n}{S^{n}}{\{p\}} \to \Lambda$ é o único homomorfismo tal que $\phi_{n}([S^{n}])=1$ e $[S^{n}]$ é a classe fundamental do par $(S^{n}, \{p\})$.
	\end{lema}
	\begin{prova}
		Será demonstrada a unicidade do homomorfismo. Sabe-se que o grupo $\homologiarel{n}{S^{n}}{\{p\}}$ tem como geradores as classes $\{[S^{n}], [0]\}$ e como $\phi_{n}$ é homomorfismo, então $\phi_{n}([0]) = 0$. Definindo $\phi_{n}([S^{n}]) = 1$ e supondo que exista outro homomorfismo tal que $\phi_{n}^{'} ([S^{n}]) = 1$, então ambos homomorfismos coincidem quando avaliados nos geradores, logo $\phi_{n}^{'}=\phi_{n}$ o que é uma contradição, portanto $\phi_{n}$ é único. Sabe-se que $\Psi_{n}$ é um isomorfismo, então existe uma única aplicação $\Phi_{n}$ tal que $\Phi_{n} \circ \Psi_{n} = 1$. Tomando $\sigma$ uma $n-$célula geradora de $\mathcal{C}_{n}(X)$,
		$$
		\begin{aligned}
		\Phi_{n}(\Psi_{n}(\sigma)) 
		&= \sum_{\beta}\phi_{n}(p_{\beta *}\Psi_{n}(\sigma))\beta
		\\
		&= \sum_{\beta}\phi_{n}(p_{\beta *}f_{\partial\sigma *}[D^{n}])\beta
		\\
		&= \sum_{\beta}\phi_{n}((p_{\beta}\circ f_{\partial\sigma})_{*}[D^{n}])\beta
		\\
		&= \phi_{n}((p_{\sigma}\circ f_{\partial\sigma})_{*}[D^{n}])\sigma
		\\
		&= \phi_{n}([S^{n}])\sigma
		\\
		&= \sigma	
		\end{aligned},
		$$
		e como $\sigma \in \mathcal{C}_{n}(X)$ é arbitrário, então $\Phi_{n} \circ \Psi_{n} = 1$.
	\end{prova}
	
	\begin{definicao}
		(Grau homológico) Seja $f: S^{n} \to S^{n}$ uma aplicação contínua e $f_{*}: \homologia{n}{S^{n}} \to \homologia{n}{S^{n}}$ o homomorfismo induzindo. Seja $[S^{n}] \in \homologia{n}{S^{n}}$ o gerador não-trivial desse grupo, então $f_{*}[S^{n}] = k[S^{n}]$ para algum $k \in \Lambda$. Denomina-se por $k=\deg(f)$ o grau homológico da aplicação $f$.
	\end{definicao}
	
	\begin{definicao}
		(Aplicação CW-bordo) Tome a tripla $(\skeleton{n}, \skeleton{n-1}, \skeleton{n-2})$ e a composição abaixo
		\[
		\xymatrix{
			\mathcal{C}_{n}(X) \ar[r]^{\Psi_{n}\qquad} &
			\homologiarelskelesimpl{n}{n} \ar[r]^{\delta_{*}} & 
			\homologiarelskele{n-1}{n-1}{n-2} \ar[r]^{\qquad \Phi_{n-1}}&
			\mathcal{C}_{n-1}(X)
		}
		\]
		onde $\delta_{*}$ é o homomorfismo de conexão da sequência da tripla. O homomorfismo $\partial_{n} = \Phi_{n-1} \circ \delta_{*} \circ \Psi_{n} : \mathcal{C}_{n}(X) \to \mathcal{C}_{n-1}(X)$ é chamado operador CW-bordo.
	\end{definicao}
	
	\begin{teorema}
		(CW-bordo) A aplicação CW-bordo é um homomorfismo tal que $\partial_{n-1}\circ\partial_{n} = 0$ e é dado por:
		$$
		\partial_{n}(\sigma) = \sum_{\beta}[\beta:\sigma]\beta,
		$$
		onde $[\beta:\sigma]$ é o grau homológico da aplicação $p_{\beta} \circ f_{\partial\sigma}:\celulabordo{n}{\sigma} \to S^{n-1}_{\sigma}$.
	\end{teorema}
	\begin{prova}
		Por definição tem-se $\partial_{n} = \Phi_{n-1} \circ \delta_{*} \circ \Psi_{n}$, logo é um homomorfismo.
		
		Considere o diagrama comutativo tal que na vertical tem-se a sequência exata longa do par $(\skeleton{n-1}, \skeleton{n-2})$
		$$
		\xymatrix{
			& \homologiaabrev{n-2}{\skeleton{n-2}}\ar[rd]^{j_{*}}
			\\
			\homologiarelskele{n}{n}{n-1} \ar[r]^{\delta_{*} \qquad}\ar[rd]_{\delta_{n}} &
			\homologiarelskele{n-1}{n-1}{n-2} \ar[u]^{\delta_{n-1}} \ar[r]^{ \delta_{*}}&
			\homologiarelskele{n-2}{n-2}{n-3}
			\\
			& \homologiaabrev{n-1}{\skeleton{n-1}}\ar[u]^{j_{*}}
		}
		$$
		Note que $\delta_{*} \circ \delta_{*} = j_{*} \circ \delta_{n-1} \circ j_{*} \circ \delta_{n}$. Pela exatidão da sequência vertical tem-se $Im(j_{*}) = Ker(\delta_{n-1})$, logo $\delta_{*}^{2}=0$. Com isso, tem-se a composição do CW-bordo $\partial_{n-1}\circ \partial_{n} = \Phi_{n-2} \circ \delta_{*} \circ \Psi_{n-1} \circ \Phi_{n-1} \circ \delta_{*} \circ \Psi_{n} = \Phi_{n-2} \circ \delta_{*}^{2} \circ \Psi_{n} =0$, pois $\Psi_{n-1} \circ \Phi_{n-1}=1$.
		
		Por definição $f_{\partial\sigma}: \celulabordo{n}{\sigma} \to \skeleton{n-1}$, assim tem-se o homomorfismo induzido $f_{\partial\sigma*}: \homologia{n-1}{\celulabordo{n}{\sigma} }\to \homologia{n-1}{\skeleton{n-1}}$. Analogamente, tem-se o homomorfismo 
		$$
		f_{\sigma*}:\homologiarelcel{n}{n}{\sigma} \to \homologiarelskelesimpl{n}{n}
		$$ 
		e o homomorfismo conectante 
		$$
		\delta_{n} : \homologiarelcel{n}{n}{\sigma} \to \homologia{n-1}{\celulabordo{n}{\sigma}}
		$$ de modo que, tomanto $[\celula{n}{\sigma}] \in \homologiarelcel{n}{n}{\sigma}$ como um elemento gerador, então $\delta_{n}\circ f_{\sigma*}[\celula{n}{\sigma}] \in \homologia{n-1}{\skeleton{n-1}}$ é um elemento gerador. Por outro lado $f_{\partial\sigma*}\circ \delta_{n}[\celula{n}{\sigma}] \in \homologia{n-1}{\skeleton{n-1}}$ também é um elemento gerador, logo $f_{\partial\sigma*}\circ \delta_{n}[\celula{n}{\sigma}] = \lambda \delta_{n}\circ f_{\sigma*}[\celula{n}{\sigma}]$, para algum $\lambda \in \Lambda$. É possível escolher um mapa caracteristico $f_{\sigma *}$ tal que $\lambda = 1$, tem-se $f_{\partial\sigma*}\circ \delta_{n} = \delta_{n}\circ f_{\partial\sigma*}$, e como $\delta_{*} = j_{*}\circ\delta_{n}$ então $f_{\partial\sigma*}\circ \delta_{*} = \delta_{*}\circ f_{\partial\sigma*}$. Assim, tem-se o operador CW-bordo
		$$
		\begin{aligned}
		\partial_{n}(\sigma) &= \Phi_{n-1}\circ\delta_{*}\circ\Psi_{n}(\sigma)
		\\
		&= \Phi_{n-1}\circ\delta_{*}\circ f_{\sigma*}([\celula{n}{\sigma}])
		\\
		&= \Phi_{n-1}\circ f_{\partial\sigma*}\circ\delta_{*}([\celula{n}{\sigma}])
		\\
		&= \Phi_{n-1}\circ f_{\partial\sigma*}\circ (j_{*}\circ \delta_{n}) ([\celula{n}{\sigma}])
		\\
		&= \Phi_{n-1} \circ f_{\partial\sigma*}([\celulabordo{n}{\sigma}])
		\\
		&= \sum_{\beta} \phi_{n-1}(p_{\beta*}\circ f_{\partial\sigma*}[\celulabordo{n}{\sigma}])\beta
		\\
		&= \sum_{\beta} \phi_{n-1}((p_{\beta}\circ f_{\partial\sigma})_{*}[S^{n-1}])\beta
		\\
		&= \sum_{\beta} \phi_{n-1}(\deg(p_{\beta}\circ f_{\partial\sigma})[S^{n-1}])\beta
		\\
		&= \sum_{\beta} \deg(p_{\beta}\circ f_{\partial\sigma})\phi_{n-1}([S^{n-1}])\beta
		\\
		&= \sum_{\beta} \deg(p_{\beta}\circ f_{\partial\sigma})\beta.
		\end{aligned}
		$$
	\end{prova}
	
	\begin{teorema}\label{teorema_cw_homologia}
		(CW-homologia) Seja $X$ um CW-complexo, então existe uma identificação natural entre a CW-homologia $\mathcal{C}_{*}(X)$ e a homologia singular $\homologia{*}{X}$, isto é 
		$$
		\homologia{k}{X} \cong \homologia{k}{\mathcal{C}_{*}(X)}\; \forall k \in \inteiros.
		$$
	\end{teorema}
	\begin{proof}
		Para a demonstração desse resultado será considerada a sequência 
		$$
		\xymatrix{
			\mathcal{C}_{k+1}(X) \ar[r]^{\partial_{k+1}} & \mathcal{C}_{k}(X) \ar[r]^{\partial_{k}} & \mathcal{C}_{k-1}(X).
		}
		$$
		Por definição tem-se $\homologia{k}{\mathcal{C}_{*}(X)} = Ker(\partial_{k})/Im(\partial_{k+1})$, e com isso será provado que 
		$$
		Ker(\partial_{k})/Im(\partial_{k+1}) \cong \homologia{k}{X}.
		$$ 
		Então pode-se concluir a equivalência entre a CW-homologia e a homologia singular do espaço topológico $X$.
		
		Tome a sequência longa exata vertical da tripla $(\skeleton{k+1}, \skeleton{k-1}, \skeleton{k-2})$ e a sequência longa exata horizontal da tripla $(\skeleton{k+1}, \skeleton{k}, \skeleton{k-1})$  no diagrama abaixo:
		$$
		\xymatrix{
			& & \homologiarelskele{k}{k-1}{k-2}= 0 \ar[d]^{i_{*}} &
			\\
			& & \homologiarelskele{k}{k+1}{k-2} \ar[d]^{j_{*}} &
			\\
			\homologiarelskele{k+1}{k+1}{k} \ar[r]^{\quad\delta_{1*}} &		\homologiarelskele{k}{k}{k-1} \ar[r]^{i_{*}} \ar[rd]^{\delta_{2*}} & \homologiarelskele{k}{k+1}{k-1} \ar[r]^{j_{*}} \ar[d]^{\delta_{3*}} & \homologiarelskele{k}{k+1}{k}=0
			\\
			& & \homologiarelskele{k-1}{k-1}{k-2} &
		}
		$$
		onde $i_{*}, \; j_{*}$ e $\delta_{*}$ são as inclusões induzidas e o homomorfismo conectante, respectivamente. Seja $\classe{\alpha} \in \homologiarelskelesimpl{k}{k}$, então $\delta_{3*}\circ i_{*}\classe{\alpha} = \delta_{2*}\classe{\alpha}$.
		
		Será caracterizado conjunto $Ker(\delta_{2*})$. Dado $[\alpha] \in Ker(\delta_{2*})$, então $\delta_{3*}\circ i_{*}\classe{\alpha} = \delta_{2*}\classe{\alpha} = 0$. Como $i_{*}$ é um epimorfismo e $j_{*}$ é monomorfismo, então existe um único $\classe{\beta} \in \homologiarelskele{k}{k+1}{k-2}$ tal que $i_{*} \classe{\alpha} = j_{*} \classe{\beta}$. 
		
		Afirmo que $\phi: Ker(\delta_{2*}) \to \homologiarelskele{k}{k+1}{k-2}$ dado por $\phi(\classe{\alpha}) = \classe{\beta}$ é um epimorfismo. Seja $\classe{\beta'} \in \homologiarelskele{k}{k+1}{k-2}$, então existe um $\classe{\alpha'} \in \homologiarelskele{k}{k}{k-1}$ tal que $i_{*} \classe{\alpha'} = j_{*} \classe{\beta'}$. Com isso $\delta_{2*}\classe{\alpha'} = \delta_{3*}\circ i_{*}\classe{\alpha'} = \delta_{3*}\circ j_{*}\classe{\beta'} = 0$ pois $Im(j_{*}) = Ker(\delta_{3*})$, logo $\classe{\alpha'} \in Ker(\delta_{2*})$ e $\phi$ é epimorfismo.
		
		Como $\phi$ é sobrejetor, existe $\classe{\alpha} \in Ker(\delta_{2*})$ tal que $\phi(\classe{\alpha}) = 0$. Pela comutatividade do diagrama tem-se que $i_{*}{\classe{\alpha}} = j_{*}\classe{0} = 0$, pois $j_{*}$ é monomorfismo, portanto $\classe{\alpha} \in Ker(i_{*})$ e, pela exatidão, tem-se $Ker(i_{*})=Im(\delta_{1*})$, logo $\classe{\alpha} \in Im(\delta_{1*})$. Com isso se conclui que $Ker(\phi) = Im(\delta_{1*})$. Pelo teorema fundamental do isomorfismo de grupos se tem que $Ker(\delta_{2*})/Ker(\phi) \cong \homologiarelskele{k}{k+1}{k-2}$, ou seja, $Ker(\delta_{2*})/Im(\delta_{1*}) \cong \homologiarelskele{k}{k+1}{k-2}$.
		
		Sem perda de generalidade, assumi-se que $X$ seja um CW-complexo de ordem $n$, isto é, $X= \skeleton{n}$. Fixe um $0 \leq j \leq n$ e tome o $j-$ésimo esqueleto $\skeleton{j}$ e defina $\skeleton{-1}$. Com isso, é possível escrever a sequência de homomorfismos de inclusão de pares:
		$$
		\xymatrix{
			\homologia{k}{\skeleton{j}} = \homologiarel{k}{\skeleton{j}}{\skeleton{-1}}\ar[r]& \homologiarel{k}{\skeleton{j}}{\skeleton{0}} \ar[r] & \dots \ar[r] & \homologiarel{k}{\skeleton{j}}{\skeleton{k-2}}
		}
		$$
		onde $k-2 \leq j$ e para cada $i-$ésimo termo $\homologiarelskele{k}{j}{i}$ no centro do diagrama abaixo, tem-se a sequência exata de triplas nas verticais $(\skeleton{j}, \skeleton{i}, \skeleton{i-1})$, com $0\leq i +1\leq j$ com $h_{i}$ sendo os homomorfismos de inclusão:
		$$
		\xymatrix{
			\homologiarel{k}{\skeleton{i-1}}{\skeleton{i-2}}=0 \ar[d] & \homologiarel{k}{\skeleton{i}}{\skeleton{i-1}}=0 \ar[d] & \homologiarel{k}{\skeleton{i+1}}{\skeleton{i}}=0 \ar[d] &	
			\\
			\homologiarel{k}{\skeleton{n}}{\skeleton{i-2}} \ar[d]^{h_{i-1}} & \homologiarel{k}{\skeleton{n}}{\skeleton{i-1}} \ar[d]^{h_{i}} & \homologiarel{k}{\skeleton{n}}{\skeleton{i}} \ar[d]^{h_{i+1}}
			\\
			\homologiarel{k}{\skeleton{n}}{\skeleton{i-1}} \ar[r]\ar[d]^{\delta_{(i-1)*}}& \homologiarel{k}{\skeleton{n}}{\skeleton{i}} \ar[r] \ar[d]^{\delta_{i*}} &  \homologiarel{k}{\skeleton{n}}{\skeleton{i+1}} \ar[d]^{\delta_{(i+1)*}} 
			\\
			\homologiarelskele{k-1}{i-1}{i-2}=0& \homologiarel{k-1}{\skeleton{i}}{\skeleton{i-1}}=0 &  \homologiarel{k}{\skeleton{i+1}}{\skeleton{i}}=0. &		
		}
		$$
		Do Lema \ref{homologiacelular} tem-se que $\homologiarelskele{k}{i}{i-1} =0$ caso $k \neq i$. Logo os grupos nas extemidades verticais do diagrama serão os triviais. Pela exatidão das sequências verticais se tem $Im(h_{i}) = Ker(\delta_{i*})$, mas como $Im(\delta_{i*}) = 0 \Rightarrow Ker(\delta_{i*}) = \homologiarel{k}{\skeleton{n}}{\skeleton{i}}$, logo $h_{i}$ é um epimorfismo, portanto um isomorfismo, isto é, $\homologiarel{k}{\skeleton{n}}{\skeleton{i-1}} \cong \homologiarel{k}{\skeleton{n}}{\skeleton{i}}$, permitindo escrever a cadeia de isomorfismos 
		$$
		\begin{aligned}
		\homologia{k}{\skeleton{j}} &= \homologiarel{k}{\skeleton{j}}{\skeleton{-1}} 
		\\
		&\cong  \homologiarel{k}{\skeleton{j}}{\skeleton{0}} \cong \dots \cong  \homologiarel{k}{\skeleton{j}}{\skeleton{i}} \cong  \dots \cong \homologiarel{k}{\skeleton{j}}{\skeleton{k-2}},
		\end{aligned}
		$$
		logo $\homologia{k}{\skeleton{j}} \cong \homologiarel{k}{\skeleton{j}}{\skeleton{i}}$.
		
		Por fim, como foi suposto que $X = \skeleton{n}$, e a construção anterior vale para $j = n$, então $\homologia{k}{X} = \homologia{k}{\skeleton{n}} \cong \homologiarelskele{k}{n}{i}$.
		
		Assim, $\homologia{k}{X} \cong \homologiarel{k}{\skeleton{k+1}}{\skeleton{k-2}} \cong Ker(\delta_{2*})/Im(\delta_{1*})$. Mas como $\partial_{k} = \Phi_{n-1}\circ\delta_{*}\circ\Psi_{n}$, então $Ker(\delta_{2*}) \cong Ker(\partial_{k})$ e $Im(\delta_{1*}) \cong Im(\partial_{k+1})$, logo $Ker(\delta_{2*})/Im(\delta_{1*}) \cong Ker(\partial_{k})/Im(\partial_{k+1}) = \homologia{k}{\mathcal{C}_{*}}$, logo $\homologia{k}{X} \cong \homologia{k}{\mathcal{C}_{*}}$ que é a equivalência entre as homologias.
	\end{proof}
	
	\begin{exemplo}
		(CW-homologia da n-esfera)
		Será exibida uma estrutura de CW-complexo para $S^{n}$. Para isso tome um ponto $p \in S^{n}$ e defina o $0-$skeleton $\skeleton{0}=\{p\}$. Em seguida, anexe uma $n-$célula a $\skeleton{0}$ onde $f_{\partial}: \celulabordo{n}{} \to \skeleton{0}$, isto é, $\skeleton{n} = \{p\}\cup_{f_{\partial}} \celula{n}{}$ (veja a Figura \ref{figura_cw_n_esfera}). Pelo teorema da CW-homologia tem-se que $\homologia{k}{S^{n}} \cong \homologiarelskelesimpl{k}{k}$, logo $\homologia{0}{S^{n}} \cong \homologiarelskelesimpl{0}{0} \cong \Lambda$, $\homologia{n}{S^{n}} \cong \homologiarelskelesimpl{n}{n} \cong \Lambda$ e $\homologia{k}{S^{n}} \cong \homologiarelskelesimpl{k}{j} =0$ caso $k \neq j$, logo
		$$
		\homologia{*}{S^{n}} = \homologia{0}{S^{n}}\oplus\homologia{n}{S^{n}} \cong \Lambda\oplus\Lambda.
		$$
		\begin{figure}[!h]
			\centering
			\includegraphics[width=9cm,height=5cm]{imagem/cw_n_esfera.pdf}
			\caption{Estrutura de CW-complexo da n-esfera $S^{n}$.}
			\label{figura_cw_n_esfera}
		\end{figure} 
	\end{exemplo}
	
	\begin{figure}[!h]
		\centering
		\includegraphics[width=15cm,height=7cm]{imagem/cw_2_toro_refeito.pdf}
		\caption{Estrutura de CW-complexo do 2-toro $T^{2}$.}
		\label{figura_cw_2_toro}
	\end{figure} 
	
	\begin{exemplo}
		(CW-homologia do 2-toro) Será exibida uma estrutura de CW-complexo para $T^{2}$. Para isso tome a identificação do toro com o quadrado cujo os lados opostos serão identificados, assim os vertices do quadrado serão um ponto $p \in T^{2}$ e definindo o $0-$skeleton $\skeleton{0} = \{p\}$. Agora será anexadas as faces do quadrado duas $1-$células, isto é, $\skeleton{1} = \skeleton{0}\cup_{f_{1\partial}}\celula{1}{1}\cup_{f_{2\partial}}\celula{1}{2}$. Por fim, deve-se cobrir o centro do quandrado anexando um $2-$célula. Com isso, $\skeleton{2} = \skeleton{1}\cup_{f_{3\partial}}\celula{2}{3}$ (veja a Figura \ref{figura_cw_2_toro}). Então
		$$
		T^{2} =\skeleton{2} = \skeleton{0}\cup_{f_{1\partial}}\celula{1}{1}\cup_{f_{2\partial}}\celula{1}{2}\cup_{f_{3\partial}}\celula{2}{3}.
		$$
		
		Tem-se os grupos de homologia não-triviais:
		$$
		\begin{aligned}
		\homologia{0}{T^{2}} &\cong \homologiarelskele{0}{0}{-1} \cong \Lambda,
		\\
		\homologia{1}{T^{2}} &\cong \homologiarelskele{1}{1}{0} \cong \somadir{i=1,2}\homologiarelcel{1}{1}{i} \cong \somadir{i=1,2}\Lambda
		\\
		\homologia{2}{T^{2}} &\cong \homologiarelskele{2}{2}{1} \cong \Lambda.
		\end{aligned}
		$$
		Logo,
		$$
		\\
		\homologia{*}{T^{2}} = \homologia{0}{T^{2}}\oplus\homologia{1}{T^{2}} \oplus\homologia{2}{T^{2}}\cong \Lambda\oplus\Lambda\oplus\Lambda\oplus\Lambda.
		$$
	\end{exemplo}	
	
	
	\chapter{Teoria de Morse}\label{capitulo_teoria_morse}
	Na investigação da topologia de variedades o principal objetivo é a determinação de seus invariantes topológicos, ou seja, características das variedades que são invariantes por homeomorfismos. Algumas abordagens são algébricas, tais como: a determinação de seus grupos de homotopia, seus grupos de homologia e cohomologia. Outras são diferenciais, tais como: teoremas de mergulho, transversalidade, teorema de Sard, etc. A Teoria de Morse faz uma conexão entre as duas metodologias. Através de uma função suave definida na variedade, determina-se seus pontos críticos, e através deles se constrói um $CW$-complexo cuja homologia coincide com a homologia singular da variedade, o que pode ser encontrado na Seção $\ref{secao_cw_complexo}$. O que torna a Teoria de Morse uma das construções matemáticas mais bonitas do século XX é justamente essa conexão entre as diferentes descrições.
	
	O propósito desse capítulo é de apresentação rápida do formalismo e alguns dos principais resultados para que se possa estender a construção aqui feita para a homologia de variedades de dimensão infinita (Homologia de Floer). Os detalhes das demonstrações e técnicas utilizadas podem ser encontrados em $\cite{milnor}$ e $\cite{banyaga_morse_homology}$.
	
	\begin{definicao}\label{definicao_variedade_fechada}
		(Variedade fechada) Seja $M$ uma n-variedade diferencial. Diz-se que $M$ é fechada se é compacta e sem bordo.
	\end{definicao}
	
	De agora em diante $M$ será uma $n$-variedade Riemanniana diferenciável fechada. Conceitos e propriedades importantes sobre variedades diferenciáveis que serão utilizados neste trabalho podem ser encontrados no Apêndice \ref{apendice_variedade_riemanniana}.
	
	\section{Teoria de Morse Clássica}\label{secao_morse_classica}
	
	\begin{definicao}
		(Função de Morse) Sejam $M$ uma n-variedade fechada, $f \in \funcoessuaves{M}$ e $\pontoscriticos{f} = \{p \in M: df_{p} = 0\}$ o conjunto dos pontos críticos de $f$. Diz-se que $f$ é uma função de Morse se a hessiana $H_{p}(f)$ é não-degenerada para todo $p \in \pontoscriticos{f}$. O conjunto das funções de Morse definidas em $M$ será denotado por $\funcoesmorse{M}$. 
	\end{definicao}

	O índice de Morse de um dado $p \in \pontoscriticos{f}$ é a dimensão do maior subespaço $V\subset T_{p}M $ tal que \ hessiana é negativa-definita, isto é, $H_{p}(f)(v,u)<0$ para todo $v,u \in V$. Esse índice é denotado por $\lambda_{p} = dim(V)$. Como a hessiana é não-degenerada, então $H_{p}(f)$ é diagonalizável e o número de auto-valores negativos é o índice $\lambda_{p}$.
	
	\begin{lema}
		(Lema de Morse) Sejam $f \in \funcoesmorse{M}$ e $p \in \pontoscriticos{f}$ com índice de Morse $\lambda_{p}$. Então existe uma carta $\{U, \phi\}$ de $p$ com $\phi(p)=0 \in \real{n}$ tal que 
		$$
		\begin{aligned}
		(f\circ \phi^{-1})(x_{1}, \dots, x_{n}) &= f(p)-x_{1}^{2}-\dots -x^{2}_{\lambda_{p}}+x^{2}_{\lambda_{p}+1}+\dots + x^{2}_{n}
		\\
		&=f(p)+x^{t}Dx,
		\end{aligned}
		$$
		onde $D$ é a representação diagonal de $H_{p}(f)$.
	\end{lema}
	
	\begin{observacao}
		Como consequência do Lema de Morse, pode-se mostrar que os pontos críticos de uma função de Morse $f \in \funcoesmorse{M}$ são isolados. Pela compacidade de $M$ tem-se que os pontos isolados são finitos.
	\end{observacao}
	
	
	A existência das funções de Morse é ilustrada pelo seguinte exemplo. Seja $f:M\to \reta$ tal que $f(x_{1}, \dots, x_{n}) = x_{n}$. Essa função é chamada de função altura e pode-se mostrar que $f$ é uma função de Morse, logo $\funcoesmorse{M} \neq \emptyset$.
	
	\begin{exemplo}
		Sejam $S^{2} \subset \real{3}$ a 2-esfera centrada na origem e $f:S^{2}\to \reta$ a função altura dadar por $f(x,y,z) = z$. Os pontos críticos de $f$ são $p_{\pm} = \{(0,0,\pm 1)\}$, cujos índices são $\lambda_{p_{-} } = 0$ e $\lambda_{p_{+}} = 2$.
	\end{exemplo}
	
	As funções de Morse não são um caso raro nessa descrição, muito pelo contrário. O seguinte teorema garante que tais funções são abundantes no conjunto das funções suaves e tal resultado pode ser encontrado em $\cite{amyia_diff_topology}$.
	
	\begin{teorema}
		Seja $g\in \funcoessuaves{M}$. Então existe $f \in \funcoesmorse{M}$ suficientemente próxima a $g$, isto é, $\funcoesmorse{M}$ é denso em $\funcoessuaves{M}$.
	\end{teorema} 
	
	Dado $a \in \reta$, defina $M^{a}= f^{-1}((-\infty, a]) = \{p \in M: f(p)\leq a\}$ como sendo o subconjunto em $M$ de nível $a$. Uma consequência imediata é que, dados $a\leq b \in \reta$, então $M^{a} \subseteq M^{b}$.
	
	\begin{teorema}
		Sejam $f \in \funcoesmorse{M}$ e $a<b \in \reta$ tais que $f^{-1}([a,b])\subset M$ não contenha pontos críticos de $f$. Então $M^{a}$ é difeomorfo a $M^{b}$. Além disso, $M^{a}$ é um retrato de deformação de $M^{b}$, de modo que a inclusão  $M^{a} \hookrightarrow M^{b}$ é uma equivalência homotópica.
	\end{teorema}
	
	O seguinte teorema afirma que, fixando $a \in \reta$ e variando $t \in \reta$, se $f^{-1}([a,t]) \cap \pontoscriticos{f} \neq \emptyset$, os conjuntos de nível $M^{a}$ e $M^{t}$ não podem ser deformados um no outro.
	
	\begin{teorema}\label{teorema cw_complexo_ponto_critico_morse}
		Sejam $f\in \funcoesmorse{M}$ e $p\in \pontoscriticos{f}$ com índice $\lambda_{p}$ tal que $f(p) = c$. Suponha que$f^{-1}([c-\epsilon,c+\epsilon])\cap \pontoscriticos{f} = \{p\}$ para algum $\epsilon>0$. Então o conjunto de nível $M^{c+\epsilon}$ tem o mesmo tipo de homotopia de $M^{c+\epsilon}$ com uma $\lambda_{p}$-célula colada, isto é, $M^{c+\epsilon} \simeq M^{c-\epsilon}\cup_{f_{\partial}} D^{\lambda_{p}}$.
	\end{teorema}
	
	\begin{observacao}
		O teorema anterior tem como hipótese a exitência de apenas um ponto crítico em $f^{-1}([c-\epsilon,c+\epsilon])$. No caso em que $f^{-1}([c-\epsilon,c+\epsilon]) \cap \pontoscriticos{f} = \{p_{j}\}_{j=1}^{r}$, tem-se $M^{c+\epsilon} \simeq M^{c-\epsilon}\cup_{f_{\partial_{1}}} D^{\lambda_{1}}\dots  \cup_{f_{\partial_{r}}} D^{\lambda_{r}}$.
	\end{observacao}
	
	\begin{observacao}
		Uma consequência imediata do teorema anterior é que, supondo $p \in \pontoscriticos{f}$ e tomando $a = f(p)$, tem-se que $M^{a}$ é um $CW$-complexo. 
	\end{observacao}
	
	Supondo que $M$ seja compacto e conexo, então $f(M) = [a,b] \subset \reta$ é um compacto. Suponha que $\pontoscriticos{f} = \{p_{j}\}_{j=1}^{k}$, $\lambda_{j}$ seja o índice do j-ésimo ponto crítico e que exista uma partição $t_{1} = a < t_{2}< \dots< t_{k+1} = b$ tal que $f^{-1}((t_{k}, t_{k+1})) \cap \pontoscriticos{f} = \{p_{k}\}$. Ao se tomar $t \in [a,b]$ obtêm-se a construção do $CW$-complexo
	$$
	\begin{aligned}
	M^{t_{1}} &\simeq \{p_{1}\}
	\\
	M^{t_{2}} & \simeq \{p_{1}\} \cup_{f_{\partial_{1}}} D^{\lambda_{1}}
	\\
	\vdots&
	\\
	M = M^{t_{k+1}} &\simeq\{p_{1}\} \cup_{f_{\partial_{1}}} D^{\lambda_{1}}\dots  \cup_{f_{\partial_{k}}} D^{\lambda_{k}}.
	\end{aligned}
	$$
	
	Caso existam pontos críticos distintos com valores críticos coincidentes a mesma construção pode ser feita perturbando a função de Morse $f$ de modo a obter uma outra $f' \in \funcoesmorse{M}$ com os mesmos pontos críticos agora com valores críticos todos distintos. Essa construção é viabilizada pela densidade das funções de Morse em $\funcoessuaves{M}$.
	
	\begin{exemplo}
		(2-toro) Sejam $T^{2} \subset \real{3}$ o toro centrado na origem do plano $Oy\times Ox = \{(x, y ,0) \in \real{3}\}$ e $\varphi:[0,2\pi]\times [0,2\pi] \to \real{3}$ uma parametrização dada por 
		$$
		\varphi(\theta, \phi) = ((2+\cos\phi)\cos\theta, (2+\cos\phi)\sin\theta, \sin\phi).
		$$
		
		Com isso, tem-se que o espaço tangente $T_{p}T^{2} $ é gerado por $ \{\derivadaparcialabrev{\theta}, \derivadaparcialabrev{\phi} \}|_{p}$ onde
		$$
		\begin{aligned}
			\derivadaparcialabrev{\theta} &= -(2+\cos\phi)\sin\theta \derivadaparcialabrev{x}+(2+\cos\phi)\cos\theta \derivadaparcialabrev{y},
			\\
			\derivadaparcialabrev{\phi} &=  - \sin\phi \cos\theta 	\derivadaparcialabrev{x} - \sin\phi \sin\theta 	\derivadaparcialabrev{y} +\cos\phi				\derivadaparcialabrev{z}.
			\end{aligned} 
		$$
				
		Seja $f:T^{2} \to \reta$ a função largura $f(x,y,z)=x$. Tem-se que o diferencial de $f$ e $df_{p} = \derivadaparcialabrev{x}$. Com isso, dado $v \in \real{3}$ tem-se que $df_{p}(v) = \produtointerno{\nabla f(p)}{v}  = \produtointerno{(1,0,0)}{(v_{x}, v_{y}, v_{z})} =v_{x}$. Assim, o caso em que $df_{p}(v) = 0$ implica que $v_{x} = 0$, ou seja, os vetores dos espaços tangentes não podem ter componentes na direção $Ox$. Com isso, basta que os elementos da base não o tenham, isto é, $\sin\phi = \sin\theta = 0$. Portanto, $\phi, \theta \in \{0, \pi\}$ e os pontos críticos de $f$ são $\varphi(0,0) = (3,0,0)$, $\varphi(0,\pi) = (-1,0,0)$, $\varphi(\pi,0) = (-3,0,0)$ e $\varphi(\pi,\pi) = (1,0,0)$, os quais estão ilustrados na Figura \ref{figura_toro_funcao_altura}.
			
		Assim, a representação matricial da Hessiana $H_{p}(f)$ é
			$$
			H(\phi, \theta) = 
			\left(
			\begin{array}{cc}
			-(2+\cos\phi)\cos\theta & \sin\phi \sin\theta  
			\\
			\sin\phi \sin\theta   & -\cos\phi \cos\theta  
			\end{array}
			\right).
			$$
		Avaliando a hessiana nos pontos críticos tem-se
			$$
			H(0, 0) = 
			\left(
			\begin{array}{cc}
			-3 & 0
			\\
			0& -1
			\end{array}
			\right),
			H(0, \pi) = 
			\left(
			\begin{array}{cc}
			3 & 0
			\\
			0& 1
			\end{array}
			\right)
			$$
			$$
			H(\pi, 0) = 
			\left(
			\begin{array}{cc}
			-1 & 0
			\\
			0& 1
			\end{array}
			\right),
			H(\pi, \pi) = 
			\left(
			\begin{array}{cc}
			1 & 0
			\\
			0& -1
			\end{array}
			\right)
			$$
			
			Com isso, os índices $\lambda_{(\phi, \theta)}$ dos pontos críticos são: $\lambda_{(0,0)} = 2$, $\lambda_{(0,\pi)} = 0$, $\lambda_{(\pi,0)} = 1$ e $\lambda_{(\pi,\pi)} = 1$.
			
			\begin{figure}[!h]
				\centering
				\includegraphics[width=8.5cm,height=5.2cm]{imagem/2_toro_funcao_largura.pdf}
				\caption{Pontos críticos da função largura em $T^{2}$.}
				\label{figura_toro_funcao_altura}
			\end{figure}
	\end{exemplo}

	Originalmente, a relação entre a topologia de $M$ e os pontos críticos de uma função de Morse $f:M \to \reta$, foi dada em termos de desigualdades, chamadas desigualdades de Morse.
	
	\begin{definicao}
		(Números de Betti) O j-ésimo número de Betti de $M$ é o inteiro $\beta_{j}(M) = dim(H_{j}(M))$, onde $H_{j}(M)$ é o j-ésimo grupo de homologia singular de $M$.
	\end{definicao}
	
	Como os grupos de homologia de $M$ são invariantes topológicos, então os números de Betti de $M$ também são.
	
	Seja $f \in \funcoesmorse{M}$. O número de pontos críticos de índice $k$ é denotado por $\nu_{k}$. Note que $\beta_{k}(M)$ contém as informações sobre a topologia da variedade, por outro lado, $\nu_{k}$ contém as informações sobre os pontos críticos de $f$.
	
	\begin{teorema}
		(Desigualdades de Morse) Sejam $f \in \funcoesmorse{M}$ e $\nu_{k}$ o número de pontos críticos de $f$ com índice de Morse $k$. Então valem as seguintes desigualdades:
		\begin{enumerate}
			\item $\beta_{k}(M) \leq \nu_{k}$ para $0\leq k\leq n$,
			
			\item $\sum_{j = 0}^{k}(-1)^{k-j}\beta_{j}(M) \leq \sum_{j = 0}^{k}(-1)^{k-j}\nu_{j} $ para $0 \leq k \leq n$ e vale a igualdade para o caso em que $k=n$.
		\end{enumerate}
	\end{teorema}
	
	\begin{observacao}
		No caso da igualdade do segundo item do teorema se tem a característica de Euler-Poincaré $\chi(M) = \sum_{j = 0}^{n}(-1)^{n-j}\beta_{j}(M)$, que também é um invariante topológico e é a generalização do teorema de Euler para poliedros convexos $V-A+F = 2$, onde $V$ é o número de vértices, $A$ é o número de arestas e $F$ é o número de faces.
	\end{observacao}
	
	
	\section{Homologia de Morse-Witten}
	\subsection{Fluxos Gradiente e Variedades de Conexão}\label{secao_fluxo_gradiente}
	Sejam $X \in \campossuaves{M}$ um campo vetorial suave em $M$ e $p \in M$. Sabe-se que existe uma curva $\gamma: \reta\to M$ que é solução do sistema de equações diferenciais 
	$$
	\derivada{\gamma(t)}{t} = X(\gamma(t)), \; \gamma(0) = p.
	$$
	Como $M$ é compacta e sem bordo, então a solução $\gamma = \gamma_{p}$ existe para todo $t\in \reta$. A aplicação $\phi: \retacartesianovariedade\to M$ definida por $\phi(t,p) = \gamma_{p}(t)$ é chamada de fluxo gerado por $X$ e, fixado $p\in M$, a curva $\gamma_{p}:\reta\to M$ é chamada de linha do fluxo de $X$. Como o fluxo $\phi$ está bem-definido para todo $t \in \reta$, então pode-se efetuar a composição $\phi_{s}\circ\phi_{t}(p) = \phi(s+t, p)$ para todo $p\in M$, logo $\phi_{s}\circ\phi_{t} = \phi_{s+t}$ e $\phi_{0} = Id$.
	\begin{definicao}
		(Órbitas) A órbita de $p \in M$ pelo fluxo $\phi$ é a imagem da curva $\phi^{p} = \phi(.,p):\reta\to M$ e será denotada por $\orbitaponto{p}$. As órbitas podem ser de três categorias
		\begin{enumerate}
			\item Singular, se $\orbitaponto{p}$ = \{p\}.
			
			\item Fechada, se existe $\tau \in \reta$ tal que $\phi_{\tau+t}(p) = p$ para todo $t \in \reta$.
			
			\item Regular, quando não é singular e não é fechada.
		\end{enumerate}
		
	\end{definicao}
	
	Pela compacidade de $M$ e pela unicidade da solução do problema de valor inicial anteriormente descrito, tem-se que $\orbitaponto{p}\subset M$ é uma imersão injetiva quando é uma órbita regular. Vista como um subconjunto de $\real{m}$, para um inteiro $m>0$ suficientemente grande (Teorema de Megulho de Whitney $\cite{guillemin_differential_topology}$), $M$ é um conjunto limitado, logo $\orbitaponto{p}$ também é limitado. Assim, tal órbita admite pontos limites. Os conjunto $\alpha$-limite e $\omega$-limite de $p \in M$ são definidos por $\alpha(p) = \{q \in M: \phi_{t}(p) \to q, t \to -\infty\}$ e $\omega(p) = \{q \in M: \phi_{t}(p) \to q, t \to \infty\}$. O seguinte resultado sobre a topologia dos conjuntos limites é feito em $\cite{palis_dynamical_systems}$.
	
	\begin{proposicao}
		Sejam $M$ uma variedade compacta e $X\in \campossuaves{M}$. Então para $p \in M$ tem-se, que $\alpha(p)$ e $\omega(p)$ são fechados, conexos e invariantes pelo fluxo de $X$, isto é, são uniões de órbitas de $X$.
	\end{proposicao}
	
	Será escolhido o campo gradiente da função de Morse $f$ para os estudos seguintes pois dele é possível extrair informações sobre o comportamento dessa função. Munindo $M$ com uma métrica Riemanniana $g: T_{p}M \times T_{p}M \to \reta$, tem-se que $df_{p}(v) = g(\nabla f(p), v)$, onde $v \in T_{p}M$. Sejam $-\gradiente \in \campossuaves{M}$ e $\gamma$ a curva integral de $-\gradiente $ tal que $\gamma(0) = p\in M$. Então
	$$
	\begin{aligned}
	\derivada{}{t}(f \circ \gamma)(t) &= g(\gradiente(\gamma(t)), \dot{\gamma}(t)) 
	\\
	&=g(\gradiente(\gamma(t)), -\gradiente(\gamma(t))) 
	\\
	&= -\norma{\gradiente(\gamma(t))}^{2}
	\\
	&\leq 0,
	\end{aligned}
	$$
	para todo $t \in \reta$.
	
	Isso mostra que a função de Morse $f$ é decrescente ao longo das linhas de fluxo do campo gradiente negativo. Além disso, supondo que $\gamma(0) =p\in \pontoscriticos{f}$, então a desigualdade acima atinge seu maior valor em $t=0$, pois $df_{p }= 0$. Como $f$ é decrescente e atinge seu máximo em $t=0$, então a órbita $\orbitaponto{p}$ é regular e não-fechada. Em $\cite{palis_dynamical_systems}$ mostra-se que, para todo $q \in M$ tem-se que $\orbitaponto{q}$ intersecta $f^{-1}(f(q))$ transversalmente. Além disso, os pontos limites das órbitas são pontos críticos, ou seja, $\alpha(q)\cup\omega(q) \subset \pontoscriticos{f}$.
	
	Sabe-se que, se $f \in \funcoesmorse{M}$, então $\pontoscriticos{f}$ é um conjunto finito. Com isso, pode-se mostrar que 
	
	\begin{lema}\label{lema_conjunto_limite_funcao_morse}
		Se 
		$p \in M$, então $\alpha(p)  = \{q\}$ e $\omega(p) = \{r\}$ onde $q, r \in \pontoscriticos{f}$.
	\end{lema}
	
	\begin{definicao}
		(Variedades instáveis/estáveis)  As variedades instável e estável de um ponto $p \in \pontoscriticos{f}$ são os conjutos $\variedadeinstavel{p} = \{q\in M: \phi_{t}(q) \to p,\; t\to -\infty\}$ e $\variedadeestavel{p} = \{q\in M: \phi_{t}(q) \to p,\; t\to \infty\}$.
	\end{definicao}
	
	\begin{observacao}
		Em termos de pontos limite, é possível reescrever as variedades instáveis e estáveis de $p \in \pontoscriticos{f}$ como sendo $\variedadeinstavel{p} = \{q\in M: \alpha(q)=p\}$ e $\variedadeestavel{p} = \{q\in M: \omega(q)=p\}$.
	\end{observacao}
	
	Da observação anterior, afirma-se que ambas as variedades instável e estável dos pontos críticos são contráteis. De fato, defina $Id: M\to M$ como sendo a aplicação identidade e $c:M\to M$ como sendo a aplicação constante $c(M) =\{p\} $, onde $p \in \pontoscriticos{f}$. Então $h:\intervalo\times \variedadeinstavel{p} \to \variedadeinstavel{p}$ tal que $h(t,q) = \phi(t/(1-t), q)$ é uma homotopia entre $Id$ e $c$, pois é contínua e $h(0, q) = Id(q)$ e $\lim_{t \to 1}h(t, q) = c(q)$, logo a variedade instável é contrátil. Com uma argumento análogo mostra-se que $\variedadeestavel{p}$ é contrátil.
	
	Foi definido o espaço tangente instável como sendo o subespaço $\espacotangentevariedadeinstavel\subset \espacotangentevariedade$ tal que a restrição da Hessiana $\hessianaponto{p}{f}$ a $\espacotangentevariedadeinstavel$ é negativa-definida. Analogamente se tem o espaço tangente estável $\espacotangentevariedadeestavel \subset \espacotangentevariedade$, onde a Hessiana é positiva-definida. Com isso, segue o teorema da variedade estável, cuja demonstração pode ser encontrada em $\cite{banyaga_morse_homology}$.
	
	\begin{teorema}\label{teorema_variedade_instavel_estavel}
		(Teorema da variedade instável/estável) Sejam $f \in \funcoesmorse{M}$ e $p \in \pontoscriticos{f}$. Então se tem a decomposição $\espacotangentevariedade=\espacotangentevariedadeinstavel\oplus\espacotangentevariedadeestavel$. Além disso, existem mergulhos suaves e sobrejetores $\espacotangentevariedadeinstavel \hookrightarrow \variedadeinstavel{p} \subseteq M$ e $\espacotangentevariedadeestavel \hookrightarrow \variedadeestavel{p} \subseteq M$. Com isso, $\variedadeinstavel{p}$ e $\variedadeestavel{p}$ são subvariedades sem bordo com dimensão $\lambda_{p}$ e $n-\lambda_{p}$, respectivamente.
	\end{teorema}
	
	\begin{observacao}
		Do mergulho dado pelo teorema anterior tem-se que as variedades instável e estável possuem o mesmo tipo de homotopia de um disco aberto cujas dimensões são $\lambda_{p}$ e $n-\lambda_{p}$, respectivamente.
	\end{observacao}

	A variedade $M$ pode ser decomposta como a união disjunta das variedades instáveis ou união disjunta da variedades estáveis, o que é garantido pelo resultado a seguir.
	
	\begin{proposicao}\label{proposicao_uniao_variedade_instavel_estavel}
		Se $f \in \funcoesmorse{M}$, então
		$$
		M = \dot{\bigcup_{p \in \pontoscriticos{f}}}\variedadeestavel{p} = \dot{\bigcup_{p \in \pontoscriticos{f}}}\variedadeinstavel{p}.
		$$
	\end{proposicao}

	 O propósito das definições desse ponto em diante é a construção de um complexo de cadeia associado ao fluxo gradiente de uma função de Morse com uma propriedade adicional, que se chama propriedade de transversalidade entre as variedades instáveis e estáveis.
	 
	 \begin{definicao}
	 	(Variedade Conectante e o Espaço Moduli) Sejam $f \in \funcoesmorse{M}$ e $p,q \in \pontoscriticos{f}$. A variedade conectante de $p$ e $q$ é definida por $\variedadeconectantepontos{p}{q} = \variedadeinstavel{p}\cap \variedadeestavel{q}$. Seja $c \in (f(q), f(p)) \subset \reta$ um valor regular de $f$. Então o espaço moduli de $p$ e $q$ é definido por $\espacomoduli{p}{q} = \variedadeconectantepontos{p}{q}\cap f^{-1}(c)$.
	 \end{definicao}
	 
	 \begin{definicao}\label{definicao_aplicao_transversal}
	 	(Aplicações tranversais) Sejam $f: N\to M$ e $g: Z \to M$ duas aplicações suaves entre variedades diferenciáveis. Diz-se que $f$ é transversal a $g$, e denote $f \pitchfork g$, sempre que $f(x) = g(z) = y$ tem-se $df_{x}(\espacotangenteponto{x}{N}) + dg_{z}(\espacotangenteponto{z}{Z}) = \espacotangenteponto{y}{M} $.
	 \end{definicao}
	 
	 \begin{observacao}
	 	Se $Z \subseteq M$ e $g$ é a aplicação de inclusão, então a transversalidade é denotada por $f\pitchfork Z$. A transversalidade nos casos em que $N, Z \subseteq M$ e $f$ e $g$ são as inclusões é denotada por $N \pitchfork Z$.
	 \end{observacao}
	 
	 \begin{definicao}
	 	(Funções de Morse-Smale) Diz-se que o gradiente negativo de $f \in \funcoesmorse{M}$ satisfaz a condição de Morse-Smale se $\variedadeinstavel{p}\pitchfork \variedadeestavel{q}$ para todos $p,q \in \pontoscriticos{f}$. O conjuntos dessas funções é denotado por $\funcoesmorsesmale{M}$.
	 \end{definicao}
	 
	 \begin{teorema}
	 	Sejam $f\in \funcoesmorsesmale{M}$ e $p,q \in \pontoscriticos{f}$. Então a variedade conectante $\variedadeconectantepontos{p}{q}$ e o espaço moduli $\espacomoduli{p}{q}$ são vazios ou subvariedades de $M$ sem bordo cujas dimensões são $dim(\variedadeconectantepontos{p}{q}) = \lambda_{p} -\lambda_{q}$ e $dim(\espacomoduli{p}{q}) = \lambda_{p} -\lambda_{q}-1$. 
	 \end{teorema}
	 
	 \begin{proposicao}
	 	Sejam $f \in \funcoesmorsesmale{M}$ e $p,q \in \pontoscriticos{f}$.
	 	\begin{enumerate}
	 		\item Se $\lambda_{p}<\lambda_{q}$, então $\variedadeconectantepontos{p}{q} = \emptyset$,
	 		
	 		\item $\variedadeconectantepontos{p}{p} = \{p\}$
	 		
	 		\item Se $\lambda_{p} = \lambda_{q}$ e $p\neq q$, então $\variedadeconectantepontos{p}{q} = \emptyset$,
	 		
	 		\item Se $p \neq q$ tal que $\variedadeconectantepontos{p}{q} \neq \emptyset$, então $\lambda_{p}>\lambda_{q}$.
	 	\end{enumerate}
	 \end{proposicao}

	\begin{observacao}
		Uma das consequências da proposição anterior é o fato de que, dada uma função de Morse-Smale, as óbitas não-singulares do fluxo do gradiente negativo dessa função sempre partem de um ponto crítico para um outro ponto crítico de índice inferior.
	\end{observacao}
	
	\begin{proposicao}
		Sejam $f \in \funcoesmorsesmale{M}$ e $p,q \in \pontoscriticos{f}$ de índice relativo 1, isto é, $\lambda_{p} - \lambda_{q} = 1$. Então $\overline{\variedadeconectantepontos{p}{q}} = \variedadeconectantepontos{p}{q} \cup \{p,q\}$. Além disso, o número de órbitas conectando $p$ a $q$ é finito.
	\end{proposicao}

	\subsection{Complexo de Morse-Smale-Witten}
	
	Sejam $V$ um n-espaço vetorial, $B=\{e_{j}\}_{j=1}^{n}$ e $B'=\{e'_{j}\}_{j=1}^{n}$ duas bases ordenadas de $V$. Diz-se que $B$ e $B'$ possuem a mesma orientação se o determinante da matriz de mudança de base, $A: B \to B'$, definida por $e_{j} = \sum_{i=1}^{n}A_{ji}e'_{i}$, possui determinante positivo. 
	
	Uma orientação em um n-espaço vetorial $V$ é uma classe de equivalência entre bases ordenadas de $V$. Um espaço vetorial munido de um orientação é um espaço vetorial orientado. Tal orientação pode ser positiva ou negativa, de acordo com o determinante da matriz de mudança de base for positiva ou negativa.
	
	Sejam $M$ uma n-variedade diferenciável. A n-upla ordenada $B=\{e_{j}\}_{j=1}^{n}$ é um referencial local se $B(p)=\{e_{j}(p)\}_{j=1}^{n}$ é uma base para $\espacotangenteponto{p}{M}$ ordenada para todo $p \in M$. Diz-se que $B$ é um referencial positivamente orientado se $B(p)$ é positivamente orientado para todo $p \in M$. Seja $A: B \to B'$ uma aplicação onde $A(p)$ é a matriz de mudança de base $B(p) $ para $B'(p)$. Se $\det(A):M \to \reta$ for uma aplicação contínua tal que $\det(A)(p) = \det(A(p))>0$ para todo $p \in M$, então é dito que $B$ possui orientação continuamente positiva. Analogamente, se $\det(A)(p)<0$ diz-se que $B$ possui uma orientação continuamente negativa.
	
	\begin{definicao}
		(Variedade orientável) Seja $M$ uma n-variedade diferenciável com um referencial $B$ continuamente orientado. Então a classe de equivalência desses referenciais $o(M)$ é chamada de orientação de $M$ e é dito que a variedade é orientável se existe uma orientção $o(M)$.
	\end{definicao}
	
	A questão de orientação de variedades e subvariedades é crucial no processo de construção do complexo de Morse-Smale-Witten.
	
	\begin{teorema}\label{teorema_orientacao_variedade_instavel}
		Sejam $f \in \funcoesmorsesmale{M}$. Fixando as orientações $o(\variedadeinstavel{p})$ para todo $p \in \pontoscriticos{f}$ tal que $\lambda_{p}>0$, então para todos $p,q \in \pontoscriticos{f}$ tem-se que $\variedadeconectantepontos{p}{q}$ e $\espacomoduli{p}{q}$ são variedades com orientações $o(\variedadeconectantepontos{p}{q})$ e $o(\espacomoduli{p}{q})$ induzidas pela orientação de $\variedadeinstavel{p}$.
	\end{teorema}
	
	\begin{observacao}
		Note que no teorema anterior não foi necessária a orientabilidade de $M$, mas apenas das variedades instáveis dos pontos críticos. O procedimento para a construção da orientação induzida da variedade conectante é:
		\begin{enumerate}
			\item Para cada $p\in \pontoscriticos{f}$ com $\lambda_{p}>0$ fixa-se uma orientação $o(\variedadeinstavel{p})$
			
			\item Considere o subespaço tangente $\espacotangenteponto{p}{\variedadeestavel{p}}$ com a orientação compatível com $\espacotangenteponto{p}{\variedadeinstavel{p}}$
			
			\item Como $\variedadeconectantepontos{p}{q} = \variedadeinstavel{p}\cap\variedadeestavel{q}$ e se tem a transversalidade $\variedadeinstavel{p}\pitchfork\variedadeestavel{q}$, então a orientação $o(\variedadeconectantepontos{p}{q})$ é determinada pelo isomorfismo $\espacotangenteponto{x}{\variedadeinstavel{p}}\cong \espacotangenteponto{x}{\variedadeconectantepontos{p}{q}}\oplus \espacotangenteponto{x}{\variedadeestavel{q}}$, onde $x \in \variedadeconectantepontos{p}{q}$.
		\end{enumerate}
	\end{observacao}

	Sejam $M$ uma n-variedade orientável, $f \in \funcoesmorsesmale{M}$, $p,q\in \pontoscriticos{f}$ tais que $\lambda_{p}-\lambda_{q} = 1$ e $\gamma :\reta \to M$ a curva integral do negativo do gradiente com as condições de contorno a seguir
	$$
	\derivada{}{t}\gamma(t) = -\gradiente(\gamma(t)), \; \lim_{t \to -\infty}\gamma(t) = p\;\;\text{e}\; \lim_{t \to \infty}\gamma(t) = q.
	$$
	Tome um ponto $x \in \gamma(\reta) \subset \variedadeconectantepontos{p}{q}$. Como a variedade instável $\variedadeinstavel{p}$ é orientável, fixe $o(\variedadeinstavel{p})$ como sendo sua orientação. Então existe um referencial $B^{u}$ nessa variedade tal que, juntamente com o campo $-\gradiente$, tem-se que $B(x) = \{-\gradiente(x), B^{u}(x)\}$ é uma base de $\espacotangenteponto{x}{\variedadeinstavel{p}}$.
	
	\begin{definicao}
		(Número de intersecção) Com as hipóteses descritas anteriormente, o sinal característico da órbita $\orbitaponto{x}$ é o índice $n_{x} \in \{\pm 1\}$ que assume os valores $1$, caso a orientação $o(B(x))$ seja equivalente a $o(\variedadeinstavel{p})$, e assume $-1$ caso contrário. O número de intersecção é definido por 
		$$
		n(p,q) = \sum_{x \in \espacomoduli{p}{q} }n_{x}.
		$$
		
		Desse modo, o número de intersecção conta as órbitas entre p e q considerando a orientação.
	\end{definicao}
	
	\begin{definicao}
		(Complexo de Morse-Smale-Witten) Seja $f \in \funcoesmorsesmale{M}$. Para cada $p \in \pontoscriticos{f}$ assuma uma orientação $o(\variedadeinstavel{p})$. Sejam $C_{k}(f)$ o grupo abeliano livremente gerado pelos pontos críticos de índice $k$ e $C_{*}(f) =\bigoplus^{m}_{k=0}C_{k}(f)$. O homomorfismo $\partial_{k}: C_{k}(f)\to C_{k-1}(f)$ definido em cada gerador p de $C_{k}$
		$$
		\partial_{k}(\gerador{p})=\sum_{q \in \pontoscriticos{f}}n(p,q)\gerador{q}
		$$
		é chamado operador bordo de Morse-Smale-Witten e o par $(C_{*}(f), \partial_{*})$ é o complexo de cadeia de Morse-Smale-Witten  da função $f$.
		
	\end{definicao}
	
	\begin{exemplo}
		(Complexo de Morse-Witten) Considere variedade 2-dimensional $M$ mergulhada em $\real{3}$ e representada pela Figura \ref{figura_fluxo_morse_smale}. Seja $F:\real{3} \to \reta$ a função definida por $F(x, y, z) = z$. Pode-se mostrar que $f=F|_{M} \in \funcoesmorsesmale{M}$,  $\pontoscriticos{f} = \{p,q,r,s\}$ e que os índices de Morse dos pontos críticos de $f$ são $\lambda_{q} = 0$, $\lambda_{p} = 1$ e $\lambda_{r}=\lambda_{s} = 2$. 
		
		Pela Figura \ref{figura_variedade_estavel_instavel} note  que $\variedadeestavel{p}$ é composto por duas linhas de fluxo: a linha conectando $r$ a $p$ e a linha conectando $s$ a $p$. Com isso, tem-se que $\variedadeestavel{q} = \complementar{M}{\variedadeestavel{p}}$ e  $\variedadeestavel{r} = \variedadeestavel{s} = \emptyset$. Logo, $M = \variedadeestavel{p}\dot{\cup} \variedadeestavel{q}$, o que está de acordo com a Proposição $\ref{proposicao_uniao_variedade_instavel_estavel}$. Examinando a Figura \ref{figura_fluxo_morse_smale} pode-se verificar que $\variedadeinstavel{p}$ é composto das linhas de fluxo que conectam $p$ a $q$, e que $\variedadeestavel{q} = \variedadeinstavel{p}$. Com isso,  $\variedadeconectantepontos{p}{q} = \variedadeinstavel{p}\cap\variedadeestavel{q} = \variedadeinstavel{p}$ e $\espacomoduli{p}{q}=\{x_{1}, x_{2}\}$. 
		
		Na Figura \ref{figura_variedade_estavel_instavel} está identificada a variedade instável $\variedadeinstavel{r}$ de $r$. Com isso, $\variedadeconectantepontos{r}{p} = \variedadeinstavel{r} \cap\variedadeestavel{p}$ é o caminho que conecta os pontos $r$ a $p$ com o ponto $r$ removido e o espaço moduli $\espacomoduli{r}{p} = \{ x_{3}\}$. Com um argumento análogo conclui-se que $\espacomoduli{s}{p} = \{x_{4} \}$.
		
		Na Figura \ref{figura_fluxo_morse_smale} as linhas de fluxo de $f$ estão representadas pelas setas nos contornos da variedade, e as orientações das variendades instáveis, em linhas roxas. Adotadas essas orentações para $\variedadeinstavel{r}$ e $\variedadeinstavel{s}$, tem-se os índices $n_{x_{3}} =n_{x_{4}}= 1$. Portanto 
		$$
		n(p,q) = n_{x_{1}}+n_{x_{2}} = 0,\; n(r,p) = n(s,p)=n_{x_{3}}= 1. 
		$$
		
		As $k$-cadeias do complexo, para $0\leq k\leq 2$, são
		$$
		C_{0}(f) = \inteiros\gerador{q},\; C_{1}(f) = \inteiros\gerador{p},\; C_{2}(f) = \inteiros\gerador{r}\oplus\inteiros\gerador{s}.
		$$
		E os $k$-operadores bordo são
		$$
		\bordo{0}\gerador{q} =0,\; \bordo{1}\gerador{p}=n(p,q)\gerador{q} = 0,\; \bordo{2}\gerador{r}=\bordo{2}\gerador{s} = n(r,p)\gerador{p} = \gerador{p}.
		$$
		
		\begin{figure}[!h]
			\centering
			\includegraphics[width=8cm,height=6cm]{imagem/pontos_cristicos_morse_smale.pdf}
			\caption{Linhas do fluxo de $-\gradiente$ e as orientações das variedades instáveis $W^{u}(s), W^{u}(p)$ e $W^{u}(r)$.}
			\label{figura_fluxo_morse_smale}
		\end{figure}
		
		
		\begin{figure}[!h]
			\centering
			\includegraphics[width=12cm,height=5cm]{imagem/variedade_estavel_instavel.pdf}
			\caption{Variedades instáveis $W^{u}(s)$ e $W^{u}(r)$ de $s$ e $r$. Variedade estável $W^{s}(q)$ de $q$.}
			\label{figura_variedade_estavel_instavel}
		\end{figure}
	\end{exemplo}

	
	O seguinte teorema é de grande importância pois afirma o isomorfismo entre a homologia do complexo de Morse-Smale-Witten e a homologia singular da variedade. Sua demonstração pode ser encontrada em $\cite{banyaga_morse_homology}$.
	
	\begin{teorema}
		(Teorema da Homologia de Morse) Sejam $f \in \funcoesmorsesmale{M}$ e o par $(C_{*}(f), \partial_{*})$ o complexo de cadeia de Morse-Smale-Witten da função $f$. Então $H_{*}((C_{*}(f), \partial_{*})) $ é isomorfo a homologia singular $ H_{*}(M, \inteiros)$.
	\end{teorema}
	
	O Teorema $\ref{teorema_cw_homologia}$ afirma que a homologia de um $CW$-complexo é isomorfa a sua homologia singular e também é sabido que a homologia de Morse é isomorfa a homologia singular da variedade. Foi visto na Seção $\ref{secao_morse_classica}$ que uma n-variedade fechada possui o mesmo tipo de homotopia que de um $CW$-complexo construido a partir dos pontos críticos de uma função de Morse, logo a homologia singular dessa variedade é isomorfa a homologia do $CW$-complexo associado. Com isso, pode-se afirmar que as homologias construidas via funções de Morse-Smale e funções de Morse clássicas são isomorfas.
	
	O cálculo da homologia singular de uma variedade com certas propriedades nem sempre é algo simples a se fazer. Contudo, é possível aplicar técnicas alternativas para a determinação de homologias isomorfas a homologia singular. Um outro ponto a ser comentado é o fato de que o complexo de Morse é utilizado para a obtenção de alguns invariantes topológicos e outras propriedades  da variedade, por exemplo, a característica de Euler-Poicaré. Portanto, a demonstração desses isomorfismos garante que ainda se está diante dos mesmos invariantes topológicos.
	
	
	\chapter{Espaços Vetoriais Simpléticos}\label{capitulo_espacos_vetoriais_simpleticos}
	
	\section{Origem na Física}
	
	A mecânica clássica visa o estudo da dinâmica de sistemas físicos, conservativos ou não. Diz-se que um campo vetorial $F:\real{3} \to \real{3}$ é conservativo quando, dados dois caminhos $\gamma,\beta:[0,1] \to \real{3}$ de classe $C^{2}$ tais que $\gamma(0)=\beta(0)$ e $\gamma(1)=\beta(1)$ (com extremos fixos), tem-se
	$$
	\tau=\int_{\intervalo} \produtointerno{F(\gamma(t))}{\gamma'(t)}=
	\int_{\intervalo} \produtointerno{F(\beta(t))}{\beta'(t)}.
	$$
	A grandeza $\tau$ é definida como sendo o trabalho realizado pelo campo $F$ ao longo do caminho $\gamma$. Logo, se o campo é conservativo, então o trabalho independe do caminho.
	
	Parte dos sistemas físicos conhecidos podem ser descritos via mecânica Newtoniana, cuja dinâmica é regida pela equação diferencial 
	$$
	F(t) = m\derivada{v(t)}{t} = \derivada{p(t)}{t},
	$$
	onde $v = \gamma'(t)$, $m\geq0$ é a massa da partículo e $p(t) = m\gamma'(t)$ é o seu momento linear. Suponha agora que $F = -\nabla U$, onde a energia potencial $U:\real{3}\to \reta$ é de classe $C^{2}$. Tomando $q=(q_{1},q_{2}, q_{3})\in \real{3}$, pode-se escrever $F(q(t)) =m \ddot{q}(t)= -\nabla U(q(t))$. Tem-se um sistema de 3 equações de segunda ordem, contudo, realizando uma mudança de variáveis, pode-se reduzir o problema de segunda ordem para um problema de primeira ordem, do seguinte modo
	$$
	\dot{q} = \frac{p}{m} \;\; e \;\;\dot{p} = -\nabla U(q).
	$$
	
	O conjunto $Q$ das possíveis posições $q$ da partícula do sistema anterior é chamado espaço de configurações e o conjunto $P=\{(q,p): q,p\in \real{3}\}$ dos pares posição e momento linear é chamado de espaço de fases. No exemplo, $Q=\real{3}$ e $P = \real{6}$, porém ambos podem ser outras variedades diferenciáveis.
	
	Uma função Hamiltoniana é uma função $H:P \to \reta$ de classe $C^{1}$ tal que 
	$$
	\dot{q} = \derivadaparcial{H}{p} \;\; e \;\; \dot{p} = -\derivadaparcial{H}{q}.
	$$
	
	O sistema de equações acima é chamado sistema equações de Hamilton. Sabe-se que a energia total de um sistema físico conservativo é a soma das energias pontecial $E_{p} = U$ e cinética $E_{c} = p^{2}/2m$ (veja $\cite{nussenzveig}$), onde $p^{2} = \produtointerno{p}{p}$. Defina a função Hamiltoniana associada a energia total do sistema por
	$$
	H(q,p) = E_{c} +E_{p} = \frac{p^{2}}{2m}+U(q). 
	$$
	
	Note que a função $H$ definida acima satisfaz as equações de Hamilton, logo é uma função Hamiltoniana e, por estar associada a energia total do sistema, é amplamente aplicada em modelos dinâmicos físicos.
	
	As equações de Hamilton nos dizem que, dada uma função Hamiltoniana, é possível recuperar as equações de Newton. Portanto, tem-se uma compatibilidade entre ambas as descrições de um problema físico.
	
	\section{Geometrização}
	
	A geometrização de problemas em física permite estudar os resultados obtidos sobre outro ponto de vista, e com isso, generalizar e analisar novos aspectos. A Hamiltoniana e as equações de Hamilton serão abordadas em um contexto geométrico, o que permitirá a apresentação dos sistemas hamiltonianos em variedades diferenciáveis.
	
	Sejam $Q= \real{n}$ e $P=\real{2n}$ os espaços de configurações e fase, respectivamente. Com coordenadas $(q, p ) = (q_{1}, \dots ,q_{n}, p_{1}, \dots ,p_{n}) \in \real{2n}$. Desse modo,  
	$$
	B(q_{0}, p_{0})=\{\partial_{q_{1}}, \dots, \partial_{q_{n}}, \partial_{p_{1}}, \dots, \partial_{p_{n}}\}= \{\partial_{q}, \partial_{p}\} $$ 
	é uma base do espaço tangente $T_{(q_{0},p_{0})} P $, onde $\partial_{q_{j}} = \partial/\partial_{q_{j}}$ e $\partial_{p_{j}} = \partial/\partial_{p_{j}}$ avaliados no ponto $(q_{0}, p_{0})$. 
	
	Seja $H:P \to \reta$ uma função Hamiltoniana de classe $C^{\infty}$. Dado o gradiente Hamiltoniano
	$$
	\nabla H =\sum_{j=1}^{n} \bigparenteses{\derivadaparcial{H}{q_{j}}\derivadaparcial{}{q_{j}} + \derivadaparcial{H}{p_{j}}\derivadaparcial{}{p_{j}} }= \derivadaparcial{H}{q}\derivadaparcialabrev{q} + \derivadaparcial{H}{p}\derivadaparcialabrev{p},
	$$
	o campo Hamiltoniano $\campohamiltonianoabrev \in \campossuaves{\real{2n}}$ é definido por 
	$$
	\campohamiltonianoabrev = -\estruturacomplexa \nabla H = \sum_{j=1}^{n}\bigparenteses{\derivadaparcial{H}{p_{j}}\derivadaparcial{}{q_{j}} - \derivadaparcial{H}{q_{j}}\derivadaparcial{}{p_{j}} } = \derivadaparcial{H}{p}\derivadaparcialabrev{q} - \derivadaparcial{H}{q}\derivadaparcialabrev{p}, 
	$$
	onde $\estruturacomplexa$ é a matriz $2n \times 2n$ dada por
	$$
	\estruturacomplexa=
	\left(
	\begin{array}{cc}
	0 & -Id
	\\
	Id & 0
	\end{array}
	\right), 
	$$
	e $Id$ a matriz identidade $n\times n$. Com isso, $\estruturacomplexa \derivadaparcialabrev{q} = \derivadaparcialabrev{p}$ e $\estruturacomplexa \derivadaparcialabrev{p} = -\derivadaparcialabrev{q}$.
	
	Seja $\psi:\reta \to P$ a curva integral do campo $\campohamiltonianoabrev$. Então as equações de Hamilton podem ser reescritas como 
	$$
	\begin{aligned}
	\dot{\psi}(t) &= \campohamiltoniano{\psi(t)}
	\\
	\left(
	\begin{array}{c}
	\dot{q}(t)
	\\
	\dot{p}(t)
	\end{array}
	\right)
	&=
	\left(
	\begin{array}{c}
	\derivadaparcial{H(t)}{p}
	\\
	-\derivadaparcial{H(t)}{q}
	\end{array}
	\right).
	\end{aligned}
	$$	
	
	Afirmo que, ao longo das soluções dos sistema Hamiltoniano, o sistema é conservativo, isto é, a energia total do sistema é constante. De fato
	
	$$
	\begin{aligned}
	\derivada{}{t}H(\psi(t)) 
	&= \produtointerno{\nabla H(\psi(t))}{\dot{\psi}(t)} 
	\\
	&= \produtointerno{\nabla H(\psi(t))}{\campohamiltoniano{\psi(t)}} 
	\\
	&= 
	\produtointerno{\nabla H}{-\estruturacomplexa \nabla H(t)} 
	\\
	&=\produtointerno{\derivadaparcial{H}{q}\derivadaparcialabrev{q} + \derivadaparcial{H}{p}\derivadaparcialabrev{p}
	}{\derivadaparcial{H}{p}\derivadaparcialabrev{q} - \derivadaparcial{H}{q}\derivadaparcialabrev{p}}
	\\
	&=\derivadaparcial{H}{p}\derivadaparcial{H}{q}\produtointerno{\derivadaparcialabrev{q}}{\derivadaparcialabrev{q}}-\bigparenteses{\derivadaparcial{H}{p}}^{2}\produtointerno{\derivadaparcialabrev{q}
	}{\derivadaparcialabrev{p}}	
	+\bigparenteses{\derivadaparcial{H}{q}}^{2}\produtointerno{\derivadaparcialabrev{q}}{\derivadaparcialabrev{p}}
	-\derivadaparcial{H}{p}\derivadaparcial{H}{q}\produtointerno{\derivadaparcialabrev{p}}{\derivadaparcialabrev{p}}
	\\
	&=0.
	\end{aligned}
	$$
	
	Tome a 2-forma $\formaSimpleticaabrev :T_{(q,p)} P \times T_{(q,p)} P \to \reta$ definida por $\formaSimpleticaabrev = \sum_{j}  dq_{j}\wedge dp_{j}$. Com isso, obtêm-se a geometrização das equações de Hamilton
	
	$$
	\begin{aligned}
	\formaSimpleticaPadrao{\campohamiltonianoabrev}{v} 
	&= \sum_{j}  dq_{j}\wedge dp_{j}(\campohamiltonianoabrev, v) 
	\\
	&= \sum_{j}  dq_{j}(\campohamiltonianoabrev)dp_{j}(v) - dq_{j}(v)dp_{j}(\campohamiltonianoabrev)
	\\
	&= \sum_{j} \derivadaparcial{H}{p_{j}}dp_{j}(v) + dq_{j}(v)\derivadaparcial{H}{q_{j}}
	\\
	&= \Big(\sum_{j} \derivadaparcial{H}{q_{j}}dq_{j} +\derivadaparcial{H}{p_{j}}dp_{j} \Big)(v)
	\\
	&= dH(v).
	\end{aligned}
	$$
	
	Define-se um espaço de fase $P$ e em cada ponto do espaço tangente de $P$ tem-se uma forma bilinear anti-simétrica e não-degenerada $\formaSimpleticaabrev$, chamada forma simplética. Com isso, o par $(T_{(q,p)}P, \formaSimpleticaabrev)$ é chamado 2n-espaço vetorial simplético, onde $dim(T_{(q,p)}P) = 2n$. A seção seguinte é dedicada ao estudo de espaços vetoriais simpléticos reais.
	
	\section{Espaços Vetoriais Simpléticos}\label{secao_espacos_vetoriais_simpleticos}
	\begin{definicao}
		(Espaço vetorial simplético) Sejam $V$ um 2n-espaço vetorial real e uma forma bilinear anti-simétrica $\omega$ em $\Lambda^{2}(V, \real{})$ tal que $\omega$ é não-degenerada, ou seja, $\omega(u,v) = 0 \; \forall v \in V \Rightarrow u=0$. O par $(V, \omega)$ é chamado de 2n-espaço vetorial simplético.
	\end{definicao}
	
	\begin{definicao}
		(Base simplética) Seja $(V, \omega)$ um 2n-espaço vetorial simplético, então uma base simplética é uma base $\{ e_{1},\dots, e_{n},f_{1},\dots f_{n}\}$ de $V$ tal que valem as relações:
		$$
		\omega(e_{i}, e_{j}) = \omega(f_{i}, f_{j}) = 0, \; \omega(e_{i}, f_{j}) = \delta_{ij}.
		$$
	\end{definicao}
	
	Sejam $(V_{1}, \omega_{1}), (V_{2}, \omega_{2})$ dois espaços vetoriais simpléticos e uma aplicação linear $\varphi: V_{1}\to V_{2}$. O pullback de $\omega_{2}$ por $\varphi$ é a 2-forma $\varphi^{*}\omega_{2}:V_{1} \times V_{1} \to \reta$ definida por $\varphi^{*}\omega_{2}(v,u) = \omega_{2}(\varphi(v), \varphi(u))$.
	
	Será provada mais adiante a existência da base Simplética. \label{convensao_base_simpletica} Quando não houver ambiguidades, a base simplética  $\{ e_{1},\dots, e_{n},f_{1},\dots f_{n}\}$  de $(V, \omega)$ será denotada por  $\{ e ,f\}$, e um dado $v =\sum_{j=1}^{n}( v_{j}e_{j} +v_{n+j}f_{j}) \in V$ por $v=v_{(1)}e+v_{(2)}f$, onde $v_{(1)}=(v_{1}, \dots, v_{n})$ e $v_{(2)}=(v_{n+1}, \dots, v_{2n})$.


	Dados $v=v_{(1)}e+v_{(2)}f$ e $u=u_{(1)}e+u_{(2)}f$ em $V$, tem-se que 
	$$
		\formaSimpletica{v}{u} = \produtointerno{v_{(1)}}{u_{(2)}}-\produtointerno{v_{(2)}}{u_{(1)}}, 
	$$
	onde $\produtointerno{v_{(1)}}{u_{(2)}}=\sum_{j=1}^{n}v_{j}u_{n+j}$ e  $\produtointerno{v_{(2)}}{u_{(1)}}=\sum_{j=1}^{n}v_{n+j}u_{j}$.
	
	\begin{definicao}
		(Simplectomorfismo) Dois espaços vetoriais simpléticos $(V_{1}, \omega_{1}), (V_{2}, \omega_{2})$ são ditos simplectomorfos se existe um isomorfismo $\varphi:V\to W$ que preserva a forma simplética, isto é, $\varphi^{*}\omega_{2} = \omega_{1}$.
	\end{definicao}
	\begin{exemplo}\label{exemplo_espaco_simpletico_real}
		Seja $V = \real{2}$, $\{e_{x}, e_{y}\}$ uma base de $V$ e $w=dx \wedge dy$. Então $\omega(e_{x}, e_{y}) = (dx \wedge dy)(e_{x}, e_{y}) = dx\otimes dy(e_{x}, e_{y})-dy\otimes dx(e_{x}, e_{y}) =dx(e_{x}) dy(e_{y}) - dx(e_{y}) dy(e_{x}) = 1-0= 1$. Por outro lado, $\omega(e_{y}, e_{x}) =dx(e_{y}) dy(e_{x}) - dx(e_{x}) dy(e_{y}) =-1 =-\omega(e_{x}, e_{y})$, logo é anti-simétrica. Além disso, $\omega(e_{x}, e_{x}) = \omega(e_{y}, e_{y}) = 0$. Fixando $v \in V$ e para qualquer $u \in V$ tem-se que $\omega(v, u) = \omega(v_{x}e_{x}+v_{y}e_{y}, u_{x}e_{x}+u_{y}e_{y}) = v_{x}u_{y}\omega(e_{x}, e_{y}) +v_{y}u_{x}\omega(e_{y}, e_{x}) = v_{x}u_{y} -v_{y}u_{x} = 0$ se, e somente se, $v_{x}=v_{y}=0$, logo $\omega$ é não-degenerada. Para ver isso basta tomar $u_{x} = 1$ e $u_{y} = 0$, isso implica que $v_{y} = 0$. Fazendo $u_{x} = 0$ e $u_{y} = 1$, $v_{x} = 0$, logo $v=0$. Dado $\varphi:V \to V$ tal que $\varphi(v) = -v$, então $\varphi^{*}\omega(v, u) = \omega(\varphi v, \varphi u)=\omega(-v, -u)=\omega(v, u)$. Portanto $\varphi^{*}\omega = \omega$ e $\varphi$ é um simplectomorfismo.
	\end{exemplo}
	
	\begin{definicao}\label{definicao_subespaco_simpletico_ortogonais}
		(Espaços $\omega$-ortogonais) Seja $(V, \omega)$ um 2n-espaço vetorial simplético e $W\subseteq V$ um subespaço vetorial. Então o complemento $\omega$-ortogonal de $W$ é o subespaço
		$$
		W^{\omega} = \{v\in V: \omega(v,u) = 0,\;\forall u\in W \}.
		$$
		O subespaço $W$ é chamado
		
		\begin{enumerate}
			\item \text{Simplético}, se $W\cap \espacoSimpleticoOrtogonal{W} = \{0\}$;
			
			\item \text{Isotrópico}, se $W \subseteq \espacoSimpleticoOrtogonal{W}$;
			
			\item \text{Coisototrópico}, se $W\supseteq \espacoSimpleticoOrtogonal{W}$;
			
			\item \text{Lagrangiano}, se $W =\espacoSimpleticoOrtogonal{W}$.
		\end{enumerate}
	\end{definicao}
	
	\begin{proposicao}\label{proposicao_subespaco_simpletico_ortogonal}
		(Caracterização de subespaços)
		\begin{enumerate}
			\item $W$ é simplético se, e somente se, então $(W, \omega|_{W \times W})$ é um espaço vetorial simplético.
			
			\item $W$ é isotrópico se, e somente se, então $\omega|_{W\times W} = 0$.
			
			\item $W$ é lagrangiano se, e somente se, então $W$ é isotrópico e máximal, isto é, $W$ não está contido propriamente em nenhum outro subespaço isotrópico. 
		\end{enumerate}
	\end{proposicao}
	\begin{prova}
		\begin{enumerate}
			\item Supondo que $\omega|_{W \times W}$ seja degenerada, então existe $0\neq v \in W$ tal que $\omega(v, u ) = 0$ para todo $u \in W$, o que implica que $v \in W\cap W^{\omega}$, contradizendo a hipótese $W\cap W^{\omega} =0$. Logo $\omega|_{W \times W} $ é não-degenerada e $(W, \omega|_{W \times W})$ é um espaço vetorial simplético. A recíproca é imediata.
			
			\item Dado $0\neq v \in W $ , então $\omega(v,u) = 0$ para todo $u \in W\cap W^{\omega} = W$. Portanto $\omega|_{W\times W}: W\times W \to \reta$ é a aplicação nula. A recíproca é imediata.
			
			\item  Como $W=W^{\omega}$, então $W\subseteq W^{\omega}$, logo $W$ é isotrópico. Seja $U \subseteq V$ um subespaço isotrópico tal que $W \subseteq U$. Então, pelo item anterior, $\omega|_{W \times W} = 0$, logo dado $v \in W \cap U$ e para todo $u \in U$, tem-se $\omega(v, u) = 0$, implicando $u \in W^{\omega}$. Portanto, $U = W^{\omega} = W$. A recíproca é imediata.
		\end{enumerate}
	\end{prova}
	
	Não há garantias de que sempre se tenha $V = W + \espacoSimpleticoOrtogonal{W}$. Contudo, a proposição a seguir dá uma relação entre as dimensões desses dois espaços vetoriais.
	
	\begin{proposicao}\label{proposicao_isomorfismo_forma_simpletica}
		Seja $(V,\omega)$ um 2n-espaço vetorial simplético. A aplicação $\omega^{*}:V\to V^{*}$ definida por $\omega^{*}(v)(u) = \omega(v,u)$ é um isomorfismo.
	\end{proposicao}
	\begin{prova}
		Dado $v \in V$, tem-se pela bilinearidade de $\omega$ que $\omega^{*}(v)$ é linear, logo $\omega^{*}(v) \in V^{*}$. Além disso, $\omega^{*}$ é injetora pois, se $v\neq 0 \in V$ tal que $\omega^{*}(v) = 0$, então $0= \omega^{*}(v)(u) = \omega(v,u)$ para todo $u \in V$, o que é contradição, pois $\omega$ é não-degenerada, logo $v =0$ e $\omega^{*}(0) = 0$. Seja $\colecaofinita{e}{2n}$ uma base de $V$. Denotando $b^{*}_{j}$ por $\omega^{*}(e_{j}) \in V^{*}$, tem-se pela injetividade que $b^{*}_{j} \neq b^{*}_{i}$ para todos $1\leq j,i\leq 2n$. Suponha que $\colecaofinita{b^{*}}{2n}$ seja linearmente dependente. Com isso, existem $\colecaofinita{\lambda}{2n} \subset \reta$ nem todos nulos tais que $0=\sum_{j=1}^{2n}\lambda_{j}b^{*}_{j} = \sum_{j=1}^{2n}\lambda_{j}\omega^{*}(e_{j}) = \omega^{*}(\sum_{j=1}^{2n}\lambda_{j}e_{j})$. Pela injetividade de $\omega^{*}$ deve-se ter $\sum_{j=1}^{2n}\lambda_{j}e_{j} = 0$, contradizendo a hipótese de que $\colecaofinita{e}{2n}$ é uma base de $V$. Portanto, $\colecaofinita{b^{*}}{2n}$ é uma base de $V^{*}$ e $\omega^{*}$ é sobrejetor, logo é um isomorfismo.
	\end{prova}
	
	\begin{proposicao}\label{proposicao_dimensao_subespaco_simpletico}
		Sejam $(V,\omega)$ um 2-espaço vetorial simplético e $W \subseteq V$ um subespaço vetorial. Então $dim(V) = dim(W) + dim(\espacoSimpleticoOrtogonal{W})$.
	\end{proposicao}
	\begin{prova}
		Seja $\omega^{*}: V \to V^{*}$ tal que $\omega^{*}(v)(u) = \omega(v,u)$. Pela Proposição $\ref{proposicao_isomorfismo_forma_simpletica}$ $\omega^{*}$ é um isomorfismo. Seja $W^{\circ}=\{f\in W^{*}: f(v) = 0,\; \forall v\in W \}$ o anulador de $W$. Tomando $f \in \omega^{*}(\espacoSimpleticoOrtogonal{W})$ tem-se $f(u) = \omega^{*}(v)(u)=\omega(v,u)$, para algum $v \in \espacoSimpleticoOrtogonal{W}$. Se $u\in W$, então $f(u) = 0$. Portanto $f \in W^{\circ}$ e  $\omega^{*}(\espacoSimpleticoOrtogonal{W})\subseteq W^{\circ}$. Por outro lado, seja $f \in W^{\circ}$. Como $\omega^{*}$ é um isomorfismo, então existe $v \in v$ tal que $f = \omega^{*}(v)$, logo $0=f(u) = \omega^{*}(v)(u) = \omega(v,u)$ para todo $u \in W$, o que implica que $v \in \espacoSimpleticoOrtogonal{W}$. Portanto, $f \in \omega^{*}(\espacoSimpleticoOrtogonal{W})$ e $W^{\circ} \subseteq \omega^{*}(\espacoSimpleticoOrtogonal{W})$. Logo $W^{\circ} =\omega^{*}(\espacoSimpleticoOrtogonal{W})$.
		Como $\omega^{*}$ é um isomorfismo, então $dim(\espacoSimpleticoOrtogonal{W}) = dim(\omega^{*}(\espacoSimpleticoOrtogonal{W}))$. Tem-se que 
		$$
		dim(V) = dim(W)+dim(W^{\circ}) = dim(W)+dim(\omega^{*}(\espacoSimpleticoOrtogonal{W})) = dim(W)+dim(\espacoSimpleticoOrtogonal{W}).
		$$ 
	\end{prova}
	
	\begin{teorema}\label{teorema_existencia_base_simpletica}
		(Existência de base simplética) Todo espaço vetorial simplético de dimensão finita possui uma base simplética.
	\end{teorema}
	\begin{prova}
		Sejam $(V, \omega)$ um 2n-espaço vetorial simplético e $e_{1} \neq 0\in V$. Como $\omega$ é não-degenerada, então existe $f_{1} \in V$ tal que $\omega(e_{1}, f_{1}) = 1$. Definindo $V_{1} = span\{e_{1}, f_{1}\}$, pode-se afirmar que $V_{1}$ é simplético, ou seja, $V_{1}\cap V_{1}^{\omega} = \{0\}$. Com isso e com a Proposição $\ref{proposicao_dimensao_subespaco_simpletico}$, pode-se afirmar que $V = V_{1}\oplus V_{1}^{\omega}$. Efetuando o mesmo procedimento n vezes tem-se $V = V_{1}\oplus \dots \oplus V_{n}$, onde $V_{i}$ é gerado por $e_{i}$ e $\omega(e_{i}, f_{i}) = 1$. Por construção, tem-se $\omega(e_{i}, e_{j})=\omega(f_{i}, f_{j}) =0$ e $\omega(e_{i}, f_{j}) = \delta_{ij}$. Portanto, $\{e_{1}, \dots e_{n}, f_{1}, \dots f_{n}\}$ é uma base simplética.
	\end{prova}
	
	\begin{corolario}
		
		Todo espaço vetorial simplético é simplectomorfo a $(\real{2n}, \omega_{0})$ para algum $n$, onde $\formaSimpleticaPadrao{u}{v}=-U^t\estruturacomplexa V$ e $U$, $V$ são as matrizes de $u$ e $v$ na base canônica, respectivamente.
	\end{corolario}
	
	\begin{observacao}\label{observacao_existencia_base_simpletica}
		A existência de uma base simplética garante a existência de uma base em que a forma simpléctica $\omega$ poderá ser representada pela matriz
		$$
		\left(
		\begin{array}{cc}
		0 & Id
		\\
		-Id & 0
		\end{array}
		\right).
		$$
	\end{observacao}
	
	\begin{teorema}
		Seja $V$ um 2n-espaço vetorial, então existe uma base $\{ e_{1},\dots, e_{n}, f_{1},\dots, f_{n}\}$ de $V$ e uma base $\{e_{1}^{*}, \dots, e_{n}^{*}, f_{1}^{*}, \dots,f_{n}^{*}\}$ de $V^{*}$ tal que dado $\alpha \in \Lambda^{2}(V)$ pode-se escrever $\alpha = \sum_{i=1}^{n} e^{*}_{i}\wedge f^{*}_{i}$.
	\end{teorema}
	\begin{prova}
		Seja $\alpha\neq 0\in \Lambda^{2}(V) $, então existem $ a_{1}, a_{1+n} \in V $ tais que $\alpha(a_{1}, a_{1+n}) = \alpha_{1} \neq 0$. Definindo $e_{1} = a_{1}/\alpha_{1}$ e $a_{1+n} = f_{1}$ tem-se $\alpha(e_{1}, f_{1}) = 1$, e pela anti-simetria tem-se $\alpha(e_{1}, e_{1}) = \alpha(f_{1}, f_{1}) = 0$. Definindo $B_{1}=span \{e_{1}, f_{1}\}$, então a matriz $(\alpha_{ij})$ de $\alpha|_{B_{1}}$ é
		$$
		\left(
		\begin{array}{cc}
		0 & 1
		\\
		-1 & 0
		\end{array}
		\right)
		$$
		Seja  $V_{2} = \{v \in V: \alpha(v, b) = 0,\; b \in B_{1}\}$, então por construção tem-se $B_{1} \cap V_{2} = 0$. Como $B_{1}, V_{2} \subseteq V$ são subespaços vetoriais então $V = B_{1}\oplus V_{2}$. Dado $v \in V$ tem-se $v_{2} =v- \alpha(v,f_{1})e_{1} +\alpha(v,e_{1})f_{1} \in V_{2}$ pois $\alpha(v_{2}, e_{1}) = \alpha(v_{2}, f_{1}) = 0$. Repetindo a construção para $V_{2}$ se pode afirmar que existem $e_{2}, f_{2} \in V_{2}$ tais que $\alpha(e_{2}, f_{2}) = 1$, $\alpha(e_{2}, e_{2}) = \alpha(f_{2}, f_{2}) = 0$, $B_{2} = span\{e_{2}, f_{2} \}$ e $V_{3} \subset V_{2}$ tal que $B_{2}\cap V_{3}=0$, onde a matriz $(\alpha_{ij})$ de $\alpha|_{B_{2}}$ é da mesma forma que a matriz de $\alpha|_{B_{1}}$. Realizando uma indução finita na construção dos planos $B_{j}$ tem-se $V = \bigoplus_{j=1}^{n}B_{j}$, logo a matriz de $\alpha$ na base  $\{ e_{1},\dots, e_{n}, f_{1},\dots, f_{n}\}$ é
		$$
		\left(
		\begin{array}{cc}
		0 & Id
		\\
		-Id & 0
		\end{array}
		\right)
		$$
		Definindo a base dual $\{e_{1}^{*}, \dots, e_{n}^{*}, f_{1}^{*}, \dots,f_{n}^{*}\}$ de $V^{*}$ se tem $\alpha = \sum_{j=1}^{n}e_{j}^{*}\wedge f_{j}^{*}$.
	\end{prova}
	
	\begin{lema}\label{lema_caracterizacao_forma_simpetica}
		(Caracterização da forma simplética) Sejam $V$ um 2n-espaço vetorial, então $\omega \in \Lambda^{2}(V)$ é uma forma simplética se, e somente se, $\omega^{\wedge n} = \omega\wedge \dots \wedge \omega \in \Lambda^{2n}(V)$ é não-nula.
	\end{lema}
	\begin{prova}
		Suponha $\omega$ uma forma simplética. O Teorema $\ref{teorema_existencia_base_simpletica}$ garante a existência de uma base simplética $\{e_{1}, \dots, e_{n}, f_{1}, \dots, f_{n}\}$ de $V$. Pela bilinearidade de $\omega$ basta analisar $\omega$ nos elementos dessa base. Considerando $\sigma$ no conjunto das permutações de $\{1, 2, \dots , 2n\}$ tem-se
		$$		
		\begin{aligned}
		\omega^{\wedge n}(e_{1}, \dots, e_{n}, f_{1}, \dots, f_{n}) &=\sum_{\sigma} \omega(e_{1}, f_{\sigma(1)})...\omega(e_{n}, f_{\sigma(n)})
		\\
		&= \sum_{\sigma}\delta_{1\sigma(1)}\dots\delta_{n\sigma(n)}
		\\
		&= \delta_{11}\dots\delta_{nn}
		\\
		&= 1.
		\end{aligned}
		$$
		Por outro lado, se $\omega^{\wedge n} \neq 0$, suponha que $\{v_{1},\dots, v_{2n}\}$ seja uma base de $V$ e que exista $v =\sum a_{j}v_{j} \neq 0$ tal que $\omega(v, u) = 0$ para todo $u=\sum b_{j}v_{j}  \in V$. Então, $0=\omega(v, u ) = \sum_{j, k} a_{j}b_{k}\omega(v_{j}, v_{k})$, o que implica em $\omega(v_{j}, v_{k}) =0$. Então 
		$$
		\omega^{\wedge n}(v_{1},\dots, v_{2n}) = \sum_{\sigma} \omega(v_{1}, v_{\sigma(1)})...\omega(v_{2n}, v_{\sigma(2n)})=0,
		$$
		contradizendo a hipótese $\omega^{\wedge n} \neq 0$. Logo, $\omega$ é não-degenerada.
	\end{prova}
	
	\begin{proposicao}\label{proposicao_preservacao_volume}
		(Preservação do volume) Sejam $(V,\omega)$ um 2n-espaço vetorial simplético e $\varphi:V\to V$ um simplectomorfismo, então $\varphi^{*}\omega^{\wedge n}=\omega^{\wedge n}$ e $\det(\varphi)=1$.
	\end{proposicao}
	\begin{prova}
		Seja $\varphi:V \to V$ um simplectomorfismo, então $\varphi^{*}\omega = \omega$, e portanto
		$$
		\begin{aligned}
		\varphi^{*}\omega^{\wedge n} 
		&= 
		\varphi^{*}(\omega\wedge \dots \wedge\omega) 
		\\
		&= \varphi^{*}\omega\wedge \dots \wedge\varphi^{*}\omega
		\\
		&=\omega\wedge \dots \wedge \omega 
		\\
		&= \omega^{\wedge n}.
		\end{aligned} 
		$$
		Aplicando esse resultado vê-se que
		$$
		\begin{aligned}
		\omega^{\wedge n}(e_{1}, \dots,e_{n}, f_{1},\dots, f_{n})
		&=(\varphi^{*}\omega^{\wedge n})(e_{1}, \dots, e_{n}, f_{1},\dots, f_{n})
		\\
		&=
		\omega^{\wedge n}(\varphi e_{1}, \dots,\varphi  e_{n}, \varphi f_{1},\dots, \varphi f_{n})
		\\
		&=\det(\varphi)\omega^{\wedge n}(e_{1}, \dots, e_{n}, f_{1},\dots, f_{n}),
		\end{aligned}
		$$
		portanto $\det(\varphi) = 1$.
	\end{prova}
	
	
	\begin{definicao}\label{definicao_transformacao_simpletica}
		(Transformação simplética) Seja $(V, \omega)$ um 2n-espaço vetorial simplético sobre $\reta$. Um operador linear $T: V \to V$ é uma transformação simplética se 
		$$
		\formaSimpletica{Tu}{Tv} = \formaSimpletica{u}{v}
		$$ para todo $u,v\in V$.
	\end{definicao}
	
	\begin{definicao}\label{definicao_grupo_simpletico}
		(Grupo simplético) O grupo simplético $\gruposimpletico{V} \subset \generalgroupreal{2n}$ de $V$ é o conjunto das matrizes associadas as transformações simpléticas definidas em $V$.
	\end{definicao}
	
	\begin{exemplo}
		(Rotações em $\real{2}$) Foi visto no Exemplo $\ref{exemplo_espaco_simpletico_real}$ que $(\real{2}, \omega)$ é um espaço vetorial simplético, onde $\omega = dx\wedge dy$ e a base canônica $\{e_{x}, e_{y}\}$ é uma base simplética. Seja $R(\theta):\real{2}\to \real{2}$ uma rotação de um ângulo $\theta$, isto é, dado $v= v_{x}e_{x}+v_{y}e_{y} \in V$, tem-se $R(\theta)v = (v_{x}\cos(\theta)-v_{y}\sin(\theta))e_{x}+(v_{x}\sin(\theta)+v_{y}\cos(\theta))e_{y} = v'_{x}e_{x}+v'_{y}e_{y}$. Então
		$$
		\begin{aligned}
			\formaSimpletica{R(\theta)u}{R(\theta)v}&=
			\formaSimpletica{u'_{x}e_{x}+u'_{y}e_{y}}{v'_{x}e_{x}+v'_{y}e_{y}}
			\\
			&=u'_{x}v'_{y}\formaSimpletica{e_{x}}{e_{y}}	+u'_{y}v'_{x}\formaSimpletica{e_{y}}{e_{x}}
			\\
			&=(u_{x}\cos(\theta)-u_{y}\sin(\theta))(v_{x}\sin(\theta)+v_{y}\cos(\theta))
			\\
			&\;\;\;\;\;- (u_{x}\sin(\theta)+u_{y}\cos(\theta))(v_{x}\cos(\theta)-v_{y}\sin(\theta))
			\\
			&=u_{x}v_{y}-u_{y}v_{x}
			\\
			&=\formaSimpletica{u_{x}e_{x}}{v_{y}e_{y}} - \formaSimpletica{u_{y}e_{y}}{v_{x}e_{x}}
			\\
			&=\formaSimpletica{u}{v}.
		\end{aligned}
		$$
		Portanto, $R(\theta)$ é uma transformação simplética para todo $\theta \in \reta$.
	\end{exemplo}
	
	\section{$\estruturascomplexaspadrao$ e sua topologia}
	
	\begin{definicao}\label{definicao_estrutura_complexa}
		(Estrutura complexa) Uma estrutura complexa em um espaço vetorial $V$ é um endomorfismo linear $J: V \to V$, onde $J^{2} = -Id$. Diz-se que uma estrutura complexa em $V$ é compatível com a forma simplética (ou $\omega$-compatível) se $g(u,v):=\omega(u, Jv)$ define um produto interno em $V$. O conjunto de todas as estruturas complexas em $V$ que sejam $\omega$-compatíveis é denotado por $\estruturascomplexaspadrao$.
	\end{definicao}
	
	\begin{observacao}\label{observacao_estrutura_complexa}
		Será fixada notação
		$$
		\estruturacomplexa=
		\left(
		\begin{array}{cc}
		0 & -Id
		\\
		Id & 0
		\end{array}
		\right).
		$$
	\end{observacao}
	
	Será demonstrado que todo produto interno em $V$ pode ser associado a um elemento de $\estruturascomplexaspadrao$ e vice-versa, sendo que dessa associação serão retiradas informações sobre sua topologia.
	
	\begin{observacao}\label{observacao_conjunto_estrutura_complexa}
		Ao se tratar $\estruturascomplexaspadrao$ como espaço topológico será adotada sua topologia induzida pela topologia de $\generalgroupreal{V}$ gerada pela norma da convergência uniforme (ou norma do sup).
	\end{observacao}
	
	
	Seja $V$ um n-espaço vetorial real munido de um produto interno positivo-definido. O conjunto dos produtos internos positivos-definidos em $V$ é denotado por $\produtosinternos{V}$.
	
	Seja $\mathcal{L}(V \times V; \real{})$ o conjunto das aplicações bilineares de $V\times V$ em $\reta$. Por definição $\produtosinternos{V} \subseteq \mathcal{L}(V \times V; \real{})$. Com isso, pode-se munir $\produtosinternos{V}$ com a topologia induzida por $\mathcal{L}(V \times V; \real{})$.
	
	\begin{lema}\label{lema_contratibilidade_produtos_internos}
		Seja $V$ um n-espaço vetorial com produto interno positivo-definido. Então $\produtosinternos{V}$ é contrátil.
	\end{lema}
	\begin{prova}
		Sejam $d_{1},d_{2} \in \produtosinternos{V}$ e $d:\intervalo\to \produtosinternos{V}$ tal que $d(s) = (1-s)d_{1}+ s d_{2}$. Afirmo que $d(s)$ é um produto interno. De fato, é bilinear e simétrico, pois é uma combinação linear de aplicações bilineares e simétricas. Resta mostrar que $d(s)$ é positivo-definido. Supondo que $v =0$ tem-se $d(s)(v,v)=(1-s)d_{1}(v,v)+ s d_{2}(v,v)  =0$, pois $d_{1}, d_{2}$ são positivos-definidos. Suponha que $v\neq 0$ e $d(s)(v,v) = 0$. Então $0=(1-s)d_{1}(v,v)+ s d_{2}(v,v)  $ o que é um absurdo pois $(1-s)d_{1}(v,v)>0 $ e $sd_{2}(v,v)> 0$. Logo $d(s)$ é positiva-definida para todo $s \in \intervalo$. Como $d_{1}, d_{2} \in \produtosinternos{V}$ são arbitrários, então $\produtosinternos{V}$ é convexo, logo é contrátil.
	\end{prova}
	
 	Dados $(V, \omega)$ n-espaço vetorial simplético e uma estrutura complexa $J$ $\omega$-compatível, tem-se que $\formaSimpletica{v}{Ju} = g_{J}(v,u)$ é um produto interno em $V$. Com isso, a aplicação $G:\estruturascomplexaspadrao \to \produtosinternos{V}$ onde $G(J)(u,v) = \omega(u,Jv) = g_{J}(u,v)$ está bem definida e é injetora. A seguinte proposição afirma que vale a recíproca desse resultado.
	
	\begin{proposicao}
		Seja $(V, \omega)$ uma 2n-espaço vetorial simplético. Então cada produto interno $g \in \produtosinternos{V}$ define uma estrutura complexa $\omega$-compatível.
	\end{proposicao}
	\begin{prova}
		Seja $g \in \produtosinternos{V}$. Sabe-se que $\hat{g}:V \to V^{*}$, definido por $\hat{g}(v)(u)=g(v,u)$, é um isomorfismo. Segue que, existe um único automorfismo $A:V\to V$ tal que $\formaSimpletica{v}{u} = g(Av,u)$ para todo $v,u \in V$ dado por $A = \hat{g}^{-1}\omega
		^{*}$. Note que $g(Av,u) = \formaSimpletica{v}{u} = -\formaSimpletica{u}{v} = -g(Au,v) = g(-A^{t}v,u)$, portanto $A$ é anti-simétrico. Pelo Corolário $\ref{corolario_decomposicao_matriz_antisimetrica}$, pode-se escrever $A=PJ$, onde $P = (-A^{2})^{1/2}$ é positiva-definida, $J$ é ortogonal tal que $J^{2} = -Id$ e $J^{t} = J^{-1}$. Com isso, tem-se $\formaSimpletica{v}{Ju} = g(Av, Ju) = g(J^{t}Av, u)  = g(J^{-1}Av, u) = g(Pv, u) = g_{P}(v,u)$, o que é um produto interno positivo-definido pois $P$ é positivo-definido. Portanto, $J$ é $\omega$-compatível. Assim, tem-se a aplicação injetora $F: \produtosinternos{V} \to \estruturascomplexas{V}{\omega}$ definida por $F(g) = J$.
	\end{prova}
	
	\begin{proposicao}\label{proposicao_J_contratil}
		$\estruturascomplexaspadrao$ é homeomorfo a $\produtosinternos{V}$, logo é contrátil.
	\end{proposicao}
	\begin{prova}
		Sejam $F: \produtosinternos{V} \to \estruturascomplexaspadrao$ e $G:\estruturascomplexaspadrao \to \produtosinternos{V}$ dos resultados anteriores. Seja $J \in \estruturascomplexaspadrao$ arbitrário. Então $(F\circ G)(J) = F(g_{J}) =J$, logo $F\circ G = Id_{\estruturascomplexaspadrao}$. Por outro lado, $(G\circ F)(g_{J}) = G(J) = g_{J}$, logo $G\circ F = Id_{\produtosinternos{V}}$. Portanto $F = G^{-1}$. Munindo $\estruturascomplexaspadrao$ com a topologia induzida por $\generalgroupreal{2n}$, então pode-se mostrar que $F$ é contínua. Analogamente, munindo $\produtosinternos{V}$ com a topologia induzida do espaço vetorial de todas as forma simétricas definida em $V$, pode-se mostrar que $G$ é contínua. Portanto $F$ é um homeomorfismo. Pelo Lema $\ref{lema_contratibilidade_produtos_internos}$, tem-se que $\produtosinternos{V}$ é contrátil, e como a contratibilidade é preservada por homeomorfismos, então $\estruturascomplexaspadrao$ é contrátil.
	\end{prova}
	
	\chapter{O Grupo Simplético $\gruposimpletico{2n}$}\label{capitulo_grupo_simpletico}
	Neste capítulo o grupo simplético será introduzido e será apresentado um estudo aprofundado das suas propriedades. O grupo simplético e suas características topológicas exercem papel central na construção dos índices de Maslov no Capítulo \ref{capitulo_indice_maslov}. Será demonstrado que o grupo fundamental de $Sp(2n)$ é isomorfo aos inteiros e tal isomorfismo será dado pelo índice de Maslov.
	
	O grupo simplético $\gruposimpletico{\real{2n}}$ será denotado por $\gruposimpletico{2n}$.
	
	\begin{proposicao}\label{proposicao_grupo_simpletico_estrutura_grupo}
		$\gruposimpletico{2n}$ é um grupo com a operação de multiplicação de matrizes.
	\end{proposicao}
	\begin{prova}
		Sejam $A,B \in \gruposimpletico{2n}$ e $u,v \in \real{2n}$, então
		\begin{enumerate}
			\item \textit{(Operação fechada)} $\omega(ABu, ABv) = \omega(Bu, Bv) = \omega(u,v)$, logo $AB \in \gruposimpletico{2n}$.
			
			\item \textit{(Associatividade)} $\gruposimpletico{2n}$ é associativo pois a operação de multiplicação de matrizes reais é associativa.
			
			\item \textit{(Elemento neutro)} O elementro neutro de $\gruposimpletico{2n}$ é a identidade.
			
			\item \textit{(Elemento inverso)} Se $A \in \gruposimpletico{2n}$, então $\omega(u, v)=\omega(AA^{-1}u, AA^{-1}v) = \omega(A^{-1}u, A^{-1}v)$, logo $A^{-1} \in \gruposimpletico{2n}$. 
		\end{enumerate}
		Portanto $\gruposimpletico{2n}$ é um grupo.
	\end{prova}
	
	A topologia considerada em $\gruposimpletico{2n}$ é a induzida pela topologia de $\generalgroupreal{2n}$ gerada pela norma da convergência uniforme.
	
	O seguinte resultado é uma caracterização do grupo simplético e sua relação com a estrutura complexa:
	
	\begin{proposicao}\label{proposicao_caracterizacao_Sp2n}
		(Caracterização de $Sp(2n)$) Se $(V, \omega)$ é um 2n-espaço vetorial simplético e $J \in \estruturascomplexaspadrao$ uma estrutura complexa $\omega$-compatível, então $A\in Sp(2n)$ se, e somente se, $A^{t}JA = J$. Além disso, pode-se escrever 
		$$
		A=
		\left(
		\begin{array}{cc}
		B & C
		\\
		D & E
		\end{array}
		\right)
		$$
		onde $B^{t}D, C^{t}E, BC^{t}, DE^{t} $ são matrizes simétricas e $B^{t}E - D^{t}C = Id$ e $BE^{t} - CD^{t} = Id$.
	\end{proposicao}
	\begin{prova}
		Suponha $A \in Sp(2n)$, então:
		$$
		\omega(Av, Jw)= g(Av,w) = g(v,A^{t}w) = \omega(v, JA^{t}w) = \omega(Av, AJA^{t}w).
		$$
		Como a igualdade vale para quaisquer $v,w$, então $J = AJA^{t}$. Por outro lado, suponha que $A \in \generalgroupreal{2n}$, tal que $A^{t}JA=J$ para $J \in \estruturascomplexaspadrao$, então:
		$$
		\begin{aligned}
		\omega(Av, Aw) &= \omega(Av, -J^{2}Aw)
		\\
		&=g(Av, -JAw) 
		\\
		&= g(v, -A^{t}JAw) 
		\\
		&= g(v, -Jw) 
		\\
		&= \omega(v, -J^{2}w) 
		\\
		&= \omega(v, w), 
		\end{aligned}
		$$
		logo $A \in \gruposimpletico{2n}$. Seja $J \in \estruturascomplexaspadrao$ e $\{e, f\}$ uma base simplética de $V$ tal que a matriz $J$ com relação a essa base é $\estruturacomplexa$. A equação $A^{t}JA=J$ nos fornece as relações entre os blocos de matrizes de $A$:
		
		$$
		\begin{aligned}
		J &= A^{t}JA
		\\
		\left(
		\begin{array}{cc}
		0 & -Id
		\\
		Id & 0
		\end{array}
		\right)
		&=
		\left(
		\begin{array}{cc}
		B^{t} & D^{t}
		\\
		C^{t} & E^{t}
		\end{array}
		\right)
		\left(
		\begin{array}{cc}
		-D & -E
		\\
		B & C
		\end{array}
		\right)
		\\
		&=
		\left(
		\begin{array}{cc}
		-B^{t}D +D^{t}B & -B^{t}E+D^{t}C
		\\
		-C^{t}D+E^{t}B & -C^{t}E+E^{t}C
		\end{array}
		\right),
		\end{aligned}
		$$
		portanto $B^{t}D = D^{t}B = (B^{t}D)^{t}$ (matriz simétrica) e $D^{t}C-B^{t}E = Id$. De forma análoga, obtêm-se as outras identidades.
	\end{prova}
	
	Dada uma matriz $A \in \generalgroupcomplexo{n}$, pode-se escrever $A = B+iC$ onde $B,C \in \generalgroupreal{n}$. Com isso, considere a aplicação $F:\generalgroupcomplexo{n} \to \generalgroupreal{2n}$ tal que 
	$$
	F(A)=
	\left(
	\begin{array}{cc}
	B & -C
	\\
	C & B
	\end{array}
	\right).
	$$
	Note que $F(A) = 0$ se, e somente se, $A=0$, logo $F$ é injetor. Pode-se verificar que $F(AB)=F(A)F(B)$, portanto é um monomorfismo. Além disso, tem-se a propriedade $F(A^{*}) = F(B^{t} - iC^{t}) = F(A)^{t}$. A aplicação $F$ é contínua pois dados $A=B+iC \in \generalgroupcomplexo{n}$ e $\epsilon > 0$, então para todo $X= Y+iZ \in \generalgroupcomplexo{n}$ tal que $||A - X||=max \{|B_{ij} - Y_{ij}|,  |C_{ij} - Z_{ij}|\} < \epsilon/2$ se tem 
	$$
	||F(A) - F(X)|| = max \{|B_{ij} - Y_{ij}|, |C_{ij} - Z_{ij}| \}< \epsilon/2 < \epsilon.
	$$
	
	Sejam $\matrizunitaria{n} = \{A\in \generalgroupcomplexo{n}: AA^{*}=Id \}$ o subgrupo das matrizes unitárias e $\matrizortogonal{n} = \{A \in \generalgroupreal{n}: AA^{t}  =Id \}$ o subgrupo das matrizes ortogonais. Então vale a seguinte proposição:
	
	\begin{proposicao}\label{proposicao_isomorfismo_U}
		Seja $F$ a aplicação contínua definida anteriormente. Então a restrição $F|_{\matrizunitaria{n}}: \matrizunitaria{n} \to \matrizSimpleticaOrtogonal $, onde $\matrizSimpleticaOrtogonal  = \gruposimpletico{2n}\cap \matrizortogonal{2n}$, é um isomorfismo. Além disso, dado $A \in \matrizSimpleticaOrtogonal $ tem-se $A\estruturacomplexa=\estruturacomplexa A$.
	\end{proposicao}
	\begin{prova}
		Note que $\matrizSimpleticaOrtogonal  = \gruposimpletico{2n} \cap \matrizortogonal{2n}$ é não-vazio, pois a identidade está na intersecção. Tomando $A \in \matrizSimpleticaOrtogonal $, então $A^{t}A= Id$. Com isso,
		$$
		\begin{aligned}
		Id=A^{t}A &=
		\left(
		\begin{array}{cc}
		B^{t} & D^{t}
		\\
		C^{t} & E^{t}
		\end{array}
		\right)
		\left(
		\begin{array}{cc}
		B & C
		\\
		D & E
		\end{array}
		\right)
		\\
		&= 
		\left(
		\begin{array}{cc}
		B^{t}B + D^{t}D & B^{t}C + D^{t}E 
		\\
		C^{t}B + E^{t}D  & CC^{t}+EE^{t}
		\end{array}
		\right)
		\\
		&=
		\left(
		\begin{array}{cc}
		B^{t}B + D^{t}D & 0 
		\\
		0 & CC^{t}+EE^{t}
		\end{array}
		\right).
		\end{aligned}
		$$
		Tal condição é satisfeita quando $C^{t}B =- E^{t}D$, $B^{t}C =- D^{t}E$ e $BB^{t} + DD^{t} = CC^{t}+EE^{t} = Id$. 
		
		Dada $Z \in \matrizunitaria{n}$, tem-se $Id=F(Id) = F(Z^{*}Z) = F(Z^{*})F(Z) = F(Z)^{t}F(Z)$, portanto $F(Z) \in \matrizortogonal{2n}$. Se $Z= B+iC$, então
		$$
		F(Z)=
		\left(
		\begin{array}{cc}
		B & -C
		\\
		C & B
		\end{array}
		\right)
		$$
		e
		$$
		F(Z^{*})F(Z)=
		\left(
		\begin{array}{cc}
		BB^{t} +CC^{t} & -B^{t}C +C^{t}B
		\\
		B^{t}C -C^{t}B & BB^{t} +CC^{t}
		\end{array}
		\right)	
		= Id
		$$
		o que implica que $B^{t}C =C^{t}B$. Pela Proposição $\ref{proposicao_caracterizacao_Sp2n}$, $F(Z) \in \gruposimpletico{2n}$. Portanto $F(Z) \in \matrizSimpleticaOrtogonal $, o que implica que $F(\matrizunitaria{n}) \subseteq \matrizSimpleticaOrtogonal$.
		
		Finalmente, será demonstrada a inclusão $\matrizSimpleticaOrtogonal \subseteq F(\matrizunitaria{n})$. 
		
		Se $A \in \matrizSimpleticaOrtogonal$, então $A^{t}=A^{-1}$ e $\estruturacomplexa=A^{t}\estruturacomplexa  A=A^{-1}\estruturacomplexa A$, o que implica que $A\estruturacomplexa=\estruturacomplexa A$. Assim:
		$$
		\begin{aligned}
		A\estruturacomplexa&=\estruturacomplexa A
		\\
		\left(
		\begin{array}{cc}
		B & C
		\\
		D & E
		\end{array}
		\right)
		\left(
		\begin{array}{cc}
		0 & -Id
		\\
		Id & 0
		\end{array}
		\right)
		&=
		\left(
		\begin{array}{cc}
		0 & -Id
		\\
		Id & 0
		\end{array}
		\right)
		\left(
		\begin{array}{cc}
		B & C
		\\
		D & E
		\end{array}
		\right)
		\\
		\left(
		\begin{array}{cc}
		C & -B
		\\
		E & -D
		\end{array}
		\right)
		&=
		\left(
		\begin{array}{cc}
		-D & -E
		\\
		B & C
		\end{array}
		\right), 
		\end{aligned}
		$$
		logo $C=-D$ e $B=E$, portanto:
		$$
		A=\left(
		\begin{array}{cc}
		B & -C
		\\
		C & B
		\end{array}
		\right).
		$$
		
		Tomando $Z = B+iC \in \matrizunitaria{n}$ tem-se que $F(Z) = A$, o que implica que $\matrizSimpleticaOrtogonal \subseteq F(\matrizunitaria{n})$. Conclusão, $\matrizSimpleticaOrtogonal = F(\matrizunitaria{n})$. 
		
		Como $F$ é monomorfismo, então $F|_{\matrizunitaria{n}}:\matrizunitaria{n} \to \matrizSimpleticaOrtogonal$ é sobrejetora sobre sua imagem, portanto é um isomorfismo.
	\end{prova}
	
	Será analizado agora o espectro $\espectrooperador{A}$ de um simplectomorfismo $A \in \gruposimpletico{2n}$ em um determinado espaço vetorial. Tal resultado será usado na demonstração da contratibilidade do quociente $\gruposimpletico{2n}/\matrizSimpleticaOrtogonal$, que por sua vez será utilizado na construção do índice de Maslov.
	
	\begin{proposicao}\label{proposicao_caracterizacao_espectro_semelhante}
		Seja $A \in \gruposimpletico{2n}$. Então, $A, A^{-1}, A^{t}$ são semelhantes. Com isso, $\sigma(A) = \sigma(A^{-1}) = \sigma(A^{t}) $.
	\end{proposicao}
	\begin{prova}
		Tem-se que $A^{t}\estruturacomplexa A = \estruturacomplexa$. Supondo que $A$ é invertível, então $A^{t} = \estruturacomplexa A^{-1} \estruturacomplexa^{-1}$. Logo $A^{t}$ é semelhante a $A^{-1}$. Sabe-se que $A$ é semelhante a $A^{t}$, logo $A$ é semelhante a $A^{-1}$. Como matrizes semelhantes possuem o mesmo polinômio característico, então $\sigma(A) = \sigma(A^{-1}) = \sigma(A^{t}) $.
	\end{prova}
	
	\begin{observacao}
		$\lambda \in \sigma(A)$ se , e somente se, $\lambda^{-1}\in \sigma(A)$. Além disso, pode-se mostrar que $\lambda$,  $\bar{\lambda}$, $\lambda^{-1}$ e $\bar{\lambda}^{-1}$ possuem as mesmas multiplicidades. 
	\end{observacao}
	
	Para estudar os auto-valores de $A \in \gruposimpleticoreal{2n}$ será realizada a complexificação do espaço vetorial simplético $(V,\omega)$ e de seu grupo de transformações simpléticas $\gruposimpleticoreal{2n}$.	Essas complexificações são denotadas por $(\complexificado{V}, \Omega)$ e $\gruposimpleticocomplexo{2n}$. Mais detalhes podem ser encontrados no Apêndice \ref{apendice_complexificacao_espacos_vetoriais}.
	
	\begin{proposicao}\label{proposicao_auto_espaco_grupo_simpletico}
		(Auto-espaços de $\gruposimpleticocomplexo{2n}$) Sejam $(\complexificado{V}, \Omega)$ um 2n-espaço vetorial simplético, $A \in \gruposimpletico{2n}$ e $r,s \in \inteiros$ tais que $r\geq 1$ e $s\geq 1$. Se $\lambda, \mu \in \sigma(A)$ tais que $\lambda\mu \neq 1$, então seus auto-espaços generalizados $E_{\lambda}=Ker(A-\lambda Id)^{r}$ e  $E_{\mu}=Ker(A-\mu Id)^{s}$ são $\omega$-ortogonais, isto é, $\omega(E_{\lambda}, E_{\mu}) = 0$.
	\end{proposicao}
	\begin{prova}
		Seja $P(r,s)$ a propriedade que se quer mostrar. Note que $P(1,1)$ é verdadeira. De fato, Dados $v\in E_{\lambda}$ e $u\in E_{\mu}$ tem-se que $\formaSimpletica{v}{u} = \formaSimpletica{Av}{Au} = \lambda\mu\formaSimpletica{v}{u}$, o que implica que $\formaSimpletica{v}{u} = 0$, pois $\lambda\mu\neq 1$. Por indução será mostrado que $P(r,s)$ é verdadeira para quaisquer $r,s$. Para aplicar a hipótese de indução no índice $s$, assuma que $P(1,s)$ é verdadeira e que $u \in Ker(A-\mu Id)^{s+1}$. Com isso, $0=(A-\mu Id)^{s+1}u = (A-\mu Id)^{s}(A-\mu Id)u $, logo $(A-\mu Id)u \in Ker(A-\mu Id)^{s}$, o que implica em
		$$
		\begin{aligned}
		\formaSimpletica{v}{u}
		&=\formaSimpletica{Av}{Au}
		\\
		&= \formaSimpletica{Av}{Au -\mu u +\mu u} 
		\\
		&= \lambda\underbrace{\formaSimpletica{v}{(A-\mu Id)u}}_{=0}+\lambda\mu\formaSimpletica{v}{u}
		\\
		&=\lambda\mu\formaSimpletica{v}{u}.
		\end{aligned}
		$$
		Como $\lambda\mu \neq 1$, então $\formaSimpletica{v}{u}=0$ e $P(1, s+1)$ é verdadeira.
		
		Conclui-se que $P(1, s)$ é verdadeira para todo $s\geq 1$. Analogamente, $P(r,1)$ é verdadeira para todo $r\geq 1$. Para mostrar que $P(r,s)$ é verdadeira por indução, basta assumir que $P(r,s+1)$ e $P(r+1,s)$ são verdadeiras e, analogamente mostra-se que $P(r+1,s+1)$ é verdadeira. Portanto, $\omega(E_{\lambda}, E_{\mu}) = 0$ para quaisquer inteiros $r\geq 1$ e $s\geq 1$.
		
	\end{prova}
	
	\begin{corolario}\label{corolario_restricao_forma_simpletica}
		Sejam $A \in \gruposimpletico{2n}$ e $\lambda, \mu \in \sigma(A)$. 
		\begin{enumerate}
			\item As restrições $\omega|_{E_{\pm 1} \times E_{\pm 1}}$ são não-degeneradas. Além disso, suas multiplicidades $m(\pm 1)$ são pares.
			
			\item Se $\lambda \neq \pm 1$, então a restrição $\omega|_{W_{\lambda} \times W_{\lambda}}$ é não-degenerada, onde $W_{\lambda } = E_{\lambda} \oplus E_{\lambda^{-1}}$.
		\end{enumerate}
	\end{corolario}
	\begin{prova}
		\begin{enumerate}
			\item  Será feito primeiro o caso em que $\lambda = \mu =1$. Suponha que $\omega|_{E_{1}\times E_{1}}$ seja degenerada. Então existe $v\neq 0 \in E_{1}$ tal que $\omega(v, u) = 0$ para todo $u \in E_{1}$. Seja $\beta \in \complementar{\sigma(A)}{\{1 \}}$. Pela Proposição $\ref{proposicao_auto_espaco_grupo_simpletico}$ se tem que $\formaSimpletica{E_{1}}{E_{\beta}} = 0$. Pelo Teorema $\ref{teorema_espectral_jordan}$ é possível escrever $V = E_{1}  \bigoplus_{\lambda \in \complementar{\sigma(A)}{ \{1\}  }}E_{\lambda}$. Com isso, $\formaSimpletica{v}{V} = 0$. Logo $\omega$ é degenerada, o que contradiz a hipótese. Portanto, $\omega|_{E_{1} \times E_{1}}$ é não-degenerada. Além disso, como $\lambda = \lambda^{-1}$, então a multiplicidade $m(1)$ é par. Com um argumento análogo $\omega|_{E_{-1} \times E_{-1}}$ é não-degenerada e $m(-1)$ é par. 
			
			\item Pelo Teorema $\ref{teorema_espectral_jordan}$ pode-se escrever $V = E_{1}  \bigoplus E_{-1}  \bigoplus_{\lambda \in \complementar{\sigma(A)}{ \{\pm 1\}  }}E_{\lambda}$. Supondo que $\omega|_{E\times E}$ seja degenerada, onde $E= E_{\lambda} \oplus E_{\lambda^{-1}}$, então existe $v \neq 0 \in E$ tal que 
			$\omega(v,a)=0$ para todo $a\in E$. Além disso, pela $\omega$-ortogonalidade,  $\omega(v,E_{\beta})=0$, para todo $\beta\in \complementar{\sigma(A)}{\{1\} }$. Segue do Teorema $\ref{teorema_espectral_jordan}$ analogamente ao item anterior que $\omega(v,V)=0$, o que contradiz a hipótese de $\omega$ ser não degenerada.
		\end{enumerate}
	\end{prova}
	
	Considere $\matrizsimetricapositiva{2n} \subset \matrizquadreal{2n}$ como sendo o conjunto de todas as matrizes positivas-definidas (veja a Definição $\ref{definicao_matriz_positiva_definida}$).
	
	\begin{proposicao}\label{proposicao_potenciacao_grupo_simpletico}
		(Potênciação em $\gruposimpletico{2n}$) Seja $\gruposimpleticopositivo{2n} = \gruposimpletico{2n} \cap \matrizsimetricapositiva{2n}$ o conjunto das matrizes simpléticas simétricas e positivas-definidas. Dado $A \in \gruposimpleticopositivo{2n}$, então $A^{\alpha} \in \gruposimpletico{2n}$ para qualquer $\alpha \in \real{}$.
	\end{proposicao}
	\begin{prova}
		Se $A \in \gruposimpleticopositivo{2n}$, então $A$ é simétrica, logo é normal, e pelo Lema $\ref{lema_caracterizacao_matriz_normal}$ é diagonalizável. Além disso, pela Observação $\ref{observacao_matriz_positiva_definida}$ seus auto-valores são todos positivos. Pode-se decompor $V$ na soma direta de seus auto-espaços $E_{\lambda}$, onde $\lambda \in \espectrooperador{A}$. Se $u \in E_{\lambda_{u}}$ e $v \in V_{\lambda_{v}}$, então
		$$
		\omega(A^{\alpha}u,A^{\alpha}v) = 		(\lambda_{u}\lambda_{v})^{\alpha}\omega(u,v).
		$$
		Se $\lambda_{u}\lambda_{v}\neq 1$, então $\omega(u,v)=0$, o que implica que $\omega(A^{\alpha}u,A^{\alpha}v)=(\lambda_{u}\lambda_{v})^{\alpha}\omega(u,v)=0$, logo $\omega(A^{\alpha}u,A^{\alpha}v) = \omega(u,v)=0$, portanto $A^{\alpha} \in \gruposimpletico{2n}$. Caso $\lambda_{u}\lambda_{v}=1$ tem-se $\omega(A^{\alpha}u,A^{\alpha}v) = \omega(u,v)$, portanto $A^{\alpha} \in \gruposimpletico{2n}$.
	\end{prova}
	
	\begin{observacao}\label{observacao_determinante_matriz_unitaria}
		Note que, dado $A \in \matrizunitaria{n}$, tem-se $1= \det(AA^{*}) = \det(A)\det(A^{*}) = \det(A)\overline{\det(A)} = ||\det(A)||^{2}$, portanto $\det(\matrizunitaria{n}) \subseteq S^{1}$.
	\end{observacao}
	
	
	\begin{lema}\label{lema_conexidade_matriz_unitaria}
		$\matrizSimpleticaOrtogonal$ é conexo por caminhos, logo é conexo.
	\end{lema}
	\begin{prova}
		Seja $A \in \matrizunitaria{n}$. Então $A$ é diagonalizável, e $\det(A) \in S^{1}$, e pela Observação $\ref{observacao_caracterizacao_matriz_normal}$ existe uma matriz unitária $U$ tal que $A=U^{*}diag\{e^{i\theta_{1}}, \dots, e^{i\theta_{n}}\}U$, com isso tem-se $\det(A) = e^{i(\theta_{1}+\dots+\theta_{n})} \in \circulo$. Definindo o caminho contínuo $\gamma:[0,1] \to \matrizunitaria{n}$ tal que $\gamma(\lambda)=U^{*}diag\{e^{i\theta_{1}\lambda}, \dots, e^{i\theta_{n}\lambda}\}U$, tem-se $\gamma(0)=Id$ e $\gamma(1)=A$. Note que $\gamma$ está bem-definida pois $D=diag\{e^{i\theta_{1}\lambda}, \dots, e^{i\theta_{n}\lambda}\}$ é tal que $D^{*}D = Id$, logo $D \in \matrizunitaria{n}$. Além disso, $U^{*}, U \in \matrizunitaria{n}$, por construção. Então $\gamma(\lambda) \in \matrizunitaria{n}$. Note que $\det(\gamma(\lambda)) = e^{i\lambda(\theta_{1}+\dots+\theta_{n})} \in \circulo$, logo $\gamma([0,1]) \subset \matrizunitaria{n}$. Com isso, toda $A \in \matrizunitaria{n}$ pode ser conectada a $Id$ por um caminho contínuo, logo $\matrizunitaria{n}$ é conexo por caminhos, portanto é conexo. Pela Proposição $\ref{proposicao_isomorfismo_U}$, tem-se que $F|_{\matrizunitaria{n}}:\matrizunitaria{n} \to \matrizSimpleticaOrtogonal$ é um isomorfismo contínuo, logo é um homeomorfimo. Como $\matrizunitaria{n}$ é conexo e a conexidade é preservada por aplicações contínuas, então $\matrizSimpleticaOrtogonal = F(\matrizunitaria{n})$ é conexo por caminhos, logo é conexo.
	\end{prova}
	
	\begin{proposicao}
		$\gruposimpleticopositivo{2n}$ é conexo por caminhos, logo é conexo.
	\end{proposicao}
	\begin{prova}
		Seja a aplicação contínua $\gamma:\gruposimpleticopositivo{2n}\times [0,1] \to \gruposimpletico{2n}$ tal que $\gamma(A,\lambda) = A^{\lambda}$. Pela Proposição $\ref{proposicao_potenciacao_grupo_simpletico}$, a aplicação $\gamma$ está bem-definida e é contínua. Fixando $A \in \gruposimpleticopositivo{2n}$, 	a curva $\gamma_{A}:[0,1]\to \gruposimpletico{2n}$ tal que $\gamma_{A}(0) = Id$ e $\gamma_{A}(1) = A$, é um caminho contínuo que conecta a identidade a matriz $A$. Como $A=A^{t}$, segue da Proposição $\ref{proposicao_potenciacao_grupo_simpletico}$ que $\gamma_{A}(\lambda)^{t} = (A^{\lambda})^{t} = (A^{t})^{\lambda} = (A)^{\lambda} = \gamma_{A}(\lambda)$. Além disso, os k-subdeterminantes $det_{k}(\gamma_{A}(\lambda)) = det_{k}(A^{\lambda}) > 0$ para $1\leq k \leq 2n$. Pelo fato de que $\gamma_{A}(\lambda)$ é simétrica e pelo Teorema $\ref{teorema_matriz_positiva_definida}$ tem-se que $\gamma_{A}(\lambda)$ é positiva-definida, logo $\gamma_{A}([0,1]) \subset \gruposimpleticopositivo{2n}$. Como $A \in \gruposimpleticopositivo{2n}$ é arbitrária, então a construção anterior vale para quaisquer elementos de $\gruposimpleticopositivo{2n}$. Com isso, dados $A, B \in \gruposimpleticopositivo{2n}$, é possível conectar $A$ a $B$ por uma curva contínua que passa pela identidade, portanto $\gruposimpleticopositivo{2n}$ é conexo por caminhos, logo é conexo.
	\end{prova}	
	
	\begin{lema}\label{lema_decomposicao_grupo_simpletico_positivo}
		Se $A \in \gruposimpletico{2n}$, então existem únicas $P \in \gruposimpleticopositivo{2n}$ e $O \in \matrizSimpleticaOrtogonal$ tais que $A=PO$.
	\end{lema}
	\begin{prova}
		Segue do Teorema $\ref{teorema_decomposicao_polar}$ que $A=PO$, onde $P=AA^t$ é positiva definida e O é uma matriz ortogonal. Como $A$ e $A^{t} $ são matrizes simpléticas, então $P=AA^{t}\in \gruposimpletico{2n}$. 
		Além disso, como $\gruposimpletico{2n}$ é grupo então $P^{-1}\in \gruposimpletico{2n}$, logo $O=P^{-1}A \in \mathcal{U}$.
	\end{prova}
	
	\begin{teorema}\label{teoerma_sp2n_conexo}
		$\gruposimpletico{2n}$ é conexo por caminhos, logo é conexo.
	\end{teorema}
	\begin{prova}
		Se $A \in \gruposimpletico{2n}$, então pelo Lema $\ref{lema_decomposicao_grupo_simpletico_positivo}$ é possível escrever $A=PO$ onde $P \in \gruposimpleticopositivo{2n}$ e $O\in \matrizSimpleticaOrtogonal$ são únicas. Pela unicidade da decomposição anterior a aplicação $G: \gruposimpletico{2n} \to \gruposimpleticopositivo{2n} \times \matrizSimpleticaOrtogonal$ definida por $G(A) = (P,O)$ é injetora. Por outro lado, dado $(P,O) \in \gruposimpleticopositivo{2n} \times \matrizSimpleticaOrtogonal$ tem-se $(PO)^{t}\estruturacomplexa PO = O^{t}P^{t}\estruturacomplexa PO = O^{t}\estruturacomplexa O = \estruturacomplexa$, logo pela Proposição $\ref{proposicao_caracterizacao_Sp2n}$, tem-se $PO \in \gruposimpletico{2n}$ e $G$ é sobrejetora. De fato, definindo $A=PO \in \gruposimpletico{2n}$ tem-se que $G(A) = (P,O)$. Portanto $G$ é bijetora. 
		
		Denotando por $G^{-1}:\gruposimpleticopositivo{2n} \times \matrizSimpleticaOrtogonal\to \gruposimpletico{2n}$ a inversa de $G$, tem-se que $G^{-1}(P,O) = PO$, o que implica que $G^{-1}$ é contínua, pois o produto de matrizes é uma operação contínua.
		
		Afirmo que $\gruposimpleticopositivo{2n}\times \matrizSimpleticaOrtogonal$ é conexo por caminhos, pois é o produto cartesianos de espaços topológicos conexos por caminhos. Além disso, como a conexidade é preservada por aplicações contínuas, então $\gruposimpletico{2n}$ é conexo por caminhos, logo é conexo.
	\end{prova}
	
	\begin{observacao}\label{observacao_decomposicao_Sp2n}
		No teorema anterior, foi exibida uma bijeção entre $\gruposimpletico{2n} $ e o produto cartesiano $\gruposimpleticopositivo{2n} \times \matrizSimpleticaOrtogonal$, isto é, toda $A \in \gruposimpletico{2n}$ pode ser decomposta unicamente como $A=PO$, onde $P\in \gruposimpleticopositivo{2n}$ e $O \in \matrizSimpleticaOrtogonal$. Essa afirmação será utilizada na demonstração de alguns resultados adiante.
	\end{observacao}
	
	\begin{observacao}\label{observacao_conexidade_grupo_simpletico}
		Esse resultado é fundamental para a construção da homologia de Floer, pois, para definir um complexo de cadeia nessa homologia deve-se ter um homomorfismo graduado. Tal graduação será dada pelo índice de Maslov e este será relacionado ao grupo fundamental $\grupofundamental{\gruposimpletico{2n}}$. Por fim, a estratégia adotada necessita que $\grupofundamental{\gruposimpletico{2n}} \cong \inteiros$. Para mostrar esse fato a conexidade é necessária.
	\end{observacao}
	
	\begin{teorema}
		O quociente $\gruposimpletico{2n}/\matrizSimpleticaOrtogonal$ é contrátil.
	\end{teorema}
	\begin{prova}
		Na Observação $\ref{observacao_decomposicao_Sp2n}$ foi mostrado que se $A \in \gruposimpletico{2n}$ então $A=PO$, onde $P \in \gruposimpleticopositivo{2n}$ e $O \in \matrizSimpleticaOrtogonal$. Com isso tem-se $AA^{t} = POO^{t}P^{t} = PP^{t}=P^{2}$, e ela Proposição $\ref{proposicao_potenciacao_grupo_simpletico}$, é possível afirmar que $P^{\alpha} \in \gruposimpletico{2n}$ para todo $\alpha \in \real{}$. Considere a aplicação $r:\gruposimpletico{2n}\times [0,1] \to \gruposimpletico{2n}$ tal que $r(A, \alpha) = (AA^{t})^{-\alpha/2}A$. Note que é contínua pois é o produto de matrizes, que é uma operação contínua em $\gruposimpletico{2n}$. Além disso, $r$ é um retrato de deformação de $\gruposimpletico{2n}$ sobre $\matrizSimpleticaOrtogonal$ pois $r(A, 0) = A$, $r(A, 1) = (AA^{t})^{-1/2}A = P^{-1}A = O \in \matrizSimpleticaOrtogonal$ e, tomando $B \in \matrizSimpleticaOrtogonal$, $r(B, 1) = (BB^{t})^{-1/2}B = B$, pois $BB^{t} = Id$.
		
		Para simplificar a notação $ \gruposimpletico{2n}/\matrizSimpleticaOrtogonal$ será denotado por $\mathcal{S}$. Definindo a aplicação $R:\mathcal{S} \times [0,1] \to \mathcal{S}$ por $R([A], \lambda) = [r(A, \lambda)] = [(AA^{t})^{-\lambda/2}A]$, tem-se que $R$ é uma contração. De fato, a imagem de $R$ é a classe de equivalência da imagem de $r$, que é contínua, portanto $R$ é contínua. Além disso, $R([A], 0) = [A]$, $R([A], 1) = [(AA^{t})^{-1/2}A] = [P^{-1}A] = [O] = [Id]$, pois $O \in \matrizSimpleticaOrtogonal$, isto é, $R(., 0) = Id_{\mathcal{S}}(.)$ é a identidade e $R(., 1) = [Id]$ é a aplicação constante, logo é uma contração e $\mathcal{S}$ é contrátil.
	\end{prova}
	
	\begin{observacao}\label{observacao_quociente_grupo_simpletico_contratil}
		Na demonstração da contratibilidade do quociente $\gruposimpletico{2n}/\matrizSimpleticaOrtogonal$ foi mostrado que $\matrizSimpleticaOrtogonal$ é um retrato por deformação de $\gruposimpletico{2n}$.
	\end{observacao}
	
	
	\begin{teorema}
		$\grupofundamental{\gruposimpletico{2n}} \cong \inteiros$.
	\end{teorema}
	\begin{prova}
		Como $\matrizSimpleticaOrtogonal$ é um retrato de deformação de $\gruposimpletico{2n}$, então $\grupofundamental{\gruposimpletico{2n}}$ e $\grupofundamental{\matrizSimpleticaOrtogonal}$ são isomorfos. Da Proposição $\ref{proposicao_isomorfismo_U}$ segue que $\matrizSimpleticaOrtogonal\cong \matrizunitaria{n}$, e como o determinante complexo $\det_{\complexo{}}: \matrizunitaria{n} \to \circulo$ induz um isomorfismo entre os grupos fundamentais, então $\grupofundamental{\matrizunitaria{n}}\cong \inteiros$ (para mais detalhes sobre o isomorfismo induzido pela fibração $\det_{\complexo{}}$, veja \cite{hatcher} e \cite{dusa}). Portanto $\grupofundamental{\matrizSimpleticaOrtogonal} \cong \grupofundamental{\matrizunitaria{n}} \cong \inteiros$ e $\grupofundamental{\gruposimpletico{2n}} \cong \inteiros$.
	\end{prova}
	
	\begin{observacao}
		No Capítulo $\ref{capitulo_indice_maslov}$ será construída uma aplicação $\gruposimpletico{2n}\mapsto \circulo$ que induz o isomorfismo entre os grupos fundamentais.
	\end{observacao}
	
	
	\chapter{O funcional de ação $\funcionalH$ e a equação de Floer}
	\section{Variedades Simpléticas}\label{secao_variedade_simpletica}
	
	Nessa seção parte dos conceitos introduzidos no Capítulo \ref{capitulo_espacos_vetoriais_simpleticos} sobre espaços vetoriais simpléticos serão estendidos para as variedades diferenciáveis.
	
	\begin{definicao}
		(Variedade Simplética) Uma estrutura simplética em uma 2n-variedade diferenciável $M$ é uma 2-forma não-degenerada $\omega\in \Omega^{2}(M)$ tal que $d\omega=0$. O par $(M, \omega)$ é uma 2n-variedade simplética.
	\end{definicao}
	
	\begin{observacao}
		Note que, se $(M, \omega)$ é uma $2n$-variedade simplética, então para todo $x \in M$ o par $(T_{x}M, \omega_{x})$ é um $2n$-espaço vetorial simplético.
	\end{observacao}
	
	\begin{observacao}
		Aplicando o Lema \ref{lema_caracterizacao_forma_simpetica}, pode-se mostrar que, se $(M, \omega)$ é uma $2n$-variedade simplética, então $\omega^{\wedge n} \in \Omega^{2n}(M)$ define uma forma-volume em $M$. Logo, $M$ é orientável (veja \cite{warner_manifolds}).
	\end{observacao}
	
	\begin{exemplo}\label{exemplo_variedade_simpletica_esfera}
		(Variedade simplética 2-esfera) Sejam $S^{2} \subset \real{3}$ a 2-esfera centrada na origem de $\real{3}$ e $\varphi:[0,2\pi]\times [0,2\pi] \to \real{3}$ uma parametrização dada por 
		$$
		\varphi(q, p) = (\cos (p) \cos (q), \cos (p) \sin (q), \sin (p)).
		$$
		
		Com isso, o espaço tangente $T_{x}S^{2} $ é gerado por $ \{\derivadaparcialabrev{ q}, \derivadaparcialabrev{ p} \}|_{x}$, onde
		$$
		\begin{aligned}
		\derivadaparcialabrev{ q} &= -\cos (p)\sin (q) \derivadaparcialabrev{x}+ \cos (p)\cos (q) \derivadaparcialabrev{y},
		\\
		\derivadaparcialabrev{ p} &=  - \sin (p) \cos (q) 	\derivadaparcialabrev{x} - \sin (p) \sin (q) 	\derivadaparcialabrev{y} +\cos (p)				\derivadaparcialabrev{z}.
		\end{aligned} 
		$$
		Seja $\omega\in \Omega^{2}(S^{2})$ definida por $\omega_{x} = dq \wedge dp$. É imediato que $d\omega_{x} = 0$ pois as componentes de $\omega$ são constantes, $\omega$ é fechada. Além disso, suponha que $\omega$ seja degenerada. Então existe um vetor não-nulo $v = v^{q}\partial_{q}+v^{p}\partial_{p} \in T_{x}S^{2}$ tal que $\omega_{x}(v, u)=0$ para todo $u\in T_{x}S^{2}$. Desse modo, fazendo $u = \partial_{q}$ tem-se que $0=\omega_{x}(v, \partial_{q}) = v^{q}$. Logo $v^{q}=0$. Com um argumento análogo pode-se verificar que $v^{p}=0$. Portanto $v=0$, contradizendo a hipótese. Logo $\omega$ é não-degenerada. Como $S^{2}$ é uma 2-variedade diferenciável, então o par $(S^{2}, \omega)$ é uma 2-variedade simplética.
	\end{exemplo}
	
	Duas variedades simpléticas $(M_{1}, \omega_{1})$ e $(M_{2}, \omega_{2})$ são simplectomorfas se existe um difeomorfismo $\varphi: M_{1}\to M_{2}$ que preserva as formas simpléticas, isto é,
	$$
	\varphi^{*}\omega_{2}=\omega_{1}.
	$$
	
	A aplicação $\varphi$ é chamada simplectomorfismo entre $M_{1}$ e $M_{2}$.
	
	Uma estrutura complexa em $M$ é um endomorfismo $J:TM \to TM$ tal que, para todo $(x,v_{x})\in TM$, tem-se $(x, J^{2}_{x}v_{x}) = (x, -v_{x})$, isto é, $J^{2}_{x}=-Id$. Além disso, se $(M, \omega)$ é uma variedade simplética, então $J$ é $\omega$-compatível se $J_{x}$ é $\omega_{x}$-compatível para todo $x\in M$. Com isso, tem-se que $g \in \Omega^{2}(M)$, definida por
	$$
	g_{x}(X, Y) =\omega_{x}(X, J_{x}Y),
	$$
	é uma métrica Riemanniana.

	 Nem toda variedade admite uma estrutura complexa, mas toda variedade simplética $(M,\omega)$ admite uma estrutura simplética compatível com omega. O conjunto das estruturas complexas definidas em $M$ é denotado por $\estruturascomplexasM$. Seja o fibrado $(\estruturascomplexasM, M,\pi_{\mathcal{J}})$, cuja fibra em $x\in M$ é $\pi_{\mathcal{J}}^{-1}(x) = \estruturascomplexas{\espacotangenteponto{x}{M}}{\omega_{x}}$. Com isso, uma estrutura complexa $\omega$-compatível $J$ pode ser vista como uma seção $M \ni x \mapsto (x, J_{x})\in \estruturascomplexasM$. Pela Proposição \ref{proposicao_J_contratil} segue que $\estruturascomplexas{\espacotangenteponto{x}{M}}{\omega_{x}}$ é contrátil para todo $x\in M$, e com isso, $\estruturascomplexasM$ é contrátil.

	\begin{definicao}
		(Função Hamiltoniana) Seja $(M, \omega)$ uma $2n$-variedade simplética. Uma função suave $H : M \to \real{}$ é chamada uma função Hamiltoniana se satisfaz as equações diferenciais de Hamilton
		$$
		\frac{\partial q_{j}}{\partial t} = \frac{\partial H}{\partial p_{j}}, \; \frac{\partial p_{j}}{\partial t} = -\frac{\partial H}{\partial p_{j}},
		$$
		para $1\leq j \leq n$ e $(q_{1}, \dots, q_{n}, p_{1}, \dots, p_{n}) \in M$.
	\end{definicao}
	
	\begin{definicao}
		(Campo Hamiltoniano) Um campo Hamiltoniano é o único campo vetorial $X_{H}\in \campossuaves{M}$ tal que
		$$
		\omega_{x}(X_{H}(x), Y) = -dH_{x}(Y)
		$$
		para todo $x\in M$ e todo $Y \in \espacotangenteponto{x}{M}$.
	\end{definicao}
	
	\begin{observacao}
		Note que 
		$$
		dH_{x}(Y)=\omega_{x}(-X_{H}(x), Y) =\omega_{x}(-J_{x}X_{H}(x), J_{x}Y) =g_{x}(-J_{x}X_{H}(x), Y),$$
		onde foi usado o fato de que $J_{x}$ é um simplectomorfismo. Com isso, o gradiente Hamiltoniano é $\nabla H = -JX_{H} \in \campossuaves{M}$.
	\end{observacao}
	
	\begin{proposicao}
		Seja $x\in M$. Então, $x\in \pontoscriticos{H}$ se, e somente se, $\campohamiltoniano{x}=0$.
	\end{proposicao}
	\begin{prova}
		Se $x\in \pontoscriticos{H}$, então $0=dH_{x}(Y)=\omega_{x}(\campohamiltonianoabrev(x), Y)$ para todo $Y\in T_{x}M$. Logo, pela não-degeneracidade da forma simplética, tem-se que $\campohamiltoniano{x} =0$. Por outro lado, se $\campohamiltoniano{x} =0$, então $\omega_{x}(\campohamiltonianoabrev(x), Y)=0$ para todo $Y\in T_{x}M$. Logo $dH_{x}=0$ e $x\in \pontoscriticos{H}$.
	\end{prova}
	
	O sistema de equações diferenciais
	$$
		\dot{x}(t)=\campohamiltoniano{x(t)},
	$$
	é chamado sistema Hamiltoniano, e suas soluções determinam uma família de difeomorfismos $\psi_{t}:M\to M$ que satisfazem as relações 
	$$
	\derivada{\psi_{t}}{t}=\campohamiltonianoabrev\circ\psi_{t} \;\;, \;\; \psi_{0}=Id \;\;\text{e}\;\;  \psi_{t}(x(0)) = x(t).
	$$
	A aplicação $\psi: M\times \reta \to M$, dada por $\psi(x,t)=\psi_{t}(x)$, é chamada de fluxo do campo Hamiltoniano $\campohamiltonianoabrev$.
	
	A fórmula de Cartan, que associa a derivada de Lie de uma p-forma a sua derivada exterior (veja em \cite{nakahara}), é utilizada na demonstração de que o fluxo Hamiltoniano preserva a forma simplética. Sejam $M$ uma variedade diferenciável com dimensão finita, $X \in T_{x}M$, $\alpha \in \Omega^{k}(M)$ e $\phi:M\times \reta \to M$ o fluxo de $X$. A derivada de Lie de $\alpha$ com respeito ao campo vetorial $X$ é dada por
	$$
	(L_{X}\alpha)(x)=\lim\limits_{\epsilon\to 0} \frac{1}{\epsilon}\big( ((\phi_{\epsilon})^{*}\alpha_{\phi_{\epsilon}(x)} - \alpha_{x} \big).
	$$
	Em \cite{nakahara} pode ser encontrada a demonstração da fórmula de Cartan	
	$$
	\liederivada{X}\alpha = i_{X}d\alpha + d(i_{X}\alpha),
	$$
	onde $i_{X}:\Omega^{k} \to \Omega^{k-1}$ por $i_{X}\alpha(Y_{1}, \dots, Y_{k-1}) = \alpha(X, Y_{1}, \dots, Y_{k-1})$.	
	
	\begin{proposicao}\label{proposicao_familia_simplectomorfismos}
		As aplicações $\psi_{t}:M\to M$ dadas pelo campo Hamiltoniano são difeomorfismos que preservam a forma simplética.
	\end{proposicao}
	\begin{prova}
		Aplicando a fórmula de Cartan para a derivada de Lie de $\omega$ com respeito ao campo $\campohamiltonianoabrev$, segue que		
		$$
		\begin{aligned}
		\frac{d}{dt}\bigparenteses{(\psi_{t})^{*}\omega} 
		&= (\psi_{t})^{*} \liederivada{X_{H}}\omega  
		\\
		&= (\psi_{t})^{*} i_{X_{H}}\underbrace{d\omega }_ {=0}+  (\psi_{t})^{*}d(i_{X_{H}}\omega) 
		\\
		&= (\psi_{t})^{*} (-ddH)=0,
		\end{aligned}
		$$
		logo $(\psi_{t})^{*} \omega = (\psi_{t'})^{*} \omega$, para quaisquer $t, t' \in \real{}$. Fazendo $t'=0$ e usando o fato de que $\psi_{0} = Id$, tem-se que  $(\psi_{t})^{*} \omega = \omega$.
	\end{prova}
	
	Os difeomorfismos $\psi_{t}$ são  chamados difeomorfismos Hamiltonianos.
	
	\begin{exemplo}\label{exemplo_funcao_hamiltoniana}
		(Função Hamiltoniana) Seja $H:\real{2} \to \reta$ dada por $H(q,p) = q^{2}/2+p^{2}/2$. Pelas equações de Hamilton, pode-se verificar que 
		$$
			\left(
			\begin{array}{c}
			\dot{q}
			\\
			\dot{p}
			\end{array}
			\right) = 
			\left(
			\begin{array}{c}
			p
			\\
			-q
			\end{array}
			\right)
			=\estruturacomplexa
			\left(
			\begin{array}{c}
			q
			\\
			p
			\end{array}
			\right).
		$$
		E o campo Hamiltoniano é $\campohamiltoniano{q,p} = p\partial_{q}-q\partial_{p}$. Com isso, a linha de fluxo que passa pelo ponto $(q,p)$ é dada por $\psi_{t}(q,p) = (q(t), p(t))$, onde $(q(0), p(0)) = (q, p)$. Então o sistema de equações diferenciais $\dot{\psi_{t}}(q,p) = \campohamiltoniano{\psi_{t}(q,p)}$ pode ser escrito em termos matriciais
		$$
		\begin{aligned}
		\dot{\psi_{t}}(q,p) &=  \campohamiltoniano{\psi_{t}(q,p)}
		\\
		\derivada{}{t}\left(
		\begin{array}{c}
		q(t)
		\\
		p(t)
		\end{array}
		\right)
		&=
		\estruturacomplexa
		\left(
		\begin{array}{c}
		q(t)
		\\
		p(t)
		\end{array}
		\right),
		\end{aligned}
		$$ 
		que tem como solução $\psi_{t}=e^{t\estruturacomplexa}$. Portanto, o fluxo do campo $\campohamiltonianoabrev$ é
		$$
		\psi_{t}(q,p) = e^{t\estruturacomplexa}
		\left(
		\begin{array}{c}
		q
		\\
		p
		\end{array}
		\right).
		$$
		Expandindo a exponencial $e^{t\estruturacomplexa}$ em série de Taylor, tem-se $e^{t\estruturacomplexa}=\cos(t)Id+\sin(t)\estruturacomplexa$. Com isso, $		\psi_{t}(q,p)=(q\cos(t)+p\sin(t), p\cos(t)-q\sin(t))$.
		
		Note que $\hessianaponto{(q,p)}{H} = Id$. Logo, a hessiana de $H$ é não-degenerada e os pontos críticos de $H$ são pontos críticos não-degenerados, portanto $H$ é uma função de Morse.
		
		Usando as identidades $D_{(q,p)}\psi_{t} = e^{t\estruturacomplexa}$ e $(e^{t\estruturacomplexa})^{*} = e^{-t\estruturacomplexa}$, tem-se que $\psi_{t}:\real{2}\to \real{2}$ é um simplectomorfismo para todo $t\in \reta$. De fato, a aplicação $\psi_{t}$ é um difeomorfismo. Além disso,
		
		$$
		\begin{aligned}
		(\psi^{*}_{t}\omega_{x})(v,u)
		&=\omega_{x}(D_{x}\psi_{t}(v), D_{x}\psi_{t}(u))
		\\
		&=\omega_{x}(e^{t\estruturacomplexa}v,e^{t\estruturacomplexa}u)
		\\
		&=\omega_{x}(e^{t\estruturacomplexa}(e^{t\estruturacomplexa})^{*}v,u)
		\\
		&=\omega_{x}(e^{t\estruturacomplexa}e^{-t\estruturacomplexa}v,u)
		\\
		&=\omega_{x}(v,u)
		\end{aligned}
		$$
		Logo $	(\psi^{*}_{t}\omega_{x}) = \omega_{x}$ para todo $t\in \reta$.
	\end{exemplo}
	
	Os conceitos desenvolvidos nas próximas seções consideram uma função Hamiltoniana dependente do tempo, isto é, uma Hamiltoniana $H:M\times \reta\to \reta$ associada ao campo Hamiltoniano $\campohamiltonianoabrev$ e que satisfaz as equações
	$$
	\dot{x}(t) = \campohamiltoniano{x(t),t}.
	$$
	
	\begin{observacao}
		A função Hamiltoniana que não tem dependência temporal 	é chamada função Hamiltoniana autônoma.
	\end{observacao}
	
	Bem como ao caso autônomo, as soluções do sistema Hamiltoniano dependente do tempo podem ser associadas a uma família de difeomorfismos $\psi_{t}:M \to M$ tal que
	$$
	\dot{\psi_{t}}(x) = \campohamiltoniano{\psi_{t}(x), t}\; \text{e} \; \psi_{0}=Id. 
	$$ 
	
	Analogamente ao que foi feito na Proposição \ref{proposicao_familia_simplectomorfismos}, pode-se mostrar que essa família de difeomorfismos é uma família de simplectomorfismos.
	
	Uma solução do sistema Hamiltoniano $x:\reta\to M$ é chamada de solução 1-periódica se $x(t) = x(t+1)$ para todo $t\in \reta$. Desse modo, $\psi_{1}(x(0)) = x(1) = x(0)$. Portanto, $x(0)\in M$ é um ponto fixo do difeomorfismo $\psi_{1}:M\to M$. Será visto adiante que, quando 
	a função Hamiltoniana $H$ também é $1$-periódica, então os difeomorfismos tem como pontos fixos, no mínimo, o total de soluções $1$-periódicas do sistema Hamiltoniano (esse resultado está relacionado diretamente com a Conjectura de Arnold).
	
	\begin{definicao}(Soluções $1$-periódicas não-degeneradas)
		Uma solução $1$-periódica de um sistema Hamiltoniano $x: \reta\to M$ é não-degenerada se $\det(D_{x(0)}\psi_{1} - Id)\neq 0$, onde $D_{x(0)}\psi_{1}:T_{x(0)}M \to T_{x(1)}M$.
	\end{definicao}
	
	\begin{proposicao}\label{proposicao_exponencial_hessiana}
		Sejam $x \in M$ um ponto crítico não-degenerado da Hamiltoniana autônoma e $\psi:M\times \reta\to M$ o fluxo de $\campohamiltonianoabrev$. Então 
		$$
		D_{x}\psi_{t} = e^{-t\estruturacomplexa \hessianaponto{x}{H}}
		$$
		em um sistema local de coordenadas de $x$.
	\end{proposicao}
	\begin{prova}
		Fixe $V$ uma vizinhança coordenada de $x$ e  $(q_{1}\dots, q_{n}, p_{1}\dots, p_{n})$ as coordenadas locais de $V$. Então o diferencial $D_{x}\psi_{t}:\espacotangenteponto{x}{M} \to \espacotangenteponto{\psi_{t}(x)}{M}$ é associado a matriz $2n\times 1$
		$$
		D_{x}\psi_{t} = 
		\left(
		\begin{array}{cc}
		\derivadaparcial{H}{\textbf{q}} &\derivadaparcial{H}{\textbf{p}}
		\end{array}
		\right),
		$$
		onde $\derivadaparcial{H}{\textbf{q}}$ denota a $n$-upla $(	\derivadaparcial{H}{q_{1}} \dots \derivadaparcial{H}{q_{n}})$, e o mesmo para $\derivadaparcial{H}{\textbf{p}}$. Fixando $x\in M$ e tomando o diferencial de $D_{x}\psi_{t}$ em termos de $t$
		$$
		\begin{aligned}
		\derivada{}{t}(D_{x}\psi_{t})
		&=
		\left(
		\begin{array}{cc}
		-(\derivadaparcialdoisdois{H}{\textbf{q}}{\textbf{p}}\dot{\textbf{q}}+\derivadaparcialdois{H}{\textbf{p}}\dot{\textbf{p}}) & 
		(\derivadaparcialdois{H}{\textbf{q}}\dot{\textbf{q}}+\derivadaparcialdoisdois{H}{\textbf{p}}{\textbf{q}}\dot{\textbf{p}})
		\end{array}
		\right)
		\\
		&=
		\left(
		\begin{array}{c}
		\left(
		\begin{array}{cc}
		0& -Id
		\\
		Id & 0
		\end{array}
		\right)
		\left(
		\begin{array}{cc}
		\derivadaparcialdois{H}{\textbf{q}} & 
		\derivadaparcialdoisdois{H}{\textbf{q}}{\textbf{p}}
		\\
		\derivadaparcialdoisdois{H}{\textbf{p}}{\textbf{q}} & \derivadaparcialdois{H}{\textbf{p}}
		\end{array}
		\right)
		\end{array}
		\right)
		\left(
		\begin{array}{c}
		\dot{\textbf{q}}
		\\
		\dot{\textbf{p}}
		\end{array}
		\right)
		\\
		&= (\estruturacomplexa\circ \hessianaponto{x}{H})(	D_{x}\psi_{t}).
		\end{aligned}
		$$
		Portanto,
		$$
		D_{x}\psi_{t} = e^{t\estruturacomplexa \hessianaponto{x}{H}},
		$$
		para todo $t\in \reta$ tal que $\psi_{t}(x(0))=x(t)\in V$.
	\end{prova}
	
	\begin{proposicao}
		Seja $H$ uma função Hamiltoniana autônoma. Se $x \in \pontoscriticos{H}$ é não-degenerado como uma solução periódica do sistema Hamiltoniano, então é não-degenerado como ponto crítico de $H$.
	\end{proposicao}
	\begin{prova}
		Seja $\psi:M\times \reta\to M$ o fluxo do campo Hamiltoniano. Se $x\in \pontoscriticos{H}$ é uma solução 1-periódica do sistema Hamiltoniano, então pela Proposição \ref{proposicao_exponencial_hessiana} tem-se que $D_{x}\psi_{1}=e^{\estruturacomplexa \hessianaponto{x}{H}}$ em uma vizinhança coordenada $V$ de $x$. Sejam $(\complexificado{V}, \Omega)$ a complexificação do espaço vetorial simplético $(\espacotangenteponto{x}{M}, \omega_{x})$ e $E_{\lambda}$ o auto-espaço generalizado associado ao auto-valor $\lambda \in \espectrooperador{D_{x}\psi_{1}}$. Para todo $v\in \autoespaco{\lambda}$ tem-se que
		$$
		D_{x}\psi_{1}v=\lambda v.
		$$
		Como $1\notin \espectrooperador{D_{x}\psi_{1}}$, então $\lambda \neq 1$. Com isso, $e^{\estruturacomplexa \hessianaponto{x}{H}}v = D_{x}\psi_{1}v \neq v$, o que implica que
		$\hessianaponto{x}{H}|_{\autoespaco{\lambda}}\neq 0$. Com um argumento análogo, pode-se mostrar que $\hessianaponto{x}{H}|_{\autoespaco{\lambda'}} \neq 0$ para todo auto-valor $\lambda'\neq \lambda$. Como $\complexificado{V} = \bigoplus_{j=1}^{k} \autoespaco{\lambda_{j}}$, para algum $1\leq k\leq2n$, então o operador $e^{\estruturacomplexa \hessianaponto{x}{H} }$ é a matriz com blocos
		$$
			e^{\estruturacomplexa \hessianaponto{x}{H}}=
			\left(
			\begin{array}{ccccc}
			e^{\estruturacomplexa \hessianaponto{x}{H}|_{\autoespaco{\lambda_{1} }} } & 0 &\dots & 0 
			\\
			0 & e^{\estruturacomplexa \hessianaponto{x}{H}|_{\autoespaco{\lambda_{2}}}} &\dots & 0 
			\\
			\vdots & \vdots & \ddots & \vdots 
			\\
			0 & 0 & \dots & e^{\estruturacomplexa \hessianaponto{x}{H}|_{\autoespaco{\lambda_{k}}}}
			\end{array}
			\right).
		$$
		Como para cada $\lambda$ tem-se que $\hessianaponto{x}{H}|_{\autoespaco{\lambda}}\neq 0$, então
		$$
		\hessianaponto{x}{H}\neq 0.
		$$
		Logo $x\in \pontoscriticos{H}$ é um ponto crítico não-degenerado de $H$.
	\end{prova}
	
	\begin{observacao}
		Uma solução $x$ do sistema Hamiltoniano é não-degenerada se, e só se, 1 não é autovalor de $e^{\estruturacomplexa \hessianaponto{x}{H}}$, o que equivale a dizer que a hessiana não tem autovalores em $2\pi \inteiros$. Em particular, se $\norma{\hessianaponto{x}{H}}<2\pi$, vale a resíproca da proposição anterior e as duas noções de não degeneracidade coincidem.
	\end{observacao}
	
	\begin{proposicao}\label{proposicao_solucao_hamiltoniana_finita}
		Seja $H:M\times\circulo\to \reta$ uma função Hamiltoniana. Suponha que todas as soluções $1-$periódicas do sistema Hamiltoniano são não-degeneradas. Então o conjunto dessas soluções é finito.
	\end{proposicao}
	\begin{prova}
		Sejam $\psi_{t}: M\to M$ a família de difeomorfismos gerados pelo fluxo do campo Hamiltoniano $\campohamiltonianoabrev$, $A =\{(y, y):y\in M\}$ o gráfico da aplicação identidade e $B=\{(y, \psi_{1}(y)) :y\in M \}$ o gráfico do difeomorfismo $\psi_{1}$. Dado $y \in M$ tem-se os espaços tangentes 
		$$\espacotangenteponto{(y,y)}{A} = \{(Y,Y):Y\in \espacotangenteponto{y}{M}\}\;\; \text{e}\;\; \espacotangenteponto{(y,\psi_{1}(y))}{B} = \{(Y,D_{y}\psi_{1}(Y)):Y\in \espacotangenteponto{y}{M}\}.
		$$
		
		Seja $S_{H}$ o conjunto das soluções $1-$periódica do sistema Hamiltoniano. Se $x\in S_{H}$, então $\psi_{1}(x(t)) = x(t)$ para todo $t\in \circulo$, e portanto $(x(t), \psi_{1}(x(t)))=(x(t), x(t))\in A\cap B$. Defina $S=\{(x(0), x(0)):x\in S_{H} \}$. Note que $S \subset A\cap B$. Com isso, 
		$$
		\espacotangenteponto{(x(0), x(0))}{S}=\{(Y,D_{x(0)}\psi_{1}(Y)):Y\in \espacotangenteponto{x(0)}{M}\}.
		$$
		Como $x\in S_{H}$ é não-degenerada, então $D_{x(0)}\psi_{1}(Y)\neq Y$. Além disso, segue da definição de $S$ que
		$$
		\espacotangenteponto{(x(0), x(0))}{S} \subset \espacotangenteponto{(x(0), x(0))}{(A \cap B)},
		$$
		o que implica em $D_{x(0)}\psi_{1}(Y) = Y$. Pela não-degeneracidade de $x$, essa igualdade é satisfeita se, e somente se, $Y=0$. Com isso, $\espacotangenteponto{(x(0), x(0))}{S}=\{(0,0)\}$ para todo $(x(0), x(0)) \in S$. Logo $S$ é uma subvariedade de dimensão zero, e portanto é um conjunto de pontos isolados. Como $M\times M$ é um compacto, então $S$ é finito. Como $S$ é finito e todo $x \in S_{H}$ é tal que $x(t) = \psi_{t}(x(0))$, então $S_{H}$ é finito.
	\end{prova}
	
	
	\section{O funcional $\funcionalH$ e a equação de Floer}\label{secao_funcional_hamiltoniano}
	
	Nesse capítulo considere $M$ como sendo uma 2n-variedade diferenciável fechada e $(M, \omega)$ uma variedade simplética.
	
	A conjectura de Arnold tem como hipótese uma Hamiltoniana dependente do tempo $H: M\times \reta \to \reta$ e, para o sistema Hamiltoniano,
	$$
	\dot{x}(t) = X_{H}(x(t), t),
	$$
	busca-se um limitante inferior para o número de soluções 1-periódicas (isto é, $x(t+1)=x(t)$), onde $\campohamiltonianoabrev$ é o único campo vetorial tal que $\formaSimpletica{\campohamiltonianoabrev}{\xi} = -dH(\xi)$ para todo $\xi \in \campossuaves{M}$. Note que laços não homotópicos estão em componentes conexas distantas no espaço dos caminhos contínuos, que não é conexo em geral. Pode-se restringir o estudo à componente conexa que contém os laços constantes. Isto é equivalente a analisar os laços contráteis em $M$, ou seja, as aplicações $x : \circulo\to M$ que satisfazem esse sistema e que são homotópicas a constante. O conjunto das aplicações $x:\circulo \to M$ de classe $C^{\infty}$ e contráteis é denotado por $\lacocontrateis$.
	
	Para demonstrar as desigualdades de Morse, foi construído o complexo de Morse-Witten de uma função de Morse-Smale $f: M\to \reta$ e a diferencial desse complexo descreve como os pontos críticos de $f$ são conectados pelas linhas do fluxo do gradiente negativo $-\gradiente$. Para a demonstração da Conjectura de Arnold, uma abordagem análoga é realizada. Contudo, ao invés de uma variedade de dimensão finita $M$ tem-se um espaço de funções $\lacocontrateis$, e ao invés de uma função de Morse $f:M\to \reta$ será construído um funcional $\funcionalH:\lacocontrateis\to \reta$ cujos pontos críticos são as trajetórias 1-periódicas do sistema Hamiltoniano.
	
	No Capítulo \ref{capitulo_aplicacoes}, a conjectura de Arnold é demonstrada construindo-se um complexo de cadeias cujo grupo graduado é gerado pelos pontos críticos desse funcional e cuja diferencial descreve como tais pontos são conectados pelas linhas de fluxo do gradiente negativo $-\gradientefuncional$. A homologia desse complexo é chamada de Homologia de Floer.
	
	No Seção \ref{apendice_variedades_banach} mostra-se que $\lacocontrateis$, munido da $\cktopologia{\infty}$, é uma variedade conexa por caminhos, mais especificamente uma variedade de Banach. Com isso, dado $x\in \lacocontrateis$ o espaço tangente $\espacotangenteponto{x}{\lacocontrateis}$ está bem-definido e pode ser visto da seguinte forma: Considere um caminho $\tilde{u}:\reta\to \lacocontrateis$ de classe $C^{\infty}$ tal que $\tilde{u}(0)=x$. Desse modo, cada $\tilde{u}$ pode ser visto como uma aplicação $u: \retacartesianocirculo \to M$ tal que $\tilde{u}(s)(t) = u(s,t)$ e $u(0,t) = x(t)$ (veja a Figura \ref{figura_caminho_espaco_laco_contrateis}). Então, para cada $t\in \circulo$ tem-se que
	$$
	\derivadaparcial{}{s}u(s,t)|_{s=0} \in \espacotangenteponto{x(t)}{M}.
	$$
	
	Assim, é natural ver um vetor tangente em $x\in \lacocontrateis$ como sendo um campo vetorial definido ao longo de $x$, ou seja, uma seção $Y$ do fibrado $\pullbackfibradotangenteM{x}$
	$$
	Y_{x(t)}\in \espacotangenteponto{x(t)}{M} \;\text{para todo  }\; t\in \circulo,
	$$
	veja a Observação \ref{observacao_pullback_fibrado_tangente}. 
	
	\begin{figure}[!h]
		\centering
		\includegraphics[width=9cm,height=5cm]{imagem/caminho_espaco_laco.pdf}
		\caption{Caminhos no espaço de laços contráteis $\lacocontrateis$.}
		\label{figura_caminho_espaco_laco_contrateis}
	\end{figure} 
	
	\begin{definicao}\label{definicao_funcional_hamiltoniano}
		(Funcional de Ação) O funcional de ação é a aplicação $\funcionalH: \lacocontrateis\to \reta$ definida por
		$$
		\funcionalHponto{x} = -\int_{D^{2}}u^{*}\omega + \int_{0}^{1}H(x(t), t)dt,
		$$
		onde $D^{2} \subset \mathbb{C}$ é o disco fechado e $u:D^{2}\to M$ é tal que $u(e^{i2\pi t})=x(t)$, ou seja, $u$ é uma extensão de $x$ para o disco.
	\end{definicao}
	
	Será assumido que a variedade $M$ satisfaz a seguintes suposições:
	
	\begin{suposicao}\label{suposicao_aesfericidade}
		(Aesfericidade) Seja $u:S^{2} \to M$ uma aplicação suave, então a condição de aesfericidade é dada por 
		$$
			\int_{S^{2}} u^{*}\omega = 0.
		$$
	\end{suposicao}
	
	\begin{suposicao}\label{suposicao_trivializacao}
		Para cada aplicação $w:S^{2}\to M$ de classe $C^{\infty}$ existe uma trivialização simplética do fibrado $\pullbackfibradotangenteM{w}$.
	\end{suposicao}
	
	\begin{observacao}
		A primeira integral na Definição \ref{definicao_funcional_hamiltoniano} do funcional de ação depende explicitamente da extensão de $x$ escolhida. Supondo que $v : D^{2}\to M$ é outra extensão de $x$, então
		$$
		\int_{D^{2}}u^{*}\omega - \int_{D^{2}}v^{*}\omega  = \int_{S^{2}}w^{*}\omega, 
		$$
		onde $w:S^{2} \to M$ é uma aplicação contínua e é definida como sendo uma colagem dos dois discos $u(D^{2})$ e $v(D^{2})$ ao longo de seus bordos. A Condição de Aesfericidade \ref{suposicao_aesfericidade} garante que $\int_{S^{2}}w^{*}\omega=0$, logo $\int_{D^{2}}u^{*}\omega = \int_{D^{2}}v^{*}\omega$, eliminando a dependência do funcional $\funcionalH$ da escolha dessas extensões, veja Figura \ref{figura_colagem_discos}.
		
		\begin{figure}[!h]
			\centering
			\includegraphics[width=5cm,height=3cm]{imagem/colagem_discos.pdf}
			\caption{Colagem dos bordos de $v(D^{2})$ e $u(D^{2})$.}
			\label{figura_colagem_discos}
		\end{figure}
	\end{observacao}
	
	Seja $T\lacocontrateis$ o fibrado tangente de $\lacocontrateis$. Então o diferencial do funcional de ação é a aplicação $D\funcionalH: T\lacocontrateis \to \real{}$. Em analogia à análise em espaços de dimensão finita, será demonstrado que o gradiente do funcional de ação é o campo suave $\gradientefuncional\in \campossuaves{\lacocontrateis}$ tal que $D_{x}\funcionalH(X) = \produtointerno{\gradientefuncional(x)}{X(x)}$ para todo $X\in \espacotangenteponto{x}{\lacocontrateis}$, onde $\produtointerno{.}{.}$ é um produto interno definido a seguir.
	
	Sejam $Y(t) \in \espacotangenteponto{x(t)}{M}$, $\tilde{x}:(-\epsilon, \epsilon)\times S^{1} \to M$  uma aplicação de classe $C^{1}$ tal que $\tilde{x}(0,t) = x(t)$ e $Y(t) = \derivadaparcial{}{s}\tilde{x}(s,t)|_{s=0}$. Com isso, o diferencial de $\funcionalH$ no ponto $x$ e avaliado em $Y$ é dado por $D_{x}\funcionalH(Y) = \derivadaparcial{}{s}\funcionalH(\tilde{x})|_{s=0}$.
	
	Para se determinar o diferencial do funcional defina $\tilde{u}:(-\epsilon, \epsilon)\times D^{2} \to M$ como sendo uma extensão de $u$, tal que $\tilde{u}(0,z) = u(z)$ e $\tilde{u}(s,e^{2i\pi t}) = \tilde{x}(s,t)$, e $\tilde{Y}(z) = \derivadaparcial{}{s}\tilde{u}(s,z)|_{s=0}$, uma extensão de $Y$, veja a Figura \ref{figura_extensao_solucao}. 
	\begin{figure}[!h]
		\centering
		\includegraphics[width=8.5cm,height=7cm]{imagem/extensao_solucao.pdf}
		\caption{As aplicações $x$ e $u$ e suas variações $\tilde{x}$ e $\tilde{u}$.}
		\label{figura_extensao_solucao}
	\end{figure}
	
	Avaliando o funcional nessas extensões, tem-se
	
	$$
	\funcionalHponto{\tilde{x}} = -\int_{D^{2}}\tilde{u}^{*}\omega + \int_{0}^{1}H(\tilde{x}(s,t),t)dt.
	$$
	
	\begin{observacao}
		Note que, $	\funcionalHponto{\tilde{x}}$ é uma função de classe $C^{1}$ e que depende explicitamente de $s\in (-\epsilon, \epsilon)$, isto é, $	\funcionalHponto{\tilde{x}}:(-\epsilon,\epsilon)\to \reta$. 
	\end{observacao}
	
	Seja $\phi:\retacartesianovariedade\to M$ o fluxo gerado pelo campo $Y\in \campossuaves{M}$. A derivada de Lie no ponto $p\in M $ de uma r-forma $\alpha \in \Omega^{r}(M)$ ao longo do campo $Y$ é definida por 
	$$
	\liederivada{Y}\alpha = \derivada{}{t}(\phi^{*}_{p}(t)\alpha)|_{t=0},
	$$ 
	o que pode ser encontrado em \cite{nakahara}. Além disso, pode-se mostrar a identidade 
	$$
	\liederivada{Y}\alpha= di_{Y}\alpha+i_{Y}d\alpha,
	$$
	onde $i_{Y}:\Omega^{r}(M)\to \Omega^{r-1}(M)$ é definida por $i_{Y}(\alpha)(X_{1}, \dots, X_{r-1}) = \alpha(Y, X_{1}, \dots, X_{r-1})$.
	
	Derivando o primeiro termo da expressão obtem-se
	$$
	\begin{aligned}
	-\frac{d}{ds} \Bigm\lvert_{s=0} \int_{D^{2}}\tilde{u}^{*}\omega 
	&=-\int_{D^{2}}\frac{d}{ds} (\tilde{u}^{*}\omega)\Bigm\lvert_{s=0} \\
	&=-\int_{D^{2}} \mathcal{L}_{\tilde{Y}(z)}(\omega)
	\\
	&=-\int_{D^{2}}u^{*} \mathcal{L}_{Y}(\omega)
	\\
	&=-\int_{D^{2}}u^{*} (d(i_{Y}\omega) + i_{Y}d\omega)
	\\
	&=-\int_{D^{2}}u^{*} d(i_{Y}\omega)
	\\
	&=-\int_{\circulo}	x^{*} (i_{Y(t)}\omega)
	\\
	&= \int_{[0,1]} \omega(\dot{x}(t), Y(t))dt.
	\end{aligned}
	$$
	
	Derivando o segundo termo:
	$$
	\begin{aligned}
	\frac{d}{ds} \Bigm\lvert_{s=0} \int_{[0,1]} H(\tilde{x}(s,t),t)dt 
	&= \int_{[0,1]} \derivadaparcial{}{s} H(\tilde{x}(s,t),t)dt \Bigm\lvert_{s=0}
	\\
	&= \int_{[0,1]} d_{\tilde{x}(0,t)}H(Y(t),t)dt
	\\
	&= \int_{[0,1]} -\omega_{x(t)}(\campohamiltonianoabrev(x(t), t), Y(t))dt. 
	\end{aligned}
	$$
	Por fim, somando ambos os termos:
	$$
	D_{x}\funcionalH(Y) = \int_{[0,1]} \omega(\dot{x} - \campohamiltonianoabrev, Y)dt.
	$$
	
	Se $x \in \lacocontrateis$ é um ponto crítico de $\funcionalH$, então $D_{x}\funcionalH(Y) = 0 $ para todo $Y(x) \in \espacotangenteponto{x}{\lacocontrateis}$. Como o integrando de $D_{x}\funcionalH(Y)$ é uma aplicação contínua, então pode-se afirmar que
	$$
	\omega(\dot{x} - \campohamiltonianoabrev, Y)=0,\; \forall Y \in \espacotangenteponto{x}{\lacocontrateis}.
	$$
	Como $\omega$ é uma forma simplética não-degenerada, então $\dot{x}(t) - \campohamiltonianoabrev(x, t)=0$. Portanto, foi demonstrado o seguinte teorema:
	
	\begin{teorema}
		Seja $x \in \lacocontrateis$. Então $x \in \pontoscriticos{\funcionalH}$ se, e somente se, $x$ é uma solução 1-periódica e contrátil do sistema Hamiltoniano.
	\end{teorema}
	
	O próximo passo é definir o campo gradiente negativo de $\funcionalH$. Sabe-se que, dado $J \in \estruturascomplexas{M}{\omega}$, tem-se que $g_{x}(., .)=\omega_{x}(.,J_{x}.):T_{x}M\times T_{x}M \to \reta$, definida por $\omega_{x}(X,J_{x}Y)$, é uma métrica Riemannian.
	
	Com isso, pode-se munir $\lacocontrateis$ com uma métrica $\produtointerno{.}{.}_{x}: \espacotangenteponto{x}{\lacocontrateis} \times \espacotangenteponto{x}{\lacocontrateis} \to \reta$, a qual é induzida por $g$ do seguinte modo:
	
	$$
	\produtointerno{X}{Y}_{x} = \int_{0}^{1}g_{x(t)}(X, Y)dt.
	$$
	
	Note que, na definição da métrica induzida foi usada a identificação dos fibrados tangentes $\espacotangenteponto{x}{\lacocontrateis} \ni X \mapsto  X|_{x} \in \pullbackfibradotangente{x}{M}$ (veja a Observação \ref{observacao_pullback_fibrado_tangente}). Não é difícil verificar que esta fórmula define uma métrica. De fato, como $g$ é bilinear e a integração é uma operação linear, então $\produtointerno{.}{.}_{x}$ é bilinear. Além disso, como $g$ é positiva-definida e a integral preserva essa positividade, então $\produtointerno{.}{.}_{x}$ herda essa propriedade.
	
	\begin{definicao}
		(Gradiente de $\funcionalH$) O campo gradiente de $\funcionalH$ é o campo suave denotado por $\gradientefuncional$ tal que $D_{x}\funcionalH(Y) = \produtointerno{\gradientefuncional(x)}{Y}_{x}$ para todo $x \in \lacocontrateis$ e $Y \in \espacotangenteponto{x}{\lacocontrateis}$.
	\end{definicao}
	
	Usando a expressão do diferencial $D_{x}\funcionalH$ determinanda anteriormente e a definição do campo gradiente, tem-se que
	$$
	\begin{aligned}
	\int_{[0,1]} \omega(\dot{x} - \campohamiltonianoabrev, Y)dt&=
	D_{x}\funcionalH(Y)
	\\ 
	&= \iprod{\gradientefuncional}{Y}_{x}
	\\
	&= \int_{[0,1]}g_{x(t)}(\gradientefuncional, Y)dt
	\\
	&=\int_{[0,1]} \omega_{x(t)}(\gradientefuncional, JY)dt
	\\
	&=\int_{[0,1]} \omega_{x(t)}(-J\gradientefuncional, Y)dt.
	\end{aligned}.
	$$
	
	Logo
	$$
	\int_{[0,1]} \omega(\dot{x} - \campohamiltonianoabrev + J\gradientefuncional, Y)dt = 0.
	$$
	
	Como a igualdade vale para todo $Y \in \espacotangenteponto{x}{\lacocontrateis}$ e o integrando é uma função contínua, então $\omega(\dot{x} - \campohamiltonianoabrev + J\gradientefuncional, Y)=0$. Além disso, a não-degeneracidade da forma simplética $\omega$ implica que
	$$
	J\gradientefuncional(x(t)) +\dot{x}(t)-\campohamiltoniano{x(t), t} = 0, \; \forall t \in \intervalo.
	$$
	Portanto, o campo gradiente de $\funcionalH$ é dado por
	$$
	\gradientefuncional(x(t))= J_{x(t)}\dot{x}(t)-J_{x(t)}\campohamiltoniano{x(t), t}.
	$$
	
	Suponha que $u :\reta\to \lacocontrateis$ seja uma solução do sistema $\derivada{}{s}u(s)|_{s=0} = -\gradientefuncional(x)$ e $u(0)=x$. Tais soluções são as trajetórias em $\lacocontrateis$ do campo gradiente negativo $-\gradientefuncional$. Note que, para cada $s\in \reta$ tem-se que $u(s):\circulo\to M$ é um caminho contínuo e fechado. Portanto as trajetórias de $-\gradientefuncional$ podem ser vistas como aplicações $u:\retacartesianocirculo \to M$ de classe $C^{1}$. Substituindo essa expressão na equação do gradiente de $\funcionalH$ tem-se uma equação diferencial chamada Equação de Floer
	$$
	\operadorFloerDefParametros{u}=0.
	$$

	\begin{observacao}
		Note que, na determinação das soluções da equação de Floer não se está diante de um problema de Cauchy, pois não se tem condições de contorno nessa formulação.
	\end{observacao}
	
	\section{Soluções de Energia Finita $\energiafinitaM$}\label{secao_funcional_energia}
	
	No Capítulo \ref{capitulo_teoria_morse} foi visto que as linhas de fluxo do gradiente negativo de uma função de Morse, definida em uma variedade fechada, conectam dois pontos críticos (veja o Lema $\ref{lema_conjunto_limite_funcao_morse}$). Ainda no caso clássico, suponha que $f \in \funcoesmorsesmale{M}$ e que $\gamma:\reta \to M$ seja uma solução do sistema de equações diferenciais
	$$
	\derivada{}{s}\gamma(s)=-\gradiente(\gamma(s)),
	$$
	isto é, $\gamma$ é uma trajetoria do gradiente negativo.
	
	Defina o funcional energia por
	$$
	E(\gamma)= -\int_{\reta}\gamma^{*} df.
	$$
	Se $\gamma$ conecta dois pontos críticos $p,q\in \pontoscriticos{f}$, ou seja, $\lim_{s\to -\infty}\gamma(s)=p$ e $\lim_{s\to \infty}\gamma(s)=q$, então a energia de $\gamma$ é dada por $E(\gamma) = f(p) -f(q)$. Como a função de Morse $f$ é decrescente ao longo das linhas de fluxo do campo $-\gradiente$, então $0\leq E(\gamma)<\infty$.
	
	Um dos objetos de interesse desse capítulo são as soluções da equação de Floer que conectam pontos críticos do funcional $\funcionalH$. Será mostrado adiante que essas soluções residem em um espaço de funções cuja compacidade é dada em certas condições (veja o Teorema \ref{teorema_compacidade_gromov}). Além disso, as soluções $u:\reta\to \lacocontrateis$, que podem ser vistas como as linhas do fluxo do campo $-\gradientefuncional$, conectam dois pontos críticos $\gamma, \gamma'\in \pontoscriticos{\funcionalH}$ se, e somente se, o funcional energia $E(u)$ é finito.
	
	Seja $\cilindrosLM$ o conjunto dos caminhos de classe $C^{1}$ em $\lacocontrateis$.
	
	\begin{definicao}
		(Funcional Energia) O funcional energia é a aplicação $E: \cilindrosLM \to \reta$ definida por
		$$
		E(u) = -\frac{1}{2}\int_{\reta}u^{*}d\funcionalH.
		$$
	\end{definicao}
	
	\begin{observacao}
		No final da seção anterior foi realizada a identificação $\lacocontrateis \ni u(s) \mapsto u_{z}(s)\in M $, onde $u(s):\circulo\to M$ e $u_{z}(s) = u(s)(z)$.
		Com isso, dado $u \in \cilindrosLM$, tem-se que $E(u)$ depende explicitamente de $z \in \circulo$ e que $E(u):\circulo\to \reta$ é uma aplicação diferenciável.
	\end{observacao}
	
	Dado $Y \in T\reta$, tem-se que
	$$
	\begin{aligned}
	(u^{*}D\funcionalH)_{s}(Y) &= D_{u}\funcionalH(D_{s}u(Y))
	\\
	&=\produtointerno{\gradientefuncional}{Y\derivada{}{\lambda}u(\lambda)|_{\lambda=s}}_{u(s)}
	\\
	&=\produtointerno{\gradientefuncional}{-Y\gradientefuncional}_{u(s)}
	\\
	&=-Y\norma{\gradientefuncional(u(s))}^{2}.
	\end{aligned}
	$$
	Como $Y\in T\reta$ é arbitrário, então $(u^{*}D\funcionalH)_{s} = -\norma{\gradientefuncional(u(s))}^{2}$. Inserindo essa expressão no funcional energia, tem-se
	$$
	\begin{aligned}
			E(u)
			&=-\frac{1}{2}\int_{\reta}u^{*}d\funcionalH 
			\\
			&= -\frac{1}{2}\int_{\reta}-\norma{\gradientefuncional(u(s))}^{2}
			\\
			&= \frac{1}{2}\int_{\reta}\bigparenteses{ \int_{\circulo} \bigparenteses{\normagrande{\derivadaparcial{u}{s}(s,t)}^{2} + \normagrande{\derivadaparcial{u}{t}(s,t) - \campohamiltoniano{u(s,t)}}^{2}} dt }ds.
	\end{aligned}
	$$
	
	Se $u$ é uma solução da equação de Floer, então
	$$
	E(u)=\int_{\retacartesianocirculo}\normagrande{\derivadaparcial{u}{s}(s,t)}^{2}.
	$$
	\begin{observacao}
		Seguem algumas propriedade do funcional energia:
		
		\begin{enumerate}
			\item Para todo $u \in \cilindrosLM$ tem-se que $E(u)\geq 0$.
			
			\item Como $u$ é uma solução da equação de Floer,  então $E(u)=0$ se, e somente se u é um ponto crítico de $\funcionalH$.
			
			\item Se $u \in \cilindrosLM$ conecta dois pontos $x^{-}, x^{+} \in \pontoscriticos{\funcionalH}$, isto é, $\lim_{s\to -\infty}u(s)=x^{-}$ e $\lim_{s\to \infty}u(s)=x^{+}$, então $E(u)=\funcionalH(x^{-}) - \funcionalH(x^{+})<\infty$.
		\end{enumerate}
		
	\end{observacao}
	
	O conjunto das soluções contráteis e de energia finita da equações de Floer é denotado por $\energiafinitaM$.

	Para definir a topologia de $\energiafinitaM$, considere $M$ mergulhada em um espaço Euclideano $\real{m}$ para $m$ suficientemente grande. Pode-se então considerar a $\cktopologia{\infty}$ 
	para as aplicaçoes de $\circulo$ em $M$. Além disso, nos subconjuntos compactos de $\retacartesianocirculo$ será adotada a topologia da convegência uniforme, e com essas topologias, denotamos o espaço das aplicações de $\retacartesianocirculo$ em $M$ por $\aplicaoessuaves{\retacartesianocirculo}{M}$ ou  $\aplicaoessuavesloc{\retacartesianocirculo}{\real{m}}$.
	
	O teorema seguinte garante que, assim como no caso Morse, as trajetórias com energia finita são exatamente as que conectam dois pontos críticos do funcional de ação. A demonstração desse resultado é bastante longa e técnica e não será apresentada nesse trabalho (veja em \cite{audi_floer_homology}).
	
	\begin{teorema}\label{teorema_limite_solucoes_energia_finita}
		Suponha que todas as trajetórias 1-periódicas do campo Hamiltoniano $\campohamiltonianoabrev$ sejam não-degeneradas. Então para cada $u \in \energiafinitaM$ existem dois pontos $x^{-}, x^{+}\in \pontoscriticos{\funcionalH}$ tais que
		$$
		\lim_{s\to -\infty}u(s)=x^{-},\; \lim_{s\to \infty}u(s)=x^{+}\;\;
		$$
		em $C^{\infty}(\circulo;M)$. Além disso, 
		$$
		\lim_{s\to \pm \infty}\derivadaparcial{u}{s}(s,t) = 0,
		$$
		converge uniformemente em $t\in \circulo$.
	\end{teorema}

	Sejam $x^{-}, x^{+}\in \pontoscriticos{\funcionalH}$. O conjunto dos caminhos $u \in \energiafinitaM$ tais $\lim\limits_{s\to \pm \infty}u(s) = x^{\pm}$ é denotado por $\energiafinitaMconectante$.

	\begin{observacao}\label{observacao_uniao_orbitas_conectantes}
		Uma consequência imediata do teorema anterior é que 
		$$
		\energiafinitaM = \bigcup_{x^{-}, x^{+} \in \pontoscriticos{\funcionalH}} \energiafinitaMconectante.
		$$
	\end{observacao}
	
	\chapter{Índice de Maslov e o Índice de Conley-Zehnder}\label{capitulo_indice_maslov}
	\section{Motivação}
	Considere a função Hamiltoniana dependente do tempo $H:M\times \reta\to \reta$ à qual é associado o campo vetorial Hamiltoniano $X_{H} \in \campossuaves{M}$, que é o único campo vetorial que satisfaz a equação $\omega_{x}(\campohamiltoniano{x}, Y) = -dH(Y)$ para todo $Y \in \espacotangenteponto{x}{M}$ e todo $x \in M$. 
	
	As soluções do sistema Hamiltoniano $\dot{x}(t) = X_{H}(x(t), t)$ que satisfazem as condições periódicas $x(t+1) = x(t)$ para todo $t\in \reta$, ou seja, as soluções 1-periódicas das equações de Hamilton, determinam uma família de simplectomorfismos $\psi_{t}:M \to M$ (difeomorfismos que preservam a forma simplética) que satisfazem a relação 
	$$
	\derivada{\psi_{t}}{t}=\campohamiltonianoabrev\circ\psi_{t} \;\;, \;\; \psi_{0}=Id \;\;\text{e}\;\;  \psi_{t}(x(0)) = x(t).
	$$
	Contudo, as soluções de interesse são aquelas 1-periódicas contráteis e não-degeneradas, pois elas são os pontos críticos do funcional de ação $\funcionalH$, onde uma solução não-degenerada é aquela em que 
	$$
	\det(D_{x(0)}\psi_{t} -Id)\neq 0.
	$$
	
	
	Nesse novo contexto, será feito o análogo ao que foi feito para o caso Morse. A aplicação $\iconleyabrev:\pontoscriticos{\funcionalH} \to \inteiros$, chamada índice de Conley-Zehnder, atribui um índice a cada um desses pontos críticos. Tal aplicação é definida a partir do índice de Maslov $\mu:\caminhosespeciais{\gruposimpletico{2n}} \to \inteiros$, onde $\caminhosespeciais{\gruposimpletico{2n}}$ é o conjunto dos caminhos contínuos $\psi:\intervalo \to \gruposimpletico{2n}$ tais que 
	$$
	\psi(0)=Id\;\;\text{e}\;\det(\psi(1)-Id)\neq 0.
	$$
	
	Para cada solução não-degenerada $x \in \pontoscriticos{\funcionalH}$ tem-se que o caminho $A_{x}:\intervalo \to \gruposimpletico{2n}$, definido por $A_{x}(t) = D_{x(0)}\psi_{t} $, é um elemento de $\caminhosespeciais{\gruposimpletico{2n}}$. Ao caminho $A_{x}$ pode-se associar o índice de Maslov $mu(A_{x})$ e este, por sua vez será o índice de Conley-Zehnder $\iconley{x}$ da solução $x$.
	
	Originalmente, o índice de Maslov foi definido para associar um caminho fechado em $\gruposimpletico{2n} $ a um número inteiro. Contudo, existe uma pluralidade de definições equivalentes desse mesmo objeto, e por equivalente entende-se aquelas definições que satistazem a mesma axiomatização. Em $\cite{cappell_maslov_index_equivalencia}$ pode-se encontrar quatro definições distintas e a demonstração de suas equivalências.
	
	
	\section{Construção de $\rho: \gruposimpletico{2n} \to \circulo$}
	Para a construção do índice de Maslov, serão adotadas as complexificações $(\complexificado{V}, \Omega) $ e $\gruposimpleticocomplexo{2n}$ do 2n-espaço vetorial simplético $(V, \omega)$ e do grupo simplético $\gruposimpletico{2n}$, respectivamente. Os detalhes da complexificação podem ser encontrados no Apêndice $\ref{apendice_complexificacao_espacos_vetoriais}$. 
	
	O primeiro passo é a construção de uma aplicação contínua $\rho: \gruposimpletico{2n}\to \circulo$, que é uma extenção contínua do determinante complexo e induz o isomorfismo entre os grupos fundamentais, além de satisfazer propriedades que serão herdadas pelo índice de Maslov.
	
	\begin{teorema}\label{teorema_aplicacao_rho}
		Para todo $n\ \in \mathbb{N}^{*}$ existe uma aplicação contínua $\rho:Sp(2n) \to S^{1}$ satisfazendo as seguintes propriedades:
		\begin{enumerate}
			\item \label{item_naturalidade_rho} \textbf{Naturalidade:}  Se $T\in \gruposimpletico{2n}$, então 
			$$
			\rho(TAT^{-1}) = \rho(A)
			$$
			para todo $A\in \gruposimpletico{2n}$.
			
			\item \label{item_produto_rho} \textbf{Produto:} Sejam $(V_{1}, \omega_{1})$ e $(V_{2}, \omega_{2})$ espaços vetoriais simpléticos com dimensão $n$ e $m$, respectivamente. Se $(V,\omega) = (V_{1}\times V_{2},\omega_{1}\times \omega_{2})$, então
			$$
			\rho(A) = \rho(A_{1})\rho(A_{2})
			$$
			para $A\in \gruposimpletico{V}$ definida por $A(v_{1}, v_{2})=(A_{1}v_{1}, A_{2}v_{2})$, onde $A_{1} \in \gruposimpletico{2n}$ e $A_{2} \in \gruposimpletico{2m}$.
			
			\item \label{item_determinante_rho} \textbf{Deteminante:} Se $A\in \matrizSimpleticaOrtogonal$, então 
			$$
			\rho(A) = det_{\complexo{}}(X+iY), \text{onde} \;	
			A=\left(
			\begin{array}{cc}
			X & -Y					\\
			Y & X
			\end{array}
			\right).
			$$
			Logo, a aplicação induzida $\rho_{*}: \grupofundamental{\gruposimpletico{2n}} \to \grupofundamental{\circulo}$ é um isomorfismo.
			
			\item \label{item_normalizacao_rho} \textbf{Normalização:} Se $A \in \gruposimpletico{2n}$ com $\sigma(A)\cap \circulo = \emptyset$, então $\rho(A) = \pm 1$.
			
			\item \label{item_inversa_rho} \textbf{Inversa:} Para todo $A \in \gruposimpletico{2n}$ tem-se que $\rho(A^{-1})=(\rho(A))^{-1}$. 
		\end{enumerate}
	\end{teorema}
	
	Considere agora a complexificação $(\complexificado{V}, \Omega)$ de $(V, \omega)$. Note que $\rho$ é uma aplicação que estende continuamente o determinante. Para construir essa aplicação, será introduzida uma ordenação nos auto-valores de $A$. Sabe-se que um dado $\lambda \in \espectrooperador{A}$ assume um dos valores $|\lambda|\leq 1$ ou $|\lambda|>1$. Além disso, se $\lambda \in \espectrooperador{A}$, então $\overline{\lambda}, \lambda^{-1}, \overline{\lambda}^{-1} \in \espectrooperador{A}$. Com isso, basta analisar os casos em que $|\lambda| \leq 1$, o que implica que $|\overline{\lambda}|\leq 1$, $|\lambda^{-1}|\geq 1$ e $|\overline{\lambda}^{-1}|\geq 1$.
	
	Um auto-valor $\lambda \in \espectrooperador{A}$ é um auto-valor de primeiro tipo se satisfaz as seguintes propriedades: 
	\begin{enumerate}
		\item Se $\lambda \in \circulo\backslash\{-1,1\}$, então $0\neq Q(v)\in \reta$. Além disso, o sinal de $Q(v)$ é independente do auto-vetor $v$ associado a $\lambda$ (veja a Corolário \ref{corolario_nao_degeneracidade_Q}). Nesse caso, $\lambda$ é um auto-valor de primeiro tipo se $Q(v)>0$.
		
		\item Se $\lambda\notin \circulo$, então $\lambda$ é um auto-valor de primeiro tipo de $|\lambda|<1$.
	\end{enumerate}
	
	O conjunto dos auto-valores de primeiro tipo de $A$ é denotado por $\autovalorprimeirotipo{A}$. 
	
	Tome $\lambda_{1}\in \sigma(A)$ tal que $|\lambda_{1}|<1$ e o escreva como $\lambda_{1} = |\lambda_{1}|\exp(i\theta_{\lambda_{1}})$, para algum $\theta_{\lambda_{1}} \in [0,2\pi]$. Tome um outro auto-valor $\lambda_{2}$ tal que $|\lambda_{2}|<1$. Se $|\lambda_{1}| < |\lambda_{2}|$ ou  $|\lambda_{1}|= |\lambda_{2}|$ e $\theta_{\lambda} \leq \theta_{\beta}$, será dito que $\lambda_{1} \prec \lambda_{2}$. Repetindo esse procedimento n-vezes tem-se a sequência $\lambda_{1} \prec \dots \prec \lambda_{n} \prec  \lambda_{n}^{-1} \prec \dots \prec \lambda_{1}^{-1}$. Essa ordenação será chamada de ordem de primeiro tipo. Com isso, segue que $\autovalorprimeirotipo{A} = \{\lambda_{1} ,\dots, \lambda_{n} \}$.
	
	Sejam $m(\lambda)$ a multiplicidade do auto-valor $\lambda \in \espectrooperador{A}$  e $\gruposimpleticoespecial{2n} =\{A\in \gruposimpletico{2n} : m(\lambda) = 1,\;\forall \lambda\in \espectrooperador{A} \}$ o conjunto das matrizes simpléticas cujos auto-valores têm multiplicidade igual a 1.
	
	Considere a aplicação $\hat{\rho}:\gruposimpleticoespecial{2n} \to \circulo$ definida por 
	$$
		\hat{\rho}(A) = \prod_{\lambda\in \autovalorprimeirotipo{A}}\frac{\lambda}{|\lambda|}.
	$$
	
	A seguir será construída uma extensão contínua dessa aplicação para $\gruposimpletico{2n}$.
	
	\begin{lema}\label{lema_forma_simetrica_B}
		A aplicação $B:\complexificado{V}\times \complexificado{V} \to \reta$ definida por $B(v, u ) = \parteImaginaria{\formaSimpleticaExtendida{\overline{v}}{u}}$ é uma forma $\reta$-bilinear simétrica e não-degenerada. Além disso
		$$
		B(iv,iu) = B(v,u)\;\text{e }\;B(\overline{v},\overline{u})=-B(v,u).
		$$
	\end{lema}
	\begin{prova} 
		A $\reta$-bilinearidade de $B$ é dada imediatamente pela bilinearidade de $\Omega$. Pela Proposição \ref{proposicao_decomposicao_Omega} se tem a decomposição da forma simplética $\Omega=g+ik$. Com isso, segue-se que  $\parteImaginaria{\formaSimpleticaExtendida{\overline{v}}{u} } = \parteImaginaria{\formaSimpleticaExtendida{\overline{u}}{v} }$ e $B(v,u) = B(u,v)$. Portanto $B$ é simétrica.
		
		Suponha que $B$ seja degenerada, isto é, existe $v \in \complexificado{V}$ com $v\neq 0$ tal que $B(v, u ) = 0$ para todo $u \in \complexificado{V}$. Então
		
		$$
		\begin{aligned}
		\formaSimpleticaExtendida{\overline{v}}{u} &=\parteReal{\formaSimpleticaExtendida{\overline{v}}{u}} +i \parteImaginaria{\formaSimpleticaExtendida{\overline{v}}{u}} 
		\\
		&= \parteImaginaria{-i\formaSimpleticaExtendida{\overline{v}}{u}} +i \parteImaginaria{\formaSimpleticaExtendida{\overline{v}}{u}}
		\\
		&= \underbrace{\parteImaginaria{\formaSimpleticaExtendida{\overline{v}}{-iu}}}_{=0} +i \underbrace{\parteImaginaria{\formaSimpleticaExtendida{\overline{v}}{u}}}_{=0}
		\\
		&=0,
		\end{aligned}
		$$
		o que implica que a forma simplética $\Omega$ é degenerada, contradizendo a hipótese. Portanto $B$ é não-degenerada.
		
		Por fim, tem-se as identidades $B(iv,iu) = \parteImaginaria{\formaSimpleticaExtendida{-i\overline{v}}{iu}}= \parteImaginaria{\formaSimpleticaExtendida{\overline{v}}{u}}=B(v,u)$ e $B(\overline{v},\overline{u})  = \parteImaginaria{\formaSimpleticaExtendida{v}{\overline{u}}} = \parteImaginaria{-\formaSimpleticaExtendida{\overline{u}}{v}} = -B(u,v) = -B(v,u)$
	\end{prova}
	
	\begin{proposicao}\label{proposicao_forma_quadratica_Q}
		Sejam  $Q: \complexificado{V} \to \reta$ a forma quadrática definida por $Q(v) = B(v,v)$ e $A \in \gruposimpletico{2n}$. Se $\lambda\in \espectrooperador{A} \cap \complementar{\circulo}{\{\pm 1 \} }$, então   $\autoespaco{\lambda} = \autoespaco{\lambda}^{+}\oplus \autoespaco{\lambda}^{-}$, onde $\autoespaco{\lambda}^{\pm}$ são os maiores subespaços do auto-espaço generalizado $\autoespaco{\lambda}$ tais que $Q$ é positiva (negativa) definida.
	\end{proposicao}
	\begin{prova}
		É imediato que $\autoespaco{\lambda}^{+} \cap \autoespaco{\lambda}^{-} = \emptyset$. Além disso, $\autoespaco{\lambda}^{\pm} \subset \autoespaco{\lambda}$ são subespaços vetoriais pois, dados $\beta\in \complexo{}$, $v, u\in \autoespaco{\lambda}^{+}$ tem-se que 
		$$
		\begin{aligned}
		Q(\beta v +u) &= \parteImaginaria{\overline{\beta}\beta\formaSimpleticaExtendida{\overline{v}}{v}+ \overline{\beta}\formaSimpleticaExtendida{\overline{v}}{u}+ \beta\formaSimpleticaExtendida{\overline{u}}{v}+ \formaSimpleticaExtendida{\overline{u}}{u}}
		\\
		&=\parteImaginaria{\overline{\beta}\beta\formaSimpleticaExtendida{\overline{v}}{v}+ \formaSimpleticaExtendida{\overline{u}}{u}}
		\\
		&=|\beta|^{2}Q(v)+Q(u)
		\\
		&\geq0,
		\end{aligned}
		$$
		onde usou-se o fato de que $\beta\formaSimpleticaExtendida{\overline{u}}{v}=-\overline{\beta}\formaSimpleticaExtendida{\overline{v}}{u}$. Portanto $\beta v +u \in \autoespaco{\lambda}^{+}$. Com um argumento análogo mostra-se que $\autoespaco{\lambda}^{-}$ é subespaço vetorial de $\autoespaco{\lambda}$. Note que, $\formaSimpleticaExtendida{\overline{v}}{v}=\formaSimpleticaExtendida{\overline{Av}}{Av} =|\lambda|^{2}\formaSimpleticaExtendida{\overline{v}}{v}$. Se $\lambda \notin \complementar{\circulo}{\{\pm 1 \} }$, então $\Omega|_{\autoespaco{\lambda}\times \autoespaco{\lambda}} = 0$.
	\end{prova}
	
	\begin{corolario}\label{corolario_nao_degeneracidade_Q}
		Se $\lambda\in \espectrooperador{A} \cap \complementar{\circulo}{\{\pm 1 \} }$, então a forma quadrática $Q$ é não-degenerada e $Q(u)>0$ ou $Q(u)<0$ para todo $v \in \autoespaco{\lambda}$.
	\end{corolario}
	\begin{prova}
		Pelo Lema \ref{lema_forma_simetrica_B} segue que $Q$ é não-degenerada. Seja $\lambda\in \espectrooperador{A} \cap \complementar{\circulo}{\{\pm 1 \} }$. Pela Proposição \ref{proposicao_forma_quadratica_Q} tem-se que $Q(v)>0$ ou $Q(v)<0$, para todo $v\in \autoespaco{\lambda}$. 
	\end{prova}
	
	Denote $dim(\autoespaco{\lambda}^{\pm})$ por $m_{\pm}(\lambda)$, respectivamente. Então $m(\lambda)=m_{+}(\lambda)+m_{-}(\lambda)$. Seja $\rho: \gruposimpletico{2n}\to \complexo{}$ definida por
	$$
	\rho(A) = 	(-1)^{m_{0}} \prod_{\lambda \in \sigma(A)\cap \complementar{\circulo}{\{\pm 1 \}  }}\lambda^{m_{+}(\lambda)},
	$$
	onde $m_{0}$ denota o número de pares $(\lambda, \lambda^{-1})$ reais negativos contados com multiplicidade.

	Afirmo que a restrição $\rho|_{\gruposimpleticoespecial{2n}}$ coincide com a aplicação $\hat{\rho}$ anteriormente definida. De fato, tomando $A \in \gruposimpleticoespecial{2n}$ tem-se que $m(\lambda)=1$ e portanto $m_{+}(\lambda) \in \{0,1\}$ para todo $\lambda\in \espectrooperador{A}$. Suponha uma ordem de primeiro tipo para os auto-valores de $A$. Sem perda de generalidade, pode-se supor que $\lambda_{j} \in \reta$ e $\lambda_{j}\in \complexo{}$ para $r+1 \leq  j\leq n$. Se $\lambda_i\in \real{+}$ então $\lambda_i/|\lambda_i|=1$. Se $\lambda_i\in \real{-}$ então $\lambda_i/|\lambda_i|=-1$. Além disso, $m_{+}(\lambda)=m_{-}(\bar{\lambda})$ e $m_{-}(\lambda)=m_{+}(\bar{\lambda})$, logo $m_{+}(\lambda)=1$ se, e somente se, $m_{+}(\bar{\lambda})=0$. Então  
	$$
	\begin{aligned}
	\hat{\rho}(A) 
	&=
	\frac{\lambda_{1}}{|\lambda_{1}|}\dots \frac{\lambda_{n}}{|\lambda_{n}|}
	\\
	&= (-1)^{m_{0}}\frac{\lambda_{r+1}}{|\lambda_{r+1}|}^{m_{+}(\lambda_{r+1})} \dots \frac{\lambda_{n}}{|\lambda_{n}|}^{m_{+}(\lambda_{n})}
	\\
	&=(-1)^{m_{0}}\prod_{\lambda\in \espectrooperador{A}\cap\complementar{\circulo}{\reta}}\lambda^{m_{+}(\lambda)}
	\\
	&= \rho(A).
	\end{aligned}
	$$

	A continuidade da aplicação $\rho$ é uma propriedade necessária para a construção do Índice de Maslov. Contudo, sua demonstração é bastante longa e técnica e por isso será omitida neste trabalho. Para os detalhes da demonstração, veja \cite{audi_floer_homology}.
	
	
	\begin{proposicao}\label{proposicao_continuidade_rho}
		A aplicação $\rho: \gruposimpletico{2n}\to \circulo$ é contínua e $\rho|_{\gruposimpleticoespecial{2n}} = \hat{\rho}$.
	\end{proposicao}
	
	\subsection*{Demonstração das propriedades de $\rho$}
	
	A construção da aplicação $\rho$ depende intrinsecamente do espaço vetorial simplético $(\complexificado{V}, \Omega)$ adotado. Caso se tenha dois espaços vetoriais simpléticos $(\complexificado{V}_{j}, \Omega_{j})$ para $j\in \{1,2\}$, tem-se as aplicações $\rho_{j}:\gruposimpletico{\complexificado{V}_{j}} \to \circulo$. Com um abuso de notação, será identificado $\rho_{j}$ com $\rho$, onde sua definição dependerá do contexto.
	
	\begin{prova}
		
		\begin{enumerate}
			\item \textbf{\textit{Naturalidade:}} Fixe $T \in \gruposimpleticocomplexo{2n}$. Para todo $A \in \gruposimpleticocomplexo{2n}$ tem-se que $\det(TAT^{-1}) = \det(A)$, logo $\espectrooperador{TAT^{-1}} = \espectrooperador{A}$. Sejam $\autoespaco{\lambda}^{TAT^{-1}}$, $\autoespaco{\lambda}^{A}$ e auto-espaços generalizados de $TAT^{-1}$ e $A$, respectivamente. Afirmo que $\autoespaco{\lambda}^{TAT^{-1}} = T(\autoespaco{\lambda}^{A})$. De fato, tomando $v \in \autoespaco{\lambda}^{TAT^{-1}}$, então $TAT^{-1}v = \lambda v$, o que implica que $AT^{-1}v = \lambda T^{-1}v$, logo $T^{-1}v \in \autoespaco{\lambda}^{A}$ e $T^{-1}(\autoespaco{\lambda}^{TAT^{-1}}) \subseteq \autoespaco{\lambda}^{A}$. Como $T$ é um isomorfismo, então $\autoespaco{\lambda}^{TAT^{-1}} = TT^{-1}(\autoespaco{\lambda}^{TAT^{-1}}) \subseteq T(\autoespaco{\lambda}^{A})$. Por outro lado, suponha que $v\in \autoespaco{\lambda}^{A}$. Como $T$ é um isomorfimo, então existe $u \in \complexificado{V}$ tal que $v=T^{-1}u$. Com isso, $AT^{-1}u = \lambda T^{-1}u$ e $ TAT^{-1}u =\lambda u$. Logo $Tv=u \in \autoespaco{\lambda}^{TAT^{-1}}$ e $T(\autoespaco{\lambda}^{A}) \subseteq \autoespaco{\lambda}^{TAT^{-1}}$. Portanto, $\autoespaco{\lambda}^{TAT^{-1}}=T(\autoespaco{\lambda}^{A}) $ e $dim(\autoespaco{\lambda}^{TAT^{-1}})=dim(T(\autoespaco{\lambda}^{A}) )=dim(\autoespaco{\lambda}^{A})$.
			
			Por fim, como $A,T\in \gruposimpleticocomplexo{2n}$, então $TAT^{-1}\in \gruposimpleticocomplexo{2n}$, e pode-se afirmar que $Q(TAT^{-1}v) = Q(v)$ para todo $v \in \complexificado{V}$. Logo  $m^{TAT^{-1}}_{\pm}(\lambda)=m^{A}_{\pm}(\lambda)$, $m_{0}^{TAT^{-1}}=m_{0}^{A}$, e com isso, $\rho(TAT^{-1})=\rho(A)$.
			
			\item \textbf{\textit{Produto:}} Sejam $(\complexificado{V}, \Omega) = (\complexificado{V}_{1}\times \complexificado{V}_{2}, \Omega_{1}\times \Omega_{2})$ um $2k$-espaço vetorial simplético, onde $\complexificado{V}_{1}, \complexificado{V}_{2}$ são $2n$ e $2m$-espaços vetoriais simpléticos, respectivamente, e $k=2n+2m$. Dado $A \in \gruposimpleticocomplexo{2k}$ tem-se que $Av = (A_{1}v_{1}, A_{2}v_{2})$, onde $v=(v_{1}, v_{2}) \in \complexificado{V}$, $A_{1}\in \gruposimpleticocomplexo{2n}$ e $A_{2}\in \gruposimpleticocomplexo{2m}$. Escrevendo $E_\lambda=0$ ($E_\lambda^i=0$) se lambda não é um autovalor de A ($A_i$), um auto-espaço generalizado $E_{\lambda}$ de $A$ pode ser escrito como $E_{\lambda} = E_{\lambda}^{1}\times E_{\lambda}^{2}$, onde $E_{\lambda}^{1}$, $E_{\lambda}^{2}$ são auto-espaços generalizados associados a $\lambda$ de $A_{1}$ e $A_{2}$, respectivamente. Desse modo, a aplicação $\rho: \gruposimpleticocomplexo{2k} \to \circulo$ satisfaz $\rho(A) = \rho(A_{1})\rho(A_{2})$.
			
			\item \textbf{\textit{Determinante:}} Sejam $A \in \mathcal{U}$ e $U \in \matrizunitaria{n}$ identificados pelo isomorfismo descrito no Capítulo \ref{capitulo_grupo_simpletico} dado por
			$$
			\begin{aligned}
				\real{2n} \ni (v_{(1)},v_{(2)}) &\mapsto v_{(1)}+iv_{(2)} \in \complexo{n},
				\\
				\mathcal{U} \ni 
				\left(
					\begin{array}{cc}
					B & -C
					\\
					C & B
					\end{array}
				\right) 
				& \mapsto B+iC \in \matrizunitaria{n}.
			\end{aligned}
			$$
			Se $(v_{(1)},v_{(2)}) \in \real{2n}$ é um auto-vetor de $A$ com auto-valor $\lambda$ em $\complexo{}$, então tem-se que $A(v_{(1)},v_{(2)})=\lambda (v_{(1)},v_{(2)}) $ implica $U(v_{(1)}+iv_{(2)}) = \lambda(v_{(1)}+iv_{(2)})$. Logo $\espectrooperador{A} \subseteq \espectrooperador{U}$. Por outro lado, se $U(v(1)+iv(2))=\lambda(v(1)+iv(2))$, então $A(v(1)+iv(2), -iv(1)+v(2))=\lambda (v(1)+iv(2), -iv(1)+v(2))$, logo $\espectrooperador{U} \subseteq\espectrooperador{A}$. Portanto $\espectrooperador{A} =\espectrooperador{U}$ em $\complexo{}$.
			
			Sabe-se que $\estruturacomplexa: \real{2n} \to \real{2n}$ é um endomorfismo, que $A\estruturacomplexa = \estruturacomplexa A$ e que ambos $A$ e $\estruturacomplexa$ são diagonalizáveis (matrizes normais são diagonalizáveis, veja o Lema \ref{lema_caracterizacao_matriz_normal}), logo seus auto-espaços generalizados são os auto-espaços usuais. Além disso, $\espectrooperador{\estruturacomplexa} = \{i,-i\}$, $\autoespaco{i}^{\estruturacomplexa} = \{u\textbf{e} +iu\textbf{f}: u \in \complexo{n}\}$, $\autoespaco{ -i}^{\estruturacomplexa} = \{u\textbf{e} -iu\textbf{f}: u \in \complexo{n}\}$ e $dim(\autoespaco{i}^{\estruturacomplexa} )=dim(\autoespaco{-i}^{\estruturacomplexa} )=n$, portanto $\complexificado{V} = \autoespaco{i}^{\estruturacomplexa}\oplus \autoespaco{-i}^{\estruturacomplexa}$. 
			Tomando $v \in \autoespaco{i}^{\estruturacomplexa}$, tem-se que 
			$$
			\begin{aligned}
				\formaSimpleticaExtendida{\overline{v}}{v} 
				&= 2i\parteImaginaria{\produtointerno{\overline{v}_{(1)}}{v_{(2)}}}
				\\
				&= 2i\parteImaginaria{\produtointerno{\overline{v}_{(1)}}{iv_{(1)}}}
				\\
				&=-2\parteImaginaria{\norma{v_{(1)}}^{2}}
				\\
				&=-2\norma{v_{(1)}}^{2} \leq 0.
			\end{aligned}
			$$
			
			Portanto a restrição da forma quadrática $Q$, dada pela Definição \ref{proposicao_forma_quadratica_Q}, ao auto-espaço $\autoespaco{ i}^{\estruturacomplexa}$, é negativa definida. Analogamente pode-se verificar que a restrição de $Q$ ao auto-espaço $\autoespaco{ -i}^{\estruturacomplexa}$ é positiva definida. Com isso, pode-se decompor o auto-espaço $\autoespaco{\lambda} = \autoespaco{\lambda}^{+} \oplus \autoespaco{\lambda}^{-}$, onde $\autoespaco{\lambda}^{+} = \autoespaco{\lambda}\cap \autoespaco{- i}^{\estruturacomplexa} $ e $\autoespaco{\lambda}^{-} = \autoespaco{\lambda}\cap \autoespaco{ i}^{\estruturacomplexa} $. Pela identificação anterior tem-se
			$$
			\autoespaco{\lambda}^{+}=\{v_{(1)}\textbf{e}-iv_{(1)}\textbf{f}\in \autoespaco{\lambda}\} = \{v_{(1)} \in \complexo{n} : Uv_{(1)}=\lambda v_{(1)} \} = \autoespaco{\lambda}^{U}.
			$$
			
			Logo $m_{+}(\lambda) = m^{U}(\lambda)$.
			
			Para o caso $\lambda=-1$, pode-se verificar que $u\textbf{e}+v\textbf{f} \in \autoespaco{-1}\cap \real{2n}$ se, e somente se, $u+iv \in \autoespaco{-1}^{U}$. Logo, os auto-espaços $\autoespaco{-1}$ e $\autoespaco{-1}^{U}$ são isomorfos. Além disso, 
			$$
			m(-1) = \dim_{\reta}(\autoespaco{-1}\cap \real{2n}) = \dim_{\reta}(\autoespaco{-1}^{U}) = 2\dim_{\complexo{}}(\autoespaco{-1}^{U}) =2m^{U}(-1).
			$$
			
			A aplicação $\rho|_{\mathcal{U}}: \mathcal{U} \to \circulo$ quando avaliada em $A\in \mathcal{U}$ é dada em termos do produto dos auto-valores $\lambda \in \espectrooperador{A}$ e suas multiplicidades $m_{+}(\lambda)$ e $m_{0}$. Como $\espectrooperador{A} = \espectrooperador{U}$ e $m_{+}(\lambda)  = m_{\lambda}^{U}$, então as multiplicidades dos auto-valores são as mesmas e $\rho(A)=\det_{\complexo{}}(U)$.

			\item \textbf{\textit{Normalização:}} Seja $A\in \gruposimpletico{2n}$ com $\sigma(A)\cap \circulo = \emptyset$. Então pela própria definição a aplicação $\rho$ não terá produto dos auto-valores, portanto $\rho(A) = (-1)^{m_{0}} = \pm 1$. 
			
			\item \textbf{\textit{Inversa:}} Pela Proposição  \ref{proposicao_caracterizacao_espectro_semelhante}, tem-se que $\espectrooperador{A^{-1}} = \espectrooperador{A}$. Sejam $\autoespaco{\lambda}$ e $F_{\lambda^{-1}}$ auto-espaços generalizados de $A$ e $A^{-1}$, associados aos auto-valores $\lambda$ e $\lambda^{-1}$, respectivamente. É imediato que $\autoespaco{\lambda} = F_{\lambda^{-1}}$, o que implica que $\autoespaco{\lambda}^{+} = F_{\lambda^{-1}}^{+}$ e que $m^{A}_{+}(\lambda) = m^{A^{-1}}_{+}(\lambda^{-1})$ e $m_0^{A^{-1}}=m_0^A$. Então
			$$
			\begin{aligned}
			\rho(A^{-1})
			&= (-1)^{m_0^{A^{-1}}}\prod (\lambda^{-1})^{m_{+}^{A^{-1}}(\lambda^{-1})}
			\\
			&=(-1)^{-m_0^{A}}\prod \lambda^{-m_{+}^{A}(\lambda)}
			\\
			&=\bigparenteses{(-1)^{m_{0}^{A}}\prod \lambda^{m_{+}^{A}(\lambda)}}^{-1}
			\\
			&=(\rho(A)) ^{-1}.
			\end{aligned}
			$$
		\end{enumerate}
	\end{prova}
	
	\section{$\gruposimpleticonaodegenerado{*}$ e sua topologia}
	
	Foi visto na Seção \ref{secao_funcional_hamiltoniano} que as transformações simpléticas de interesse em construções posteriores a esse capítulo são aquelas correspondentes aos elementos de $\gruposimpletico{2n}$ que não possuem auto-valores iguais a 1. Considere então $\gruposimpleticonaodegenerado{*}= \{ A \in \gruposimpletico{2n}: \det(A-Id)\neq 0 \}$ o conjunto dessas transformações. Além disso, defina os conjuntos $\gruposimpleticonaodegenerado{+}= \{ A \in \gruposimpletico{2n}: \det(A-Id)> 0 \}$ e $\gruposimpleticonaodegenerado{-}= \{ A \in \gruposimpletico{2n}: \det(A-Id)<0 \}$. É imediato da definição que $\gruposimpleticonaodegenerado{*}= \gruposimpleticonaodegenerado{+} \cup \gruposimpleticonaodegenerado{-} $.
	
	Será demonstrado agora que $\gruposimpleticonaodegenerado{+}$ e  $\gruposimpleticonaodegenerado{-}$ são as duas componentes conexas de $\gruposimpleticonaodegenerado{*}$. Para isso, será utilizado o lema técnico a seguir, cuja demonstração pode ser encontrada em  $\cite{audi_floer_homology}$.
	
	\begin{lema}\label{lema_conectividade_grupo_simlpetico_nao_degenerado}
		Seja $A\in \gruposimpleticonaodegenerado{*}$. Então existe um caminho contínuo em $\gruposimpleticonaodegenerado{*}$ conectando $A$ a uma matriz $B \in \gruposimpleticonaodegenerado{*}$ cujos  auto-valores são todos distintos com uma das seguintes condições: 1) se $A\in \gruposimpleticonaodegenerado{+}$, então $B$ não possui auto-valores reais positivos ou 2) se $A\in \gruposimpleticonaodegenerado{-}$, então $B$ possui apenas 2 auto-valores reais positivos.
	\end{lema}
	
	\begin{proposicao}\label{proposicao_desconexidade_grupo_simpletico_nao_degenerado}
		$ \gruposimpleticonaodegenerado{+}$ e $\gruposimpleticonaodegenerado{-}$ são conexos por caminhos, logo são conexos. Além disso, esse conjuntos são as componentes conexas de $\gruposimpleticonaodegenerado{*} $.
	\end{proposicao}
	\begin{prova}
		Sejam $(\complexificado{V}, \Omega)$ um 2n-espaço vetorial simplético e $\{\textbf{e}, \textbf{f}\}$ a base simplética de $\complexificado{V}$. Defina $W: \complementar{\reta}{\{0,1\}} \to \generalgroupcomplexo{2n}$ por
		$$
		W(a) = diag\{\underbrace{a, -1, \dots, -1}_{n}, \underbrace{a^{-1}, -1, \dots , -1}_{n}\}.
		$$
		
		Afirmo que $W(\complementar{\reta}{\{0,1\}}) \subset \gruposimpleticonaodegenerado{*}$. De fato, 
		$$
		\det(W(a)- Id) = (a-1)(a^{-1}-1)(-2)^{2n-2} \neq 0,
		$$
		pois $a\neq 1$. Tomando $v,u \in \complexificado{V}$ tais que $v=v_{(1)}\textbf{e}+ v_{(2)}\textbf{f}$ e $u=u_{(1)}\textbf{e}+ u_{(2)}\textbf{f}$, tem-se 
		$$
		\begin{aligned}	
		W(a)v &= (av_{1}, -v_{2}, \dots, -v_{n}, a^{-1}v_{n+1}, -v_{n+2}, \dots, -v_{2n})
		\\
		&= av_{1}e_{1}  -\sum_{j=2}^{n} v_{j}e_{j} + a^{-1}v_{n+1}f_{1} -\sum_{j=2}^{n} v_{n+j}f_{j},
		\\
		\formaSimpleticaExtendida{W(a)v}{W(a)u} 
		&=v_{1}u_{n+1}\formaSimpleticaExtendida{e_{1}}{f_{1}} + \sum_{j=2}^{n}v_{j}u_{n+j}\formaSimpleticaExtendida{e_{j}}{f_{j}}
		\\
		&+v_{n+1}u_{1}\formaSimpleticaExtendida{f_{1}}{e_{1}} + \sum_{j=2}^{n} v_{n+j}u_{j}\formaSimpleticaExtendida{f_{j}}{e_{j}}
		\\
		&=\sum_{j=1}^{n}\bigparenteses{v_{j}u_{n+j} - 	v_{n+j}u_{j}}
		\\
		&= \produtointerno{v_{(1)}}{u_{(2)}} - \produtointerno{v_{(2)}}{u_{(1)}}
		\\
		&= \formaSimpleticaExtendida{v}{u}.
		\end{aligned}
		$$ 
		
		Portanto, $W(a) \in \gruposimpletico{2n}$ e $W(\complementar{\reta}{\{0,1\}}) \subset \gruposimpleticonaodegenerado{*}$.
		
		Defina $W^{+} = W(-1)$ e $W^{-} = W(2)$. Tem-se que $\det(W^{+} - Id) = 4^{n}>0$  e $\det(W^{-} - Id) = -4^{n-1}/2 <0$, logo $W^{+}\in \gruposimpleticonaodegenerado{+}$ e $W^{-}\in \gruposimpleticonaodegenerado{-}$. 
		
		Seja $A \in \gruposimpleticonaodegenerado{*}$. Pelo Lema $\ref{lema_conectividade_grupo_simlpetico_nao_degenerado}$ existe $B \in \gruposimpleticonaodegenerado{*}$ com todos os seus auto-valores distintos  e um caminho contínuo $\alpha:[0,1] \to \gruposimpleticonaodegenerado{*}$ tal que $\alpha(0)=A$ e $\alpha(1) = B$. Seja $\{v_{\lambda_{1}}, \dots , v_{\lambda_{2n}} \}$ uma base de auto-valores de $B$.
		
		Para cada $\lambda \in \complementar{\espectrooperador{B}}{\real{+}}$ considere um caminho contínuo $\gamma:[0,1]\to \reta$ tal que $\gamma(0) = \lambda$ e $\gamma(1) = -1$ tal que $1 \notin \gamma([0,1])$. Seja $\Gamma:[0,1 ]\to \gruposimpleticonaodegenerado{*}$ o caminho contínuo definido por
		$$
		\funcaocond{\Gamma(s)v_{\lambda}}{\gamma(s)v_{\lambda}}{\lambda \notin \real{+}}{\lambda v_{\lambda} }{\lambda \in \real{+}}
		$$
		com $\Gamma(0) = B$. 
		
		Para os auto-valores $\overline{\lambda}$ e $1/\lambda$ tome os caminhos $\overline{\gamma}$ e $1/\gamma$, respectivamente. É imediato da definição que $\Gamma(s)(E_{\lambda}) = E_{\lambda}$. Afirmo que $\Gamma(s) \in \gruposimpleticonaodegenerado{*}$ para todo $s\in \intervalo$. De fato, supondo $\lambda,\lambda' \in \complementar{\sigma(B)}{\real{+}}$ tais que $\lambda\lambda' \neq 1$ tem-se que $\formaSimpleticaExtendida{\Gamma(s)v_{\lambda}}{\Gamma(s)v_{\lambda'}}= \gamma(s)\gamma'(s)\formaSimpleticaExtendida{v_{\lambda}}{v_{\lambda'}} = 0$, pela $\Omega$-ortoganalidade (veja a Proposição \ref{proposicao_auto_espaco_grupo_simpletico}). No caso em que $\lambda\lambda'=1$ se tem  $\lambda'=\lambda^{-1}$, o que implica em $\gamma'(s) = 1/\gamma(s)$, logo $\Omega(\Gamma(s)v_{\lambda},\Gamma(s)v_{\lambda'}) = \Omega(v_{\lambda},v_{\lambda'})$ e $\Gamma(s) \in \gruposimpletico{2n}$. Como $1 \notin \gamma(\intervalo)$, então $\det(\Gamma(s)- Id)\neq 0 $ e $\Gamma(s)\in \gruposimpleticonaodegenerado{*}$ para todo $s\in \intervalo$.
		
		Seja $\alpha:\intervalo\to \gruposimpleticonaodegenerado{*}$ um caminho contínuo conectando $A$ a $B$, cuja existência é garantida pelo Lema $\ref{lema_conectividade_grupo_simlpetico_nao_degenerado}$.
		Por definição de $\gruposimpleticonaodegenerado{*}$ tem-se que $\gruposimpleticonaodegenerado{+} \cap \gruposimpleticonaodegenerado{-}=\emptyset$, portanto $\alpha([0,1]) \subset \gruposimpleticonaodegenerado{+}$, caso $A \in \gruposimpleticonaodegenerado{+}$, ou $\alpha([0,1]) \subset \gruposimpleticonaodegenerado{-}$, caso $A\in \gruposimpleticonaodegenerado{-}$.
		
		\begin{enumerate}
			\item Se $\alpha([0,1]) \subset \gruposimpleticonaodegenerado{+}$, então $B$ não possui auto-valores reais positivos. Logo $\Gamma(0) = B$ e $\Gamma(1) = diag \{-1, \dots, -1\} = W^{+}$. Portanto, $\Gamma$ é um caminho contínuo que conecta $B$ a $W^{+}$. A justaposição $\alpha * \Gamma:\intervalo \to \gruposimpleticonaodegenerado{+}$
			é um caminho contínuo conectando $A$ a $W^{+}$. Tome $A' \in \gruposimpleticonaodegenerado{+}$ tal que $A'\neq A$. Então com uma construção análoga a anterior, conecta-se $A'$ a $W^{+}$ por um caminho contínuo em $\gruposimpleticonaodegenerado{+}$. Tem-se caminhos conectado $A$ a $W^{+}$ e $A'$ a $W^{+}$, respectivamente. Logo conecta-se $A$ a $A'$ através de um caminho contínuo. Portanto $\gruposimpleticonaodegenerado{+}$ é conexo por caminhos, logo é conexo.
			
			\item Se $\alpha([0,1]) \subset \gruposimpleticonaodegenerado{-}$, então $B$ possui apenas 2 auto-valores positivos. Supondo que $\lambda \in \sigma(A)$ tal que $\lambda > 0$, então $\lambda^{-1} > 0$, pois os auto-valores de $A$ são determinados aos pares $\lambda, \lambda^{-1}$. Reordene a base de auto-valores de $B$ tal que $\lambda_{1}=\lambda$ e $\lambda_{n+1}=\lambda^{-1}$ sejam esses auto-valores positivos. Assim, tem-se que $\Gamma(0) = B$ e $\Gamma(1) = diag\{\lambda, -1, \dots, -1, \lambda^{-1}, -1, \dots, -1\}$. Seja $\Gamma': \intervalo \to \gruposimpleticonaodegenerado{-}$ o caminho definido por $\Gamma'(t)=diag\{ \theta(t), -1,\dots, -1, 1/\theta(t) ,-1,\dots, -1\}$,
			onde $\theta(t)=(2- \lambda)t + \lambda$. Note que $\Gamma'$ é um caminho contínuo que conecta $\Gamma(1)$ a $W^{-}$. Com isso, a justaposição $\Gamma'*\Gamma$ é um caminho conectando $B$ a $W^{-}$ e a justaposição $\alpha * \Gamma'*\Gamma: \intervalo \to \gruposimpleticonaodegenerado{-}$ é um caminho contínuo conectando $A$ a $W^{-}$. Com uma construção análoga a do item anterior se pode-se concluir que qualquer $A' \in \gruposimpleticonaodegenerado{-}$ pode ser conectado a $A$ por um caminho contínuo que passa por $W^{-}$. Portanto, $\gruposimpleticonaodegenerado{-}$ é conexo por caminhos, logo é conexo.
		\end{enumerate}
		
		Afirmo que $\gruposimpleticonaodegenerado{\pm}$ são abertos. De fato, defina $g:\gruposimpletico{2n}\to \reta$ por $g(A) = \det(A-Id)$. Como o determinante é uma função contínua, então $g$ é contínua. Suponha que $A \in \gruposimpleticonaodegenerado{+}$. Então $g(A)>0$ e pode-se afirmar que existe um intervalo $(a,b) \subset \real{+}$ onde $g(A)\in (a,b)$. Afirmo que $g^{-1}((a,b))\subset \gruposimpleticonaodegenerado{+}$.  De fato, supondo que $g^{-1}((a,b)) \varsubsetneq \gruposimpleticonaodegenerado{+}$, então existe $A' \in g^{-1}((a,b))$ tal que $g(A')\leq 0$, o que contradiz o fato de que $(a,b) \subset \real{+}$. Portanto $g^{-1}((a,b)) \subset \gruposimpleticonaodegenerado{+}$ e $A$ é um ponto interno de $\gruposimpleticonaodegenerado{+}$, logo esse conjunto é um aberto. Um argumento análogo mostra que $\gruposimpleticonaodegenerado{-}$ é aberto.
	\end{prova}
	
	Finalmente, a Proposição \ref{proposicao_desconexidade_grupo_simpletico_nao_degenerado} será usada para mostrar que as inclusões $\gruposimpleticonaodegenerado{\pm} \hookrightarrow \gruposimpletico{2n}$ induzem os homomorfismos triviais nos grupos fundamentais. Para isso, será utilizado o seguinte lema, o qual está demonstrado em \cite{audi_floer_homology}.
	
	\begin{lema}\label{lema_levantamento_rho}
		Existem duas aplicações contínuas $\alpha^{\pm}_{\rho}: \gruposimpleticonaodegenerado{\pm}\to \reta$ que tornam o seguinte diagrama comutativo.
		$$
		\xymatrix{
			& & \real{}\ar[d]\ar[d]^{\text{exp}}
			\\
			\gruposimpleticonaodegenerado{\pm}\ar[urr]^{\alpha^{\pm}_{\rho}} \ar[r]_{i^{\pm}} & \gruposimpletico{2n} \ar[r]_{\rho} & S^{1}
		}
		$$	
		
	\end{lema}
	
	\begin{proposicao}\label{proposicao_homomorfismo_trivial_grupos_simpletico}
		As inclusões $\gruposimpleticonaodegenerado{+} \hookrightarrow \gruposimpletico{2n}$ e $\gruposimpleticonaodegenerado{-} \hookrightarrow \gruposimpletico{2n}$ induzem os homomorfismos triviais entre os grupos fundamentais.
	\end{proposicao}
	\begin{prova}
		Pelo Lema \ref{lema_levantamento_rho} pode-se afirmar que $\rho\circ i^{+} = \exp\circ \alpha^{+}_{\rho}$. Logo $\rho_{*}\circ i^{+}_{*} = \exp_{*}\circ (\alpha^{+}_{\rho})_{*}$. Usando o fato de que $\grupofundamental{\reta}=\{\classe{0}\}$ e tomando $\classe{\gamma}\in \grupofundamental{\gruposimpleticonaodegenerado{+}}$ verifica-se que 
		$$
		(\rho_{*}\circ i^{+}_{*})(\classe{\gamma}) = (\exp_{*}\circ (\alpha^{+}_{\rho})_{*})(\classe{\gamma}) = \exp_{*}(\classe{0})=\classe{0}.
		$$
		
		Do Item \ref{item_determinante_rho} do Teorema \ref{teorema_aplicacao_rho} tem-se que sua induzida $\rho_{*}$ é um isomorfismo. Logo, $	i^{+}_{*}(\classe{\gamma})=\classe{0}$. Como $\classe{\gamma}\in \gruposimpleticonaodegenerado{+}$ é arbitrário, então $i^{+}_{*} = 0$.
		
		Com um argumento análogo pode-se mostrar que a inclusão $\gruposimpleticonaodegenerado{-}\hookrightarrow \gruposimpletico{2n}$ induz o homomorfismo trivial no grupo fundamental.
	\end{prova}
		
	
	\section{Índice de Maslov $\mu : \dominioMaslov \to \inteiros$}\label{secao_indice_maslov}
	
	\begin{definicao}
		(Índice de Maslov) 
		Seja 
		$$
		\dominioMaslov=\{ \psi \in \caminhos(\gruposimpletico{2n}): \psi(0)=Id \;\; \text{e}\;\; \psi(1)\in \gruposimpleticonaodegenerado{*}\}.
		$$
		Para um dado caminho $\psi \in \dominioMaslov$ fixe um caminho $\gamma \in \caminhos(\gruposimpleticonaodegenerado{*})$ tal que $\psi(1)=\gamma(0)$ e $\gamma(1)=W^{\pm}$. Defina o prolongamento de $\psi$ por $\Psi = \psi*\gamma \in   \dominioMaslov$. Seja $\alpha_{\Psi}: \intervalo \to \reta$ o levantamento do caminho $\rho_{\Psi} = \rho \circ \Psi:\intervalo\to \circulo$ que faz o diagrama abaixo comutar
		$$
		\xymatrix{
			& & \real{}\ar[d]\ar[d]^{\text{exp}}
			\\
			[0,1 ]\ar[urr]^{\alpha_{\Psi}} \ar[r]_{\Psi} & \gruposimpleticonaodegenerado{*} \ar[r]_{\rho} & S^{1}
		}
		$$	
		Então o grau de $\rho_{\Psi}$ é dado por $\deg(\rho_{\Psi}) = (\alpha_{\Psi}(1)-\alpha_{\Psi}(0))/2\pi$. 
		
		O índice de Maslov de $\psi$ é definido por
		$$
		\mu(\psi)= 2\deg(\rho_{\Psi}) = 2\deg(\rho_{\psi}) + 2\deg(\rho_{\gamma}).
		$$
		\begin{figure}[!h]
			\centering
			\includegraphics[width=10cm,height=3.5cm]{imagem/caminhos_especiais_em_Sp2n.pdf}
			\caption{Um caminho $\psi$ em $\caminhosespeciais{\gruposimpletico{2n}}$ e seu prolongamento $\Psi$.}
		\end{figure}
	\end{definicao}
	
	\begin{observacao}
		Sejam $\psi^{+}, \psi^{-} \in  \caminhosespeciais{\gruposimpletico{2n}}$ caminhos tais que $\psi^{+}(1)=W^{+}$ e $\psi^{-}(1)=W^{-}$. Com isso, tem-se que $\rho_{\psi^{+}}(1) =\rho(W^{+}) = (-1)^{n}$ e $\rho_{\psi^{-}}(1) =\rho(W^{-}) = (-1)^{n-1}$, o que implica em $\alpha_{\psi^{+}}(1)-\alpha_{\psi^{+}}(0) = k^{+}\pi$ e $\alpha_{\psi^{-}}(1)-\alpha_{\psi^{-}}(0) = k^{-}\pi$, para algum $k^{+}, k^{-}\in \inteiros$.
	\end{observacao}
	
	Os dois próximos resultados são importantes propriedades do índice de Maslov que são úteis no Teorema \ref{teorema_indice_maslov}.

	\begin{lema}\label{lema_independencia_prolongamento}
		Sejam $\psi,\psi' \in  \dominioMaslov$. Então $\psi \sim \psi'$ se, e somente se, $\Psi \sim \Psi'$, onde $\Psi$ e $\Psi'$ são prolongamentos de $\psi$ e $\psi'$, respectivamente. Além disso, o índice de Maslov de $\psi$ não depende da escolha do prolongamento $\Psi$.
	\end{lema}
	\begin{prova}
		Suponha que $\psi \sim \psi'$. Pela Proposição \ref{proposicao_homomorfismo_trivial_grupos_simpletico} tem-se que $\gamma\sim \gamma'$. Portanto, $\Psi = \psi*\gamma \sim \psi'*\gamma' = \Psi'$, o que é demonstrado no Lema \ref{lema_compatibilidade_produto_caminhos}. 
		
		Por outro lado, suponha que $H: \intervalo \times \intervalo \to \gruposimpletico{2n}$ seja uma homotopia entre $\Psi $ e $\Psi'$ tal que $H(t,0) = \Psi(t)$ e $H(t,1) = \Psi'(t)$. Logo $\Psi(t) = \psi(2t)$ e $\Psi'(t) = \psi'(2t)$ para $0 \leq t \leq 1/2$. Seja $h: \intervalo \times \intervalo \to \gruposimpleticonaodegenerado{*}$ definida por $h(t,s) = H(t/2, s)$. Então  $h(t,0) = H(t/2, 0) =  \psi(2t)$ e $h(t,1)=H(t/2,1) = \psi'(2t)$. Portanto $h$ é uma homotopia entre $\psi$ e $\psi'$.
		
		Sejam $\Psi=\psi*\gamma$ e $\Psi''=\psi*\gamma''$ são dois prolangamentos de $\psi$, onde $\gamma, \gamma''\in \caminhos(\gruposimpleticonaodegenerado{\pm})$. A Proposição \ref{proposicao_homomorfismo_trivial_grupos_simpletico} permite afirmar que $\gamma\sim \gamma''$. Com isso, $\Psi\sim \Psi''$ e $2\deg(\rho_{\Psi})=2\deg(\rho_{\Psi'})=\mu(\psi)$. Logo $\mu(\psi)$ não depende do prolongamento escolhido.
	\end{prova}
	
	\begin{observacao}
		O lema anterior mostra que o índice de Maslov de um caminho $\psi\in \dominioMaslov$ está bem definido.
	\end{observacao}
	
	Dado $\psi\in \caminhos(\gruposimpletico{2n})$, seja $\psi^{-1}$ tal que $\psi^{-1}(t) = (\psi(t))^{-1}$ para todo $t\in \reta$. Como a inversa de uma transformação simplética é uma transformação simplética, então $\psi^{-1}$ está bem definido e $\psi^{-1} \in \caminhos(\gruposimpletico{2n})$ .
	
	\begin{lema}\label{lema_inversa_caminho_especial}
		Se $\psi \in   \dominioMaslov$, então $\psi^{-1} \in  \dominioMaslov$. Além disso, $\rho(\psi^{-1}(t)) = (\rho(\psi(t)))^{-1}$.
	\end{lema}
	\begin{prova}
		Como $\psi(0) = Id$, então $\psi^{-1}(0) = (\psi(0))^{-1} = Id$. Sabe-se que, se $\lambda\in \espectrooperador{\psi(1)}$, então $1/\lambda\in \espectrooperador{\psi^{-1}(1)}$. Logo, $1\notin \espectrooperador{\psi^{-1}(1)}$ pois $1\notin \espectrooperador{\psi(1)}$. Pelo Item \ref{item_inversa_rho} do Teorema \ref{teorema_aplicacao_rho}, tem-se que $\rho(\psi^{-1}(t))= (\rho(\psi(t)))^{-1}$ para todo $t\in \intervalo$.
	\end{prova}
	
	
	\begin{teorema}\label{teorema_indice_maslov}
		Se $\psi \in \dominioMaslov$, então o índice de Maslov satisfaz as seguintes propriedades:
		\begin{enumerate}
			\item $\mu(\psi) \in \inteiros$.
			
			\item \textbf{(Naturalidade)}\label{item_naturalidade_maslov} Se $\phi\in \caminhossempontobase{\gruposimpletico{2n}}$, então o caminho $\phi\psi\phi^{-1}: \intervalo \to \gruposimpletico{2n}$ definido por $(\phi\psi\phi^{-1})(t) = \phi(t)\psi(t)\phi^{-1}(t)$ é um elemento de $\caminhosespeciais{\gruposimpletico{2n}}$ e $\mu(\phi\psi\phi^{-1}) = \mu(\psi)$.
			
			\item \textbf{(Homotopia)} \label{item_homotopia_caminhos_teorema_indice_maslov} Dois caminhos $\psi_{1}, \psi_{2}\in \dominioMaslov $ com $\psi_{1}(0) = \psi_{2}(0)$ e $\psi_{1}(1) = \psi_{2}(1)$ são homotópicos se, e somente se, $\mu(\psi_{1}) = \mu(\psi_{2})$.
			
			\item \textbf{(Nullidade)} se $\espectrooperador{\psi(t)}\cap \circulo = \emptyset$ para todo $t\neq 0$, então $\mu(\psi) = 0$.
			
			\item \textbf{(Produto)} Se $n=p+q$, então identifica-se $\gruposimpletico{2p}\times \gruposimpletico{2q}$ com um subgrupo de $\gruposimpletico{2n}$ e $\mu:\caminhosespeciais{\gruposimpletico{2p}\times \gruposimpletico{2q}} \to \inteiros$ é dado por $\mu((\psi_{p}, \psi_{q})) = \mu(\psi_{p})+\mu(\psi_{q})$.
			
			\item \textbf{(Determinante)} O sinal de $ \det(\psi(1)-Id)$ é $(-1)^{\mu(\psi)-n}$.
			
			\item \textbf{(Assinatura)}\label{item_assinatura_maslov} Se $S \in GL(2n)$ é uma matriz simétrica com a norma $||S|| < 2\pi$ e se $\psi(t) = exp(t\estruturacomplexa S)$, então 
			$$
			\mu(\psi) = Ind(S) - n,
			$$
			onde $Ind(S)$ é o número de auto-valores negativos de $S$ contatos com multiplicidade.
			
			\item \textbf{(Inversa)} $\mu(\psi^{-1}) = -\mu(\psi)$.
			
		\end{enumerate}
	\end{teorema}
	\begin{prova}
		
		Considere para os itens da demosntração $\psi \in \caminhos{(\gruposimpleticonaodegenerado{*})}$ e $\Psi=\psi * \gamma$ um prologamento de $\psi$.
		\begin{enumerate}
			\item Tem-se que $\rho_{\Psi}(0) = \rho(\Psi(0)) = 1$ e $\rho_{\Psi}(1) = \rho(\Psi(1)) = \rho(W^{\pm})$, onde $\rho(W^{+})=(-1)^{n}$ e $\rho(W^{-})=(-1)^{n-1}$, o que implica em $\rho(W^{\pm}) \in \{ -1,1\}$. Sabe-se que $\rho_{\Psi}(t) = \exp(i\alpha_{\Psi}(t))$. Afirmo que $\alpha_{\Psi}(0)=0$ e $\alpha_{\Psi}(1)=k\pi$. De fato, para $t=0$ é imediato que $\alpha_{\Psi}(0)=0$. Para $t=1$ tem-se que $\rho(W^{\pm}) = \pm 1$, o que implica que $\alpha_{\Psi}(1) = k\pi$, para algum $k\in \inteiros$. Logo $\mu(\psi) = 2\deg(\rho_{\Psi}) = (k\pi-0)/\pi \in \inteiros$.
			
			\item \textbf{(Naturalidade)}
			Como $\phi\psi\phi^{-1}$ é o produto de aplicações contínuas, então é uma aplicação contínua. Veja que $(\phi\psi\phi^{-1})(0) = \phi(0)\psi(0)\phi^{-1}(0) = \phi(0)(\phi(0))^{-1} = Id$. Defina $\psi'=\phi\psi\phi^{-1}$. Como $1\notin \espectrooperador{\psi(1)}$ e $\psi'(1)$ é uma matriz semelhante a $\psi(1)$, então $\espectrooperador{\psi'(1)}=\espectrooperador{\psi(1)}$. Logo $1\notin \espectrooperador{\psi'(1)}$ e $\psi'\in \caminhosespeciais{\gruposimpletico{2n}}$.
			
			Seja $\phi'\in \caminhos(\gruposimpletico{2n})$ tal que $\phi'(0) = \phi(1)$ e $\phi'(1) = Id$, e defina $\gamma'=\phi'\gamma\phi'^{-1}$. Afirmo que $\Psi'=\psi'*\gamma'$ é um prolongamento de $\psi'$. De fato, 
			$$
			\gamma'(0)=\phi'(0)\gamma(0)\phi'^{-1}(0) = \phi(1)\psi(1)\phi^{-1}(1) = \psi'(1)
			$$ 
			e 
			$$
			\gamma'(1)=\phi'(1)\gamma(1)\phi'^{-1}(1) = Id\gamma(1)Id = W^{\pm}.
			$$
			
			Pelo Item \ref{item_naturalidade_rho} do Teorema \ref{teorema_aplicacao_rho} tem-se que  $\rho(\phi(t)\psi(t)\phi^{-1}(t)) = \rho(\psi(t))$ para todo $t\in \intervalo$, logo $\rho_{\phi\psi\phi^{-1}} = \rho_{\psi}$. Com isso, tem-se que 
			$$
			\begin{aligned}
			\mu(\phi\psi\phi^{-1})
			&=\mu(\psi')
			\\
			&= 2\deg(\rho_{\Psi'})
			\\
			&= 2\deg(\rho_{\psi'}) + 2\deg(\rho_{\gamma'})
			\\
			&=2\deg(\rho_{\phi\psi\phi^{-1}}) + 2\deg(\rho_{\phi'\gamma\phi'^{-1}})
			\\
			&=2\deg(\rho_{\psi}) + 2\deg(\rho_{\gamma})
			\\
			&=2\deg(\rho_{\Psi})
			\\
			&= \mu(\psi).			
			\end{aligned}
			$$
			
			\begin{figure}[!h]
				\centering
				\includegraphics[width=13cm,height=6cm]{imagem/caminho_naturalidade.pdf}
				\caption{Os prolongamentos $\Psi=\psi*\gamma$ de $\psi$ e $\Psi'  = \psi'*\gamma'$ de $\psi'$.}
			\end{figure}
			
			\item \textbf{(Homotopia)} Sejam $\Psi_{1}, \Psi_{2}$ prolongamentos de $\psi_{1}, \psi_{2} \in \caminhosespeciais{\gruposimpletico{2n}}$. No Lema \ref{lema_independencia_prolongamento} é demonstrado que $\Psi_{1}\sim \Psi_{2}$ se, e somente se, $\psi_{1}\sim \psi_{2}$. Supondo que $\psi_{1}\sim \psi_{2}$, então $\mu(\psi_{1}) = 2\deg(\rho_{\Psi_{1}}) = 2\deg(\rho_{\Psi_{2}}) = \mu(\psi_{2})$. A recíproca $\mu(\psi_{1}) = \mu(\psi_{2})$ implica imediatamente que $\Psi_{1}\sim \Psi_{2}$, logo $\psi_{1}\sim \psi_{2}$.
			
			\item \textbf{(Nullidade)} Por definição tem-se que 
			$$
			\rho(\psi(t)) = (-1)^{m_{0}} \prod_{\lambda \in \sigma(\psi(t))\cap \complementar{\circulo}{\{\pm 1 \}  }}\lambda^{m_{+}(\lambda)}.
			$$
			
			Como  $\espectrooperador{\psi(t)}\cap \circulo = \emptyset$ para todo $t\neq 0$, então $	\rho(\psi(t))=\pm 1$. Resta avaliar os seguintes casos: $\espectrooperador{\psi(0)}\cap \circulo = \emptyset$ e $\espectrooperador{\psi(0)}\cap \circulo \neq \emptyset$. No primeiro caso tem-se que $	\rho(\psi(0))=\pm 1$, logo o caminho  $\rho\circ\psi$ não é sobrejetor, o que implica em $\deg(\rho_{\Psi}) = 0$ e $\mu(\psi) = 0$. No segundo caso tem-se que $	\rho(\psi(0))=e^{i\theta}$, para alguma constante $\theta \in \reta$ e, analogamente ao caso anterior, o caminho $\rho\circ\psi$ não é sobrejetor. Logo $\mu(\psi)=0$. 
			
			\item \textbf{(Produto)} Sabe-se que a identificação $\real{2n} \mapsto \real{2p}\times \real{2q}$ é um isomorfismo, o que induz o isomorfismo $\gruposimpletico{2n} \ni A \mapsto (A_{p}, A_{q}) \in \gruposimpletico{2p} \times \gruposimpletico{2q}$. Sejam $\psi=(\psi_{p}, \psi_{q}) \in \caminhosespeciais{\gruposimpletico{2p}\times \gruposimpletico{2q}}$, $\Psi=(\Psi_{p}, \Psi_{q})$ os respectivos prolongamentos de $\psi, \psi_{p}$ e $\psi_{q}$. Então pelo Item \ref{item_produto_rho} do Teorema \ref{teorema_aplicacao_rho} tem-se que $\rho(\Psi(t))=\rho(\Psi_{p}(t))\rho(\Psi_{q}(t))$ para todo $t\in \intervalo$. Supondo que $\rho(\Psi_{p}(t))=\exp(i\alpha_{\Psi_{p}}(t))$ e $\rho(\Psi_{q}(t))=\exp(i\alpha_{\Psi_{q}}(t))$, então $\rho(\Psi(t)) = \exp(i(\alpha_{\Psi_{p}}+\alpha_{\Psi_{q}})(t))$. Com isso, tem-se que 
			
			$$
			\begin{aligned}
			\mu(\psi) &= 2\deg(\rho_{\Psi}) 
			\\
			&= \frac{1}{2}\big(\alpha_{\Psi_{p}}(1)+\alpha_{\Psi_{q}}(1) -\alpha_{\Psi_{p}}(0)-\alpha_{\Psi_{q}}(0) \big) 
			\\
			&= 2\deg(\rho_{\Psi_{p}}) +2\deg(\rho_{\Psi_{p}}) 
			\\
			&=\mu(\psi_{p})+\mu(\psi_{q}).
			\end{aligned}
			$$
			
			\item \textbf{(Determinante)} Suponha que  $\det(\psi(1)- Id)>0$. Por definição $\Psi(1)=W^{+}$ e $\rho_{\Psi}(1) = \rho(W^{+}) = (-1)^{n}$. Se $\rho_{\Psi}(1) =1$, então $n\in 2\inteiros$. Sabendo que $\rho_{\Psi}(t) = \exp(i\alpha_{\Psi}(t))$, e dos fatos $\rho_{\Psi}(1) = 1$ e $\alpha_{\Psi}(0) = 0$, pode-se concluir que $\alpha_{\Psi}(1)=k\pi$ para algum $k \in 2\inteiros$. Logo $\mu(\psi) = (k\pi-0)/\pi \in 2\inteiros$ e $\mu(\psi)-n \in 2\inteiros$. Se $\rho_{\Psi}(1) =-1$, então $n\in 2\inteiros+1$. Com argumentos análogos ao caso anterior, tem-se que $\alpha_{\Psi}(1) = k\pi$ para algum $k\in 2\inteiros+1$. Logo $\mu(\psi) = (k\pi-0)/\pi \in 2\inteiros+1$ e $\mu(\psi)-n \in 2\inteiros$. Com argumentos análogos pode-se mostrar que, para $\det(\psi(1)- Id)<0$, tem-se  $\rho_{\Psi}(1) = \rho(W^{-}) = (-1)^{n-1}$ e portanto $\mu(\psi)-n \in 2\inteiros +1$.
			
			\item \textbf{(Assinatura)} Afirmo que, $\exp(\estruturacomplexa S) \in\gruposimpletico{2n}$. Para isso é suficiente mostrar que $(\exp(\estruturacomplexa S))^{t}\estruturacomplexa \exp(\estruturacomplexa S)= \estruturacomplexa$. Note que $(\estruturacomplexa S)^{t} 
			= S^{t}\estruturacomplexa^{t} = -S\estruturacomplexa$. Com isso, dado um inteiro $k\geq 0$ tem-se a identidade 
			$$
			\begin{aligned}
			\estruturacomplexa (\estruturacomplexa S)^{k} 
			&= \estruturacomplexa\underbrace{\estruturacomplexa  S\dots \estruturacomplexa S}_{k-vezes} 
			\\
			&= \estruturacomplexa^{2}  S\estruturacomplexa \dots \estruturacomplexa S \underbrace{(-\estruturacomplexa^{2})}_{Id} 
			\\
			&=\underbrace{ S\estruturacomplexa \dots  S \estruturacomplexa }_{k-vezes}\estruturacomplexa 
			\\
			&= (S\estruturacomplexa)^{k}\estruturacomplexa.
			\end{aligned}
			$$
			
			Além disso, note que $((\estruturacomplexa S)^{k})^{t} = (\estruturacomplexa S)^{t}\dots (\estruturacomplexa S)^{t} = (S^{t}\estruturacomplexa^{t} )^{k} = (-S\estruturacomplexa)^{k}$, e com isso
			$$
			\begin{aligned}
			\big(\exp(\estruturacomplexa S)\big)^{t}\estruturacomplexa\exp(\estruturacomplexa S)
			&=\Big(\sum_{j=0}^{\infty}\frac{ (\estruturacomplexa S)^{j}}{j!}\Big)^{t} \estruturacomplexa \sum_{k=0}^{\infty}\frac{(\estruturacomplexa S)^{k}}{k!}
			\\
			&=\sum_{j=0}^{\infty}\frac{\big( (\estruturacomplexa S)^{j}\big)^{t}}{j!}\estruturacomplexa \sum_{k=0}^{\infty}\frac{(\estruturacomplexa S)^{k}}{k!}
			\\
			&=\sum_{j=0}^{\infty}\frac{(-S\estruturacomplexa)^{j}}{j!} \sum_{k=0}^{\infty}\frac{\estruturacomplexa(\estruturacomplexa S)^{k}}{k!}
			\\
			&=\sum_{j=0}^{\infty}\frac{(-S\estruturacomplexa)^{j}}{j!} \sum_{k=0}^{\infty}\frac{(S\estruturacomplexa)^{k}}{k!}\estruturacomplexa
			\\
			&=\exp(-S\estruturacomplexa)\exp(S\estruturacomplexa)\estruturacomplexa
			\\
			&=\estruturacomplexa.
			\end{aligned}
			$$
			Portanto, $\exp(S\estruturacomplexa) \in \gruposimpletico{2n}$. 
			
			O seguinte fato será utilizado nessa em argumentos a seguir: Sejam $A \in \generalgroupcomplexo{2n}$ e $\epsilon>0$ tal que $\norma{A}<\epsilon$. Suponha $u$ um auto-vetor não-nulo de $A$ associado ao auto-valor $\lambda\in \espectrooperador{A}$. Então 
			$$
				|\lambda|=\normagrande{A\bigparenteses{\frac{u}{\norma{u}}} }\leq \sup_{\norma{v}=1 } \norma{Av} = \norma{A}<\epsilon.
			$$
				
			
			Seja $v$ um auto-vetor de $\estruturacomplexa S$ associado ao auto-valor $\lambda \in \espectrooperador{\estruturacomplexa S}$. Então 
			$$
			\exp(\estruturacomplexa S)v = e^{\lambda}v.
			$$
			
			Além disso, note que $\norma{\estruturacomplexa S}\leq \norma{\estruturacomplexa}\norma{S} = \norma{S} <2\pi$, o que implica em $|\lambda|<2\pi$, conforme mostrado anteriormente.
			
			Afirmo que $1\notin \espectrooperador{\exp(S\estruturacomplexa)}$. De fato, escreva $\lambda = a+ib\in \complexo{}$. Supondo $e^{\lambda} = 1$, tem-se que $a=0$ e $b \in 2\pi k$ para algum $k\in \inteiros$, o que contradiz o fato de que $|\lambda| <2\pi$. Portanto, $1\notin\espectrooperador{\exp(\estruturacomplexa S)}$ e $\exp(\estruturacomplexa S)\in \gruposimpleticonaodegenerado{*}$.
			
			Como $S$ é simétrica, existe uma matriz ortogonal $P \in \matrizortogonal{2n}$ tal que $P^{t}SP=\diag(a_{1},\dots,a_{2n})$ (tais resultados podem ser encontrados em \cite{hoffman_kunze}). Sejam $\mathcal{P}:[0,1]\to \matrizortogonal{2n}$ um caminho contínuo tal que $\mathcal{P}(0) = P$ e $\mathcal{P}(1)=Id$. 
			Então o caminho contínuo $\mathcal{P}^{t}S\mathcal{P}:\intervalo\to \gruposimpleticonaodegenerado{*}$, definido por $(\mathcal{P}^{t}S\mathcal{P})(s)=\mathcal{P}^{t}(s)S\mathcal{P}(s)$, conecta a matriz diagonal $P^{t}SP$ a $S$. Como $ ((\mathcal{P}^{t}S\mathcal{P})(s))^{t} =(\mathcal{P}^{t}(s)S\mathcal{P}(s))^{t} = \mathcal{P}^{t}(s)S\mathcal{P}(s) =(\mathcal{P}^{t}S\mathcal{P})(s)$, então $(\mathcal{P}^{t}S\mathcal{P})(s)$ é simétrica para todo $s\in \intervalo$, e com os argumentos análogos aos anteriores, pode-se afirmar que $\exp(\estruturacomplexa(\mathcal{P}^{t}S\mathcal{P})(s)) \in \gruposimpleticonaodegenerado{*}$. Como $\exp(\estruturacomplexa(\mathcal{P}^{t}S\mathcal{P})(s))$ é contínuo e não contém auto-valores iguais a 1, então $\exp(\estruturacomplexa(\mathcal{P}^{t}S\mathcal{P})(s))$ está na mesma componente conexa de $\gruposimpleticonaodegenerado{*}$ que a matriz $\exp(\estruturacomplexa S)$ para todo $s\in \intervalo$.
			
			Note que, o índice da matriz simétrica $(\mathcal{P}^{t}S\mathcal{P})(s)$ independe de s. De fato, 
			$$
			\det(Id-(\mathcal{P}^{t}S\mathcal{P})(s)) = 			\det(\mathcal{P}^{t}(Id-S)\mathcal{P}) = \det(S).
			$$ 
			Portanto, $\ind((\mathcal{P}^{t}S\mathcal{P})(s)) = \ind(S)$ para todo $s \in \reta$. Além disso pela Proposição \ref{proposicao_homomorfismo_trivial_grupos_simpletico}, pode-se assumir, pela invariância homotópica do índice de Maslov, que $S$ é a matriz diagonal $\diag(a_{1},\dots,a_{2n})$. Denotando as n-duplas $(a_{1}, \dots, a_{n})$ e $(a_{n+1}, \dots, a_{2n})$ por $a_{(1)}$ e $a_{(2)}$, respectivamente, e $a_{1}a_{n+1}\dots a_{n}a_{2n}$ por $a_{(1)}.a_{(2)}$, tem-se que
			$$
			\begin{aligned}
			(\estruturacomplexa S)^{2j} &=(-a_{(1)}.a_{(2)})^{j} Id
			\\
			(\estruturacomplexa S)^{2j+1} 
			&= (-a_{(1)}.a_{(2)})^{j} 
			\left(
			\begin{array}{cc}
			0 & a_{(2)}Id
			\\
			-a_{(1)}Id & 0  
			\end{array}
			\right),
			\end{aligned}
			$$
			onde $a_{(1)}Id=diag\{a_{1}, \dots, a_{n}\}$. O análogo vale para $a_{(2)}Id$ . Com isso, tem-se a exponencial
			$$
			\begin{aligned}
			\exp(\estruturacomplexa S) &= \sum_{j=0}^{\infty}\frac{(\estruturacomplexa S)^{2j}}{(2j)!} + \sum_{j=0}^{\infty} \frac{(\estruturacomplexa S)^{2j+1}}{(2j+1)!}
			\\
			&= \sum_{j=0}^{\infty}\frac{(-1)^{j}(a_{(1)}.a_{(2)})^{j}}{(2j)!}Id + \sum_{j=0}^{\infty} \frac{(-1)^{j}(a_{(1)}.a_{(2)})^{j}}{(2j+1)!}	\left(
			\begin{array}{cc}
			0 & a_{(2)}Id
			\\
			-a_{(1)}Id & 0  
			\end{array}
			\right)
			\\
			&= \cos\big((a_{(1)}.a_{(2)})^{1/2}\big)Id +\sin\big((a_{(1)}.a_{(2)})^{1/2}\big)
			\left(
			\begin{array}{cc}
			0 & a_{(2)}Id
			\\
			-a_{(1)}Id & 0  
			\end{array}
			\right).
			\end{aligned}
			$$


			Como $\norma{S}<2\pi$, então $0<|a_{j}|<2\pi$. Tomando uma aplicação contínua $f_{j}:\intervalo \to [0,\pi]$ tal que $f_{j}(0) = a_{j}$ e $f_{j}(1) = \pi$ para cada $1\leq j \leq 2n$ e definindo $D_{\pi} = \diag\{\pi, \dots, \pi\}$, pode-se afirmar que os caminhos $t\mapsto \exp(t\estruturacomplexa S)$ e $t\mapsto \exp(t\estruturacomplexa D_{\pi})$ são homotopicamente equivalentes, e portanto possuem o mesmo índice de Maslov. Esse fato  permite reduzir a análise a condição em que $|a_{j}|=\pi$ para $1\leq j\leq 2n$. Assim, tem-se os seguintes casos:
			
			\begin{enumerate}
				\item \textbf{Caso $a_{(1)}.a_{(2)}>0$ e $a_{(1)} = a_{(2)} = (\pi,\dots,\pi)$:}
				$$
				\exp(t\estruturacomplexa S) 
				= 
				\cos(t\pi)Id+ \sin(t\pi)\estruturacomplexa.
				$$
				
				Decompondo o espaço $\real{2n}$ no produto de n espaços $\real{2}$, a exponencial de $t\estruturacomplexa S$ é representada por n blocos diagonais de matrizes $2\times 2$.
				
				$$
				\begin{aligned}
					\exp(t\estruturacomplexa S) 
					&=
					\left(
					\begin{array}{ccccccc}
					\cos(t\pi) & \sin(t\pi) & 0 & 0 & 0 & \dots & 0
					\\
					-\sin(t\pi) & \cos(t\pi) & 0 & 0 & 0 & \dots & 0  
					\\
					0 & 0 & \cos(t\pi) & \sin(t\pi) & 0 & \dots & 0
					\\
					0 & 0 & -\sin(t\pi) & \cos(t\pi) & 0 & \dots & 0
					\\
					\vdots & \vdots & \vdots & \vdots & \vdots & \ddots & \vdots
					\\
					0 & 0 & 0 & 0 & 0 & \cos(t\pi) & \sin(t\pi)
					\\
					0 & 0 & 0 & 0 & 0 & -\sin(t\pi) & \cos(t\pi)
					\end{array}
					\right)
					\\
					&=
					\left(
					\begin{array}{ccccc}
					r(-t\pi) & 0 &\dots & 0 
					\\
					0 & r(-t\pi) &\dots & 0 
					\\
					\vdots & \vdots & \ddots & \vdots 
					\\
					0 & 0 & \dots & r(-t\pi)
					\end{array}
					\right),
				\end{aligned}
				$$
				onde $r(-t\pi)$ é a matriz de rotação no plano $\real{2}$ por um ângulo $-t\pi \in \reta$. Note que, $r(-t\pi) \in \gruposimpletico{2}$ é uma matriz otogonal, logo pode ser identificada com uma matriz unitária em $e^{-i\pi}\in U(1)$ (veja o Proposição \ref{proposicao_isomorfismo_U}). Pela propriedade do determinante de $\rho$ descrita no Item \ref{item_determinante_rho} e pela propriedade produto de $\rho$ descrita no Item \ref{item_produto_rho} do Teorema \ref{teorema_aplicacao_rho} tem-se que 
				$$
				\rho(\exp(t\estruturacomplexa S)) = (\rho(r(t\pi)))^{n} = e^{-itn\pi}.
				$$ 
				
				Defina $\psi \in \caminhosespeciais{\gruposimpletico{2n}}$ por $\psi(t) = \exp(t\estruturacomplexa S)$. Com isso, 
				$$
				\mu(\psi) = 2\deg(\rho_{\psi}) = \frac{2(-n\pi +0)}{2\pi}=-n.
				$$
				
				Como o número de auto-valores negativos de $\ind(S)$ é zero, então  
				$$
				\ind(S)-n=\mu(\psi).
				$$
				
				\item \textbf{Caso $a_{(1)}.a_{(2)}>0$ e $a_{(1)} = a_{(2)} = (-\pi,\dots,-\pi)$:} tem-se que
				$$
				\exp(t\estruturacomplexa S) = \cos(t\pi)Id- \sin(t\pi)\estruturacomplexa.
				$$
				
				Analogamente ao item anterior, 	decomponha o espaço $\real{2n}$ no produto de n espaços $\real{2}$. Então a exponencial de $t\estruturacomplexa S$ é representada por n blocos diagonais de matrizes $2\times 2$
				
				$$
				\exp(t\estruturacomplexa S) =
				\left(
				\begin{array}{ccccc}
				r(t\pi) & 0 &\dots & 0 
				\\
				0 & r(t\pi) &\dots & 0 
				\\
				\vdots & \vdots & \ddots & \vdots 
				\\
				0 & 0 & \dots & r(t\pi)
				\end{array}
				\right),
				$$
				onde $r(t\pi)$ é uma rotação no plano $\real{2}$ de um ângulo $t\pi\in \reta$.
				
				Com isso, identificando $r(t\pi)$ com a matriz $e^{i\pi }\in U(1)$, aplicando a propriedade do determinante e a propriedade do produto da aplicação $\rho$, tem-se que 
				$$
				\rho(\exp(t\estruturacomplexa S)) = (\rho(r(t\pi)))^{n} = e^{itn\pi}.
				$$
				
				
				Defina $\psi \in \caminhosespeciais{\gruposimpletico{2n}}$ por $\psi(t) = \exp(t\estruturacomplexa S)$. Com isso, $\mu(\psi) = n $, e como $\ind(S) = 2n$, então 
				$$ 
				\ind(S)-n = n = \mu(\psi).
				$$
				
				\item \textbf{Caso $a_{(1)}.a_{(2)}<0$ e $a_{(1)}=(\pi,\dots,\pi), \; a_{(2)} = -a_{(1)}$:} usando o fato que $\sin(it\epsilon)=\sinh(t\epsilon)$ e $\cos(it\epsilon)=\cosh(t\epsilon)$, tem-se que
				$$
				\begin{aligned}
					\exp(t\estruturacomplexa S) 
					&= \cosh(t\pi)Id- \sinh(t\pi)\left(
					\begin{array}{cc}
					0 & Id
					\\
					Id & 0  
					\end{array}
					\right)
					\\
					&=	
					\left(
					\begin{array}{ccccc}
					h(t\pi) & 0 &\dots & 0 
					\\
					0 & h(t\pi) &\dots & 0 
					\\
					\vdots & \vdots & \ddots & \vdots 
					\\
					0 & 0 & \dots & h(t\pi)
					\end{array}
					\right),
				\end{aligned}
				$$
				onde foi usado o fato que 
				$$
				h(t\pi) = 
				\left(
				\begin{array}{cc}
				\cosh(t\pi) & -\sinh(t\pi)
				\\
				-\sinh(t\pi) & \cosh(t\pi)
				\end{array}
				\right).
				$$
				
				Defina $\psi \in \caminhosespeciais{\gruposimpletico{2n}}$ por $\psi(t) = \exp(t\estruturacomplexa S)$. Note que, $\espectrooperador{\psi}= \espectrooperador{h(t\pi)} $, onde $\espectrooperador{h(t\pi)} = \{e^{t\pi} , e^{-t\pi}\}$. Como os auto-valores $\lambda(t)$ de $\psi(t)$ são números reais positivos, então $\lambda(t)$ estão no primeiro quadrante do plano complexo. Com isso, caminho $\rho_{\psi}:\intervalo\to \circulo$ não é sobrejetor em $\circulo$, e portanto $\deg(\rho_{\psi}) = 0$. Logo, $\mu(\psi) = 0$. Como $\ind(S) = n$, então
				$$
				\ind(S) - n = 0 =\mu(\psi).
				$$
			
			\end{enumerate}
	
			\item \textbf{(Inverso)} Pelo Lema \ref{lema_inversa_caminho_especial}, tem-se que $\psi^{-1} \in \caminhosespeciais{\gruposimpletico{2n}}$. Afirmo que $\Psi^{-1}= \psi^{-1}*\gamma^{-1} \in \caminhosespeciais{\gruposimpletico{2n}}$. De fato, se $\lambda(t)$ é um auto-valor de $\gamma(t)$, então $1/\lambda(t)$ é um auto-valore de $\gamma^{-1}(t)$. Como $1 \notin\espectrooperador{\gamma(t)}$, então $1 \notin\espectrooperador{\gamma^{-1}(t)}$
			para todo $t\in \intervalo$. Logo $\gamma^{-1}(\intervalo)\subset\caminhos(\gruposimpleticonaodegenerado{*})$. 
			
			Note que $\Psi(t)\Psi^{-1}(t) = \psi(t)\psi^{-1}(t) = Id$ para $t \in [0,1/2]$ e $\Psi(t)\Psi^{-1}(t) = \gamma(t)\gamma^{-1}(t) = Id$ para $t \in [1/2,1]$. Logo $\Psi(t)\Psi^{-1}(t) = Id$ para todo $t \in \intervalo$ e $\Psi\Psi^{-1} = Id$.
			
			Nem sempre pode-se afirmar que $\Psi^{-1}$ é um prolongamento de $\psi^{-1}$ pois $\Psi^{-1}(1)$ não necessariamente está em $\{W^{+},W^{-}\}$ em certas condições. No caso em que $\gamma(1)=W^{-}$, tem-se que $\Psi^{-1}(1) = (W^{-})^{-1}=W^{-}$. Logo $\Psi^{-1}$ é um prolongamento de $\psi^{-1}$. Contudo, o caso em que $\gamma(1)=W^{+}$ implica em $\psi^{-1}(1)= (\gamma(1))^{-1}=(W^{+})^{-1} \neq W^{+}$. Por isso é necessário prolongar $\Psi^{-1}$ até a matriz $W^{+}$.
			
			Seja $\gamma'\in \caminhos(\gruposimpleticonaodegenerado{+})$ tal que $\gamma'(0) = \gamma^{-1}(1)$ e $\gamma'(1)=W^{+}$. Então o caminho $\Psi'=\Psi^{-1}*\gamma'$ é um prolongamento de $\Psi^{-1}$. Note que $\Psi' = (\psi^{-1}*\gamma^{-1})*\gamma'$ é homotopicamente equivalente a $\Psi'' = \psi^{-1}*\gamma''$, para algum $\gamma''\in \caminhos(\gruposimpleticonaodegenerado{+})$, como ilustrado na Figura	 \ref{figura_prolongamento_homotopico}. Logo, $\Psi''$ é um prolongamento de $\psi^{-1}$ .  
			
			\begin{figure}[!h]
				\centering
				\includegraphics[width=9.5cm,height=7cm]{imagem/prolongamento_homotopico.pdf}
				\caption{O prolongamento $\Psi'$ e seu equivalente homotópico $\Psi''$.}
				\label{figura_prolongamento_homotopico}
			\end{figure}
			
			\begin{figure}[!h]
				\centering
				\includegraphics[width=7cm,height=4.2cm]{imagem/prolongamento_homotopico_2.pdf}
				\caption{O prolongamento $\Psi''$ e seu equivalente homotópico $\Psi^{-1}$.}
				\label{figura_prolongamento_homotopico_2}
			\end{figure}
			
		
			
			Note que $\Psi'\sim \Psi''$ e $\Psi''\sim \Psi^{-1}$ (veja a Figura \ref{figura_prolongamento_homotopico_2}). Pelo Item \ref{item_inversa_rho} do Teorema \ref{teorema_aplicacao_rho}, tem-se que $\rho_{\Psi^{-1}}(t) = (\rho_{\Psi}(t))^{-1}$. Escrevendo $\rho_{\Psi}(t) = \exp(i\alpha_{\Psi}(t))$, então $\rho_{\Psi^{-1}}(t) =(\exp(i\alpha_{\Psi}(t)))^{-1}= \exp(-i\alpha_{\Psi}(t))$. Logo 
			$$
			\deg(\rho_{\Psi^{-1}})=\frac{1}{2\pi}(-\alpha_{\Psi}(1)+\alpha_{\Psi}(0)) =  -\deg(\rho_{\Psi}).
			$$
			
			Usando o fato de que $\Psi''$ e $\Psi^{-1}$ possuem o mesmo grau pois são caminhos homotópicos, então
			$$
			\mu(\psi^{-1})=2\deg(\rho_{\Psi''}) = 2\deg(\rho_{\Psi^{-1}}) = -2\deg(\rho_{\Psi})= -\mu(\psi).
			$$
		\end{enumerate}
	\end{prova}
	
	\section{Índice de Conley-Zehnder $\iconleyabrev: \pontoscriticos{\funcionalH}\to \inteiros$}\label{secao_indice_conley}
	
	Na tentativa de se construir invariantes topológicos para uma 2n-variedade simplética $(M, \omega)$, e que contenham informações da estrutura simplética $\omega$, foi utilizada uma abordagem variacional, na qual atribui-se um número inteiro para cada ponto crítico do funcional de ação $\funcionalH$, onde $H: M\times \reta\to \reta$ é uma função Hamiltoniana depedente do tempo. Foi mostrado que as soluções 1-periódicas não-degeneradas e contráteis do sistema Hamiltonino $\dot{x} = \campohamiltoniano{x, t}$ são os pontos críticos do funcional de ação. Além disso, ao campo Hamiltoniano $\campohamiltonianoabrev$ existe um fluxo associado $\psi_{t}:M\to M$ que satisfaz $\dot{\psi_{t}} =\campohamiltonianoabrev\circ\psi_{t}$ e $\psi_{0}=Id$.
	
	Fixe $x $ uma solução 1-periódica não degenerada e contrátil do sistema Hamiltoniano. Escolha uma base simplética $B(0) = \{e_{1}(0),\dots,e_{2n}(0)\}$ de $\espacotangenteponto{x(0)}{M}$. Como $\psi_{t}$ é um simplectomorfismo para todo $t\in \reta$, então pode-se afirmar que $B(t) = \{e_{1}(t), \dots, e_{2n}(t)\}$ é uma base simplética de $\espacotangenteponto{x(t)}{M}$, onde $e_{j}(t) = D_{x(0)}\psi_{t}(e_{j}(0))$ para $1\leq j \leq 2n$ e $D_{x(0)}\psi_{t}: \espacotangenteponto{x(0)}{M}\to \espacotangenteponto{x(t)}{M}$. Esse fato segue da seguinte identidade:
	$$
	\begin{aligned}
	\omega_{x(t)}(e_{j}(t), e_{k}(t))
	&=
	\omega_{x(t)}(D_{x(0)}\psi_{t}(e_{j}(0)), D_{x(0)}\psi_{t}(e_{k}(0)))
	\\
	&=	(\psi^{*}_{t}\omega_{x(0)})(e_{j}(0), e_{k}(0))
	\\
	&=	\omega_{x(0)}(e_{j}(0), e_{k}(0)).
	\end{aligned}
	$$
	Com isso, a $\omega$-ortogonalidade de $e_{j}(t)$ e $e_{k}(t)$ decorre da $\omega$-ortogonalidade de $e_{j}(0)$ e $e_{k}(0)$. Portanto $B(t)$ é uma base simplética para todo $t\in \reta$. Aplicando Corolário o \ref{corolario_fibrado_sobre_Dn} para uma extensão $u:D^{2} \to M$ de $x$, essa família de bases simpléticas é $C^{\infty}$ em $t$ formando um frame simplético ao longo de $x$. 
	
	Seja a aplicação contínua $A_{x}:\reta \to \gruposimpletico{2n}$ definida por $A_{x}(t) = D_{x(0)}\psi_{t}$. Note que, $A_{x}(0) = D_{x(0)}\psi_{0}=Id$, pois $\psi_{0}=Id$. Como $x$ é uma solução não-degenerada, então $\det(D_{x(0)}\psi_{1}-Id) \neq 0$, o que implica em $A_{x}(1)\in \gruposimpleticonaodegenerado{*}$. Tome $v:D^{2}\to M$ uma outra extensão de $x$ ao disco $D^{2}$. Se $B'(t)= \{e'_{1}(t), \dots, e'_{2n}(t)\}$ é uma outra trivialização ao longo de $v$, então, colocando as extensãoes $u$ e $u'$ como na Figura \ref{figura_colagem_discos}, constrói-se uma aplicação $w:S^{2}\to M$ tal que $A_{x}(t), A'_{x}(t), A''(t)$ são trivializações ao longo de $u$, $v$ e $w$, respectivamente, sendo esta última devido a Suposição \ref{suposicao_trivializacao}. Portanto, os caminhos $A_{x}, A'_{x}$ e $ A''$ são homotópicos em $\dominioMaslov$, e pela propriedade de homotopia descrita no Item \ref{item_naturalidade_maslov}, tem-se que 
	$$
	\mu(A_{x}) = \mu(A'_{x}) = \mu(A''_{x}).
	$$

	Com isso, a aplicação $A: \pontoscriticos{\funcionalH} \to \caminhosespeciais{\gruposimpletico{2n}}$ dada por $A(x)(t) =A_{x}(t)= D_{x(0)}\psi_{t}$, onde $t \in \reta$, está bem definida.
	
	\begin{definicao}
		O índice de Conley-Zehnder $\iconleyabrev:\pontoscriticos{\funcionalH} \to \inteiros$ é dado por 
		$$
		\iconley{x} = \mu(A_{x}).
		$$
	\end{definicao}
	
	No caso em que a Hamiltoniana é uma função de Morse é natural questionar-se sobre a relação entre o índice de Conley-Zehnder e o índice de Morse. O seguinte resultado esclarece essa questão.
	
	Note que $\hessianaponto{x(0)}{H}$ é uma matriz simétrica. Além disso, pela Proposição \ref{proposicao_exponencial_hessiana} tem-se que $D_{x(0)}\psi_{1}= e^{-\estruturacomplexa \hessianaponto{x(0)}{H}}$. Segue da demonstração do Item \ref{item_assinatura_maslov} do Teorema \ref{teorema_indice_maslov} que se $\norma{\hessianaponto{x}{H}}<2\pi$, então 
	$$
	\exp(-\estruturacomplexa \hessianaponto{x}{H}) \in \gruposimpletico{2n}
	$$
	e
	$$ 
	A_{x}\in \caminhosespeciais{\gruposimpletico{2n}},\;\text{onde}\; A_{x}(t) = D_{x(0)}\psi_{t}.
	$$ 
	
	
	\begin{corolario}
		Sejam $(M,\omega)$ uma 2n-variedade simplética, $H:M\to \reta$ uma função Hamiltoniana autônoma e $x \in \pontoscriticos{H}$. Se $\norma{\hessianaponto{x}{H}}<2\pi$ (assumindo a $\cktopologia{2}$), então o índice de Conley-Zender $\iconley{x}$ de $x$ como uma solução periódica do sistema Hamiltoniano e seu índice $\ind(x)$ como ponto crítico da função $H$ são relacionados por
		$$
		\iconley{x} = \ind(x)-n.
		$$
	\end{corolario}
	\begin{prova}
		Segue da hipótese de que $x\in \pontoscriticos{H}$ é um ponto crítico não-degenerado. Note que, $S=\hessianaponto{x}{H}$ é uma matriz simétrica, e portanto é diagonalizável. Além disso, $\ind(S)$ é o número de auto-valores negativos de $S$ contados com a multiplicidade. Logo, $\ind(S)$ coincide com o índice de Morse $\ind(x)$ de $x$ e pelo Item \ref{item_assinatura_maslov} do Teorema \ref{teorema_indice_maslov} tem-se que tem-se que
		$$
		\iconley{x} = \ind(S)-n = \ind(x)-n.
		$$
	\end{prova}
	
	\chapter{Aplicações: Homologia de Floer e a Conjectura de Arnold}\label{capitulo_aplicacoes}
	
	A Teoria de Morse é uma estrátegia de se estudar a topologia de uma variedade diferenciável de dimensão finita através da análise do comportamento de funções suaves definidas na dada variedade, chamadas funções de Morse. Uma das principais ideias dessa teoria é construir um complexo graduado pelo índice de Morse, o qual recupera o homologia singular da variedade.
	
	Em 1966 em um congresso em Moscou, o matemático russo Vladimir Igorevich Arnold, nos estudos de sistemas Hamiltonianos e topologia do 2-toro, formulou uma conjectura a respeito do número de pontos fixos que um simplectomorfismo definido em uma variedade simplética possui e o número de pontos críticos de uma função de Morse, que deu origem a seguinte generalização:
	
	\textit{\textbf{(Conjectura de Arnold):} Sejam $(M, \omega)$ uma variedade simplética 2n-dimensioanl e $\psi : M \to M $ um simplectomorfismo Hamiltoniano, então $\psi$ deve ter tantos pontos fixos quanto uma função suave em $M$ deve ter de pontos críticos. Se os pontos fixos forem não-degenerados, então os número de pontos fixos é, no mínimo, o mesmo número de pontos críticos de uma função de Morse em $M$.}
	
	Foi no contexto de sua demonstração que nasceu a Homologia de Floer, quando na tentativa de se construir uma técnica análoga àquela utilizada no caso Morse-Witten deparou-se com algumas barreiras técnicas, tais como: variedades de dimensão infinita e a definição de um análogo ao índice de Morse para os pontos críticos não-degenerados de um funcional de ação definido nessas variedades. Floer utiliza um funcional de ação definido no espaço dos laços em M e estuda seus pontos críticos e linhas de fluxo de seu gradiente para a construção da homologia da variedade simplética. Ao final, mostra que essa homologia é isomorfa a homologia singular da variedade.
	
	Esse capítulo é constituído de duas partes: a primeira é dedicada a construção espaço das trajetórias e a apresentação de algumas de suas propriedades necessárias para a definição do complexo de Floer e seu operador bordo. A segunda parte é dedicada a construção desse complexo e a demonstração da conjectura de Arnold como aplicação direta da homologia de Floer. 
	
	Serão apresentados os principais resultados e o roteiro para as construções e demonstrações mais importantes. Para detalhes sobre os temas desse capítulo de aplicações, veja \cite{audi_floer_homology}, \cite{salamon_lecture},  \cite{salamon_conley_index} e \cite{salamon_zehnder}.
	
	
	\section{O Operador de Floer e o Espaço de Trajetórias}\label{apendice_variedades_banach}
	
	Sejam $H:M\times \reta\to \reta$ uma função Hamiltoniana dependente do tempo e $\energiafinitaM$ o conjunto das trajetórias de energia finita. Dadas duas soluções 1-periódicas do sistema Hamiltoniano $x^{-}, x^{+} \in \pontoscriticos{\funcionalH}$, o objetivo desta seção é apresentar uma ideia da demonstração de que o conjunto $\energiafinitaMconectante \subset \energiafinitaM$ das soluções da equação de Floer que conectam $x^{-}$ a $x^{+}$ é uma variedade de dimensão finita igual a diferença entre os índices de Conley-Zehnder de cada uma das soluções. A estratégia é a seguinte: pelo Teorema de Sard-Smale, o análogo ao Teorema de Sard para dimensões infinitas, conclui-se o conjunto dos valores regulares do operador de Floer é um subconjunto denso de seu contra-domínio. Logo, escolhendo uma perturbação da função Hamiltoniana, o conjunto $\energiafinitaMconectante$ é a pré-imagem de do valor regular de uma determinada aplicação, e portanto é uma variedade. Contudo, esse resultado tem como hipótese de que o diferencial do operador de Floer satisfaz as propriedades de um operador de Fredholm. Com isso, parte desse capítulo envolve uma análise do diferencial do operador de Floer $\diferencialfloerponto{u}$ para concluir que o mesmo satisfaz essas propriedades para todo $u\in \energiafinitaMconectanteHamiltoniana$ e que o índice de Fredholm desse operador é a diferença entre os índices de Conley-Zehnder dos pontos críticos $x^{-}$ e $x^{+}$
	$$
	\ind(\diferencialfloerponto{u})=\iconley{x^{-}}-\iconley{x^{+}}.
	$$
	A transversalidade do operador de Floer é necessária para a demonstração de que o conjunto  $\energiafinitaMconectanteHamiltoniana$ é uma variedade e sua dimensão coincide com o índice de Fredholm de $\diferencialfloerponto{u}$
	$$
	\dim(\energiafinitaMconectante)= \ind(\diferencialfloerponto{u}) = \iconley{x^{-}}-\iconley{x^{+}}.
	$$
	
	Por fim, são apresentados alguns resultados necessários para a demonstração da compacidade de $\energiafinitaMconectante$. Essa propriedade é essencial para a definição do operador bordo do complexo de Floer e, consequentemente, de sua homologia (veja a Seção \ref{secao_complexo_floer}). Tais resultados podem ser encontrados na Subseção \ref{secao_transversalidade_compacidade_M}.
	
	\subsection{Operador de Floer $\operadorFloer$ e a Variedade de Banach $\caminhosexponenciaisconectantespadrao$}\label{secao_variedade_banach_P}
	
	Uma variedade de Banach é uma generalização do conceito de variedade em dimensão finita para dimensão infinita. Mais precisamente, tem-se um espaço topológico em que cada ponto possui uma vizinhaça que é homeomorfa a um aberto de um espaço de Banach. Parte dos resultados desse trabalho está apoiado em operadores definidos em espaços de funções que, mesmo não sendo de dimensão finita, possuem uma estrutura de variedades de Banach, e com isso, pode-se definir uma noção de espaço tangente nos mesmos. A referência utilizada nessa subseção foi \cite{chang_morse_infinito}.
	
	\begin{definicao}
		(Variedade de Banach) Sejam $B$ um espaço de banach e $M$ um espaço topológico conexo de Hausdorff. $M$ é uma variedade de Banach de classe $C^{r}$ sobre $B$ se
		
		\begin{enumerate}
			\item existe uma família de abertos $\{U_{i}\}_{i\in I}$ de $M$ tal que $M=\bigcup_{i\in I}U_{i}$,
			
			\item existe uma família de homeomorfismos $ \phi_{i}:U_{i}\to \phi_{i}(U_{i})\subset X $ onde $i\in I$,
			
			\item $\phi_{j}\circ \phi_{i}^{-1}: \phi_{i}(U_{i}\cap U_{j}) \to \phi_{j}(U_{i}\cap U_{j}) $ é um difeomorfismo de classe $C^{r}$ (no sentido de derivadas de Fréchet) para todo $i,j \in I$.
		\end{enumerate}
		
		Cada par $(U_{i}, \phi_{i})$ é chamado carta de $M$ e  o conjunto $\{(U_{i}, \phi_{i})\}_{i\in I}$ e chamado atlas de $M$.
	\end{definicao}
	
	Seja $\operadorFloer :\aplicaoessuaves{\retacartesianocirculo}{M}\to \aplicaoessuaves{\retacartesianocirculo}{TM}$ o operador de Floer definido por 
	$$
	\operadorFloerParametro{u}(s,t)=\operadorFloerDefParametros{u}
	$$
	para todo $s\in \reta$ e $t\in \circulo$. Como os espaços das funções $\aplicaoessuaves{\retacartesianocirculo}{M}$ não são uma variedade de Banach, precisamos restringir o estudo a um domínio para o qual se possa definir uma estrutura de variedade de classe $C^{\infty}$. Mais especificamente,  será considerado um conjunto em que se possa construir um atlas em que os abertos das cartas serão bolas abertas de um espaço de Sobolev (mais detalhes estão na seção \ref{secao_espaco_sobolev}) e os homeomorfismos serão dados pela aplicação exponencial (veja o Apêndice \ref{apendice_variedade_riemanniana}).
	
	Considere as aplicações $x\in\lacocontrateis$ e $u \in \energiafinitaM$ dadas por
	$$
	\circulo \ni t \mapsto x(t) \in M\; \text{e}\;\;\retacartesianocirculo \ni (s,t)\mapsto u(s,t) \in M.
	$$ 
	As seções dos fibrados $\pullbackfibradotangenteM{x}$ e $\pullbackfibradotangenteM{u}$ são as seções de $TM$ restritas aos subconjuntos $x(\circulo )$ e  $u(\retacartesianocirculo )$ de $M$, respectivamente (veja a Observação \ref{observacao_pullback_fibrado_tangente}). Com isso, os espaços de Sobolev $\espacosobolevcontradominio{\retacartesianocirculo}{\pullbackfibradotangenteM{u}}$ e $\espacosobolevcontradominio{\circulo}{\pullbackfibradotangenteM{x}}$, também denotados por  $\espacosobolev{\pullbackfibradotangenteM{u}}$ e $\espacosobolev{\pullbackfibradotangenteM{x}}$, respectivamente, são definidos do seguinte modo: a variedade $M$ pode ser mergulhada em um espaço euclidiano $\real{m}$ para $m\geq 4n+1$ (veja o Teorema \ref{teorema_whitney}). Com isso, tem-se o mergulho $i:TM \hookrightarrow T\real{m} = \real{m} \times \real{m}$. Diz-se que $Y\in \espacosobolev{\pullbackfibradotangenteM{u}}$ se $Y$ é uma aplicação contínua $Y:\retacartesianocirculo\to TM$ tal que $Y(s,t)\in \espacotangenteponto{u(s,t)}{M}$ para cada $s,t $ e a 	composição $\pi_{2}\circ i\circ Y:\retacartesianocirculo\to \real{m}$ é um elemento de $\espacosobolevcontradominio{\retacartesianocirculo}{\real{m}}$, onde $\pi_{2}$ é a projeção do fibrado $T\real{m}$. Analogamente, diz-se que $X \in \espacosobolev{\pullbackfibradotangenteM{x}}$ se a composição $\pi_{2}\circ i\circ X:\circulo \to \real{m}$ é um elemento de $\espacosobolevcontradominio{\circulo}{\real{m}}$.
	
	\begin{observacao}
		Em alguns resultados, a análise realizada nesta seção fica simplificada se identificarmos os espaços de Sobolev $\espacosobolev{\pullbackfibradotangenteM{u}}$ e $\espacosobolev{\pullbackfibradotangenteM{x}}$ aos espaços de Sobolev $\espacosobolevcontradominio{\retacartesianocirculo}{\real{2n}}$ e  $\espacosobolevcontradominio{\circulo}{\real{2n}}$, respectivamente.
	\end{observacao}
	
	Os espaços de Banach  $\espacoLp{\pullbackfibradotangenteM{u}}$, $\espacoLp{\pullbackfibradotangenteM{x}}$, que são identificados respectivamente a $\espacoLpcontradominio{\retacartesianocirculo}{\real{2n}}$ e $\espacoLpcontradominio{\circulo}{\real{2n}}$, são definidos de forma análoga às construções anteriores.
	
	Sejam $D \subset TM$ uma vizinhança aberta da seção nula de $TM$ e $\exp:D\to M$ a aplicação exponencial definida no Apêndice \ref{apendice_variedade_riemanniana}.
	Fixado $x\in \lacocontrateis$, tem-se que $\Exp_{x}: \espacosobolev{x^{*}D} \to C^{0}(\circulo;M)$ dada por $\Exp_{x}(Y)(t) = \exp(x(t),Y(t))$ está bem-definida. Além disso, $\Exp_{x}(0)(t) = \exp(x(t),0) = x(t)$ para todo $t\in \circulo$, logo $\Exp_{x}(0)=x$.
	Tome a coleção $\{ (\espacosobolev{x^{*}D}, \Exp_{x}) \}_{x \in \lacocontrateis}$ e defina $\caminhosexponenciaisSobolev$ como sendo o conjunto das aplicações $\beta:\circulo\to M$ tais que $\beta=\Exp_{x}(Y)$ para algum $x\in \lacocontrateis$ e algum $Y \in \espacosobolev{x^{*}D}$.
	
	\begin{observacao}\label{observacao_inclusao_lacoescontrateis}
		Note que, dado $x\in \lacocontrateis$, tem-se que $x=\Exp_{x}(0)$. Logo $x\in \caminhosexponenciaisSobolev$ e $\lacocontrateis \subset \caminhosexponenciaisSobolev$.
	\end{observacao}
	
	\begin{proposicao}\label{proposicao_caminhos_variedade_banach}
		O conjunto $\caminhosexponenciaisSobolev$ munido com o atlas $\{(\espacosobolev{x^{*}D_{x}}, \Exp_{x})\}_{x \in \lacocontrateis }$, onde $D_{x} \subset TM$ é uma vizinhança aberta da seção nula de $TM$, é uma variedade de Banach de classe $C^{\infty}$. Além disso, 
		$$
		\lacocontrateis = C^{\infty}(\circulo;M) \subset  \caminhosexponenciaisSobolev\subset C^{0}(\circulo;M),
		$$
		onde cada um dos conjuntos da sequência é denso no conjunto que o contém.
	\end{proposicao}
	\begin{prova}
		Afirmo que $\caminhosexponenciaisSobolev$ satisfaz os axiomas das variedades de Banach. De fato,
		\begin{enumerate}
			\item Para todo $\beta \in \caminhosexponenciaisSobolev $ existem $x \in \lacocontrateis$ e $Y \in \espacosobolev{x^{*}D_{x}}$ tal que $\beta(t)=\Exp_{x}(Y)(t)$. A Proposição \ref{proposicao_difeomorfismo_exponencial} garante a existência de uma vizinhança aberta $D_{x} \subset TM$ da seção nula de $TM$ onde a aplicação exponencial é difeomorfa sobre a sua imagem. Denotando $\espacosobolev{x^{*}D_{x}}$ por $W_{x}$, tem-se que
			$$
			\caminhosexponenciaisSobolev = \bigcup_{x \in \lacocontrateis} W_{x}.
			$$
			
			\item Pela escolha de $D_{x}$ do item anterior e pela Proposição \ref{proposicao_difeomorfismo_exponencial}, pode-se afirmar que $\Exp_{x}:W_{x} \to \Exp_{x}(W_{x})$ é um difeomorfismo sobre sua imagem. Logo é um homeomorfismo.
			
			\item Como $\Exp_{x}$ e $\Exp_{y}$ são difeomorfismos sobre suas imagens, então a composição $\Exp_{x}^{-1} \circ \Exp_{y} :W_{x} \cap W_{y} \to W_{x} \cap W_{y}$ é um difeomorfismo. Além disso, como as aplicações exponenciais são de classe $C^{\infty}$, então a composição anterior também o é.
		\end{enumerate}
		
		Portanto $\{(W_{x}, \Exp_{x}) \}_{x \in \lacocontrateis}$ é um atlas suave e $\caminhosexponenciaisSobolev$ é uma variedade de Banach de classe $C^{\infty}$. 
		
		Por definição tem-se que $\caminhosexponenciaisSobolev\subset C^{0}(\circulo;M)$ e pela Observação \ref{observacao_inclusao_lacoescontrateis} $\aplicaoessuaves{\circulo}{M} \subset \caminhosexponenciaisSobolev$. Sabe-se que $\aplicaoessuaves{\circulo}{M}$ é denso em $C^{0}(\circulo;M)$. Logo, $\caminhosexponenciaisSobolev$ é denso em $C^{0}(\circulo;M)$.
	\end{prova}
	
	\begin{observacao}
		Uma consequência da construção anterior é que o espaço tangente em $x\in \caminhosexponenciaisSobolev$ dessa variedade de Banach é
		$$
		\espacotangenteponto{x}{\caminhosexponenciaisSobolev} = \espacosobolev{\pullbackfibradotangenteM{x}}.
		$$
	\end{observacao}
	
	\begin{definicao}\label{definicao_caminhos_decaimentos_exponenciais}
		(Caminhos de decaimentos exponenciais) Sejam $x^{-}, x^{+} \in \pontoscriticos{\funcionalH}$. O conjunto das aplicações $w \in \aplicaoessuavesreatacirculo$ tais que $\lim_{s \to \pm \infty} w(s,t) = x^{\pm}(t)$ para todo  $t\in \circulo $ e que 
		$$
		\normagrande{\derivadaparcial{w}{s}(s,t)} \leq Ke^{-\delta|s|} \;\; \text{e} \;\; \normagrande{\derivadaparcial{w}{t}(s,t) -X_{H}(w(s,t))} \leq Ke^{-\delta|s|}
		$$
		para $K,\delta \in \reta_{+}$ é chamado de conjunto dos caminhos com decaimento exponencial conectando $x^{-}$ a $x^{+}$ e é denotado por $\caminhosdecaimentoexponencialpadrao$.
	\end{definicao}
	
	\begin{definicao}
		(Exponenciais Conectantes) O conjunto das aplicações 
		$$
		\retacartesianocirculo \ni (s,t) \mapsto \exp(w(s,t), Y(s,t)) \in M,
		$$
		onde $w \in \caminhosdecaimentoexponencialpadrao$ e $Y\in \espacosobolev{\pullbackfibradotangenteM{w}}$ é chamado conjunto das exponenciais conectantes e é denotado por $\caminhosexponenciaisconectantespadrao$.
	\end{definicao}
	
	\begin{proposicao}\label{proposicao_variedade_banach}
		Sejam $x^{-}, x^{+} \in \pontoscriticos{\funcionalH}$. Então $\caminhosexponenciaisconectantespadrao$ é uma variedade de Banach para $p>2$.
	\end{proposicao}
	
	\begin{prova}
		Seja $w \in \caminhosdecaimentoexponencialpadrao$. Pelo Item \ref{item_tres_teorema_inclusao_sobolev} do Teorema \ref{teorema_inclusao_sobolev} tem-se que a inclusão $\espacosobolevcontradominio{\retacartesianocirculo}{\real{2n}} \hookrightarrow \espacoLpGeral{\infty}{\retacartesianocirculo;\real{2n}}$ é um operador contínuo para $p>2$. Com isso, existe uma constante $K\in \reta_{+}$ tal que 
		$$
		\norma{Y}_{L^{\infty}}\leq (K+1)\normaWp{Y}\;\forall Y\in \espacosobolevcontradominio{\retacartesianocirculo}{\real{2n}}.
		$$
		Note que 
		$$
		\norma{Y(s,t)}\leq \sup_{(s,t)\in \retacartesianocirculo}\norma{Y(s,t)} = \norma{Y}_{L^{\infty}}.
		$$
		Seja $\epsilon\in \reta_{+}$ um raio da bola aberta $B_{\epsilon}(0)$ centrada em $0\in \espacotangenteponto{w(s,t)}{M}$ tal que valha o difeomorfismo da Proposição \ref{proposicao_difeomorfismo_exponencial}. Tome  $0<r_{w}< \epsilon/(K+1)$. Logo, para todo $Y$ no disco $D_{w}=\{Y \in \espacosobolev{\pullbackfibradotangenteM{w}}: \normaWp{Y}<r_{w} \}$ tem-se que
		$$
		\norma{Y(s,t)} \leq (K+1)\normaWp{Y} < \epsilon.
		$$
		Com isso, $r_{w}$ é um raio tal que a aplicação 
		$$
		D_{w}\ni Y\mapsto \Exp_{w}(Y) \in \caminhosexponenciaisconectantespadrao
		$$ 
		é um difeomorfismo sobre sua imagem. 
		
		Com argumentos análogos aos utilizandos na demonstração da Proposição \ref{proposicao_caminhos_variedade_banach}, pode-se afirmar que $\{(D_{w}, \Exp_{w})\}_{w\in \caminhosdecaimentoexponencialpadrao} $ é um atlas de $\caminhosexponenciaisconectantespadrao$ e este conjunto é uma variedade de Banach.
	\end{prova}
	
	Seja $(\fibradocaminhosexponenciais, \caminhosexponenciaisconectantespadrao, \pi_{\mathcal{E}})$ um fibrado vetorial cuja fibra em $u \in \caminhosexponenciaisconectantespadrao$ é $\espacoLp{\pullbackfibradotangenteM{u}}$.
	
	\begin{definicao}\label{definicao_operador_floer}
		(Operador de Floer) Sejam $x^{-}, x^{+}\in \pontoscriticos{\funcionalH}$. O operador de Floer é a seção $s_{\operadorFloer}: \caminhosexponenciaisconectantesabrev \to \fibradocaminhosexponenciais$ definida por
		$$
		s_{\operadorFloer}(u)= \bigparenteses{u,\operadorFloerParametro{u}}.
		$$
	\end{definicao}
	
	\begin{observacao}
		Daqui em diante será utilizada a identificação da seção $s_{\operadorFloer}$ com o operador diferencial $\operadorFloer=\partial_{s}+J\partial_{t}-J\campohamiltonianoabrev$ (veja o Apêndice \ref{apendice_fibrados}). Com essa identificação o operador de Floer pode ser visto como um elemento de $\campossuaves{\caminhosexponenciaisconectantespadrao}$.
	\end{observacao}
	
	Dado $\exp(w,Y) \in \caminhosexponenciaisconectantesabrev$, pode-se mostrar que, identificando a fibra $\espacoLp{\pullbackfibradotangenteM{u}}$ com $\espacoLp{\retacartesianocirculo;\real{2n}}$, tem-se que $\operadorFloerParametro{\exp(w,Y)}\in \espacoLp{\retacartesianocirculo;\real{2n}}$ (veja \cite{audi_floer_homology}). Logo, o operador de Floer esta bem-definido.
	
	\subsection{O Diferencial $\diferencialfloer$}\label{secao_operador_floer}
	
	Seja $w \in \caminhosexponenciaisconectantespadrao$ e tome $Y $ um vetor da fibra $ \espacoLp{\retacartesianocirculo;\real{2n}}$. Então $Y$ pode ser visto como uma aplicação suave $Y:\retacartesianocirculo\to TM$ tal que $Y(s,t)\in \espacotangenteponto{w(s,t)}{M}$. Além disso, existe uma vizinhança $V$ de $w$ em $\caminhosexponenciaisconectantespadrao$ onde todo $w' \in V$ é dado unicamente por $w'=\exp(w, Y')$ para algum $Y'\in \espacoLp{\pullbackfibradotangenteM{w}}$. Como $M$ é uma variedade Riemanniana, então o diferencial de $\diferencialfloerponto{u}$ avaliado no vetor $Y$ é dado via o transporte paralelo de $\operadorFloerParametro{u}$ ao longo da geodésica $\tau \mapsto \exp_{w}(tY)$ por 
	$$
	\diferencialfloerponto{w}(Y) = \nabla_{s}Y + J_{w}\nabla_{t}Y + dJ_{w}(Y)(Y - X_{H}(w)) - J_{w}d_{w}X_{H}(Y),
	$$
	onde $\nabla:\campossuaves{M}\times \campossuaves{M} \to \campossuaves{M}$ é a conexão afim. Mais detalhes dessa descrição podem ser encontrados em \cite{salamon_lecture}. 
	
	Contudo, analisar as propriedades desse operador com conexões afins adiciona complexidade ao estudo, e por isso nesse texto é adotada uma abordagem via mergulho de variedades em espaços euclidianos.
	
	Pelo Teorema $\ref{teorema_whitney}$, a variedade $M$ pode ser mergulhada em $\real{m}$ para $m\geq 4n+1$. Fixando $m$, tem-se o mergulho de $i: TM \hookrightarrow \real{m}\times\real{m}$, e com isso, os campos tangentes a $M$ ao longo de todo $u \in \caminhosexponenciaisconectantespadrao$ podem ser vistos como vetores em $\real{m}$ da seguinte forma
	$$
	\begin{aligned}
	\pi_{2}\circ i : TM &\to \real{m}
	\\
	(x, Y(x)) &\mapsto Y(x),
	\end{aligned}
	$$
	onde $\pi_{2}:\real{m}\times \real{m}\to \real{m} 
	$ é definida por $\pi_{2}(a,b) = b$.
	
	Logo, o operador de Floer $\operadorFloer$ pode ser visto como uma aplicação $\operadorFloer: \caminhosexponenciaisconectantespadrao\to \espacoLpcontradominio{\retacartesianocirculo}{\real{m}}$. Essa identificação permite que se calcule o diferencial de $\operadorFloer$ de forma análoga ao que se faz em análise em espaços euclidianos, pois $u(s,t)$ pode ser visto como elemento de $\real{m}$ para todo $(s,t)\in \retacartesianocirculo$ e todo $u\in \caminhosexponenciaisconectantespadrao$.
	
	Fixe $u \in \caminhosexponenciaisconectantespadrao$ e tome uma carta $(D_{w}, \Exp_{w})$ em $u$, onde $D_{w}=\{Y \in \espacosobolev{\pullbackfibradotangenteM{w}}: \normaWp{Y}<r_{w} \}$ (veja a Proposição \ref{proposicao_variedade_banach}). Com isso, todo $u_{0} \in D_{w}$ pode ser escrito como $u_{0}=\Exp_{w}(Y)$ para algum $Y\in \espacosobolev{\pullbackfibradotangenteM{w}}$. Sejam $\epsilon>0$ e o caminho $\tilde{u}:(-\epsilon, \epsilon)\to \caminhosexponenciaisconectantespadrao$ definido por $\tilde{u}(\tau)=\Exp_{w}((1+\tau )Y)$. Usando os fatos que 
	$$
	\tilde{u}(0) = u \;\;\text{e}\;\; \lim\limits_{\tau\to 0}\frac{\tilde{u}(\tau)-u}{\tau}=\derivada{\tilde{u}}{\tau}(0)=Y
	$$ 
	(veja o Apêndice \ref{apendice_variedade_riemanniana}), o diferencial diferencial de $\operadorFloer$ em $u \in \caminhosexponenciaisconectantespadrao$ é dado por
	$$
	\begin{aligned}
	\diferencialfloerponto{u}(Y)
	&=	\lim\limits_{\tau\to 0}\frac{\operadorFloerParametro{\tilde{u}(\tau)} - \operadorFloerParametro{u}}{\tau}
	\\
	&=\lim\limits_{\tau\to 0}\frac{1}{\tau}
	\bigparenteses{
		\derivadaparcial{\tilde{u}}{s} + J_{\tilde{u}}\derivadaparcial{\tilde{u}}{t}-J_{\tilde{u}}\campohamiltoniano{\tilde{u}}
		-\derivadaparcial{u}{s} - J_{u}\derivadaparcial{u}{t} + J_{u}\campohamiltoniano{u}
	}
	\\
	&=
	\lim\limits_{\tau\to 0}\frac{1}{\tau}
	\bigparenteses{
		\derivadaparcial{(\tilde{u}-u)}{s}
		+(J_{\tilde{u}}-J_{u})\derivadaparcial{\tilde{u}}{t}
		+J_{u}\Big(\derivadaparcial{\tilde{u}}{t}-\derivadaparcial{u}{t}\Big)
		\\
		&\;\;\;\;\;\;\;\;\;\;\;\;\;\;\;-J_{\tilde{u}}\campohamiltoniano{\tilde{u}}+J_{u}\campohamiltoniano{\tilde{u}}
		-J_{u}(\campohamiltoniano{\tilde{u}}-\campohamiltoniano{u})}
	\\
	&=\lim\limits_{\tau\to 0}
	\bigparenteses{
		\derivadaparcial{}{s}\bigparenteses{\frac{\tilde{u}-u}{\tau}}
		+\bigparenteses{\frac{J_{\tilde{u}}-J_{u}}{\tau}}\derivadaparcial{\tilde{u}}{t} + J_{u}\derivadaparcial{}{t}\bigparenteses{\frac{\tilde{u}-u}{\tau}}
		\\&\;\;\;\;\;\;\;\;\;\;\;\;\;\;\;-\bigparenteses{\frac{J_{\tilde{u}}-J_{u}}{\tau}}\campohamiltoniano{\tilde{u}}-J_{u}\bigparenteses{\frac{\campohamiltoniano{\tilde{u}}-\campohamiltoniano{u} }{\tau}}
	}
	\\
	&=\derivadaparcial{Y}{s}+(d_{u}J)(Y)\derivadaparcial{u}{t}+J_{u}\derivadaparcial{Y}{t} -(d_{u}J)(Y)\campohamiltoniano{u}-J_{u}(d_{u}\campohamiltonianoabrev)(Y)
	\\
	&=\bigparenteses{\derivadaparcial{}{s}+J_{u}\derivadaparcial{}{t}}(Y)+(d_{u}J)(Y)\bigparenteses{ \derivadaparcial{u}{t} - \campohamiltoniano{u}}-J_{u}(d_{u}\campohamiltonianoabrev)(Y).
	\end{aligned}
	$$
	
	Portanto, o diferencial $\diferencialfloerponto{u}: \espacosobolev{\pullbackfibradotangenteM{u}}\to \espacoLp{\pullbackfibradotangenteM{u}}$ é o operador
	$$
		\diferencialfloerponto{u} =  \derivadaparcial{}{s} + J_{u}\derivadaparcial {}{t} + dJ_{u}(.)\bigparenteses{\derivadaparcial{u}{t} - X_{H}(u)} - J_{u} \diferencialhamiltoniano{u}.
	$$
		
	Em \cite{audi_floer_homology} é demonstrado que $\diferencialfloerponto{u}(\espacosobolev{\pullbackfibradotangenteM{u}})\subset \espacoLp{\pullbackfibradotangenteM{u}}$, e portanto o diferencial de $\operadorFloer$  está bem definido.
	
	\begin{observacao}
		Note que, ao se mergulhar a variedade $M$ em $\real{m}$, a conexão afim coincide com a operação de diferenciação usual, isto é,  $\nabla=\partial$. Assim, identificando $\nabla_{s}$ e $\nabla_{t}$ com os operadores de diferenciação $\derivadaparcial{}{s}$ e $\derivadaparcial{}{t}$, respectivamente, e voltando na equação de Floer escrita em termos da conexão afim, pode-se verificar que ela coincide com a equação obtida via mergulho.
	\end{observacao}
	
	O seguinte teorema e a sua observação caracterizam o comportamento assintótico das soluções da equação de Floer.
	
	\begin{teorema}\label{teorema_decaimento_exponencial_solucao_floer}
		Se $Y \in \ker(\diferencialfloerponto{u})$ é uma aplicação de classe $C^{2}$ e $u \in \energiafinitaM$, então 
		$$
		\lim\limits_{a\to \infty}\int_{-a}^{a}\int_{\circulo}\norma{Y(s,t)}^{2} = \infty \;\;\text{ou}\;\; \norma{Y(s,t)} \leq Ce^{-\delta|s|},
		$$
		onde $\delta$ e $C>0$ são constantes.
	\end{teorema}
	
	\begin{observacao}\label{observacao_decaimento_exponencial_solucao_floer}
		Se $u \in \energiafinitaM$ é uma solução da equação de Floer, então $Y=\derivadaparcial{u}{s}$ decai exponencialmente, e em particular, $Y \in \espacoLpcontradominio{\retacartesianocirculo}{\real{2n}}$ para cada $p>1$. De fato, 
		$$
		\int_{\retacartesianocirculo}\normagrande{Y(s,t)}^{2} = \int_{\retacartesianocirculo}\normagrande{\derivadaparcial{u}{s}(s,t) }^{2}=E(u)<\infty.
		$$
		Portanto, pelo teorema anterior, as soluções da equação de Floer possuem decaimento exponencial. Além disso, 
		$$
		\normaLpdefinicao{\norma{Y}}{\retacartesianocirculo} \leq C\bigparenteses{\int_{\retacartesianocirculo}e^{-\delta|s|p}dsdt}^{1/p} = 2\pi C\bigparenteses{\int_{\reta}e^{-\delta|s|p}ds }^{1/p}<\infty.
		$$ 
		Logo $Y \in \espacoLpcontradominio{\retacartesianocirculo}{\real{2n}}$ para $p>1$.
	\end{observacao}
	Sejam $x^{-}, x^{+} \in \pontoscriticos{\funcionalH}$ e $\energiafinitaMconectante =\{u\in \energiafinitaM: \lim_{s\to \pm \infty}u(s) =x^{\pm}\}$.
	A seguinte proposição, que é consequência co Teorema \ref{teorema_decaimento_exponencial_solucao_floer} e da Observacao \ref{observacao_decaimento_exponencial_solucao_floer}, afirma que toda solução de energia finita $u \in \energiafinitaMconectante$ satisfaz as condições de decaimento exponencial e, além disso, pode ser escrita como $u=\Exp(w,Y)$ para algum $w\in \caminhosdecaimentoexponencialpadrao$ e algum $Y\in \espacosobolev{\pullbackfibradotangenteM{w}}$.
	
	\begin{proposicao}\label{proposicao_cadeia_caminhos_conectantes}
		$\energiafinitaMconectante \subset \caminhosdecaimentoexponencialpadrao \subset  \caminhosexponenciaisconectantespadrao$.
	\end{proposicao}
	
	
	A seguir o diferencial $\diferencialfloer$ será escrito como a soma de dois operadores lineares. Como $x^{-}, x^{+}$ são contráteis, existem $\tilde{x}^{-}:D^{2}\subset \complexo{} \to M$ uma extensão suave de $x^{-}$ ao disco unitário $D^{2}$ tal que  $\tilde{x}^{-}|_{\circulo} = x^{-}$ e $\tilde{x}^{+}:D^{2}\subset \complexo{} \to M$ é uma extensão de suave de $x^{+}$. Defina uma extensão contínua $\tilde{u}:S^{2}\to M$ do cilindro $u:\retacartesianocirculo\to M$ para a esfera $S^{2}$ como ilutrado na Figura \ref{figura_colagem_disco_cilindro}. Como $\tilde{u}(S^{2})$ é homotópico a $S^{2}$ e todo fibrado vetorial sobre $S^{2}$ é trivial, então $\pullbackfibradotangenteM{\tilde{u}}$ é um $2n$-fibrado vetorial trivial (veja o Apêndice \ref{apendice_fibrados}).
	
	\begin{figure}[!h]
		\centering
		\includegraphics[width=11.5cm,height=5.5cm]{imagem/colagem_disco_cilindro.pdf}
		\caption{A extensão $\tilde{u}$ de $u$ para a esfera $S^{2}$ através da colagem das extensões $\tilde{x}^{-}$ e $\tilde{x}^{+}$}
		\label{figura_colagem_disco_cilindro}
	\end{figure}
	
	Com isso, pode-se escolher uma base simplética orgononal $\{e_{1}(s,t), \dots, e_{2n}(s,t)\}$ de $\espacotangenteponto{u(s,t)}{M}$ que varia suavemente com $(s,t)\in \retacartesianocirculo$. Além disso, suponha que
	$$
	\lim\limits_{s\to \pm \infty}\derivadaparcial{e_{j}}{s}(s,t)=0,\;\;	\lim\limits_{s\to \pm \infty}\frac{\partial^{2}e_{j}}{\partial s^{2}}(s,t)=0,\;\;\frac{\partial^{2}e_{j}}{\partial s \partial t}(s,t)=0.
	$$
	
	O referencial $\{e_{1}, \dots, e_{2n}\}$ define uma carta em $u$ da variedade de Banach $\caminhosexponenciaisconectantespadrao$ dada por
	$$
	\espacosobolevcontradominio{\retacartesianocirculo}{\real{2n}} \ni (y_{1}, \dots ,y_{2n})\mapsto \Exp_{u}\Big(\sum_{j=1}^{2n}y_{j}e_{j}\Big) \in \caminhosexponenciaisconectantespadrao.
	$$
	
	Nessa carta o diferencial de $\diferencialfloerponto{u}$ avaliado em $Y=\sum_{j=1}^{2n}y_{j}e_{j} \in \real{2n} \subset \real{m}$ é escrito como
	
	$$
	\begin{aligned}
	\diferencialfloerponto{u}(Y)
	&=\underbrace{
		\sum_{j=1}^{2n}\bigparenteses{\derivadaparcial{y_{j}}{s}+J_{u}\derivadaparcial{y_{j}}{t}}(e_{j}) }_{	\overline{\partial}_{u}(Y)}
	+ \underbrace{\sum_{j=1}^{2n}
		y_{j}\bigparenteses{(d_{u}J)(e_{j})\bigparenteses{ \derivadaparcial{u}{t} - \campohamiltoniano{u}}-J_{u}(d_{u}\campohamiltonianoabrev)(e_{j})}}_{S_{u}(Y)}
	\\
	&=
	\overline{\partial}_{u}(Y)+S_{u}(Y).
	\end{aligned}
	$$
	
	Como $\{e_{1}, \dots,e_{2n} \}$ é um referencial simplético, então $J_{u}$ coincide com a estrutura complexa $\estruturacomplexa$ nessa carta. Logo, $	\overline{\partial}_{u}(Y) =\sum_{j=1}^{2n}\big( \derivadaparcial{y_{j}}{s}+\estruturacomplexa \derivadaparcial{y_{j}}{t}\big)(e_{j})$.
	
	A seguir será calculado o operador dual de $\diferencialfloer$ no contexto distribucional, e os objetos utilizados nessa demonstração estão no Apêndice \ref{apendice_preliminares_analise}.


	\begin{proposicao}\label{proposicao_diferencial_floer_adjunto}
		Em termos de distribuição, o operador adjunto $\diferencialfloeradjponto{u}:\espacoLpcontradominio{\retacartesianocirculo}{\real{m}} \to \espacosobolevcontradominio{\retacartesianocirculo}{\real{m}}$ de $\diferencialfloer$ é dado por
		$$
		\diferencialfloeradjponto{u}=-\derivadaparcial{}{s}+\estruturacomplexa \derivadaparcial{}{t}+S^{t}.
		$$
	\end{proposicao}
	\begin{prova} Tome $f \in \espacoLpdual$ e $Y \in \espacoLpGeral{q}{\retacartesianocirculo;\real{m}}$ tal que 
		$$
		f(.) = \int_{\retacartesianocirculo}\produtointerno{Y}{.},
		$$
		de acordo com o Teorema \ref{teorema_representacao_reiz}. Seja  $\funcoessupcompactcontradom{\retacartesianocirculo}{\real{m}}$ o conjunto das aplicações suaves e de suporte compacto de $\retacartesianocirculo$ em $\real{m}$. Por simplicidade, denote $\diferencialfloerponto{u}$ por $\diferencialfloerabrev$. Então, para todo $X$ no espaço de funções teste $\espacosobolevcontradominio{\retacartesianocirculo}{\real{m}} \cap \funcoessupcompactcontradom{\retacartesianocirculo}{\real{m}}$ tem-se que
		$$
		\begin{aligned}
		f(\diferencialfloerabrev X)
		&= \int_{\retacartesianocirculo}\produtointerno{Y}{\diferencialfloerabrev X}
		\\
		&=\int_{\retacartesianocirculo}\produtointerno{Y}{\derivadaparcial{ X}{s}+\estruturacomplexa\derivadaparcial{ X}{t}+S X}
		\\
		&=
		\int_{\retacartesianocirculo}\produtointerno{Y}{\derivadaparcial{ X}{s}} + \int_{\retacartesianocirculo}\produtointerno{Y}{\estruturacomplexa\derivadaparcial{ X}{t}} + \int_{\retacartesianocirculo}\produtointerno{Y}{S X}.
		\end{aligned}
		$$
		
		Efetuando integração por partes da primeira integral e usando o fato de que $ X(s,t) \to  0$ quanto $s \to \infty$, pois tem suporte compacto, tem-se
		$$
		\begin{aligned}
		\int_{\retacartesianocirculo}\produtointerno{Y}{\derivadaparcial{ X}{s}}
		&=\int_{\retacartesianocirculo}\derivadaparcial{}{s}\produtointerno{Y}{ X} - \int_{\retacartesianocirculo}\produtointerno{\derivadaparcial{Y}{s}}{ X} 
		\\
		&=\int_{\circulo}\bigparenteses{\produtointerno{Y(s,t)}{ X(s,t)}\Big|_{s=-\infty}^{s=\infty}}dt -\int_{\retacartesianocirculo}\produtointerno{\derivada{Y}{s}}{ X}
		\\
		&=-\int_{\retacartesianocirculo}\produtointerno{\derivada{Y}{s}}{ X}.
		\end{aligned}
		$$
		
		Usando o fato que $X(s, 0)=X(s, 1)$ para todo $s\in \reta$, e aplicando integração por parte na segunda integral, tem-se
		$$
		\begin{aligned}
		\int_{\retacartesianocirculo}\produtointerno{Y}{\estruturacomplexa\derivadaparcial{ X}{t}} 
		&= \int_{\retacartesianocirculo}\bigparenteses{\derivadaparcial{}{t}\produtointerno{Y}{ \estruturacomplexa X}-\produtointerno{\derivadaparcial{Y}{t} }{\estruturacomplexa X}}
		\\
		&= \int_{\reta}\bigparenteses{\produtointerno{Y(s,t)}{\estruturacomplexa X(s,t)}\Big|_{0}^{1}}ds + \int_{\retacartesianocirculo}\produtointerno{\estruturacomplexa \derivadaparcial{Y}{t}}{ X}
		\\
		&=\int_{\retacartesianocirculo}\produtointerno{\estruturacomplexa \derivadaparcial{Y}{t}}{ X}.
		\end{aligned}
		$$
		E para a terceira integral
		$$
		\int_{\retacartesianocirculo}\produtointerno{Y}{S X}= \int_{\retacartesianocirculo}\produtointerno{S^{t}Y}{ X}.
		$$
		Portanto
		$$
		\begin{aligned}
		f(\diferencialfloerabrev X)
		&= \int_{\retacartesianocirculo}\produtointerno{Y}{\diferencialfloerabrev X}
		\\
		&=
		\int_{\retacartesianocirculo}\produtointerno{-\derivadaparcial{Y}{s}+\estruturacomplexa \derivadaparcial{Y}{t} + S^{t}Y}{ X}
		\\
		&=\int_{\retacartesianocirculo}\produtointerno{\diferencialfloerabrev^{*}Y}{ X}
		\\
		&=(\diferencialfloerabrev^{*}f)(X).
		\end{aligned}
		$$
		Como o domínio $\espacosobolevcontradominio{\retacartesianocirculo}{\real{m}} $ de $\diferencialfloerabrev$ é denso em $\espacoLpcontradominio{\retacartesianocirculo}{\real{m}}$ (veja em \cite{breazis_sobolev_spaces}), então o operador adjunto de $\diferencialfloerabrev$ é dado em termos de distribuição por
		$$
		\diferencialfloeradjponto{u}=\diferencialfloerabrev^{*}=-\derivadaparcial{}{s}+\estruturacomplexa \derivadaparcial{}{t}+S^{t}.
		$$
	\end{prova}
	
	A priori, as soluções da equação $\diferencialfloerponto{u}(Y)=0$ são dadas em termos distribucionais. Contudo, os dois seguintes resultados afirmam sob quais hipóteses essas soluções são na verdade soluções de classe $C^{\infty}$. Mais detalhes sobre a caracterização das soluções desse sistema e das soluções da equação de Floer $\operadorFloerParametro{u}=0$ podem ser encontradas em \cite{audi_floer_homology}.
	
	\begin{teorema}\label{teorema_regularidade_eliptica}
		(Regularidade Elíptica Linear) Se $Y \in \espacoLpcontradominio{\retacartesianocirculo}{\real{m}}$ é uma solução fraca (no sentido distribucional) da equação $\diferencialfloerponto{u}(Y) =0$, então 
		$$
		Y \in \espacosobolevcontradominio{\retacartesianocirculo}{\real{m}} \cap C^{\infty}(\retacartesianocirculo;\real{m}).
		$$
		Além disso, se $p>2$, então $Y \in \espacosobolevgeral{q}{\retacartesianocirculo;\realprojetivo{m}}$ para todo $q>1$.
	\end{teorema} 
	
	A regularidade elíptica linear é utilizada para demonstrar que $\diferencialfloerponto{u}$ é um operador de Fredholm para $p>1$ no Teorema \ref{teorema_propriedade_fredholm}. Além disso, é importante nas Proposições  \ref{proposicao_orbitasConectantesZ},  \ref{proposicao_gamma_sobrejetora} e o Teorema \ref{teorema_variedade_espaco_trajetoria}.
	
	O próximo resultado é chamado regularidade elíptica não linear. É importante notar que a regularidade elíptica não linear vale somente quando $p>2$. Seja $W^{1,p}_{loc}(\retacartesianocirculo;\real{m})$ o conjunto das aplicações de Sobolev localmente integráveis, ou seja, é o conjunto dos elementos $Y \in \espacosobolevcontradominio{\retacartesianocirculo}{\real{m}}$ integráveis em todo compacto de $\retacartesianocirculo$. Analogamente, considere o conjunto $C^{\infty}_{loc}(\retacartesianocirculo;\real{m})$ das aplicações $\aplicaoessuaves{\retacartesianocirculo}{\real{m}}$ localmente integraveis.
	
	\begin{proposicao}
		Para $p>2$, as soluções $u\in W^{1,p}_{loc}(\retacartesianocirculo;M)$ da equação de Floer são de classe $C^{\infty}$. No espaço de soluções $\energiafinitaM\cap W^{1,p}_{loc}(\retacartesianocirculo;M)$ as topologias $W^{1,p}_{loc}$ e $C^{\infty}_{loc}$ coincidem.
	\end{proposicao}
	
	A regularidade elíptica não linear é usada para se mostrar que as soluções da equação de Floer de $\caminhosexponenciaisconectantespadrao$ são de classe $C^{\infty}$ e que em $\energiafinitaM$ as topologias $C^{0}_{loc}$, $C^{1}_{loc}$ e $C^{\infty}_{loc}$ coincidem. Uma consequência desse resultado é que as topologias induzidas por $\caminhosexponenciaisconectantespadrao$ e por $\energiafinitaM$ no espaço das soluções coincidem e são ambas iguais a topologia $C^{\infty}_{loc}$.
	
	Seja $T:A\to B$ um operador linear entre espaços vetoriais. Suponha que $T$ seja limitado e que existam subespaços fechados $V \subset A$ e $W \subset B$ tais que $A=V\oplus Ker(T)$ e $B=W\oplus T(A)$. Então $B/T(A) \cong W$ e $T:V \to T(A)$ é um isomorfismo (veja $\cite{abramovich}$). Nesse sentido, fica claro que as dimensões dos espaços $Ker(T)$ e $W$ são indicadores de quanto o operador se distancia de um isomorfismo. Com essa motivação define-se os operadores de Fredholm.
	
	\begin{definicao}\label{definicao_operador_fredholm}
		(Operador de Fredholm) Sejam $A, B$ espaços de Banach e $T: A\to B$ um operador linear limitado. O cokernel de $T$ é o quociente $Coker(T)=B/T(A)$. O operador $T$ é chamado de operador de Fredholm se $T(A)=\overline{T(A)}$ (imagem fechada), $k(T) = dim(Ker(T)) < \infty$ e $c(T)=dim(Coker(T)) < \infty$. O índice do operador $T$ é o número inteiro $\ind(T) = k(T) - c(T)$. 
	\end{definicao}
	
	\begin{definicao}
		(Aplicação de Fredholm) Uma aplicação diferenciável entre espaços de Banach é uma aplicação de Fredholm se o seu diferencial é um operador de Fredholm em todos os pontos de seu domínio.
	\end{definicao}
	
	
	O teorema a seguir afirma que o diferencial $\diferencialfloerponto{u}$ é um operador de Fredholm, e seu subsequente, que o índice de Fredholm de $\diferencialfloerponto{u}$ é dado pela diferença dos índices de Maslov dos pontos críticos $x^{-}, x^{+} \in \pontoscriticos{\funcionalH}$ que são conectados por $u$. Esse resultado é utilizado na demonstração da transversalidade do operador de Floer $\operadorFloer$, e como consequência, no fato de que $\energiafinitaMconectante$ é uma variedade de dimensão finita e igual ao índice de Fredholm $\ind(\diferencialfloerponto{u})$. Ambos os resultados podem ser encontrados em \cite{audi_floer_homology} e \cite{salamon_zehnder}.
	
	\begin{teorema}\label{teorema_propriedade_fredholm}
		(Propriedade de Fredholm de $\diferencialfloerponto{u}$) Sejam $S^{\pm}$ operadores de acordo com o Teorema \ref{teorema_regularidade_eliptica}. Suponha que $\Psi^{\pm}$ sejam soluções do sistema $\dot{\Psi}(t) = \estruturacomplexa S^{\pm}(t)\Psi(t)$, onde $\Psi(0)=Id$ tal que $\det(Id-\Psi^{\pm}(1))\neq 0$. Então o operador $\diferencialfloerponto{u}: \espacosobolevcontradominio{\retacartesianocirculo}{\real{m}}\to \espacoLpcontradominio{\retacartesianocirculo}{\real{m}}$ é um operador de Fredholm para $1<p<\infty$.
	\end{teorema}
	
	\begin{teorema}\label{teorema_indice_diferencial_operador_floer}
		Sejam $ x^{-}, x^{+} \in \pontoscriticos{\funcionalH}$ e $u\in \caminhosexponenciaisconectantespadrao$. O índice de Fredholm de $\diferencialfloerponto{u}$ é
		$$
		\ind(\diferencialfloerponto{u}) = \iconley{x^{-}}-\iconley{x^{+}}.
		$$
	\end{teorema}
	
	\subsection{A Transversalidade do Operador $\operadorFloer$ e a Compacidade de $\energiafinitaMconectante$}\label{secao_transversalidade_compacidade_M}
	
	A propriedade da transversalidade do operador de Floer é utilizada na demonstração de que o espaço das soluções da equação de Floer que conectam esse dois pontos $\energiafinitaMconectante$ é uma variedade com dimensão igual a $\iconley{x^{-}}-\iconley{x^{+}}$.
	
	Por fim, será introduzido o espaço quociente $\quocientetrajetorias$ e a compacidade implica que esse conjuntos possui um número finito de elementos. Esse resultado é fundamental na definição do operador bordo do complexo de cadeia de Floer e sua homologia definidos na Seção \ref{secao_complexo_floer}.
	
	A finalidade dessa seção é esboçar um roteiro para a demonstração dos resultados mencionados e parte dos objetos matemáticos envolvidos. Os detalhes das demonstrações podem ser encontradas em \cite{audi_floer_homology}.
	
	Tome $x^{-},x^{+}\in \pontoscriticos{\funcionalH}$, e considere o operador de Floer 
	$$
	\operadorFloer: \caminhosexponenciaisconectantespadrao \to \fibradocaminhosexponenciais
	$$  
	definido por
	$$
	\operadorFloerParametro{u} =\operadorFloerDefinicao{u}
	$$
	e seu diferencial $\diferencialfloerponto{u}: \espacosobolev{\pullbackfibradotangenteM{u}} \to \espacoLp{\pullbackfibradotangenteM{u}}$ definido por
	$$
	\diferencialfloerponto{u}(Y)= \Big( \derivadaparcial{}{s} + J_{u}\derivadaparcial {}{t}\Big)(Y)+ \Big(dJ_{u}(.)\bigparenteses{\derivadaparcial{u}{t} - X_{H}(u)} - J_{u} \diferencialhamiltoniano{u}\Big)(Y).
	$$
	
	Suponha que $X:\caminhosexponenciaisconectantespadrao \to \fibradocaminhosexponenciais$ seja seção nula de $\fibradocaminhosexponenciais$. Afirmo que $\operadorFloer \pitchfork X$ se, e somente se, $p_{u}\circ\diferencialfloerponto{u}$ é sobrejetor para todo $u\in \caminhosexponenciaisconectantespadrao$, onde $p_{u}$ é a projeção da imagem de $\diferencialfloerponto{u}$ em $\espacoLp{\pullbackfibradotangenteM{u}}$. De fato, suponha que $\operadorFloer$ e $X$ sejam aplicações transversais. Como $X_{u} = 0$ para todo $u \in \caminhosexponenciaisconectantespadrao$, então $DX_{u} = 0$, o que implica em
	$$
	\espacoLp{\pullbackfibradotangenteM{u}} = (p_{u}\circ \diferencialfloerponto{u})(\pullbackfibradotangenteM{u}) + (p_{u}\circ DX_{u})(\pullbackfibradotangenteM{u}) = (p_{u}\circ \diferencialfloerponto{u})(\pullbackfibradotangenteM{u}).$$ 
	Logo, $p_{u}\circ\diferencialfloerponto{u}$ é sobrejetor. A recíproca é imediata. Por simplicidade, a projeção $p_{u}$ será omitida da composição $p_{u}\circ\diferencialfloerponto{u}$ nos resultados a seguir.
	
	As soluções da equação de Floer dependem da escolha da estrutura complexa $J$ e da função Hamiltoniana $H$, e com isso a transversalidade de $\operadorFloer$ não é garantida. Porém, pode-se mostrar que arbitrariamente próximo de $H$ existe uma outra função Hamiltoniana com as mesmas orbitas periódicas, satisfazendo a propriedade da transversalidade.
	
	Considere o espaço de Banach $\perturbacaoHamiltoniana{H}$ das perturbações 
	$h:M\times \circulo \to \reta$ tais que $H+h$ está $\epsilon$-próximo de H. Mais especificamente,  $h(x,t)=0$ em uma vizinhança das soluções 1-periódicas, e portanto, as soluções periódicas de $H+h$ e de $H$ são as mesmas. Pode se mostrar existe uma escolha de $\epsilon$ tal que $\perturbacaoHamiltoniana{H}$ é denso $\aplicaoessuaves{M\times \circulo}{\reta}$ na $\cktopologia{1}$.
	
	Um par $(H, J)$ é chamado de regular se satisfaz a condição de transversalidade, e o conjunto das funções $h:M\times \circulo \to \reta$ tais que $(H+h,J)$ é um par regular é denotado por $\hamiltonianasRegulares$.

	Não há garantias de que, dada função Hamiltoniana $H$, a transversalidade sempre será satisfeita. Contudo, fixada a Hamiltoniana $H$, pode-se encontrar uma outra função Hamiltoniana suficientemente próxima a $H$, no sentido $\cktopologia{1}$, tal que valha essa propriedade. 
	
	Dado um par $(H+h, J)$, denote o operador de Floer associado a função Hamiltoniana $H+h:M\times \reta \to \reta$ por $\operadorFloer^{H+h}$. Sejam $\energiafinitaMHamiltoniana{H+h}$ o conjunto das soluções da equação de Floer $\operadorFloer^{H+h}(u)=0$ e  $\energiafinitaMconectanteHamiltoniana \subset \energiafinitaMHamiltoniana{H+h}$ o conjunto das soluções de Floer que conectam os pontos $x^{-}, x^{+} \in \pontoscriticos{\funcionalH}$.
	
	Seja 
	$$
	\orbitasConectantesZ = \{ (u, H+h): u \in  \energiafinitaMconectanteHamiltoniana \;\;\text{e}\;\; h\in \perturbacaoHamiltoniana{H} \}
	$$
	o conjunto das soluções da equação de Floer conectando $x^{-}$ a $x^{+}$ de todas as perturbações da função Hamiltoniana $H$ tais que as soluções 1-periódicas de $H+h$ são as mesmas que as de $H$.
	
	\begin{proposicao}\label{proposicao_gamma_sobrejetora}
		(Propriedade de transversalidade) Se $(u, H+h) \in \orbitasConectantesZ$, então a aplicação 
		$$
		\begin{aligned}
		\Gamma:\espacosobolev{\pullbackfibradotangenteM{u}}\times \perturbacaoHamiltoniana{H} 
		&\to \espacoLp{\pullbackfibradotangenteM{u}}
		\\
		(Y, h) 
		&\mapsto \diferencialfloerponto{u}(Y) + \nabla_{u}h
		\end{aligned}
		$$
		é sobrejetora e admite uma inversa contínua a direita.
	\end{proposicao}
	
	Como consequência da proposição anterior, e via o teorema da função implícita, pode-se mostrar que
	
	\begin{proposicao}\label{proposicao_orbitasConectantesZ}
		Dados $x^{-},x^{+}\in \pontoscriticos{\funcionalH}$ com $x^{+}\neq  x^{-}$ tem-se que $\orbitasConectantesZ$ é uma variedade (de Banach). 
	\end{proposicao}
	
	Seja $\pi:\orbitasConectantesZ \to \perturbacaoHamiltoniana{H}$ a projeção definida por $\pi(u, H+h) = h$ e seu diferencial
	$$
	\begin{aligned}
	D_{(u,H+h)}\pi: T_{(u,H+h)}\orbitasConectantesZ 
	&\to T_{h}\perturbacaoHamiltoniana{H}=\perturbacaoHamiltoniana{H}
	\\
	(Y,h) &\mapsto h.
	\end{aligned}
	$$
	
	Afirmo que $\pi$ é uma aplicação de Fredholm. De fato, pela Proposição \ref{proposicao_gamma_sobrejetora} pode-se afirmar que $\ker(\Gamma)\neq \emptyset$. Com isso, para todo $(Y,h)\in\ker(\Gamma)$ tem-se que $\nabla_{u}h =- \diferencialfloerponto{u}(Y)$. Como $D_{(u,H+h)}\pi(V) = \nabla_{u} h$, então $\ker(D_{(u,H+h)}\pi) = \ker(\nabla_{u} h) = \ker(\diferencialfloerponto{u})$ e a dimensão do kernel de $D\pi$ é finita. Além disso, a imagem de $D_{u, H+h}\pi$ é igual a $G^{-1}(\diferencialfloerponto{u}(\espacoLp{\pullbackfibradotangenteM{u}}))$, onde $G:\perturbacaoHamiltoniana{H} \to \pullbackfibradotangenteM{u}$ é definida por $G(h) = \nabla_{u}h$. Logo, o cokernel de $D_{u, H+h}\pi$ tem dimensão finita. Portanto $\pi$ é uma aplicação de Fredholm.

	O teorema de Sard-Smale afirma que o conjunto de valores regulares de um mapa de Fredholm entre espaços de Banach separáveis é um subconjunto denso (mais detalhes podem ser encontrados em \cite{audi_floer_homology}).
	
	\begin{teorema}\label{teorema_sard_smale}
		(Corolário de Sard-Smale) Como $\pi$ é uma aplicação de Fredholm, então o conjunto de seus valores regulares é uma intersecção enumerável de conjuntos densos.
	\end{teorema}
	
	\begin{lema}\label{lema_valores_regular_pi}
		Os valores regulares de $\pi$ são os elementos $h$ de $\perturbacaoHamiltoniana{H}$ tais que $u\in \energiafinitaM(H+h, J)$ e $\diferencialfloerponto{u}$ é sobrejetor.
	\end{lema}

	O teorema a seguir, que é consequência da do Teorema \ref{teorema_sard_smale} e do Lema \ref{lema_valores_regular_pi}, formaliza resultado em termos da sobrejetividade do diferencial do operador de Floer, como foi discutido anteriormente.
	
	\begin{teorema}\label{teorema_sobrejetividade_DF_todo_u}
		Fixe uma função Hamiltoniana não-degenerada $H:M\times \reta\to \reta$ dependente do tempo. Então existe uma vizinhança $U$ de $0$ em $\perturbacaoHamiltoniana{H}$ e uma intersecção enumerável de subconjuntos abertos de $\hamiltonianasRegulares$ densos em $U$ tal que se $h\in \paresregulares$, então $H+h$ é não-degenerada e o diferencial $\diferencialfloerponto{u}^{H+h}$ é sobrejetor para todo $u \in \energiafinitaMHamiltoniana{H+h}$, onde $\operadorFloer^{H+h}$ é o operador de Floer correspondente a Hamiltoniana $H+h$.
	\end{teorema}
	
	Como $\diferencialfloerponto{u}^{H+h}$ é sobrejetor para todo $u \in \energiafinitaMHamiltoniana{H+h}$, então o operador de Floer $\operadorFloer^{H+h}$ satisfaz a propriedade de transversalidade para todo $u \in \energiafinitaMHamiltoniana{H+h}$.

	Assumindo os resultados anteriores, tem-se que 
	
	\begin{teorema}\label{teorema_variedade_espaco_trajetoria}
		Para cada $h \in \hamiltonianasRegulares$ e para todas as soluções $x^{-}$ e $x^{+}$ contráteis 1-periódicas de $H$, $\energiafinitaM(x^{-}, x^{+}, H+h)$ é uma variedade de dimensão $\iconley{x^{-}}-\iconley{x^{+}}$.
	\end{teorema}
	\begin{prova} Tome $h \in \perturbacaoHamiltoniana{H}$. Pelo teorema da pré-imagem de valores regulares, tem-se que $\pi^{-1}(h)$ é uma variedade e sua dimensão é dada por
			$$
			\begin{aligned}
			\dim(\ker(D_{(u, H+h)}\pi)) &= \dim(\ker(\diferencialfloerponto{u}))
			\\
			&=\ind{(\diferencialfloerponto{u})}
			\\
			&=\mu(x^{-})-\mu(x^{+}).
			\end{aligned}
			$$
			
			Note que, os elementos de $\pi^{-1}$ são as soluções em $\caminhosexponenciaisconectantespadrao$ da equação de Floer para o par $(H+h, J)$. Pela Proposição \ref{proposicao_cadeia_caminhos_conectantes}, pode-se afirmar que 
			$$
			\energiafinitaMconectanteHamiltoniana \subset\pi^{-1}(h).
			$$
			
			Do Teorema da Regularidade Elíptica \ref{teorema_regularidade_eliptica}, pode-se concluir que 
			$$
			\pi^{-1}(h) \subset\energiafinitaMconectanteHamiltoniana.
			$$
			Portanto $\energiafinitaMconectanteHamiltoniana =\pi^{-1}(h)$.
	\end{prova}
	
	A compacidade de $\energiafinitaM$ exerce um papel importante na definição da homologia de Floer. O teorema a seguir garante que, com a hipótese da Aesfericidade, essa propriedade é garantida e sua demonstração pode ser encontrada em $\cite{audi_floer_homology}$.
	
	\begin{teorema}\label{teorema_compacidade_gromov}
		(Teorema da Compacidade de Gromov) Supondo a Condição de Aesfericidade $\ref{suposicao_aesfericidade}$, então $\energiafinitaM$ é compacto em $C^{\infty}_{loc}(\retacartesianocirculo; M)$.
	\end{teorema}
	
	
	Note que, tomando $\alpha\in \reta$, tem-se que $\tilde{u}\in \aplicaoessuaves{\retacartesianocirculo}{M}$ definido por $ \tilde{u}(s,t)=u(s+\alpha,t)$, também é solução da equação de Floer. Com isso, pode-se afirmar que o grupo aditivo $\reta$ age a direita em $\energiafinitaM$ pela operação $*:\energiafinitaM\times \reta\to \energiafinitaM$ definida por $*(u,\alpha)(s,t) = u(s+\alpha,t)$ para todo $s,t$. Essa operação será denotada por $u*\alpha$. A $\reta-$órbita de $u \in \energiafinitaM$ é o conjunto $u*\reta = \{u*\alpha  : \alpha\in \reta\}$ e o quociente $\energiafinitaM/\reta$ é o conjunto das $\reta$-óbitas. Esse quociente é um espaço topologia munida da topologia quociente. Nessa topologia uma sequência $\tilde{u}_{n}$ converge para $\tilde{u}$ se, e somente se, existe uma sequência $(s_{n})$ de números reais tais que $\lim\limits_{n\to \infty}u(s+s_{n}, t) = u(s,t)$ em $\energiafinitaM$ para todo $t \in \circulo$.
	
	Defina o quociente 
	$$\quocientetrajetorias=\energiafinitaM(x^{-},x^{+}, H, J)/\reta.
	$$

	Se $x^{-}=x^{+}$, então $\energiafinitaM(x^{-},x^{+}, H, J)$ é o conjunto das soluções constantes em $s$ e iguais a $x^{-}$. Caso $x^{-}\neq x^{+}$, então $\quocientetrajetorias$ é uma variedade de dimensão $\iconley{x^{-}}-\iconley{x^{+}}-1$. Se $\iconley{x^{-}}=\iconley{x^{+}}$ e $x^{-}\neq x^{+}$, então $\energiafinitaM(x^{-},x^{+}, H, J)=\emptyset$, e portanto $\quocientetrajetorias=\emptyset$.

	Considere os pontos $x^{-}, x^{+} \in \pontoscriticos{\funcionalH}$ tais que $\iconley{x^{-}}-\iconley{x^{+}}=1$. Em \cite{audi_floer_homology} e \cite{salamon_lecture} é demonstrado que nesse caso $\quocientetrajetorias$ é uma variedade compacta de dimensão zero, e portanto é um conjunto finito. Esse resultado é formalizado na proposição a seguir.

	\begin{proposicao}\label{proposicao_quociente_orbitas_finitas}
		Se $(H,J)$ é um par regular, então o quociente $\widehat{\energiafinitaM}(x^{-},x^{+}, H, J)$ é um conjunto finito para cada par $x^{-}, x^{+} \in \pontoscriticos{\funcionalH}$.
	\end{proposicao}
	
	\section{O Complexo de Floer e a Conjectura de Arnold}\label{secao_complexo_floer}
	
	Seja $(M,\omega)$ uma 2n-variedade simplética e $H: M\times \reta\to \reta$ uma função Hamiltoniana dependente do tempo não degenerada e uma estrutura complexa $J$. Fazendo uma perturbação dessa Hamiltoniana, caso seja necessário, pode-se supor que o par $(H,J)$ é regular. Suponha que $M$ satisfaça as condições \ref{suposicao_aesfericidade} e \ref{suposicao_trivializacao}. Defina
	$$
	\pontoscriticosordem{k}{\funcionalH} = \{ y \in \pontoscriticos{\funcionalH}: \iconley{y} \ =	 k\}.
	$$
	
	Considere $C^{F}_{k}(M, \inteiros_{2}; H,J)$ o espaço vetorial sobre o corpo $\inteiros_{2}$ e gerado pelos elementos $x \in \pontoscriticosordem{k}{\funcionalH} $.
	
	Por simplicidade, denote $C^{F}_{k}(M, \inteiros_{2}, H,J)$ por $C^{F}_{k}(H,J)$. Com isso, $C^{F}_{*}(H,J)=\bigoplus_{k}C^{F}_{k}(H,J)$ é um espaço vetorial sobre o corpo $\inteiros_{2}$ e graduado pelo índice de Conley-Zehnder $\iconleyabrev$.

	A Proposição \ref{proposicao_quociente_orbitas_finitas} afirma que, dados $x^{-},x^{+} \in \pontoscriticos{\funcionalH}$, o conjunto $\quocientetrajetorias$ é finito. Com isso, denote o total de elementos de $\quocientetrajetorias$ módulo 2 por $n(x^{-},x^{+})$.
	
	Considere o homomorfismo $\bordo{k}^{F}: C^{F}_{k}(H,J)\to C^{F}_{k-1}(H,J)$ de grau -1 definido em cada gerador por
	$$
	\bordo{k}^{F}(\gerador{x}) = \sum_{\iconley{y}=k-1}n(x,y)\gerador{y}
	$$
	e estendido por linearidade.
	
	O teorema a seguir mostra que o par $(C^{F}_{*}(H,J), \partial^{F})$ é um complexo de cadeias, e tal resultado pode ser encontrado em  \cite{audi_floer_homology} e \cite{salamon_lecture}.
	
	\begin{teorema}
		Se $(H,J)$ é um par regular, então $\partial^{F}_{k-1}\circ \partial^{F}_{k} = 0$.
	\end{teorema}
	
	A Homologia de Floer da 2n-variedade simplética $(M,\omega)$ é definida por 
	$$
		H^{F}_{*}(M;H,J)=\bigoplus_{k=0}^{2n}H^{F}_{k}(C^{F}_{*}(H,J), \partial^{F}),
	$$
	onde $H^{F}_{k}(C^{F}_{*}(H,J))$ é o k-ésimo grupo de homologia do complexo $C^{F}_{*}(H,J)$ da variedade $M$, ou seja,
			
	$$
		H^{F}_{k}(C^{F}_{*}(H,J)) = \frac{\ker(\partial^{F}_{k})}{\text{Im}(\partial^{F}_{k+1})}.
	$$ 
	
	Note que nas definições do complexo de cadeia e da homologia de Floer foram mantidas as referências a função Hamiltoniana $H$ escolhida para se realizar sua construção. Contudo,  pode-se mostrar que a homologia de Floer não depende de tal escolha, conforme o seguinte resultado.
	
	\begin{teorema}
		Sejam $(H,J)$ e $(H',J')$ dois pares regulares. Então existe um homomorfismo de cadeias induzindo um isomorfismo nas homologias de Floer
		$$
		H^{F}_{*}(C^{F}_{*}(H,J)) \cong H^{F}_{*}(M; H',J'). 
		$$
	\end{teorema}
	
	As definições e resultados obtidos até aqui tiveram como objetivo a construção de um invariante topológico que fornecesse informações sobre a topologia da 2n-variedade $M$ e, ao mesmo tempo, de sua estrutura simplética $\omega$.
	
	O próximo resultado, que pode ser encontrado em \cite{salamon_zehnder}, permite que se recupere a homologia singular da variedade a partir da homologia de Floer.
	
	\begin{teorema}\label{teorema_isomorfismo_homologia_floer}
		Se $(H,J)$ é um par regular, então 
		existe um isomorfismo natural entre a homologia de Floer e a homologia singular de $M$
		$$
		H^{F}_{*}(M;H,J) \cong H_{*}(M;\inteiros_{2}). 
		$$
		
	\end{teorema}
	
	Com a definição da homologia de Floer e com o isomorfismo dessa com a homologia singular da 2n-variedade, é imediata a demonstração da Conjectura de Arnold.
	
	\begin{teorema}
			(Conjectura de Arnold) Seja $(M,\omega)$ uma 2n-variedade compacta e simplética. Defina $H:M\times \real{} \to \reta$ uma função Hamiltoniana 1-periódica e suponha que as soluções 1-periódicas do sistema Hamiltoniano sejam não-degeneradas. Então o número de soluções $\mathcal{N}$ desse sistema é limitado inferiormente pela soma dos números de Betti de M, isto é:
				$$
				\mathcal{N}\geq \sum_{i=0}^{2n}\beta_{i}(M),
				$$
				onde $\beta_{i}(M)$ é a dimensão do i-ésimo grupo de homologia singular de $M$.
	\end{teorema}
	\begin{prova}
		Pela Proposição \ref{proposicao_solucao_hamiltoniana_finita} pode-se afirmar que $\pontoscriticos{\funcionalH}$ é um conjunto finito, logo $\pontoscriticosordem{k}{\funcionalH} \subset \pontoscriticos{\funcionalH}$ também o é. Denote a cardinalidade de $\pontoscriticosordem{k}{\funcionalH}$ por $\mathcal{N}_{k}$. Com isso, $\mathcal{N} =\sum_{j=0}^{2n} \mathcal{N}_{j}$ é o total de pontos críticos do funcional de ação. Como $C^{F}_{k}(H,J)$ é gerado por todos os pontos críticos de ordem $k$, então $\mathcal{N}_{k}\geq \dim(	H^{F}_{k}(C^{F}_{*}(H,J)))$. Como os grupos de homologia de Floer e de homologia singular são isomorfos (veja o Teorema \ref{teorema_isomorfismo_homologia_floer}), então $\dim(	H^{F}_{k}(C^{F}_{*}(H,J)))  = \dim(	H_{k}(M;\inteiros_{2}) = \beta_{k}(M)$. Portanto
		$$
		\mathcal{N} =\sum_{j=0}^{2n} \mathcal{N}_{j} \geq \sum_{j=0}^{2n} \beta_{j}(M).
		$$ 
	\end{prova}
	
	\appendix
	
	\chapter{Complexificação de espaços vetoriais}\label{apendice_complexificacao_espacos_vetoriais}
	
	Seja $V$ um n-espaço vetorial real. Deseja-se construir um espaço vetorial complexo a partir de $V$ e uma forma de realizar essa construção é tomando $v \in V $ e $a+ib \in \complexo{}$ e efetuando a multiplicação $(a+ib)v = av+ibv$. Porém, a multiplicação por $i$ não tem sentido nesse contexto, já que trata-se de um espaço vetorial real. Pode-se contornar essa dificuldade tratando a multiplicação formal $(a+ib)v = av+ibv$ como o par ordenado $(av, bv)$, resultando na seguinte definição de complexificação:
	
	\begin{definicao}
		(Complexificação do n-espaço vetorial real V) A complexificação do n-espaço vetorial real $V$ é o n-espaço vetorial complexo $\complexificacao{V}=V\oplus V$ munido da multiplicação por escalares $(a+ib)(v_{1}, v_{2}) = (av_{1}-bv_{2}, bv_{1}+ av_{2})$, onde $a+ib \in \complexo{}$ e $(v_{1}, v_{2})\in \complexificacao{V}$.
	\end{definicao}
	
	\begin{observacao}
		Essa definição tem como motivação a seguinte situação: supondo $V$ complexo, então $V \ni (a+ib)v = (a+ib)(v_{1}+iv_{2}) = av_{1}-bv_{2}+i(bv_{2}+av_{1}) \mapsto (av_{1}-bv_{2}, bv_{1}+ av_{2}) \in \complexificacao{V}$.
	\end{observacao}
	
	É possível demonstrar que $\complexificacao{V}$ é um espaço vetorial sobre $\complexo{}$. De fato, basta mostrar que esse espaço é fechado pela operação de adição e multiplicação por escalares, pois as outras operaçoes são triviais. Dados $a+ib \in \complexo{}$ e $v,u \in \complexificacao{V}$, então $(a+ib )v+u = (av_{1}-bv_{2}, bv_{1}+ av_{2}) + (u_{1}, u_{2}) = (av_{1}-bv_{2} + u_{1}, bv_{1}+ av_{2} + u_{2}) \in \complexificacao{V}$. As demais propriedades dos axiomas de espaços vetoriais resultam imediatamente dessa.
	
	\begin{exemplo}\label{exemplo_conjugado_reta_real}
		(Complexificação de $\reta$) Definindo $V = \reta$ tem-se que $\complexificacao{\reta}$ é $\reta\oplus \reta$ munido da operação $(a+ib)(x,y) = (ax-by, bx+ay)$. Assim, a identificação $(a+ib)(x, y ) = (ax-by, bx+ay) \mapsto ax-by+i(bx+ay) \in \complexo{}$ nos dá o isomorfimo $\complexificacao{\reta} \ni (x,y)\mapsto x +iy\in \complexo{}$.
	\end{exemplo}
	
	\begin{exemplo}\label{exemplo_complexificacao_matrizes}
		Analogamente ao exemplo anterior, tem-se que as aplicações $\complexificacao{\real{n}} \ni (x,y) \mapsto x+iy \in \complexo{n}$ e  $\complexificacao{\matrizquadreal{n}} \ni (x,y) \mapsto x+iy \in \matrizquadcomplexa{n}$ também são isomorfismos.
	\end{exemplo}
	
	Note que os espaços vetoriais $V\oplus\{0\}$ e $\{0\}\oplus V$ são isomorfos a $V$. Com isso, a inclusão $i_{V} :V \hookrightarrow \complexificacao{V}$ definida por $i(v) = (v, 0)$ é chamada de mergulho padrão de $V$ em $\complexificacao{V}$.
	
	Complexifica-se os espaços vetoriais reais, porém, as aplicações lineares entre tais espaços vetoriais são outros tipos de objetos a serem complexificados. O teorema a seguir afirma a unicidade da complexificação de aplicações lineares entre as complexificações de espaços vetoriais reais.
	
	\begin{teorema}
		(Complexificação de aplicações lineares) Seja $\varphi : V \to W$ uma aplicação
		$\reta$-linear entre espaços vetoriais reais de dimensão finita. Então existe uma única aplicação linear $\complexificacao{\varphi}:\complexificacao{V} \to \complexificacao{W}$ tal que $i_{W}\circ \varphi = \complexificacao{\varphi} \circ i_{V}$, isto é, o diagrama abaixo é comutativo.
		$$
		\xymatrix{
			V\ar[d]_{i_{V}}\ar[r]^{\varphi} & W\ar[d]^{i_{W}} 
			\\
			\complexificacao{V} \ar[r]_{\complexificacao{\varphi}}&\complexificacao{W} 
		}
		$$
	\end{teorema}
	\begin{prova}
		Defina $\complexificacao{\varphi}:\complexificacao{V} \to \complexificacao{W}$ por $\complexificacao{\varphi}(v_{1}, v_{2}) = (\varphi v_{1}, \varphi v_{2})$. Como $\varphi v_{1}, \varphi v_{2} \in W$, então $(\varphi v_{1}, \varphi v_{2}) \in W\oplus W = \complexificacao{W}$. Além disso, dados $\lambda = (a+ib)\in \complexo{}$ e $(v_{1}, v_{2}), (u_{1}, u_{2})\in \complexificacao{V}$ tem-se 
		$$
		\begin{aligned}
		\complexificacao{\varphi} (\lambda(v_{1}, v_{2}) + (u_{1}, u_{2})) 
		&= (\varphi(a v_{1} - bv_{2} + u_{1}), \varphi(b v_{1}+a v_{2} + u_{2}))
		\\
		&=(a\varphi v_{1} - b\varphi v_{2} + \varphi u_{1}, b\varphi v_{1}+a\varphi v_{2} + \varphi u_{2})
		\\
		&=(a\varphi v_{1} - b\varphi v_{2} , b\varphi v_{1}+a\varphi v_{2} )+(\varphi u_{1},\varphi u_{2})
		\\
		&=\lambda(\varphi v_{1},\varphi v_{2})+(\varphi u_{1},\varphi u_{2})
		\\
		&=\lambda 	\complexificacao{\varphi} (v_{1},v_{2})+\complexificacao{\varphi} (u_{1},u_{2}).
		\end{aligned}
		$$
		Portanto $\complexificacao{\varphi}$ é $\complexo{}$-linear. Dado $v \in V$ tem-se $(i_{W}\circ \varphi)(v) = (\varphi v,0) = \complexificacao{\varphi}(v,0) = (\complexificacao{\varphi} \circ i_{V})(v)$, logo $i_{W}\circ \varphi =\complexificacao{\varphi} \circ i_{V}$. Por fim, suponha que exista $\complexificacao{\varphi'} $ definida por $\complexificacao{\varphi'}(v_{1}, v_{2}) = (\varphi' v_{1}, \varphi' v_{2})$ e tal que $i_{W}\circ \varphi =\complexificacao{\varphi'} \circ i_{V}$, onde $\varphi':V\to W$ é uma aplicação linear. Então para todo $v \in V$ tem-se $(i_{W}\circ \varphi )(v)=(\complexificacao{\varphi} \circ i_{V})(v) = (\complexificacao{\varphi'} \circ i_{V})(v)$, o que implica em $(\varphi v, 0) = (\varphi' v, 0)$. Como $v \in V$ é arbitrário, então $\varphi = \varphi'$, logo $\complexificacao{\varphi} = \complexificacao{\varphi'}$, demonstrando a unicidade. 
	\end{prova}
	
	Considerando o corpo dos complexos como um espaço vetorial sobre $\reta$, será mostrado que o produto tensorial $\complexificacaotensorial{V}$ é equivalente a construção da complexificação feita anteriormente. Está construção será utilizada na complexificação do 2n-espaço vetorial simplético $(V, \omega)$.
	
	\begin{teorema}\label{teorema_isomorfismo_complexificacao}
		Seja $V$ um espaço vetorial real de dimensão finita. A aplicação $f_{V}: \complexificacao{V} \to \complexificacaotensorial{V}$ definida por  $f_{V}(v_{1},v_{2}) = 1\otimes_{\reta} v_{1} + i \otimes_{\reta} v_{2}$ é um isomorfismo entre os espaços vetoriais complexos que faz o diagrama
		$$
		\xymatrix{
			& V\ar[dl]_{i_{V}}\ar[rd]^{j} &
			\\
			\complexificacao{V}\ar[rr]_{f_{V}}& & \complexificacaotensorial{V},
		}
		$$
		comutar, onde $i_{V}(v) = (v,0)$ (é o mergulho padrão) e $j(v) = 1\otimes_{\reta} v$.
	\end{teorema}
	\begin{prova}
		Dados $z=(a+ib) \in \complexo{}$ e $v,u \in \complexificacao{V}$ tem-se
		$$
		\begin{aligned}
		f_{V}(zv+u) &= 1\otimes_{\reta} (a v_{1} - bv_{2} + u_{1}) + i \otimes_{\reta} (bv_{1}+av_{2} + u_{2})
		\\
		&=1\otimes_{\reta} (a v_{1} - bv_{2}) + i \otimes_{\reta} (bv_{1}+av_{2})+1\otimes_{\reta} u_{1} + i \otimes_{\reta} u_{2}
		\\
		&=(a+ib)\otimes_{\reta}v_{1} + (-b+ia)\otimes_{\reta} v_{2}+f_{V}(u)
		\\
		&=(a+ib)\otimes_{\reta}v_{1} + (a+ib)i\otimes_{\reta} v_{2}+f_{V}(u)
		\\
		&=zf_{V}(v)+f_{V}(u).
		\end{aligned}
		$$
		Portanto $f_{V}$ é $\complexo{}$-linear. Além disso, é injetora pois $f_{V}(u) = 0$ se, e somente se, $u = 0$. Tomando um elemento $w = (a+ib)\otimes_{\reta}v \in \complexificacaotensorial{V} $ e definindo $u = (av, bv) \in \complexificacao{V}$ tem-se que $f_{V}(u) = \complexificacaoelemento{1}{av} +\complexificacaoelemento{i}{bv} = (a+ib)\otimes_{\reta} v = w$. Portanto $f_{V}$ é sobrejetora. Logo é um isomorfismo.
		
		Para todo $v \in V$ se tem que $(f_{V}\circ i_{V})(v) = f_{V}(v, 0) = 1\otimes_{\reta} v = j(v)$. Portanto o diagrama é comutativo.
	\end{prova}
	
	O conjugado de cada $z=(a+ib) \in \complexo{}$ é definido por $\overline{z} = a-ib$. Analogamente, o conjugado de $\complexificacaoelemento{z}{v} \in \complexificacaotensorial{V}$ é definido por $\complexificacaoelemento{\overline{z}}{v} \in \complexificacaotensorial{V}$. Definindo $v_{1} =av$ e $v_{2} = -bv$ se tem $f_{V}(v_{1}, v_{2}) =\complexificacaoelemento{1}{v_{1}} +\complexificacaoelemento{i}{v_{2}} = \complexificacaoelemento{a}{v} -\complexificacaoelemento{ib}{v}  =\complexificacaoelemento{\overline{z}}{v}  $. Portanto, tem-se a identificação entre os conjugados $ \complexificacao{V} \ni (av, -bv) \mapsto \complexificacaoelemento{\overline{z}}{v} \in \complexificacaotensorial{V}$.
	
	Sejam $k\geq 1$ um inteiro e $\{V_{j}\}_{j=1}^{k}$ um conjunto de espaços vetoriais de dimensão finita sobre um corpo $\mathbb{K}$. Sabe-se que o produto tensorial $\produtotensorial{V}{k}$ é um espaço vetorial de dimensão finita sobre $\mathbb{K}$. Além disso, sabe-se que, dado um $W$ espaço vetorial também sobre $\mathbb{K}$, tem-se que $W\otimes_{\mathbb{K}}(\produtotensorial{V}{k}) \cong W\otimes_{\mathbb{K}} \produtotensorial{V}{k}$. 
	
	Daqui em diante $ \complexificacaotensorial{V}$ será denotado por $\complexificado{V}$ para a complexificação de $V$ via produto tensorial.
	
	\begin{observacao}
		Visto como espaço vetorial complexo, tem-se que $\{1\}$ é uma base de $\complexo{}$, logo se tem que $\{1^{*}\}$ é uma base do dual $\complexo{*}$, tal que $1^{*}(1) = 1$. Com isso, dado $z = (a+ib).1 \in \complexo{}$, tem-se $1^{*}(z) = 1^{*}((a+ib).1) = (a+ib)1^{*}(1) = a+ib$.
	\end{observacao}
	
	Sejam $B = \colecaofinita{e}{n}$ uma base do n-espaço vetorial real $V$ e $B^{*} = \colecaofinita{e^{*}}{n}$ a base de seu dual $V^{*}$ tal que $e^{*}_{j}(e_{k}) =\delta_{jk}$. Definindo $1 \otimes B = \colecaofinita{\complexificacaoelemento{1}{e}}{n}$ e
	$1^{*} \otimes B^{*} = \colecaofinita{\complexificacaoelemento{1^{*}}{e^{*}}}{n}$, tem-se que $1\otimes B$ gera $\complexificado{V}$. De fato, todo $v \in \complexificado{V}$ pode ser escrito como 
	$$
	\begin{aligned}
	v &= \sum_{j=1}^{n} \complexificacaoelemento{z_{j}}{v_{j}}
	\\
	&= \sum_{j=1}^{n} \complexificacaoelemento{z_{j}}{\bigparenteses{\sum_{i=1}^{n}\alpha_{ji}e_{i}}} 
	\\
	&= \sum_{j,i=1}^{n}\complexificacaoelemento{z_{j}\alpha_{ji}}{e_{i}}
	\\
	&= \sum_{i=1}^{n}\complexificacaoelemento{z'_{i}}{e_{i}}
	\\
	&= \sum_{i=1}^{n}z'_{i}(\complexificacaoelemento{1}{e_{i}}),
	\end{aligned}
	$$
	onde $z'_{i} = \sum_{j=1}^{n}z_{j}\alpha_{ji}$ e $ z_{j}$ estão em $\complexo{}$ e $v_{j} \in V$.
	
	
	\begin{observacao}
		Os elementos $v \in \complexificado{V}$ do tipo $v = \complexificacaoelemento{z}{\alpha}$ são chamados de vetores simples. Foi visto anteriormente que os vetores mais gerais são combinações lineares dos vetores simples. Por esse fato, as aplicações lineares e multilineares estudadas nessa seção serão avaliadas apenas nos tensores simples.
	\end{observacao}
	
	Para simplificar a notação, denotando os elementos das bases $1\otimes B$ e $1^{*}\otimes B^{*}$ por $\{\textbf{a}_{j}\} $ e $\{\textbf{a}^{*}_{j}\}$, respectivamente.
	
	\begin{proposicao}\label{proposicao_base_complexificada}
		$1\otimes B$ e $1^{*}\otimes B^{*}$ são bases de $\complexificado{V}$ e $\complexificado{V}^{*}$, respectivamente, tais que $\textbf{a}^{*}_{j}(\textbf{a}_{k})=\delta_{ik}$.
	\end{proposicao}
	\begin{prova}
		Por construção $1\otimes B$ gera $\complexificado{V}$. Afirmo que $B$ é linearmente independente pois, $0=\sum_{j}z_{j}\textbf{a}_{j} = \sum_{j}z_{j}(1\otimes e_{j}) = \sum_{j}(b_{j} +ic_{j})(1\otimes e_{j}) = 1\otimes \sum_{j}b_{j}e_{j}+i\otimes \sum_{j}c_{j}e_{j}$, o que implica que $\sum_{j}b_{j}e_{j}=\sum_{j}c_{j}e_{j} = 0$. Logo, $b_{j}=c_{j}=0$  e $z_{j} = 0$ para $1\leq j\leq k$. O isomorfismo $\complexificado{V} \mapsto \complexificado{V}^{*}$ garante que $B^{*}$ é base de $\complexificado{V}^{*}$. Por fim, tem-se que $\textbf{a}^{*}_{j}(\textbf{a}_{k})=(\complexificacaoelemento{1^{*}}{e^{*}_{j}})(\complexificacaoelemento{1}{e_{k}}) = 1^{*}(1)e^{*}_{j}(e_{k}) = \delta_{ik}$.
	\end{prova}
	
	Como o produto tensorial real $\otimes_{j=1}^{k}V_{j}$ é um espaço vetorial real, então sua complexificação será o espaço vetorial complexo $\complexificacaotensorial{(\produtotensorialreal{k}{V})} \cong \complexificacaotensorial{V_{1} \otimes_{\reta} \dots \otimes_{\reta} V_{k}}$, pelo isomorfismo anteriormente citado. Contudo, pode-se se mostrar que essa complexificação é isomorfa ao produto tensorial sobre os complexos $\bigotimes^{k}_{j=1}\complexificado{V}$, a qual se denota por $\complexificado{V}^{k\otimes}$. 
	
	Na seção seguinte será utilizado o produto tensorial complexificado $\complexificado{V}^{k\otimes}$ pois ela facilita a demonstração de algumas propriedades da complexificação da forma simplética, que é um tensor de ordem 2.
	
	\begin{exemplo}\label{exemplo_complexificacao_tensorial}
		Sejam $V$ um n-espaço vetorial real e $B = \colecaofinita{e}{n}$ uma base ortonormal. Supondo que $\complexificado{V}$ seja a complexificação de $V$, então pela Proposição $\ref{proposicao_base_complexificada}$ se tem que $1\otimes B^{*}$ é uma base de $\complexificado{V}^{*}$. Com isso, dado $T \in \produtotensorialdual$ é possível escrever $T = \sum_{i,j = 1}^{n} T^{ij}\textbf{a}^{*}_{i}\otimes \textbf{a}^{*}_{j}$, onde $T^{ij} \in \complexo{}$. Dados $v, u\in \complexificado{V}$ tais que $v=\complexificacaoelemento{x}{\alpha}$ e $u=\complexificacaoelemento{y}{\beta}$, tem-se
		$$
		\begin{aligned}
		T(v,u) &= \sum_{i,j = 1}^{n}
		T^{ij}\textbf{a}^{*}_{i}\otimes \textbf{a}^{*}_{j}(v,u)
		\\
		&= \sum_{i,j = 1}^{n}
		T^{ij}\textbf{a}^{*}_{i}(v)\textbf{a}^{*}_{j}(u)
		\\
		&= \sum_{i,j = 1}^{n}
		T^{ij}(\complexificacaoelemento{1^{*}}{e^{*}_{i}})(\complexificacaoelemento{x}{\alpha})(\complexificacaoelemento{1^{*}}{e^{*}_{j}})(\complexificacaoelemento{y}{\beta})
		\\
		&= \sum_{i,j = 1}^{n}
		T^{ij}1^{*}(x)e^{*}_{i}(\alpha)1^{*}(y)e^{*}_{i}(\beta)
		\\
		&= \sum_{i,j = 1}^{n}
		xyT^{ij}\alpha_{i}\beta_{j},
		\end{aligned}
		$$ 
		onde $\alpha_{j}$ e $\beta_{j}$ são as $j$-ésimas coordenadas de $\alpha, \beta \in V$. 
	\end{exemplo}
	
	Para estudar algumas características dos auto-espaços do grupo simplético é preciso  decompor esse espaço vetorial em soma direta de determinados  auto-espaços. Tal composição será garantida pelo Teorema $\ref{teorema_espectral_jordan}$, que depende de um espaço vetorial complexo como hipótese. Como consequencia desse fato, é necessário um espaço vetorial simplético complexo e que seja gerado a partir de um dado espaço vetorial simplético.
	
	Nessa seção, viu-se que uma das construções da complexificação de $V$ é o espaço vetorial complexo $\complexificado{V}$. Dado $(V, \omega)$ um 2n-espaço vetorial simplético real, tem-se que $\omega \in V^{*}\otimes V^{*}$. Sejam $B = \{e_{j}\}_{j=1}^{2n}$ uma base de $V$ e $\omega^{ij} = \omega(e_{i}, e_{j})$ as componentes da forma simplética nessa base. Dessa definição, tem-se que $\omega^{ij} = -\omega^{ji}$.
	
	Utilizando as notações da Proposição $\ref{proposicao_base_complexificada}$, defina $\Omega=\sum_{i,j}\omega^{ij}\textbf{a}^{*}_{i}\otimes \textbf{a}^{*}_{j} \in \produtotensorialdual$. Tem-se que
	
	\begin{proposicao} $(\complexificado{V}, \Omega)$ é um 2n-espaço vetorial simplético sobre $\complexo{}$.
	\end{proposicao}\label{proposicao_complexificacao_espaco_simpletico}
	\begin{prova}
		Pelo Teorema $\ref{teorema_isomorfismo_complexificacao}$, tem-se que $\complexificado{V}$ é um 2n-espaço vetorial complexo. Por construção $\Omega$ é uma aplicação $\complexo{}$-bilinear. Como $\omega^{ij}  = -\omega^{ji} $, tem-se que $\Omega$ é anti-simétrica. Afirmo que $\Omega$ é não degenerada. De fato, tomando $v, u \in \complexificado{V}$ tal que $v= \complexificacaoelemento{x}{\alpha}$ e $u=\complexificacaoelemento{y}{\beta} $ tem-se, pelo Exemplo $\ref{exemplo_complexificacao_tensorial}$, que 
		$$
		\begin{aligned}
		\Omega(v,u) &= \sum_{i,j = 1}^{2n}
		\omega^{ij}\textbf{a}^{*}_{i}\otimes \textbf{a}^{*}_{j}(v,u)
		\\
		&= \sum_{i,j = 1}^{2n}
		xy\omega^{ij}\alpha_{i}\beta_{j}
		\\
		&= xy\sum_{i,j = 1}^{2n}
		\formaSimpletica{\alpha_{i}e_{i}}{\alpha_{j}e_{j}}
		\\
		&= xy\formaSimpletica{\alpha}{\beta}.
		\end{aligned}
		$$
		Como $\omega$ é não-degenerada, então $\Omega$ é não-degenerada. Portanto, $(\complexificado{V}, \Omega)$ é um 2n-espaço vetorial simplético sobre $\complexo{}$.
	\end{prova}
	
	\begin{observacao}
		Note que $\complexificado{V}_{0} = \{\complexificacaoelemento{1}{v} \in \complexificado{V} \}$ é um subespaço vetorial de $\complexificado{V}$ isomorfo a $\real{2n}$. Além disso, $\Omega|_{\complexificado{V}_{0}} = \omega$ e $\Omega$ é uma extensão contínua de $\omega$.
	\end{observacao}
	
	\begin{proposicao}\label{proposicao_decomposicao_Omega}
		A complexificação $\Omega$ de $\omega$ pode ser decomposta em $\Omega = g +i k$, onde $g,k$ são ambas formas simpléticas definidas em $V$.
	\end{proposicao}
	\begin{prova} O Teorema $\ref{teorema_existencia_base_simpletica}$ garante que $\complexificado{V}$ possui uma base simplética $B$. Pela Observação $\ref{observacao_existencia_base_simpletica}$ tem-se que a representação matricial de $\Omega$ nessa base simplética é $\estruturacomplexa$. Suponha que $V, U \in \matrizquadcomplexa{2n}$ sejam as representações matriciais de $v,u \in \complexificado{V}$ nessa base. Então pode-se escrever 
	$$
		\begin{aligned}
		\formaSimpleticaExtendida{v}{u} &= V^{*}\estruturacomplexa^{*} U 
		\\
		&= (V_{1}+iV_{2})^{*}\estruturacomplexa^{*} (U_{1}+iU_{2})
		\\
		&= \underbrace{V_{1}^{t}\estruturacomplexa^{t} U_{1}}_{\formaSimpletica{v_{1}}{u_{1}}} -V_{2}^{t}\estruturacomplexa^{t} U_{2}+i(\underbrace{ V_{1}^{t}\estruturacomplexa^{t} U_{2}}_{\formaSimpletica{v_{1}}{u_{2}}} - V_{2}^{t}\estruturacomplexa^{t} U_{1})
		\\
		&= \underbrace{\formaSimpletica{v_{1}}{u_{1}} -\formaSimpletica{v_{2}}{u_{2}}}_{g(v,u)}+i(\underbrace{ \formaSimpletica{v_{1}}{u_{2}}- \formaSimpletica{v_{2}}{u_{1}}}_{k(v,u)})
		\\
		&= g(v,u)+ik(v,u),
		\end{aligned}
		$$
		onde foi feito  $v_{j}, u_{j}\in V$ e $V_{j}, U_{j}$ sendo a representação matricial de cada um deles para $j \in \{1,2\}$.
		Tem-se que $g,k:\complexificado{V}\times \complexificado{V} \to \reta$ são aplicações $\complexo{}$-bilineares e anti-simétricas pois são combinações lineares de aplicações $\complexo{}$-bilineares anti-simétricas. Pelo mesmo argumento, $g,k$ são não-degeneradas.
	\end{prova}
	
	\begin{proposicao}\label{proposicao_base_simpletica_conjugada}
		Seja $B=\{\textbf{e}, \textbf{f} \}$ uma base simplética de $\complexificado{V}$. Então $\overline{B}=\{\overline{\textbf{e}}, \overline{\textbf{f}}\}$ é uma base simplética de $\complexificado{V}$.
	\end{proposicao}
	\begin{prova}
		Tem-se que a aplicação de conjugação $\complexificado{V} \ni z\otimes v \mapsto \overline{z}\otimes v \in \complexificado{V}$ é um isomorfismo. Logo, leva base em base. Como $B$ é uma base simplética, tem-se $\Omega(\textbf{e}_{i}, \textbf{e}_{j}) = \Omega(\textbf{f}_{i}, \textbf{f}_{j}) =0$ e $\Omega(\textbf{e}_{i}, \textbf{f}_{j}) = \delta_{ij}$. Com isso, fazendo $\textbf{e}_{i} = \complexificacaoelemento{z_{i}}{v_{i}}$, e pela Proposição $\ref{proposicao_complexificacao_espaco_simpletico}$ se tem $\Omega(\overline{\textbf{e}}_{i}, \overline{\textbf{e}}_{j}) =	\overline{z_{i}}\overline{z_{j}}\formaSimpletica{v_{i}}{v_{j}}=\overline{z_{i}z_{j}\formaSimpletica{v_{i}}{v_{j}}} = \overline{\Omega(\textbf{e}_{i}, \textbf{e}_{j})} = 0$. Com argumento análogo pode-se ver que $\Omega(\overline{\textbf{e}}_{i}, \overline{\textbf{e}}_{j}) =0$ e $\Omega(\overline{\textbf{e}}_{i}, \overline{\textbf{e}}_{j}) =\delta_{ij}$. Logo, $\overline{B}$ é uma base simplética.
	\end{prova}
	
	Um operador linear $T : \complexificado{V} \to \complexificado{V}$ pode ser escrito como $T=\complexificacaoelemento{\lambda}{A}$, tal que $T(v) = \complexificacaoelemento{\lambda}{A}(\complexificacaoelemento{z}{\alpha}) = \complexificacaoelemento{\lambda z}{A\alpha}$, onde $\lambda \in \complexo{}$ e $A$ é um operador linear $A:V \to V$. Com isso, $T$ está bem definido.
	
	\begin{definicao}
		(Transformação simplética) Seja $(\complexificado{V}, \Omega)$ um 2n-espaço vetorial simplético sobre $\complexo{}$. Um operador linear $T: \complexificado{V} \to \complexificado{V}$ é uma transformação simplética se 
		$$
		\formaSimpleticaExtendida{Tu}{Tv} = \formaSimpleticaExtendida{u}{v}
		$$ para todo $u,v\in \complexificado{V}$.
	\end{definicao}
	
	Suponha que $T=\complexificacaoelemento{1}{A}$ para algum operador linear $A:V \to V$. Afirmo que se $T$ é uma transformação simplética, então $A$ é uma transformação simplética. De fato, tomando $u = \complexificacaoelemento{x}{\alpha}$ e $v=\complexificacaoelemento{y}{\beta}$ em $\complexificado{V}$ tem-se que $Tu = \complexificacaoelemento{x}{A\alpha}$ e $Tv= \complexificacaoelemento{y}{A\beta}$
	$$
	\begin{aligned}
	\Omega(Tv,Tu)
	&= \sum_{i,j = 1}^{2n}
	\omega^{ij}\textbf{a}^{*}_{i}\otimes \textbf{a}^{*}_{j}(Tv,Tu)
	\\
	&= 
	\sum_{i,j = 1}^{2n}
	\omega^{ij}\textbf{a}^{*}_{i}\otimes \textbf{a}^{*}_{j}(\complexificacaoelemento{x}{A\alpha},\complexificacaoelemento{y}{A\beta})
	\\
	&= \sum_{i,j = 1}^{2n}
	xy\omega^{ij}e^{*}_{i}(Au)e^{*}_{j}(Av)
	\\
	&= xy
	\formaSimpletica{A\alpha}{A\beta}.
	\end{aligned}
	$$  
	Tem-se que $\formaSimpleticaExtendida{Tu}{Tv} = \formaSimpleticaExtendida{u}{v}$ o que implica que $xy
	\formaSimpletica{A\alpha}{A\beta} = xy
	\formaSimpletica{\alpha}{\beta}$. Como $\alpha, \beta \in V$ são arbitrários, então $A$ é uma transformação simplética. A recíproca é imediata.
	
	Adotando a notação da convencionada na Seção \ref{secao_espacos_vetoriais_simpleticos}, pode-se enunciar o seguinte resultado:
	
	\begin{proposicao}\label{proposicao_forma_simpletica_vetor_conjugado}
		Seja $\{\textbf{e}, \textbf{f}\}$ uma base simplética de $\complexificado{V}$. Então para um dado $v = v_{(1)}\textbf{e}+v_{(2)}\textbf{f} \in \complexificado{V}$ tem-se que $\formaSimpleticaExtendida{\overline{v}}{v}
		=i2\parteImaginaria{\produtointerno{\overline{v}_{(1)}}{v_{(2)}}} \in i\reta$.
	\end{proposicao}
	\begin{prova}
		Seja $\{\textbf{e}, \textbf{f} \}$ uma base simplética de $\complexificado{V}$. Tomando $v = v_{(1)}\textbf{e}+v_{(2)}\textbf{f}$, tem-se
		$$
		\begin{aligned}
		\formaSimpleticaExtendida{\overline{v}}{v} 
		&= \produtointerno{\overline{v}_{(1)}}{v_{(2)}}- \produtointerno{\overline{v}_{(2)}}{v_{(1)}}
		\\
		&= \produtointerno{\overline{v}_{(1)}}{v_{(2)}}- \overline{\produtointerno{v_{(2)}}{\overline{v_{(1)}}}}
		\\
		&= \produtointerno{\overline{v}_{(1)}}{v_{(2)}}-\overline{ \produtointerno{\overline{v}_{(1)}}{v_{(2)}}}
		\\
		&=i2\parteImaginaria{\produtointerno{\overline{v}_{(1)}}{v_{(2)}}}\in i\reta,
		\end{aligned}
		$$
		onde foi usada a notação da Seção \ref{secao_espacos_vetoriais_simpleticos}.
	\end{prova}
		
	\begin{definicao}
		(Grupo simplético complexo) O grupo simplético complexo $\gruposimpletico{\complexificado{V}} \subset \generalgroupcomplexo{2n}$ de $\complexificado{V}$ é o conjunto das matrizes associadas as transformações simpléticas definidas em $\complexificado{V}$.
	\end{definicao}
	
	\chapter{Preliminares de Álgebra}
	
	Nessa seção, serão apresentados alguns resultados preliminares necessários para o estudo da topologia de $\gruposimpletico{2n}$ e do conjunto das estruturas complexas $\estruturascomplexas{V}{\omega}$.
	
	
	Sejam $V$ um n-espaço vetorial real, $B= \{e_{j}\}_{j=1}^{n}$ uma base ordenada e $f:V\times V\to \reta$ uma aplicação bilinear. Definindo $A$ como a matriz de $f$ na base $B$, isto é, $A_{ij} = f(e_{i}, e_{j})$, pode-se mostrar que $f$ é positiva-definida se, e somente se, a função $g_{A}: M_{n\times 1}(\reta)\times M_{n\times 1}(\reta) \to \reta$, definida por $
	g_{A}(U,V) = U^{t}AV$, é positiva-definida.
	
	\begin{definicao}
		(k-subdeterminante) Seja $A \in \matrizquadreal{n}$. O seu k-subdeterminante é
		$$
		det_{k}(A) =
		det \left(
		\begin{array}{ccc}
		A_{11} & \dots & A_{1k}
		\\
		\vdots & \ddots & \vdots
		\\
		A_{k1} & \dots & A_{kk}
		\end{array}
		\right),\;\; 1\leq k \leq n.
		$$
	\end{definicao}
	
	\begin{teorema}\label{teorema_matriz_positiva_definida}
		Sejam $V$ um n-espaço vetorial real, $f: V\times V\to \reta$ uma aplicação bilinear e $A$ a matriz de $f$ na base ordenada $B$. Então $f$ é positiva-definida se, e somente se, $A=A^{t}$ e $det_{k}(A)>0$ para $1\leq k\leq n$.
	\end{teorema}
	
	\begin{observacao}\label{observacao_matriz_positiva_definida}
		Do teorema anterior pode-se afirmar que, se $A$ for diagonalizável, todos os seus auto-valores serão positivos.
	\end{observacao}
	
	Note a relação $f(u, v) = \sum_{i, j}u_{i}v_{j}f(e_{i}, e_{j}) = \sum_{i, j}u_{i}v_{j}A_{ij} = U^{t}AV = g_{A}(U, V)$, onde $U, V$ são as representações matriciais de $u$ e $v$, respectivamente. Assim, tem-se a seguinte definição.
	
	\begin{definicao}\label{definicao_matriz_positiva_definida}
		(Matriz positiva-definida) Se uma matriz $A \in \matrizquadreal{n}$ é tal que $g_{A}(U, V)\geq 0$ para todos $U,V \in M_{n\times 1}$, então diz-se que $A$ é positiva-definida. O conjunto de todas as matrizes em $\matrizquadreal{n}$ positivas-definidas é denotado por $\matrizsimetricapositiva{n}$ .
	\end{definicao}
	
	Supondo que $\espectrooperador{A}$ seja o espectro de $A$, então $\autoespaco{\lambda}$ é o auto-espaço de $A$ associado a $\lambda \in \espectrooperador{A}$.
	
	\begin{definicao}\label{definicao_potenciacao_matriz}
		(Potênciação de matriz) Sejam $V$ um n-espaço vetorial real, $A:V \to V$ um operador linear diagonalizável. Dado $\alpha \in \real{}$ o operador linear $A^{\alpha}:V \to V$ é o operador definido por $A^{\alpha} = B^{-1}diag\{ \lambda_{1}^{\alpha} , \dots, \lambda_{2n}^{\alpha} \}B$, onde $B$ é a matriz tal que $A= B^{-1}diag\{ \lambda_{1}, \dots, \lambda_{2n}\}B$.
	\end{definicao}
	
	\begin{observacao}
		Note que, para um dado $v \in E_{\lambda}$, tem-se $A^{\alpha}v = \lambda^{\alpha}v$, logo $v \in E_{\lambda^{\alpha}}$. Portanto, $E_{\lambda}\subset E_{\lambda^{\alpha}}$.
	\end{observacao}
	
	\begin{observacao}\label{observacao_transposta_potenciacao_matriz}
		Denotando $D = diag\{ \lambda_{1}^{\alpha} , \dots, \lambda_{2n}^{\alpha} \}$, tem-se da definição que
		$$
		(A^{\alpha})^{t} = B^{t}D^{t}(B^{-1})^{t} = B^{t}D(B^{t})^{-1} = (A^{t})^{\alpha}.
		$$
	\end{observacao}
	
	\begin{definicao}
		(Matriz ortogonal) O conjunto $\matrizortogonal{n} =\{ A \in \generalgroupreal{n}: AA^{t}=Id \}$ é denominado conjunto das matrizes ortogonais.
	\end{definicao}
	
	\begin{observacao}
		O conjunto das matrizes ortogonais forma um grupo com a operação de multiplicação de matrizes.
	\end{observacao}
	
	O resultado seguinte é uma caracterização de matrizes normais  utilizando matrizes ortogonais e matrizes positivas definidas. Sua demonstração pode ser encontrada em $\cite{hoffman_kunze}$.
	
	\begin{lema}\label{lema_caracterizacao_matriz_normal}
		(Caracterização matriz normal) Seja $A\in \matrizquadreal{n}$ uma matriz normal, isto é, $A^{t}A=AA^{t}$. Então existem uma matriz diagonal $D \in \matrizquadreal{n}$ positiva-definida e uma matriz $O\in \matrizortogonal{n}$ tais que $A=ODO^{t}$. Nesse caso, seus auto-valores são positivos. 
	\end{lema}
	
	\begin{observacao}\label{observacao_caracterizacao_matriz_normal}
		No caso de uma matriz complexa $A \in M_{n\times n}(\mathbb{C})$ tem-se a decomposição $A=U^{*}DU$, onde $U \in \matrizunitaria{n}$.
	\end{observacao}
	
	\begin{lema}\label{lema_raiz_matriz_normal}
		(Raíz de matriz normal) Sejam $V$ um n-espaço vetorial e $A\in \matrizquadreal{n}$ uma matriz normal, então existe uma única $P\in \matrizsimetricapositiva{n}$ tal que $A=P^{2}$. 
	\end{lema}
	\begin{prova}
		Como $A$ é normal, então pelo Lema $\ref{lema_caracterizacao_matriz_normal}$ pode-se escrever $A=ODO^{t}$, onde $D=(D_{ii})$ é uma matriz diagonal com entradas positivas e $O\in \matrizortogonal{n}$. Logo se pode definir $C \in \generalgroupreal{n}$ como sendo $C = (\sqrt{D_{ii}})$, o que implica em $C^{2} = D$. Definindo $P = OCO^{t}$ se tem $P^{2} = OCO^{t}OCO^{t} = OC^{2}O^{t} = ODO^{t}=A$. Tem-se que $P^{t} = (OCO^{t})^{t} = OCO^{t} = P$, pois $C$ é diagonal. Além disso, $P$ é semelhante a uma matriz diagonal, então pelo Teorema $\ref{teorema_matriz_positiva_definida}$ se tem que $P \in \matrizsimetricapositiva{n}$. A unicidade vem do fato de que $C^{2} = D$ é única, logo $P$ é única.
	\end{prova}
	
	\begin{observacao}\label{observacao_raiz_matriz_normal}
		A matriz do Lema $\ref{lema_raiz_matriz_normal}$ é chamada raíz de $A$ e é denotada por $P=\sqrt{A}$.
	\end{observacao}
	
	O seguinte teorema é de grande importância pois é fundamental na caracterização do grupo simplético e investigação de sua topologia. Um caso mais geral que pode ser encontrado em $\cite{hoffman_kunze}$.
	
	\begin{teorema}\label{teorema_decomposicao_polar}
		(Decomposição polar) Se $(V, \produtointerno{}{})$ é um n-espaço vetorial real com um produto interno positivo-definido e $A \in \generalgroupreal{n}$, então pode-se escrever $A=PO$ onde $P \in  \matrizsimetricapositiva{n}$ e $O \in \matrizortogonal{n}$. Além disso, essa decomposição é única.
	\end{teorema}
	\begin{prova}
		Por construção tem-se que $AA^{t}$ é normal. Então pelo Lema $\ref{lema_raiz_matriz_normal}$ existe uma única $P \in \matrizsimetricapositiva{n}$ tal que $P^{2} = AA^{t}$, isto é, $P = \sqrt{AA^{t}}$. Como $P$ é invertível, pode-se definir $O = P^{-1}A$. Veja que $OO^{t} = P^{-1}AA^{t}(P^{-1})^{t} = P^{-1}AA^{t}(P^{t})^{-1} = P^{-1}AA^{t}P^{-1} = P^{-1}P^{2}P^{-1} = Id$, logo $O \in \matrizortogonal{n}$. Pela unicidade de $P$ tem-se que $O=P^{-1}A$ é única. Portanto $A=PO$, onde $P \in \matrizsimetricapositiva{n}$ e $O \in \matrizortogonal{n}$ são únicas.
	\end{prova}
	
	\begin{corolario}\label{corolario_decomposicao_matriz_antisimetrica}
		(Decomposição polar anti-simétrica) Considere as hipóteses no Teorema $\ref{teorema_decomposicao_polar}$ e tome $A \in \generalgroupreal{n}$ uma matriz anti-simétrica. Então $A = PO$, onde $P \in \matrizsimetricapositiva{n}$, $O \in \matrizortogonal{n}$ com  $O^{2} = -Id$ e $O^{t} = -O$.
	\end{corolario}
	\begin{prova}
		Tem-se que $AA^{t}$ é uma matriz normal, logo pelo Lema $\ref{lema_caracterizacao_matriz_normal}$ é diagonalizável e é possível escrever $AA^{t} = B^{-1}diag\{\lambda_{1}, \dots, \lambda_{n}\}B$. Logo tem-se 
		$$
		(AA^{t})^{1/2} =B^{-1}diag\{\lambda_{1}^{1/2}, \dots, \lambda_{n}^{1/2}\}B.
		$$
		Portanto, $P=(AA^{t})^{1/2}$ é positiva-definida, logo $P=P^{t}$. Sejam $D_{P}$ e $D_{A}$ as matrizes diagonais semelhantes a $P$ e $A$, respectivamente. Afirmo que $PA = AP$. De fato, $PA = B^{-1}D_{P}B B^{-1}D_{A}B = B^{-1}D_{A} D_{P}B = B^{-1}D_{A}B B^{-1}D_{P}B = AP$. Definindo $O = P^{-1}A$ tem-se $OO^{t} = P^{-1}A(P^{-1}A)^{t} = P^{-1}AA^{t}(P^{t})^{-1} = P^{-1}AA^{t}P^{-1} = P^{-1}P^{2}P^{-1}  = Id$, portanto $O \in \matrizortogonal{n}$. Além disso, $O^{2} = P^{-1}AP^{-1}A = P^{-2}A^{2} = (AA^{t})^{-1}A^{2} = -Id$, onde foram usados a anti-simetria e o fato de que $P^{-1}$ comuta com $A$. Pelo Teorema $\ref{teorema_decomposicao_polar}$ a decomposição $A=PO$ é única.
	\end{prova}
	
	Seja $T:V\to V$ um operador linear e $V$ um n-espaço vetorial sobre os complexos. Suponha que $A \in \matrizquadcomplexa{n}$ seja sua representação matricial. Mesmo que $A$ seja invertível, isso não garante que $T$ seja diagnalizável. Contudo, sob algumas hipósteses, pode-se escrever $A$ de um modo que se assemelha a uma matriz diagonal, e com isso, decompõe-se $V$ como soma direta de subespaços vetoriais construidos a partir de $T$. Essa construção é chamada forma canônica de Jordan e será utilizada na análise dos auto-valores do grupo simplético.
	
	A diagonalização de um dado operador linear $T:V\to V$ nem sempre é garantida. Para contornar esse dificuldade técnica será utilizada a forma canônica de Jordan de $T$ na construção de uma generalização de auto-espaços desse operador.
	
	\begin{teorema}\label{teorema_espectral_jordan}
		(Teorema Espectral) Sejam $T:V \to V$ um operador linear e $V$ um n-espaço vetorial complexo. Suponha que o polinômio característico de $T$ seja
		$$
		p_{T}(\lambda) = (\lambda - \lambda_{1})^{n_{1}}\dots (\lambda - \lambda_{k})^{n_{k}}
		$$
		em que $\lambda_{j} \in \sigma(T)$ sejam todos distintos. Então existem subespaços $V_{j} \subset V$ com $1\leq j \leq k$ invariantes por $T$ tais que 
		$$
		V = E_{1}\oplus \dots \oplus E_{k}.
		$$
		Além disso, $dim(E_{j}) = n_{j}$,  o polinômio de $T|_{E_{j}}$ é $m_{j}(\lambda) = (\lambda - \lambda_{j})^{m_{j}}$ e $E_{j} = Ker(T-\lambda_{j})^{m_{j}}$, em que $1\leq m_{j}\leq n_{j}$ é o comprimento do maior bloco de Jordan associado ao auto-valor $\lambda_{j}$.
	\end{teorema}
	
	\begin{definicao}\label{definicao_autoespaco_generalizado}
		(Auto-espaços generalizados) Sejam $A$ como no teorema anterior e $\lambda \in \sigma(A)$. O $\lambda-$auto-espaço de generalizado de $A$ é o conjunto $E_{\lambda} = \bigcup_{r \in \mathbb{N}} Ker(A - \lambda)^{r}$.
	\end{definicao}
	
	\chapter{Fibrados Vetoriais}\label{apendice_fibrados}
	
	Mais detalhes sobre fibrados vetoriais podem ser encontrados em \cite{husemoller_fibrados}, \cite{nakahara} e \cite{steenrod_fibre_bundles}.
	
	\begin{definicao}
		Um n-fibrado vetorial diferenciável $\eta$ sobre um espaço topológico $B$ é a quíntupla $\eta = (E, B, \pi, F, G)$ em que
		
		\begin{enumerate}
			\item Uma variedade diferenciável $E$ chamada espaço total.
			
			\item Uma variedade diferenciável $B$ chamada espaço base.
			
			\item Um n-espaço vetorial $F$ chamado fibra.
			
			\item Uma aplicação sobrejetora $\pi: E\to B$ chamada projeção. A pré-imagem $\pi^{-1}(b) = F_{b} \cong F$ chamada fibra em $b\in B$. 
			
			\item Um grupo de Lie $G$ chamado grupo de estrutura, o qual age à esquerda em $F$.
			
			\item Existe uma cobertura por abertos $\{U_{i}\}_{i\in I}$ de $B$ e uma família de difeomorfismos $\phi_{i}: U_{i}\times F\to \pi^{-1}(U_{i})$ tal que $(\pi\circ \phi_{i})(b,f) = b$. A aplicação $\phi_{i}$ é chamada trivialização local.
			
			\item A aplicação $\phi_{i,b}: F\to F_{b}$ definida tal que $\phi_{i,b}(f) = \phi_{i}(b,f)$ é um difeomorfismo. Na intersecção $U_{i}\cap U_{j}\neq \emptyset$ é exigido que $t_{ij}(b)=\phi_{i,b}^{-1}\circ \phi_{j,b}:F\to F$ é um elemento de $G$. Então $\phi_{i}$ e $\phi_{j}$ estão relacionados pela aplicação suave $t_{ij}: U_{i}\cap U_{j}\to G$ tal que $\phi_{j}(b, f) = \phi_{i}(b, t_{ij}(b)f)$. As aplicações $t_{ij}$ são chamadas funções de transição.
		\end{enumerate}
		
	\end{definicao}
	
	Um n-fibrado vetorial diferenciável $\eta$ é um fibrado trivial quando todas as funções de transição coincidem com a aplicação identidade. Nesse caso o espaço total $E$ é homeomorfo a $B\times F$.
	
	\begin{exemplo}\label{exemplo_fibrado_circulo}
		Seja $\eta=(E, \circulo,\pi, G \reta)$ um 1-fibrado vetorial com as fibras sendo a reta $\reta$ e $G$ o grupos das transformações lineares em $\reta$ que preservam sua orientação. Sejam $U_{1} = (0,2\pi)$ e $U_{2} = (-\pi,\pi)$ uma cobertura de $\circulo$. Então $U_{1}\cap U_{2} =(0,\pi)\cup(\pi,2\pi)$. As trivializações locais são 
		$$
		\phi^{-1}_{1}(b) = (\theta, t),\; 		\phi^{-1}_{2}(b) = (\theta, t)
		$$
		para $\theta\in U_{1}\cap U_{2}$ e $t\in \reta$. A função de transição $t_{12}(\theta) = \theta$. Portanto $t_{12}=Id$ e $\eta$ é um fibrado trivial.
	\end{exemplo}
	
	Daqui em diante o grupo de estrutura será $G = GL(n,\real{})$ e um n-fibrado vetorial será denotado por $\eta = (E,B,\pi)$.
	
	\begin{exemplo}
		(1-Fibrado vetorial) Sejam $E=S^{1} \times \real{}$, $B=S^{1}$, $F_{b} = \real{}$ para qualquer $b \in S^{1}\subset \real{2}$ e $\pi:S^{1} \times \real{}\to S^{1}$ tal que $\pi(b, e)=b \in S^{1}$. Então $\eta=(S^{1} \times \real{}, S^{1}, \pi)$ é um fibrado vetorial. Além disso, $\eta$ é um fibrado vetorial trivial pois, tomando os abertos $S^{+} = S^{1} \backslash \{(0,1)\}$ e $S^{-} = S^{1} \backslash \{(0,-1)\}$, tem-se as projeções $\pi^{-1}(S^{\pm}) = S^{\pm} \times \real{} \subset E$, e com isso, $(S^{+}\times\real{}) \cup ( S^{-}\times\real{}) = (S^{+}\cup S^{-})\times\real{} = S^{1} \times\real{} =E$.
	\end{exemplo}
	
	\begin{exemplo}
		(Fibrado tangente) Sejam $M$ uma variedade suave n-dimensional. Definindo $B=M$, dado $q\in M$ se tem as fibras $F_{q} = T_{q}M$ e o espaço total é $TM=\{(q, v): q\in M, v\in T_{q}M \}$. A tripla $\eta = (TM, M, \pi)$, denominada fibrado tangente de $M$, é localmente trivial, portanto é um fibrado vetorial.
	\end{exemplo}

	Os fibrados tangentes $\eta$ serão denotados por $TM$.
 
 	Seja $TM$ um $n$-fibrado tangente de uma $n$-variedade diferenciável $M$. Uma seção de $TM$ é uma aplicação suave $s:M \to TM$ tal que $\pi\circ s=Id$.
	
	Os campos vetoriais definidos em $M$ podem ser vistos como seções do fibrado tangente $TM$ do seguinte modo: tome $X\in \campossuaves{M}$ e defina $s_{X}$ a seção do fibrado tangente por $s_{X}(b) = (b, X(b))$.
	
	\begin{definicao}
		(Pullback de fibrado) Sejam $B, B'$ dois espaços topológicos, $\eta=(E', B', \pi')$ um n-fibrado vetorial e $f:B\to B'$ uma aplicação contínua. Defina $E \subseteq B\times E'$ como sendo o conjunto $\{(b, e): b \in B,\; e \in \pi'^{-1}(f(b)) \}$. Além disso, defina o mapa de projeção $\pi:E\to B$ tal que $\pi(b,e) = b$. O pullback de $\eta$ por $f$ é a tripla $f^{*}\eta = (E,B, \pi)$.
	\end{definicao}
	
	\begin{observacao}\label{observacao_pullback_fibrado_tangente}
		Seja $\gamma:\reta \to M$ um caminho contínuo na variedade $M$. Então o pullback $\pullbackfibradotangenteM{\gamma}$ é o conjunto $\{(\gamma(t), X(\gamma(t)): X\in \campossuaves{M} \; \text{e}\; t \in \reta \}$. Note que, $\pullbackfibradotangenteM{\gamma}$ é o $2n$-fibrado vetorial cujas fibras são $\espacotangenteponto{\gamma(t)}{M}$ para cada $t\in \reta$ e o espaço base é $\gamma(\reta)\subset M$. De fato, se $\pi: \pullbackfibradotangenteM{\gamma} \to \gamma(\reta)$ e o mapa de projeção desse fibrado, então $\pi(\gamma(t), Y(\gamma(t))) = \gamma(t)$ e $\pi^{-1}(\gamma(t)) = \espacotangenteponto{\gamma(t)}{M}$ para todo $t\in \reta$. Com isso, toda seção $s_{X}: \gamma(\reta)\to \pullbackfibradotangenteM{\gamma}$ dada por $s_{X}(\gamma(t)) = (\gamma(t), X(\gamma(t)))$, pode ser vista como a restrição de alguma seção $s_{X'}: M\to TM$ ao subconjunto $\gamma(\reta) \subset M$, isto é, $s_{X}= s_{X'}|_{\gamma}$. Logo, $X=X'|_{\gamma}$.
	\end{observacao}
	
	\begin{lema}\label{pullback_composicao}
		(Composição de pullbacks) Sejam $\eta =(E, C, \pi)$ um n-fibrado vetorial, $A, B, C$ espaços topológicos, $f:A\to B$ e $g:B\to C$ funções contínuas. Então $(g\circ f)^{*} \eta= f^{*}(g^{*}\eta)$.
	\end{lema}
	\begin{prova}
		Como $g\circ f:A\to C$ é contínua, logo $(g\circ f)^{*}\eta = (E_{g\circ f}, A, \pi_{g\circ f})$ é um n-fibrado vetorial onde as fibras são dadas por $F_{(g\circ f)(a)} = \pi^{-1}((g\circ f)(a))$. Tem-se que $g^{*}\eta=(E_{g}, B, \pi_{g})$, $f^{*}(g^{*}\eta) = (E_{f}, A, \pi_{f})$, e suas fibras são dadas por $F_{g(b)} = \pi^{-1}(g(b))$, sendo que $b=f(a)$, logo  $F_{g(f(a))} = \pi^{-1}((g \circ f)(a))$, que é uma fibra de $(g\circ f)^{*}\eta$. Como o espaço base, as fibras e as projeções de $(g\circ f)^{*}\eta$ e $f^{*}(g^{*}\eta)$ coincidem, então $f^{*}(g^{*}\eta) = (g\circ f)^{*}\eta$.
	\end{prova}
	\begin{lema}\label{pullback_trivial}
		(Pullback trivial) Se $\eta = (E, B, \pi)$ é um n-fibrado vetorial trivial, $A$ um espaço topológico e $f:A\to B$ uma função contínua, então $f^{*}\eta$ é trivial.
	\end{lema}
	
	Os seguintes resultados nos dão a relação entre os fibrados vetoriais obtidos pelo pullback de aplicações homotópicas e podem ser encontrados em \cite{nakahara} e \cite{steenrod_fibre_bundles}.
	
	Sejam $ \eta'=(E',B',\pi'), \eta=(E,B,\pi)$ dois fibrados vetoriais.  Uma aplicação suave $\overline{f}:E'\to E$ é chamada mapa de fibrados se $\overline{f}$ é uma sobrejeção de cada fibra $F'_{p}$ de $\eta'$ em cada fibra $F_{q}$ de $\eta$. Então $\overline{f}$ induz uma aplicação suave $f:B'\to B$ tal que $f(p)=q$. Com isso, tem-se a comutatividade $\pi\circ \overline{f} = f\circ\pi'$, ilustrado no diagrama a seguir. 
	$$
	\xymatrix{
		E'\ar[d]_{\pi'}\ar[r]^{\overline{f}} & E\ar[d]^{\pi}
		\\
		B'\ar[r]^{f} & B
	}
	$$
	
	
	Sejam $\eta'$ e $\eta$ dois fibrados vetoriais sobre o mesmo espaço base $B$. Então $\eta'$ e $\eta$ são ditos fibrados vetoriais equivalentes se a aplicação induzida pelo mapa de fibrados $\overline{f}$ é a identidade, isto é, $f=Id_{B}$.
	
	\begin{teorema}\label{pullback_isomorfismo}
		Sejam $A, B$ dois espaços topológicos, $\eta=(E, B, \pi)$ um n-fibrado vetorial e $f,g: A\to B$ duas aplicações homotópicas. Então $f^{*}\eta $ e $g^{*}\eta$ são fibrados vetoriais equivalentes sobre $A$.
	\end{teorema}
	
	\begin{corolario}\label{pullback_contratil}
		Sejam $\eta$ um n-fibrado vetorial sobre um espaço base contrátil. Então $\eta$ é trivial.
	\end{corolario}
	\begin{prova}
		Sejam $B$ um compacto contrátil, $\eta=(E, B, p)$ um n-fibrado vetorial e $\{*\} \subset B$ um conjunto unitário. Como $B$ é contrátil, então existem $f:B\to \{*\}$ e $g:\{*\}\to B$ tais que $f\circ g = Id_{\{*\}}$ e $g\circ f \simeq Id_{B}$. Pela definição tem-se que $g^{*}\eta = (E', \{*\}, p')$ é um n-fibrado trivial e pelo Teorema $\ref{pullback_isomorfismo}$ tem-se $(g\circ f)^{*}\eta $ é equivalente a $ Id_{B}^{*}\eta = \eta$. Pelo Lema $\ref{pullback_composicao}$ $(g\circ f)^{*}\eta = f^{*}(g^{*}\eta) $ é trivial pois $g^{*}\eta$ é trivial pelo Lema $\ref{pullback_trivial}$, logo $ \eta $ é equivalente a $f^{*}(g^{*}\eta)$ e $\eta$ é trivial.
	\end{prova}
	
	\begin{corolario}\label{corolario_fibrado_sobre_Dn}
		Se $\eta$ é um n-fibrado vetorial sobre $D^{n} = \{ x\in \real{n}:\norma{x}\leq 1\}$, então $\eta$ é trivial.
	\end{corolario}

	\chapter{Variedades Riemannianas}\label{apendice_variedade_riemanniana}
	
	Essa seção contém algumas definições e resultados em geometria Riemanniana que serão utilizados em diferentes pontos do trabalho. Uma explanação detalhada pode ser encontrada em \cite{manfredo_riemannian_geo}.
	
	\begin{definicao}\label{definicao_variedade_riemanniana}
		(Variedade Riemanniana) Seja $M$ uma n-variedade diferenciável. Uma métrica Riemanniana é uma 2-forma simétrica $g \in \Omega^{2}(M)$ tal que $g_{p}:T_{p}M \times T_{p}M \to \real{}$ é um produto interno positivo-definido para todo $p \in M$. O par $(M, g)$ é chamado n-variedade Riemanniana.
	\end{definicao}
	
	O conceito de conexão afim está intimamente relacionado com a forma de comparar um campo vetorial avaliado em pontos distintos da variedade. Uma das estratégias de se efetuar essa comparação é chamada de transporte paralelo e uma boa discussão pode ser encontrada em \cite{nakahara}. Além disso, pode-se mostrar que a conexão afim é uma generalização do conceito de diferenciação usual.
	
	\begin{definicao}\label{definicao_conexao_afim}
		(Conexão afim) Uma conexão afim $\nabla$ definida em uma n-variedade diferenciável é a aplicação $\nabla: \campossuaves{M} \times \campossuaves{M} \to \campossuaves{M}$ tal que, dadas $f,h \in \funcoessuaves{M}$ e $X,Y,Z \in \campossuaves{M}$:
		\begin{enumerate}
			\item $\nabla_{fX+hY}Z = f\nabla_{X}Z+h\nabla_{Y}Z$
			\item $\nabla_{X}(Y+Z) = \nabla_{X}Y+ \nabla_{X}Z$
			\item $\nabla_{X}(fY) = X(f)Y+f\nabla_{X}Y$.
		\end{enumerate}
	\end{definicao}
	
	\begin{observacao}\label{observacao_conexao_afim}
		Sejam $(U, \phi)$ uma carta local em $p\in M$, onde $(x_{1}(q), \dots, x_{n}(q))$ é a parametrização local, e $\{\partial_{1}, \dots, \partial_{n}\}$ referencial ortonormal local, onde $\partial_{j} = \partial/\partial x_{j}$. Pode-se mostrar que, dados $X=\sum X_{j}\partial_{j}, Y=\sum Y_{j}\partial_{j} \in \campossuaves{M}$, tem-se que
		$$
		\begin{aligned}
		\nabla_{X}Y &= 
		\sum_{k=1}^{n} \Big( \sum_{i,j=1}^{n} X(Y_{k}) + X_{i}Y_{j} \Gamma^{k}_{ij}\Big)\partial_{k} 
		\\
		&= 
		\sum_{k=1}^{n} \Big( \sum_{i,j=1}^{n} X_{i} (\partial_{i}(Y_{k}) + Y_{j} \Gamma^{k}_{ij})\Big)\partial_{k} 
		\\
		&= \sum_{k=1}^{n} (\nabla_{X}Y)^{k}\partial_{k}.
		\end{aligned} 
		$$
		
		Com isso, tomando $X = \partial_{i}$, o operador linear $\nabla_{\partial_{i}}: \campossuaves{M} \to \campossuaves{M}$, dado por 
		$$
		\nabla_{\partial_{i}}Y = \sum_{k,j=1}^{n}  (\partial_{i}(Y_{k}) + Y_{j} \Gamma^{k}_{ij})\partial_{k},
		$$
		generaliza o conceito de derivada direcional em espaços euclidianos. De fato, supondo $M=\real{n}$, pode-se mostrar que os símbolos $\Gamma^{k}_{ij}$ são funções suaves e $\Gamma^{k}_{ij}(p)=0$ para todo $p \in \real{n}$ (veja em \cite{manfredo_riemannian_geo}). Portanto, 
		$$
		\nabla_{\partial_{i}}Y = \sum_{k=1}^{n}  \partial_{i}(Y_{k})\partial_{k}.
		$$
	\end{observacao}
	
	Sejam um $\gamma:I\subset\reta\to M$ caminho suave definido no intervalo $I$ e $v=\derivada{\gamma}{t}\in \campossuaves{M}$ seu campo de velocidades. Um campo $X \in \campossuaves{M}$ é dito paralelamente transportado ao longo de $\gamma$ se $\nabla_{v}X=0$. Além disso, o caminho $\gamma$ é chamado geodésica em $t_{0} \in \reta$ se o seu campo de velocidades é paralelamente transportado ao logo dele mesmo no ponto $t=t_{0}$, isto é, $\nabla_{v}v=0$. Se $\gamma$ é uma geodésica em todo $t\in \reta$, então diz-se que $\gamma$ é uma geodésica. Pode-se mostrar que se $\gamma$ é uma geodésica, então $\norma{v(t)} = \sqrt{g_{\gamma(t)}(v(t), v(t))}$ é constante.
	Mais detalhes sobre podem ser encontrados em \cite{manfredo_riemannian_geo} e \cite{nakahara}.
	
	O comprimento de arco de um caminho suave $\gamma:\intervalo\to M$ é definido por 
	$$
	L(\gamma) =  \int_{0}^{1}\normagrande{\derivada{\gamma}{t}(t)}dt.
	$$
	Se o caminho $\gamma$ é uma geodésica, então $L(\gamma) =\norma{v}$.
	
	A seguinte proposição é necessária para a definição da aplicação exponencial que será apresentada a seguir e pode ser encontrada em \cite{manfredo_riemannian_geo}.
	
	\begin{proposicao}\label{proposicao_existencia_geodesica}
		Dado $p\in M$, existem uma vizinhança $V$ de $p$ em $M$, um número $\epsilon>0$, $U=\{(q, v) \in TM: q\in V,\;v\in \espacotangenteponto{q}{Y}\;\text{e}\;\norma{v}<\epsilon \}$ e uma aplicação suave $\gamma:(-2,2)\times U \to M$ tal que $t\mapsto \gamma(t, q, v)$, é a única geodésica de $M$ onde $\gamma(0, q, v)=q$, $v=\derivada{}{t}\gamma(0,q,v)$ para todo $q\in V$ e todo $v\in \espacotangenteponto{q}{M}$ com $\norma{v}<\epsilon$.
	\end{proposicao}
	
	\begin{definicao}\label{definicao_aplicacao_exponencial}
		(Aplicação exponencial) Seja $U$ um aberto de $TM$ conforme a Proposição \ref{proposicao_existencia_geodesica}. A aplicação $\exp:U \to M$ definida por
		$$
		\exp(p,v) = \gamma(1, p,v)=\gamma(\norma{v}, p,\frac{v}{\norma{v}})
		$$
		é chamada aplicação exponencial em $U$.
	\end{definicao}
	
	Fixado um ponto $p \in M$, defina $\exp_{p}:T_{p}M \to M$ por $\exp_{p}(v) = \exp(p,v)$. Com isso, $\exp_{p}(v)$ é o ponto de $M$ que é conectado a $p$ por uma geodésica com comprimento de arco igual a $\norma{v}$ e tem a direção de $v$. Além disso, tem-se que
	$$
	\derivada{\exp_{p}}{t}(tv)|_{t=0} = \derivada{\gamma}{t}(1,p,tv)|_{t=0}=\derivada{\gamma}{t}(t,p,v)|_{t=0}=v.
	$$
	
	A proposição a seguir, que pode ser encontrada em $\cite{manfredo_riemannian_geo}$, tem sua importância na demonstração de outros resultados do texto.
	
	\begin{proposicao}\label{proposicao_difeomorfismo_exponencial}
		(Difeomorfismo exponencial) Seja  $\exp_{p}:T_{p}M \to M$ a aplicação exponencial. Então existe uma bola aberta $B_{\epsilon}(0) \subset \espacotangenteponto{p}{M}$ de raio $\epsilon>0$ e centro em $0 \in T_{p}M$ tal que a restrição $\exp_{p}|_{B_{\epsilon}(0)}:B_{\epsilon}(0) \to W \subset M$ é um difeomorfismo sobre algum aberto $W \subset M$.
	\end{proposicao}
	
	\begin{definicao}\label{definicao_gradiente_hessiana}
		(Campo gradiente e a Hessiana) Sejam $M$ uma n-variedade diferenciável Riemanniana e $f\in \funcoessuaves{M}$. Então o gradiente de $f$ é definido como sendo o campo vetorial $\gradiente \in \campossuaves{M}$ tal que $D_{p}f(v) = g_{p}(\gradiente(p), v)$ para todo $p\in M$ e todo $v \in T_{p}M$. A Hessiana de $f$ é a aplicação bilinear $H_{p}(f): T_{p}M\times T_{p}M \to \reta$ dada pelo diferencial do gradiente de $f$, isto é, $H_{p}(f) = D_{p}\gradiente$.
	\end{definicao}
	
	\begin{definicao}\label{definicao_mergulho_variedades}
			(Mergulho de variedades) Sejam $M, N$ m,n-variedade s diferenciávei, respectivamente. Uma aplicação diferenciável $\psi:M\to N$ é uma imersão se $D\psi_{p}:T_{p}M\to T_{\psi(p)}N$ é injetora para todo $p \in M$. Diz-se que $\psi$ é um mergulho se for uma imersão e um homeomorfismo sobre $\psi(M) \subseteq N$, na topologia induzida por $N$. 
	\end{definicao}
	
	O seguinte teorema permite tratar uma variedade diferenciável como um subconjunto de algum espaço euclidiano de dimensão suficientemente grande. Sua demonstração pode ser encontrada em \cite{guillemin_differential_topology}.
	
	\begin{teorema}\label{teorema_whitney}
			(Mergulho de Whitney) Toda variedade n-dimensional pode ser mergulhada em $\real{2n+1}$.
	\end{teorema}
	
	
	\chapter{Espaços de Sobolev}
	\section{Distribuições}\label{secao_distribuicoes}
	Muitos problemas de análise clássica tiveram suas origens na modelagem de sistemas físicos, tais como: a dinâmica de uma corda vibrante, a propagação do calor em um meio condutor, a descrição de um oscilador com uma força externa, etc. Contudo, em alguns desses modelos surgiram objetos matemáticos cuja definição não fora sistematizada. Por exemplo, sabe-se que uma distribuição de carga elétrica é a fonte de um campo elétrico, sendo esse campo a solução de uma das equações de Maxwell expressa por
	$$
	\nabla.E(x) = \rho(x),
	$$
	onde $E:\real{3} \to \real{3} $ é o campo elétrico, $\nabla.E$ é o divergente de $E$ e $\rho:\real{3}\to \reta$ é a densidade de carga, ambas aplicações de classe $C^{\infty}$. Sejam $q:\overline{B_{\epsilon}(x_{0})} \subset \real{3}\to \reta$ a função carga elétrica definida na bola fechada de raio $\epsilon>0$ centrada em $x_{0}\in \real{3}$, tal que $q|_{\partial B} = 0$, onde $\partial B$ é a fronteira de $B=\overline{B_{\epsilon}(x_{0})}$. A função de distribuição da carga é definida por $\rho(x)=q(x)/V(\epsilon)$, onde $V(\epsilon)=\int_{B}dx$ é o volume de $B$.
	
	O problema surge em modelos físicos em que se toma o limite $\lim\limits_{\epsilon\to 0}V(\epsilon)=0$, para descrever uma densidade de carga puntual. Para modelar esse sistema foi introduzido o símbolo $\delta$ que satisfaz as seguintes propriedades:
	$$
	\int_{\real{3}}\delta(x-x_{0}) = 1\;\;\text{e}\;\; 	f(x_{0})=\int_{\real{3}}f(x)\delta(x-x_{0})
	$$
	para toda função não-nula em um compacto de $\real{3}$. Com isso, pode-se escrever a densidade de carga elétrica como
	
	$$
	\rho(x) = \lim_{\epsilon\to 0} \frac{q(x)}{V(\epsilon)} = q_{0}\delta(x - x_{0}),
	$$
	onde $q_{0} =\int_{\real{3}}q(x)$ é a carga total do sistema. Definida dessa forma, $\rho$ é de fato uma densidade de carga pois
	$$
	\int_{\real{3}}\rho(x) = \int_{\real{3}}q_{0}\delta(x-x_{0}) = q_{0}.
	$$
	
	Portanto, tem-se o problema de EDP
	$$
	\nabla.E(x) = q_{0}\delta(x-x_{0}),
	$$
	cujas soluções podem não existir no sentido usual, mas podem ser determinadas usando o formalismo de distribuições, como será mostrado a seguir.
	

	Nessa seção serão apresentadas apenas algumas definições desse formalismo com o objetivo principal de definir os espaços de Sobolev.
	
	Sejam $f:\Omega\to \reta$ uma função e $\Omega \subseteq \real{n}$ um aberto. O suporte de $f$ é o definido por $supp(f) = \overline{\{ x\in \Omega: f(x)\neq 0 \}}$. O conjunto $\funcoesteste \subset \funcoesdiferenciaveis{k}{\Omega}$ das funções de classe $C^{k}$ e com suporte compacto em $\Omega$ é chamado de espaço de funções teste.
	
	\begin{observacao}
		É possível se mostrar que o espaço das funções de teste $\funcoesteste$ forma um espaço vetorial real, veja em $\cite{friedlander}$.
	\end{observacao}
	
	Seja $l^{p}(\Omega)$ o conjunto das funções Riemann integráveis em $\Omega$ tais que 
	$$
	\int_{\Omega}|f(x)|^{p} <\infty.$$ 
	
	Sejam $f,g \in l^{p}(\Omega)$ e $\Omega(f,g) = \{x\in \Omega: f(x) \neq g(x)\}$. É dito que $f \sim g$ se $\Omega(f,g)$ possui medida nula.
	
	\begin{definicao}\label{definicao_espaco_Lp}
		(Espaços $\espacoLpGeral{p}{\Omega}$) O quociente $\espacoLpGeral{p}{\Omega}=l^{p}(\Omega)/\sim$ é chamado espaço das funções p-integráveis. O conjunto $\espacoLpcomp{\Omega} = \{f:\Omega\to \reta: f \in \espacoLp{K}, \forall K \subset \Omega \;\text{compacto}\}$ é chamado espaço das funções localmente p-integráveis.
	\end{definicao}
	
	
	É possível mostrar que $\espacoLpGeral{p}{\Omega}$ é um espaço vetorial real e, munido da norma
	$$
	\normaLp{f}=\normaLpdefinicao{|f(x)|}{\Omega},
	$$
	forma um espaço de Banach para $1\leq p\leq \infty$ (veja em \cite{breazis_sobolev_spaces}). 
	
	\begin{observacao}
		No caso em que $p=2$ tem-se que $\espacoLdois{\Omega}$ é um espaço de Hilbert, o que está demonstrado em $\cite{kreyszig_analise_funcional}$, ou seja, é um espaço vetorial munido de um produto interno positivo-definido e completo na métrica gerada por esse produto interno.
	\end{observacao}
	
	Dada $f \in \funcoesdiferenciaveis{k}{\Omega}$, denote a $j$-ésima derivada parcial $\partial f/\partial x_{j}$ por $\derivadaparcialabrev{j}f $. Um q-multi-índice é uma q-tupla $\alpha = (\alpha_{1}, \dots, \alpha_{q})$, onde $\alpha_{i} \in \mathbb{N}$ para $0\leq i \leq q$ e seu comprimento é definido por $|\alpha| = \sum_{i}\alpha_{i}$. Sendo $\beta$ um q-multi-índice, então $\alpha+\beta=(\alpha_{1}+\beta_{1}, \dots, \alpha_{q}+\beta_{q})$ e $\alpha\leq \beta$ se $\alpha_{i}\leq\beta_{i}$ para todo $0\leq i \leq q$. O operador multi-diferencial é definido por 
	$$
	\partial^{\alpha} f(x) = \frac{\partial^{|\alpha|} f}{ \partial{x^{\alpha_{1}}_{1}} \dots \partial{x^{\alpha_{q}}_{q}} }(x),
	$$
	onde $|\alpha|\leq k$.
	
	\begin{definicao}
		(Espaço das distribuições) Seja $\Omega \subseteq \real{n}$ um aberto. O espaço das distribuições $\distribuicoesgeral{\Omega}$ é o espaço dual de $\funcoesteste$. Mais precisamente, é o conjunto dos funcionais lineares $f:\funcoesteste \to \reta$ que são contínuos no sentido que, 	para todo compacto $K\subset \Omega$ existe um inteiro $p$ e uma constante $C$ tal que, para todo $\phi\in \funcoesteste$ com suporte em $K$, tem-se que
		$$
		|f(\phi)| \leq C \sup\limits_{x\in K, |\alpha|\leq p}|\partial^{\alpha}\phi(x)|.
		$$
	\end{definicao}
	
	Pode-se identificar uma função localmente integrável com um elemento de $\distribuicoes$ do seguinte modo: seja $f \in \espacoLpcomp{\Omega}$ e tome funcional linear
	$$
	g_{f}(\phi)=\int_{\Omega}f(x)\phi(x).
	$$
	
	Sejam $\phi \in \funcoesteste$ e $K = supp(\phi) \subset \Omega$. É imediata a cota superior $|\phi(x)|\leq \sup\limits_{x\in K, |\alpha|\leq 0}|\partial^{\alpha}\phi(x)| = \sup\limits_{x\in K}|\phi(x)|$. Usando o fato que $C=\int_{K}|f(x)|<\infty$, tem-se
	$$
	\begin{aligned}
	|g_{f}(\phi)|
	&=\Big|\int_{\Omega}f(x)\phi(x)\Big|
	\\
	&=\Big|\int_{K}f(x)\phi(x)\Big|
	\\
	&\leq \int_{K}|f(x)||\phi(x)|
	\\
	&\leq \sup\limits_{x\in K, |\alpha|\leq p}|\partial^{\alpha}\phi(x)| \int_{K}|f(x)|
	\\
	&< C\sup\limits_{x\in K, |\alpha|\leq p}|\partial^{\alpha}\phi(x)|.
	\end{aligned}
	$$
	Portanto, $g_{f} \in \distribuicoes$. Com isso, pode-se mostrar que a aplicação $g:\espacoLpcomp{\Omega}\to \distribuicoesgeral{\Omega}$ definida por $g(f)(\phi) = g_{f}(\phi)$ é um operador linear injetor.
	
	Denote a aplicação $g:\espacoLpcomp{\Omega}\to \distribuicoesgeral{\Omega}$ por $g(f)=\produtointerno{f}{.}$.
	
	\begin{definicao}
		Sejam $g_{f}\in \distribuicoes$ e $\alpha$ um k-multi-índice. A $\alpha$-diferenciação de $g_{f}$ é definida por 
		
		$$
		\partial^{\alpha}g_{f}(\phi)=\produtointerno{\partial^{\alpha}f}{\phi}=(-1)^{|\alpha|}		\produtointerno{f}{\partial^{\alpha}\phi}		$$
		para todo $\phi \in \funcoesteste$. Quando existir $\partial^{\alpha}f \in \espacoLpcomp{\Omega}$ satisfazendo a equação anterior diz-se que a derivada da distribuição existe no sentido fraco (ou sentido distribucional).
	\end{definicao}
		
	 
	\begin{definicao}
		(Delta de Dirac) A distribuição $\delta \in \distribuicoes$, chamada Delta de Dirac, é definida por $\produtointerno{\delta}{\phi} = \phi(0)$, para todo $\phi \in \funcoesteste$.
	\end{definicao}
	
	\begin{exemplo}
		(Função de Heaviside) A função $\theta:\reta \to \reta$ definida por
		$$
		\funcaocond{\theta(x)}{1}{x\geq0}{0}{x<0},
		$$
		é chamada  função de Heaviside.
		
		Afirmo que $\partial\theta = \delta \in \distribuicoes$. De fato, é imediato que $\theta$ é localmente integrável, logo $\theta \in \distribuicoesgeral{\reta}$. Dado $\phi \in \funcoestestegeral{\reta}$ e definindo $K=supp(\phi)$, tem-se que 
		
		$$
		\begin{aligned}
		\produtointerno{\partial\theta}{\phi} 
		&= -\produtointerno{\theta}{\partial\phi} 
		\\
		&=-\int_{\reta}\theta(x)d\phi(x)
		\\
		&= -\int_{K\cap\reta_{+}}  d\phi(x) 
		\\
		&= -\phi(x)\Big|^{\infty}_{0} 
		\\
		&= \phi(0) 
		\\
		&= \produtointerno{\delta}{\phi},
		\end{aligned}
		$$
		onde foi usado o fato que, por ter o suporte compacto, então $\lim_{x\to \infty}\phi(x)=0$.
		Como $\phi \in \funcoestestegeral{\reta}$ é arbitrária, então $\partial \theta = \delta \in \distribuicoesgeral{\reta}$.
	\end{exemplo}
	
	\begin{observacao}
		O exemplo anterior permite sistematizar o problema físico da densidade puntual de carga escrevendo $\produtointerno{\nabla.E}{.} = \produtointerno{q_{0}\delta}{.} \in \distribuicoesgeral{\real{3}}$, onde a solução dos sistema é um elemento $E\in \espacoLpcomp{\real{3};\real{3}}$. As soluções de EDPs nesse contexto são chamadas soluções fracas.
	\end{observacao}

	\section{Espaço de Sobolev $W^{k,p}$}\label{secao_espaco_sobolev}
	
	\begin{definicao}\label{definicao_espaco_sobolev}
		(Espaço de Sobolev) Seja $\Omega \subset \real{n}$ um aberto e fixe $k, p\in \inteiros$, onde $1\leq p \leq \infty$ e $0\leq k$. O espaço de Sobolev $W^{k,p} (\Omega)$ é o conjunto das funções $f:\Omega\to \reta$ tais que $\partial^{\alpha}f$ existe no sentido fraco e $\partial^{\alpha}f \in \espacoLp{\Omega}$ para todos os multi-índices $\alpha$ com $|\alpha|\leq k$.
	\end{definicao}
	
	Quando existe g em $\espacoLp{\Omega}$ satisfazendo a equação $\produtointerno{f}{\partial^{\alpha}\phi} = (-1)^{|\alpha|}\produtointerno{g}{\phi}$ para todo $\phi\in \funcoesteste$ diz-se que $g = \partial^{\alpha}f$. Logo $W^{k,p}$ pode ser visto como um espaço de distribuições.
	
	
	\begin{definicao}
		(Espaços de aplicações p-integráveis) Seja $\Omega$ é um aberto de $\real{n}$. O espaço de $\espacoLpcontradominio{\Omega}{\real{m}}$ é o conjunto das aplicações 
		$$
		\begin{aligned}
		f:\Omega \subset\real{n} &\to \real{m}
		\\
		x &\mapsto (f_{1}(x), \dots, f_{k}(x))
		\end{aligned}
		$$ 
		tais que $f_{j} \in \espacoLp{\Omega}$ para $1\leq j \leq k$.
 	\end{definicao}
	
	\begin{definicao}\label{definicao_espalo_sobolev_generalizado}
		(Espaço de Sobolev Generalizado) Seja $\Omega$ é um aberto de $\real{n}$. O espaço de Sobolev $\espacosobolevgeneralizadocontra{\Omega}{\real{m}}$ é o conjunto das aplicações  
		$$
		\begin{aligned}
		f:\Omega \subset\real{n} &\to \real{m}
		\\
		x &\mapsto (f_{1}(x), \dots, f_{k}(x))
		\end{aligned}
		$$ 
		tais que $f_{j} \in \espacosobolev{\Omega}$ para $1\leq j \leq k$.
	\end{definicao}
	
	A importante proposição a seguir está demonstrada em \cite{breazis_sobolev_spaces}.
	
	\begin{proposicao}
		Os espaços $\espacoLpcontradominio{\Omega}{\real{m}}$ e $\espacosobolevgeneralizadocontra{\Omega}{\real{m}}$, munidos das normas 
		$$
		\normaLp{f} =\normaLpdefinicao{\sum_{j=1}^{m} |f_{j}(x)|}{\Omega},
		$$
		e
		$$
		\normaWpgeneralizado{f} =\normaLpdefinicao{\sum_{j=1}^{m} \sum_{|\alpha|\leq k}|\partial^{\alpha}f_{j}(x)|}{\Omega},
		$$
		respectivamente, são espaços de Banach.
	\end{proposicao}
	
	
	\begin{observacao}
		Da proposição anterior é imediato que $\normaLp{f} \leq \normaWpgeneralizado{f}$.
	\end{observacao}
	
	\begin{exemplo}
		(O espaço $\espacosobolevgeral{p}{B_{1}(0)}$) Sejam $B_{1}(0)\subset \real{n}$ a bola aberta centrada na origem e $f:B_{1}(0) \to \reta$ definida por $f(x) = |x_{1}|$, onde $x = (x_{1}, \dots, x_{n})$. Afirmo que $\partial^{0}f, \partial^{1}f $ existem e são elementos de $\espacoLp{B_{1}(0)}$. De fato, como $f$ é contínua e limitada na bola $B_{1}(0)$, então a integral $\int_{B_{1}(0)}f(x)$ existe. Note que $f(x)<1$ para todo $x\in B_{1}(0)$. Com isso, $\int_{B_{1}(0)}(f(x))^{p}\leq \int_{B_{1}(0)}f(x)<\infty$ e 
		$$
		\normaLp{f}=\normaLpdefinicao{|f(x)|}{B_{1}(0)}=\normaLpdefinicao{|x_{1}|}{B_{1}(0)}<\infty.
		$$
		
		Portanto, $f \in \espacoLp{B_{1}(0)}$. Como $\partial^{0}f = f$, então $\partial^{0}f\in \espacoLp{B_{1}(0)}$. Defina $g:B_{1}(0)\to \reta$ por
		$$
		\funcaocond{g(x)}{\pm 1}{\pm x_{1}>0}{0}{x_{1} = 0}.
		$$
		Como $B_{1}(0)$ é limitado e $\int_{B_{1}(0)}|g(x)|^{p}\leq \int_{B_{1}(0)}dx<\infty$, então $g\in \espacoLp{B_{1}(0)}$. Defina $B^{\pm} = \{x \in B_{1}(0): \pm x_{1} >0\}$ e $B^{0}= \{x \in B_{1}(0):  x_{1} =0\}$. Dado $\phi \in \funcoestestegeral{B_{1}(0)}$ tem-se que
		$$
		\begin{aligned}
		\produtointerno{g}{\phi} 
		&= \int_{B_{1}(0)}g(x)\phi(x)
		\\
		&=\int_{B^{+}}g(x)\phi(x)+\int_{B^{-}}g(x)\phi(x)+\underbrace{\int_{B^{0}}g(x)\phi(x)}_{=0}
		\\
		&=\int_{B^{+}}\phi(x)+\int_{B^{-}}(-1)\phi(x)
		\\ &=\int_{\Omega^{+}}\bigparenteses{\derivadaparcial{}{x_{1}}(x_{1}\phi(x)) - x_{1}\derivadaparcial{\phi(x)}{x_{1}}} 
		\\
		&+\int_{\Omega^{-}}\bigparenteses{\derivadaparcial{}{x_{1}}(-x_{1}\phi(x)) + x_{1}\derivadaparcial{\phi(x)}{x_{1}}}
		\\ 
		&=
		\int_{\Omega^{+}\cup \Omega^{-}}\derivadaparcial{}{x_{1}}(|x_{1}|\phi(x)) - \int_{\Omega^{+}\cup \Omega^{-}}|x_{1}|\derivadaparcial{\phi(x)}{x_{1}}
		\\
		&=
		|x_{1}|\phi(x)\Big|_{0}^{\infty}+|x_{1}|\phi(x)\Big|_{-\infty}^{0}- \int_{\Omega}|x_{1}|\derivadaparcial{\phi(x)}{x_{1}}
		\\
		&=
		-\produtointerno{f}{\derivadaparcial{}{x_{1}}\phi},
		\end{aligned}
		$$
		onde foi usado o fato que $\lim_{x_{1}\to \pm \infty}\phi(x) = 0$, pois $supp(\phi) \subset B_{1}(0)$ é um compacto. Portanto, $g = \derivadaparcial{}{x_{1}}f \in \distribuicoesgeral{B_{1}(0)}$ no sentido fraco e $\derivadaparcial{}{x_{1}}f \in \espacoLp{B_{1}(0)}$. É imediato que $\derivadaparcial{}{x_{j}}f\in \espacoLp{B_{1}(0)}$ para $2\leq j \leq n$ pois todas elas são nulas. Com isso, $\partial^{\alpha}f$ existe e é um elemento de $\espacoLp{B_{1}(0)}$ para todo multi-índice $\alpha$ tal que $|\alpha|\leq 1$. Portanto $f\in \espacosobolev{B_{1}(0)}$.
	\end{exemplo}

	No caso em que $\Omega\subset\real{n}$ e $n=1$, tem-se $\espacosobolev{\Omega} \subset C^{0}(\overline{\Omega})$ para $p \geq 1$. E no caso em que $n>1$ tem-se $\espacosobolev{\Omega} \subset C^{0}(\overline{\Omega})$ apenas para $p>n$. Contudo, o Teorema do Mergulho de Sobolev (ou Teorema de Sobolev) garante que a inclusão em $\espacosobolev{\Omega}\hookrightarrow \espacoLp{\Omega}$ é um operador compacto, o que pode ser encontrado em $\cite{breazis_sobolev_spaces}$.	
	
	\begin{teorema}\label{teorema_sobolev}
		(Mergulho de Sobolev) Sejam $\Omega \subseteq \real{n}$ e $1 \leq p < n$. Então $\espacosobolev{\Omega} \hookrightarrow L^{q}(\Omega)$ é uma inclusão contínua, onde $1/q = 1/p -1/n$. Além disso, se $\Omega$ for limitado com bordo $\partial\Omega$ de classe $C^{1}$, então a inclusão é um operador compacto para $1/q>1/p -1/n$.
	\end{teorema}
	

	\begin{observacao}
		O teorema implica que existe uma constate $C>0$ tal que 
	
		para todo $\xi\in \espacosobolev{\Omega}$.
	\end{observacao}
	
	Os seguintes resultados podem ser encontrados em \cite{audi_floer_homology}.
	
	
	
	\begin{teorema}\label{teorema_inclusao_sobolev}
		Sejam $\Omega \subset\real{n}$ um aberto limitado com fronteira de classe $C^{1}$, $k\in \inteiros$ tal que $k\geq 1$ e $1<p <\infty$. Então todas as inclusão a seguir são contínuas:
		\begin{enumerate}
			\item Se $p<n/k$, então $\espacosobolevgeneralizado{\Omega}\subset \espacoLpGeral{q}{\Omega} $ com $1/q=1/p - k/n$.
			
			\item Se $p=n/k$, então $\espacosobolevgeneralizado{\Omega}\subset \espacoLpGeral{q}{\Omega} $ para todo $q\geq p$.
			
			\item\label{item_tres_teorema_inclusao_sobolev} Se $p>n/k$, então $\espacosobolevgeneralizado{\Omega}\subset \espacoLpGeral{\infty}{\Omega} $.
		\end{enumerate}
	\end{teorema}
	
	\begin{teorema}\label{teorema_rellich}
		(Rellich-Kondrachov) Sejam $\Omega \subseteq \real{n}$ um aberto limitado e $p<n$. Então existe uma constante $C=C(p,\Omega)>0$ tal que, para cada $\xi: \Omega \to \real{n}$, tem-se 
		$$
		\normaLgGeral{\xi}{q}{\Omega} \leq C \normaWpGeralDominio{\xi}{p}{\Omega}\;\;\text{para cada}\; q\leq \frac{n}{n-p}.
		$$
		Além disso, para $q<np/(n-p)$, a inclusão $\espacosobolevcontradominio{\Omega}{\real{m}} \hookrightarrow \espacoLpGeral{q}{\Omega;\real{m}}$ é um operador compacto.
	\end{teorema}
	
	\section{Preliminares de Análise}\label{apendice_preliminares_analise}
	\begin{definicao}
		Sejam $A,B$ espaços normados. Um operador linear $T:A\to B$ é um operador linear limitado se existe uma constante $C>0$ tal que
		$$
		\norma{T\xi} \leq C\norma{\xi}\;\;\forall \xi \in A.
		$$
		Caso contrário, é dito que $T$ é um operador não-limitado.
	\end{definicao}
	
	\begin{teorema}\label{teorema_hahn_banach}
		(Teorema de Hahn-Banach) Seja $f:B \to \reta$ um funcional sublinear definido no espço vetorial $B$. Suponha que $g:Z \subset B \to \reta$ seja um funcional linear tal que $g(x)\leq f(x)$ para todo $x \in Z$. Então $g$ tem uma única extensão linear $g': B \to \reta$ tal que $g'(x)\leq f(x)$ para todo $x \in B$ e $g=g'|_{Z}$.
	\end{teorema}
	
	Sejam $E$ e $F$ espaços de Banach e $E^{*}$ e $F^{*}$ seus respectivos espaços duais. Suponha que $A: D(A)\subset E\to F$ com o domínio $D(A)$ denso em $E$. Defina o conjunto 
	$$
	D(A^{*})\{ f\in F^{*}: \exists\;c\geq 0\;\;\text{tal que }\;\; |f(Au)|\leq c\norma{u}\;\forall u\in D(A)\}.
	$$
	Pode-se mostrar que $D(A^{*})$ é um subespaço vetorial de $F^{*}$.
	
	Dado $f\in D(A^{*})$, considere a aplicação $g:D(A)\to \reta$ definida por $g(u) = f(Au)$. Com isso, $|g(u)|\leq c\norma{u}$ para todo $u\in D(A)$. Pelo Teorema \ref{teorema_hahn_banach} existe uma única aplicação linear $h\in E^{*}$ tal que 
	$$
	|h(u)|\leq c\norma{u} \;\forall u \in E \;\;\text{e }\;\; g=h|_{D(A)}.
	$$
	
	O operador linear não-limitado $A^{*}:D(A^{*})\subset F^{*}\to E^{*}$ definido por $A^{*}f=f'$ é chamado adjunto de $A$ e satisfaz a relação
	$$
	f(Au) = (A^{*}f)(u).
	$$
	
	\begin{teorema}\label{teorema_representacao_reiz}
		(Teorema da representação de Reisz) Seja $f \in \espacoLpdual$ um funcional linear e $1<p<\infty$. Então existe uma única aplicação $Y \in \espacoLpadjuntoretacirculo$ tal que
		$$
		f(X) = \int_{\retacartesianocirculo}\produtointerno{Y(s,t)}{X(s,t)},
		$$
		para todo $X \in \espacoLpcontradominio{\retacartesianocirculo}{\real{m}}$, onde $\produtointerno{.}{.}$ é o produto interno canônico de $\real{m}$.
	\end{teorema}
	
	
	\begin{thebibliography}{9}
		\bibitem{abramovich}
		Abramovich, Y. A.; Aliprantis, C. D.
		\emph{An Invitation to Operator Theory},
		American Mathematical Society, 2002.
		
		\bibitem{amyia_diff_topology}
		Amyia, Mukherjee.
		\emph{Topics in Differential Topology},
		Texts and Reading in Mathematics 34,
		2005.
		
		\bibitem{audi_floer_homology}
		Audi, Michèlle; Damian, Mihai.
		\emph{Morse Theory and Floer Homology},
		Springer, first edition,
		2010.
		
		\bibitem{banyaga_morse_homology}
		Banyaga, Augustin; Hurtubise, David
		\emph{Lectures on Morse Homology},
		Springer Science + Business Media, first edition,
		2004.
		
		\bibitem{breazis_sobolev_spaces}
		Brezis, Haim.
		\emph{Functional Analysis, Sobolev Spaces and Partial Differential Equantions},
		Springer, first edition,
		2011.
		
		\bibitem{cappell_maslov_index_equivalencia}
		Cappell, Sylvain E.; Lee, Ronnie; Miller, Edward Y.
		\emph{On Maslov Index}, Communication on Pure and Applied Mathematics, Vol. XLVII, 121-186 (1994).
		
		\bibitem{chang_morse_infinito}
		Chang, Kung-Ching.
		\emph{Infinite Dimensional Morse Theory and Multiple Solutions Problems}, Birkhauser, Progress in nonlinear differential equations and
		their applications, vol.6 (1993).
		
		\bibitem{manfredo_riemannian_geo}
		Do Carmo, Manfredo P.
		\emph{Riemannian Geometry},
		Birkhauser, 2nd edition,
		1992.
		
		\bibitem{doering_ode}
		Doering, Claus I.; Lopes, Artur O.
		\emph{Equações Diferenciais Ordinárias - Coleção Matemática Universitária},
		IMPA, terceira edição,
		2008.
		
		\bibitem{friedlander}
		Friedlander, F. G.; Joshi, M.
		\emph{Introduction to Theory of Distributions},
		Cambridge University Press, 2nd edition
		1998.
		
		\bibitem{guillemin_differential_topology}
		Guillemin, Victor; Pollack, Alan.
		\emph{Differential Topolgy},
		Prentice-Hall,
		1974.	
		
		\bibitem{hatcher}
		Hatcher, Allen.
		\emph{Algebraic Topology},
		Cambridge University Press, 2001.
		
		\bibitem{hoffman_kunze}
		Hoffman, Kenneth; Kunze, Ray.
		\emph{Linear Algebra},
		John Wiley and Sons, 2nd edition, 1971.
		
		\bibitem{husemoller_fibrados}
		Husemoller, Dale.
		\emph{Fibre Bundles},
		Springer-Verlag, third edition
		1993.
		
		\bibitem{kreyszig_analise_funcional}
		Kreyszig, Erwin.
		\emph{Introduction to Functional Analysis with Applications},
		John Wiley and Sons, 1978.
		
		
		\bibitem{elon_grupo_fundamental}
		Lima, Elon Lages.
		\emph{Fundamental Group and Covering Spaces},
		A K Peters, 2003.

		\bibitem{elon_homologia}
		Lima, Elon Lages.
		\emph{Homologia Básica},
		IMPA, 2012, segunda edição.
		
		\bibitem{massey}
		Massey, William S.
		\emph{A Basic Course in Algebraic Topology},
		Springer-Verlag, first edition,
		1991.
	
		\bibitem{dusa}
		McDuff, Dusa; Salamon, Dietmar.
		\emph{Introduction to Symplectic Topology},
		Clarendon Press, Oxford, 1988.
	
		\bibitem{milnor}
		Milnor, J.
		\emph{Morse Theory},
		Princeton University Press, 1963.

		\bibitem{munkres_topology}
		Munkres, James R.
		\emph{Topology},
		Prentice Hall Inc., 2000.
				
		\bibitem{nakahara}
		Nakahara, Mikio.
		\emph{Geometry, Topology and Physics},
		Graduate Student Series in Physics, 2nd edition,
		2003.
		
		\bibitem{nussenzveig}
		Nussenzveig, Moyses.
		\emph{Mecânica - Curso de Física Básica Vol 1},
		Edgard Blucher, 4 edição,
		2002.
		
		\bibitem{palis_dynamical_systems}
		Palis, Jaboc; Melo, Welington.
		\emph{Geometric Theory od Dynamical Systems},
		Springer-Verlag,
		1982.
		
		\bibitem{pontryagin_ode}
		Pontryagin, L.S.
		\emph{Ordinary Differential Equations},
		Addison-Wesley Publishing Company,
		1962.
		
		\bibitem{salamon_lecture}
		Salamon, Dietmar.
		\emph{Lectures on Floer Homology},
		University of Warnick,
		1997.


		\bibitem{salamon_conley_index}
		Salamon, Dietmar.
		\emph{Morse Theory, the Conley Index and Floer Homology},
		LMS 22, 113-140
		1990.

		\bibitem{salamon_zehnder}
		Salamon, Dietmar; Zehnder Eduard.
		\emph{Morse Theory for Periodic Solutions of Hamiltonian Systems and Maslov Index},
		Communications on Pure and Applied Mathematics, Vol. XLV, 1303-1360
		1992.
		
		\bibitem{matthias_morse_homology}
		Schwarz, Matthias.
		\emph{Morse Homology},
		Birkhauser, Boston
		1993.
		
		\bibitem{steenrod_fibre_bundles}
		Steenrod, Norman.
		\emph{The Topology of Fibre Bundles},
		Princeton University Press, third printing, 
		1960.
		
		\bibitem{vick_homology}
		Vick, James W..
		\emph{Homology Theory - An Introduction to Algebraic Topology},
		Academic Press, 1973.
		
		\bibitem{warner_manifolds}
		Warner, Frank W.
		\emph{Foundations of Differentiable Manifolds and Lie Groups},
		Scott, Foresman and Company.
			
		\bibitem{witten_supersymmetry_morse}
		Witten, Edward.
		\emph{Supersymmetry and Morse Theory},
		J. Differential Geometry 17,
		1982.
	\end{thebibliography}
	
\end{document}