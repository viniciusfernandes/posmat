\documentclass[12pt]{book}
\usepackage[portuguese]{babel}
\usepackage{graphicx}
\usepackage{indentfirst}
\usepackage[utf8]{inputenc}
\usepackage{amssymb}
\usepackage{enumitem}
\usepackage{color}
\usepackage[fleqn]{amsmath}
\usepackage[a4paper, margin=1.0in]{geometry}
\usepackage{verbatim}

% INICIO CONFIGURACAO DO HYPER LINK %
\usepackage{hyperref}
\usepackage[dvipsnames]{xcolor}
\newcommand\myshade{85}
\colorlet{mylinkcolor}{violet}

\hypersetup{
	linkcolor  = mylinkcolor!\myshade!black,
	citecolor  = mylinkcolor!\myshade!black,
	urlcolor   = mylinkcolor!\myshade!black,
	colorlinks = true,
}
% FIM CONFIGURACAO DO HYPER LINK %

\usepackage{amsthm, amssymb, amsfonts, amsmath}
\usepackage{graphicx}
\usepackage{tikz}
\usetikzlibrary{calc,shapes}
\usepackage{mathtools}
\usepackage{mathrsfs}
\usepackage{tikz-cd}

% PACOTES PARA DIAGRAMAS %
\usepackage[all,cmtip]{xy}
% PACOTES PARA DIAGRAMAS %

% QUALIFICADORES DOS RESULTADOS: TEOREMAS, LEMAS, COROLARIOS E PROVA %
\newtheorem{teorema}{Teorema}[section]
\newtheorem{corolario}[teorema]{Corolário}
\newtheorem{convensao}[teorema]{Convensão}
\newtheorem{lema}[teorema]{Lema}
\newtheorem{definicao}[teorema]{Definição}
\newtheorem{exemplo}[teorema]{Exemplo}
\newtheorem{observacao}[teorema]{Observação}
\newtheorem{proposicao}[teorema]{Proposição}
\newenvironment{prova}[1]{$\square$ #1}{\hfill$\blacksquare$}
% QUALIFICADORES DOS RESULTADOS: TEOREMAS, LEMAS, COROLARIOS E PROVA %

% AQUI ESTAO OS COMANDOS %
\newcommand{\aplicacaoexponencial}[2]{exp_{#1}(#2)}
\newcommand{\aplicacaoexponencialgeral}[1]{exp_{#1}}
\newcommand{\aplicaoessuaves}[2]{C^{\infty}(#1, #2)}
\newcommand{\aplicaoessuavesloc}[2]{C^{\infty}_{\text{loc}}(#1, #2)}
\newcommand{\aplicaoessuavesreatacirculo}{C^{\infty}(\retacartesianocirculo, M)}
\newcommand{\autoespaco}[1]{E_{#1}}
\newcommand{\bigmodulo}[1]{\Bigm\lvert #1 \Bigm\lvert }
\newcommand{\bigparenteses}[1]{\Big( #1 \Big) }
\newcommand{\bordo}[1]{\partial_{#1}}
\newcommand{\bordorel}[1]{\overline{\partial}_{#1}}
\newcommand{\cadeia}[2]{C_{#1}(#2; A)}
\newcommand{\caminhosfechadoscirculo}[2]{L([#1,#2], S^{1})}
\newcommand{\caminhosfechadospontobase}[2]{\mathcal{L}^{o}_{#1}(#2)}
\newcommand{\caminhosfechadosSp}[2]{L([#1,#2], \gruposimpletico{2n})}
\newcommand{\caminhosespeciais}[1]{\mathcal{L}^{*}(#1)}
\newcommand{\caminhosdecaimentoexponencial}[2]{C^{\infty}_{\searrow}(#1, #2)}
\newcommand{\caminhosdecaimentoexponencialpadrao}{\caminhosdecaimentoexponencial{x^{-}}{x^{+}}}
\newcommand{\caminhosexponenciaisconectantesabrev}{\mathcal{P}}
\newcommand{\caminhosexponenciaisconectantes}[2]{\mathcal{P}^{1,p}(#1, #2)}
\newcommand{\caminhosexponenciaisSobolev}{\mathcal{L}^{1,p}M}
\newcommand{\caminhosexponenciaisconectantespadrao}{\caminhosexponenciaisconectantes{x^{-}}{x^{+}}}
\newcommand{\caminhos}{\mathcal{L}}
\newcommand{\caminhosfechados}[1]{\caminhos^{o}(#1)}
\newcommand{\caminhoslagrangianos}[3]{\caminhos_{#3}(#1,#2)}
\newcommand{\caminhoslagrangianosV}[2]{\caminhoslagrangianos{#1}{#2}{V}}
\newcommand{\caminhospontobase}[1]{\caminhos_{#1}}
\newcommand{\caminhossempontobase}[1]{\caminhos(#1)}
\newcommand{\caminhosNaoDegeneradosSp}{\caminhos^{*}(\gruposimpletico{2n})}
\newcommand{\caminhospontobasegeral}[2]{\caminhos_{#1}(#2)}
\newcommand{\caminhossuavesconectantes}[2]{\caminhos(#1, #2)}
\newcommand{\caminhossubespacoslagrangianos}[2]{L[#1,#2]}
\newcommand{\campogradiente}{\mathcal{X}}
\newcommand{\campogradientefuncional}{\mathcal{X}_{\mathcal{A}}}
\newcommand{\campohamiltoniano}[1]{X_{H}(#1)}
\newcommand{\campohamiltonianoabrev}{X_{H}}
\newcommand{\campossuaves}[1]{\mathfrak{X}(#1)}
\newcommand{\celula}[2]{D^{#1}_{#2}}
\newcommand{\celulabordo}[2]{\partial D^{#1}_{#2}}
\newcommand{\circulo}{S^{1}}
\newcommand{\circulovariedade}{\circulo\times M}
\newcommand{\cktopologia}[1]{\mathcal{C}^{#1}\text{-topologia}}
\newcommand{\classe}[1]{[#1]}
\newcommand{\cohomologia}[2]{H^{#1}(#2)}
\newcommand{\cohomologiadual}[2]{H^{#1}(#2)^{*}}
\newcommand{\cohomologiacompac}[2]{H^{#1}_{c}(#2)}
\newcommand{\cohomologiacompacdual}[2]{H^{#1}_{c}(#2)^{*}}
\newcommand{\colecao}[1]{\{#1_{k} \}_{k\in \inteiros}}
\newcommand{\colecaoabrev}[1]{\{#1 \}_{k\in \inteiros}}
\newcommand{\colecaofinita}[2]{\{#1_{j} \}_{j=1}^{#2}}
\newcommand{\colecaofinitaabrev}[2]{\{#1 \}_{j=1}^{#2}}
\newcommand{\complementar}[2]{#1 \backslash #2}
\newcommand{\complexificacao}[1]{#1_{\complexo{}}}
\newcommand{\complexificacaotensorial}[1]{\complexo{}\otimes_{\reta} #1}
\newcommand{\complexificado}[1]{\mathcal{#1}}
\newcommand{\complexificacaoelemento}[2]{#1\otimes_{\reta} #2}
\newcommand{\complexo}[1]{\mathbb{C}^{#1}}
\newcommand{\coordenada}[2]{#1_{(#2)}}
\newcommand{\diferencialfloer}{D\operadorflor}
\newcommand{\diferencialfloerabrev}{\mathcal{D}}
\newcommand{\derivada}[2]{\frac{d #1}{d #2}}
\newcommand{\derivadaparcial}[2]{\frac{\partial #1}{\partial #2}}
\newcommand{\derivadaparcialabrev}[1]{\partial_{#1}}
\newcommand{\diferencialhamiltoniano}[1]{(dX_{H})_{#1}}
\newcommand{\distribuicoes}{\distribuicoesgeral{\Omega}}
\newcommand{\distribuicoesgeral}[1]{\mathcal{D'}(#1)}
\newcommand{\energiafinitaM}{\mathcal{E}M}
\newcommand{\energiafinitaMconectante}{\energiafinitaM(x^{-}, x^{+})}
\newcommand{\espacoLdois}[1]{L^{2}(#1)}
\newcommand{\espacoLp}[1]{L^{p}(#1)}
\newcommand{\espacoLpcomp}[1]{L^{p}_{loc}(#1)}
\newcommand{\espacoLpcontradominio}[2]{L^{p}(#1;#2)}
\newcommand{\espacoLpGeral}[2]{L^{#1}(#2)}
\newcommand{\espacoLpretacirculo}{\espacoLpcontradominio{\retacartesianocirculo}{\real{2n}}}
\newcommand{\espacomoduli}[2]{\mathcal{M}_{#1#2}}
\newcommand{\espacoSimpleticoOrtogonal}[1]{#1^{\omega}}
\newcommand{\espacosobolev}[1]{W^{1,p}(#1)}
\newcommand{\espacosobolevcontradominio}[2]{W^{1,p}(#1;#2)}
\newcommand{\espacosobolevretacirculo}{\espacosobolevcontradominio{\retacartesianocirculo}{\real{2n}}}
\newcommand{\espacosobolevgeral}[2]{W^{1,#1}(#2)}
\newcommand{\espacotangente}[1]{\espacotangenteponto{p}{#1}}
\newcommand{\espacotangentevariedadeestavel}{T^{s}_{p}M}
\newcommand{\espacotangentevariedadeinstavel}{T^{u}_{p}M}
\newcommand{\espacotangenteponto}[2]{T_{#1}#2}
\newcommand{\espacotangentevariedade}{\espacotangenteponto{p}{M}}
\newcommand{\espectrooperador}[1]{\sigma(#1)}
\newcommand{\estruturacomplexa}{J_{0}}
\newcommand{\estruturascomplexas}[2]{\mathcal{J}(#1, #2)}
\newcommand{\estruturascomplexaspadrao}{\mathcal{J}(V, \omega)}
\newcommand{\fibradocaminhosexponenciais}{\mathcal{E}(x^{-}, x^{+})}
\newcommand{\fibradocaminhosexponenciaisabrev}{\mathcal{E}}
\newcommand{\formaSimpletica}[2]{\omega(#1, #2)}
\newcommand{\formaSimpleticaabrev}{\omega_{0}}
\newcommand{\formaSimpleticaExtendida}[2]{\Omega(#1, #2)}
\newcommand{\formaSimpleticaPadrao}[2]{\omega_{0}(#1, #2)}
\newcommand{\funcaocond}[5]{
	#1 = 
	\left\{
	\begin{array}{cc}
		#2, & #3\\
		#4, & #5\\
	\end{array}
	\right.
}
\newcommand{\funcionalH}{\mathcal{A}_{H}}
\newcommand{\funcionalHponto}[1]{\mathcal{A}_{H}(#1)}
\newcommand{\funcoesdiferenciaveis}[2]{C^{#1}(#2)}
\newcommand{\funcoesdiferenciaveissupp}[2]{C^{#1}_{c}(#2)}
\newcommand{\funcoesmorse}[1]{\mathcal{M}_{o}(#1)}
\newcommand{\funcoesmorsesmale}[1]{\mathcal{M}^{S}_{o}(#1)}
\newcommand{\funcoessuaves}[1]{C^{\infty}(#1, \real{})}
\newcommand{\funcoesteste}{\mathcal{D}(\Omega)}
\newcommand{\generalgroup}[2]{GL(#1, #2)}
\newcommand{\generalgroupreal}[1]{\generalgroup{#1}{\real{}}}
\newcommand{\generalgroupcomplexo}[1]{\generalgroup{#1}{\complexo{}}}
\newcommand{\gerador}[1]{\langle #1\rangle}
\newcommand{\gradiente}{\nabla f}
\newcommand{\gradientefuncional}{\nabla \funcionalH}
\newcommand{\grupofundamental}[1]{\pi_{1}(#1)}
\newcommand{\grupofundamentalpontobase}[2]{\pi_{1}(#1; #2)}
\newcommand{\gruposimpletico}[1]{Sp(#1)}
\newcommand{\gruposimpleticocomplexo}[1]{Sp(#1; \complexo{})}
\newcommand{\gruposimpleticoreal}[1]{Sp(#1;\reta)}
\newcommand{\gruposimpleticoespecial}[1]{Sp^{1}(#1)}
\newcommand{\gruposimpleticonaodegenerado}[1]{Sp^{#1}(2n)}
\newcommand{\gruposimpleticopositivo}[1]{Sp_{+}(#1)}
\newcommand{\hessiana}{H_{p}(f)}
\newcommand{\homologia}[2]{H_{#1}(#2;A)}
\newcommand{\homologiaabrev}[2]{H_{#1}(#2)}
\newcommand{\homologiarel}[3]{H_{#1}(#2,#3)}
\newcommand{\homologiarelcel}[3]{H_{#1}(D^{#2}_{#3}, \partial D^{#2}_{#3})}
\newcommand{\homologiarelskele}[3]{H_{#1}(X^{(#2)}, X^{(#3)})}
\newcommand{\homologiarelskelesimpl}[2]{H_{#1}(X^{(#2)}, X^{(#2-1)})}
\newcommand{\imagembordo}[2]{B_{#1}(#2;A)}
\newcommand{\imagembordoabrev}[2]{B_{#1}(#2)}
\newcommand{\induzida}[1]{#1_{\#}}
\newcommand{\inteiros}{\mathbb{Z}}
\newcommand{\inteirospos}{\inteiros_{+}}
\newcommand{\iprod}[2]{\langle #1, #2 \rangle}
\newcommand{\intervalo}{[0,1]}
\newcommand{\kernelbordo}[2]{Z_{#1}(#2;A)}
\newcommand{\kernelbordoabrev}[2]{Z_{#1}(#2)}
\newcommand{\liederivada}[1]{\mathcal{L}_{#1}}
\newcommand{\operadorflor}{\mathcal{F}}
\newcommand{\operadorflordefinicao}[1]{\derivadaparcial{#1}{s} + J(#1)\derivadaparcial{#1}{t} - J(#1)X_{H}(#1)}
\newcommand{\operadorflorparametro}[1]{\mathcal{F}(#1)}
\newcommand{\operadorflorpadrao}{\operadorflorparametro{u}}
\newcommand{\matrizantisimetrica}[1]{Asym(#1)}
\newcommand{\matrizortogonal}[1]{O(#1)}
\newcommand{\matrizquadcomplexa}[1]{M_{#1 \times #1}(\complexo{})}
\newcommand{\matrizquadreal}[1]{M_{#1 \times #1}(\real{})}
\newcommand{\matrizsimetrica}[1]{Sym(#1)}
\newcommand{\matrizSimpleticaOrtogonal}{\mathcal{U}}
\newcommand{\matrizsimetricapositiva}[1]{Sym_{+}(#1)}
\newcommand{\matrizsimpleticapositiva}[1]{Sp_{+}(#1)}
\newcommand{\matrizunitaria}[1]{U(#1)}
\newcommand{\norma}[1]{||#1||}
\newcommand{\normagrande}[1]{\Big|\Big|#1\Big|\Big|}
\newcommand{\normaLp}[1]{||#1||_{L^{p}}}
\newcommand{\normaLpdefinicao}[2]{ \Big(\int_{#2}#1^{p}\Big)^{1/p}}
\newcommand{\normaLpDominio}[2]{||#1||_{L^{p}(#2)}}
\newcommand{\normaLgGeral}[3]{\norma{#1}_{\espacoLpGeral{#2}{#3}}}
\newcommand{\normapequenaLpdefinicao}[2]{ \normaLpdefinicao{\norma{#1}}{#2}}
\newcommand{\normagrandeLpdefinicao}[2]{ \normaLpdefinicao{\normagrande{#1}}{#2}}
\newcommand{\normasubscrito}[2]{\norma{#1}_{#2}}
\newcommand{\normaWp}[1]{||#1||_{W^{1,p}}}
\newcommand{\normaWpgeral}[2]{||#1||_{W^{1,#2}}}
\newcommand{\normaWpGeralDominio}[3]{\norma{#1}_{W^{1,#2}(#3)}}
\newcommand{\normaWpDominio}[2]{||#1||_{W^{1,p}(#2)}}
\newcommand{\operadorcauchyabrev}[1]{\overline{\partial}_{#1}}
\newcommand{\operadorfloer}{\mathfrak{F}}
\newcommand{\operadoresfredholm}[2]{\mathcal{F}r(#1, #2)}
\newcommand{\operadoreslimitados}[2]{\mathcal{B}(#1, #2)}
\newcommand{\orbitaponto}[1]{\mathcal{O}(#1)}
\newcommand{\orbitasconectantes}[2]{\mathcal{M}(#1, #2)}
\newcommand{\orbitasconectantespadrao}{\mathcal{M}(x^{-}, x^{+})}
\newcommand{\parteImaginaria}[1]{\Im(#1)}
\newcommand{\parteReal}[1]{\Re(#1)}
\newcommand{\pontoscriticos}[1]{\textit{Cr}(#1)}
\newcommand{\produtointerno}[2]{\langle #1, #2 \rangle}
\newcommand{\produtointernoabrev}{\langle ., .\rangle}
\newcommand{\produtosinternos}[1]{Riem(#1)}
\newcommand{\produtotensorial}[2]{ #1_{1} \otimes_{\mathbb{K}} \dots \otimes_{\mathbb{K}} #1_{#2}}
\newcommand{\produtotensorialabrev}[2]{#1\otimes #2}
\newcommand{\produtotensorialdual}{\produtotensorialabrev{\complexificado{V}^{*}}{\complexificado{V}^{*}}}
\newcommand{\produtotensorialreal}[2]{\bigotimes_{j=1}^{#1} #2_{j}}
\newcommand{\pullbackfibradotangente}[2]{#1^{*}T#2}
\newcommand{\pullbackfibradotangenteM}[1]{\pullbackfibradotangente{#1}{M}}
\newcommand{\pullbackfibradotangenteMpadrao}{\pullbackfibradotangente{u}{M}}
\newcommand{\retacartesianocirculo}{\real{} \times \circulo}
\newcommand{\retacartesianovariedade}{\real{} \times M}
\newcommand{\real}[1]{\mathbb{R}^{#1}}
\newcommand{\realprojetivo}[1]{\mathbb{R}P^{#1}}
\newcommand{\reta}{\real{}}
\newcommand{\subespacoslagrangianos}{L(V)}
\newcommand{\lacocontrateis}{\mathcal{L}M}
\newcommand{\somadir}[1]{\bigoplus \limits_{#1}}
\newcommand{\cilindrosLM}{\mathcal{C}M}
\newcommand{\morsefunc}[1]{\mathcal{M}o(#1)}
\newcommand{\skeleton}[1]{X^{(#1)}}
\newcommand{\variedadeconectante}{\variedadeconectantepontos{p}{q}}
\newcommand{\variedadeconectantepontos}[2]{W_{#1#2}}
\newcommand{\variedadeestavel}[1]{W^{s}(#1)}
\newcommand{\variedadeinstavel}[1]{W^{u}(#1)}
\newcommand{\vermelho}[1]{{\color{red}#1}}
% AQUI ESTAO OS COMANDOS %

\begin{document}
	
	\title{Homologia de Floer, Índice de Maslov e a Conjectura de Arnold}
	
	\author{Vinicius Fernandes}
	
	\maketitle
	
	\tableofcontents
	
	\chapter*{Introdução}\label{capitulo_introducao}
	Como muitas das construções desenvolvidas na matemática a Homologia de Floer teve como motivação um problema de origem na Física, mais especificamente, no estudo de sistemas Hamiltonianos. A descrição Newtoniana da mecânica clássica tem como um de seus equivalentes a descrição Hamiltoniana. Nesse contexto, grande parte dos sistemas dinâmicos Hamiltonianos podem ser desenvolvidos em uma 2n-variedade diferenciável $M$ munida de uma 2-forma fechada e não degenerada $\omega$. O sistema dinâmico é modelado por uma função Hamiltoniana dependente do tempo $H: \retacartesianovariedade \to \reta$ satisfazendo a equação geométrica $\omega(\campohamiltonianoabrev(t), v) = dH_{t}(v)$, onde o campo vetorial $\campohamiltonianoabrev(t) \in \campossuaves{M}$ é chamado de campo Hamiltoniano associado a $H$. As equações de movimento são as soluções do fluxo
	$$
	\derivada{\psi(t)}{t} = \campohamiltoniano{\psi(t)}.
	$$ 
	Foi estudando sistemas dinâmicos Hamiltonianos que V. I. Arnold formulou a primeira versão da conjectura que atualmente leva seu nome. Sejam $(M, \omega )$ uma 2n-variedade simplética e $\phi:M \to M$ um difeomorfismo tal que $\phi^{*}\omega = \omega$. Então $\phi$ é chamado de simplectomorfismo. Pode-se mostrar que $\psi(t):M\to M$ é um simplectomorfismo e o conjunto $G=\{\psi(t): t \in \reta\}$ forma um grupo de difeomorfismos com a operação de composição, chamado grupo de aplicações Hamiltonianas (ou grupo a 1-parâmetro de difeomorfismos). De acordo com a Teoria de Lefschetz toda aplicação Hamiltoniana possui ao menos um ponto fixo, quando a característica de Euler de $M$ é não-nula. A ideia de Arnold era generalizar esse resultado, o que levou a seguinte versão da conjectura:
	
	\textbf{Conjectura de Arnold (V1):} \textit{Seja $\phi$ uma aplicação Hamiltoniana em uma 2n-variedade compacta simplética $(M, \omega)$. Então, o número de pontos fixos de $\phi$ é, no mínimo, o número de pontos críticos de uma função Hamiltoniana em $M$.}
	
	Posteriormente, trabalhando em particularizações do problema, tais como $T^{2}$ e $S^{2}$, Arnold observou que, se a função Hamiltoniana $H$ depende periodicamente no tempo, isto é, $H:\circulovariedade\to \reta$ tem período 1, então as soluções 1-periódicas do campo Hamiltoniano estão em bijeção com os pontos fixos da função Hamiltoniana. Com isso, ao invés de determinar diretamente os pontos fixos da aplicação Hamiltoniana $\phi$, basta determinar as soluções 1-periódicas do sistema Hamiltoniano. O que resultou na seguinte reformulação da conjectura
	
	\textbf{Conjectura de Arnold (V2):} \textit	{Seja $(M,\omega)$ uma 2n-variedade compacta e simplética. Defina $H:M\times \real{} \to \reta$ uma função hamiltoniana 1-periódica e suponha que as soluções 1-periódicas do sistema hamiltoniano sejam não-degeneradas. Então o número de soluções $\mathcal{N}$ desse sistema será limitado inferiormente pela soma dos números de Betti de M, isto é:
		$$
		\mathcal{N}\geq \sum_{i=0}^{2n}\beta_{i}(M),
		$$
		onde $\beta_{i}(M)$ é a dimensão do i-ésimo grupo de homologia singular de $M$.}
	
	Motivado pela demonstração dessa conjectura, nasce um dos principais formalismos no estudo da topologia simplética: a Homologia de Floer.
	
	O fato de estar presente a soma dos números de Betti na conjectura nos conduz a ideia de construirmos uma homologia que tenha como ingredientes as soluções do sistema Hamiltoniano. Mas uma abordagem similiar já era conhecida: a Homologia de Morse-Witten. Também tem origem em problemas de sistemas dinâmicos da Física, mais espeficicamente, em estudos dos aspectos topológicos de sistemas quânticos por Edward Witten $\cite{witten_supersymmetry_morse}$.
	
	A construção de Witten traz um entendimento geométrico do operador bordo do complexo de cadeias. Munindo a variedade $M$ com uma métrica Riemanniana, determina-se os pontos críticos de função de Morse $f: M \to \reta$ e mostra-se que, para cada par de pontos críticos com índices consecutivos, existe um número finito de linhas do fluxo gradiente conectando-os. O conjunto de tais linhas são subvariedades de $M$ chamadas variedades de conexão. Escolhemos  uma orientação para a variedade de conexão. Witten definiu o grupo das $k$-cadeias como sendo o grupo abeliano gerado pelos pontos críticos com índice de Morse igual a $k$. Já o operador bordo, quando avaliado em uma k-cadeia, é a soma sob todas as $(k-1)$-cadeias, sendo que os coeficientes dessa soma são inteiros cujo sinal esta relacionado com a orientação da variedade de conexão e com a orientação da variedade instável da k-cadeia. Essa é uma metodologia simples e elegante de cálculo da homologia da variedade que permite o estudo dos aspectos topológicos das variedades via sistemas dinâmicos, sendo que a dinâmica é materializada no gradiente da função de Morse $f$.
	
	É tentador nos enveredarmos na demonstração da conjectura de Arnold via construção de Witten. Contudo, dificuldades técnicas fizeram com que uma abordagem ligeiramente diferente fosse realizada por Floer. Em vez de estudar o gradiente de uma função Hamiltoniana definida na variedade $M$ a estratégia foi analisar o comportamento de um funcional, chamado funcional Hamiltoniano, definido no espaço de curvas 1-periódicas e contráteis em $M$. Aplicando o princípio variacional clássico, determina-se que as soluções 1-periódicas e contráteis do sistema Hamiltoniano são os pontos críticos do funcional Hamiltoniano. Essa técnica é conhecido como abdordagem variacional. Uma das grandes dificuldades nesse cenário é que o conjunto de curvas 1-periódicas e contráteis forma uma variedade de dimensão infinita, com isso, estabelecer uma análogo ao índice de Morse para os pontos críticos do funcional é uma tarefa árdua. Foi necessária a construção de um operador de Fredholm e de um índice de Maslov, que generaliza o ideia do índice de Morse para dimensão infinita.
	
	O princípio variacional aplicado no funcional Hamiltoniano resulta na determinação das solução de um sistema de equações diferenciais parciais de primeira ordem. A esse sistema de equações diferenciais é associado um operador diferencial, chamado operador de Floer, cujas soluções são as curvas integrais que conectam os pontos críticos do funcional Hamiltoniano, situação análoga a determinação das linhas de fluxo do gradiente negativo.
	
	O operador de Floer é um operador linear e limitado definido no espaço de todas as curvas que conectam um par de pontos críticos do funcional Hamiltoniano. Mostra-se que esse espaço é uma variedade de dimensão infinita (uma variedade de Banach) e que o diferencial de tal operador é um operador de Fredholm. A todo operador de Fredholm associa-se um número inteiro, chamado índice de Fredholm, que no caso em que estamos interessados, coincide com a diferença dos índices de Maslov de cada um dos pontos críticos. Feito isso, pode-se mostrar que, no caso de uma função Hamiltoniana independente do tempo e cuja Hessiana é limitada, tal índice coincide com o índice de Morse, a menos de uma constante. Tem-se aqui a generalização da Homologia de Morse-Witten.
	
	\chapter{Grupo Fundamental}\label{capitulo_grupo_fundamental}
	A topologia algébrica é o ramo da matemática no qual se utiliza estruturas algébricas para se analisar a topologia de um conjunto (um espaço topológico) e seus invariantes. Nesse capítulo abordaremos um caso específico de estrutura algébrica, o grupo fundamental $\grupofundamental{X}$, de um espaço topológico $X$.
	
	\section{Definições}
	
	Sejam $X$ um espaço topológico e $\caminhossempontobase{X}$ o conjunto dos caminhos contínuos $\gamma:[0,1]\to X$. Tome $\gamma, \beta \in \caminhossempontobase{X}$ tais que $\gamma(1) = \beta(0)$. Definimos a justaposição $*:\caminhossempontobase{X}\times \caminhossempontobase{X} \to \caminhossempontobase{X}$ entre $\gamma$ e $\beta$ por
	$$
	\funcaocond{(\gamma*\beta)(t)}{\gamma(2t)}{0\leq t \leq 1/2}{\beta(2t-1)}{1/2 \leq t \leq 1}.
	$$
	
	Quando não houver ambiguidades denotaremos por simplicidade $\caminhos=\caminhossempontobase{X}$.
	
	\begin{definicao}\label{definicao_caminhos_homotopicos}
		(Homotopia entre caminhos) Sejam $\gamma, \gamma' \in \caminhos$ tais que $\gamma(0)=\gamma'(0)$ e $\gamma(1)=\gamma'(1)$. Uma aplicação contínua $F:\intervalo \times \intervalo \to X$ é chamada homotopia entre $\gamma$ e $\gamma'$ se $F(t, 0) = \gamma(t)$ e $F(t, 1) = \gamma'(t)$ para qualquer $t\in \intervalo$. Dizemos que  $\gamma$ e $\gamma'$ são homotópicos se, e somente se, existe uma homotopia entre eles. Nesse caso denotaremos $\gamma \sim \gamma'$. 
	\end{definicao}
	
	\begin{definicao}\label{definicao_homotopia_extremos_fixos}
		(Homotopia de extremos fixos entre caminhos) Suponha as condições da definição anterior. Dizemos que a homotopia entre os caminhos $\gamma$ e $\gamma'$ é uma homotopia de extremos fixos se $F(0,s) = \gamma(0) = \gamma'(0)$ e $F(1,s) = \gamma(1) = \gamma'(1)$ para todo $s\in \intervalo$.
	\end{definicao}
	
	\begin{observacao}
		Por definição, toda homotopia de extremos fixos entre caminhos é uma homotopia entre caminhos. Portanto, também diremos que dois caminhos são homotópicos quando existir uma homotopia de extremos fixos entre ambos.
	\end{observacao}
	
	\begin{observacao}
		A definição de caminhos homotópicos não depende da homotopia escolhida, pois, dadas $F,G$ homotopias entre $\gamma$ e $\gamma'$, podemos definir uma aplicação contínua $H: [0,1] \times [0,1] \times [0,1] \to X$ tal que $H(t,s ,0) = F(t,s)$ e $H(t,s, 1) = G(t,s)$ que é uma homotopia entre as homotopias, logo será uma homotopia entre ambas as curvas.
	\end{observacao}
	
	O próximo lema demonstra a compatibilidade entre a justaposição de caminhos e a relação de homotopia.
	
	\begin{lema}\label{lema_compatibilidade_produto_caminhos}
		Sejam $\gamma, \gamma', \alpha, \alpha' \in \caminhos$, onde $\gamma(1) = \alpha(0), \gamma'(1) = \alpha'(0) $, tais que $\gamma \sim \gamma'$ e $\alpha \sim \alpha'$, então $\gamma * \alpha \sim \gamma' * \alpha'$.
	\end{lema}
	\begin{prova}
		Sejam  $F, G:[0,1] \times [0,1] \to X$ homotopias entre caminhos tais que $F(t,0)=\gamma(t)$, $F(t,1)=\gamma'(t)$, $G(t,0)=\alpha(t)$, $G(t,1)=\alpha'(t)$. Então, $F_{s}, G_{s}:\intervalo \to X$, definidas por $F_{s}(t) = F(s,t)$ e $G_{s}(t) = G(s,t)$, são curvas contínuas para todo $s \in \intervalo$ tais que $F_{s}(1) = G_{s}(0)$. Com isso, $H :\intervalo\times \intervalo\to X$ tal que $H(t, s)=(F_{s}*G_{s})(t)$ é uma homotopia entre $\gamma*\alpha$ e $\gamma'*\alpha'$. De fato, $H(s) :\intervalo\to X$ é uma curva contínua pois é a justaposição de caminhos contínuos para todo $s \in \intervalo$. Além disso, $H(t, 0) = (\gamma*\alpha)(t)$ e $H(t, 1) = (\gamma'*\alpha')(t)$. Logo, $\gamma*\alpha \sim \gamma'*\alpha'$.
	\end{prova}
	
	A justaposição de caminhos em $\caminhos$ nem sempre é associativo. Contudo, o seguinte lema afirma que tal associatividade vale a menos de uma homotopia. Sua demonstração pode ser encontrada em $\cite{massey}$.
	
	\begin{lema}\label{lema_associatividade_produto_caminhos}
		Dados $\alpha, \beta, \gamma\in \caminhos$, então $(\alpha*\beta)*\gamma \sim \alpha*(\beta*\gamma)$.
	\end{lema}
	
	\begin{lema}\label{lema_caminho_inverso}
		Sejam $\gamma, \gamma^{-1}, c \in \caminhos$ tais que $c(t) = \gamma(0)$ para todo $t\in \intervalo$ (c é o caminho constante) e $\gamma^{-1}(t) = \gamma(1-t)$. Então $\gamma*\gamma^{-1} \sim \gamma^{-1}*\gamma \sim c$.
	\end{lema} 	
	\begin{prova}
		Seja a aplicação contínua $F:[0,1]\times [0,1]\to X$ tal que
		$$
		F(t,s) = 
		\left\{
		\begin{array}{cc}
		\gamma(2t), & 0\leq t \leq s/2\\
		\gamma(s), & s/2 \leq t \leq 1-s/2\\
		\gamma(2-2t), & 1-s/2 \leq t \leq 1\\
		\end{array}
		\right.
		$$
		Temos que $F(t,1) = (\gamma*\gamma^{-1})(t)$ e $F(t,0) = c(t) = p$, logo $\gamma*\gamma^{-1}\sim c$ pois $F$ deforma continuamente a justaposição de $\gamma$ e $\gamma'$ no caminho constante. Podemos definir uma homotopia $\hat{F}$ análoga a $F$ substituindo $\gamma$ por $\gamma^{-1}$, consequentemente teremos $\gamma^{-1}*\gamma \sim c$, logo $\gamma*\gamma^{-1} \sim \gamma^{-1}*\gamma \sim c$.
	\end{prova}
	
	Seja $\caminhospontobasegeral{p}{X} = \{\gamma\in \caminhossempontobase{X}: \gamma(0)=\gamma(1)=p \}$ o conjunto dos caminhos contínuos e fechados com o ponto base $p$. Denote o quociente $\{ \classe{\gamma} : \gamma' \in \caminhospontobasegeral{p}{X},\;\;\gamma \sim \gamma'\}$ por $\caminhospontobasegeral{p}{X}/\sim $. Então, o produto definido por $\classe{\gamma}.\classe{\alpha} = \classe{\gamma*\alpha}$ esta bem-definido. De fato, tomando $\gamma, \gamma',\alpha, \alpha' \in \caminhospontobasegeral{p}{X}$ tal que $\gamma \sim \gamma'$ e $\alpha \sim \alpha'$, então $\classe{\gamma}.\classe{\alpha} = \classe{\gamma*\alpha} = \classe{\gamma'*\alpha'} = \classe{\gamma'}.\classe{\alpha'}$.
	
	\begin{definicao}
		(Grupo Fundamental) Dado um espaço topológico $X$, o grupo fundamental é
		$$
		\grupofundamentalpontobase{X}{p} = (\caminhospontobase{p}/\sim, .).
		$$
	\end{definicao}
	
	\begin{teorema}
		O grupo fundamental $\grupofundamentalpontobase{X}{p}$ é um grupo.
	\end{teorema}
	\begin{prova}
		Sejam $\classe{\alpha}, \classe{\beta}, \classe{\gamma} \in  \grupofundamentalpontobase{X}{p}$ e $c\in \caminhospontobase{p}$ o caminho constante.
		\begin{enumerate}
			\item \textit{(Operação fechada)} Pelas construções anteriores de produto entre caminhos e pela construções da equivalência homotópica, temos que $\classe{\alpha}.\classe{\beta} \in \grupofundamentalpontobase{X}{p}$, logo é fechada.
			\item \textit{(Associatividade)} Segue do Lema $\ref{lema_associatividade_produto_caminhos}$ $(\classe{\alpha}. \classe{\beta}). \classe{\gamma} = \classe{\alpha*\beta}. \classe{\gamma} = \classe{(\alpha*\beta)*\gamma} = \classe{\alpha*(\beta*\gamma)} = \classe{\alpha}. (\classe{\beta}. \classe{\gamma} )$.
			\item \textit{(Elemento neutro)} Pela definição de produto temos $(\gamma*c)(t) = \gamma(2t)$ para $0\leq t \leq 1/2$ e  $(\gamma*c)(t) = c(2t-1) = p$ para $1/2 \leq t \leq 1$, logo $(\gamma*c)(t) = \gamma(t)$ para todo $t \in \intervalo$. Com isso, podemos afirmar que $\classe{\gamma}.\classe{c} = \classe{\gamma*c} = \classe{\gamma}$, portanto $\classe{c}$ é o elemento neutro $Id_{p}$.
			\item \textit{(Elemento inverso)} Pelo Lema $\ref{lema_caminho_inverso}$, vimos que $\gamma *\gamma^{-1} \sim \gamma^{-1} *\gamma \sim c$, logo $Id_{p} =\classe{c} =  \classe{\gamma*\gamma^{-1}} = \classe{\gamma}.\classe{\gamma^{-1}}$, portanto $\classe{\gamma^{-1}}$ é o elemento inverso de $\classe{\gamma}$.
		\end{enumerate}
		Portanto $\grupofundamentalpontobase{X}{p}$ é um grupo.
	\end{prova}
	
	Na definição de grupo fundamental se mantém a escolha do ponto base, contudo, pode-se mostrar que, para espaços topológicos conexos por caminhos, a definição de grupo fundamental independe da escolha do ponto base. A demonstração do teorema a seguir pode ser encontrado em $\cite{massey}$.
	
	\begin{teorema}
		Se $X$ é um espaço topológico conexo por caminhos, então $\grupofundamentalpontobase{X}{p} \cong \grupofundamentalpontobase{X}{q}$ para quaisquer $p,q \in X$.
	\end{teorema}
	
	No caso em que $X$ é um espaço espaços topológico conexo por caminhos denotaremos $\grupofundamental{X}=\grupofundamentalpontobase{X}{p}$ e $\caminhossempontobase{X}=\caminhospontobasegeral{p}{X}$.
	
	\section{Homomorfismos induzidos}
	Sejam $f:X\to Y$ uma aplicação contínua entre espaços topológicos e $\gamma \in \caminhospontobasegeral{p}{X}$, então temos a composição $\beta=f\circ \gamma \in \caminhospontobasegeral{f(p)}{Y}$. Suponha que $\gamma' \in \caminhospontobasegeral{p}{X}$ seja tal que $\gamma \sim \gamma'$. Se $F$ é a homotopia entre ambas, então $(f\circ F)(t,0) =  f(\gamma(t))$ e $(f\circ F)(t,1) =  f(\gamma'(t)) $, logo $G=f\circ F$ é uma homotopia entre $\beta=f\circ \gamma$ e $\beta' = f\circ \gamma'$ em $\caminhospontobasegeral{f(p)}{Y}$, assim, $\beta \sim \beta'$.
	
	\begin{lema}
		(Homomorfismo induzido) Seja $f:X\to Y$ uma aplicação contínua entre espaços topológicos, então a aplicação $f_{*}:\grupofundamentalpontobase{X}{p} \to \grupofundamentalpontobase{Y}{f(p)}$ dada por $f_{*}\classe{\gamma} = \classe{f\circ\gamma}$ é um homomorfismo.
	\end{lema}
	\begin{prova}
		Seja $\gamma \in \caminhospontobasegeral{p}{X}$ uma curva constante, então $\classe{\gamma} = Id_{p}$, logo $f_{*}Id_{p} = f_{*}\classe{\gamma} = \classe{f\circ\gamma} = Id_{f(p)}$, pois a composição $f\circ\gamma$ é uma curva constante em $Y$ com $(f\circ\gamma)(t) = f(p)$.
		
		Se $\classe{\gamma}, \classe{\gamma'}\in \grupofundamentalpontobase{p}{X}$, então 
		$$
		f_{*}(\classe{\gamma}.\classe{\gamma'}) = f_{*}(\classe{\gamma*\gamma'}) = \classe{f\circ(\gamma*\gamma')} = \classe{f\circ\gamma*f\circ\gamma'} =
		\classe{f\circ\gamma}.\classe{f\circ\gamma'}=	f_{*}\classe{\gamma}.f_{*}\classe{\gamma'}.
		$$
		Portanto $f_{*}$ é um homomorfismo de grupos.
	\end{prova}
	
	Como consequência do lema anterior o seguinte resultado mostra que o grupo fundamental é um invariante topológico.
	\begin{teorema}
		Seja $f:X\to Y$ um homeomorfismo entre espaços topológicos. Então $\grupofundamentalpontobase{X}{p}$ e $\grupofundamentalpontobase{Y}{f(p)}$ são isomorfos.
	\end{teorema}
	
	Dados $X, Y$ espaços topológicos, $p\in X$ e $q\in Y$ pontos base, então podemos construir o espaço topológico $X\times Y$ com o ponto base $(p,q)$.
	
	\begin{proposicao}\label{proposicao_produto_grupo_fundamental}
		Se $X, Y$ são espaços topológicos, $p\in X$ e $q\in Y$ são os respectivos pontos base, então $\grupofundamentalpontobase{X\times Y}{(p,q)} \cong \grupofundamentalpontobase{X}{p}\times \grupofundamentalpontobase{Y}{q}$
	\end{proposicao}
	\begin{prova}
		Dado o caminho $\Gamma \in \caminhospontobasegeral{(p,q)}{X\times Y}$ podemos escrever $\Gamma(t) = (\alpha(t), \beta(t)) \in X\times Y$ onde $\alpha \in \caminhospontobasegeral{p}{X}$ e $\beta \in \caminhospontobasegeral{q}{Y}$. Seja $\Gamma' \in \caminhospontobasegeral{(p,q)}{X\times Y}$. Então $\Gamma \sim \Gamma'$ implica que $ (\alpha, \beta) \sim (\alpha', \beta')$, logo $\alpha \sim \alpha'$ e $\beta \sim \beta'$. Portanto, temos a bijeção $I(\classe{ (\alpha, \beta)}) =  (\classe{\alpha}, \classe{\beta}) \in \grupofundamentalpontobase{p}{X}\times \grupofundamentalpontobase{q}{Y}$. Tomando $\classe{(\gamma, \lambda)} \in \grupofundamentalpontobase{X\times Y}{(p,q)}$, então $I(\classe{(\alpha, \beta)} .\classe{(\gamma, \lambda)} ) = I(\classe{(\alpha*\gamma, \beta*\lambda)} ) = (\classe{\alpha*\gamma}, \classe{\beta*\lambda})=(\classe{\alpha}, \classe{\beta}) .(\classe{\gamma}, \classe{\lambda}) = I(\classe{(\alpha, \beta)} ).I(\classe{(\gamma, \lambda)} )$, portanto é um isomorfismo.
	\end{prova}
	
	\begin{definicao}
		Seja $X$ um espaço topológico e
		$c \in \caminhospontobasegeral{p}{X}$ o caminho constante tal que $c(t) = p$ para todo $t \in \intervalo$. Se para todo $\gamma \in \caminhospontobasegeral{p}{X}$ tem-se que $\gamma \sim c$, então dizemos que $X$ é simplesmente conexo.
	\end{definicao}
	
	\begin{proposicao}\label{proposicao_grupo_fundamental_simplesmente_conexo}
		Se $X$ é um espaço topológico simplesmente conexo, então $\grupofundamental{X}$ é trivial.
	\end{proposicao}
	\begin{prova}
		Sejam $\gamma, c \in \caminhospontobase{p}$, onde $c(t)=p$ para todo $t\in \intervalo$. Como $X$ é simplesmente conexo, então toda curva pode ser deformada continuamente em $p=c(t)$, logo $\gamma \sim c$ e $\classe{\gamma} = \classe{c} = Id_{p}$. Portanto $\grupofundamental{X}$ é trivial.
	\end{prova}
	
	\section{Aplicação Grau e o Grupo Fundamental de $\circulo$}
	Nesta seção mostraremos que o grupo fundamental de $S^{1}$ é um grupo cíclico infinito pois é isomorfo a $\inteiros$. Para demonstrar este fato, a cada caminho fechado $\gamma:[0,1] \to S^{1}$, associa-se um número $deg(\gamma) \in \inteiros$, chamado de \textit{grau de $\gamma$}, de modo que dois caminhos $\gamma, \beta$ em $S^{1}$ são homotópicos se, e somente se, $deg(\gamma) = deg(\beta)$ (possuem o mesmo grau). Finalmente, mostraremos que induz um isomorfismo entre $\pi_{1}(S^{1})$ e $\inteiros$.
	
	A proposição a seguir é necessária para a definição e também para o estudo das propriedades da aplicação grau e sua demonstração pode ser encontrada em $\cite{elon_grupo_fundamental}$.
	
	\begin{proposicao}\label{proposicao_levantamento_curvas}
		(Levantamento de caminhos) Seja $\gamma:[a,b] \to S^{1}$ uma aplicação contínua e $t_{a}\in \real{}$ tal que $\gamma(a) = e^{it_{a}}$. Então existe uma única aplicação contínua $\alpha:[a,b] \to \real{}$ tal que $\gamma(t) = e^{i\alpha(t)}$ para todo $t\in [a,b]$ e $\alpha(a) = t_{a}$. A aplicação $\alpha$ é chamada de levantamento do caminho $\gamma$ e faz com que o diagrama abaixo comute:
		$$
		\xymatrix{
			& \real{}\ar[d]\ar[d]^{\text{exp}}
			\\
			[a,b]\ar[ur]^{\alpha} \ar[r]_{\gamma} & S^{1}
		}
		$$
	\end{proposicao}
	
	Seja $\gamma \in \caminhossempontobase{\circulo}$. Então podemos escrever $\gamma(t) = \exp(i\alpha(t))$, onde $\alpha(t) \in \reta$ é uma aplicação contínua. Como $\gamma^{-1}(\{1\})$ é um subconjunto fechado do compacto $\intervalo$, então é um compacto, e com isso, finito. Denote $t_{\gamma} = \max\{t \in \gamma^{-1}(\{0\})\}$ e defina os caminhos $\gamma_{0}(t) = \gamma(t_{\gamma}t)$ e $\gamma_{1}(t) = \gamma((1-t_{\gamma})t+t_{\gamma})$. Com isso, podemos escrever $\gamma = \gamma_{0}*\gamma_{1}$. Temos que $\gamma_{0}$ é um caminho fechado pois $\gamma_{0}(0) = \gamma_{0}(1)$ e $\gamma_{1}$ um caminho aberto. Além disso, $\gamma_{1}$ é contrátil e $\gamma = \gamma_{0}*\gamma_{1} \sim \gamma_{0}$. Portanto, qualquer $\gamma \in \caminhos(\circulo)$ é homotópico a um caminho fechado $\gamma_{0} \in \caminhos(\circulo)$.
	
	Vamos definir o grau de um caminho como sendo uma aplicação que associa um dado caminho em $\circulo$ a um múltiplo de $2\pi$, isto é, o número de vezes que o tal caminho envolve $\circulo$. Na construção anterior vimos que todo caminho em $\circulo$ é homotópico a um caminho fechado. Assim, vamos restringir nossas hipóteses a caminhos fechados.

	\begin{definicao}\label{definicao_aplicacao_grau}
		(Aplicação grau) Sejam $p \in \circulo$ um ponto base e $deg: \caminhospontobasegeral{p}{\circulo} \to \inteiros$ a aplicação dada por $deg(\gamma) = (\alpha(1)-\alpha(0))/2\pi$, onde $\alpha:[0,1] \to \real{}$ é o levantamento de $\gamma$ da Proposição $\ref{proposicao_levantamento_curvas}$. A aplicação $deg$ é chamada de aplicação grau e o valor $deg(\gamma)$ é chamado grau do caminho $\gamma$.
	\end{definicao}
	
	Temos que $\gamma(0) = \gamma(1)$, o que é equivalente a $e^{i\alpha(0)} = e^{i\alpha(1)}$ pela proposição $\ref{proposicao_levantamento_curvas}$. Com isso, $e^{i(\alpha(1)-\alpha(0))} = 1$ e teremos $\alpha(1)-\alpha(0) = 2\pi k$ para algum $k \in \inteiros$. Portanto, $deg(\gamma) = (\alpha(1)-\alpha(0))/2\pi \in \inteiros$ e $deg$ está bem-definida.
	
	\begin{proposicao}\label{proposicao_grau_aplicacao}
		(Propriedades da aplicação grau) Sejam $\gamma, \beta \in \caminhospontobasegeral{p}{\circulo}$, então
		\begin{enumerate}
			\item $deg(\caminhospontobasegeral{p}{\circulo}) = \inteiros$ e, se $\gamma(t) = e^{i2\pi t}$, então $deg(\gamma) = 1$.
			\item $deg(\gamma*\beta)=deg(\gamma)+deg(\beta)$.
			\item $\gamma\sim \beta$ se, e somente se, $deg(\gamma)=deg(\beta)$
		\end{enumerate}
	\end{proposicao}
	\begin{prova}
		\begin{enumerate}
			\item Supondo que $\gamma(t) = e^{i2\pi t}$, então $deg(\gamma) = (2\pi -0)/2\pi =1$. Dado $k \in \inteiros$ e supondo que $\beta(t) = e^{i2\pi kt}$ temos $deg(\beta) = k$, logo $deg$ é sobrejetora e $deg(\caminhospontobasegeral{p}{\circulo}) = \inteiros$.
			
			\item Suponha que $\alpha$ e $\lambda$ sejam os respectivos levatamentos de $\gamma$ e $\beta$. Pela justaposição de caminhos temos $(\gamma*\beta)(t) = e^{i\alpha(2t)}$ para $0\leq t\leq 1/2$ e  $(\gamma*\beta)(t) = e^{i\lambda(2t - 1)}$ para $1/2\leq t\leq 1$. Logo 
			$$
			\begin{aligned}
			deg(\gamma*\beta) &= (\lambda(1)- \alpha(0))/2\pi 
			\\
			&= (\lambda(1) -\lambda(0)+ \alpha(1)- \alpha(0))/2\pi
			\\
			&= deg(\gamma)+deg(\beta).
			\end{aligned}
			$$  
			
			\item Suponha que $h:[0,1]\times [0,1]\to S^{1}$ seja uma homotopia de extremos fixos entre $\gamma$ e $\beta$ tal que $h(t,0) = \gamma(t)$ e $h(t,1) = \beta(t)$. Então, para cada $s \in [0,1]$ fixo temos $h_{s}(0) = h_{s}(1)$, portanto $h_{s} \in \caminhospontobasegeral{p}{\circulo}$. Pela Proposição $\ref{proposicao_levantamento_curvas}$ podemos escrever $h_{s}(t) = e^{i\alpha_{s}t}$. Com isso, podemos escolher $s, s_{0 }\in [0,1]$ tais que $\norma{h_{s}(t)-h_{s_{0}}(t)} =\norma{e^{i\alpha_{s}(t)} - e^{i\alpha_{s_{0}}(t)}} <2$, ou seja, $h_{s}(t), h_{s_{0}}(t) \in \circulo$ não são anti-podais, logo $|\alpha_{s}(t)-\alpha_{s_{0}}(t)| <\pi$ para todo $t\in [0,1]$. Seja $0=s_{0}<s_{1}<s_{2}\dots s_{m-1}<s_{m} = 1$ uma partição do intervalo $[0,1]$, tal que $\norma{h_{s_{j+1}}(t)-h_{s_{j}}(t)}<2$ para $0\leq j \leq m-1$. A partição escolhida implica em $|\alpha_{s_{j+1}}(t)-\alpha_{s_{j}}(t)| <\pi$. Então
			$$
			\begin{aligned}
			2\pi|deg(h_{s_{j+1}})-deg(h_{s_{j}})| 
			&= |\alpha_{s_{j+1}}(1)-\alpha_{s_{j+1}}(0) - \alpha_{s_{j}}(1)+\alpha_{s_{j}}(0)|
			\\
			&\leq |\alpha_{s_{j+1}}(1)-\alpha_{s_{j}}(1)| + |\alpha_{s_{j}}(0)+\alpha_{s_{j}}(0)|
			\\
			&<2\pi,
			\end{aligned} 
			$$
			logo $|deg(h_{s_{j+1}})-deg(h_{s_{j}})| <1$. Portanto $deg(h_{s_{j+1}})=deg(h_{s_{j}})$ para todo $0\leq j \leq m-1$. Logo, 
			$$
			deg(\beta) = deg(h_{s_{m}})=deg(h_{s_{m-1}})=\dots=deg(h_{s_{0}}) = deg(\gamma).
			$$
			Por outro lado, vamos supor que $n = deg(\gamma)=deg(\beta)$. Considere os levantamentos $\alpha$ e $\lambda$ dos caminhos $\gamma$ e $\beta$, respectivamente. Seja $H:[0,1]\times [0,1] \to \real{}$ definida por $H(t,s) = (1-s)\alpha(t) + s\lambda(t)$. Então $H(t,0)=\alpha(t)$ e $H(t,1)=\lambda(t)$, logo $H$ é uma homotopia entre $\alpha$ e $\lambda$. Além disso, 
			$$
			\begin{aligned}
			H(1,s) - H(0,s) 
			&= (1-s)\alpha(1) + s\lambda(1) - (1-s)\alpha(0) + s\lambda(0) 
			\\
			&= (1-s)(\alpha(1)-\alpha(0)) + s(\lambda(1)-\lambda(0))
			\\
			&=\big( (1-s)deg(\gamma) +sdeg(\beta)\big)2\pi
			\\
			&= 2\pi n.
			\end{aligned}
			$$ 
			
			Seja a aplicação contínua $G(t, s) = e^{iH(t,s)}$. Então, para $s \in [0,1]$ fixo temos que $G_{s} \in \caminhospontobasegeral{p}{\circulo}$ com $deg(G_{s}) = n$. Além disso, $G(t,0) = e^{i\alpha(t)} = \gamma(t)$ e $G(t,1) = e^{i\lambda(t)} = \beta(t)$, logo $G$ é uma homotopia entre $\gamma$ e $\beta$, portanto $\gamma \sim \beta$.
		\end{enumerate}
	\end{prova}
	
	Dados $\gamma \in \caminhospontobasegeral{p}{X}$ e $k \in \mathbb{N}$, denotaremos a k-justaposição $\gamma*\dots * \gamma$ por $\gamma^{k}$. Além disso, $\gamma^{-k}$ é o caminho tal que $(\gamma^{k}*\gamma^{-k})(t) = p$ para todo $t \in \intervalo$.
	
	O seguinte resultado é uma consequência imediata da da aplicação grau e mostra que o grupo fundamental de $\circulo$ é um grupo abeliano cíclico infinito.
	
	\begin{proposicao}\label{proposicao_gerador_grupo_fundamental_ciruclo}
		O grupo $\grupofundamental{\circulo}$ é gerado por $\gamma(t) = e^{i2\pi t}$, onde $t \in [0,1]$, isto é, $\grupofundamental{\circulo} = \gerador{\classe{\gamma}} $.
	\end{proposicao}
	\begin{prova}
		Supondo $\beta \in \caminhospontobasegeral{p}{\circulo}$, então pela Proposição $\ref{proposicao_grau_aplicacao}$ podemos escrever $\beta(t)=e^{i2\pi \alpha(t)}$ e $deg(\gamma) = 1$, logo 
		$$
		deg(\beta) = k = \underbrace{1+\dots+1}_{k-vezes} = deg(\gamma)+\dots+deg(\gamma) = deg(\gamma *\dots *\gamma) = deg(\gamma^{k}).
		$$
		
		Portanto $\beta \sim \gamma^{n}$ e $\classe{\beta} = \classe{\gamma^{n}}=\classe{\gamma}^{n}$. Logo $\grupofundamental{\circulo} = \gerador{\classe{\gamma}}$.
	\end{prova}
	
	\begin{teorema}\label{teorema_grupo_fundamental_circulo}
		(Grupo fundamental de $\circulo$) $\grupofundamental{\circulo} \cong \inteiros$.
	\end{teorema}
	\begin{prova}
		Da proposição anterior temos que $\grupofundamental{\circulo} = \gerador{\classe{\gamma}}$. É evidente que aplicação $\grupofundamental{\circulo}  \ni \classe{\gamma}^{n} \mapsto n \in \inteiros$ é um isomorfismo.
	\end{prova}
	
	\section{Exemplos}
	\begin{exemplo}
		(Grupo fundamental de $\real{n}$) O espaço $\real{n}$ é simplesmente conexo, então pela Proposição $\ref{proposicao_grupo_fundamental_simplesmente_conexo}$ temos que $\grupofundamental{\real{n}}$ é trivial.
	\end{exemplo}
	\begin{exemplo}
		(Grupo fundamental de $T^{2}$) Seja $T^{2}=\circulo \times \circulo$ o 2-toro. Como $T^{2}$ é conexo por caminhos, então o grupo fundamental não depende do ponto base escolhido, logo temos $\grupofundamental{T^{2}} \cong \grupofundamental{\circulo} \times \grupofundamental{\circulo} \cong \inteiros \times \inteiros$, pelo Lema $\ref{proposicao_produto_grupo_fundamental}$ e o Teorema $\ref{teorema_grupo_fundamental_circulo}$. 
	\end{exemplo}
	
	\begin{exemplo}\label{exemplo_grupo_fundamental_plano_furo}
		(Grupo fundamental $\grupofundamental{\real{2}\backslash\{p\}}$) Sejam $X = \real{2}\backslash \{(0,0)\}$, $p=(1,0) \in X$ o ponto base e $\gamma,c \in \caminhospontobase{p}$ onde $\gamma$ é o círculo envolvendo a origem e $c$ a curva constante. Se $\alpha \in \caminhospontobase{p}$ é um caminho fechado que não envolve a origem, então $\alpha$ pode ser deformada contínuamente para $p$, logo $\alpha \sim c$ e $\classe{\alpha} = Id$. Por definição $deg(\gamma) = 1$ e $deg(c) =0$, o que implica que $0= deg(c)=deg(\gamma*\gamma^{-1}) = 1 +deg(\gamma^{-1})$, portanto $deg(\gamma^{-1})=-1$. Com isso, dado $k \in \inteiros$ tem-se que $deg(\gamma^{k}) = deg(\gamma)\dots deg(\gamma) = k$. Seja $\beta \in \caminhospontobasegeral{p}{X}$ tal que $deg(\beta)=n$. Então $deg(\beta) = deg(\gamma^{n})$, e $\beta \sim \gamma^{n}$, pela Proposição $\ref{proposicao_grau_aplicacao}$. Portanto $\classe{\beta} =  \classe{\gamma^{n}}=\classe{\gamma}^{n}$. Logo, o grupo fundamental de $X$ é um grupo infinito cíclico gerado por $\classe{\gamma}$, isto é, $\grupofundamental{X} = \gerador{\classe{\gamma}}$. A aplicação $\grupofundamental{X} \ni \classe{\gamma}^{n} \mapsto n \in \inteiros$ é um isomorfismo, portanto $\grupofundamental{X} \cong \inteiros$.
	\end{exemplo}
	
	\begin{exemplo}
		(Grupo fundamental $\grupofundamental{\real{2}\backslash\{p,q\}}$) Sejam $X = \real{2}\backslash\{p,q\}$ e $p ,q \in \real{2}$ tais que $p = (-1,0)$ e $q=(1,0)$. Defina $r=(0,0)\in \real{2}$ e $\gamma_{1}, \gamma_{2}, c \in \caminhospontobase{r}$, onde $c$ é a curva constante, $\gamma_{1}$ e $\gamma_{2}$ são os círculos unitários de centro em $p$ e $q$, respectivamente. Seja $\gamma \in \caminhospontobase{r}$, tal que $deg(\gamma) = 1$. Então teremos as seguintes possibildiades para $\classe{\gamma}$: 1) se $\gamma$ não envolve nenhum dos pontos $p,q$, então $\gamma$ pode ser deformada continuamente ao ponto $r$, logo $\gamma \sim c$ e $\classe{\gamma} =\classe{c}= Id_{r}$. 2) se $\gamma$ envolver apenas $p$ ou apenas $q$, então $\gamma \sim \gamma_{1}$ ou $\gamma \sim \gamma_{2}$, logo $\classe{\gamma }=\classe{\gamma_{1}}$ ou $\classe{\gamma }=\classe{\gamma_{2}}$. 3) se $\gamma$ envolver $p$ e $q$, então $\gamma$ pode ser deformada continuamente no laço em formato de "8" que passa pela origem, mas esse laço pode ser deformado na justaposição $\gamma_{1}*\gamma_{2}$, ou seja, $\gamma \sim \gamma_{1}*\gamma_{2}$ e $\classe{\gamma} = \classe{\gamma_{1}}.\classe{\gamma_{2}}$. Analogamente ao Exemplo $\ref{exemplo_grupo_fundamental_plano_furo}$, qualquer classe de $\grupofundamental{X}$ pode ser escrita como $\classe{\gamma_{1}}^{n} \classe{\gamma_{2}}^{m}$, para dados $n,m \in \inteiros$. Com isso, o grupo fundamental de $X$ é um grupo cíclico infinito gerado por $\classe{\gamma_{1}}$ e $\classe{\gamma_{2}}$, isto é, $\grupofundamental{X} = \gerador{\classe{\gamma_{1}}, \classe{\gamma_{2}}}$. A aplicação $\grupofundamental{X} \ni \classe{\gamma_{1}}^{n} \classe{\gamma_{2}}^{m} \mapsto (n,m) \in \inteiros\times\inteiros$ é um isomorfismo, portanto $\grupofundamental{X}  \cong \inteiros\times\inteiros$.
	\end{exemplo}
		
	\chapter{Homologia}
	
	\section{Homologia singular}
	Para as próximas definições, considere $X$ um espaço topológico e $A$ um anel.
	\begin{definicao}
		(Simplexo singular) Denotaremos por $\Delta_{k}$ o simplexo k-dimesional cujos vértices $e_{0}, \dots, e_{k}$ formam um base canônica de $\real{k+1}$ de modo que $\Delta_{k} = \{(x_{0}, \dots, x_{k}) \in \real{k+1}: x_{j}\geq 0, \;\sum_{j}x_{j}=1\}$. Um k-simplexo singular no espaço topológico $X$ é uma aplicação contínua $\sigma:\Delta_{k} \to X$.
	\end{definicao}
	
	Denotaremos por $\cadeia{k}{X}$ o A-módulo livre gerado pelas k-cadeias singulares de $X$. Consequentemente, cada um dos seus elementos é uma soma formal finita $\alpha = \sum_{\sigma} \alpha_{\sigma}.\sigma $ de k-simplexos singulares $\sigma$, onde $\alpha_{\sigma} \in A$.
	
	\begin{definicao}
		(Operador bordo) A i-ésima face do k-simplexo singular $\sigma: \Delta_{k} \to X$ é o (k-1)-simplexo singular $\partial^{i}\sigma:\Delta_{k-1} \to X$ onde $\partial^{i}\sigma = \sigma(t_{0}, \dots, t_{i-1},0,t_{i+1}, \dots, t_{k-1})$. O k-operador bordo é o homomorfismo denotado por $\bordo{k}: \cadeia{k}{X} \to \cadeia{k-1}{X}$ onde $\bordo{k}\sigma = \sum_{i} (-1)^{i}\partial^{i}\sigma$.
	\end{definicao}
	
	Como o $k$-operador bordo é um homomorfismo, então $ker(\bordo{k})$ e $Im(\bordo{k+1})$ são subgrupos abelianos de $\cadeia{k}{X}$, os quais denotamos por $\kernelbordo{k}{X}$ e $\imagembordo{k}{X}$, respectivamente.
	
	\begin{teorema}
		A composição $\bordo{k-1}\circ\bordo{k}: \cadeia{k}{X} \to \cadeia{k-2}{X}$ é o homomorfismo trivial para todo $k>0$.
	\end{teorema}
	
	Uma consequência imediata do teorema anterior é que $\imagembordo{k}{X} \subseteq \kernelbordo{k}{X}$, logo $\imagembordo{k}{X}$ é um submódulo normal de $\kernelbordo{k}{X}$ e o quociente $\kernelbordo{k}{X}/\imagembordo{k}{X}$ define um módulo quociente.
	
	\begin{definicao}
		(k-módulo de homologia singular) Seja $X$ um espaço topológico. Então para cada $k \in \inteirospos$ o módulo quociente
		$$
		\homologia{k}{X} = \frac{\kernelbordo{k}{X}}{\imagembordo{k}{X}}
		$$
		é o $k$-ésimo módulo de homologia singular de $X$ com coeficientes em $A$.
	\end{definicao}
	
	Um módulo graduado $G$ é uma coleção de módulos $\colecao{G}$ tal que as operações de $G$ induzidas em cada $G_{k}$ são fechadas para todo $k \in \inteiros$. Se $G$ e $G'$ são módulos graduados, então um homomorfismo $f:G\to G'$ de grau $r$, onde $r \in \inteiros$ é uma coleção de homomorfismos $\colecao{f}$, onde 
	$$
	f_{k}:G_{k}\to G'_{k+r}.
	$$
	
	Um subgrupo $H\subseteq G$ de um módulo graduado é um módulo graduado $H=\colecao{H}$, onde $H_{k} \subseteq G_{k}$ é um submódulo para $k \in \inteiros$. Nesse caso, o módulo quociente $G/H$ é o módulo graduado $\{G_{k}/H_{k} \}_{k\in \inteiros}$.
	
	Notemos que, dado um espaço topológico $X$, as coleções $\cadeia{*}{X} = \colecaoabrev{\cadeia{k}{X}}$, $\kernelbordo{*}{X}=\colecaoabrev{\kernelbordo{k}{X}}$, $\imagembordo{*}{X}=\colecaoabrev{\imagembordo{k}{X}}$ e $\homologia{*}{X}=\colecaoabrev{\homologia{k}{X}}$ são grupos graduados.
	
	O módulo graduado $\homologia{*}{X}$ é chamado de homologia singular do espaço $X$ e é denotado por
	$$
	\bigoplus_{k\in \inteiros}\homologia{k}{X}.
	$$
	
	\begin{definicao}
		(Complexo de cadeia) Um complexo de cadeia com coeficientes em $A$ é um A-módulo graduado $C_{*} = \colecao{C}$ juntamente com um homomorfismo $\bordo{}:C_{*} \to C_{*}$ (de grau -1)
		$$
		\xymatrix{
			\dots \ar[r]^{\bordo{k+1}}  & C_{k} \ar[r]^{\bordo{k}} & C_{k-1}\ar[r]^{\bordo{k-1}} &\dots
		}
		$$
		tal que a composição $\bordo{k-1}\circ\bordo{k} = 0$ para $k \in \inteiros$. Denotamos o complexo de cadeia por $(C_{*}, \bordo{*})$ e chamamos o homomorfismo $\bordo{*}$ de operador bordo do complexo de cadeia.
	\end{definicao}
	
	Note que o par $(\cadeia{k}{X}, \bordo{*})$, onde $\bordo{*}=\colecaoabrev{\bordo{k}}$, forma um complexo de cadeia.
	
	\begin{definicao}
		(Homologia do complexo de cadeia) Seja $C=(C_{*}, \bordo{*})$ um complexo de cadeia. O $k$-ésimo módulo de homologia do complexo de cadeia $C$ é o módulo quociente
		$\homologiaabrev{k}{C} = \kernelbordoabrev{k}{C}/\imagembordoabrev{k}{C}$. O módulo graduado $\homologiaabrev{*}{C} = \colecaoabrev{\homologiaabrev{k}{C}}$ é chamado homologia do complexo de cadeia $C$ e é denotado por
		$$
		\bigoplus_{k\in \inteiros}\homologiaabrev{k}{C}.
		$$				
	\end{definicao}

	\begin{definicao}
		(Aplicação de cadeias) Sejam $C$ e $C'$ dois complexos de cadeias. Uma aplicação de cadeias é um homomorfismo $f:C\to C'$ de grau zero tal que o diagrama abaixo comuta para todo $k \in \inteiros$
		$$
		\xymatrix{
			C_{k}\ar[d]_{\bordo{k}}\ar[r]^{f_{k}}  & C'_{k}\ar[d]^{\bordo{k}'} 
			\\
			C_{k-1}\ar[r]_{f_{k-1}} & C'_{k-1},
		}
		$$
		isto é, $\bordo{k}'\circ f_{k} = f_{k-1}\circ \bordo{k}$.
	\end{definicao}
	
	Dada uma aplicação de cadeias $f:C\to C'$, temos que $f(\imagembordoabrev{k}{C})\subseteq \imagembordoabrev{k}{C'}$ e $f(\kernelbordoabrev{k}{C})\subseteq \kernelbordoabrev{k}{C'}$. Logo, $f$ induz um homomorfismo $f_{*}:\homologiaabrev{*}{C}  \to \homologiaabrev{*}{C'}$ de grau 0 definido por $f_{*}(\classe{a}) = \classe{f(a)}$.
	
	\begin{observacao}
		Note que, dadas as aplicações de cadeias $f:C\to C'$ e $f':C'\to C''$, a composição de aplicações de cadeias $f'\circ f:C\to C''$ é uma aplicação de cadeias.
	\end{observacao}
	
	Suponha que $f:X\to Y$ seja uma aplicação contínua entre espaços topológicos. Definimos o homomorfismo induzido por $f$ como sendo a aplicação $\induzida{f}:\cadeia{*}{X}\to \cadeia{*}{Y}$ tal que, dada $\phi \in \cadeia{k}{X}$, temos $\induzida{f}(\phi) = f\circ \phi$, para todo $k \in \inteiros$. Pode-se mostra que $\bordo{}'\induzida{f}=\induzida{f}\bordo{}$, ou seja, $\induzida{f}$ é uma aplicação de cadeias.
	
	O teorema a seguir mostra que a homologia singular é um invariante topológico.
	
	\begin{teorema}
		(Isomorfimo induzido) Sejam $f:X \to Y$ um homeomorfismo entre espaços topológicos e $C = \cadeia{*}{X}$ e $C'=\cadeia{*}{Y}$ os complexos de cadeias singulares de $X$ e $Y$, respectivamente. Então $f_{*}: \homologiaabrev{*}{C} \to \homologiaabrev{*}{C'}$ é um isomorfismo.
	\end{teorema}
	
	Sejam $f, g: X\to Y$ duas aplicações contínuas entre espaços topológicos. Dizemos que $f$ e $g$ são homotópicas se existe uma aplicação contínua $h:\intervalo\times X \to Y$, tal que $h(0, .) = f$ e $h(1, .) = g$. Com isso, denotamos $f\sim g$ e a aplicação $h$ é chamada homotopia entre $f$ e $g$.
	
	\begin{teorema}
		Sejam $f, g: X\to Y$ duas aplicações homotópicas entre espaços topológicos. Então $f_{*}=g_{*}: \homologiaabrev{*}{C}\to \homologiaabrev{*}{C'}$, onde $C$ e $C'$ são complexos de cadeia de $X$ e $Y$, respectivamente.
	\end{teorema}
	
	\begin{definicao}
		(Equivalência homotópica) Seja $f:X\to Y$ uma aplicação entre espaços topológicos. Dizemos que $f$ é uma equivalência homotópica se existe uma aplicação $g:Y \to X$ tal que $f\circ g \sim Id_{Y}$ e $g\circ f \sim Id_{X}$. Neste caso, dizemos que $X$ e $Y$ são homotopicamente equivalentes e denotamos por $X\sim Y$.
	\end{definicao}
	
	O resultado seguinte mostra que a equivalência de homotopia é condição suficiente para que os módulos de homologia sejam isomorfos.
	
	\begin{teorema}
		Se $f: X \to Y$ é uma equivalência homotópica entre espaços topológicos, então $f_{*}:\homologiaabrev{*}{C} \to \homologiaabrev{*}{C'}$ é um isomorfismo, onde $C$ e $C'$ são os complexos de cadeias singulares de $X$ e $Y$, respectivamente.
	\end{teorema}
	
	\section{Sequências exatas}
	\begin{definicao}
		Uma tripla $\xymatrix{C\ar[r]^{f} & D\ar[r]^{g} & E }$ de módulos abelianos e homomorfismos é chamada exata se $Im(f) = Ker(g)$. Uma sequência de módulos abelianos e homomorfismos
		$$
		\xymatrix{\dots \ar[r]& G_{1}\ar[r]^{f_{1}} & G_{2}\ar[r]^{f_{2}} & G_{3}\ar[r]^{f_{3}} & \dots \ar[r]\ar[r]^{f_{k-1}} & G_{k}\ar[r]^{f_{k}} &\dots}
		$$
		é chamada de sequência exata longa se cada tripla é exata, isto é, $Im(f_{k}) = Ker(f_{k+1})$ para todo $k\in \inteiros$.
	\end{definicao}
	
	Dizemos que uma sequência do tipo $ \xymatrix{0 \ar[r] & C\ar[r]^{f} & D\ar[r]^{g} & E \ar[r]& 0}$ é uma sequência exata curta. Nesse caso, temos que $f$ é um monomorfismo e $g$ é um epimorfismo. Sejam $C, D, E$ complexos de cadeias. Uma sequência exata curta de complexos de cadeias é uma coleção de sequências exatas
	$$ 
	\xymatrix{0 \ar[r] & C_{k}\ar[r]^{f_{k}} & D_{k}\ar[r]^{g_{k}} & E_{k} \ar[r]& 0}
	$$
	para todo $k \in \inteiros$.
	
	Utilizando a sequência exata curta de complexos de cadeias, um homomorfismo entre as sequências exatas curtas, chamado homomorfismo conectante, será utilizado na construção da sequência exata longa. Considere o diagrama abaixo
	$$
	\xymatrix{
		 & \vdots \ar[d]^{\bordo{k+1}} & \vdots \ar[d]^{\bordo{k+1}'} & \vdots \ar[d]^{\bordo{k+1}''}& 
		\\
		0 \ar[r] & C_{k}\ar[d]^{\bordo{k}}\ar[r]^{f_{k}} & D_{k}\ar[d]^{\bordo{k}'}\ar[r]^{g_{k}} & E_{k} \ar[d]^{\bordo{k}''}\ar[r]& 0
		\\
		0 \ar[r] & C_{k-1}\ar[d]^{\bordo{k-1}}\ar[r]^{f_{k-1}} & D_{k-1}\ar[d]^{\bordo{k-1}'}\ar[r]^{g_{k-1}} & E_{k-1}\ar[d]^{\bordo{k-1}''} \ar[r]& 0
		\\
		 & \vdots & \vdots  & \vdots & 
	}
	$$
	Como $g$ é epimorfismo, dado $e \in \kernelbordoabrev{k}{E}$ existe $d \in D_{k}$ tal que $g_{k}(d) = e$. Pela comutatividade do diagrama, temos que $g_{k-1}(\bordo{k}'(d)) = \bordo{k}''(g_{k}(d)) = \bordo{k}''(e) = 0$. Portanto, $\bordo{k}'(d) \in Ker(g_{k-1}) = Im(f_{k-1})$, pela exatidão. Com isso, podemos tomar $c \in C_{k-1}$ tal que $f_{k-1}(c)=\bordo{k}'(d)$. Então $f_{k-2}(\bordo{k-1}(c)) = \bordo{k-1}'(f_{k-1}(c)) = \bordo{k-1}'(\bordo{k}'(d)) = 0$. Como $f$ é monomorfismo, então $\bordo{k-1}(c) \in \kernelbordoabrev{k-1}{C}$. Logo, temos uma correpondência $\kernelbordoabrev{k}{E} \ni e \mapsto c \in \kernelbordoabrev{k-1 }{C}$, a qual denotaremos por $\delta_{k}(e) = c$. 
	
	\begin{teorema}\label{teorema_homomorfismo_conectante}
		(Homomorfismo conectante) Se $ \xymatrix{0 \ar[r] & C\ar[r]^{f} & D\ar[r]^{g} & E \ar[r]& 0}$ é uma sequência exata curta de complexos de cadeias, então a sequência longa 
		$$
		\xymatrix{\dots \ar[r]^{f_{*}} & \homologiaabrev{k}{D}\ar[r]^{g_{*}} & \homologiaabrev{k}{E}\ar[r]^{\delta_{k}}& \homologiaabrev{k-1}{C}\ar[r]^{f_{*}} & \homologiaabrev{k-1}{D} \ar[r]^{g_{*}}& \dots} 
		$$
		é exata.
	\end{teorema}
	
	Como consequência do Teorema $\ref{teorema_homomorfismo_conectante}$, temos um importante método de cálculo da homologia de espaços topológicos. Dados $U, V$ subconjuntos abertos de um espaço topológico $X$ tal que $X = U\cup V$, considere a sequência exata curta de complexos de cadeias
	$$ 
	\xymatrix{0 \ar[r] & C_{k}(U\cap V)\ar[r]^{f} & C_{k}(U)\oplus C_{k}(V)\ar[r]^{g} & C_{k}(X) \ar[r]& 0},
	$$
	onde $f(x) = (x,-x)$ e $g(x) = x+y$.
	
	A sequência exata longa induzida como no Teorema $\ref{teorema_homomorfismo_conectante}$ é chamada sequência de Mayer-Vietoris.
	
	\begin{teorema}
		(Sequência de Mayer-Vietoris) Sejam $U, V$ subcojuntos abertos do espaço topológico $X$ tais que $X = U \cup V$. Então existe uma sequência exata longa 
		\[
		\xymatrix{\dots \ar[r]^{\delta_{k+1}} & \homologiaabrev{k}{U\cap V}\ar[r]^{f_{*}} & \homologiaabrev{k}{U}\oplus \homologiaabrev{k}{V}\ar[r]^{g_{*}}& \homologiaabrev{k}{X}\ar[r]^{\delta_{k}} & \homologiaabrev{k-1}{U \cap V} \ar[r]^{f_{*}}& \dots} 
		\]
	\end{teorema}
	
	Seja $U \subseteq X$ um subespaço do espaço topológico $X$ e considere o par topológico $(X, U)$. Caso $U = \emptyset$, identificaremos $(X, \emptyset)$ com $X$. Desse modo, a aplicação de inclusão $U \hookrightarrow X$ induz a aplicação de cadeias $\induzida{i}: \cadeia{*}{U} \to \cadeia{*}{X}$ tal que, para todo $k \in \inteirospos$, $i_{k}$ é um monomorfismo. Assim, podemos identificar $\cadeia{*}{U}$ como um subcomplexo de $\cadeia{*}{X}$ e podemos definir o quociente entre ambos de modo que
	
	\begin{definicao}
		O complexo de cadeia singular do par topológico $(X,U)$ é o complexo de cadeia $(\cadeia{*}{X,U},\bordorel{*} )$ onde 
		$$
		\cadeia{*}{X,U} = \frac{\cadeia{*}{X}}{\cadeia{*}{U}}
		$$
		e $\bordorel{k}: \cadeia{k}{X} \to \cadeia{k-1}{U}$ é definido por $\bordorel{k}(\classe{c}) = \classe{\bordo{k}(c)}$.
	\end{definicao}
	
	\begin{definicao}
		(Homologia relativa) Seja $(X, U)$ um par topológico. O $k$-ésimo grupo de homologia relativa de $X$ mod $U$ é o grupo quociente
		$$
		\homologia{k}{X,U} = \frac{\kernelbordo{k}{X,U}}{\imagembordo{k}{X, U}}.
		$$
	\end{definicao}
	
	\begin{proposicao}
		Dado um par topológico $(X, U)$ tal que $U$ seja um retrato de deformação de $X$, então $\homologia{*}{X,U} = 0$.
	\end{proposicao}
	
	\begin{teorema}
		(Sequência exata do par) Seja $(X, U)$ um par topológico. Então a sequência exata longa desse par é a sequência
		$$
		\xymatrix{\dots \ar[r]^{\delta_{k+1}} & \homologiaabrev{k}{U}\ar[r]^{i_{*}} & \homologiaabrev{k}{X}\ar[r]^{\pi_{*}}& \homologiaabrev{k}{X, U}\ar[r]^{\delta_{k}} & \homologiaabrev{k-1}{U} \ar[r]^{i_{*}}& \dots},
		$$
		onde $\delta_{k}$ é o homomorfismo conectante para $k \in \inteirospos$.
	\end{teorema}
	
	\begin{teorema}
		(Excisão) Seja $(X, U)$ um par topológico. Se $U' \subset U$ com $\overline{U'} \subset int(U)$, então a inclusão $i : (\complementar{X}{U'}, \complementar{U}{U'}) \hookrightarrow (X, U)$ induz um isomorfismo 
		$$
		i_{*}: \homologiaabrev{*}{\complementar{X}{U'}, \complementar{U}{U'}} \to \homologiaabrev{*}{X, U}.
		$$
	\end{teorema}
	
	
	Sejam $U'\subseteq U \subseteq X$ subespaços do espaço topológico $X$ e considere a tripla topológica $(X, U, U')$. Analogamente ao caso do par, o teorema seguinte garante a existência de uma sequência exata longa da tripla.
	
	\begin{teorema}
		(Sequência exata da tripla) Seja $(X, U, U')$ uma tripla topológica. Então a sequência exata longa da tripla é a sequência
		$$
		\xymatrix{\dots \ar[r]^{\delta_{k+1}} & \homologiaabrev{k}{U, U'}\ar[r]^{i_{*}} & \homologiaabrev{k}{X, U'}\ar[r]^{\pi_{*}}& \homologiaabrev{k}{X, U}\ar[r]^{\delta_{k}} & \homologiaabrev{k-1}{U, U'} \ar[r]^{i_{*}}& \dots},
		$$
		onde $\delta_{k}$ é o homomorfismo conectante para $k \in \inteirospos$.
	\end{teorema}
	
	
	
	\section{CW-Homologia}\label{secao_cw_complexo}
	Um CW-complexo é um tipo de espaço topológico introduzido por J. H. C. Whitehead, que teve como objetivo inicial facilitar alguns cálculos em teoria de homotopia. A ideia é que um CW-complexo é construído com sucessivas colagens (identificações) de outros espaços mais simples (células), de modo que, para se determinar a homologia do todo basta determinar a homologia das partes mais simples, as células.
	\begin{definicao}
		(Colagem de célula) Sejam $X$ um espaço topológico, $D^{n}=\{x\in \mathbb{R}^{n} : ||x|| \leq 1\}$ e $S^{n-1} = \partial D^{n}=\{x\in \mathbb{R}^{n} : ||x|| = 1\}$. Se $f_{\partial}:S^{n-1} \to X$ é uma função contínua, denotaremos por $X\cup_{f_{\partial}}D^{n}$ o espaço quociente da união disjunta $X \coprod D^{n}$ pela identificação de $x \in \partial D^{x} = S^{n-1}$ com $f_{\partial}(x) \in X$. Diremos que $X\cup_{f_{\partial}}D^{n}$ é obtido a partir de $X$ colando uma $n-$célula e $f_{\partial}$ é chamado de mapa de colagem.
	\end{definicao}
	
	\begin{definicao}
		(CW-complexo) Dizemos que um espaço topológico $X$ tem uma CW-estrutura se existem uma sequência de espaços
		$$
		\skeleton{0} \subseteq \skeleton{1} \subseteq \dots \subseteq X = \bigcup \limits_{n\in \mathbb{N}} \skeleton{n}
		$$ 
		tais que:
		\begin{enumerate}
			\item $\skeleton{0}$ é um conjunto discreto de pontos.
			
			\item $\skeleton{n+1}$ é obtido anexando $(n+1)-$células a $\skeleton{n}$.
			
			\item $X$ tem uma topologia fraca, ou seja, um dado $A \subseteq X$ é dito um aberto se, e somente se, $A \cap \skeleton{n}$ for um aberto em $\skeleton{n}$ para todo $n \in \mathbb{N}$.
		\end{enumerate}
	\end{definicao}
	
	Um espaço $X$ com uma CW-estrutura é chamado de CW-complexo e cada subespaço $\skeleton{n}$ é chamado $n-$esqueleto do CW-complexo $X$. Uma aplicação $f_{\partial}:S^{n-1} \to \skeleton{n-1}$ estende a uma aplicação $f:D^{n} \to \skeleton{n}$ chamada aplicação característica. Chamaremos a imagem de $D^{n}$ por $f$ de célula fechada em $X$, e a imagem de $D^{n} - \partial D^{n}$ de célula aberta em $X$.
	
	\begin{exemplo}
		(n-esfera) Vamos exibir uma estrutura de CW-complexo para $S^{n}$. Fixemos um ponto-base $p \in S^{n}$ e definamos o $0-$esqueleto $\skeleton{0}=\{p\}$. Anexando uma $n-$célula a $\skeleton{0}$ teremos $f_{\partial}: \partial D^{n} \to \skeleton{0}$, isto é, $S^{n} \approx \skeleton{n} = \{p\}\cup_{f_{\partial}} \celula{n}{}$.
	\end{exemplo}
	
	\begin{exemplo}
		(Disco com alça) Sejam $p=(1,0), q=(-1,0) \in D^{2}$ e $I=[a,b] \subset \reta$. Temos $\partial I=\{a,b\}$. Definindo $f_{\partial_{0}}: \partial I \to D^{2}$ tal que $f_{\partial_{0}}(a)=p$ e $f_{\partial_{0}}(b)=q$ teremos o disco com alça $X=D^{2}\cup_{f_{\partial_{0}}}I$.   
	\end{exemplo}
	
	\begin{exemplo}
		(2-toro) Vamos exibir uma estrutura de CW-complexo para $T^{2}$. Representando o toro como o quadrado cujos os lados opostos estão identificados preservando a orientação, todos os vértices do quadrado serão identificados a um único ponto $p \in T^{2}$. Definamos o $0-$esqueleto como sendo $\skeleton{0} = \{p\}$. As arestas horizontais representam o mesmo $S^{1}$ no toro. Isso equivale a colar uma 1-célula ao 0-esqueleto, ou seja, $\skeleton{0}\cup_{f_{1\partial}}\celula{1}{1}$. Analogamente, as faces verticais também representam o mesmo $S^{1}$ no toro, o que indica que devemos anexar uma outra 1-célula o espaço anexado anteriormente, isto é, $\skeleton{0}\cup_{f_{1\partial}}\celula{1}{1}\cup_{f_{2\partial}}\celula{1}{2}$. Por fim, temos que anexar um 2-célula para cobrir o interior do quadrado. Então $T^{2} =\skeleton{2} = \skeleton{0}\cup_{f_{1\partial}}\celula{1}{1}\cup_{f_{2\partial}}\celula{1}{2}\cup_{f_{3\partial}}\celula{2}{3}$.
	\end{exemplo}
	
	\begin{exemplo}
		(n-espaço projetivo) Vamos exibir uma estrutura de CW-complexo para $\realprojetivo{n}$. Para $n=0$ temos que $\realprojetivo{0} = \{\classe{p}\}$, para um determinado $p \in \real{}$. Já para $n=1$, sabemos que existe um homeomorfismo $\realprojetivo{1} \approx S^{1}_{\sim} = \{\classe{p}: p \in S^{1},\; p \sim -p\}$. Note esse espaço quociente já possui, naturalmente, uma CW-estrutura pois, na passagem ao quociente, identificamos todos os pontos do equador com um único ponto desse conjunto. Digamos este ponto $p_{0} = (1,0)$, sem perda de generalidade. Isso quer dizer que ao colarmos uma 1-célula no ponto $p_{0}$ teremos $S^{1}_{\sim} \approx \{[p_{0}]\} \cup_{f}D^{1} \approx \realprojetivo{0}\cup_{f}D^{1} $ pois $\realprojetivo{0} \approx \{[p_{0}]\}$, portanto $ \realprojetivo{1} \approx \realprojetivo{0}\cup_{f}D^{1}$. Repetindo o procedimento anterior para $\realprojetivo{n} \approx S^{n}_{\sim} $ onde $p_{0} = (1,0,\dots, 0)$, teremos $\realprojetivo{n} \approx \realprojetivo{n-1} \cup_{f_{\partial}}D^{n}$. Assim, temos a CW-estrutura $\realprojetivo{j-1} \subseteq \realprojetivo{j}$ e $\realprojetivo{j} = \skeleton{j} \approx \skeleton{j-1}\cup_{f_{j\partial}}D^{j}$ para $1\leq j \leq n$.
	\end{exemplo}
	
	\begin{lema}\label{homologiacelular}
		(Homologia celular relativa) Sejam $A$ um anel comutativo com unidade e $X$ um CW-complexo, então
		$$
		\homologiarelskelesimpl{k}{n} \cong 
		\left\{
		\begin{array}{cc}
		\mathcal{C}_{n}(X), & k = n\\
		0, & k\neq n\\
		\end{array}
		\right.,
		$$
		onde $\mathcal{C}_{n}(X)$ é um $A-$módulo livre e finitamente gerado pelas $n-$células de $X$. Além disso,
		$$
		\mathcal{C}_{n}(X) \cong \somadir{\sigma} \homologiarelcel{n}{n}{\sigma} \cong \somadir{\sigma} A
		$$
		em que 
		$$
		\somadir{\sigma}f_{\sigma*}: \somadir{\sigma} \homologiarelcel{n}{n}{\sigma} \to \homologiarelskelesimpl{n}{n}
		$$
		denota o isomorfismo descrito.
	\end{lema}
	\begin{prova}
		Por definição, temos  $\skeleton{n} = \skeleton{n-1} \bigcup_{f_{\partial \sigma} } \celula{n}{\sigma}$ e também $\celula{n}{\sigma} \subset \skeleton{n}$, onde $f_{\sigma}:\celula{n}{\sigma} \to \skeleton{n}$ é a aplicação característica. Sejam $C_{\sigma}$ e $A_{\sigma}$ discos fechado e abertos, respectivamente, contendo o hemisfério norte de $f_{\sigma}(\celula{n}{\sigma})$. Definindo $N_{\sigma} = f_{\sigma}(\celula{n}{\sigma}) - C_{\sigma}$, $M_{\sigma} = f_{\sigma}(\celula{n}{\sigma}) - A_{\sigma}$ temos $\overline{N_{\sigma}} \subset M_{\sigma}$, e considerando $U = \skeleton{n} - \bigcup C_{\sigma}$, $Y = \skeleton{n} - \bigcup A_{\sigma}$ temos $U \subseteq Y$. Note que $\skeleton{n-1}$ é um retrato de deformação de $Y$, logo, pela invariância homotópica temos $\homologiarel{k}{\skeleton{n}}{\skeleton{n-1}} \cong  \homologiarel{k}{\skeleton{n}}{Y}$. Como $\skeleton{n} - U = \bigcup C_{\sigma}$ e $Y - U $ é homotópico a $\bigcup S^{n}_{\sigma}$, pelo teorema da excisão $\homologiarel{k}{\skeleton{n} - U}{Y- U} \cong \homologiarel{k}{\skeleton{n}}{Y}$. Portanto $\homologiarel{k}{\skeleton{n} - U}{Y- U} = \homologiarel{k}{\bigcup C_{\sigma}}{\bigcup S^{n}_{\sigma}} \cong \homologia{k}{\bigcup (\celula{n}{\sigma}, \celulabordo{n}{\sigma})} \cong \somadir{\sigma} \homologiarelcel{k}{n}{\sigma}$, pois $C_{\sigma} \approx \celula{n}{\sigma}$ e $S^{n}_{\sigma} = \celulabordo{n}{\sigma}$. Enfim, temos o diagrama comutativo:
		\[
		\xymatrix{
			\somadir{\sigma} \homologiarelcel{k}{n}{\sigma} \ar[r]^{id_{*}} \ar[d]^{\cong} & 
			\somadir{\sigma} \homologiarelcel{k}{n}{\sigma} \ar[r]^{id_{*}} \ar[d]^{\somadir{\sigma}f_{\sigma*}} & 
			\somadir{\sigma} \homologiarelcel{k}{n}{\sigma} \ar[d]^{\somadir{\sigma}f_{\sigma*}} 
			\\
			\homologiarel{k}{\skeleton{n} - U}{Y- U} \ar[r]^{\cong} & \homologiarel{k}{\skeleton{n}}{Y} \ar[r]^{\cong} & 
			\homologiarelskelesimpl{k}{n}.
		}
		\]
		Por fim, sabemos que $\homologiarelcel{k}{n}{\sigma} \cong A$ para $k=n$ e é trivial para $k\neq n$, então pela sequência anterior temos $\homologiarelskelesimpl{k}{n} \cong \somadir{\sigma}\homologiarelcel{k}{n}{\sigma} \cong \somadir{\sigma} A$ se $k=n$ e é trivial caso $k\neq n$, como desejávamos.
	\end{prova} 
	
	\begin{definicao}
		(Aplicação de Pares) Seja $X = \skeleton{n}$ um CW-complexo e tome $p \in \skeleton{n-1}$ como um ponto-base. Identificaremos um dado $q \in \skeleton{n}$ com o ponto-base se $q \in \skeleton{n-1}$, e diremos que $q \sim p$. Definimos $\skeleton{n}/\skeleton{n-1} = \{[q]: q \in \skeleton{n}, \; q \nsim p\}$ e a aplicação quociente $\pi : \skeleton{n} \to \skeleton{n}/\skeleton{n-1}$ por:
		$$
		\pi(q) = 
		\left\{
		\begin{array}{cc}
		\classe{p}, & q \in \skeleton{n-1}\\
		\classe{q}, & q \notin \skeleton{n-1}\\
		\end{array}
		\right..
		$$
		Seja $\bigvee_{\sigma} S^{n}_{\sigma}$ o buquet de $n-$esferas com o ponto-base $p$, então $\skeleton{n}/\skeleton{n-1} \approx \bigvee_{\sigma} S^{n}_{\sigma}$. Agora, definindo $s_{\sigma} : \skeleton{n}/\skeleton{n-1} \to S^{n}_{\sigma}$ por 
		$$
		\funcaocond{s_{\sigma}([q])}{q}{q \in \celula{n}{\sigma}}{p}{q \notin \celula{n}{\sigma}}
		$$
		chamamos de $\sigma-$aplicação de pares $p_{\sigma} = s_{\sigma} \circ \pi : (\skeleton{n}, \skeleton{n-1}) \to (S^{n}_{\sigma}, \{p\})$.
	\end{definicao}
	
	Denotaremos por $\Psi_{n}:\mathcal{C}_{k}(X) \to \homologiarelskelesimpl{k}{n}$ o isomorfismo definido no lema anterior dado por 
	$$
	\Psi(\sum_{\sigma} n_{\sigma} \sigma) = \sum_{\sigma} n_{\sigma} f_{\sigma *}[D^{n}],
	$$
	onde $[D^{n}]$ é um gerador do módulo $\homologiarelcel{n}{n}{}$.
	
	\begin{lema}
		(Inversa de $\Psi_{n}$) A aplicação inversa $\Phi_{n} : \homologiarelskelesimpl{n}{n} \to \mathcal{C}_{n}(X)$ do isomorfismo definido anteriormente é dada por
		$$
		\Phi_{n}(\alpha) = \sum_{\sigma} \phi_{n}(p_{\sigma *}\alpha)\sigma,
		$$
		onde $\phi_{n}: \homologiarel{n}{S^{n}}{\{p\}} \to \Lambda$ é o único homomorfismo tal que $\phi_{n}([S^{n}])=1$ e $[S^{n}]$ é a classe fundamental do par $(S^{n}, \{p\})$.
	\end{lema}
	\begin{prova}
		Mostremos a unicidade do homomorfismo. Sabemos que o grupo $\homologiarel{n}{S^{n}}{\{p\}}$ tem como geradores as classes $\{[S^{n}], [0]\}$ e como $\phi_{n}$ é homomorfismo, então $\phi_{n}([0]) = 0$. Definindo $\phi_{n}([S^{n}]) = 1$ e supondo que exista outro homomorfismo tal que $\phi_{n}^{'} ([S^{n}]) = 1$, então ambos homomorfismos coincidem quando avaliados nos geradores, logo $\phi_{n}^{'}=\phi_{n}$ o que é uma contradição, portanto $\phi_{n}$ é único. Sabemos o que $\Psi_{n}$ é um isomorfismo, então existe uma única aplicação $\Phi_{n}$ tal que $\Phi_{n} \circ \Psi_{n} = 1$. Tomando $\sigma$ uma $n-$célula geradora de $\mathcal{C}_{n}(X)$,
		$$
		\begin{aligned}
		\Phi_{n}(\Psi_{n}(\sigma)) 
		&= \sum_{\beta}\phi_{n}(p_{\beta *}\Psi_{n}(\sigma))\beta
		\\
		&= \sum_{\beta}\phi_{n}(p_{\beta *}f_{\partial\sigma *}[D^{n}])\beta
		\\
		&= \sum_{\beta}\phi_{n}((p_{\beta}\circ f_{\partial\sigma})_{*}[D^{n}])\beta
		\\
		&= \phi_{n}((p_{\sigma}\circ f_{\partial\sigma})_{*}[D^{n}])\sigma
		\\
		&= \phi_{n}([S^{n}])\sigma
		\\
		&= \sigma	
		\end{aligned},
		$$
		e como $\sigma \in \mathcal{C}_{n}(X)$ é arbitrário, então $\Phi_{n} \circ \Psi_{n} = 1$, como desejávamos.
	\end{prova}
	
	\begin{definicao}
		(Grau homológico) Seja $f: S^{n} \to S^{n}$ uma aplicação contínua e $f_{*}: \homologia{n}{S^{n}} \to \homologia{n}{S^{n}}$ o homomorfismo induzindo. Seja $[S^{n}] \in \homologia{n}{S^{n}}$ o gerador não-trivial desse grupo, então $f_{*}[S^{n}] = k[S^{n}]$ para algum $k \in \Lambda$. Denominamos $k=deg(f)$ o grau homológico da aplicação $f$.
	\end{definicao}
	
	\begin{definicao}
		(Aplicação CW-bordo) Tome a tripla $(\skeleton{n}, \skeleton{n-1}, \skeleton{n-2})$ e a composição abaixo
		\[
		\xymatrix{
			\mathcal{C}_{n}(X) \ar[r]^{\Psi_{n}\qquad} &
			\homologiarelskelesimpl{n}{n} \ar[r]^{\delta_{*}} & 
			\homologiarelskele{n-1}{n-1}{n-2} \ar[r]^{\qquad \Phi_{n-1}}&
			\mathcal{C}_{n-1}(X)
		}
		\]
		onde $\delta_{*}$ é o homomorfismo de conexão da sequência da tripla. Denominamos por operador CW-bordo o homomorfismo $\partial_{n} = \Phi_{n-1} \circ \delta_{*} \circ \Psi_{n} : \mathcal{C}_{n}(X) \to \mathcal{C}_{n-1}(X)$.
	\end{definicao}
	
	\begin{teorema}
		(CW-bordo) A aplicação CW-bordo é um homomorfismo tal que $\partial_{n-1}\circ\partial_{n} = 0$ e é dado por:
		$$
		\partial_{n}(\sigma) = \sum_{\beta}[\beta:\sigma]\beta,
		$$
		onde $[\beta:\sigma]$ é o grau homológico da aplicação $p_{\beta} \circ f_{\partial\sigma}:\celulabordo{n}{\sigma} \to S^{n-1}_{\sigma}$.
	\end{teorema}
	\begin{prova}
		Por definição temos que $\partial_{n} = \Phi_{n-1} \circ \delta_{*} \circ \Psi_{n}$, logo é um homomorfismo.
		
		Consideremos o diagrama comutativo tal que na vertical temos a sequência exata longa do par $(\skeleton{n-1}, \skeleton{n-2})$
		$$
		\xymatrix{
			& \homologiaabrev{n-2}{\skeleton{n-2}}\ar[rd]^{j_{*}}
			\\
			\homologiarelskele{n}{n}{n-1} \ar[r]^{\delta_{*} \qquad}\ar[rd]_{\delta_{n}} &
			\homologiarelskele{n-1}{n-1}{n-2} \ar[u]^{\delta_{n-1}} \ar[r]^{ \delta_{*}}&
			\homologiarelskele{n-2}{n-2}{n-3}
			\\
			& \homologiaabrev{n-1}{\skeleton{n-1}}\ar[u]^{j_{*}}
		}
		$$
		Note que $\delta_{*} \circ \delta_{*} = j_{*} \circ \delta_{n-1} \circ j_{*} \circ \delta_{n}$. Pela exatidão da sequência vertical temos $Im(j_{*}) = Ker(\delta_{n-1})$, logo $\delta_{*}^{2}=0$. Com isso, temos o composição do CW-bordo $\partial_{n-1}\circ \partial_{n} = \Phi_{n-2} \circ \delta_{*} \circ \Psi_{n-1} \circ \Phi_{n-1} \circ \delta_{*} \circ \Psi_{n} = \Phi_{n-2} \circ \delta_{*}^{2} \circ \Psi_{n} =0$, pois $\Psi_{n-1} \circ \Phi_{n-1}=1$.
		
		Por definição temos $f_{\partial\sigma}: \celulabordo{n}{\sigma} \to \skeleton{n-1}$, assim temos o homomorfismo induzido $f_{\partial\sigma*}: \homologia{n-1}{\celulabordo{n}{\sigma} }\to \homologia{n-1}{\skeleton{n-1}}$. Analogamente, temos o homomorfismo $f_{\sigma*}:\homologiarelcel{n}{n}{\sigma} \to \homologiarelskelesimpl{n}{n}$ e o homomorfismo conectante $\delta_{n} : \homologiarelcel{n}{n}{\sigma} \to \homologia{n-1}{\celulabordo{n}{\sigma}}$ de modo que, tomanto $[\celula{n}{\sigma}] \in \homologiarelcel{n}{n}{\sigma}$ como um elemento gerador, então $\delta_{n}\circ f_{\sigma*}[\celula{n}{\sigma}] \in \homologia{n-1}{\skeleton{n-1}}$ é um elemento gerador. Por outro lado $f_{\partial\sigma*}\circ \delta_{n}[\celula{n}{\sigma}] \in \homologia{n-1}{\skeleton{n-1}}$ também é um elemento gerador, logo $f_{\partial\sigma*}\circ \delta_{n}[\celula{n}{\sigma}] = \lambda \delta_{n}\circ f_{\sigma*}[\celula{n}{\sigma}]$, para algum $\lambda \in \Lambda$. Como sempre podemos escolher um mapa caracteristico $f_{\sigma *}$ tal que $\lambda = 1$, temos $f_{\partial\sigma*}\circ \delta_{n} = \delta_{n}\circ f_{\partial\sigma*}$, e como $\delta_{*} = j_{*}\circ\delta_{n}$ então $f_{\partial\sigma*}\circ \delta_{*} = \delta_{*}\circ f_{\partial\sigma*}$. Assim, temos o operador CW-bordo
		$$
		\begin{aligned}
		\partial_{n}(\sigma) &= \Phi_{n-1}\circ\delta_{*}\circ\Psi_{n}(\sigma)
		\\
		&= \Phi_{n-1}\circ\delta_{*}\circ f_{\sigma*}([\celula{n}{\sigma}])
		\\
		&= \Phi_{n-1}\circ f_{\partial\sigma*}\circ\delta_{*}([\celula{n}{\sigma}])
		\\
		&= \Phi_{n-1}\circ f_{\partial\sigma*}\circ (j_{*}\circ \delta_{n}) ([\celula{n}{\sigma}])
		\\
		&= \Phi_{n-1} \circ f_{\partial\sigma*}([\celulabordo{n}{\sigma}])
		\\
		&= \sum_{\beta} \phi_{n-1}(p_{\beta*}\circ f_{\partial\sigma*}[\celulabordo{n}{\sigma}])\beta
		\\
		&= \sum_{\beta} \phi_{n-1}((p_{\beta}\circ f_{\partial\sigma})_{*}[S^{n-1}])\beta
		\\
		&= \sum_{\beta} \phi_{n-1}(deg(p_{\beta}\circ f_{\partial\sigma})[S^{n-1}])\beta
		\\
		&= \sum_{\beta} deg(p_{\beta}\circ f_{\partial\sigma})\phi_{n-1}([S^{n-1}])\beta
		\\
		&= \sum_{\beta} deg(p_{\beta}\circ f_{\partial\sigma})\beta,
		\end{aligned}
		$$
		como desejávamos.
	\end{prova}
	
	\begin{teorema}\label{teorema_cw_homologia}
		(CW-homologia) Seja $X$ um CW-complexo, então existe uma identificação natural entre a CW-homologia $\mathcal{C}_{*}(X)$ e a homologia singular $\homologia{*}{X}$, isto é 
		$$
		\homologia{k}{X} \cong \homologia{k}{\mathcal{C}_{*}(X)}\; \forall k \in \inteiros.
		$$
	\end{teorema}
	\begin{proof}
		Para a demonstração desse resultado vamos considerar a sequência 
		$$
		\xymatrix{
			\mathcal{C}_{k+1}(X) \ar[r]^{\partial_{k+1}} & \mathcal{C}_{k}(X) \ar[r]^{\partial_{k}} & \mathcal{C}_{k-1}(X).
		}
		$$
		Por definição temos $\homologia{k}{\mathcal{C}_{*}(X)} = Ker(\partial_{k})/Im(\partial_{k+1})$, e com isso provaremos que 
		$$
		Ker(\partial_{k})/Im(\partial_{k+1}) \cong \homologia{k}{X}
		$$ 
		e então concluir a equivalência entre a CW-homologia e a homologia singular do espaço topológico $X$.
		
		Tomemos a sequência longa exata vertical da tripla $(\skeleton{k+1}, \skeleton{k-1}, \skeleton{k-2})$ e a sequência longa exata horizontal da tripla $(\skeleton{k+1}, \skeleton{k}, \skeleton{k-1})$  no diagrama abaixo:
		$$
		\xymatrix{
			& & \homologiarelskele{k}{k-1}{k-2}= 0 \ar[d]^{i_{*}} &
			\\
			& & \homologiarelskele{k}{k+1}{k-2} \ar[d]^{j_{*}} &
			\\
			\homologiarelskele{k+1}{k+1}{k} \ar[r]^{\quad\delta_{1*}} &		\homologiarelskele{k}{k}{k-1} \ar[r]^{i_{*}} \ar[rd]^{\delta_{2*}} & \homologiarelskele{k}{k+1}{k-1} \ar[r]^{j_{*}} \ar[d]^{\delta_{3*}} & \homologiarelskele{k}{k+1}{k}=0
			\\
			& & \homologiarelskele{k-1}{k-1}{k-2} &
		}
		$$
		onde $i_{*}, \; j_{*}$ e $\delta_{*}$ são as inclusões induzidas e o homomorfismo conectante, respectivamente. Seja $\classe{\alpha} \in \homologiarelskelesimpl{k}{k}$, então $\delta_{3*}\circ i_{*}\classe{\alpha} = \delta_{2*}\classe{\alpha}$.
		
		Vamos agora caracterizar $Ker(\delta_{2*})$. Dado $[\alpha] \in Ker(\delta_{2*})$, então $\delta_{3*}\circ i_{*}\classe{\alpha} = \delta_{2*}\classe{\alpha} = 0$. Como $i_{*}$ é um epimorfismo e $j_{*}$ é monomorfismo, então existe um único $\classe{\beta} \in \homologiarelskele{k}{k+1}{k-2}$ tal que $i_{*} \classe{\alpha} = j_{*} \classe{\beta}$. 
		
		Afirmo que $\phi: Ker(\delta_{2*}) \to \homologiarelskele{k}{k+1}{k-2}$ dado por $\phi(\classe{\alpha}) = \classe{\beta}$ é um epimorfismo. Seja $\classe{\beta'} \in \homologiarelskele{k}{k+1}{k-2}$, então existe um $\classe{\alpha'} \in \homologiarelskele{k}{k}{k-1}$ tal que $i_{*} \classe{\alpha'} = j_{*} \classe{\beta'}$. Com isso $\delta_{2*}\classe{\alpha'} = \delta_{3*}\circ i_{*}\classe{\alpha'} = \delta_{3*}\circ j_{*}\classe{\beta'} = 0$ pois $Im(j_{*}) = Ker(\delta_{3*})$, logo $\classe{\alpha'} \in Ker(\delta_{2*})$ e $\phi$ é epimorfismo.
		
		Como $\phi$ é sobrejetor, existe $\classe{\alpha} \in Ker(\delta_{2*})$ tal que $\phi(\classe{\alpha}) = 0$. Pela comutatividade do diagrama temos $i_{*}{\classe{\alpha}} = j_{*}\classe{0} = 0$, pois $j_{*}$ é monomorfismo, portanto $\classe{\alpha} \in Ker(i_{*})$ e, pela exatidão, temos $Ker(i_{*})=Im(\delta_{1*})$, logo $\classe{\alpha} \in Im(\delta_{1*})$. Com isso podemos concluir que $Ker(\phi) = Im(\delta_{1*})$. Pelo teorema fundamental do isomorfismo de grupos temos que $Ker(\delta_{2*})/Ker(\phi) \cong \homologiarelskele{k}{k+1}{k-2}$, ou seja, $Ker(\delta_{2*})/Im(\delta_{1*}) \cong \homologiarelskele{k}{k+1}{k-2}$.
		
		Sem perda de generalidade, vamos assumir que $X$ seja um CW-complexo de ordem $n$, isto é, $X= \skeleton{n}$. Fixemos um $0 \leq j \leq n$ e tomemos o $j-$ésimo esqueleto $\skeleton{j}$ e definamos $\skeleton{-1}$. Com isso, podemos escrever a sequência de homomorfismos de inclusão de pares:
		$$
		\xymatrix{
			\homologia{k}{\skeleton{j}} = \homologiarel{k}{\skeleton{j}}{\skeleton{-1}}\ar[r]& \homologiarel{k}{\skeleton{j}}{\skeleton{0}} \ar[r] & \dots \ar[r] & \homologiarel{k}{\skeleton{j}}{\skeleton{k-2}}
		}
		$$
		onde $k-2 \leq j$ e para cada $i-$ésimo termo $\homologiarelskele{k}{j}{i}$ no centro do diagrama abaixo, teremos a sequência exata de triplas nas verticais $(\skeleton{j}, \skeleton{i}, \skeleton{i-1})$, com $0\leq i +1\leq j$ com $h_{i}$ sendo os homomorfismos de inclusão:
		$$
		\xymatrix{
			\homologiarel{k}{\skeleton{i-1}}{\skeleton{i-2}}=0 \ar[d] & \homologiarel{k}{\skeleton{i}}{\skeleton{i-1}}=0 \ar[d] & \homologiarel{k}{\skeleton{i+1}}{\skeleton{i}}=0 \ar[d] &	
			\\
			\homologiarel{k}{\skeleton{n}}{\skeleton{i-2}} \ar[d]^{h_{i-1}} & \homologiarel{k}{\skeleton{n}}{\skeleton{i-1}} \ar[d]^{h_{i}} & \homologiarel{k}{\skeleton{n}}{\skeleton{i}} \ar[d]^{h_{i+1}}
			\\
			\homologiarel{k}{\skeleton{n}}{\skeleton{i-1}} \ar[r]\ar[d]^{\delta_{(i-1)*}}& \homologiarel{k}{\skeleton{n}}{\skeleton{i}} \ar[r] \ar[d]^{\delta_{i*}} &  \homologiarel{k}{\skeleton{n}}{\skeleton{i+1}} \ar[d]^{\delta_{(i+1)*}} 
			\\
			\homologiarelskele{k-1}{i-1}{i-2}=0& \homologiarel{k-1}{\skeleton{i}}{\skeleton{i-1}}=0 &  \homologiarel{k}{\skeleton{i+1}}{\skeleton{i}}=0. &		
		}
		$$
		Do Lema \ref{homologiacelular} temos que $\homologiarelskele{k}{i}{i-1} =0$ caso $k \neq i$, logo os grupos nas extemidades verticais do diagrama serão os triviais. Pela exatidão das sequências verticais temos $Im(h_{i}) = Ker(\delta_{i*})$, mas como $Im(\delta_{i*}) = 0 \Rightarrow Ker(\delta_{i*}) = \homologiarel{k}{\skeleton{n}}{\skeleton{i}}$, logo $h_{i}$ é um epimorfismo, portanto um isomorfismo, isto é, $\homologiarel{k}{\skeleton{n}}{\skeleton{i-1}} \cong \homologiarel{k}{\skeleton{n}}{\skeleton{i}}$, o que nos pertmite escrever a cadeia de isomorfismos 
		$$
		\begin{aligned}
		\homologia{k}{\skeleton{j}} &= \homologiarel{k}{\skeleton{j}}{\skeleton{-1}} 
		\\
		&\cong  \homologiarel{k}{\skeleton{j}}{\skeleton{0}} \cong \dots \cong  \homologiarel{k}{\skeleton{j}}{\skeleton{i}} \cong  \dots \cong \homologiarel{k}{\skeleton{j}}{\skeleton{k-2}},
		\end{aligned}
		$$
		logo $\homologia{k}{\skeleton{j}} \cong \homologiarel{k}{\skeleton{j}}{\skeleton{i}}$.
		
		Por fim, como supusemos que $X = \skeleton{n}$ e a construção anterior vale para $j = n$, então $\homologia{k}{X} = \homologia{k}{\skeleton{n}} \cong \homologiarelskele{k}{n}{i}$.
		
		Assim, $\homologia{k}{X} \cong \homologiarel{k}{\skeleton{k+1}}{\skeleton{k-2}} \cong Ker(\delta_{2*})/Im(\delta_{1*})$. Mas como $\partial_{k} = \Phi_{n-1}\circ\delta_{*}\circ\Psi_{n}$, então $Ker(\delta_{2*}) \cong Ker(\partial_{k})$ e $Im(\delta_{1*}) \cong Im(\partial_{k+1})$, logo $Ker(\delta_{2*})/Im(\delta_{1*}) \cong Ker(\partial_{k})/Im(\partial_{k+1}) = \homologia{k}{\mathcal{C}_{*}}$, logo $\homologia{k}{X} \cong \homologia{k}{\mathcal{C}_{*}}$ que é a equivalência entre as homologias, como desejávamos.
	\end{proof}
	
	\begin{exemplo}
		(CW-homologia da n-esfera)
		Vamos exibir uma estrutura de CW-complexo para $S^{n}$. Para isso tomemos um ponto $p \in S^{n}$ e definamos o $0-$skeleton $\skeleton{0}=\{p\}$. Em seguida, anexemos uma $n-$célula a $\skeleton{0}$ onde $f_{\partial}: \celulabordo{n}{} \to \skeleton{0}$, isto é, $\skeleton{n} = \{p\}\cup_{f_{\partial}} \celula{n}{}$. Pelo teorema da CW-homologia temos que $\homologia{k}{S^{n}} \cong \homologiarelskelesimpl{k}{k}$, logo $\homologia{0}{S^{n}} \cong \homologiarelskelesimpl{0}{0} \cong \Lambda$, $\homologia{n}{S^{n}} \cong \homologiarelskelesimpl{n}{n} \cong \Lambda$ e $\homologia{k}{S^{n}} \cong \homologiarelskelesimpl{k}{j} =0$ caso $k \neq j$, logo
		$$
		\homologia{*}{S^{n}} = \homologia{0}{S^{n}}\oplus\homologia{n}{S^{n}} \cong \Lambda\oplus\Lambda.
		$$
	\end{exemplo}
	
	\begin{exemplo}
		(CW-homologia do 2-toro) Vamos exibir uma estrutura de CW-complexo para $T^{2}$. Para isso tomemos a identificação do toro com o quadrado cujo os lados opostos serão identificados, assim os vertices do quadrado serão um ponto $p \in T^{2}$ e definindo o $0-$skeleton $\skeleton{0} = \{p\}$, agora vamos anexar às faces do quadrado duas $1-$células, isto é, $\skeleton{1} = \skeleton{0}\cup_{f_{1\partial}}\celula{1}{1}\cup_{f_{2\partial}}\celula{1}{2}$, e por fim, cobrir o centro do quandrado anexando um $2-$célula, com isso, $\skeleton{2} = \skeleton{1}\cup_{f_{3\partial}}\celula{2}{3}$. Então
		$$
		T^{2} =\skeleton{2} = \skeleton{0}\cup_{f_{1\partial}}\celula{1}{1}\cup_{f_{2\partial}}\celula{1}{2}\cup_{f_{3\partial}}\celula{2}{3}.
		$$
		Teremos os grupos de homologia não-triviais:
		$$
		\begin{aligned}
		\homologia{0}{T^{2}} &\cong \homologiarelskele{0}{0}{-1} \cong \Lambda,
		\\
		\homologia{1}{T^{2}} &\cong \homologiarelskele{1}{1}{0} \cong \somadir{i=1,2}\homologiarelcel{1}{1}{i} \cong \somadir{i=1,2}\Lambda
		\\
		\homologia{2}{T^{2}} &\cong \homologiarelskele{2}{2}{1} \cong \Lambda.
		\end{aligned}
		$$
		Logo,
		$$
		\\
		\homologia{*}{T^{2}} = \homologia{0}{T^{2}}\oplus\homologia{1}{T^{2}} \oplus\homologia{2}{T^{2}}\cong \Lambda\oplus\Lambda\oplus\Lambda\oplus\Lambda.
		$$
	\end{exemplo}	
	
	
	\chapter{Teoria de Morse}\label{capitulo_teoria_morse}
	Na investigação da topologia de variedades o principal objetivo é a determinação de seus invariantes topológicos, ou seja, características das variedades que são invariantes por homeomorfismos. Algumas abordagens são algébricas, tais como: a determinação de seus grupos de homotopia, seus grupos de homologia e cohomologia. Outras são diferenciais, tais como: teoremas de mergulho, transversalidade, teorema de Sard, etc. A Teoria de Morse faz uma conexão entre as duas metodologias. Através de uma função suave definida na variedade, determina-se seus pontos críticos, e através deles se constrói um $CW$-complexo cuja homologia coincide com a homologia singular da variedade, o que pode ser encontrado na Seção $\ref{secao_cw_complexo}$. O que torna a Teoria de Morse uma das construções matemáticas mais bonitas do século XX é justamente essa conexão entre as diferentes descrições.
	
	O propósito desse capítulo é de apresentação rápida do formalismo e alguns dos principais resultados para que possamos estender a construção aqui feita para a homologia de variedades de dimensão infinita (Homologia de Floer). Os detalhes das demonstrações e técnicas utilizadas podem ser encontrados em $\cite{milnor}$ e $\cite{banyaga_morse_homology}$.
	
	\begin{definicao}\label{definicao_variedade_fechada}
		(Variedade fechada) Seja $M$ uma n-variedade diferencial. Diz-se que $M$ é fechada se é compacta e sem bordo.
	\end{definicao}
	
	De agora em diante $M$ será uma $n$-variedade Riemanniana diferenciável fechada, cujas definições podem ser encontradas no Apêndice \ref{apendice_variedade_riemanniana}.
	
	\section{Teoria de Morse Clássica}\label{secao_morse_classica}
	
	\begin{definicao}
		(Função de Morse) Sejam $M$ uma n-variedade fechada, $f \in \funcoessuaves{M}$ e $\pontoscriticos{f} = \{p \in M: df_{p} = 0\}$ o conjunto dos pontos críticos de $f$. Dizemos que $f$ é uma função de Morse se a hessiana $H_{p}(f)$ é não-degenerada para todo $p \in \pontoscriticos{f}$. O conjunto das funções de Morse definidas em $M$ será denotado por $\funcoesmorse{M}$. 
	\end{definicao}

	O índice de Morse de um dado $p \in \pontoscriticos{f}$ é a dimensão do maior subespaço $V\subset T_{p}M $ tal que \ hessiana é negativa-definita, isto é, $H_{p}(f)(v,u)<0$ para todo $v,u \in V$. Denotaremos esse índice por $\lambda_{p} = dim(V)$. Como a hessiana é não-degenerada, então $H_{p}(f)$ é diagonalizável e o número de auto-valores negativos é o índice $\lambda_{p}$.
	
	\begin{lema}
		(Lema de Morse) Sejam $f \in \funcoesmorse{M}$ e $p \in \pontoscriticos{f}$ com índice de Morse $\lambda_{p}$. Então existe uma carta $\{U, \phi\}$ de $p$ com $\phi(p)=0 \in \real{n}$ tal que 
		$$
		\begin{aligned}
		(f\circ \phi^{-1})(x_{1}, \dots, x_{n}) &= f(p)-x_{1}^{2}-\dots -x^{2}_{\lambda_{p}}+x^{2}_{\lambda_{p}+1}+\dots + x^{2}_{n}
		\\
		&=f(p)+x^{t}Dx,
		\end{aligned}
		$$
		onde $D$ é a representação diagonal de $H_{p}(f)$.
	\end{lema}
	
	\begin{observacao}
		Como consequência do Lema de Morse, pode-se mostrar que os pontos críticos de uma função de Morse $f \in \funcoesmorse{M}$ são isolados. Pela compacidade de $M$ tem-se que os pontos isolados são finitos.
	\end{observacao}
	
	
	A existência das funções de Morse é ilustrada pelo seguinte exemplo. Seja $f:M\to \reta$ tal que $f(x_{1}, \dots, x_{n}) = x_{n}$. Essa função é chamada de função altura e pode-se mostrar que $f$ é uma função de Morse, logo $\funcoesmorse{M} \neq \emptyset$.
	
	\begin{exemplo}
		Sejam $S^{2} \subset \real{3}$ a 2-esfera centrada na origem e $f:S^{2}\to \reta$ a função altura dadar por $f(x,y,z) = z$. Os pontos críticos de $f$ são $p_{\pm} = \{(0,0,\pm 1)\}$, cujos índices são $\lambda_{p_{-} } = 0$ e $\lambda_{p_{+}} = 2$.
	\end{exemplo}
	
	As funções de Morse não são um caso raro nessa descrição, muito pelo contrário. O seguinte teorema garante que tais funções são abundantes no conjunto das funções suaves e tal resultado pode ser encontrado em $\cite{amyia_diff_topology}$.
	
	\begin{teorema}
		Seja $g\in \funcoessuaves{M}$. Então existe $f \in \funcoesmorse{M}$ suficientemente próxima a $g$, isto é, $\funcoesmorse{M}$ é denso em $\funcoessuaves{M}$.
	\end{teorema} 
	
	Dado $a \in \reta$, definimos o conjunto $M^{a}= f^{-1}((-\infty, a]) = \{p \in M: f(p)\leq a\}$ como sendo o conjunto em $M$ de nível $a$. Uma consequência imediata é que, dados $a\leq b \in \reta$, então $M^{a} \subseteq M^{b}$.
	
	\begin{teorema}
		Sejam $f \in \funcoesmorse{M}$ e $a<b \in \reta$ tais que $f^{-1}([a,b])\subset M$ não contenha pontos críticos de $f$. Então $M^{a}$ é difeomorfo a $M^{b}$. Além disso, $M^{a}$ é um retrato de deformação de $M^{b}$, de modo que a inclusão  $M^{a} \hookrightarrow M^{b}$ é uma equivalência homotópica.
	\end{teorema}
	
	O seguinte teorema afirma que, fixando $a \in \reta$ e variando $t \in \reta$, se $f^{-1}([a,t]) \cap \pontoscriticos{f} \neq \emptyset$, os conjuntos de nível $M^{a}$ e $M^{t}$ não podem ser deformados um no outro.
	
	\begin{teorema}\label{teorema cw_complexo_ponto_critico_morse}
		Sejam $f\in \funcoesmorse{M}$ e $p\in \pontoscriticos{f}$ com índice $\lambda_{p}$ tal que $f(p) = c$. Suponhamos que$f^{-1}([c-\epsilon,c+\epsilon])\cap \pontoscriticos{f} = \{p\}$ para algum $\epsilon>0$. Então o conjunto de nível $M^{c+\epsilon}$ tem o mesmo tipo de homotopia de $M^{c+\epsilon}$ com uma $\lambda_{p}$-célula colada, isto é, $M^{c+\epsilon} \simeq M^{c-\epsilon}\cup_{f_{\partial}} D^{\lambda_{p}}$.
	\end{teorema}
	
	\begin{observacao}
		O teorema anterior tem como hipótese a exitência de apenas um ponto crítico em $f^{-1}([c-\epsilon,c+\epsilon])$. No caso em que $f^{-1}([c-\epsilon,c+\epsilon]) \cap \pontoscriticos{f} = \{p_{j}\}_{j=1}^{r}$, teremos $M^{c+\epsilon} \simeq M^{c-\epsilon}\cup_{f_{\partial_{1}}} D^{\lambda_{1}}\dots  \cup_{f_{\partial_{r}}} D^{\lambda_{r}}$.
	\end{observacao}
	
	\begin{observacao}
		Uma consequência imediata do teorema anterior é que, supondo $p \in \pontoscriticos{f}$ e tomando $a = f(p)$, temos que $M^{a}$ é um $CW$-complexo. 
	\end{observacao}
	
	Supondo que $M$ seja compacto e conexo, então $f(M) = [a,b] \subset \reta$ é um compacto. Suponha que $\pontoscriticos{f} = \{p_{j}\}_{j=1}^{k}$, $\lambda_{j}$ seja o índice do j-ésimo ponto crítico e que exista uma partição $t_{1} = a < t_{2}< \dots< t_{k+1} = b$ tal que $f^{-1}((t_{k}, t_{k+1})) \cap \pontoscriticos{f} = \{p_{k}\}$. Ao variarmos $t \in [a,b]$ teremos a construção do $CW$-complexo
	$$
	\begin{aligned}
	M^{t_{1}} &\simeq \{p_{1}\}
	\\
	M^{t_{2}} & \simeq \{p_{1}\} \cup_{f_{\partial_{1}}} D^{\lambda_{1}}
	\\
	\vdots&
	\\
	M = M^{t_{k+1}} &\simeq\{p_{1}\} \cup_{f_{\partial_{1}}} D^{\lambda_{1}}\dots  \cup_{f_{\partial_{k}}} D^{\lambda_{k}}.
	\end{aligned}
	$$
	
	Caso existam pontos críticos distintos com valores críticos coincidentes a mesma construção pode ser feita perturbando a função de Morse $f$ de modo a obter uma outra $f' \in \funcoesmorse{M}$ com os mesmos pontos críticos agora com valores críticos todos distintos. Essa construção é viabilizada pela densidade das funções de Morse em $\funcoessuaves{M}$.
	
	\begin{exemplo}
		(2-toro) Sejam $T^{2} \subset \real{3}$ o toro centrado na origem do plano $Oy\times Ox = \{(x, y ,0) \in \real{3}\}$ e $\phi:[0,2\pi]\times [0,2\pi] \to \real{3}$ uma parametrização dada por 
		$$
		\varphi(\theta, \phi) = ((2+cos\phi)cos\theta, (2+cos\phi)sin\theta, sin\phi).
		$$
		
		Com isso, tem-se que o espaço tangente $T_{p}T^{2} $ é gerado por $ \{\derivadaparcialabrev{\theta}, \derivadaparcialabrev{\phi} \}|_{p}$ onde
		$$
		\begin{aligned}
			\derivadaparcialabrev{\theta} &= -(2+cos\phi)sin\theta \derivadaparcialabrev{x}+(2+cos\phi)cos\theta \derivadaparcialabrev{y},
			\\
			\derivadaparcialabrev{\phi} &=  - sin\phi cos\theta 	\derivadaparcialabrev{x} - sin\phi sin\theta 	\derivadaparcialabrev{y} +cos\phi				\derivadaparcialabrev{z}.
			\end{aligned} 
		$$
				
		Seja $f:T^{2} \to \reta$ a função largura $f(x,y,z)=x$. Tem-se que o diferencial de $f$ e $df_{p} = \derivadaparcialabrev{x}$. Com isso, dado $v \in \real{3}$ tem-se que $df_{p}(v) = \produtointerno{\nabla f(p)}{v}  = \produtointerno{(1,0,0)}{(v_{x}, v_{y}, v_{z})} =v_{x}$. Assim, o caso em que $df_{p}(v) = 0$ implica que $v_{x} = 0$, ou seja, os vetores dos espaços tangentes não podem ter componentes na direção $Ox$. Com isso, basta que os elementos da base não o tenham, isto é, $sin\phi = sin\theta = 0$. Portanto, $\phi, \theta \in \{0, \pi\}$ e os pontos críticos de $f$ são $\varphi(0,0) = (3,0,0)$, $\varphi(0,\pi) = (-1,0,0)$, $\varphi(\pi,0) = (-3,0,0)$ e $\varphi(\pi,\pi) = (1,0,0)$.
			
		Assim, a representação matricial da Hessiana $H_{p}(f)$ é
			$$
			H(\phi, \theta) = 
			\left(
			\begin{array}{cc}
			-(2+cos\phi)cos\theta & sin\phi sin\theta  
			\\
			sin\phi sin\theta   & -cos\phi cos\theta  
			\end{array}
			\right).
			$$
		Avaliando a hessiana nos pontos críticos tem-se
			$$
			H(0, 0) = 
			\left(
			\begin{array}{cc}
			-3 & 0
			\\
			0& -1
			\end{array}
			\right),
			H(0, \pi) = 
			\left(
			\begin{array}{cc}
			3 & 0
			\\
			0& 1
			\end{array}
			\right)
			$$
			$$
			H(\pi, 0) = 
			\left(
			\begin{array}{cc}
			-1 & 0
			\\
			0& 1
			\end{array}
			\right),
			H(\pi, \pi) = 
			\left(
			\begin{array}{cc}
			1 & 0
			\\
			0& -1
			\end{array}
			\right)
			$$
			
			Com isso, os índices $\lambda_{(\phi, \theta)}$ dos pontos críticos são: $\lambda_{(0,0)} = 2$, $\lambda_{(0,\pi)} = 0$, $\lambda_{(\pi,0)} = 1$ e $\lambda_{(\pi,\pi)} = 1$.
			
	\end{exemplo}

	Originalmente, a relação entre a topologia de $M$ e os pontos críticos de uma função de Morse $f:M \to \reta$, foi dada em termos de desigualdades, chamadas desigualdades de Morse.
	
	\begin{definicao}
		(Números de Betti) O j-ésimo número de Betti de $M$ é o inteiro $\beta_{j}(M) = dim(H_{j}(M))$, onde $H_{j}(M)$ é o j-ésimo grupo de homologia singular de $M$.
	\end{definicao}
	
	Como os grupos de homologia de $M$ são invariantes topológicos, então os números de Betti de $M$ também são.
	
	Seja $f \in \funcoesmorse{M}$. O número de pontos críticos de índice $k$ é denotado por $\nu_{k}$. Note que $\beta_{k}(M)$ contém as informações sobre a topologia da variedade, por outro lado, $\nu_{k}$ contém as informações sobre os pontos críticos de $f$.
	
	\begin{teorema}
		(Desigualdades de Morse) Sejam $f \in \funcoesmorse{M}$ e $\nu_{k}$ o número de pontos críticos de $f$ com índice de Morse $k$. Então valem as seguintes desigualdades:
		\begin{enumerate}
			\item $\beta_{k}(M) \leq \nu_{k}$ para $0\leq k\leq n$,
			
			\item $\sum_{j = 0}^{k}(-1)^{k-j}\beta_{j}(M) \leq \sum_{j = 0}^{k}(-1)^{k-j}\nu_{j} $ para $0 \leq k \leq n$ e vale a igualdade para o caso em que $k=n$.
		\end{enumerate}
	\end{teorema}
	
	\begin{observacao}
		No caso da igualdade do segundo item do teorema temos a característica de Euler-Poincaré $\chi(M) = \sum_{j = 0}^{n}(-1)^{n-j}\beta_{j}(M)$, que também é um invariante topológico e é a generalização do teorema de Euler para poliedros convexos $V-A+F = 2$, onde $V$ é o número de vértices, $A$ é o número de arestas e $F$ é o número de faces.
	\end{observacao}
	
	\section{Fluxos Gradiente e Variedades de Conexão}\label{secao_fluxo_gradiente}
	Sejam $X \in \campossuaves{M}$ um campo vetorial suave em $M$ e $p \in M$. Sabe-se que existe uma curva $\gamma: \reta\to M$ que é solução do sistema de equações diferenciais 
	$$
	\derivada{\gamma(t)}{t} = X(\gamma(t)), \; \gamma(0) = p.
	$$
	Como $M$ é compacta e sem bordo, então a solução $\gamma = \gamma_{p}$ existe para todo $t\in \reta$. A aplicação $\phi: \retacartesianovariedade\to M$ definida por $\phi(t,p) = \gamma_{p}(t)$ é chamada de fluxo gerado por $X$ e, fixado $p\in M$, a curva $\gamma_{p}:\reta\to M$ é chamada de linha do fluxo de $X$. Como o fluxo $\phi$ está bem-definido para todo $t \in \reta$, então podemos efetuar a composição $\phi_{s}\circ\phi_{t}(p) = \phi(s+t, p)$ para todo $p\in M$, logo $\phi_{s}\circ\phi_{t} = \phi_{s+t}$ e $\phi_{0} = Id$.
	\begin{definicao}
		(Órbitas) A órbita de $p \in M$ pelo fluxo $\phi$ é a imagem da curva $\phi^{p} = \phi(.,p):\reta\to M$ e será denotada por $\orbitaponto{p}$. As órbitas podem ser de três categorias
		\begin{enumerate}
			\item Singular, se $\orbitaponto{p}$ = \{p\}.
			
			\item Fechada, se existe $\tau \in \reta$ tal que $\phi_{\tau+t}(p) = p$ para todo $t \in \reta$.
			
			\item Regular, quando não é singular e não é fechada.
		\end{enumerate}
		
	\end{definicao}
	
	Pela compacidade de $M$ e pela unicidade da solução do problema de valor inicial anteriormente descrito, temos que $\orbitaponto{p}\subset M$ é uma imersão injetiva quando é uma órbita regular. Vista como um subconjunto de $\real{m}$, para um inteiro $m>0$ suficientemente grande (Teorema de Megulho de Whitney $\cite{guillemin_differential_topology}$), $M$ é um conjunto limitado, logo $\orbitaponto{p}$ também é limitado. Assim, tal órbita admite pontos limites. Os conjunto $\alpha$-limite e $\omega$-limite de $p \in M$ são definidos por $\alpha(p) = \{q \in M: \phi_{t}(p) \to q, t \to -\infty\}$ e $\omega(p) = \{q \in M: \phi_{t}(p) \to q, t \to \infty\}$. O seguinte resultado sobre a topologia dos conjuntos limites é feito em $\cite{palis_dynamical_systems}$.
	
	\begin{proposicao}
		Sejam $M$ uma variedade compacta e $X\in \campossuaves{M}$. Então para $p \in M$ tem-se, que $\alpha(p)$ e $\omega(p)$ são fechados, conexos e invariantes pelo fluxo de $X$, isto é, são uniões de órbitas de $X$.
	\end{proposicao}
	
	Vamos trabalhar com o campo gradiente da função de Morse $f$ pois dele podemos extrair informações sobre o comportamento dessa função. Munindo $M$ com uma métrica Riemanniana $g: T_{p}M \times T_{p}M \to \reta$, temos que $df_{p}(v) = g(\nabla f(p), v)$, onde $v \in T_{p}M$. Sejam $-\gradiente \in \campossuaves{M}$ e $\gamma$ a curva integral de $-\gradiente $ tal que $\gamma(0) = p\in M$. Então
	$$
	\begin{aligned}
	\derivada{}{t}(f \circ \gamma)(t) &= g(\gradiente(\gamma(t)), \dot{\gamma}(t)) 
	\\
	&=g(\gradiente(\gamma(t)), -\gradiente(\gamma(t))) 
	\\
	&= -\norma{\gradiente(\gamma(t))}^{2}
	\\
	&\leq 0,
	\end{aligned}
	$$
	para todo $t \in \reta$.
	
	Isso mostra que a função de Morse $f$ é decrescente ao longo das linhas de fluxo do campo gradiente negativo. Além disso, supondo que $\gamma(0) =p\in \pontoscriticos{f}$, então a desigualdade acima atinge seu maior valor em $t=0$, pois $df_{p }= 0$. Como $f$ é decrescente e atinge seu máximo em $t=0$, então a órbita $\orbitaponto{p}$ é regular e não-fechada. Em $\cite{palis_dynamical_systems}$ mostra-se que, para todo $q \in M$ tem-se que $\orbitaponto{q}$ intersecta $f^{-1}(f(q))$ transversalmente. Além disso, os pontos limites das órbitas são pontos críticos, ou seja, $\alpha(q)\cup\omega(q) \subset \pontoscriticos{f}$.
	
	Sabemos que, se $f \in \funcoesmorse{M}$, então $\pontoscriticos{f}$ é um conjunto finito. Com isso, pode-se mostrar, 
	
	\begin{lema}\label{lema_conjunto_limite_funcao_morse}
		Se 
		$p \in M$, então $\alpha(p)  = \{q\}$ e $\omega(p) = \{r\}$ onde $q, r \in \pontoscriticos{f}$.
	\end{lema}
	
	\begin{definicao}
		(Variedades instáveis/estáveis)  As variedades instável e estável de um ponto $p \in \pontoscriticos{f}$ são os conjutos $\variedadeinstavel{p} = \{q\in M: \phi_{t}(q) \to p,\; t\to -\infty\}$ e $\variedadeestavel{p} = \{q\in M: \phi_{t}(q) \to p,\; t\to \infty\}$.
	\end{definicao}
	
	\begin{observacao}
		Em termos de pontos limite, podemos reescrever as variedades instáveis e estáveis de $p \in \pontoscriticos{f}$ como sendo $\variedadeinstavel{p} = \{q\in M: \alpha(q)=p\}$ e $\variedadeestavel{p} = \{q\in M: \omega(q)=p\}$.
	\end{observacao}
	
	Da observação anterior, podemos afirmar que ambas as variedades instável e estável dos pontos críticos são contráteis. De fato, defina $Id: M\to M$ como sendo a aplicação identidade e $c:M\to M$ como sendo a aplicação constante $c(M) =\{p\} $, onde $p \in \pontoscriticos{f}$. Então $h:\intervalo\times \variedadeinstavel{p} \to \variedadeinstavel{p}$ tal que $h(t,q) = \phi(t/(1-t), q)$ é uma homotopia entre $Id$ e $c$, pois é contínua e $h(0, q) = Id(q)$ e $\lim_{t \to 1}h(t, q) = c(q)$, logo a variedade instável é contrátil. Com uma argumento análogo mostra-se que $\variedadeestavel{p}$ é contrátil.
	
	Definimos o espaço tangente instável como sendo o subespaço $\espacotangentevariedadeinstavel\subset \espacotangentevariedade$ tal que a restrição da Hessiana $\hessiana$ a $\espacotangentevariedadeinstavel$ é negativa-definida. Analogamente temos o espaço tangente estável $\espacotangentevariedadeestavel \subset \espacotangentevariedade$, onde a Hessiana é positiva-definida. Com isso, segue o teorema da variedade estável, cuja demonstração pode ser encontrada em $\cite{banyaga_morse_homology}$.
	
	\begin{teorema}\label{teorema_variedade_instavel_estavel}
		(Teorema da variedade instável/estável) Sejam $f \in \funcoesmorse{M}$ e $p \in \pontoscriticos{f}$. Então temos a decomposição $\espacotangentevariedade=\espacotangentevariedadeinstavel\oplus\espacotangentevariedadeestavel$. Além disso, existem mergulhos suaves e sobrejetores $\espacotangentevariedadeinstavel \hookrightarrow \variedadeinstavel{p} \subseteq M$ e $\espacotangentevariedadeestavel \hookrightarrow \variedadeestavel{p} \subseteq M$. Com isso, $\variedadeinstavel{p}$ e $\variedadeestavel{p}$ são subvariedades sem bordo com dimensão $\lambda_{p}$ e $n-\lambda_{p}$, respectivamente.
	\end{teorema}
	
	\begin{observacao}
		Do mergulho dado pelo teorema anterior temos, que as variedades instável e estável possuem o mesmo tipo de homotopia de um disco aberto cujas dimensões são $\lambda_{p}$ e $n-\lambda_{p}$, respectivamente.
	\end{observacao}

	Podemos decompor a variedade $M$ como a união disjunta das variedades instáveis ou união disjunta da variedades estáveis, o que é garantido pelo resultado a seguir.
	
	\begin{proposicao}\label{proposicao_uniao_variedade_instavel_estavel}
		Se $f \in \funcoesmorse{M}$, então
		$$
		M = \dot{\bigcup_{p \in \pontoscriticos{f}}}\variedadeestavel{p} = \dot{\bigcup_{p \in \pontoscriticos{f}}}\variedadeinstavel{p}.
		$$
	\end{proposicao}

	 O propósito das definições desse ponto em diante é a construção de um complexo de cadeia associado ao fluxo gradiente de uma função de Morse com uma propriedade adicional, que chamaremos de propriedade de transversalidade entre as variedades instáveis e estáveis.
	 
	 \begin{definicao}
	 	(Variedade Conectante e o Espaço Moduli) Sejam $f \in \funcoesmorse{M}$ e $p,q \in \pontoscriticos{f}$. A variedade conectante de $p$ e $q$ é definida por $\variedadeconectantepontos{p}{q} = \variedadeinstavel{p}\cap \variedadeestavel{q}$. Se $a = f(q)$, $b = f(p)$ e tomando $c \in (a, b) \subset \reta$ um valor regular de $f$, o espaço moduli de $p$ e $q$ é definido por $\espacomoduli{p}{q} = \variedadeconectantepontos{p}{q}\cap f^{-1}(c)$.
	 \end{definicao}
	 
	 Sejam $f: N\to M$ e $g: Z \to M$ duas aplicações suaves entre variedades diferenciáveis. Dizemos que $f$ é transversal a $g$ e denotemos $f \pitchfork g$ sempre que $f(x) = g(z) = y$ tem-se $df_{x}(\espacotangenteponto{x}{N}) + dg_{z}(\espacotangenteponto{z}{Z}) = \espacotangenteponto{y}{M} $. Se $Z \subseteq M$ e $g$ é a aplicação de inclusão, então denotaremos a transversalidade por $f\pitchfork Z$. Trataremos constatemente os casos em que $N, Z \subseteq M$ e $f$ e $g$ são as inclusões e denotaremos a transversalidade por $N \pitchfork Z$.
	 
	 \begin{definicao}
	 	(Funções de Morse-Smale) Dizemos que o gradiente negativo de $f \in \funcoesmorse{M}$ satisfaz a condição de Morse-Smale se $\variedadeinstavel{p}\pitchfork \variedadeestavel{q}$ para todos $p,q \in \pontoscriticos{f}$. O conjuntos dessas funções é denotado por $\funcoesmorsesmale{M}$.
	 \end{definicao}
	 
	 \begin{teorema}
	 	Sejam $f\in \funcoesmorsesmale{M}$ e $p,q \in \pontoscriticos{f}$. Então a variedade conectante $\variedadeconectantepontos{p}{q}$ e o espaço moduli $\espacomoduli{p}{q}$ são vazios ou subvariedades de $M$ sem bordo cujas dimensões são $dim(\variedadeconectantepontos{p}{q}) = \lambda_{p} -\lambda_{q}$ e $dim(\espacomoduli{p}{q}) = \lambda_{p} -\lambda_{q}-1$. 
	 \end{teorema}
	 
	 \begin{proposicao}
	 	Sejam $f \in \funcoesmorsesmale{M}$ e $p,q \in \pontoscriticos{f}$.
	 	\begin{enumerate}
	 		\item Se $\lambda_{p}<\lambda_{q}$, então $\variedadeconectantepontos{p}{q} = \emptyset$,
	 		
	 		\item $\variedadeconectantepontos{p}{p} = \{p\}$
	 		
	 		\item Se $\lambda_{p} = \lambda_{q}$ e $p\neq q$, então $\variedadeconectantepontos{p}{q} = \emptyset$,
	 		
	 		\item Se $p \neq q$ tal que $\variedadeconectantepontos{p}{q} \neq \emptyset$, então $\lambda_{p}>\lambda_{q}$.
	 	\end{enumerate}
	 \end{proposicao}

	\begin{observacao}
		Uma das consequências da proposição anterior é o fato de que, dada uma função de Morse-Smale, as óbitas não-singulares do fluxo do gradiente negativo dessa função sempre partem de um ponto crítico para um outro ponto crítico de índice inferior.
	\end{observacao}
	
	\begin{proposicao}
		Sejam $f \in \funcoesmorsesmale{M}$ e $p,q \in \pontoscriticos{f}$ de índice relativo 1, isto é, $\lambda_{p} - \lambda_{q} = 1$. Então $\overline{\variedadeconectantepontos{p}{q}} = \variedadeconectantepontos{p}{q} \cup \{p,q\}$. Além disso, o número de órbitas conectando $p$ a $q$ é finito.
	\end{proposicao}

	\section{Complexo de Morse-Smale-Witten}
	
	Sejam $V$ um n-espaço vetorial, $B=\{e_{j}\}_{j=1}^{n}$ e $B'=\{e'_{j}\}_{j=1}^{n}$ duas bases ordenadas de $V$. Dizemos que $B$ e $B'$ possuem a mesma orientação se o determinante da matriz de mudança de base, $A: B \to B'$, definida por $e_{j} = \sum_{i=1}^{n}A_{ji}e'_{i}$, possui determinante positivo. 
	
	Uma orientação em um n-espaço vetorial $V$ é uma classe de equivalência entre bases ordenadas de $V$. Um espaço vetorial munido de um orientação é um espaço vetorial orientado. Tal orientação pode ser positiva ou negativa, de acordo com o determinante da matriz de mudança de base for positiva ou negativa.
	
	Sejam $M$ uma n-variedade diferenciável. A n-upla ordenada $B=\{e_{j}\}_{j=1}^{n}$ é um referencial local se $B(p)=\{e_{j}(p)\}_{j=1}^{n}$ é uma base para $\espacotangenteponto{p}{M}$ ordenada para todo $p \in M$. Dizemos que $B$ é um referencial positivamente orientado se $B(p)$ é positivamente orientado para todo $p \in M$. Seja $A: B \to B'$ uma aplicação onde $A(p)$ é a matriz de mudança de base $B(p) $ para $B'(p)$. Se $det(A):M \to \reta$ for uma aplicação contínua tal que $det(A)(p) = det(A(p))>0$ para todo $p \in M$, então dizemos que $B$ possui orientação continuamente positiva. Analogamente, se tivermos $det(A)(p)<0$ diremos que $B$ possui uma orientação continuamente negativa.
	
	\begin{definicao}
		(Variedade orientável) Seja $M$ uma n-variedade diferenciável com um referencial $B$ continuamente orientado. Então a classe de equivalência desses referenciais $o(M)$ é chamada de orientação de $M$ e dizemos que a variedade é orientável se existe uma orientção $o(M)$.
	\end{definicao}
	
	A questão de orientação de variedades e subvariedades é crucial no processo de construção do complexo de Morse-Smale-Witten.
	
	\begin{teorema}\label{teorema_orientacao_variedade_instavel}
		Sejam $f \in \funcoesmorsesmale{M}$. Fixando as orientações $o(\variedadeinstavel{p})$ para todo $p \in \pontoscriticos{f}$ tal que $\lambda_{p}>0$, então para todos $p,q \in \pontoscriticos{f}$ temos que $\variedadeconectantepontos{p}{q}$ e $\espacomoduli{p}{q}$ são variedades com orientações $o(\variedadeconectantepontos{p}{q})$ e $o(\espacomoduli{p}{q})$ induzidas pela orientação de $\variedadeinstavel{p}$.
	\end{teorema}
	
	\begin{observacao}
		Note que no teorema anterior não foi necessária a orientabilidade de $M$, mas apenas das variedades instáveis dos pontos críticos. O procedimento para a construção da orientação induzida da variedade conectante é:
		\begin{enumerate}
			\item Para cada $p\in \pontoscriticos{f}$ com $\lambda_{p}>0$ fixa-se uma orientação $o(\variedadeinstavel{p})$
			
			\item Considere o subespaço tangente $\espacotangenteponto{p}{\variedadeestavel{p}}$ com a orientação compatível com $\espacotangenteponto{p}{\variedadeinstavel{p}}$
			
			\item Como $\variedadeconectantepontos{p}{q} = \variedadeinstavel{p}\cap\variedadeestavel{q}$ e temos a transversalidade $\variedadeinstavel{p}\pitchfork\variedadeestavel{q}$, então a orientação $o(\variedadeconectantepontos{p}{q})$ é determinada pelo isomorfismo $\espacotangenteponto{x}{\variedadeinstavel{p}}\cong \espacotangenteponto{x}{\variedadeconectantepontos{p}{q}}\oplus \espacotangenteponto{x}{\variedadeestavel{q}}$, onde $x \in \variedadeconectantepontos{p}{q}$.
		\end{enumerate}
	\end{observacao}

	Sejam $M$ uma n-variedade orientável, $f \in \funcoesmorsesmale{M}$, $p,q\in \pontoscriticos{f}$ tais que $\lambda_{p}-\lambda_{q} = 1$ e $\gamma :\reta \to M$ a curva integral do negativo do gradiente com as condições de contorno a seguir
	$$
	\derivada{}{t}\gamma(t) = -\gradiente(\gamma(t)), \; \lim_{t \to -\infty}\gamma(t) = p\;\;\text{e}\; \lim_{t \to \infty}\gamma(t) = q.
	$$
	Tome um ponto $x \in \gamma(\reta) \subset \variedadeconectantepontos{p}{q}$. Como a variedade instável $\variedadeinstavel{p}$ é orientável, podemos fixar $o(\variedadeinstavel{p})$ como sendo sua orientação. Então existe um referencial $B^{u}$ nessa variedade tal que, juntamente com o campo $-\gradiente$, tem-se que $B(x) = \{-\gradiente(x), B^{u}(x)\}$ é uma base de $\espacotangenteponto{x}{\variedadeinstavel{p}}$.
	
	\begin{definicao}
		(Número de intersecção) Com as hipóteses descritas anteriormente, o sinal característico da órbita $\orbitaponto{x}$ é o índice $n_{x} \in \{\pm 1\}$ que assume os valores $1$, caso a orientação $o(B(x))$ seja equivalente a $o(\variedadeinstavel{p})$, e assume $-1$ caso contrário. O número de intersecção é definido por 
		$$
		n(p,q) = \sum_{x \in \espacomoduli{p}{q} }n_{x}.
		$$
		
		Desse modo, o número de intersecção conta as órbitas entre p e q considerando a orientação.
	\end{definicao}
	
	\begin{definicao}
		(Complexo de Morse-Smale-Witten) Seja $f \in \funcoesmorsesmale{M}$. Para cada $p \in \pontoscriticos{f}$ assuma uma orientação $o(\variedadeinstavel{p})$. Sejam $C_{k}(f)$ o grupo abeliano livremente gerado pelos pontos críticos de índice $k$ e $C_{*}(f) =\bigoplus^{m}_{k=0}C_{k}(f)$. O homomorfismo $\partial_{k}: C_{k}(f)\to C_{k-1}(f)$ definido em cada gerador p de $C_{k}$
		$$
		\partial_{k}(p)=\sum_{q \in \pontoscriticos{f}}n(p,q)q
		$$
		é chamado operador bordo de Morse-Smale-Witten e o par $(C_{*}(f), \partial_{*})$ é o complexo de cadeia de Morse-Smale-Witten  da função $f$.
		
	\end{definicao}
	
	\begin{exemplo}
		(Complexo de Morse-Witten) Considere variedade 2-dimensional $M$ mergulhada em $\real{3}$ e representada pela Figura \ref{figura_fluxo_morse_smale}. Seja $F:\real{3} \to \reta$ a função definida por $F(x, y, z) = z$. Pode-se mostrar que $f=F|_{M} \in \funcoesmorsesmale{M}$,  $\pontoscriticos{f} = \{p,q,r,s\}$ e que os índices de Morse dos pontos críticos de $f$ são $\lambda_{q} = 0$, $\lambda_{p} = 1$ e $\lambda_{r}=\lambda_{s} = 2$. 
		
		\vermelho{(Devo inserir uma figura ilustrando as variedades instável/estável para facilitar a identificação de tais conjuntos.)}
		
		Definindo $A$ como sendo a imagem do caminho conectando os ponto críticos $r$ e $s$, temos que $\variedadeestavel{q} = \complementar{M}{A}$. Além disso, $\variedadeestavel{p} = A$ e $\variedadeestavel{r} = \variedadeestavel{s} = \emptyset$. Logo, $M = \variedadeestavel{p}\dot{\cup} \variedadeestavel{q}$, o que está de acordo com a Proposição $\ref{proposicao_uniao_variedade_instavel_estavel}$. Seja $B$ a imagem dos caminhos conectando $p$ a $q$, então $\variedadeinstavel{p} = B$. Com isso, temos $\variedadeconectantepontos{p}{q} = B$ e $\espacomoduli{p}{q}=\{x_{1}, x_{2}\}$. 
		
		Seja $D$ a imagem do caminho conectando $r$ a $p$. Temos que $\variedadeinstavel{r}$ é a região da figura colorida em azul. $\variedadeconectantepontos{r}{p} = \variedadeinstavel{r} \cap A = D$. Com isso, $\espacomoduli{r}{p} = \{ x_{3}\}$. Com um procedimento análogo concluimos que $\espacomoduli{s}{p} = \{x_{4} \}$.
		
		Pela orientação escolhida para a variedade instável $\variedadeinstavel{p}$, temos os sinais característicos $n_{x_{1}}=1$, $n_{x_{2}} = -1$. E a orientação escolhida para $\variedadeinstavel{r}$ e $\variedadeinstavel{s}$ nos dá $n_{x_{3}} =n_{x_{4}}= 1$. Portanto 
		$$
		n(p,q) = n_{x_{1}}+n_{x_{2}} = 0,\; n(r,p) = n(s,p)=n_{x_{3}}= 1. 
		$$
		
		As $k$-cadeias do complexo, para $0\leq k\leq 2$, são
		$$
		C_{0}(f) = \inteiros\gerador{q},\; C_{1}(f) = \inteiros\gerador{p},\; C_{2}(f) = \inteiros\gerador{r}\oplus\inteiros\gerador{s}.
		$$
		E os $k$-operadores bordo são
		$$
		\bordo{0}\gerador{q} =0,\; \bordo{1}\gerador{p}=n(p,q)\gerador{q} = 0,\; \bordo{2}\gerador{r}=\bordo{2}\gerador{s} = n(r,p)\gerador{p} = \gerador{p}.
		$$
		
		\begin{figure}
			\centering
			\includegraphics[width=5cm,height=3cm]{imagem/fluxoesfera}
			\caption{Linhas do fluxo de $-\gradiente$.}
			\label{figura_fluxo_morse_smale}
		\end{figure}
	\end{exemplo}

	
	O seguinte teorema é de grande importância pois afirma o isomorfismo entre a homologia do complexo de Morse-Smale-Witten e a homologia singular da variedade. Sua demonstração pode ser encontrada em $\cite{banyaga_morse_homology}$.
	
	\begin{teorema}
		(Teorema da Homologia de Morse) Sejam $f \in \funcoesmorsesmale{M}$ e o par $(C_{*}(f), \partial_{*})$ o complexo de cadeia de Morse-Smale-Witten da função $f$. Então $H_{*}((C_{*}(f), \partial_{*})) $ é isomorfo a homologia singular $ H_{*}(M, \inteiros)$.
	\end{teorema}
	
	O Teorema $\ref{teorema_cw_homologia}$ afirma que a homologia de um $CW$-complexo é isomorfa a sua homologia singular e também sabemos que a homologia de Morse é isomorfa a homologia singular da variedade. Vimos na Seção $\ref{secao_morse_classica}$ que uma n-variedade fechada possui o mesmo tipo de homotopia que de um $CW$-complexo construido a partir dos pontos críticos de uma função de Morse, logo a homologia singular dessa variedade é isomorfa a homologia do $CW$-complexo associado. Com isso, podemos afirmar que as homologias construidas via funções de Morse-Smale e funções de Morse clássicas são isomorfas.
	
	O cálculo da homologia singular de uma variedade com certas propriedades nem sempre é algo simples a se fazer. Contudo, podemos aplicar técnicas alternativas para a determinação de homologias isomorfas a homologia singular. Essa é importância dos resultados dos teoremas de isomorfismos anteriormente citados. Um outro ponto a ser comentado é o fato de que as homologias singulares são utilizadas para a descrição de alguns invariantes topológicos, por exemplo, a característica de Euler-Poicaré. Portanto, a demonstração desses isomorfismos garante que ainda estamos diantes dos mesmo invariantes topológicos.
	
	
	\chapter{Espaços Vetoriais Simpléticos}
	
	\section{Origem na Física}
	
	A mecânica clássica visa o estudo da dinâmica de sistemas físicos, conservativos ou não. Diz-se que um campo vetorial $F:\real{3} \to \real{3}$ é conservativo quando, dados dois caminhos $\gamma,\beta:[0,1] \to \real{3}$ de classe $C^{2}$ tais que $\gamma(0)=\beta(0)$ e $\gamma(1)=\beta(1)$ (com extremos fixos), temos
	$$
	\tau=\int_{\intervalo} \produtointerno{F(\gamma(t))}{\gamma'(t)}=
	\int_{\intervalo} \produtointerno{F(\beta(t))}{\beta'(t)}.
	$$
	A grandeza $\tau$ é definida como sendo o trabalho realizado pelo campo $F$ ao longo do caminho $\gamma$. Logo, se o campo é conservativo, então o trabalho independe do caminho.
	
	Parte dos sistemas físicos conhecidos podem ser descritos via mecânica Newtoniana, cuja dinâmica é regida pela equação diferencial 
	$$
	F(t) = m\derivada{v(t)}{t} = \derivada{p(t)}{t},
	$$
	onde $v = \gamma'(t)$, $m\geq0$ constante e $p(t) = m\gamma'(t)$, este último chamado de momento linear. Suponha agora que exista uma função, chamada de energia potencial, de classe $C^{2}$ tal que $U:\real{3}\to \reta$ e $F = -\nabla U$. Tomando $q=(q_{1},q_{2}, q_{3})\in \real{3}$, então $\gamma(t)=(q_{1}(t),q_{2}(t), q_{3}(t)) = q(t)$ e podemos escrever $F(q(t)) =m \ddot{q}(t)= -\nabla U(q(t))$. Temos um sistema de 3 equações de segunda ordem, contudo, realizando uma mudança de variáveis, podemos reduzir o problema de segunda ordem para um problema de primeira ordem, do seguinte modo
	$$
	\dot{q} = \frac{p}{m} \;\; e \;\;\dot{p} = -\nabla U(q).
	$$
	
	O conjunto $Q$ das posições $q$ do sistema anterior é chamado espaço de configurações e o conjunto $P=\{(q,p): q,p\in \real{3}\}$ dos pares posição e momento linear é chamado de espaço de fases. No exemplo temos $Q=\real{3}$ e $P = \real{6}$, porém ambos podemo ser outras variedades diferenciáveis.
	
	Uma função Hamiltoniana é uma função $H:P \to \reta$ de classe $C^{1}$ tal que 
	$$
	\dot{q} = \derivadaparcial{H}{p} \;\; e \;\; \dot{p} = -\derivadaparcial{H}{q}.
	$$
	
	O sistema de equações acima é chamado de equações de Hamilton. Sabe-se que a energia total de um sistema físico conservativo é a soma das energias pontecial $E_{p} = U$ e cinética $E_{c} = p^{2}/2m$ (veja $\cite{nussenzveig}$), onde $p^{2} = \produtointerno{p}{p}$. Definamos a função Hamiltoniana associada a energia total do sistema por
	$$
	H(q,p) = E_{c} +E_{p} = \frac{p^{2}}{2m}+U(q). 
	$$
	
	Note que a função $H$ definida acima satisfaz as equações de Hamilton, logo é uma função Hamiltoniana e, por estar associada a energia total do sistema, é amplamente aplicada em modelos dinâmicos físicos.
	
	As equações de Hamilton nos dizem que, dada uma função Hamiltoniana, podemos recuperar as equações de Newton. Portanto, temos uma compatibilidade entre ambas as descrições de um problema físico.
	
	\section{Geometrização}
	
	A geometrização de problemas em física nos permite estudar os resultados obtidos sobre outro ponto de vista, e com isso, generalizar e analisar novos aspectos. Faremos o mesmo com a descrição Hamiltoniana, de modo que definiremos objetos utilizados adiante na análise da topologia de uma categoria específica de variedades diferenciáveis.
	
	Sejam $Q= \real{n}$ e $P=\real{2n}$ os espaços de configurações e fase, respectivamente. Por simplicidade vamos utilizar a notação $(q, p ) = (q_{1}, \dots ,q_{n}, p_{1}, \dots ,p_{n}) \in \real{2n}$. Desse modo,  temos que $B(q_{0}, p_{0})=\{\partial_{q_{1}}, \dots, \partial_{q_{n}}, \partial_{p_{1}}, \dots, \partial_{p_{n}}\}= \{\partial_{q}, \partial_{p}\} $ é uma base do espaço tangente $T_{(q_{0},p_{0})} P $, onde $\partial_{q_{j}} = \partial/\partial_{q_{j}}$ e $\partial_{p_{j}} = \partial/\partial_{p_{j}}$ avaliados no ponto $(q_{0}, p_{0})$. 
	
	Seja $H:P \to \reta$ uma função Hamiltoniana de classe $C^{\infty}$. O gradiente Hamiltoniano é dado por
	$$
	\nabla H =\sum_{j=1}^{n} \bigparenteses{\derivadaparcial{H}{q_{j}}\derivadaparcial{}{q_{j}} + \derivadaparcial{H}{p_{j}}\derivadaparcial{}{p_{j}} }= \derivadaparcial{H}{q}\derivadaparcialabrev{q} + \derivadaparcial{H}{p}\derivadaparcialabrev{p}.
	$$
	
	Definimos o campo Hamiltoniano $\campohamiltonianoabrev \in \campossuaves{\real{2n}}$ por 
	$$
	\campohamiltonianoabrev = -\estruturacomplexa \nabla H = \sum_{j=1}^{n}\bigparenteses{\derivadaparcial{H}{p_{j}}\derivadaparcial{}{q_{j}} - \derivadaparcial{H}{q_{j}}\derivadaparcial{}{p_{j}} } = \derivadaparcial{H}{p}\derivadaparcialabrev{q} - \derivadaparcial{H}{q}\derivadaparcialabrev{p}, 
	$$
	onde $\estruturacomplexa$ é a matriz $2n \times 2n$ dada por
	$$
	\estruturacomplexa=
	\left(
	\begin{array}{cc}
	0 & -Id_{n\times n}
	\\
	Id_{n\times n} & 0
	\end{array}
	\right), 
	$$
	e $Id_{n\times n}$ a matriz identidade $n\times n$. Com isso, $\estruturacomplexa \derivadaparcialabrev{q} = \derivadaparcialabrev{p}$ e $\estruturacomplexa \derivadaparcialabrev{p} = -\derivadaparcialabrev{q}$.
	
	Vejamos que o campo Hamiltoniano representa um campo conservativo. Seja $\psi:\reta \to P$ a curva integral do campo $\campohamiltonianoabrev$. Então as equações de Hamilton podem ser reescritas como 
	$$
	\begin{aligned}
	\dot{\psi}(t) &= \campohamiltoniano{\psi(t)}
	\\
	\left(
	\begin{array}{c}
	\dot{q}(t)
	\\
	\dot{p}(t)
	\end{array}
	\right)
	&=
	\left(
	\begin{array}{c}
	\derivadaparcial{H(t)}{p}
	\\
	-\derivadaparcial{H(t)}{q}
	\end{array}
	\right).
	\end{aligned}
	$$	
	
	Afirmo que, ao longo das soluções dos sistema Hamiltoniano, o sistema é conservativo, isto é, a energia total do sistema é constante. De fato
	
	$$
	\begin{aligned}
	\derivada{}{t}H(\psi(t)) 
	&= \produtointerno{\nabla H}{\dot{\psi}(t)} 
	\\
	&= \produtointerno{\nabla H}{\campohamiltoniano{\psi(t)}} 
	\\
	&= 
	\produtointerno{\nabla H}{-\estruturacomplexa \nabla H(t)} 
	\\
	&=\produtointerno{\derivadaparcial{H}{q}\derivadaparcialabrev{q} + \derivadaparcial{H}{p}\derivadaparcialabrev{p}
	}{\derivadaparcial{H}{p}\derivadaparcialabrev{q} - \derivadaparcial{H}{q}\derivadaparcialabrev{p}}
	\\
	&=\derivadaparcial{H}{p}\derivadaparcial{H}{q}\produtointerno{\derivadaparcialabrev{q}}{\derivadaparcialabrev{q}}-\bigparenteses{\derivadaparcial{H}{p}}^{2}\produtointerno{\derivadaparcialabrev{q}
	}{\derivadaparcialabrev{p}}	
	+\bigparenteses{\derivadaparcial{H}{q}}^{2}\produtointerno{\derivadaparcialabrev{q}}{\derivadaparcialabrev{p}}
	-\derivadaparcial{H}{p}\derivadaparcial{H}{q}\produtointerno{\derivadaparcialabrev{p}}{\derivadaparcialabrev{p}}
	\\
	&=0.
	\end{aligned}
	$$
	
	Tome a 2-forma $\formaSimpleticaabrev :T_{(q,p)} P \times T_{(q,p)} P \to \reta$ definida por $\formaSimpleticaabrev = \sum_{j}  dq_{j}\wedge dp_{j}$. Com isso, temos a geometrização das equações de Hamilton
	
	$$
	\begin{aligned}
	\formaSimpleticaPadrao{\campohamiltonianoabrev}{v} 
	&= \sum_{j}  dq_{j}\wedge dp_{j}(\campohamiltonianoabrev, v) 
	\\
	&= \sum_{j}  dq_{j}(\campohamiltonianoabrev)dp_{j}(v) - dq_{j}(v)dp_{j}(\campohamiltonianoabrev)
	\\
	&= \sum_{j} \derivadaparcial{H}{p_{j}}dp_{j}(v) + dq_{j}(v)\derivadaparcial{H}{q_{j}}
	\\
	&= \Big(\sum_{j} \derivadaparcial{H}{q_{j}}dq_{j} +\derivadaparcial{H}{p_{j}}dp_{j} \Big)(v)
	\\
	&= dH(v).
	\end{aligned}
	$$
	
	Definimos um espaço de fase $P$ e em cada ponto do espaço tangente de $P$ temos uma forma bilinear anti-simétrica e não-degenerada $\formaSimpleticaabrev$, chamada forma simplética. Com isso, o par $(T_{(q,p)}P, \formaSimpleticaabrev)$ é chamado 2n-espaço vetorial simplético, onde $dim(T_{(q,p)}P) = 2n$. A seção seguinte é dedicada ao estudo de espaços vetoriais simpléticos reais.
	
	\section{Espaços Vetoriais Simpléticos}
	\begin{definicao}
		(Espaço vetorial simplético) Sejam $V$ um 2n-espaço vetorial real e uma forma bilinear anti-simétrica $\omega$ em $\Lambda^{2}(V, \real{})$ tal que $\omega(u,v) = 0 \; \forall v \in V \Rightarrow u=0$. Então dizemos que $\omega$ é não-degenerada e o par $(V, \omega)$ é chamado de 2n-espaço vetorial simplético.
	\end{definicao}
	
	\begin{definicao}
		(Base simplética) Seja $(V, \omega)$ um 2n-espaço vetorial simplético, então uma base simplética é uma base $\{ e_{1},\dots, e_{n},f_{1},\dots f_{n}\}$ de $V$ tal que valem as relações:
		$$
		\omega(e_{i}, e_{j}) = \omega(f_{i}, f_{j}) = 0, \; \omega(e_{i}, f_{j}) = \delta_{ij}.
		$$
	\end{definicao}
	
	Sejam $(V_{1}, \omega_{1}), (V_{2}, \omega_{2})$ dois espaços vetoriais simpléticos e uma aplicação linear $\varphi: V_{1}\to V_{2}$. O pullback de $\omega_{2}$ por $\varphi$ é a 2-forma $\varphi^{*}\omega_{2}:V_{1} \times V_{1} \to \reta$ definida por $\varphi^{*}\omega_{2}(v,u) = \omega_{2}(\varphi(v), \varphi(u))$.
	
	\begin{convensao}\label{convensao_base_simpletica}
		Quando não houver ambiguidades, denotaremos a base simplética  $\{ e_{1},\dots, e_{n},f_{1},\dots f_{n}\}$  de $(V, \omega)$ por  $\{ e ,f\}$, e um dado $v =\sum_{j=1}^{n}( v_{j}e_{j} +v_{n+j}f_{j}) \in V$ por $v=v_{(1)}e+v_{(2)}f$, onde $v_{(1)}=(v_{1}, \dots, v_{n})$ e $v_{(2)}=(v_{n+1}, \dots, v_{2n})$.
	\end{convensao}
	
	\begin{observacao}\label{observacao_convensao_base_simpletica}
		Dados $v=v_{(1)}e+v_{(2)}f$ e $u=u_{(1)}e+u_{(2)}f$ em $V$, tem-se que 
		$$
		\formaSimpletica{v}{u} = \produtointerno{v_{(1)}}{u_{(2)}}-\produtointerno{v_{(2)}}{u_{(1)}}, 
		$$
		onde $\produtointerno{v_{(1)}}{u_{(2)}}=\sum_{j=1}^{n}v_{j}u_{n+j}$. O mesmo vale para $\produtointerno{v_{(2)}}{u_{(1)}}$.
	\end{observacao}
	
	\begin{definicao}
		(Simplectomorfismo) Dois espaços vetoriais simpléticos $(V_{1}, \omega_{1}), (V_{2}, \omega_{2})$ são ditos simplectomorfos se existe um isomorfismo $\varphi:V\to W$ que preserva a forma simplética, isto é, $\varphi^{*}\omega_{2} = \omega_{1}$.
	\end{definicao}
	\begin{exemplo}\label{exemplo_espaco_simpletico_real}
		Seja $V = \real{2}$, $\{e_{x}, e_{y}\}$ uma base de $V$ e $w=dx \wedge dy$. Então $\omega(e_{x}, e_{y}) = (dx \wedge dy)(e_{x}, e_{y}) = dx\otimes dy(e_{x}, e_{y})-dy\otimes dx(e_{x}, e_{y}) =dx(e_{x}) dy(e_{y}) - dx(e_{y}) dy(e_{x}) = 1-0= 1$. Por outro lado, $\omega(e_{y}, e_{x}) =dx(e_{y}) dy(e_{x}) - dx(e_{x}) dy(e_{y}) =-1 =-\omega(e_{x}, e_{y})$, logo é anti-simetrica. Além disso, $\omega(e_{x}, e_{x}) = \omega(e_{y}, e_{y}) = 0$. Fixando $v \in V$ e para qualquer $u \in V$ temos que $\omega(v, u) = \omega(v_{x}e_{x}+v_{y}e_{y}, u_{x}e_{x}+u_{y}e_{y}) = v_{x}u_{y}\omega(e_{x}, e_{y}) +v_{y}u_{x}\omega(e_{y}, e_{x}) = v_{x}u_{y} -v_{y}u_{x} = 0$ se, e somente se, $v_{x}=v_{y}=0$, logo $\omega$ é não-degenerada. Para ver isso basta tomar $u_{x} = 1$ e $u_{y} = 0$, isso implica que $v_{y} = 0$. Fazendo $u_{x} = 0$ e $u_{y} = 1$ teremos $v_{x} = 0$, logo $v=0$. Seja $\varphi:V \to V$ tal que $\varphi(v) = -v$. É imediato que $\varphi$ é um isomorfismo. Então $\varphi^{*}\omega(v, u) = \omega(\varphi v, \varphi u)=\omega(-v, -u)=\omega(v, u)$, logo $\varphi^{*}\omega = \omega$ e $\varphi$ é um simplectomorfismo.
	\end{exemplo}
	
	\begin{definicao}\label{definicao_subespaco_simpletico_ortogonais}
		(Espaços $\omega$-ortogonais) Seja $(V, \omega)$ um 2n-espaço vetorial simplético e $W\subseteq V$ um subespaço vetorial simplético. Então o complemento $\omega$-ortogonal de $W$ é o subespaço vetorial simplético
		$$
		W^{\omega} = \{v\in V: \omega(v,u) = 0,\;\forall u\in W \}.
		$$
		Além disso, $W$ pode ser classificado de acordo com as seguintes características
		\begin{enumerate}
			\item \text{Simplético}, se $W\cap \espacoSimpleticoOrtogonal{W} = \{0\}$;
			
			\item \text{Isotrópico}, se $W \subseteq \espacoSimpleticoOrtogonal{W}$;
			
			\item \text{Coisototrópico}, se $W\supseteq \espacoSimpleticoOrtogonal{W}$;
			
			\item \text{Lagrangiano}, se $W =\espacoSimpleticoOrtogonal{W}$.
		\end{enumerate}
	\end{definicao}
	
	\begin{lema}\label{lema_subespaco_simpletico_ortogonal}
		(Caracterização de subespaço simplético)
		\begin{enumerate}
			\item Se $W$ for simplético, então $(W, \omega|_{W \times W})$ é um espaço vetorial simplético.
			
			\item Se $W$ for isotrópico, então $\omega|_{W\times W} = 0$.
			
			\item Se $W$ for lagrangiano, então $W$ é isotrópico e máximal, isto é, $W$ não está contido propriamente em nenhum outro subespaço isotrópico. 
		\end{enumerate}
	\end{lema}
	\begin{prova}
		\begin{enumerate}
			\item Supondo que $\omega|_{W \times W}$ seja degenerada, então existe $0\neq v \in W$ tal que $\omega(v, u ) = 0$ para todo $u \in W$, o que implica que $u \in W\cap W^{\omega}$, contradizendo a hipótese $W\cap W^{\omega} =0$. Logo $\omega|_{W \times W} $ é não-degenerada e $(W, \omega|_{W \times W})$ é um espaço vetorial simplético.
			
			\item Dado $0\neq v \in W $ , então $\omega(v,u) = 0$ para todo $u \in W\cap W^{\omega} = W$. Portanto $\omega|_{W\times W}: W\times W \to \reta$ é a aplicação nula.
			
			\item  Como $W=W^{\omega}$, então $W\subseteq W^{\omega}$, logo $W$ é isotrópico. Seja $U \subseteq V$ um subespaço isotrópico tal que $W \subseteq U$. Então, pelo item anterior, $\omega|_{W \times W} = 0$, logo dado $v \in W \cap U$ e para todo $u \in U$, temos $\omega(v, u) = 0$, implicando $u \in W^{\omega}$. Portanto, $U = W^{\omega} = W$.
		\end{enumerate}
	\end{prova}
	
	Não há garantias de que sempre tenhamos $V = W + \espacoSimpleticoOrtogonal{W}$. Contudo, a proposição a seguir nos dá uma relação entre as dimensões desses dois espaços vetoriais.
	
	\begin{lema}\label{lema_isomorfismo_forma_simpletica}
		Seja $(V,\omega)$ um 2n-espaço vetorial simplético. A aplicação $\omega^{*}:V\to V^{*}$ definida por $\omega^{*}(v)(u) = \omega(v,u)$ é um isomorfismo.
	\end{lema}
	\begin{prova}
		Dado $v \in V$, temos pela bilinearidade de $\omega$ que $\omega^{*}(v)$ é linear, logo $\omega^{*}(v) \in V^{*}$. Além disso, $w^{*}$ é injetora pois, suponha 
		que exista $v\neq 0 \in V$ tal que $\omega^{*}(v) = 0$. Com isso, temos $0= \omega^{*}(v)(u) = \omega(v,u)$ para todo $u \in V$, o que é contradição, pois $\omega$ é não-degenerada, logo $v =0$ e $\omega^{*}(0) = 0$. Portanto $\omega^{*}$ é injetora. Seja $\colecaofinita{e}{2n}$ uma base de $V$. Denotando $b^{*}_{j}$ por $\omega^{*}(e_{j}) \in V^{*}$, temos pela injetividade que $b^{*}_{j} \neq b^{*}_{i}$ para todos $1\leq j,i\leq 2n$. Suponha que $\colecaofinita{b^{*}}{2n}$ seja linearmente dependente. Com isso, existem $\colecaofinita{\lambda}{2n} \subset \reta$ nem todos nulos tais que $0=\sum_{j=1}^{2n}\lambda_{j}b^{*}_{j} = \sum_{j=1}^{2n}\lambda_{j}\omega^{*}(e_{j}) = \omega^{*}(\sum_{j=1}^{2n}\lambda_{j}e_{j})$. Pela injetividade de $\omega^{*}$ devemos ter que $\sum_{j=1}^{2n}\lambda_{j}e_{j} = 0$, contradizendo a hipótese de que $\colecaofinita{e}{2n}$ é uma base de $V$. Portanto, $\colecaofinita{b^{*}}{2n}$ é uma base de $V^{*}$ e $\omega^{*}$ é sobrejetor, logo é um isomorfismo.
	\end{prova}
	
	\begin{proposicao}\label{proposicao_dimensao_subespaco_simpletico}
		Sejam $(V,\omega)$ um 2-espaço vetorial simplético e $W \subseteq V$ um subespaço vetorial. Então $dim(V) = dim(W) + dim(\espacoSimpleticoOrtogonal{W})$.
	\end{proposicao}
	\begin{prova}
		Seja $\omega^{*}: V \to V^{*}$ tal que $\omega^{*}(v)(u) = \omega(v,u)$. Pelo Lema $\ref{lema_isomorfismo_forma_simpletica}$ $\omega^{*}$ é um isomorfismo. Seja $W^{\circ}=\{f\in W^{*}: f(v) = 0,\; \forall v\in W \}$ o anulador de $W$. Tomando $f \in \omega^{*}(\espacoSimpleticoOrtogonal{W})$ temos $f(u) = \omega^{*}(v)(u)=\omega(v,u)$, para algum $v \in \espacoSimpleticoOrtogonal{W}$. Se $u\in W$, então $f(u) = 0$. Portanto $f \in W^{\circ}$ e  $\omega^{*}(\espacoSimpleticoOrtogonal{W})\subseteq W^{\circ}$. Por outro lado, seja $f \in W^{\circ}$. Como $\omega^{*}$ é um isomorfismo, então existe $v \in v$ tal que $f = \omega^{*}(v)$, logo $0=f(u) = \omega^{*}(v)(u) = \omega(v,u)$ para todo $u \in W$, o que implica que $v \in \espacoSimpleticoOrtogonal{W}$. Portanto, $f \in \omega^{*}(\espacoSimpleticoOrtogonal{W})$ e $W^{\circ} \subseteq \omega^{*}(\espacoSimpleticoOrtogonal{W})$. Logo $W^{\circ} =\omega^{*}(\espacoSimpleticoOrtogonal{W})$.
		Como $\omega^{*}$ é um isomorfismo, então $dim(\espacoSimpleticoOrtogonal{W}) = dim(\omega^{*}(\espacoSimpleticoOrtogonal{W}))$. Temos que 
		$$
		dim(V) = dim(W)+dim(W^{\circ}) = dim(W)+dim(\omega^{*}(\espacoSimpleticoOrtogonal{W})) = dim(W)+dim(\espacoSimpleticoOrtogonal{W}).
		$$ 
	\end{prova}
	
	\begin{teorema}\label{teorema_existencia_base_simpletica}
		(Existência de base simplética) Todo espaço vetorial simplético de dimensão finita possui uma base simplética.
	\end{teorema}
	\begin{prova}
		Sejam $(V, \omega)$ um 2n-espaço vetorial simplético e $e_{1} \neq 0\in V$. Como $\omega$ é não-degenerada, então existe $f_{1} \in V$ tal que $\omega(e_{1}, f_{1}) = 1$. Definindo $V_{1} = span\{e_{1}, f_{1}\}$, podemos afirmar que $V_{1}$ é simplético, ou seja, $V_{1}\cap V_{1}^{\omega} = \{0\}$. Com isso e com a Proposição $\ref{proposicao_dimensao_subespaco_simpletico}$, podemos afirmar que $V = V_{1}\oplus V_{1}^{\omega}$. Efetuando o mesmo procedimento n vezes teremos $V = V_{1}\oplus \dots \oplus V_{n}$, onde $V_{i}$ é gerado por $e_{i}$ e $\omega(e_{i}, f_{i}) = 1$. Por construção temos $\omega(e_{i}, e_{j})=\omega(f_{i}, f_{j}) =0$ e $\omega(e_{i}, f_{j}) = \delta_{ij}$. Portanto, $\{e_{1}, \dots e_{n}, f_{1}, \dots f_{n}\}$ é uma base simplética.
	\end{prova}
	
	\begin{observacao}\label{observacao_existencia_base_simpletica}
		A existência de uma base simplética garante a existência de uma base em que a forma simpléctica $\omega$ poderá ser representada pela matriz
		$$
		\left(
		\begin{array}{cc}
		0 & Id
		\\
		-Id & 0
		\end{array}
		\right).
		$$
	\end{observacao}
	
	\begin{teorema}
		Seja $V$ um 2n-espaço vetorial, então existe uma base $\{ e_{1},\dots, e_{n}, f_{1},\dots, f_{n}\}$ de $V$ e uma base $\{e_{1}^{*}, \dots, e_{n}^{*}, f_{1}^{*}, \dots,f_{n}^{*}\}$ de $V^{*}$ tal que dado $\alpha \in \Lambda^{2}(V)$ pode-se escrever $\alpha = \sum_{i=1}^{n} e^{*}_{i}\wedge f^{*}_{i}$.
	\end{teorema}
	\begin{prova}
		Seja $\alpha\neq 0\in \Lambda^{2}(V) $, então existem $ a_{1}, a_{1+n} \in V $ tais que $\alpha(a_{1}, a_{1+n}) = \alpha_{1} \neq 0$. Definindo $e_{1} = a_{1}/\alpha_{1}$ e $a_{1+n} = f_{1}$ teremos $\alpha(e_{1}, f_{1}) = 1$, e pela anti-simetria temos $\alpha(e_{1}, e_{1}) = \alpha(f_{1}, f_{1}) = 0$. Definindo $B_{1}=span \{e_{1}, f_{1}\}$, então a matriz $(\alpha_{ij})$ de $\alpha|_{B_{1}}$ é
		$$
		\left(
		\begin{array}{cc}
		0 & 1
		\\
		-1 & 0
		\end{array}
		\right)
		$$
		Seja  $V_{2} = \{v \in V: \alpha(v, b) = 0,\; b \in B_{1}\}$, então por construção temos $B_{1} \cap V_{2} = 0$. Como $B_{1}, V_{2} \subseteq V$ são subespaços vetoriais então $V = B_{1}\oplus V_{2}$. Dado $v \in V$ temos $v_{2} =v- \alpha(v,f_{1})e_{1} +\alpha(v,e_{1})f_{1} \in V_{2}$ pois $\alpha(v_{2}, e_{1}) = \alpha(v_{2}, f_{1}) = 0$. Repetindo a construção para $V_{2}$ podemos afirmar que existem $e_{2}, f_{2} \in V_{2}$ tais que $\alpha(e_{2}, f_{2}) = 1$, $\alpha(e_{2}, e_{2}) = \alpha(f_{2}, f_{2}) = 0$, $B_{2} = span\{e_{2}, f_{2} \}$ e $V_{3} \subset V_{2}$ tal que $B_{2}\cap V_{3}=0$, onde a matriz $(\alpha_{ij})$ de $\alpha|_{B_{2}}$ é da mesma forma que a matriz de $\alpha|_{B_{1}}$. Realizando uma indução finita na construção dos planos $B_{j}$ teremos $V = \bigoplus_{j=1}^{n}B_{j}$, logo a matriz de $\alpha$ na base  $\{ e_{1},\dots, e_{n}, f_{1},\dots, f_{n}\}$ é
		$$
		\left(
		\begin{array}{cc}
		0 & Id
		\\
		-Id & 0
		\end{array}
		\right)
		$$
		Definindo a base dual $\{e_{1}^{*}, \dots, e_{n}^{*}, f_{1}^{*}, \dots,f_{n}^{*}\}$ de $V^{*}$ teremos $\alpha = \sum_{j=1}^{n}e_{j}^{*}\wedge f_{j}^{*}$.
	\end{prova}
	
	\begin{lema}
		(Caracterização da forma simplética) Sejam $V$ um 2n-espaço vetorial, então $\omega \in \Lambda^{2}(V)$ é uma forma simplética se, e somente se, $\omega^{\wedge n} = \omega\wedge \dots \wedge \omega \in \Lambda^{2n}(V)$ é não-nula.
	\end{lema}
	\begin{prova}
		Suponha $\omega$ uma forma simplética. O Teorema $\ref{teorema_existencia_base_simpletica}$ garante a existência de uma base simplética $\{e_{1}, \dots, e_{n}, f_{1}, \dots, f_{n}\}$ de $V$. Pela bilinearidade de $\omega$ basta analisarmos $\omega$ nos elementos dessa base. Considerando $\sigma$ no conjunto das permutações de $\{1, 2, \dots , 2n\}$ temos
		$$		
		\begin{aligned}
		\omega^{\wedge n}(e_{1}, \dots, e_{n}, f_{1}, \dots, f_{n}) &=\sum_{\sigma} \omega(e_{1}, f_{\sigma(1)})...\omega(e_{n}, f_{\sigma(n)})
		\\
		&= \sum_{\sigma}\delta_{1\sigma(1)}\dots\delta_{n\sigma(n)}
		\\
		&= \delta_{11}\dots\delta_{nn}
		\\
		&= 1.
		\end{aligned}
		$$
		Por outro lado, se $\omega^{\wedge n} \neq 0$, suponha que $\{v_{1},\dots, v_{2n}\}$ seja uma base de $V$ e que exista $v =\sum a_{j}v_{j} \neq 0$ tal que $\omega(v, u) = 0$ para todo $u=\sum b_{j}v_{j}  \in V$. Então, $0=\omega(v, u ) = \sum_{j, k} a_{j}b_{k}\omega(v_{j}, v_{k})$, o que implica em $\omega(v_{j}, v_{k}) =0$. Então 
		$$
		\omega^{\wedge n}(v_{1},\dots, v_{2n}) = \sum_{\sigma} \omega(v_{1}, v_{\sigma(1)})...\omega(v_{2n}, v_{\sigma(2n)})=0,
		$$
		contradizendo a hipótese $\omega^{\wedge n} \neq 0$. Logo, $\omega$ é não-degenerada.
	\end{prova}
	
	\begin{proposicao}\label{proposicao_preservacao_volume}
		(Preservação do volume) Sejam $(V,\omega)$ um 2n-espaço vetorial simplético e $\varphi:V\to V$ um simplectomorfismo, então $\varphi^{*}\omega^{\wedge n}=\omega^{\wedge n}$ e $det(\varphi)=1$.
	\end{proposicao}
	\begin{prova}
		Seja $\varphi:V \to V$ um simplectomorfismo, então $\varphi^{*}\omega = \omega$, e portanto
		$$
		\begin{aligned}
		\varphi^{*}\omega^{\wedge n} 
		&= 
		\varphi^{*}(\omega\wedge \dots \wedge\omega) 
		\\
		&= \varphi^{*}\omega\wedge \dots \wedge\varphi^{*}\omega
		\\
		&=\omega\wedge \dots \wedge \omega 
		\\
		&= \omega^{\wedge n}.
		\end{aligned} 
		$$
		Aplicando esse resultado vemos que
		$$
		\begin{aligned}
		\omega^{\wedge n}(e_{1}, \dots,e_{n}, f_{1},\dots, f_{n})
		&=(\varphi^{*}\omega^{\wedge n})(e_{1}, \dots, e_{n}, f_{1},\dots, f_{n})
		\\
		&=
		\omega^{\wedge n}(\varphi e_{1}, \dots,\varphi  e_{n}, \varphi f_{1},\dots, \varphi f_{n})
		\\
		&=det(\varphi)\omega^{\wedge n}(e_{1}, \dots, e_{n}, f_{1},\dots, f_{n}),
		\end{aligned}
		$$
		portanto $det(\varphi) = 1$.
	\end{prova}
	
	
	\begin{definicao}\label{definicao_transformacao_simpletica}
		(Transformação simplética) Seja $(V, \omega)$ um 2n-espaço vetorial simplético sobre $\reta$. Um operador linear $T: V \to V$ é uma transformação simplética se 
		$$
		\formaSimpletica{Tu}{Tv} = \formaSimpletica{u}{v}
		$$ para todo $u,v\in V$.
	\end{definicao}
	
	\begin{definicao}\label{definicao_grupo_simpletico}
		(Grupo simplético) O grupo simplético $\gruposimpletico{V} \subset \generalgroupreal{2n}$ de $V$ é o conjunto das matrizes associadas as transformações simpléticas definidas em $V$.
	\end{definicao}
	
	\begin{exemplo}
		(Rotações em $\real{2}$) Vimos no Exemplo $\ref{exemplo_espaco_simpletico_real}$ que $(\real{2}, \omega)$ é um espaço vetorial simpletico, onde $\omega = dx\wedge dy$ e a base canônica $\{e_{x}, e_{y}\}$ é uma base simplética. Seja $R(\theta):\real{2}\to \real{2}$ uma rotação de um ângulo $\theta$, isto é, dado $v= v_{x}e_{x}+v_{y}e_{y} \in V$, temos $R(\theta)v = (v_{x}\cos(\theta)-v_{y}\sin(\theta))e_{x}+(v_{x}\sin(\theta)+v_{y}\cos(\theta))e_{y} = v'_{x}e_{x}+v'_{y}e_{y}$. Então
		$$
		\begin{aligned}
			\formaSimpletica{R(\theta)u}{R(\theta)v}&=
			\formaSimpletica{u'_{x}e_{x}+u'_{y}e_{y}}{v'_{x}e_{x}+v'_{y}e_{y}}
			\\
			&=u'_{x}v'_{y}\formaSimpletica{e_{x}}{e_{y}}	+u'_{y}v'_{x}\formaSimpletica{e_{y}}{e_{x}}
			\\
			&=(u_{x}\cos(\theta)-u_{y}\sin(\theta))(v_{x}\sin(\theta)+v_{y}\cos(\theta))
			\\
			&\;\;\;\;\;- (u_{x}\sin(\theta)+u_{y}\cos(\theta))(v_{x}\cos(\theta)-v_{y}\sin(\theta))
			\\
			&=u_{x}v_{y}-u_{y}v_{x}
			\\
			&=\formaSimpletica{u_{x}e_{x}}{v_{y}e_{y}} - \formaSimpletica{u_{y}e_{y}}{v_{x}e_{x}}
			\\
			&=\formaSimpletica{u}{v}.
		\end{aligned}
		$$
		Portanto, $R(\theta)$ é uma transformação simplética para todo $\theta \in \reta$.
	\end{exemplo}
	
	\section{$\estruturascomplexaspadrao$ e sua topologia}
	
	\begin{definicao}\label{definicao_estrutura_complexa}
		(Estrutura complexa) Uma estrutura complexa em um espaço vetorial $V$ é um endomorfismo linear $J: V \to V$, onde $J^{2} = -Id$. Dizemos que uma estrutura complexa em $V$ é compatível com a forma simplética (ou $\omega$-compatível) se $g(u,v):=\omega(u, Jv)$ define um produto interno em $V$. Denotaremos por $\estruturascomplexaspadrao$ o conjunto de todas as estruturas complexas em $V$ que sejam $\omega$-compatíveis.
	\end{definicao}
	
	\begin{observacao}\label{observacao_estrutura_complexa}
		Fixaremos a notação $\estruturacomplexa \in \estruturascomplexaspadrao$ como sendo o caso em que
		$$
		\estruturacomplexa=
		\left(
		\begin{array}{cc}
		0 & Id
		\\
		-Id & 0
		\end{array}
		\right).
		$$
	\end{observacao}
	
	Vamos mostrar que todo produto interno em $V$ pode ser associado a um elemento de $\estruturascomplexaspadrao$ e vice-versa, sendo que dessa associação vamos retirar informações sobre sua topologia.
	
	\begin{observacao}\label{observacao_conjunto_estrutura_complexa}
		Ao tratarmos $\estruturascomplexaspadrao$ como espaço topológico adotaremos sua topologia induzida pela topologia de $\generalgroupreal{V}$ gerada pela norma da convergência uniforme (ou norma do sup).
	\end{observacao}
	
	\begin{definicao}
		Seja $V$ um n-espaço vetorial real munido de um produto interno positivo-definido, então $\produtosinternos{V}$ é o conjunto de todos os produtos internos positivos-definidos em $V$.
	\end{definicao}
	
	\begin{observacao}
		Por definição $\produtosinternos{V} \subseteq \mathcal{L}(V \times V; \real{})$. Com isso, podemos muni-lo com a toplogia induzida por $\mathcal{L}(V \times V; \real{})$.
	\end{observacao} 
	
	\begin{lema}\label{lema_contratibilidade_produtos_internos}
		Seja $V$ um n-espaço vetorial com produto interno positivo-definido. Então $\produtosinternos{V}$ é contrátil.
	\end{lema}
	\begin{prova}
		Sejam $d_{1},d_{2} \in \produtosinternos{V}$ e $d:\intervalo\to \produtosinternos{V}$ tal que $d(s) = (1-s)d_{1}+ s d_{2}$. Afirmo que $d(s)$ é um produto interno. De fato, é bilinear e simétrico, pois é uma combinação linear de aplicações bilineares e simétricas. Resta-nos mostrar que $d(s)$ é positivo-definido. Supondo que $v =0$ teremos $d(s)(v,v)=(1-s)d_{1}(v,v)+ s d_{2}(v,v)  =0$, pois $d_{1}, d_{2}$ são positivos-definidos. Suponha que $v\neq 0$ e $d(s)(v,v) = 0$. Então $0=(1-s)d_{1}(v,v)+ s d_{2}(v,v)  $ o que é um absurdo pois $(1-s)d_{1}(v,v)>0 $ e $sd_{2}(v,v)> 0$. Logo $d(s)$ é positiva-definida para todo $s \in \intervalo$. Como $d_{1}, d_{2} \in \produtosinternos{V}$ são arbitrários, e $d$ é uma reta, então $\produtosinternos{V}$ é convexo, logo é contrátil.
	\end{prova}
	
	Vimos que, dados $(V, \omega)$ n-espaço vetorial simplético e uma estrutura complexa $J$ $\omega$-compatível, temos que $\formaSimpletica{v}{Ju} = g_{J}(v,u)$ é um produto interno em $V$. Com isso, a aplicação $G:\estruturascomplexaspadrao \to \produtosinternos{V}$ onde $G(J)(u,v) = \omega(u,Jv) = g_{J}(u,v)$ esta bem-definida e é injetora.
	
	A seguinte proposição afirma que vale a recíproca do resultado anterior.
	
	\begin{proposicao}
		Seja $(V, \omega)$ uma 2n-espaço vetorial simplético. Então cada produto interno $g \in \produtosinternos{V}$ define uma estrutura complexa $\omega$-compatível.
	\end{proposicao}
	\begin{prova}
		Seja $g \in \produtosinternos{V}$. Sabe-se que $\hat{g}:V \to V^{*}$, definido por $\hat{g}(v)(u)=g(v,u)$, é um isomorfismo. Além disso, existe um único automorfismo $A:V\to V$ tal que $\formaSimpletica{v}{u} = g(Av,u)$ para todo $v,u \in V$ dado por $A = \hat{g}^{-1}\omega
		^{*}$. Note que $g(Av,u) = \formaSimpletica{v}{u} = -\formaSimpletica{u}{v} = -g(Au,v) = g(-A^{t}v,u)$, portanto $A$ é anti-simétrico. Pelo Corolário $\ref{corolario_decomposicao_matriz_antisimetrica}$, podemos escrever $A=PJ$, onde $P = (-A^{2})^{1/2}$ é positiva-definida, $J$ é ortogonal tal que $J^{2} = -Id$ e $J^{t} = J^{-1}$. Com isso, temos $\formaSimpletica{v}{Ju} = g(Av, Ju) = g(J^{t}Av, u)  = g(J^{-1}Av, u) = g(Pv, u) = g_{P}(v,u)$, o que é um produto interno positivo-definido pois $P$ é positivo-definido. Portanto, $J$ é $\omega$-compatível. Assim, temos a aplicação injetora $F: \produtosinternos{V} \to \estruturascomplexas{V}{\omega}$ definida por $F(g) = J$.
	\end{prova}
	
	\begin{proposicao}
		$\estruturascomplexaspadrao$ é homeomorfo a $\produtosinternos{V}$, logo é contrátil.
	\end{proposicao}
	\begin{prova}
		Sejam $F: \produtosinternos{V} \to \estruturascomplexaspadrao$ e $G:\estruturascomplexaspadrao \to \produtosinternos{V}$ dos resultados anteriores. Seja $J \in \estruturascomplexaspadrao$ arbitrário. Então $(F\circ G)(J) = F(g_{J}) =J$, logo $F\circ G = Id_{\estruturascomplexaspadrao}$. Por outro lado, $(G\circ F)(g_{J}) = G(J) = g_{J}$, logo $G\circ F = Id_{\produtosinternos{V}}$. Portanto $F = G^{-1}$. Munindo $\estruturascomplexaspadrao$ com a topologia induzida por $\generalgroupreal{2n}$, então pode-se mostrar que $F$ é contínua. Analogamente, munindo $\produtosinternos{V}$ com a topologia induzida do espaço vetorial de todas as forma simétricas definida em $V$, pode-se mostrar que $G$ é contínua. Portanto $F$ é um homeomorfismo. Pelo Lema $\ref{lema_contratibilidade_produtos_internos}$, temos que $\produtosinternos{V}$ é contrátil, e como a contratibilidade é preservada por homeomorfismos, então $\estruturascomplexaspadrao$ é contrátil.
	\end{prova}
	
	\chapter{O Grupo Simplético $\gruposimpletico{2n}$}\label{capitulo_grupo_simpletico}
	Neste capítulo definiremos o grupo simplético e apresentaremos um estudo aprofundado das suas propriedades. O grupo simplético e suas características topológicas exercem papel central na construção dos índices de Maslov. Vamos mostrar que o grupo fundamental de $Sp(2n)$ é isomorfo aos inteiros e tal isomorfismo será dado pelo índice de Maslov.
	
	Daqui em diante vamos denotar por $\gruposimpletico{2n}$ o grupo simplético real $\gruposimpletico{\real{2n}}$.
	
	\begin{proposicao}\label{proposicao_grupo_simpletico_estrutura_grupo}
		$\gruposimpletico{2n}$ é um grupo com a operação de multiplicação de matrizes.
	\end{proposicao}
	\begin{prova}
		Sejam $A,B \in \gruposimpletico{2n}$ e $u,v \in \real{2n}$, então
		\begin{enumerate}
			\item \textit{(Operação fechada)} $\omega(ABu, ABv) = \omega(Bu, Bv) = \omega(u,v)$, logo $AB \in \gruposimpletico{2n}$.
			
			\item \textit{(Associatividade)} $\gruposimpletico{2n}$ é associativo pois a operação de multiplicação de matrizes reais é associativa.
			
			\item \textit{(Elemento neutro)} o elementro neutro de $\gruposimpletico{2n}$ é a identidade.
			
			\item \textit{(Elemento inverso)} Se $A \in \gruposimpletico{2n}$, então $\omega(u, v)=\omega(AA^{-1}u, AA^{-1}v) = \omega(A^{-1}u, A^{-1}v)$, logo $A^{-1} \in \gruposimpletico{2n}$. 
		\end{enumerate}
		Portanto $\gruposimpletico{2n}$ é um grupo.
	\end{prova}
	
	\begin{observacao}
		Quando tratarmos $\gruposimpletico{2n}$ como espaço topológico adotaremos sua topologia induzida pela topologia de $\generalgroupreal{2n}$ gerada pela norma da convergência uniforme.
	\end{observacao}
	
	Vejamos a seguinte caracterização do grupo simplético e sua relação com a estrutura complexa:
	
	\begin{lema}\label{lema_caracterizacao_Sp2n}
		(Caracterização de $Sp(2n)$) Se $(V, \omega)$ é um 2n-espaço vetorial simplético e $J \in \estruturascomplexaspadrao$ uma estrutura complexa $\omega$-compatível, então $A\in Sp(2n)$ se, e somente se, $A^{t}JA = J$. Além disso, podemos escrever 
		$$
		A=
		\left(
		\begin{array}{cc}
		B & C
		\\
		D & E
		\end{array}
		\right)
		$$
		onde $B^{t}D, C^{t}E, BC^{t}, DE^{t} $ são matrizes simétricas e $B^{t}E - D^{t}C = Id$ e $BE^{t} - CD^{t} = Id$.
	\end{lema}
	\begin{prova}
		Suponha $A \in Sp(2n)$, então:
		$$
		\omega(Av, Jw)= g(Av,w) = g(v,A^{t}w) = \omega(v, JA^{t}w) = \omega(Av, AJA^{t}w).
		$$
		Como a igualdade vale para quaisquer $v,w$, então $J = AJA^{t}$. Por outro lado, suponha que $A \in \generalgroupreal{2n}$, tal que $A^{t}JA=J$ para $J \in \estruturascomplexaspadrao$, então:
		$$
		\begin{aligned}
		\omega(Av, Aw) &= \omega(Av, -J^{2}Aw)
		\\
		&=g(Av, -JAw) 
		\\
		&= g(v, -A^{t}JAw) 
		\\
		&= g(v, -Jw) 
		\\
		&= \omega(v, -J^{2}w) 
		\\
		&= \omega(v, w), 
		\end{aligned}
		$$
		logo $A \in \gruposimpletico{2n}$. Seja $J \in \estruturascomplexaspadrao$ e $\{e, f\}$ uma base simplética de $V$ tal que a matriz $J$ com relação a essa base é $\estruturacomplexa$. A equação $A^{t}JA=J$ nos fornece as relações entre os blocos de matrizes de $A$:
		
		$$
		\begin{aligned}
		J &= A^{t}JA
		\\
		\left(
		\begin{array}{cc}
		0 & -Id
		\\
		Id & 0
		\end{array}
		\right)
		&=
		\left(
		\begin{array}{cc}
		B^{t} & D^{t}
		\\
		C^{t} & E^{t}
		\end{array}
		\right)
		\left(
		\begin{array}{cc}
		-D & -E
		\\
		B & C
		\end{array}
		\right)
		\\
		&=
		\left(
		\begin{array}{cc}
		-B^{t}D +D^{t}B & -B^{t}E+D^{t}C
		\\
		-C^{t}D+E^{t}B & -C^{t}E+E^{t}C
		\end{array}
		\right),
		\end{aligned}
		$$
		portanto $B^{t}D = D^{t}B = (B^{t}D)^{t}$ (matriz simétrica) e $D^{t}C-B^{t}E = Id$. De forma análoga, obtemos as outras identidades.
	\end{prova}
	
	Dada uma matriz $A \in \generalgroupcomplexo{n}$, podemos escrever $A = B+iC$ onde $B,C \in \generalgroupreal{n}$. Com isso, consideremos a aplicação $F:\generalgroupcomplexo{n} \to \generalgroupreal{2n}$ tal que 
	$$
	F(A)=
	\left(
	\begin{array}{cc}
	B & -C
	\\
	C & B
	\end{array}
	\right).
	$$
	Note que $F(A) = 0$ se, e somente se, $A=0$, logo $F$ é injetor. Pode-se verificar que $F(AB)=F(A)F(B)$, portanto é um monomorfismo. Além disso, temos a propriedade $F(A^{*}) = F(B^{t} - iC^{t}) = F(A)^{t}$. A aplicação $F$ é contínua pois dados $A=B+iC \in \generalgroupcomplexo{n}$ e $\epsilon > 0$, então para todo $X= Y+iZ \in \generalgroupcomplexo{n}$ tal que $||A - X||=max \{|B_{ij} - Y_{ij}|,  |C_{ij} - Z_{ij}|\} < \epsilon/2$ temos
	$$
	||F(A) - F(X)|| = max \{|B_{ij} - Y_{ij}|, |C_{ij} - Z_{ij}| \}< \epsilon/2 < \epsilon.
	$$
	
	Sejam $\matrizunitaria{n} = \{A\in \generalgroupcomplexo{n}: AA^{*}=Id \}$ o subgrupo das matrizes unitárias e $\matrizortogonal{n} = \{A \in \generalgroupreal{n}: AA^{t}  =Id \}$ o subgrupo das matrizes ortogonais, então temos o seguinte lema:
	
	
	
	\begin{lema}\label{lema_isomorfismo_U}
		Seja $F$ a aplicação contínua definida anteriormente. Então a restrição $F|_{\matrizunitaria{n}}: \matrizunitaria{n} \to \matrizSimpleticaOrtogonal $, onde $\matrizSimpleticaOrtogonal  = \gruposimpletico{2n}\cap \matrizortogonal{2n}$, é um isomorfismo. Além disso, dado $A \in \matrizSimpleticaOrtogonal $ temos $A\estruturacomplexa=\estruturacomplexa A$.
	\end{lema}
	\begin{prova}
		Temos que $\matrizSimpleticaOrtogonal  = \gruposimpletico{2n} \cap \matrizortogonal{2n}$ é não-vazio, pois a identidade esta na intersecção. Tomando $A \in \matrizSimpleticaOrtogonal $, então $A^{t}A= Id$. Com isso, temos
		$$
		\begin{aligned}
		Id=A^{t}A &=
		\left(
		\begin{array}{cc}
		B^{t} & D^{t}
		\\
		C^{t} & E^{t}
		\end{array}
		\right)
		\left(
		\begin{array}{cc}
		B & C
		\\
		D & E
		\end{array}
		\right)
		\\
		&= 
		\left(
		\begin{array}{cc}
		B^{t}B + D^{t}D & B^{t}C + D^{t}E 
		\\
		C^{t}B + E^{t}D  & CC^{t}+EE^{t}
		\end{array}
		\right)
		\\
		&=
		\left(
		\begin{array}{cc}
		B^{t}B + D^{t}D & 0 
		\\
		0 & CC^{t}+EE^{t}
		\end{array}
		\right),
		\end{aligned}
		$$
		onde a condição é satisfeita quando $C^{t}B =- E^{t}D$, $B^{t}C =- D^{t}E$ e $BB^{t} + DD^{t} = CC^{t}+EE^{t} = Id$. 
		
		Dada $Z \in \matrizunitaria{n}$, temos $Id=F(Id) = F(Z^{*}Z) = F(Z^{*})F(Z) = F(Z)^{t}F(Z)$, portanto $F(Z) \in \matrizortogonal{2n}$. Fazendo $Z= B+iC$ teremos
		$$
		F(Z)=
		\left(
		\begin{array}{cc}
		B & -C
		\\
		C & B
		\end{array}
		\right)
		$$
		e
		$$
		F(Z^{*})F(Z)=
		\left(
		\begin{array}{cc}
		BB^{t} +CC^{t} & -B^{t}C +C^{t}B
		\\
		B^{t}C -C^{t}B & BB^{t} +CC^{t}
		\end{array}
		\right)	
		= Id
		$$
		o que implica que devemos ter $B^{t}C =C^{t}B$, logo pelo Lema $\ref{lema_caracterizacao_Sp2n}$ temos que $F(Z) \in \gruposimpletico{2n}$. Portanto $F(Z) \in \matrizSimpleticaOrtogonal $, o que implica que $F(\matrizunitaria{n}) \subseteq \matrizSimpleticaOrtogonal$.
		
		Vamos mostrar a inclusão $\matrizSimpleticaOrtogonal \subseteq F(\matrizunitaria{n})$. 
		
		Se $A \in \matrizSimpleticaOrtogonal$, então $A^{t}=A^{-1}$ e $\estruturacomplexa=A^{t}\estruturacomplexa  A=A^{-1}\estruturacomplexa A$, o que implica que $A\estruturacomplexa=\estruturacomplexa A$. Assim:
		$$
		\begin{aligned}
		A\estruturacomplexa&=\estruturacomplexa A
		\\
		\left(
		\begin{array}{cc}
		B & C
		\\
		D & E
		\end{array}
		\right)
		\left(
		\begin{array}{cc}
		0 & -Id
		\\
		Id & 0
		\end{array}
		\right)
		&=
		\left(
		\begin{array}{cc}
		0 & -Id
		\\
		Id & 0
		\end{array}
		\right)
		\left(
		\begin{array}{cc}
		B & C
		\\
		D & E
		\end{array}
		\right)
		\\
		\left(
		\begin{array}{cc}
		C & -B
		\\
		E & -D
		\end{array}
		\right)
		&=
		\left(
		\begin{array}{cc}
		-D & -E
		\\
		B & C
		\end{array}
		\right), 
		\end{aligned}
		$$
		logo temos $C=-D$ e $B=E$, portanto:
		$$
		A=\left(
		\begin{array}{cc}
		B & -C
		\\
		C & B
		\end{array}
		\right).
		$$
		
		Tomando $Z = B+iC \in \matrizunitaria{n}$ temos que $F(Z) = A$, o que implica que $\matrizSimpleticaOrtogonal \subseteq F(\matrizunitaria{n})$. Conclusão, $\matrizSimpleticaOrtogonal = F(\matrizunitaria{n})$. 
		
		Como $F$ é monomorfismo, então $F|_{\matrizunitaria{n}}:\matrizunitaria{n} \to \matrizSimpleticaOrtogonal$ é sobrejetora sobre sua imagem, portanto é um isomorfismo.
	\end{prova}
	
	Analisaremos agora o espectro $\espectrooperador{A}$ de um simplectomorfismo $A \in \gruposimpletico{2n}$ de um determinado espaço vetorial. Tal resultado será usado na demonstração da contratibilidade do quociente $\gruposimpletico{2n}/\matrizSimpleticaOrtogonal$, que por sua vez será utilizado na construção do índice de Maslov.
	
	\begin{lema}\label{lema_caracterizacao_espectro_semelhante}
		Seja $A \in \gruposimpletico{2n}$. Então, $A, A^{-1}, A^{t}$ são semelhantes. Com isso, $\sigma(A) = \sigma(A^{-1}) = \sigma(A^{t}) $.
	\end{lema}
	\begin{prova}
		Temos que $A^{t}\estruturacomplexa A = \estruturacomplexa$. Supondo que $A$ é invertível, então $A^{t} = \estruturacomplexa A^{-1} \estruturacomplexa^{-1}$. Logo $A^{t}$ é semelhante a $A^{-1}$. Sabe-se que $A$ é semelhante a $A^{t}$, logo $A$ é semelhante a $A^{-1}$. Como matrizes semelhantes possuem o mesmo polinômio característico, então $\sigma(A) = \sigma(A^{-1}) = \sigma(A^{t}) $.
	\end{prova}
	
	\begin{observacao}
		$\lambda \in \sigma(A)$ se , e somente se, $\lambda^{-1}\in \sigma(A)$.
	\end{observacao}
	
	\begin{lema}\label{lema_auto_espaco_grupo_simpletico}
		(Auto-espaços de $\gruposimpletico{2n}$) Sejam $(V, \omega)$ um 2n-espaço vetorial simplético, $A \in \gruposimpletico{2n}$ e $r,s \in \inteiros$ tais que $r\geq 1$ e $s\geq 1$. Se $\lambda, \mu \in \sigma(A)$ tais que $\lambda\mu \neq 1$, então seus auto-espaços generalizados $E_{\lambda}=Ker(A-\lambda Id)^{r}$ e  $E_{\mu}=Ker(A-\mu Id)^{s}$ são $\omega$-ortogonais, isto é, $\omega(E_{\lambda}, E_{\mu}) = 0$.
	\end{lema}
	\begin{prova}
		Seja $P(r,s)$ a propriedade que queremos mostrar. Por indução finita, vejamos que $P(1,1)$ é verdadeira. Dados $v\in E_{\lambda}$ e $u\in E_{\mu}$ temos que $\formaSimpletica{v}{u} = \formaSimpletica{Av}{Au} = \lambda\mu\formaSimpletica{v}{u}$, o que implica que $\formaSimpletica{v}{u} = 0$, pois $\lambda\mu\neq 1$. Aplicando a hipótese de indução no índice $s$, podemos assumir que $P(1,s)$ é verdadeira e que $u \in Ker(A-\mu Id)^{s+1}$. Com isso, $0=(A-\mu Id)^{s+1}u = (A-\mu Id)^{s}(A-\mu Id)u $, logo $(A-\mu Id)u \in Ker(A-\mu Id)^{s}$, o que implica em
		$$
		\begin{aligned}
		\formaSimpletica{v}{u}
		&=\formaSimpletica{Av}{Au}
		\\
		&= \formaSimpletica{Av}{Au -\mu u +\mu u} 
		\\
		&= \lambda\underbrace{\formaSimpletica{v}{(A-\mu Id)u}}_{=0}+\lambda\mu\formaSimpletica{v}{u}
		\\
		&=\lambda\mu\formaSimpletica{v}{u}.
		\end{aligned}
		$$
		Como $\lambda\mu \neq 1$, então $\formaSimpletica{v}{u}=0$ e $P(1, s+1)$ é verdadeira.
		
		Concluímos que $P(1, s)$ é verdadeira para todo $s\geq 1$. Analogamente, mostramos por indução que P(r,1) é verdadeira para todo $r\geq 1$. Para mostrar que P(r,s) é verdadeira por indução, assumimos que P(r,s+1) e P(r+1,s) são verdadeiras e, analogamente mostramos que P(r+1,s+1) é verdadeira. Portanto, $\omega(E_{\lambda}, E_{\mu}) = 0$ para quaisquer inteiros $r\geq 1$ e $s\geq 1$.
		
	\end{prova}
	
	\begin{corolario}\label{corolario_restricao_forma_simpletica}
		Sejam $A \in \gruposimpletico{2n}$ e $\lambda, \mu \in \sigma(A)$. 
		\begin{enumerate}
			\item As restrições $\omega|_{E_{\pm 1} \times E_{\pm 1}}$ são não-degeneradas. Além disso, suas multiplicidades $m(\pm 1)$ são pares.
			
			\item Se $\lambda \neq \pm 1$, então a restrição $\omega|_{W_{\lambda} \times W_{\lambda}}$ é não-degenerada, onde $W_{\lambda } = E_{\lambda} \oplus E_{\lambda^{-1}}$.
		\end{enumerate}
	\end{corolario}
	\begin{prova}
		\begin{enumerate}
			\item  Faremos primeiro o caso em que $\lambda = \mu =1$. Suponha que $\omega|_{E_{1}}$ seja degenerada. Então existe $v\neq 0 \in E_{1}$ tal que $\omega(v, u) = 0$ para todo $u \in E_{1}$. Seja $\beta \in \complementar{\sigma(A)}{\{1 \}}$. Pelo Lema $\ref{lema_auto_espaco_grupo_simpletico}$ temos que $\formaSimpletica{E_{1}}{E_{\beta}} = 0$. Pelo Teorema $\ref{teorema_espectral_jordan}$ podemos escrever $V = E_{1}  \bigoplus_{\lambda \in \complementar{\sigma(A)}{ \{1\}  }}E_{\lambda}$. Com isso, $\formaSimpletica{v}{V} = 0$. Logo $\omega$ é degenerada, o que contradiz a hipótese. Portanto, $\omega|_{E_{1} \times E_{1}}$ é não-degenerada. Além disso, como $\lambda = \lambda^{-1}$, então a multiplicidade $m(1)$ é par. Com argumento análogo para $\lambda = \mu = -1$ temos que $\omega|_{E_{-1} \times E_{-1}}$ é não-degenerada e $m(-1)$ é par. 
			
			\item Pelo Teorema $\ref{teorema_espectral_jordan}$ podemos escrever $V = E_{1}  \bigoplus E_{-1}  \bigoplus_{\lambda \in \complementar{\sigma(A)}{ \{\pm 1\}  }}E_{\lambda}$. Supondo que $\omega|_{E\times E}$ seja degenerada, onde $E= E_{\lambda} \oplus E_{\lambda^{-1}}$, então $v \neq 0 \in E$ tal que 
			$\omega(v,a)=0$ para todo $a\in E$. Além disso, pela $\omega$-ortogonalidade,  $\omega(v,E_{\beta})=0$, para todo $\beta\in \complementar{\sigma(A)}{\{1\} }$. Segue do Teorema $\ref{teorema_espectral_jordan}$ analogamente ao item anterior que $\omega(v,V)=0$, o que contradiz a hipótese de $\omega$ ser não degenerada.
		\end{enumerate}
	\end{prova}
	
	Considere $\matrizsimetricapositiva{2n} \subset \matrizquadreal{2n}$ como sendo o conjunto de todas as matrizes positivas-definidas (veja a Definição $\ref{definicao_matriz_positiva_definida}$).
	
	\begin{proposicao}\label{proposicao_potenciacao_grupo_simpletico}
		(Potênciação em $\gruposimpletico{2n}$) Seja $\gruposimpleticopositivo{2n} = \gruposimpletico{2n} \cap \matrizsimetricapositiva{2n}$ o conjunto das matrizes simpléticas simétricas e positivas-definidas. Dado $A \in \gruposimpleticopositivo{2n}$, então $A^{\alpha} \in \gruposimpletico{2n}$ para qualquer $\alpha \in \real{}$.
	\end{proposicao}
	\begin{prova}
		Seja $A \in \gruposimpleticopositivo{2n}$, então $A$ é simétrica, logo é normal, e pelo Lema $\ref{lema_caracterizacao_matriz_normal}$ é diagonalizável. Além disso, pela Observação $\ref{observacao_matriz_positiva_definida}$ seus auto-valores são todos positivos. Podemos decompor $V$ na soma direta de seus auto-espaços $E_{\lambda}$, onde $\lambda \in \espectrooperador{A}$. Se $u \in E_{\lambda_{u}}$ e $v \in V_{\lambda_{v}}$, então
		$$
		\omega(A^{\alpha}u,A^{\alpha}v) = 		(\lambda_{u}\lambda_{v})^{\alpha}\omega(u,v).
		$$
		Se $\lambda_{u}\lambda_{v}\neq 1$, então $\omega(u,v)=0$, o que implica que $\omega(A^{\alpha}u,A^{\alpha}v)=(\lambda_{u}\lambda_{v})^{\alpha}\omega(u,v)=0$, logo $\omega(A^{\alpha}u,A^{\alpha}v) = \omega(u,v)=0$, portanto $A^{\alpha} \in \gruposimpletico{2n}$. Caso $\lambda_{u}\lambda_{v}=1$ temos $\omega(A^{\alpha}u,A^{\alpha}v) = \omega(u,v)$, portanto $A^{\alpha} \in \gruposimpletico{2n}$.
	\end{prova}
	
	\begin{observacao}\label{observacao_determinante_matriz_unitaria}
		Note que, dado $A \in \matrizunitaria{n}$, temos $1= det(AA^{*}) = det(A)det(A^{*}) = det(A)\overline{det(A)} = ||det(A)||^{2}$, portanto $det(\matrizunitaria{n}) \subseteq S^{1}$.
	\end{observacao}
	
	
	\begin{lema}\label{lema_conexidade_matriz_unitaria}
		$\matrizSimpleticaOrtogonal$ é conexo por caminhos, logo é conexo.
	\end{lema}
	\begin{prova}
		Seja $A \in \matrizunitaria{n}$. Então $A$ é diagonalizável, e $det(A) \in S^{1}$, e pela Observação $\ref{observacao_caracterizacao_matriz_normal}$ existe uma matriz unitária $U$ tal que $A=U^{*}diag\{e^{i\theta_{1}}, \dots, e^{i\theta_{n}}\}U$, com isso temos $det(A) = e^{i(\theta_{1}+\dots+\theta_{n})} \in \circulo$. Definindo o caminho contínuo $\gamma:[0,1] \to \matrizunitaria{n}$ tal que $\gamma(\lambda)=U^{*}diag\{e^{i\theta_{1}\lambda}, \dots, e^{i\theta_{n}\lambda}\}U$, temos $\gamma(0)=Id$ e $\gamma(1)=A$. Temos que $\gamma$ está bem-definida pois $D=diag\{e^{i\theta_{1}\lambda}, \dots, e^{i\theta_{n}\lambda}\}$ é tal que $D^{*}D = Id$, logo $D \in \matrizunitaria{n}$. Além disso, $U^{*}, U \in \matrizunitaria{n}$, por construção. Então temos que $\gamma(\lambda) \in \matrizunitaria{n}$. Notemos que $det(\gamma(\lambda)) = e^{i\lambda(\theta_{1}+\dots+\theta_{n})} \in \circulo$, logo $\gamma([0,1]) \subset \matrizunitaria{n}$. Com isso, toda $A \in \matrizunitaria{n}$ pode ser conectada a $Id$ por um caminho contínuo, logo $\matrizunitaria{n}$ é conexo por caminhos, portanto é conexo. Pelo Lema $\ref{lema_isomorfismo_U}$, temos que $F|_{\matrizunitaria{n}}:\matrizunitaria{n} \to \matrizSimpleticaOrtogonal$ é um isomorfismo contínuo, logo é um homeomorfimo. Como $\matrizunitaria{n}$ é conexo e a conexidade é preservada por aplicações contínuas, então $\matrizSimpleticaOrtogonal = F(\matrizunitaria{n})$ é conexo por caminhos, logo é conexo.
	\end{prova}
	
	\begin{lema}
		$\gruposimpleticopositivo{2n}$ é conexo por caminhos, logo é conexo.
	\end{lema}
	\begin{prova}
		Seja a aplicação contínua $\gamma:\gruposimpleticopositivo{2n}\times [0,1] \to \gruposimpletico{2n}$ tal que $\gamma(A,\lambda) = A^{\lambda}$. Pela Proposição $\ref{proposicao_potenciacao_grupo_simpletico}$, a aplicação $\gamma$ está bem-definida e é contínua. Fixando $A \in \gruposimpleticopositivo{2n}$ temos a curva $\gamma_{A}:[0,1]\to \gruposimpletico{2n}$ tal que $\gamma_{A}(0) = Id$ e $\gamma_{A}(1) = A$, ou seja, é um caminho contínuo que conecta a identidade a matriz $A$. Como $A=A^{t}$, segue da Proposição $\ref{proposicao_potenciacao_grupo_simpletico}$ que $\gamma_{A}(\lambda)^{t} = (A^{\lambda})^{t} = (A^{t})^{\lambda} = (A)^{\lambda} = \gamma_{A}(\lambda)$. Além disso, os k-subdeterminantes $det_{k}(\gamma_{A}(\lambda)) = det_{k}(A^{\lambda}) > 0$ para $1\leq k \leq 2n$. Pelo fato de que $\gamma_{A}(\lambda)$ é simétrica e pelo Teorema $\ref{teorema_matriz_positiva_definida}$ temos que $\gamma_{A}(\lambda)$ é positiva-definida, logo $\gamma_{A}([0,1]) \subset \gruposimpleticopositivo{2n}$. Como $A \in \gruposimpleticopositivo{2n}$ é arbitrária, então a construção anterior vale para quaisquer elementos de $\gruposimpleticopositivo{2n}$. Com isso, dados $A, B \in \gruposimpleticopositivo{2n}$, podemos conectar $A$ a $B$ por uma curva contínua que passa pela identidade, portanto $\gruposimpleticopositivo{2n}$ é conexo por caminhos, logo é conexo.
	\end{prova}	
	
	\begin{lema}\label{lema_decomposicao_grupo_simpletico_positivo}
		Se $A \in \gruposimpletico{2n}$, então existem únicas $P \in \gruposimpleticopositivo{2n}$ e $O \in \matrizSimpleticaOrtogonal$ tais que $A=PO$.
	\end{lema}
	\begin{prova}
		Segue do Teorema $\ref{teorema_decomposicao_polar}$ que $A=PO$, onde $P=AA^t$ é positiva definida e O é uma matriz ortogonal. Como $A$ e $A^{t} $ são matrizes simpléticas, então $P=AA^{t}\in \gruposimpletico{2n}$. 
		Além disso, como $\gruposimpletico{2n}$ é grupo então $P^{-1}\in \gruposimpletico{2n}$, logo $O=P^{-1}A \in \mathcal{U}$.
	\end{prova}
	
	\begin{teorema}\label{teoerma_sp2n_conexo}
		$\gruposimpletico{2n}$ é conexo por caminhos, logo é conexo.
	\end{teorema}
	\begin{prova}
		Se $A \in \gruposimpletico{2n}$, então pelo Lema $\ref{lema_decomposicao_grupo_simpletico_positivo}$ podemos escrever $A=PO$ onde $P \in \gruposimpleticopositivo{2n}$ e $O\in \matrizSimpleticaOrtogonal$ são únicas. Pela unicidade da decomposição anterior a aplicação $G: \gruposimpletico{2n} \to \gruposimpleticopositivo{2n} \times \matrizSimpleticaOrtogonal$ definida por $G(A) = (P,O)$ é injetora. Por outro lado, dado $(P,O) \in \gruposimpleticopositivo{2n} \times \matrizSimpleticaOrtogonal$ temos $(PO)^{t}\estruturacomplexa PO = O^{t}P^{t}\estruturacomplexa PO = O^{t}\estruturacomplexa O = \estruturacomplexa$, logo pelo Lema $\ref{lema_caracterizacao_Sp2n}$, temos $PO \in \gruposimpletico{2n}$ e $G$ é sobrejetora. De fato, definindo $A=PO \in \gruposimpletico{2n}$ temos que $G(A) = (P,O)$. Portanto $G$ é bijetora. 
		
		Denotando $G^{-1}:\gruposimpleticopositivo{2n} \times \matrizSimpleticaOrtogonal\to \gruposimpletico{2n}$ pela inversa de $G$, temos que $G^{-1}(P,O) = PO$, o que implica que $G^{-1}$ é contínua, pois o produto de matrizes é uma operação contínua.
		
		Afirmo que $\gruposimpleticopositivo{2n}\times \matrizSimpleticaOrtogonal$ é conexo por caminhos, pois é o produto cartesianos de espaços topológicos conexos por caminhos. Além disso, como a conexidade é preservada por aplicações contínuas, então $\gruposimpletico{2n}$ é conexo por caminhos, logo é conexo.
	\end{prova}
	
	\begin{observacao}\label{observacao_decomposicao_Sp2n}
		No teorema anterior, exibimos uma bijeção entre $\gruposimpletico{2n} $ e $\gruposimpleticopositivo{2n} \times \matrizSimpleticaOrtogonal$, isto é, toda $A \in \gruposimpletico{2n}$ pode ser decomposta unicamente como $A=PO$, onde $P\in \gruposimpleticopositivo{2n}$ e $O \in \matrizSimpleticaOrtogonal$. Usaremos essa afirmação na demonstração de alguns resultados adiante.
	\end{observacao}
	
	\begin{teorema}
		O quociente $\gruposimpletico{2n}/\matrizSimpleticaOrtogonal$ é contrátil.
	\end{teorema}
	\begin{prova}
		Na Observação $\ref{observacao_decomposicao_Sp2n}$ foi mostrado que se $A \in \gruposimpletico{2n}$ temos $A=PO$, onde $P \in \gruposimpleticopositivo{2n}$ e $O \in \matrizSimpleticaOrtogonal$. Com isso temos $AA^{t} = POO^{t}P^{t} = PP^{t}=P^{2}$, e ela Proposição $\ref{proposicao_potenciacao_grupo_simpletico}$, podemos afirmar que $P^{\alpha} \in \gruposimpletico{2n}$ para todo $\alpha \in \real{}$. Definindo a aplicação $r:\gruposimpletico{2n}\times [0,1] \to \gruposimpletico{2n}$ tal que $r(A, \alpha) = (AA^{t})^{-\alpha/2}A$ é contínua pois é o produto de matrizes, que é uma operação contínua em $\gruposimpletico{2n}$. Temos que $r$ é um retrato de deformação de $\gruposimpletico{2n}$ sobre $\matrizSimpleticaOrtogonal$ pois $r(A, 0) = A$, $r(A, 1) = (AA^{t})^{-1/2}A = P^{-1}A = O \in \matrizSimpleticaOrtogonal$ e, por fim, tomando $B \in \matrizSimpleticaOrtogonal$ temos $r(B, 1) = (BB^{t})^{-1/2}B = B$, pois $BB^{t} = Id$.
		
		Por brevidade, denotaremos $\mathcal{S} = \gruposimpletico{2n}/\matrizSimpleticaOrtogonal$. Definindo a aplicação $R:\mathcal{S} \times [0,1] \to \mathcal{S}$ por $R([A], \lambda) = [r(A, \lambda)] = [(AA^{t})^{-\lambda/2}A]$, temos que $R$ é uma contração. De fato, a imagem de $R$ é a classe de equivalência da imagem de $r$, que é contínua, portanto $R$ é contínua. Além disso, $R([A], 0) = [A]$, $R([A], 1) = [(AA^{t})^{-1/2}A] = [P^{-1}A] = [O] = [Id]$, pois $O \in \matrizSimpleticaOrtogonal$, isto é, $R(., 0) = Id_{\mathcal{S}}(.)$ é a identidade e $R(., 1) = [Id]$ é a aplicação constante, logo é uma contração e $\mathcal{S}$ é contrátil.
	\end{prova}
	
	\begin{observacao}\label{observacao_quociente_grupo_simpletico_contratil}
		Na demonstração da contratibilidade do quociente $\gruposimpletico{2n}/\matrizSimpleticaOrtogonal$ mostramos que $\matrizSimpleticaOrtogonal$ é um retrato por deformação de $\gruposimpletico{2n}$, logo todo caminho contínuo em $\gruposimpletico{2n}$ pode ser deformado contínuamente em um caminho contínuo em $\matrizSimpleticaOrtogonal$.
	\end{observacao}
	
	\begin{observacao}\label{observacao_conexidade_grupo_simpletico}
		Até o momento, foram demonstrados muitos lemas técnicos afim de examinarmos algumas características da topologia de $\gruposimpletico{2n}$ com o objetivo de mostrar que esse conjunto é conexo. Esse resultado é fundamental para a construção da homologia de Floer, pois, para definirmos um complexo de cadeia nessa homologia devemos ter um homomorfismo graduado. Tal graduação será dada pelo índice de Maslov e este será relacionado ao grupo fundamental $\grupofundamental{\gruposimpletico{2n}}$. Por fim, a estratégia adotada necessita que $\grupofundamental{\gruposimpletico{2n}} \cong \inteiros$. Para mostrar esse fato precisamos da conexidade.
	\end{observacao}

	Seja $Det:\caminhossempontobase{\matrizunitaria{n}} \to \caminhossempontobase{\circulo}$ dada por $Det(\gamma)(t) = \det(\gamma(t))$. Essa aplicação esta bem-definida pois, pela Observação \ref{observacao_determinante_matriz_unitaria}, $\det(\matrizunitaria{n}) \subseteq \circulo$. Sejam $\gamma, \beta \in\caminhospontobasegeral{p}{\matrizunitaria{n}} $ tais que $\gamma(0)=\beta(0)$, $\gamma(1)=\beta(1)$, e $h : \intervalo \times \intervalo \to \matrizunitaria{n}$ uma homotopia entre $\gamma$ e $\beta$ tal que $h(t,0) = \gamma(t)$ e $h(t,1)=\beta(t)$.
	
	Definina a função contínua $H = Det \circ h:\intervalo \times \intervalo \to \circulo$ por $H(t,s) = Det(h(t,s))$. Então $H(t,0) = Det(h(t,0)) = Det(\gamma(t)) = Det(\gamma)(t)$ e $H(t,1) = Det(h(t,1)) = Det(\beta(t)) = Det(\beta)(t)$. Logo $H$ é uma homotopia entre $Det(\gamma)$ e $Det(\beta)$. Portanto, se $\gamma \sim \beta$ então $Det(\gamma) \sim Det(\beta)$.
	
	Além disso, a justaposição dos caminhos $Det(\gamma),Det(\beta) \in \caminhossempontobase{\circulo}$ é dada por
	$$
	(Det(\gamma)*Det(\beta)) (t)= Det(\gamma)(t)*Det(\beta) (t) = Det(\gamma*\beta)) (t).
	$$

	\begin{proposicao}\label{proposicao_aplicacao_det_homomorfismo}
		A aplicação $Det_{*}: \grupofundamental{\matrizunitaria{n}} \to \grupofundamental{\circulo}$ definida por $Det_{*}(\classe{\gamma}) = \classe{Det(\gamma)}$ é um homomorfismo.
	\end{proposicao}	
	\begin{prova}
		De fato, se $\classe{\gamma}, \classe{\beta} \in \grupofundamental{\matrizunitaria{n}}$, então $Det_{*}(\classe{\gamma}.\classe{\beta}) = Det_{*}(\classe{\gamma *\beta}) =\classe{Det(\gamma *\beta)} = \classe{Det(\gamma) *Det(\beta)} = Det_{*}(\classe{\gamma})*Det_{*}(\classe{\beta})$.
	\end{prova}
	\begin{lema}\label{lema_isomorfismo_grupo_fundamental_Un}
		$Det_{*}: \grupofundamental{\matrizunitaria{n}} \to \grupofundamental{\circulo}$ é um isomorfimo, logo $\grupofundamental{\matrizunitaria{n}} \cong \inteiros$.
	\end{lema}
	\begin{prova}
		Como $Det$ é sobrejetora, por construção, então $Det_{*}$ é sobrejetora. Suponha que $Det_{*}$ não é injetora. Então existem $\classe{\gamma} , \classe{\beta} \in \grupofundamental{\matrizunitaria{n}}$ tais que $\classe{\gamma} \neq \classe{\beta}$ e $Det_{*}(\classe{\gamma})= Det_{*}(\classe{\beta} )$, isto é, $\classe{Det(\gamma)}= \classe{Det(\beta)}$. Com isso, tem-se que $Det(\gamma) \sim Det(\beta)$ e $\gamma \sim \beta$, logo $\classe{\gamma} = \classe{\beta}$, contradizendo a hipótese de que $\classe{\gamma} \neq \classe{\beta}$. Portanto $Det_{*}$ é injetora e sobrejetora, logo é um isomorfismo e $\grupofundamental{\matrizunitaria{n}} \cong \grupofundamental{\circulo} \cong \inteiros$.
	\end{prova}
	
	\begin{teorema}
		$\grupofundamental{\gruposimpletico{2n}} \cong \inteiros$.
	\end{teorema}
	\begin{prova}
		Como $\matrizSimpleticaOrtogonal$ é um retrato de deformação de $\gruposimpletico{2n}$, então $\grupofundamental{\gruposimpletico{2n}}$ e $\grupofundamental{\matrizSimpleticaOrtogonal}$ são isomorfos. Do Lema $\ref{lema_isomorfismo_U}$ temos que $\matrizSimpleticaOrtogonal\cong \matrizunitaria{n}$, logo $\grupofundamental{\matrizSimpleticaOrtogonal} \cong \grupofundamental{\matrizunitaria{n}} \cong \inteiros$ e $\grupofundamental{\gruposimpletico{2n}} \cong \inteiros$.
	\end{prova}
	
	\chapter{Índice de Maslov - A Construção de Conley-Zehnder}
	\section{Motivação}
	Considere a função Hamiltoniana dependente do tempo $H:M\times \reta\to \reta$ ao qual associamos o campo vetorial $X_{H} \in \campossuaves{M}$, chamado campo Hamiltoniano, e definido pela equação $\formaSimpletica{\campohamiltoniano{t}}{Y} = -dH(Y)$, cujo fluxo $\psi:M\times \reta \to M$ é solução do sistema Hamiltoniano
	$$
	\derivadaparcial{\psi(t)}{t} = X_{H}(\psi(t), t)
	$$
	satisfazendo as condições periódicas de contorno $\psi(t+1) = \psi(t)$, $\psi(0) = \psi_{0}$ e $\dot{\psi}(0) = \campohamiltoniano{0}$. Tais soluções são chamadas de soluções 1-periódicas das equações de Hamilton. Além disso, buscamos as soluções 1-periódicas que são contráteis, isto é, homotópicas a uma curva constante. Elas são pontos críticos do funcional de ação $f_{H}$. Analogamente a Teoria de Morse vamos atribuir um índice (um número inteiro) a cada um desses pontos críticos, o qual denominaremos por índice de Maslov.
	
	Vamos realizar a associação $\reta/\mathbb{Z} \ni t \mapsto A(t) \in \gruposimpletico{2n}$, onde $A$ é um caminho contínuo tal que $A(0) = Id$ e $det(Id - A(t))\neq 0$. Para cada um desses caminhos teremos a associação $\gruposimpletico{2n} \ni A(t) \mapsto \rho(A) \in \circulo$. Por fim, teremos o índice de Maslov $A \mapsto \mu(A) \in \inteiros$.
	
	Originalmente, o índice de Maslov foi definido para associar um caminho fechado em $\gruposimpletico{2n} $ a um número inteiro. Contudo, existe uma pluralidade de definições equivalentes desse mesmo objeto, e por equivalente entende-se aquelas definições que satistazem a mesma axiomatização. Em $\cite{cappell_maslov_index_equivalencia}$, pode-se encontrar quatro definições distintas e a demonstração de suas equivalências.
	
	\section{Construção de $\rho: \gruposimpletico{2n} \to \circulo$}
	Para a construção do índice de Maslov, vamos adotar as complexificações $(\complexificado{V}, \Omega) $ e $\gruposimpleticocomplexo{2n}$ do 2n-espaço vetorial simplético $(V, \omega)$ e do grupo simplético $\gruposimpletico{2n}$, respectivamente. Os detalhes da complexificação podem ser encontrados na Apêndice $\ref{apendice_complexificacao_espacos_vetoriais}$. 
	
	O primeiro passo é a construção de uma aplicação contínua $\rho: \gruposimpletico{2n}\to \circulo$ satisfazendo determinadas propriedades, as quais serão herdadas pelo índice de Maslov e serão aplicadas nas demonstrações de sua caracterização, como está feito na Seção \ref{secao_indice_maslov}.
	
	\begin{teorema}\label{teorema_aplicacao_rho}
		Sejam $(V_{1}, \omega_{1})$ e $(V_{2}, \omega_{2})$ 2n-espaços vetoriais simpléticos. Existe uma aplicação contínua $\rho:Sp(2n) \to S^{1}$ satisfazendo as seguintes propriedades:
		\begin{enumerate}
			\item \label{item_naturalidade_rho} \textbf{Naturalidade:}  Se $T:V_{1} \to V_{1}$ é um isomorfismo simplético, isto é, $T^{*}\omega_{1} = \omega_{1}, $então 
			$$
			\rho(TAT^{-1}) = \rho(A)
			$$
			para todo $A\in \gruposimpletico{V_{1}}$.
			
			\item \label{item_produto_rho} \textbf{Produto:} Se $(V,\omega) = (V_{1}\times V_{2},\omega_{1}\times \omega_{2})$, então
			$$
			\rho(A) = \rho(A_{1})\rho(A_{2})
			$$
			para $A\in \gruposimpletico{V}$ definida por $A(v_{1}, v_{2})=(A_{1}v_{1}, A_{2}v_{2})$, onde $A_{i} \in \gruposimpletico{V_{i}}$.
			
			\item \label{item_determinante_rho} \textbf{Deteminante:} Se $A\in \matrizSimpleticaOrtogonal$, então 
			$$
			\rho(A) = det(X+iY), \text{onde} \;	
			A=\left(
			\begin{array}{cc}
			X & -Y					\\
			Y & X
			\end{array}
			\right).
			$$
			Além disso, a aplicação induzida $\rho_{*}: \grupofundamental{\gruposimpletico{2n}} \to \grupofundamental{\circulo}$ é um isomorfismo.
			
			\item \label{item_normalizacao_rho} \textbf{Normalização:} Se $A \in \gruposimpletico{2n}$ com $\sigma(A)\cap \circulo = \emptyset$, então $\rho(A) = \pm 1$.
			
			\item \label{item_inversa_rho} \textbf{Inversa:} Para todo $A \in \gruposimpletico{2n}$ tem-se que $\rho(A^{-1})=(\rho(A))^{-1}$. 
		\end{enumerate}
	\end{teorema}
	
	No Apêndice $\ref{apendice_complexificacao_espacos_vetoriais}$ mostrou-se que, dados $\lambda \in \sigma(A)$, temos que $\lambda, \lambda^{-1}, \overline{\lambda}, \overline{\lambda}^{-1}  \in \sigma(A)$. Suponha que $\lambda, \overline{\lambda} \in \circulo$. Então $0 \neq \formaSimpleticaExtendida{E_{\lambda}}{E_{\overline{\lambda}}} \in i\reta$. De fato, dados $v \in E_{\lambda}$ e $\overline{v} \in E_{\overline{\lambda}}$, a Proposição \ref{proposicao_forma_simpletica_vetor_conjugado} garante que  $\formaSimpleticaExtendida{\overline{v}}{v} \in i\reta$.
	
	Para construir uma aplicação que estenda continuamente o determinante se introduz uma ordenação nos auto-valores de $A$. Sabe-se que um dado $\lambda \in \espectrooperador{A}$ assume um dos valores $|\lambda|\leq 1$ ou $|\lambda|>1$. Além disso, se $\lambda \in \espectrooperador{A}$, então $\overline{\lambda}, \lambda^{-1}, \overline{\lambda}^{-1} \in \espectrooperador{A}$. Com isso, basta analisar os casos em que $|\lambda| \leq 1$, o que implica que $|\overline{\lambda}|\leq 1$, $|\lambda^{-1}|\geq 1$ e $|\overline{\lambda}^{-1}|\geq 1$.
	
	Tome $\lambda_{1}\in \sigma(A)$ tal que $|\lambda_{1}|<1$ e o escreva como $\lambda_{1} = |\lambda_{1}|\exp(i\theta_{\lambda_{1}})$, para algum $\theta_{\lambda_{1}} \in [0,2\pi]$. Tome um outro auto-valor $\lambda_{2}\neq \lambda_{1}$ tal que $|\lambda_{2}|<1$. Se $|\lambda_{1}| < |\lambda_{2}|$ ou  $|\lambda_{1}|= |\lambda_{2}|$ com $\theta_{\lambda} \leq \theta_{\beta}$, diremos que $\lambda_{1} \prec \lambda_{2}$. Repetindo esse procedimento n-vezes teremos a sequência $\lambda_{1} \prec \dots \prec \lambda_{n} \prec  \lambda_{n}^{-1} \prec \dots \prec \lambda_{1}^{-1}$. Essa ordenação será chamada de ordem de primeiro tipo.
	
	Sejam $m(\lambda)$ a multiplicidade do auto-valor $\lambda \in \espectrooperador{A}$  e $\gruposimpleticoespecial{2n} =\{A\in \gruposimpletico{2n} : m(\lambda) = 1,\;\forall \lambda\in \sigma(A) \}$ o conjunto das matrizes simpléticas cujos auto-valores têm multiplicidade igual a 1.
	
	\begin{lema}\label{lema_aplicacao_rho}
		A aplicação $\rho:\gruposimpleticoespecial{2n} \to \circulo$ definida por 
		$$
		\rho(A) = \prod_{j=1}^{n}\frac{\lambda_{j}}{|\lambda_{j}|}
		$$
		é contínua.
	\end{lema}
	\begin{prova}
		Como os auto-valores de $A$ são não-nulos, então podemos escrever $\lambda/|\lambda| = e^{i\theta}$. Com isso, $\rho(A)=\prod \lambda_{j}/|\lambda_{j}| = \prod e^{i\theta_{j}} = e^{i(\theta_{1}+\dots +\theta_{n})} \in \circulo$, onde $\lambda_{j}/|\lambda_{j}| = e^{i\theta_{j}}$ para $1\leq j \leq n$. Logo $\rho$ esta bem-definida.
		
		Considere como a topologia de  $\gruposimpleticoespecial{2n}$ a topologia herdada de $\generalgroupcomplexo{2n}$. Como todos os elementos de $\gruposimpleticoespecial{2n}$ possuem auto-valores distintos, então são diagonalizáveis. Com isso, dados $A, B \in \gruposimpleticoespecial{2n}$ e $\epsilon>0$ tais que $\norma{A-B}< \epsilon$, tem-se $\norma{D_{A}-D_{B}}< \epsilon$, onde $D_{A}, D_{B}$ são as representações diagonais de $A$ e $B$, respectivamente. O que implica que $\max\{|\lambda_{j} - \beta_{j}| \}<\epsilon$, onde $\lambda_{j} \in \espectrooperador{A}$ e $\beta_{j} \in \espectrooperador{B}$ para $1\leq j\leq 2n$.
		
		Dados $A \in \gruposimpleticoespecial{2n}$ e $\epsilon>0$. Então para todo $B\in \gruposimpleticoespecial{2n}$ tal que $\norma{A-B}<\epsilon$, tem-se que
		$$
		|\rho(A) -\rho(B)| =\Big|\prod_{j=1}^{n} \bigparenteses{\frac{\lambda_{j}}{|\lambda_{j}|} - \frac{\beta_{j}}{|\beta_{j}|}} \Big|< \Big|\prod_{j=1}^{n} \bigparenteses{\frac{\lambda_{j}}{\alpha} - \frac{\beta_{j}}{\alpha} }\Big| <\frac{1}{\alpha}\prod_{j=1}^{n} \max|\lambda_{j} - \beta_{j}|=\frac{\epsilon^{n}}{\alpha}<\epsilon,
		$$
		onde $\alpha = min\{|\lambda_{j}|,|\beta_{j}|\}$. Portanto, $\rho$ é contínua.
	\end{prova}		
	
	\begin{lema}
		A aplicação $B:\complexificado{V}\times \complexificado{V} \to \reta$ definida por $B(v, u ) = \parteImaginaria{\formaSimpleticaExtendida{\overline{v}}{u}}$ é uma forma $\reta$-bilinear simétrica e não-degenerada. Além disso
		$$
		B(iv,iu) = B(v,u)\;\text{e }\;B(\overline{v},\overline{u})=-B(v,u).
		$$
	\end{lema}
	\begin{prova} 
		A $\reta$-bilinearidade de $B$ é dada imediatamente pela bilinearidade de $\Omega$. 
		
		A partir do cálculo anterior da decomposição de $\formaSimpleticaExtendida{\overline{u}}{v} $ em suas partes real e imaginária pode-se concluir que $\parteImaginaria{\formaSimpleticaExtendida{\overline{v}}{u} } = \parteImaginaria{\formaSimpleticaExtendida{\overline{u}}{v} }$, o que implica que $B(v,u) = B(u,v)$ e $B$ é simétrica.
		
		Suponha que $B$ seja degenerada, isto é, existe $v \in \complexificado{V}$ com $v\neq 0$ tal que $B(v, u ) = 0$ para todo $u \in \complexificado{V}$. Então
		
		$$
		\begin{aligned}
		\formaSimpleticaExtendida{\overline{v}}{u} &=\parteReal{\formaSimpleticaExtendida{\overline{v}}{u}} +i \parteImaginaria{\formaSimpleticaExtendida{\overline{v}}{u}} 
		\\
		&= \parteImaginaria{-i\formaSimpleticaExtendida{\overline{v}}{u}} +i \parteImaginaria{\formaSimpleticaExtendida{\overline{v}}{u}}
		\\
		&= \underbrace{\parteImaginaria{\formaSimpleticaExtendida{\overline{v}}{-iu}}}_{=0} +i \underbrace{\parteImaginaria{\formaSimpleticaExtendida{\overline{v}}{u}}}_{=0}
		\\
		&=0,
		\end{aligned}
		$$
		o que implica que a forma simplética $\Omega$ é degenerada, contradizendo a hipótese. Portanto $B$ é não-degenerada.
		
		Por fim, temos as identidades $B(iv,iu) = \parteImaginaria{\formaSimpleticaExtendida{-i\overline{v}}{iu}}= \parteImaginaria{\formaSimpleticaExtendida{\overline{v}}{u}}=B(v,u)$ e $B(\overline{v},\overline{u})  = \parteImaginaria{\formaSimpleticaExtendida{v}{\overline{u}}} = \parteImaginaria{-\formaSimpleticaExtendida{\overline{u}}{v}} = -B(u,v) = -B(v,u)$
	\end{prova}
	
	\begin{proposicao}\label{proposicao_forma_quadratica_Q}
		Sejam  $Q: \complexificado{V} \to \reta$ a forma quadrática definida por $Q(v) = B(v,v)$ e $A \in \gruposimpletico{2n}$. Se $\lambda\in \espectrooperador{A} \cap \complementar{\circulo}{\{\pm 1 \} }$, então   $\autoespaco{\lambda} = \autoespaco{\lambda}^{+}\oplus \autoespaco{\lambda}^{-}$, onde $\autoespaco{\lambda}^{\pm}$ são os maiores subespaços do auto-espaço generalizado $\autoespaco{\lambda}$ tais que $Q$ é positiva (negativa) definida.
	\end{proposicao}
	\begin{prova}
		É imediato que $\autoespaco{\lambda}^{+} \cap \autoespaco{\lambda}^{-} = \emptyset$. Além disso, $\autoespaco{\lambda}^{\pm} \subset \autoespaco{\lambda}$ são subespaços vetoriais pois, dados $\beta\in \complexo{}$, $v, u\in \autoespaco{\lambda}^{+}$ tem-se que 
		$$
		\begin{aligned}
		Q(\beta v +u) &= \parteImaginaria{\overline{\beta}\beta\formaSimpleticaExtendida{\overline{v}}{v}+ \overline{\beta}\formaSimpleticaExtendida{\overline{v}}{u}+ \beta\formaSimpleticaExtendida{\overline{u}}{v}+ \formaSimpleticaExtendida{\overline{u}}{u}}
		\\
		&=\parteImaginaria{\overline{\beta}\beta\formaSimpleticaExtendida{\overline{v}}{v}+ \formaSimpleticaExtendida{\overline{u}}{u}}
		\\
		&=|\beta|^{2}Q(v)+Q(u)
		\\
		&\geq0,
		\end{aligned}
		$$
		onde usou-se o fato de que $\beta\formaSimpleticaExtendida{\overline{u}}{v}=-\overline{\beta}\formaSimpleticaExtendida{\overline{v}}{u}$. Portanto $\beta v +u \in \autoespaco{\lambda}^{+}$. Com um argumento análogo mostra-se que $\autoespaco{\lambda}^{-}$ é subespaço vetorial de $\autoespaco{\lambda}$. Note que, $\formaSimpleticaExtendida{\overline{v}}{v}=\formaSimpleticaExtendida{\overline{Av}}{Av} =|\lambda|^{2}\formaSimpleticaExtendida{\overline{v}}{v}$. Se $\lambda \notin \complementar{\circulo}{\{\pm 1 \} }$, então $\Omega|_{\autoespaco{\lambda}\times \autoespaco{\lambda}} = 0$.
	\end{prova}
	
	Denote $dim(\autoespaco{\lambda}^{\pm})$ por $m_{\pm}(\lambda)$, respectivamente.  Seja $\rho: \gruposimpletico{2n}\to \complexo{}$ definida por
	$$
	\rho'(A) = 	(-1)^{m_{0}} \prod_{\lambda \in \sigma(A)\cap \complementar{\circulo}{\{\pm 1 \}  }}\lambda^{m_{+}(\lambda)},
	$$
	onde $m_{0}$ denota o número de pares $(\lambda, \lambda^{-1})$ reais negativos.

	Afirmo que a restrição $\rho|_{\gruposimpleticoespecial{2n}}$ coincide com a aplicação $\rho$ anteriormente definida no Lema \ref{lema_aplicacao_rho}. De fato, tomando $A \in \gruposimpleticoespecial{2n}$ tem-se que $m_{+}(\lambda) = 1$ para todo $\lambda\in \espectrooperador{A}$. Suponha uma ordem de primeiro tipo para os auto-valores de $A$. Com isso, tem-se o produto $\lambda_{1}\dots \lambda_{n} = \lambda_{1}^{m_{+}(\lambda_{1})}\dots \lambda_{n}^{m_{+}(\lambda_{n})}$. Sem perda de generalidade, pode-se supor que $\lambda_{j} \in \reta$ e que $\lambda_{k}\in \circulo$ para $1\leq j \leq r $ e $r +1\leq k \leq n$. Com isso, $\lambda_{j}/|\lambda_{j}|=\pm 1$ e $\lambda_{k}/|\lambda_{k}| = \lambda_{k}$. Seja $m_{0}$ seja a quantidade de auto-valores reais negativos de $A$. Então tem-se o produto 
	$$
	\begin{aligned}
	\rho(A) 
	&=
	\frac{\lambda_{1}}{|\lambda_{1}|}\dots \frac{\lambda_{n}}{|\lambda_{n}|}
	\\
	&= (-1)^{m_{0}}\frac{\lambda_{r+1}}{|\lambda_{r+1}|}^{m_{+}(\lambda_{r+1})} \dots \frac{\lambda_{n}}{|\lambda_{n}|}^{m_{+}(\lambda_{n})}
	\\
	&=(-1)^{m_{0}}\prod_{\lambda\in \espectrooperador{A}\cap\complementar{\circulo}{\reta}}\lambda^{m_{+}(\lambda)}
	\\
	&= \rho'(A).
	\end{aligned}
	$$

	\vermelho{(Ainda não ficou clara o produto sobre todos os autovalores que não pertencem a $\circulo$. Ver novamente o Salamon artigo. Ele afirma algo sobre os auto-valores de primeira ordem e suas multiplicidades, mas não entendi como eles não contribuirão para o produto acima.)}

	A continuidade da aplicação $\rho'$ é uma propriedade necessária para a construção do Índice de Maslov e tal resultado pode se encontrado em \cite{audi_floer_homology}.
	
	\begin{proposicao}
		A aplicação $\rho': \gruposimpletico{2n}\to \circulo$ é contínua e $\rho'|_{\gruposimpleticoespecial{2n}} = \rho$.
	\end{proposicao}
	
	No restante do texto denotaremos $\rho'$ por $\rho$ para simplificar a notação quando não houver ambiguidades.
	
	\subsection*{Demonstração das propriedades de $\rho$}
	
	A construção da aplicação $\rho$ depende intrinsecamente do espaço vetorial simplético $(\complexificado{V}, \Omega)$ adotado. Caso se tenha dois espaços vetoriais simpléticos $(\complexificado{V}_{j}, \Omega_{j})$ para $j\in \{1,2\}$, tem-se as aplicações $\rho_{j}:\gruposimpletico{\complexificado{V}_{j}} \to \circulo$. Com um abuso de notação, identificaremos $\rho_{j}$ com $\rho$, onde sua definição dependerá do contexto.
	
	\begin{prova}
		
		\begin{enumerate}
			\item \label{item_naturalidade_rho} \textbf{\textit{Naturalidade:}} Sejam $T : \complexificado{V}_{1} \to \complexificado{V}_{1}$ um isomorfismo simplético e $A \in \gruposimpletico{\complexificado{V}_{1}}$. Como $det(TAT^{-1}) = det(A)$, então $\espectrooperador{TAT^{-1}} = \espectrooperador{A}$. Sejam $\autoespaco{\lambda}^{TAT^{-1}}$, $\autoespaco{\lambda}^{A}$ e auto-espaços generalizados de $TAT^{-1}$ e $A$, respectivamente. Afirmo que $\autoespaco{\lambda}^{TAT^{-1}} = T(\autoespaco{\lambda}^{A})$. De fato, tomando $v \in \autoespaco{\lambda}^{TAT^{-1}}$, então $TAT^{-1}v = \lambda v$, o que implica que $AT^{-1}v = \lambda T^{-1}v$, logo $T^{-1}v \in \autoespaco{\lambda}^{A}$ e $T^{-1}(\autoespaco{\lambda}^{TAT^{-1}}) \subseteq \autoespaco{\lambda}^{A}$. Como $T$ é um isomorfismo, então $\autoespaco{\lambda}^{TAT^{-1}} = TT^{-1}(\autoespaco{\lambda}^{TAT^{-1}}) \subseteq T(\autoespaco{\lambda}^{A})$. Por outro lado, suponha que $v\in \autoespaco{\lambda}^{A}$. Como $T$ é um isomorfimo, então existe $u \in \complexificado{V}_{1}$ tal que $v=T^{-1}u$. Com isso, $AT^{-1}u = \lambda T^{-1}u$ e $ TAT^{-1}u =\lambda u$. Logo $Tv=u \in \autoespaco{\lambda}^{TAT^{-1}}$ e $T(\autoespaco{\lambda}^{A}) \subseteq \autoespaco{\lambda}^{TAT^{-1}}$. Portanto, $\autoespaco{\lambda}^{TAT^{-1}}=T(\autoespaco{\lambda}^{A}) $ e $dim(\autoespaco{\lambda}^{TAT^{-1}})=dim(T(\autoespaco{\lambda}^{A}) )=dim(\autoespaco{\lambda}^{A})$.
			
			Por fim, como $A,T\in \gruposimpletico{2n}$, então $TAT^{-1}\in \gruposimpletico{2n}$, e pode-se afirmar que $Q(TAT^{-1}v) = Q(v)$ para todo $v \in \complexificado{V}_{1}$. Logo  $m^{TAT^{-1}}_{\pm}(\lambda)=m^{A}_{\pm}(\lambda)$, $m_{0}^{TAT^{-1}}=m_{0}^{A}$, e com isso, $\rho(TAT^{-1})=\rho(A)$.
			
			\item \textbf{\textit{Produto:}} Sejam $(\complexificado{V}, \Omega) = (\complexificado{V}_{1}\times \complexificado{V}_{2}, \Omega_{1}\times \Omega_{2})$ um $2k$-espaço vetorial simplético, onde $\complexificado{V}_{1}, \complexificado{V}_{2}$ são $2n_{1}$ e $2n_{2}$-espaços vetoriais simpléticos, respectivamente, e $k=2n_{1}+2n_{2}$. Dado $A \in \gruposimpletico{\complexificado{V}}$ tem-se que $Av = (A_{1}v_{1}, A_{2}v_{2})$, onde $v=(v_{1}, v_{2}) \in \complexificado{V}$, $A_{1}\in \gruposimpletico{\complexificado{V}_{1}}$ e $A_{2}\in \gruposimpletico{\complexificado{V}_{2}}$. Com isso, um auto-espaço generalizado $E_{\lambda}$ de $A$ pode ser escrito como $E_{\lambda} = E_{\lambda_{1}}\times E_{\lambda_{2}}$, onde $E_{\lambda_{1}}$, $E_{\lambda_{2}}$ são auto-espaços generalizados de $A_{1}$ e $A_{2}$, respectivamente. Desse modo, a aplicação $\hat{\rho}: \gruposimpletico{\complexificado{V}} \to \circulo$ satisfaz $\hat{\rho}(A) = \rho(A_{1})\rho(A_{2})$.
			
			\item \textbf{\textit{Deteminante:}} Sejam $A \in \mathcal{U}$ e $U \in \matrizunitaria{n}$ identificados pelo isomorfismo descrito no Capítulo \ref{capitulo_grupo_simpletico} dado por
			$$
			\begin{aligned}
				\real{2n} \ni (v_{(1)},v_{(2)}) &\mapsto v_{(1)}+iv_{(2)} \in \complexo{n},
				\\
				\mathcal{U} \ni 
				\left(
					\begin{array}{cc}
					B & -C
					\\
					C & B
					\end{array}
				\right) 
				& \mapsto B+iC \in \matrizunitaria{n}.
			\end{aligned}
			$$
			Se $(v_{(1)},v_{(2)}) \in \real{2n}$ é um auto-vetor de $A$ com auto-valor $\lambda$ em $\complexo{}$, então tem-se que $A(v_{(1)},v_{(2)})=\lambda (v_{(1)},v_{(2)}) $ implica $U(v_{(1)}+iv_{(2)}) = \lambda(v_{(1)}+iv_{(2)})$. Logo $\espectrooperador{A} \subseteq \espectrooperador{U}$. Analogamente, a identificação dos auto-valores de $U$ com os auto-valores de $A$ em $\complexo{}$ implica que $\espectrooperador{U} \subseteq\espectrooperador{A}$. Portanto $\espectrooperador{A} =\espectrooperador{U}$ em $\complexo{}$.
			
			Sabedo-se que $\estruturacomplexa: \real{2n} \to \real{2n}$ é um endomorfismo, que $A\estruturacomplexa = \estruturacomplexa A$ e que ambos $A$ e $\estruturacomplexa$ são diagonalizáveis (matrizes normais são diagonalizáveis, veja o Lema \ref{lema_caracterizacao_matriz_normal}). Além disso, $\espectrooperador{\estruturacomplexa} = \{i,-i\}$, $\autoespaco{i}^{\estruturacomplexa} = \{u\textbf{e} +iu\textbf{f}: u \in \complexo{n}\}$, $\autoespaco{ -i}^{\estruturacomplexa} = \{u\textbf{e} -iu\textbf{f}: u \in \complexo{n}\}$ e $dim(\autoespaco{i}^{\estruturacomplexa} )=dim(\autoespaco{-i}^{\estruturacomplexa} )=n$, portanto $\complexificado{V} = \autoespaco{i}^{\estruturacomplexa}\oplus \autoespaco{-i}^{\estruturacomplexa}$. 
			Tomando $v \in \autoespaco{i}^{\estruturacomplexa}$, tem-se que 
			$$
			\begin{aligned}
				\formaSimpleticaExtendida{\overline{v}}{v} 
				&= 2i\parteImaginaria{\produtointerno{\overline{v}_{(1)}}{v_{(2)}}}
				\\
				&= 2i\parteImaginaria{\produtointerno{\overline{v}_{(1)}}{iv_{(1)}}}
				\\
				&=-2\parteImaginaria{\norma{v_{(1)}}^{2}}
				\\
				&=-2\norma{v_{(1)}}^{2} \leq 0.
			\end{aligned}
			$$
			
			Portanto a restrição da forma quadrática $Q$, dada pela Definição \ref{proposicao_forma_quadratica_Q}, ao auto-espaço $\autoespaco{ i}^{\estruturacomplexa}$, é negativa definida. Analogamente pode-se verificar que a restrição de $Q$ ao auto-espaço $\autoespaco{ -i}^{\estruturacomplexa}$ é positiva definida. Com isso, pode-se decompor o auto-espaço $\autoespaco{\lambda} = \autoespaco{\lambda}^{+} \oplus \autoespaco{\lambda}^{-}$, onde $\autoespaco{\lambda}^{+} = \autoespaco{\lambda}\cap \autoespaco{- i}^{\estruturacomplexa} \neq \emptyset$ e $\autoespaco{\lambda}^{-} = \autoespaco{\lambda}\cap \autoespaco{ i}^{\estruturacomplexa} \neq \emptyset$. Pela identificação anterior tem-se
			$$
			\autoespaco{\lambda}^{+}=\{v_{(1)}\textbf{e}-iv_{(1)}\textbf{f}\in \autoespaco{\lambda}\} = \{v_{(1)} \in \complexo{n} : Uv_{(1)}=\lambda v_{(1)} \} = \autoespaco{\lambda}^{U}.
			$$
			
			Logo $m_{+}(\lambda) = m^{U}(\lambda)$.
			
			A aplicação $\rho|_{\mathcal{U}}: \mathcal{U} \to \circulo$ quando avaliada em $A\in \mathcal{U}$ é dada em termos do produto dos auto-valores $\lambda \in \espectrooperador{A}$ e suas multiplicidades $m_{+}(\lambda)$ e $m_{0}$. Como $\espectrooperador{A} = \espectrooperador{U}$ e $m_{+}(\lambda)  = m_{\lambda}^{U}$, então as multiplicidades dos auto-valores são as mesmas e $\rho(A)=det(U)$.

			Como $\matrizSimpleticaOrtogonal \subset \gruposimpletico{2n}$, então $\grupofundamental{\matrizSimpleticaOrtogonal} \subset \grupofundamental{\gruposimpletico{2n}}$. Por outro lado, a Observação \ref{observacao_quociente_grupo_simpletico_contratil} afirma que $\matrizSimpleticaOrtogonal$ é um retrato por deformação de $\gruposimpletico{2n}$. Com isso, todo elemento de $\grupofundamental{\gruposimpletico{2n}} $ é um elemento de $\grupofundamental{\matrizSimpleticaOrtogonal}$, $\grupofundamental{\gruposimpletico{2n}}\subset\grupofundamental{\matrizSimpleticaOrtogonal}$. Portanto $\grupofundamental{\gruposimpletico{2n}}=\grupofundamental{\matrizSimpleticaOrtogonal}$.
			
			Mostrou-se que $\rho|_{\matrizSimpleticaOrtogonal} =det$, o que implica que $(\rho|_{\matrizSimpleticaOrtogonal})_{*} = Det_{*}$, onde $Det_{*}:\grupofundamental{\matrizunitaria{n}} \to \grupofundamental{\circulo}$ é o isomorfismo dado pelo Lema \ref{lema_isomorfismo_grupo_fundamental_Un}. Como $\grupofundamental{\gruposimpletico{2n}}=\grupofundamental{\matrizSimpleticaOrtogonal}$, então $\rho_{*}=(\rho|_{\matrizSimpleticaOrtogonal})_{*}$ e $\rho_{*}: \grupofundamental{\gruposimpletico{2n}} \to \grupofundamental{\circulo}$ é um isomorfismo.
			
			\item \textbf{\textit{Normalização:}} Seja $A\in \gruposimpletico{2n}$ com $\sigma(A)\cap \circulo = \emptyset$. Então pela própria definição a aplicação $\rho$ não terá produto dos auto-valores, portanto $\rho(A) = (-1)^{m_{0}} = \pm 1$. 
			
			\item \textbf{\textit{Inversa:}} 	Pelo Lema \ref{lema_caracterizacao_espectro_semelhante}, tem-se que $\espectrooperador{A^{-1}} = \espectrooperador{A}$. Sejam $\autoespaco{\lambda}$ e $F_{\lambda^{-1}}$ auto-espaços generalizados de $A$ e $A^{-1}$, associados aos auto-valores $\lambda$ e $\lambda^{-1}$, respectivamente. É imediato que $\autoespaco{\lambda} = F_{\lambda^{-1}}$, o que implica que $\autoespaco{\lambda}^{+} = F_{\lambda}^{+}$ e que $m^{A}_{+}(\lambda) = m^{A^{-1}}_{+}(\lambda^{-1})$. Então
			$$
			\begin{aligned}
			\rho(A^{-1})
			&= (-1)^{m_{0}}\prod (\lambda^{-1})^{m_{+}^{A^{-1}}(\lambda^{-1})}
			\\
			&=(-1)^{-m_{0}}\prod \lambda^{-m_{+}^{A}(\lambda)}
			\\
			&=\bigparenteses{(-1)^{m_{0}}\prod \lambda^{m_{+}^{A}(\lambda)}}^{-1}
			\\
			&=(\rho(A)) ^{-1}.
			\end{aligned}
			$$
		\end{enumerate}
	\end{prova}
	
	\section{$\gruposimpleticonaodegenerado{*}$ e sua topologia}
	
	Defina os conjuntos $\gruposimpleticonaodegenerado{*} = \gruposimpleticonaodegenerado{+} \cup \gruposimpleticonaodegenerado{-}$ e  $\Sigma = \{ A \in \gruposimpletico{2n}: det(Id-A)= 0 \}$, onde $\gruposimpleticonaodegenerado{+} = \{ A \in \gruposimpletico{2n}: det(Id-A)> 0 \}$ e $\gruposimpleticonaodegenerado{-} = \{ A \in \gruposimpletico{2n}: det(Id-A)< 0 \}$. Note que o complementar de $\gruposimpleticonaodegenerado{*} \subset \gruposimpletico{2n}$ é $\Sigma$.
	
	O seguinte lema será usado nas próximas demostrações e sua demonstração pode ser encontrada em $\cite{audi_floer_homology}$.
	
	\begin{lema}\label{lema_conectividade_grupo_simlpetico_nao_degenerado}
		Seja $A\in \gruposimpleticonaodegenerado{*}$. Então existe um caminho contínuo em $\gruposimpleticonaodegenerado{*}$ conectando $A$ a uma matriz $B \in \gruposimpleticonaodegenerado{*}$ cujos  auto-valores são todos distintos com uma das seguintes condições: 1) se $A\in \gruposimpleticonaodegenerado{+}$, então $B$ não possui auto-valores reais positivos ou 2) se $A\in \gruposimpleticonaodegenerado{-}$, então $B$ possui apenas 2 auto-valores reais positivos.
	\end{lema}
	
	\begin{lema}
		$\gruposimpleticonaodegenerado{*} = \gruposimpleticonaodegenerado{+}\cup \gruposimpleticonaodegenerado{-}$, onde $\gruposimpleticonaodegenerado{\pm}$ são as duas componentes conexas por caminhos, portanto conexas.
	\end{lema}
	\begin{prova}
		Sejam $(\complexificado{V}, \Omega)$ um 2n-espaço vetorial simplético e $\{\textbf{e}, \textbf{f}\}$ a base simplética de $\complexificado{V}$. Defina $W: \complementar{\reta}{\{0,1\}} \to \generalgroupcomplexo{2n}$ por
		$$
		W(a) = diag\{\underbrace{a, -1, \dots, -1}_{n}, \underbrace{a^{-1}, -1, \dots , -1}_{n}\}.
		$$
		
		Afirmo que $W(\complementar{\reta}{\{0,1\}}) \subset \gruposimpleticonaodegenerado{*}$. De fato, $det(Id - W(a)) = (1-a)(1-a^{-1})4^{n-1} \neq 0$, pois $a\neq 1$. Tomando $v,u \in \complexificado{V}$ tais que $v=v_{(1)}\textbf{e}+ v_{(2)}\textbf{f}$ e $u=u_{(1)}\textbf{e}+ u_{(2)}\textbf{f}$, temos 
		$$
		\begin{aligned}	
		W(a)v &= (av_{1}, -v_{2}, \dots, -v_{n}, a^{-1}v_{n+1}, -v_{n+2}, \dots, -v_{2n})
		\\
		&= av_{1}e_{1}  -\sum_{j=2}^{n} v_{j}e_{j} + a^{-1}v_{n+1}f_{1} -\sum_{j=2}^{n} v_{n+j}f_{j},
		\\
		\formaSimpleticaExtendida{W(a)v}{W(a)u} 
		&=v_{1}u_{n+1}\formaSimpleticaExtendida{e_{1}}{f_{1}} + \sum_{j=2}^{n}v_{j}u_{n+j}\formaSimpleticaExtendida{e_{j}}{f_{j}}
		\\
		&+v_{n+1}u_{1}\formaSimpleticaExtendida{f_{1}}{e_{1}} + \sum_{j=2}^{n} v_{n+j}u_{j}\formaSimpleticaExtendida{f_{j}}{e_{j}}
		\\
		&=\sum_{j=1}^{n}\bigparenteses{v_{j}u_{n+j} - 	v_{n+j}u_{j}}
		\\
		&= \produtointerno{v_{(1)}}{u_{(2)}} - \produtointerno{v_{(2)}}{u_{(1)}}
		\\
		&= \formaSimpleticaExtendida{v}{u}.
		\end{aligned}
		$$ 
		
		Portanto, $W(a) \in \gruposimpletico{2n}$ e $W(\complementar{\reta}{\{0,1\}}) \subset \gruposimpleticonaodegenerado{*}$.
		
		Defina $W^{+} = W(-1)$ e $W^{-} = W(2)$. Temos que $det(Id - W^{+}) = 4^{n}>0$  e $det(Id - W^{-}) = -4^{n-1}/2 <0$, logo $W^{+}\in \gruposimpleticonaodegenerado{+}$ e $W^{-}\in \gruposimpleticonaodegenerado{-}$. 
		
		Seja $A \in \gruposimpleticonaodegenerado{*}$. Pelo Lema $\ref{lema_conectividade_grupo_simlpetico_nao_degenerado}$ existe $B \in \gruposimpleticonaodegenerado{*}$ com todos os seus auto-valores distintos  e um caminho contínuo $\alpha:[0,1] \to \gruposimpleticonaodegenerado{*}$ tal que $\alpha(0)=A$ e $\alpha(1) = B$. Seja $\{v_{\lambda_{1}}, \dots , v_{\lambda_{2n}} \}$ uma base de auto-valores de $B$ 
		
		Para cada $\lambda \in \complementar{\espectrooperador{B}}{\real{+}}$ considere um caminho contínuo $\gamma:[0,1]\to \reta$ tal que $\gamma(0) = \lambda$ e $\gamma(1) = -1$ tal que $1 \notin \gamma([0,1])$. Seja $\Gamma:[0,1 ]\to \gruposimpleticonaodegenerado{*}$ o caminho contínuo definido por
		$$
		\funcaocond{\Gamma(s)v_{\lambda}}{\gamma(s)v_{\lambda}}{\lambda \notin \real{+}}{\lambda v_{\lambda} }{\lambda \in \real{+}}
		$$
		
		com $\Gamma(0) = B$. Com um abuso de notação, escreva $\Gamma(s)v_{\lambda}=\gamma(s)v_{\lambda}$. Para os auto-valores $\overline{\lambda}$ e $1/\lambda$ tem-se $\overline{\gamma}(s)$ e $1/\gamma(s)$, respectivamente. É imediato da definição que $\Gamma(s)(E_{\lambda}) = E_{\lambda}$. Afirmo que $\Gamma(s) \in \gruposimpleticonaodegenerado{*}$ para todo $s\in \intervalo$. De fato, supondo $\lambda,\lambda' \in \sigma(B)$ tais que $\lambda\lambda' \neq 1$ tem-se que $\formaSimpleticaExtendida{\Gamma(s)v_{\lambda}}{\Gamma(s)v_{\lambda'}}= \gamma(s)\gamma'(s)\formaSimpleticaExtendida{v_{\lambda}}{v_{\lambda'}} = 0$, pela $\Omega$-ortoganalidade (veja o Lema \ref{lema_auto_espaco_grupo_simpletico}). No caso em que $\lambda\lambda'=1$ teremos $\lambda'=\lambda^{-1}$, o que implica em $\gamma'(s) = 1/\gamma(s)$, logo $\omega(\Gamma(s)v_{\lambda},\Gamma(s)v_{\lambda'}) = \omega(v_{\lambda},v_{\lambda'})$. Por fim, caso $\lambda = \lambda'$, teremos $\omega(\Gamma(s)v_{\lambda},\Gamma(s)v_{\lambda}) = \gamma^{2}(s)\omega(v_{\lambda},v_{\lambda}) =0$. Portanto, $\Gamma^{*}(s)\omega = \omega$ e $\Gamma(s) \in \gruposimpletico{2n}$. Como $1 \notin \gamma(\intervalo)$, então $det(Id - \Gamma(s))\neq 0 $ e $\Gamma(s)\in \gruposimpleticonaodegenerado{*}$ para todo $s\in \intervalo$.
		
		Seja $\alpha:\intervalo\to \gruposimpleticonaodegenerado{*}$ um caminho contínuo conectando $A$ a $B$, cuja exitência é garantida pelo Lema $\ref{lema_conectividade_grupo_simlpetico_nao_degenerado}$.
		Por definição de $\gruposimpleticonaodegenerado{*}$ temos que $\gruposimpleticonaodegenerado{+} \cap \gruposimpleticonaodegenerado{-}=\emptyset$, portanto está em apenas uma dessas componentes, isto é, $\alpha([0,1]) \subset \gruposimpleticonaodegenerado{+}$, caso $A \in \gruposimpleticonaodegenerado{+}$, ou $\alpha([0,1]) \subset \gruposimpleticonaodegenerado{-}$, caso $A\in \gruposimpleticonaodegenerado{-}$.
		
		Vamos mostrar que ambas as componentes são conexas por caminhos.
		\begin{enumerate}
			\item Se $\alpha([0,1]) \subset \gruposimpleticonaodegenerado{+}$, então $B$ não possui auto-valores reais positivos. Pela hipótese, temos que $\Gamma(0) = B$ e $\gamma_{\lambda_{j}}(1) = -1$ para $1\leq j\leq 2n$. Com isso, $\Gamma(1) = diag \{-1, \dots, -1\} = W^{+}$. Portanto, $\Gamma$ é um caminho contínuo que conecta $B$ a $W^{+}$. A justaposição $\alpha * \Gamma:\intervalo \to \gruposimpleticonaodegenerado{+}$
			é um caminho contínuo conectando $A$ a $W^{+}$. Tome $A' \in \gruposimpleticonaodegenerado{+}$ tal que $A'\neq A$. Então com uma construção análoga a anterior, conecta-se $A'$ a $W^{+}$ por um caminho contínuo em $\gruposimpleticonaodegenerado{+}$. Tem-se caminhos conectado $A$ a $W^{+}$ e $A'$ a $W^{+}$, respectivamente. Logo conecta-se $A$ a $A'$ através de um caminho contínuo. Portanto $\gruposimpleticonaodegenerado{+}$ é conexo por caminhos, logo é conexo.
			
			\item Se $\alpha([0,1]) \subset \gruposimpleticonaodegenerado{-}$, então $B$ possui apenas 2 auto-valores positivos. Supondo que $\lambda \in \sigma(A)$ tal que $\lambda > 0$, então $\lambda^{-1} > 0$, pois os auto-valores de $A$ são determinados aos pares $\lambda, \lambda^{-1}$. Reordene a base de auto-valores de $B$ tal que $\lambda_{1}=\lambda$ e $\lambda_{n+1}=\lambda^{-1}$ sejam esses auto-valores positivos. Assim, tem-se que $\Gamma(0) = B$ e $\Gamma(1) = diag\{\lambda, -1, \dots, -1, \lambda^{-1}, -1, \dots, -1\}$. Seja $\Gamma': \intervalo \to \gruposimpleticonaodegenerado{-}$ o caminho definido por $\Gamma'(t)=diag\{ \theta(t), -1,\dots, -1, 1/\theta(t) ,-1,\dots, -1\}$,
			onde $\theta(t)=(2- \lambda)t + \lambda$. Note que $\Gamma'$ é um caminho contínuo que conecta $\Gamma(1)$ a $W^{-}$. Com isso, a justaposição $\Gamma'*\Gamma$ é um caminho conectando $B$ a $W^{-}$ e a justaposição $\alpha * \Gamma'*\Gamma: \intervalo \to \gruposimpleticonaodegenerado{-}$ é um caminho contínuo conectando $A$ a $W^{-}$. Com uma construção análoga a do item anterior se pode-se concluir que qualquer $A' \in \gruposimpleticonaodegenerado{-}$ pode ser conectado a $A$ por um caminho contínuo que passa por $W^{-}$. Portanto, $\gruposimpleticonaodegenerado{-}$ é conexo por caminhos, logo é conexo.
		\end{enumerate}
		
		Afirmo que $\gruposimpleticonaodegenerado{\pm}$ são abertos. De fato, defina $g:\gruposimpletico{2n}\to \reta$ por $g(A) = det(Id-A)$. Como o determinante é uma função contínua, então $g$ é contínua. Suponha que $A \in \gruposimpleticonaodegenerado{+}$. Então $g(A)>0$ e existe $V=(a,b)$, onde $a,b>0$, tal que $g(A)\in V$. Pela continuidade $g^{-1}(V)\subset \gruposimpleticonaodegenerado{+}$. Portanto $A \in \gruposimpleticonaodegenerado{+}$ é um ponto interno e $\gruposimpleticonaodegenerado{+}$ é um aberto. Um argumento análogo mostra que $\gruposimpleticonaodegenerado{-}$ é aberto.
	\end{prova}
	
	\begin{corolario}\label{corolario_homomorfismo_trivial_grupos_simpletico}
		A inclusão $\gruposimpleticonaodegenerado{*} \hookrightarrow \gruposimpletico{2n}$ induz o homomorfismo trivial entre os grupos fundamentais
	\end{corolario}
	\begin{prova}
		Sabe-se que $\rho(\gruposimpletico{2n})=\circulo$. Então dado $A\in \gruposimpletico{2n}$ tem-se $\rho(A) = \exp(i\theta)$, para algum $\theta\in [0,2\pi]$. Sejam $\gamma\in \caminhosfechados{\gruposimpleticonaodegenerado{+}}$ e $\alpha_{j}:\gruposimpleticonaodegenerado{+}\to [0,2\pi]$ funções contínuas para $1\leq j\leq n$ tais que
		$$
		\rho(A)=\exp\bigparenteses{i\sum_{j=1}^{n}\alpha_{j}(A)},
		$$
		onde $\alpha_{j}$ é definida do seguinte modo: tome os números reais $\alpha_{j}(A)\in \reta$ tais que $0\leq \alpha_{1}(A)\leq \dots\leq \alpha_{n}(A)\leq 2\pi$, onde $\exp(i\alpha_{j}(A)) = \lambda_{j}/|\lambda_{j}|$ e $\lambda_{j}\in \espectrooperador{A}$.
		 
		Note que a composição $\alpha\circ\gamma:\intervalo\to [0,2\pi]$ são funções contínuas e periódicas, pois é a composição de funções com essas propriedades. Com isso, o caminho fechado $\rho\circ\gamma:\intervalo \to \circulo$, dado por $\rho(\gamma(t)) = \exp(i\sum_{j}(\alpha_{j}\circ\gamma)(t)) \in \circulo$, é contrátil. Seja $i_{*}:\grupofundamental{\gruposimpleticonaodegenerado{+}} \hookrightarrow \grupofundamental{\gruposimpletico{2n}}$ o homomorfismo induzido pela inclusão $i:\gruposimpleticonaodegenerado{+}\hookrightarrow \gruposimpletico{2n}$. Como todo caminho fechado $\gamma\in \caminhosfechados{\gruposimpleticonaodegenerado{+}}$ é contrátil, então $i_{*}(\classe{\gamma}) = \classe{i\circ \gamma} = \classe{0}$. Com uma argumento análogo mostra-se que $i_{*}(\classe{\beta}) = \classe{i\circ \beta} = \classe{0}$ para todo $\beta \in \gruposimpleticonaodegenerado{-}$. Portanto a inclusão $i_{*}(\grupofundamental{\gruposimpleticonaodegenerado{*}}) \hookrightarrow \grupofundamental{\gruposimpletico{2n}}$ é o homomorfismo trivial.
	\end{prova}
		
	
	\section{Índice de Maslov $\mu : \caminhosespeciais{\gruposimpletico{2n}} \to \inteiros$}\label{secao_indice_maslov}
	
	Na seção anterior mostrou-se que $\gruposimpleticonaodegenerado{\pm}$ são as duas componentes conexas por caminhos de $\gruposimpleticonaodegenerado{*}$, logo $\gruposimpletico{2n}=\gruposimpleticonaodegenerado{+}\cup\gruposimpleticonaodegenerado{-}\cup \Sigma$.
	
	Seja $\caminhos^{\pm}(\gruposimpleticonaodegenerado{*})$ o conjunto de todos os caminhos contínuos $\alpha:\intervalo \to \gruposimpleticonaodegenerado{*}$, onde $\alpha(1) = W^{\pm}$, isto é, $\alpha(\intervalo) \subset \gruposimpleticonaodegenerado{+}$ ou $\alpha(\intervalo) \subset \gruposimpleticonaodegenerado{-}$. Defina $\caminhosespeciais{\gruposimpletico{2n}}$ como sendo o conjunto de todos os caminhos $\psi=\gamma*\phi:\intervalo\to\gruposimpletico{2n}$, onde $\gamma(0)=Id $, $\gamma(1) \in \gruposimpleticonaodegenerado{*}$ e $\phi \in \caminhos^{\pm}(\gruposimpleticonaodegenerado{*})$, onde $\psi(1)=\phi(0)$.
	
	\begin{observacao}
		Note que, um dado $\psi\in \caminhosespeciais{\gruposimpletico{2n}}$ é um caminho contínuo, não necessariamente fechado e que conecta a identidade $Id\in\Sigma$ a um elemento de $\gruposimpleticonaodegenerado{\pm}$. Como $\Sigma$ é conexo por caminhos, poderia-se ter escolhido qualquer outro elemento desse conjunto e distinto de $Id$.
	\end{observacao}
	
	\begin{definicao}
		(Índice de Maslov) Sejam $\exp:\real{} \to \circulo \subset \mathbb{C}$ a aplicação exponencial. Para cada caminho $\psi \in \caminhosespeciais{\gruposimpletico{2n}}$ existe um único levantamento $\alpha_{\psi}:[0,1] \to \real{}$ do caminho $\rho_{\psi} = \rho \circ \psi:\intervalo\to \circulo$ que faz o diagrama abaixo comutar
		$$
		\xymatrix{
			& & \real{}\ar[d]\ar[d]^{\text{exp}}
			\\
			[0,1 ]\ar[urr]^{\alpha_{\psi}} \ar[r]_{\psi} & Sp(2n) \ar[r]_{\rho} & S^{1}
		}
		$$	
		O índice de Maslov é definido por $\mu(\psi)= 2deg(\rho_{\psi}) = (\alpha_{\psi}(0) - \alpha_{\psi}(1))/\pi$.
	\end{definicao}
	
	\begin{observacao}
		Seja $\psi:\intervalo\to \gruposimpletico{2n}$ um caminho tal que $\psi(0)=Id$ e $\psi(1)=W^{-}$. Com isso, tem-se $\rho_{\psi}(0) =\rho(W^{-}) = (-1)^{n-1}$. Se $n\in 2\mathbb{N}$, então $\rho_{\psi}(0) =1$ e $\rho_{\psi}(1)=-1$. Portanto a imagem de $\rho_{\psi}$ envolve ao menos a metade do cícurlo, isto é, $\alpha_{\psi}(1)-\alpha_{\psi}(0) = k\pi$, para algum $k\in \inteiros$. Por esse fato defini-se o índice de Maslov com sendo $2deg(\rho_{\psi})$.
	\end{observacao}
	
	Os dois próximos resultados são importantes propriedades do índice de Maslov que são úteis no cálculo do mesmo.

	\begin{lema}
		Se $\psi, \psi'\in \caminhosespeciais{\gruposimpletico{2n}}$ tais que $\psi=\gamma*\alpha$ e $\psi'=\gamma*\alpha'$, então $\mu(\psi) = \mu(\psi')$.
	\end{lema}
	\begin{prova}
		Pelo Corolário \ref{corolario_homomorfismo_trivial_grupos_simpletico} tem-se que $\grupofundamental{\gruposimpleticonaodegenerado{*}} = \{\classe{0}\}$, logo $\alpha, \alpha'\in \classe{0}$ e $\alpha\sim \alpha'$. Com isso, $\psi\sim \psi'$ e $\mu(\psi) = 2deg(\rho_{\psi})=2deg(\rho_{\psi'})=\mu(\psi')$, pois caminhos homotopicamente equivalentes possuem o mesmo grau.
	\end{prova}
	
	\begin{observacao}
		O lema anterior afirma que o índice de Maslov de um caminhos $\psi\in \caminhosespeciais{\gruposimpletico{2n}}$ não depende da escolha de $\alpha:\intervalo\to \gruposimpleticonaodegenerado{*}$.
	\end{observacao}
	
	\begin{lema}\label{lema_inversa_caminho_especial}
		Se $\psi \in \caminhosespeciais{\gruposimpletico{2n}}$, então $\psi^{-1} \in \caminhosespeciais{\gruposimpletico{2n}}$ e é dado por $\psi^{-1}(t)=(\psi(t))^{-1}$ para todo $t\in \intervalo$. Além disso, $\rho(\psi^{-1}(t)) = (\rho(\psi(t)))^{-1}$.
	\end{lema}
	\begin{prova}
		Tem-se que $(\psi^{-1}\psi)(t)=Id$, o que implica em $\psi^{-1}(t)=(\psi(t))^{-1}$ para todo $t\in \intervalo$. Como a inversa de toda transformação simplética é uma transformação simplética, então $\psi^{-1}(\intervalo) \subset \gruposimpletico{2n}$. Pelo Item \ref{item_inversa_rho} do Teorema \ref{teorema_aplicacao_rho}, tem-se que $\rho(\psi^{-1}(t))= (\rho(\psi(t)))^{-1}$ para todo $t\in \intervalo$.
	\end{prova}
	
	
	\begin{teorema}\label{teorema_indice_maslov}
		Seja $\psi=\gamma*\alpha \in \caminhosespeciais{\gruposimpletico{2n}} $. Então o índice de Maslov satisfaz as seguintes propriedades:
		\begin{enumerate}
			\item $\mu(\psi) \in \inteiros$.
			
			\item \textbf{(Naturalidade)} Para todo $\phi\in \caminhossempontobase{\gruposimpletico{2n}}$ tem-se que $\mu(\phi\psi\phi^{-1}) = \mu(\psi)$.
			
			\item \textbf{(Homotopia)} \label{homotopia_caminhos_teorema_indice_maslov} Dois caminhos $\psi_{1}, \psi_{2}\in \caminhosespeciais{\gruposimpletico{2n}} $ com $\psi_{1}(0) = \psi_{2}(0)$ e $\psi_{1}(1) = \psi_{2}(1)$ são homotopicamente equivalentes se, e somente se, $\mu(\psi_{1}) = \mu(\psi_{2})$.
			
			\item \textbf{(Nullidade)} se $\espectrooperador{\psi(s)}\cap \circulo = \emptyset$ para todo $s\in \intervalo$, então $\mu(\psi) = 0$.
			
			\item \textbf{(Produto)} Se $n=p+q$, então identifica-se $\gruposimpletico{2p}\times \gruposimpletico{2q}$ com um subgrupo de $\gruposimpletico{2n}$ e $\mu:\caminhosespeciais{\gruposimpletico{2p}\times \gruposimpletico{2p}} \to \inteiros$ é dado por $\mu((\psi_{p}, \psi_{q})) = \mu(\psi_{p})+\mu(\psi_{q})$.
			
			\item \textbf{(Determinante)} $sign(det(Id-\gamma(1))) = (-1)^{\mu(\psi)-n}$.
			
			\item \textbf{(Assinatura)} Se $S \in GL(2n)$ é uma matriz simétrica com a norma $||S|| < 2\pi$ e se $\psi(t) = exp(t\estruturacomplexa S)$, então 
			$$
			\mu(\psi) = Ind(S) - n,
			$$
			onde $Ind(S)$ é a multiplicidade dos auto-valores negativos de $S$.
			
			\item \textbf{(Inversa)} $\mu(\psi^{-1}) = -\mu(\psi)$, onde $\psi^{-1} $ é o caminho contínuo tal que $\psi^{-1}*\psi = Id$.
			
		\end{enumerate}
	\end{teorema}
	\begin{prova}
		\begin{enumerate}
			\item Para todo $\psi \in \caminhosespeciais{\gruposimpletico{2n}}$ tem-se que $\rho_{\psi}(0) = \rho(\psi(0)) = 1$ e $\rho_{\psi}(1) = \rho(\psi(1)) = \rho(W^{\pm})$, onde $\rho(W^{+})=(-1)^{n}$ e $\rho(W^{-})=(-1)^{n-1}$, o que implica em $\rho(W^{\pm}) \in \{ -1,1\}$. Sabe-se que $\rho_{\psi}(t) = \exp(i\alpha_{\psi}(t))$. Afirmo que $\alpha_{\psi}(0)=0$ e $\alpha_{\psi}(1)=k\pi$. De fato, para $t=0$ é imediato que $\alpha_{\psi}(0)=0$. Para $t=1$ tem-se que $\rho(W^{\pm}) = \pm 1$, o que implica que $\alpha(1) = k\pi$, para algum $k\in \inteiros$. Logo $\mu(\psi) = (k\pi-0)/\pi \in \inteiros$.
			
			\item \textbf{(Naturalidade)} Pelo Item \ref{item_naturalidade_rho} do Teorema \ref{teorema_aplicacao_rho} tem-se que  $\rho(\phi(t)\psi(t)\phi^{-1}(t)) = \rho(\psi(t))$ para todo $t\in \intervalo$, logo $\rho_{\phi\psi\phi^{-1}} = \rho_{\psi}$. Com isso, $\mu(\phi\psi\phi^{-1})= 2deg(\rho_{\phi\psi\phi^{-1}} )=2deg(\rho_{\psi})=\mu(\psi)$.
			
			\item \textbf{(Homotopia)} Sejam $\psi_{1}, \psi_{2}\in \caminhosespeciais{\gruposimpletico{2n}}$ e $h:\intervalo\times \intervalo \to \gruposimpletico{2n}$ uma homotopia tal que $h(t,0) = \psi_{1}(t)$ e $h(t,1) = \psi_{2}(t)$.
			Considere a aplicação contínua $H:\intervalo\times \intervalo\to \circulo$ definida por $H(t,s) = (\rho\circ h)(t,s)$. Então $H(t,0) = \rho_{\psi_{1}}(t)$ e $H(t,0) = \rho_{\psi_{2}}(1)$, portanto $H$ é uma homotopia entre os caminhos $\rho_{\psi_{1}}$ e $\rho_{\psi_{2}}$. Da Proposição \ref{proposicao_grau_aplicacao} tem-se que $\rho_{\psi_{1}}\sim \rho_{\psi_{2}}$ se, e somente se $deg(\rho_{\psi_{1}})=deg(\rho_{\psi_{2}})$. Logo $\rho_{\psi_{1}}\sim \rho_{\psi_{2}}$ se, e somente se, $\mu(\psi_{1})=\mu(\psi_{2})$. 
			
			\item \textbf{(Nullidade)} Como  $\espectrooperador{\psi(s)}\cap \circulo = \emptyset$ para todo $s\in \intervalo$, então pelo Item \ref{item_normalizacao_rho} do Teorema \ref{teorema_aplicacao_rho}, tem-se que $\rho(\psi(t)) = \pm 1$ para todo $t\in \intervalo$. Com isso, $deg(\rho_{\psi}) = 0$ e $\mu(\psi) = 0$.
			
			\item \textbf{(Produto)} Sabe-se que a identificação $\real{2n} \mapsto \real{2p}\times \real{2q}$ é um isomorfismo, o que induz o isomorfismo $\gruposimpletico{2n} \ni A \mapsto (A_{p}, A_{q}) \in \gruposimpletico{2p} \times \gruposimpletico{2q}$. Se $\Psi=(\psi_{p}, \psi_{q}) \in \caminhosespeciais{\gruposimpletico{2p}\times \gruposimpletico{2q}}$, então pelo Item \ref{item_produto_rho} do Teorema \ref{teorema_aplicacao_rho} tem-se que $\rho(\Psi(t))=\rho(\psi_{p}(t))\rho(\psi_{q}(t))$ para todo $t\in \intervalo$. Supondo que $\rho(\psi_{p}(t))=\exp(i\alpha_{\psi_{p}}(t))$ e $\rho(\psi_{q}(t))=\exp(i\alpha_{\psi_{q}}(t))$, então $\rho(\Psi(t)) = \exp(i(\alpha_{\psi_{p}}+\alpha_{\psi_{q}})(t))$, e com isso tem-se que 
			$$
			\begin{aligned}
			\mu(\Psi) &= 2deg(\rho_{\Psi}) 
			\\
			&= \frac{1}{2}\big(\alpha_{\psi_{p}}(1)+\alpha_{\psi_{q}}(1) -\alpha_{\psi_{p}}(0)-\alpha_{\psi_{q}}(0) \big) 
			\\
			&= 2deg(\rho_{\psi_{p}}) +2deg(\rho_{\psi_{p}}) 
			\\
			&=\mu(\psi_{p})+\mu(\psi_{q}).
			\end{aligned}
			$$
			
			\item \textbf{(Determinante)} Suponha que $\rho_{\psi}(t)=\exp(i\alpha_{\psi}(t))$ e que $det(Id-\gamma(1) )>0$, o que implica em  $sign(det(Id-\gamma(1) )) = (-1)^{m}$ para algum $m\in 2\inteiros$. Por definição tem-se que $\psi(1)=W^{+}$ e $\rho_{\psi}(1) = \rho(W^{+}) = (-1)^{n}$. Se $\rho_{\psi}(1) =1$, então $n\in 2\inteiros$, o que implica em $\alpha_{\psi}(1) = k\pi$ para algum $k\in 2\inteiros$. Logo $\mu(\psi) = (k\pi-0)/\pi \in 2\inteiros$ e $\mu(\psi)-n \in 2\inteiros$. Se $\rho_{\psi}(1) =-1$, então então $n\in 2\inteiros+1$, o que implica em $\alpha_{\psi}(1) = k\pi$ para $k\in 2\inteiros+1$. Logo $\mu(\psi) = (k\pi-0)/\pi \in 2\inteiros+1$ e $\mu(\psi)-n \in 2\inteiros$. Portanto $m = \mu(\psi)-n \in 2\inteiros$ esta bem-definido para $det(Id-\gamma(1) )>0$. Com argumentos análogos, pode-se mostrar que  $m = \mu(\psi)-n \in 2\inteiros+1$ para $det(Id-\gamma(1) )<0$, onde $\rho_{\psi}(1) = \rho(W^{-}) = (-1)^{n-1}$.
			
			\item \textbf{(Assinatura)} Afirmo que, $\Psi(t)=\exp(t\estruturacomplexa S) \in\gruposimpletico{2n}$. Para isso é suficiente mostrar que $\Psi(t)^{t}\estruturacomplexa\Psi(t)= \estruturacomplexa$ para todo $t \in \reta$. Note que $(\estruturacomplexa S)^{t} 
			= S^{t}\estruturacomplexa^{t} = -S\estruturacomplexa$. Com isso, dado um inteiro $k\geq 0$ tem-se a identidade 
			$$
			\begin{aligned}
			\estruturacomplexa (\estruturacomplexa S)^{k} 
			&= \estruturacomplexa\underbrace{\estruturacomplexa  S\dots \estruturacomplexa S}_{k-vezes} 
			\\
			&= \estruturacomplexa^{2}  S\estruturacomplexa \dots \estruturacomplexa S \underbrace{(-\estruturacomplexa^{2})}_{Id} 
			\\
			&=\underbrace{ S\estruturacomplexa \dots  S \estruturacomplexa }_{k-vezes}\estruturacomplexa 
			\\
			&= (S\estruturacomplexa)^{k}\estruturacomplexa.
			\end{aligned}
			$$
			
			Além disso, note que $((\estruturacomplexa S)^{k})^{t} = (\estruturacomplexa S)^{t}\dots (\estruturacomplexa S)^{t} = (S^{t}\estruturacomplexa^{t} )^{k} = (-S\estruturacomplexa)^{k}$. Com isso
			$$
			\begin{aligned}
			\Psi(t)^{t}\estruturacomplexa\Psi(t)
			&=
			\exp(t\estruturacomplexa S)^{t}\estruturacomplexa\exp(t\estruturacomplexa S)
			\\
			&=\sum_{j=0}^{\infty}\frac{\big( (t\estruturacomplexa S)^{j}\big)^{t}}{j!}\estruturacomplexa \sum_{k=0}^{\infty}\frac{(t\estruturacomplexa S)^{k}}{k!}
			\\
			&=\sum_{k,j=0}^{\infty}\frac{(-tS\estruturacomplexa)^{j}}{j!} \sum_{k=0}^{\infty}\frac{\estruturacomplexa(t\estruturacomplexa S)^{k}}{k!}
			\\
			&=\sum_{j=0}^{\infty}\frac{(-tS\estruturacomplexa)^{j}}{j!} \sum_{k=0}^{\infty}\frac{(tS\estruturacomplexa)^{k}}{k!}\estruturacomplexa
			\\
			&=exp(-tS\estruturacomplexa)exp(tS\estruturacomplexa)\estruturacomplexa
			\\
			&=\estruturacomplexa.
			\end{aligned}
			$$
			Portanto, $\Psi(t) \in \gruposimpletico{2n}$ para todo $t\in \reta$.
			
			Como $S$ é simétrica, então é diagonalizável. Além disso, existe uma matriz ortogonal $P \in \matrizortogonal{2n}$ tal que $P^{t}SP$ é diagonal (tais resultados podem ser encontrados em \cite{hoffman_kunze}). Sejam $\mathcal{P}:[0,1]\to \matrizortogonal{2n}$ um caminho contínuo tal que $\mathcal{P}(0) = Id$ e $\mathcal{P}(1)=P$ e $\mathcal{S}:\intervalo\to \generalgroupcomplexo{2n}$ dada por $\mathcal{S}(s)=\mathcal{P}^{t}(s)S\mathcal{P}(s)$. Do fato de que $\mathcal{P}^{t}(s)=\mathcal{P}^{-1}(s)$ tem-se que $det(\mathcal{S}(s)) =det(\mathcal{P}^{t}(s)S \mathcal{P}(s)) = det(S)$, portanto $\mathcal{S}(s)$ é diagonalizável e $\espectrooperador{\mathcal{S}(s)} = \espectrooperador{S}$ para todo $s \in [0,1]$. Além disso, $\mathcal{S}(s)^{t}$ é simétrica. De fato, $\mathcal{S}(s)^{t} = (\mathcal{P}^{t}(s)S\mathcal{P}(s))^{t} = \mathcal{P}^{t}(s)S^{t}\mathcal{P}(s) = \mathcal{P}^{t}(s)S\mathcal{P}(s) = \mathcal{S}(s)$, o que implica em $\Psi_{s}(t)=\exp(t\estruturacomplexa \mathcal{S}(s)) \in \gruposimpletico{2n}$.

			Dado um inteiro $k\geq 0$ temos que $\estruturacomplexa^{2k} = (-1)^{k}Id$ e $\estruturacomplexa^{2k+1} = (-1)^{k}\estruturacomplexa$. Com isso, tem-se que
			$$
			\begin{aligned}
			\exp(t\estruturacomplexa ) 
			&= \sum_{k=0}^{\infty}\frac{(t\estruturacomplexa)^{k}}{k!} 
			\\
			&= \sum_{k=0}^{\infty}\frac{(t\estruturacomplexa)^{2k}}{(2k)!} + \sum_{k=0}^{\infty}\frac{(t\estruturacomplexa)^{2k+1}}{(2k+1)!}
			\\
			&=\sum_{k=0}^{\infty}\frac{(-1)^{k} t^{2k} }{(2k)!} Id+ \sum_{k=0}^{\infty}\frac{(-1)^{k} t^{2k+1}}{(2k+1)!}\estruturacomplexa
			\\
			&=\cos(t)Id+\sin(t)\estruturacomplexa
			\\
			&=
			\left(
			\begin{array}{cc}
			\cos(t)Id & \sin(t)Id
			\\
			-\sin(t)Id & \cos(t)Id
			\end{array}
			\right),
			\
			\end{aligned}
			$$
			onde $\cos(t)Id$ e $\sin(t)Id$ são as matrizes diagonais $n \times n$ cujos elementos são $\cos(t)$ e $\sin(t)$, respectivamente.
			
			Seja $v_{j}$ um auto-vetor de $\mathcal{S}(s)$ com auto-valor $\lambda_{j}$. Então $\Psi_{s}(1)v_{j} = \exp(\estruturacomplexa \mathcal{S}(s))v_{j}=\exp(\lambda_{j}\estruturacomplexa )v_{j} $. Como $\norma{S}<2\pi$ e $\espectrooperador{\mathcal{S}(s)} = \espectrooperador{S}$, então $|\lambda_{j}|<2\pi$ para $1\leq j\leq 2n$. Além disso, $\exp(\lambda_{j}\estruturacomplexa ) =\cos(\lambda_{j})Id+\sin(\lambda_{j})\estruturacomplexa$. Logo $det(Id-\exp(\lambda_{j}\estruturacomplexa ) ) \neq 0$, o que implica em $1\notin \espectrooperador{\Psi_{s}(1)}$ e $\Psi_{s}(1) \in \gruposimpleticonaodegenerado{*}$ para todo $s\in \intervalo$. Com isso, pode-se afirmar que o caminho contínuo $\Psi(1):\intervalo\to \gruposimpleticonaodegenerado{*}$, definido por $\Psi(1)(s) = \Psi_{s}(1)$, é tal que $\Psi(1)(\intervalo)\subseteq \gruposimpleticonaodegenerado{+}$ ou $\Psi(1)(\intervalo)\subseteq \gruposimpleticonaodegenerado{-}$.
			
			Como os conjuntos $\gruposimpleticonaodegenerado{\pm}$ são conexos por caminhos e $\grupofundamental{\gruposimpleticonaodegenerado{\pm}} = \{\classe{0}\}$, quando visto como subgrupo de $\grupofundamental{\gruposimpletico{2n}}$, então todo caminho em $\gruposimpleticonaodegenerado{\pm}$ é homotopicamente equivalente ao caminho $\Psi_{s}(t)=\exp(t\estruturacomplexa \mathcal{S}(s))$, onde $\mathcal{S}(s)$ é a matriz diagonal $diag\{a_{1}, \dots, a_{2n}\}$. Denotando as n-duplas $(a_{1}, \dots, a_{n})$ e $(a_{n+1}, \dots, a_{2n})$ por $a_{(1)}$ e $a_{(2)}$, respectivamente, e $a_{(1)}.a_{(2)} = a_{1}a_{n+1}\dots a_{n}a_{2n}$, tem-se que
			$$
			\begin{aligned}
				(\estruturacomplexa \mathcal{S}(s))^{2j} &=(-a_{(1)}.a_{(2)})^{j} Id
				\\
				(\estruturacomplexa \mathcal{S}(s))^{2j+1} 
				&= (-a_{(1)}.a_{(2)})^{j} 
				\left(
				\begin{array}{cc}
				0 & a_{(2)}Id
				\\
				-a_{(1)}Id & 0  
				\end{array}
				\right),
			\end{aligned}
			$$
			onde $a_{(1)}Id=diag\{a_{1}, \dots, a_{n}\}$. O análogo vale para $a_{(2)}Id$ . Com isso, tem-se a exponencial
			$$
			\begin{aligned}
			\exp(\estruturacomplexa \mathcal{S}(s)) &= \sum_{j=0}^{\infty}\frac{(\estruturacomplexa \mathcal{S}(s))^{2j}}{(2j)!} + \sum_{j=0}^{\infty} \frac{(\estruturacomplexa \mathcal{S}(s))^{2j+1}}{(2j+1)!}
			\\
			&= \sum_{j=0}^{\infty}\frac{(-1)^{j}(a_{(1)}.a_{(2)})^{j}}{(2j)!}Id + \sum_{j=0}^{\infty} \frac{(-1)^{j}(a_{(1)}.a_{(2)})^{j}}{(2j+1)!}	\left(
			\begin{array}{cc}
			0 & a_{(2)}Id
			\\
			-a_{(1)}Id & 0  
			\end{array}
			\right).
			\end{aligned}
			$$
			
			A hipótese de que $\norma{S}<2\pi$ implica em $|a_{j}|<2\pi$ para $0\leq j \leq 2n$, o que permite a redução da análise ao caso em que  $|a_{j}|=\epsilon \in (0,2\pi)$. Com isso, tem-se os seguintes casos:
			\begin{enumerate}
				\item \textbf{Caso $a_{(1)}.a_{(2)}>0$ e $a_{(1)} = a_{(2)} = (\epsilon,\dots,\epsilon)$:}
				$$
				\begin{aligned}
				\exp(t\estruturacomplexa \mathcal{S}(s)) 
				&= 
				\cos(t\epsilon)Id+ \sin(t\epsilon)\estruturacomplexa
				\\
				&=	
				\left(
				\begin{array}{cc}
				\cos(t\epsilon)Id & \sin(t\epsilon)Id
				\\
				-\sin(t\epsilon)Id & \cos(t\epsilon)Id
				\end{array}
				\right)
				\\
				&\sim
				\left(
				\begin{array}{cccc}
				r(-t\epsilon) & 0 &\dots & 0
				\\
				0 & r(-t\epsilon) & \dots & 0
				\\
				\vdots & \vdots & \ddots & \vdots
				\\
				0 & \dots & r(-t\epsilon) & 0
				\\
				0 & \dots & 0 & r(-t\epsilon) 
				\end{array}
				\right)
				\\
				&=R(t\epsilon),
				\end{aligned}
				$$
				onde $r(-t\epsilon)$
				é a matriz $2\times 2$ de rotação por um ângulo $-t\epsilon <0$ no plano $\real{2}$ definida por
				$$
				r(-t\epsilon) = \left(
				\begin{array}{cc}
				\cos(t\epsilon) & \sin(t\epsilon)
				\\
				-\sin(t\epsilon) & \cos(t\epsilon)
				\end{array}
				\right),
				$$ 
				e $R(t\epsilon)$ é uma matriz com $n$ blocos diagonais resultante da permutação das linhas e colunas de $\exp(t\estruturacomplexa S)$. Realizando as identificações $r(-t\epsilon) \mapsto \exp(-it\epsilon)$ e $R(t\epsilon)\mapsto \exp(-it\epsilon)Id$, tem-se o determinante $det_{\complexo{}}(R(t\epsilon)) = \exp(-int\epsilon)\in \circulo$.
				
				Note que, $\exp(t\estruturacomplexa \mathcal{S}(s))$ é uma matriz ortogonal e, como foi mostrado anteriormente, é uma transformação simplética, logo $\exp(t\estruturacomplexa \mathcal{S}(s)) \in \matrizSimpleticaOrtogonal$. Relizando a identificação $\matrizSimpleticaOrtogonal\ni \Psi_{s}(t)=\exp(t\epsilon\estruturacomplexa) \mapsto \cos(t\epsilon)Id+i\sin(t\epsilon)\estruturacomplexa\in \matrizunitaria{n}$, e aplicando o Item \ref{item_determinante_rho} do Teorema \ref{teorema_aplicacao_rho}, tem-se que 
				$$
				\rho(\Psi_{s}(t)) = det_{\complexo{}}(\cos(t\epsilon)Id +i\sin(t\epsilon)\estruturacomplexa)=det_{\complexo{}}(R(t\epsilon)) = \exp(-int\epsilon).
				$$
				
				Portanto $\mu(\Psi_{s}) = 2deg(\rho_{\Psi(t)}) = -n$. Como o número de auto-valores negativos de $Ind(S)$ é zero, então  $ Ind(S)-n=\mu(\Psi_{s})$ para todo $s\in \intervalo$, como desejávamos.
				
				\item \textbf{Caso $a_{(1)}.a_{(2)}>0$ e $a_{(1)} = a_{(2)} = (-\epsilon,\dots,-\epsilon)$:} analogamente ao anterior, tem-se $R(t\epsilon)$,  onde $r(t\epsilon)$ é a matriz $2\times 2$ de rotação por um ângulo de $t\epsilon>0$ no plano $\real{2}$ e $det_{\complexo{}}(R(t\epsilon)) = \exp(int\epsilon)$. Isso implica em $\mu(\Psi_{s}) = 2deg(\exp(t\estruturacomplexa \mathcal{S}(s))) = n$. Como $Ind(S) = 2n$, então $ Ind(S)-n = 2n-n = n = \mu(\Psi_{s})$, como desejávamos.
				
				\item \textbf{Caso $a_{(1)}.a_{(2)}<0$ e $a_{(1)}=(\epsilon,\dots,\epsilon), \; a_{(2)} = -a_{(1)}$:} tem-se agora as funções hiporbólicas
				$$
				\exp(t\estruturacomplexa \mathcal{S}(s)) = 
				\left(
				\begin{array}{cc}
				\cosh(t\epsilon)Id & \sinh(t\epsilon)Id
				\\
				-\sinh(t\epsilon)Id & \cosh(t\epsilon)Id
				\end{array}
				\right).
				$$
				
				O sistema $\exp(t\estruturacomplexa \mathcal{S}(s)) v = \lambda v$ tem como solução os auto-valores $\lambda \in \{\cosh(t\epsilon)\pm i\sinh(t\epsilon)\}$. Portanto para todo $t\in (0,1]$ tem-se que $|\lambda| \neq 1$, logo $\lambda \notin \circulo$ e $\espectrooperador{\Psi_{s}}\cap\circulo = \emptyset$. Como a aplicação $\rho$ depende apenas dos auto-valores em $\espectrooperador{\Psi_{s}}\cap\circulo$, então $\mu(\Psi_{s}) = 0$. Além disso, não há auto-valores negativos, então $Ind(S)=0$ e $Ind(S)=\mu(\Psi_{s})$.
			\end{enumerate}
	
			\item \textbf{(Inverso)} Pode-se escrever que $\rho_{\psi}(t) = \exp(i\alpha_{\psi}(t))$. Pelo Lema \ref{lema_inversa_caminho_especial}, tem-se que $\psi^{-1} \in \caminhosespeciais{\gruposimpletico{2n}}$, o que implica em $\rho_{\psi^{-1}}(t) = (\rho_{\psi}(t))^{-1} = \exp(-i\alpha_{\psi}(t))$. Logo 
			$$
			\mu(\psi^{-1})=2deg(\rho_{\psi^{-1}}) = (-\alpha_{\psi}(1)+\alpha_{\psi}(0))/\pi = -2deg(\rho_{\psi}) = -\mu(\psi).
			$$
		\end{enumerate}
	\end{prova}
	
	O caso de sistemas Hamiltonianos autônomos fica reduzido ao caso de uma função de Morse $H:M\to \real{}$ cujo conjunto de soluções 1-periódicas $\lacocontrateis$ será o conjunto de pontos críticos isolados de $H$. Com isso, o índice de Maslov terá uma relação direta com índice de Morse da seguinte forma:
	
	\begin{corolario}
		Sejam $(M, \omega)$ uma 2n-variedade simplética, $H : M \to \real{}$ uma função hamiltoniana autônoma e $x \in Crit(H)$. Assumindo que $||Hess_{x}(H)|| < 2\pi$, então o índice de Maslov $\mu(x)$ da solução periódica $x$ do sistema hamiltoniano $\dot{x}(t) = X_{H}(x(t))$ pode ser relacionado o índice de Morse $Ind(x)$ do ponto crítico da função $H$ da seguinte forma:
		$$
		\mu(x) = Ind(x) - n.
		$$
	\end{corolario}	
	
	\chapter{Variedades Simpléticas e o Operador de Floer}
	\section{Variedades Simpléticas}
	\begin{definicao}
		(Variedade Simplética) Uma variedade simplética de dimensão 2m é o par $(M, \omega)$ onde $M$ é uma 2m-variedade diferenciável e $\omega \in \Omega^{2}(M^{m})$ é uma 2-forma fechada e não-degenerada, isto é, $d\omega = 0$ e para cada $p \in M$ temos que $(T_{p}M, \omega_{p})$ é um espaço vetorial simplético. Cada $x = (q_{1}, \dots, q_{m}, p_{1}, \dots, p_{m}) \in M$  será abreviado pelo par $(q,p)$.
	\end{definicao}
	
	\begin{definicao}
		(Gradiente simplético) Seja $f : M \times \real{} \to \real{}$ uma função suave, então o gradiente simplético é o único campo vetorial $X \in \mathfrak{X}(M)$ tal que $\omega(X, Y) = df(Y)$ para todo $Y \in \mathfrak{X}(M)$.
	\end{definicao}
	
	\begin{definicao}
		(Campo hamiltoniano) Uma função suave $H : M \times \real{} \to \real{}$ é chamada uma função hamiltoniana se satisfaz as equações diferenciais de Hamilton
		$$
		\frac{\partial q}{\partial t} = \frac{\partial H}{\partial p}, \; \frac{\partial p}{\partial t} = -\frac{\partial H}{\partial p}. 
		$$
		Um campo hamiltoniano é o único campo vetorial tal que $\omega(X_{H}, Y) = -dH(Y)$. Disso, podemos definir o fluxo hamiltoniano como sendo a solução dos sitema de equações 
		$$
		\label{sisHamilt}
		\derivadaparcial{\psi(t)}{t} = X_{H}(\psi(t), t).
		$$
		
		As soluções dessa equação geram uma família de simplectomorfismos a 1-parametro tal que $\psi_{t}(x(0)) = x(t)$.
	\end{definicao}
	Para demonstrar o invariância da forma simplética pelo fluxo Hamiltoniano, vamos usar o seguinte resultado:
	\begin{proposicao}\label{proposicao_identidade_cartan}
		Sejam $M$ uma variedade diferenciável com dimensão finita, $X \in T_{p}M$ e $\alpha \in \Omega^{k}(M)$. Definindo $i_{X}:\Omega^{k} \to \Omega^{k-1}$ por $i_{X}\alpha(Y_{1}, \dots, Y_{k-1}) = \alpha(X, Y_{1}, \dots, Y_{k-1})$ então
		$$
		\liederivada{X}\alpha = i_{X}d\alpha + di_{X}\alpha.
		$$
	\end{proposicao}
	
	\begin{lema}
		O fluxo hamiltoniano preserva a forma simplectica.
	\end{lema}
	\begin{prova}\label{fluxo_convervativo}
		$$
		\begin{aligned}
		\frac{d}{dt}\bigparenteses{(\psi_{t})^{*}\omega} 
		&= (\psi_{t})^{*} \liederivada{X_{H}}\omega  
		\\
		&= (\psi_{t})^{*} i_{X_{H}}\underbrace{d\omega }_ {=0}+ di_{X_{H}}\omega 
		\\
		&= (\psi_{t})^{*} (-ddH)=0,
		\end{aligned}
		$$
		logo $(\psi_{t})^{*} \omega = (\psi_{0})^{*} \omega = \omega$, para qualquer $t \in \real{}$
	\end{prova}

	\section{Fibrados Vetoriais - alguns resultados}
	\begin{definicao}
		(n-Fibrado vetorial) Um fibrado vetorial $\eta$ sobre um espaço topológico $B$ é denotado por $\eta = (E(\eta), B, p, G)$, onde $E=E(\eta)$ é um espaço topológico chamado espaço total, $p:E\to B$ é uma aplicação contínua chamada mapa de projeção e para cada $p \in B$ temos que $F_{b}=p^{-1}(b)$ é um espaço vetorial real com $dim(F_{b}) = n < \infty$, chamado de fibra de $b$. $G$ é um grupo de Lie, denominado grupo de estrutura, que age a esquerda de $F_{p}$, isto é, $\varphi:G\times F_{p} \to F_{p}$. Além disso, temos a condição de trivialidade local: para cada $b_{0} \in B$ existe uma vizinhança $U$ de $b_{0}$ e um homeomorfismo $h:U\times\real{n} \to p^{-1}(U)$ tal que $h(\{b\} \times \real{n})$ é isomorfo a $\real{n}$.
	\end{definicao}
	
	\begin{observacao}\label{observacao_fibrado_vetorial_trivial}
		Dizemos que um n-fibrado vetorial $\eta$ é trivial quando existe uma trivialização local $(U, h)$ onde $U = B$, isto é, o espaço total $E(\eta)$ é homeomorfo a $B\times \real{n}$, isto é, $E(\eta)\cong B\times \real{n}$.
	\end{observacao}
	
	Daqui em diante o grupo de estrutura será $G = GL(n,\real{})$ e um n-fibrado vetorial será denotado por $\eta = (E,B,p)$.
	
	\begin{definicao}
		(Mapa de fibrados) Sejam $\eta, \xi$ dois fibrados vetoriais. Chamamos de mapa de fibrados o par $(F, f)$ tal que comute o diagrama
		$$
		\xymatrix{
			E \ar[r]^{F}\ar[d]^{p} & E'\ar[d]^{p'}
			\\
			B\ar[r]^{f} & B'
		},
		$$
		e tal que $F$ é um isomorfismo entre $p^{-1}(b)$ e $p'^{-1}(f(b))$. A aplicação $f:B\to B'$ é chamada mapa de espaço base.
	\end{definicao}
	
	Pode-se mostrar que, nas condições da definição anterior, $f^{*}\eta$ satisfaz a condição de trivialidade local, logo é um n-fibrado vetorial.
	
	\begin{exemplo}
		(1-Fibrado vetorial) Sejam $E=S^{1} \times \real{}$, $B=S^{1}$, $F_{b} = \real{}$ para qualquer $b \in S^{1}\subset \real{2}$ e $p:S^{1} \times \real{}\to S^{1}$ tal que $p(b, e)=p \in S^{1}$. Então $\eta=(S^{1} \times \real{}, S^{1}, p)$ é um fibrado vetorial pois $p$ é uma aplicação contínua, $E, B$ são espaços topológicos e $F_{b} = p^{-1}(b) = \real{}$ é um 1-espaço vetorial. Além disso, $\eta$ é um fibrado vetorial trivial pois, tomando os abertos $S^{+} = S^{1} \backslash \{(0,1)\}$ e $S^{-} = S^{1} \backslash \{(0,-1)\}$, tem-se as projeções $p^{-1}(S^{\pm}) = S^{\pm} \times \real{} \subset E$, e com isso, $(S^{+}\times\real{}) \cup ( S^{-}\times\real{}) = (S^{+}\cup S^{-})\times\real{} = S^{1} \times\real{} =E$, logo $\eta$ é trivial. 
		
		\vermelho{
		Defina $f:[0,\pi/2] \to S^{1}$ por $f(t) = (cos(t), sin(t))$. Sejam $B' =[0,\pi/2]$ e $E' \subset [0,\pi/2]\times E$ o conjunto dos pares $(b, e)$ com $e \in p^{-1}(f(b))$, então $p': E'\to B'$ tal que $p'(b, e) = b \in [0,\pi/2]$ temos o pull-back $f^{*}\eta = (E', B', p') = ([0,\pi/2]\times S^{1}\times \real{}, [0,\pi/2], p')$. (Acho que deve ser feito posteriormente)}
	\end{exemplo}
	
	\begin{proposicao}
		(Fibrado tangente) Sejam $M$ uma variedade suave n-dimensional. Definindo $B=M$, dado $q\in M$ temos as fibras $F_{q} = T_{q}M$ e o espaço total é $TM=\{(p, v_{p}): p\in M, v_{q}\in T_{q}M \}$. A tripla $\eta = (TM, M, p)$, denominado fibrado tangente de $M$, é localmente trivial, portanto é um fibrado vetorial.
	\end{proposicao}
	\begin{prova}
		Note que a projeção $p(q,v_{q}) = q$ é uma aplicação contínua, $p^{-1}(\{q\}) = T_{q}M$ é um n-espaço vetorial em $q\in M$ e  o espaço base $M$ é um espaço topológico pois é uma variedade. Seja $U$ uma vizinhança de $q$. Como $T_{q}M$ é homeomorfo a $\real{n}$, então $p^{-1}(U) = \bigcup_{q\in M}(\{q\}\times T_{q}M)$ é homeomorfo a $U\times \real{n}$ e $\eta$ é localmente trivial. Portanto $\eta$ é um n-fibrado vetorial.
	\end{prova}
	
	\begin{observacao}
		Os fibrados tangentes $\eta$ serão denotados por $TM$, o que é convencionado por grande parte da literatura.
	\end{observacao}
	
	\begin{definicao}
		(Seção do fibrado tangente) Seja $TM$ um $n$-fibrado tangente de uma $n$-variedade diferenciável $M$. Uma seção de $TM$ é uma aplicação suave $s:M \to TM$ tal que $p\circ s=Id$.
	\end{definicao}
	
	\begin{observacao}\label{observacao_identificacao_secao_campo_vetorial}
		Os campos vetoriais definidos em $M$ podem ser identificados com as seções do fibrado tangente $TM$ do seguinte modo: seja $X\in \campossuaves{M}$ e definida $s_{X}$ a seção do fibrado tangente por $s_{X}(q) = (q, X(q))$. tem-se que, $(p\circ s_{X})(q) = q$, logo $p\circ s_{X}=Id$. Com isso, segue-se a identificação $s_{X} \mapsto X$.
	\end{observacao}
	
	\begin{definicao}
		(Pullback de um fibrado) Sejam $B, B'$ dois espaços topológicos, $\eta=(E', B', p')$ um n-fibrado vetorial e $f:B\to B'$ uma aplicação contínua. Defina $E \subseteq B\times E'$ como sendo o conjunto $\{(b, e): b \in B,\; e \in p'^{-1}(f(b)) \}$. Além disso, defina o mapa de projeção $p:E\to B$ tal que $p(b,e) = b$. O pullback de $\eta$ por $f$ é a tripla $f^{*}\eta = (E,B, p)$.
	\end{definicao}
	
	\begin{lema}\label{pullback_composicao}
		(Composição de pullbacks) Sejam $\eta =(E, C, p)$ um n-fibrado vetorial, $A, B, C$ espaços topológicos, $f:A\to B$ e $g:B\to C$ funções contínuas. Então $(g\circ f)^{*} \eta= f^{*}(g^{*}\eta)$.
	\end{lema}
	\begin{prova}
		Como $f, g$ são contínuas, então $g\circ f:A\to C$ é contínua, logo $(g\circ f)^{*}\eta = (E_{g\circ f}, A, p_{g\circ f})$ é um n-fibrado vetorial onde dado as fibradas são dadas por $F_{(g\circ f)(a)} = p^{-1}((g\circ f)(a))$. Tem-se que $g^{*}\eta=(E_{g}, B, p_{g})$, $f^{*}(g^{*}\eta) = (E_{f}, A, p_{f})$, onde suas fibras são dadas por $F_{g(b)} = p^{-1}(g(b))$, sendo que $b=f(a)$, logo  $F_{g(f(a))} = p^{-1}(g(f(a))) = p^{-1}((g \circ f)(a))$, que é uma fibra de $(g\circ f)^{*}\eta$. Como o espaço base, as fibras e as projeções de $(g\circ f)^{*}\eta$ e $f^{*}(g^{*}\eta)$ coincidem, então $f^{*}(g^{*}\eta) = (g\circ f)^{*}\eta$.
	\end{prova}
	\begin{lema}\label{pullback_trivial}
		(Pullback trivial) Sejam $\eta = (E, B, p)$ um n-fibrado vetorial trivial, $A$ um espaço topológico e $f:A\to B$ uma função contínua, então $f^{*}\eta$ é trivial.
	\end{lema}
	\begin{prova}
		Por definição tem-se que $f^{*}\eta = (E_{f}, A, p_{f})$, onde $E_{f}=\{(a, e): a\in A,\; e \in p^{-1}(f(a)) \}$ e $E_{f} \subseteq A\times E$. Como $\eta$ é trivial, então $E \cong B\times \real{n}$ e existe um homeomorfismo $h: A\times E \to A\times B\times \real{n}$. Note que, $h|_{E_{f}}: E_{f}\to A\times f(A)\times \real{n} $ é um homeomorfismo. Como $p_{f}^{-1}(A) = A\times f(A)$ e $p^{-1}_{f}$ é contínua, então $h|_{E_{f}}\circ f\circ p_{f}: E_{f}\to p_{f}^{-1}(A)\times \real{n} $ e portanto $f^{*}\eta$ é trivial.
		\vermelho{
			
			Portanto $E_{f} \subset A\times E \cong f(A)\times \real{n}$ e $E_{f}$ é trivial....concluir essa demonstracao...}
	\end{prova}
	
	Os seguintes resultados nos dão informação sobre a homotopia dos fibrados vetoriais e serão usados adiante na construção dos índices de Maslov.
	
	\begin{teorema}\label{pullback_isomorfismo}
		Sejam $A, B$ dois espaços topológicos, $\eta=(E, B, p)$ um n-fibrado vetorial e $f,g: A\to B$ duas apliações homotópicas, então $f^{*}\eta \cong g^{*}\eta$.
	\end{teorema}
	
	\begin{corolario}\label{pullback_contratil}
		Sejam $\eta$ um n-fibrado vetorial sobre um espaço base compacto e contrátil, então $\eta$ é trivial.
	\end{corolario}
	\begin{prova}
		Sejam $B$ um compacto contrátil, $\eta=(E, B, p)$ um n-fibrado vetorial e $\{*\} \subset B$ um conjunto unitário. Definindo $f:B\to \{*\}$ e $g:\{*\}\to B$ teremos $f\circ g : \{*\} \to \{*\}$ e $g\circ f:B\to B$, logo $f\circ g = Id_{\{*\}}$ e pela hipótese da contratibilidade de $B$ temos $g\circ f \simeq Id_{B}$. Pela definição temos que $g^{*}\eta = (E', \{*\}, p')$ é um n-fibrado trivial e pelo Teorema $\ref{pullback_isomorfismo}$ teremos $(g\circ f)^{*}\eta \cong Id_{B}^{*}\eta = \eta$. Pelo Lema $\ref{pullback_composicao}$ $(g\circ f)^{*}\eta = f^{*}(g^{*}\eta) $ é trivial pois $g^{*}\eta$ é trivial pelo Lema $\ref{pullback_trivial}$, logo $ \eta \cong f^{*}(g^{*}\eta)$ é trivial.
	\end{prova}
	
	\section{O funcional $\funcionalH$ e a equação de Floer}
	Nesse capítulo é considerado $M$ como sendo uma 2n-variedade diferenciável fechada (veja a Definição \ref{definicao_variedade_fechada}).
	
	Como foi abordado na Introdução, a conjectura de Arnold tem como hipótese uma hamiltoniana dependente do tempo $H: \retacartesianovariedade \to \reta$ e, para o sistema Hamiltoniano abaixo,
	$$
	\derivadaparcial{\psi(t)}{t} = X_{H}(\psi(t), t),
	$$
	busca-se as soluções 1-periódicas (isto é, $\psi(t+1)=\psi(t)$) e contráteis, onde $\campohamiltonianoabrev$ é o único campo vetorial tal que $\formaSimpletica{\campohamiltonianoabrev}{\xi} = -dH(\xi)$ para todo $\xi \in \campossuaves{M}$. O que é equivalente a analisar as aplicações $\psi : \circulo\to M$ que satisfazem esse sistema.
	
	\begin{definicao}
		(Laços contráteis) O conjunto das aplicações $\psi:\circulo \to M$ de classe $C^{\infty}$ e contráteis é denotado por $\lacocontrateis$.
	\end{definicao}
	
	A construção do complexo de Morse-Witten basea-se na determinação dos pontos críticos de uma função de Morse-Smale $f: M\to \reta$ e como tais pontos são conectados pelas linhas do fluxo do gradiente negativo $-\gradiente$. Para a demonstração da Conjectura de Arnold uma abordagem análoga é realizada. Contudo, em vez de uma variedade de dimensão finita $M$ tem-se um espaço de funções $\lacocontrateis$, e em vez de uma função $f:M\to \reta$ tem-se um funcional $\funcionalH:\lacocontrateis\to \reta$. A seguir é mostrado que os pontos críticos desse funcional são as soluções 1-periódicas do sistema Hamiltoniano, e por isso, $\funcionalH$ é chamado funcional Hamiltoniano. Por fim, a conjectura de Arnold é demonstrada construindo-se um complexo de cadeias baseado nos pontos críticos desse funcional e em como tais pontos são conectados pelas linhas de fluxo do gradiente negativo $-\gradientefuncional$. A homologia desse complexo é chamada de Homologia de Floer.
	
	No Apêndice \ref{apendice_variedades_banach} mostra-se que $\lacocontrateis$ é uma variedade, mais especificamente uma variedade de Banach. Com isso, dado $\gamma\in \lacocontrateis$ o espaço tangente $\espacotangenteponto{\gamma}{\lacocontrateis}$ está bem-definido.
	
	Sejam $\gamma \in \lacocontrateis$ e $X \in \espacotangenteponto{\gamma}{\lacocontrateis}$. Considere o caminho $u : \reta \to \lacocontrateis$ como solução do sistema $u'(0)=X $ e $u(0) = \gamma$. Com isso, $u(s)\in \lacocontrateis$ para todo $s\in \reta$, isto é, $u(s):\circulo\to M$, onde $u(0,z) = \gamma(z)$ para todo $z\in \circulo$. Portanto
	$$
	\derivadaparcial{}{s}u(s,z)|_{s=0}=X(z) \in \espacotangenteponto{\gamma(z)}{M}.
	$$
	
	
	Dado que $\gamma:\circulo\to M$ é um caminho contínuo, então o pullback $\gamma^{*}TM$ é o $2n$-fibrado vetorial onde as fibras são $\espacotangenteponto{\gamma(t)}{M}$, e com isso, seus elementos são as restrições dos campos vetoriais $X \in \campossuaves{M}$ ao caminho $\gamma$. Essa descrição resulta na identificação $\espacotangenteponto{\gamma}{\lacocontrateis} \ni X \mapsto  X|_{\gamma} \in \pullbackfibradotangente{\gamma}{M}$. 
	\begin{definicao}\label{definicao_funcional_hamiltoniano}
		(Funcional Hamiltonianno) O funcional Hamiltoniano é a aplicação $\funcionalH: \lacocontrateis\to \reta$ definida por
		$$
		\funcionalHponto{\gamma} = -\int_{D^{2}}u^{*}\omega + \int_{0}^{1}H(\gamma(t), t)dt,
		$$
		onde $D^{2} \subset \mathbb{C}$ é o disco fechado e $u(e^{i2\pi t})=\gamma(t)$. A aplicação $u:D^{2}\to M$ é chamada extensão de $\gamma$ para o disco.
	\end{definicao}
	
	\begin{definicao}\label{definicao_condicao_aesfericidade}
		(Condição de Aesfericidades) Seja $u:S^{2} \to M$ uma aplicação suave, então a condição de aesfericidade é dada por 
		$$
		\int_{S^{2}} u^{*}\omega = 0.
		$$
	\end{definicao}
	
	\begin{observacao}
		A primeira integral na Definição \ref{definicao_funcional_hamiltoniano} do funcional Hamiltoniano depende explicitamente da extensão de $\gamma$ escolhida. Supondo que $v : D^{2}\to M$ é outra extensão de $\gamma$, então
		$$
		\int_{D^{2}}u^{*}\omega - \int_{D^{2}}v^{*}\omega  = \int_{S^{2}}w^{*}\omega, 
		$$
		onde $w:S^{2} \to M$ é uma aplicação contínua e é definida como sendo uma colagem dos dois discos $u(D^{2})$ e $v(D^{2})$ ao longo de seus bordos. A Condição de Aesfericidade \ref{definicao_condicao_aesfericidade} garante que $\int_{S^{2}}w^{*}\omega=0$, logo $\int_{D^{2}}u^{*}\omega = \int_{D^{2}}v^{*}\omega$, eliminando a dependência do funcional $\funcionalH$ da escolha dessas extensões.
	\end{observacao}
	
	\vermelho{(Colocar uma figura islustrando a colagem aqui!!!!)}
	
	Seja $T\lacocontrateis$ o fibrado tangente de $\lacocontrateis$. Então o diferencial do funcional Hamiltoniano é a aplicação $d\funcionalH: T\lacocontrateis \to \real{}$ e , em analogia à análise em espaços de dimensão finita, o gradiente é o campo suave $\gradientefuncional\in \campossuaves{\lacocontrateis}$ tal que $d_{\gamma}\funcionalH(X) = \produtointerno{\gradientefuncional(\gamma)}{X(\gamma)}$ para todo $X\in \espacotangenteponto{\gamma}{\lacocontrateis}$, onde $\produtointerno{.}{.}$ é um produto interno definido a seguir.
	
	Sejam $Y(t) \in \espacotangenteponto{\gamma(t)}{M}$, $\tilde{\gamma}:(-\epsilon, \epsilon)\times S^{1} \to M$  uma aplicação de classe $C^{1}$ tal que $\tilde{\gamma}(0,t) = \gamma(t)$ e $Y(t) = \derivadaparcial{}{s}\tilde{\gamma}(s,t)|_{s=0}$. Com isso, o diferencial de $\funcionalH$ no ponto $\gamma$ e avaliado em $Y$ é dado por $d_{\gamma}\funcionalH(Y) = \derivadaparcial{}{s}\funcionalH(\tilde{\gamma})|_{s=0}$.
	
	Para se determinar o diferencial do funcional defina $\tilde{u}:(-\epsilon, \epsilon)\times D^{2} \to M$ como sendo uma extensão de $u$, tal que $\tilde{u}(0,z) = u(z)$ e $\tilde{u}(s,e^{2i\pi t}) = \tilde{\gamma}(s,t)$, e $\tilde{Y}(z) = \derivadaparcial{}{s}\tilde{u}(s,z)|_{s=0}$, uma extensão de $Y$. Avaliando o funcional nessas extensões, tem-se
	
	$$
	\funcionalHponto{\tilde{\gamma}} = -\int_{D^{2}}\tilde{u}^{*}\omega + \int_{0}^{1}H(\tilde{\gamma}(s,t))dt.
	$$
	
	\begin{observacao}
		Note que, $	\funcionalHponto{\tilde{\gamma}}$ é uma função de classe $C^{1}$ e que depende explícitamente de $s\in (-\epsilon, \epsilon)$, isto é, $	\funcionalHponto{\tilde{\gamma}}:(-\epsilon,\epsilon)\to \reta$. 
	\end{observacao}
	
	Seja $\phi:\retacartesianovariedade\to M$ o fluxo gerado pelo campo $Y\in \campossuaves{M}$. A derivada de Lie no ponto $p\in M $ de uma r-forma $\alpha \in \Omega^{r}(M)$ ao longo do campo $Y$ é definida por 
	$$
	\liederivada{Y}\alpha = \derivada{}{t}(\phi^{*}_{p}(t)\alpha)|_{t=0},
	$$ 
	o que pode ser encontrado em \cite{nakahara}. Além disso, pode-se mostrar a identidade 
	$$
	\liederivada{Y}\alpha= di_{Y}\alpha+i_{Y}d\alpha,
	$$
	onde $i_{Y}:\Omega^{r}(M)\to \Omega^{r-1}(M)$ é definida por $i_{Y}(\alpha)(X_{1}, \dots, X_{r-1}) = \alpha(Y, X_{1}, \dots, X_{r-1})$.
	
	Seja $Y\in \campossuaves{M}$ o campo gerador do fluxo $\tilde{u}:\intervalo\times D^{2}\to M$. Na determinação do diferencial do funcional deriva-se o primeiro termo
	$$
	\begin{aligned}
	-\frac{d}{ds} \Bigm\lvert_{s=0} \int_{D^{2}}\tilde{u}^{*}\omega 
	&=-\int_{D^{2}}\frac{d}{ds} (\tilde{u}^{*}\omega)\Bigm\lvert_{s=0} \\
	&=-\int_{D^{2}} \mathcal{L}_{\tilde{Y}(z)}(\omega)
	\\
	&=-\int_{D^{2}}u^{*} \mathcal{L}_{Y}(\omega)\vermelho{?????????}
	\\
	&=-\int_{D^{2}}u^{*} (d(i_{Y}\omega) + i_{Y}d\omega)
	\\
	&=-\int_{D^{2}}u^{*} d(i_{Y}\omega)
	\\
	&=-\int_{\circulo}	\gamma^{*} (i_{Y(t)}\omega)
	\\
	&= \int_{[0,1]} \omega(\dot{\gamma}(t), Y(t))dt.
	\end{aligned}
	$$
	
	Diferenciando o segundo termo:
	$$
	\begin{aligned}
	\frac{d}{ds} \Bigm\lvert_{s=0} \int_{[0,1]} H(\tilde{\gamma}(s,t)) 
	&= \int_{[0,1]} \derivadaparcial{}{s} H(\tilde{\gamma}(s,t)) \Bigm\lvert_{s=0}
	\\
	&= \int_{[0,1]} d_{\gamma}H(Y)dt
	\\
	&= \int_{[0,1]} -\omega_{\gamma}(\campohamiltonianoabrev(\gamma, t), Y)dt. 
	\end{aligned}
	$$
	Por fim, somando ambos os termos:
	$$
	d_{\gamma}\funcionalH(Y) = \int_{[0,1]} \omega(\dot{\gamma} - \campohamiltonianoabrev, Y)dt.
	$$
	
	Se $\gamma \in \lacocontrateis$ é um ponto crítico de $\funcionalH$, então $d_{\gamma}\funcionalH(Y) = 0 $ para todo $Y(\gamma) \in \espacotangenteponto{\gamma}{\lacocontrateis}$. Como o integrando de $d_{\gamma}\funcionalH(Y)$ é uma aplicação contínua, então pode-se afirmar que
	$$
	\omega(\dot{\gamma} - \campohamiltonianoabrev, Y)=0,\; \forall Y \in \espacotangenteponto{\gamma}{\lacocontrateis}.
	$$
	
	Como $\omega$  é uma forma simplética não-degenerada, então deve-se ter $\dot{\gamma}(t) - \campohamiltonianoabrev(\gamma, t)=0$. Portanto, um dado $\gamma \in \lacocontrateis$ é um ponto crítico do funcional Hamiltoniano $\funcionalH$ se, e somente se, $\gamma$ é uma solução 1-periódica e contrátil das equações de Hamilton.
	
	Foi determinado o diferencial do funcional Hamiltoniano e foi caracterizado os pontos críticos do mesmo. A seguir será definido o campo gradiente negativo de $\funcionalH$. Sabe-se que, dado $J \in \estruturascomplexas{M}{\omega}$, tem-se que $g_{p}(., .)=\omega_{p}(.,J.):T_{p}M\times T_{p}M \to \reta$, definida por $\omega_{p}(X,JY)$, é uma métrica Riemannian (veja \cite{manfredo_riemannian_geo}).
	
	Com isso, pode-se munir $\lacocontrateis$ com uma métrica $\produtointerno{.}{.}_{\gamma}: \espacotangenteponto{\gamma}{\lacocontrateis} \times \espacotangenteponto{\gamma}{\lacocontrateis} \to \reta$, a qual é induzida por $g$ do seguinte modo:
	
	$$
	\produtointerno{X}{Y}_{\gamma} = \int_{0}^{1}g_{\gamma(t)}(X, Y)dt.
	$$
	
	Note que, na definição da métrica induzida foi usada a identificação dos fibrados tangentes $\espacotangenteponto{\gamma}{\lacocontrateis} \ni X \mapsto  X|_{\gamma} \in \pullbackfibradotangente{\gamma}{M}$. De fato é uma métrica. Como $g$ é bilinear e a integração é uma operação linear, então $\produtointerno{.}{.}_{\gamma}$ é bilinear. Além disso, como $g$ é positiva-definida e a integral preserva essa positividade, então $\produtointerno{.}{.}_{\gamma}$ herda essa propriedade.

	\begin{definicao}
		(Gradiente de $\funcionalH$) O campo gradiente de $\funcionalH$ é o campo suave denotado por $\gradientefuncional$ tal que $d_{\gamma}\funcionalH(Y) = \produtointerno{\gradientefuncional}{Y}_{\gamma}$ para todo $\gamma \in \lacocontrateis$ e $Y(\gamma) \in \espacotangenteponto{\gamma}{\lacocontrateis}$.
	\end{definicao}

	Usando a expressão do diferencial $d\funcionalHponto{\gamma}$ determinanda anteriormente e a definição do campo gradiente, tem-se que
	$$
	\begin{aligned}
	\int_{[0,1]} \omega(\dot{\gamma} - \campohamiltonianoabrev, Y)dt&=
	d_{\gamma}\funcionalH(Y)
	\\ 
	&= \iprod{\gradientefuncional}{Y}_{\gamma}
	\\
	&= \int_{[0,1]}g_{\gamma(t)}(\gradientefuncional, Y)dt
	\\
	&=\int_{[0,1]} \omega_{\gamma(t)}(\gradientefuncional, JY)dt
	\\
	&=\int_{[0,1]} \omega_{\gamma(t)}(-J\gradientefuncional, Y)dt.
	\end{aligned}.
	$$
	Subtraindo ambos os lados identidade, tem-se que
	$$
	\int_{[0,1]} \omega(\dot{\gamma} - \campohamiltonianoabrev + J\gradientefuncional, Y)dt = 0.
	$$
	
	Como a igualdade vale para todo $Y \in \espacotangenteponto{\gamma}{\lacocontrateis}$ e o integrando é uma função contínua, então $\omega(\dot{\gamma} - \campohamiltonianoabrev + J\gradientefuncional, Y)=0$. Além disso, a não-degenerecência da forma simplética $\omega$ implica que
	$$
	J\gradientefuncional(\gamma(t)) +\dot{\gamma}(t)-\campohamiltoniano{\gamma(t), t} = 0, \; \forall t \in \intervalo.
	$$
	Portanto, o campo gradiente de $\funcionalH$ é dado por
	$$
	\gradientefuncional(\gamma(t))= J(\gamma(t))\dot{\gamma}(t)-J(\gamma(t))\campohamiltoniano{\gamma(t), t}.
	$$
	
	Suponha que $u :\reta\to \lacocontrateis$ seja uma solução do sistema $\derivada{}{s}u(s)|_{s=0} = -\gradientefuncional(\gamma)$ e $u(0)=\gamma$. Com isso, pode-se afirmar que tais soluções são as trajetórias em $\lacocontrateis$ do campo gradiente negativo $-\gradientefuncional$. Note que, para casa $s\in \reta$ tem-se que $u(s):\circulo\to M$ é um caminho contínuo e fechado. Portanto as trajetórias de $-\gradientefuncional$ podem ser vistas como aplicações $u:\retacartesianocirculo \to M$ de classe $C^{1}$. Substituindo essa expressão na equação do gradiente de $\funcionalH$ tem-se a Equação de Floer
	
	$$
	\derivadaparcial{}{s}u(s,t) + J(u(s,t))\derivadaparcial{}{t}u(s,t) -J(u(s,t)) \campohamiltoniano{u(s,t)}=0. 
	$$
	
	\begin{observacao}
		Note que, na determinação das soluções da equação de Floer não estamos diante de um problema de Cauchy, pois não temos condições de contorno nessa formulação.
	\end{observacao}
		
	\section{Soluções de Energia Finita $\energiafinitaM$}\label{secao_funcional_energia}
	
	No Capítulo \ref{capitulo_teoria_morse} mostrou-se que as linhas de fluxo do gradiente negativo de uma função de Morse, definida em uma variedade fechada, conectam dois pontos críticos (veja o Lema $\ref{lema_conjunto_limite_funcao_morse}$). Ainda no caso clássico, suponha que $f \in \funcoesmorsesmale{M}$ e que $\gamma:\reta \to M$ seja uma solução do sistema de equações diferenciais
	$$
	\derivada{}{s}\gamma(s)=-\gradiente(\gamma(s)),
	$$
	isto é, $\gamma$ é uma trajetoria do gradiente negativo. A compacidade de $M$ implica que $\gamma(\reta) \subset M$ é, em algum sentido, limitado. Por essa limitação de $\gamma(\reta)$ diz-se que existem $a, b \in M$ tais que $\lim_{s\to -\infty}\gamma(s)=a$ e $\lim_{s\to \infty}\gamma(s)=b$. 
	
	Defina o funcional energia por
	$$
	E(\gamma)= -\int_{\reta}\gamma^{*} df.
	$$
	Se $\gamma$ conecta dois pontos críticos $p,q\in \pontoscriticos{f}$, ou seja, $\lim_{s\to -\infty}\gamma(s)=p$ e $\lim_{s\to \infty}\gamma(s)=q$, então a energia de $\gamma$ é dada por $E(\gamma) = f(p) -f(q)$. Como a função de Morse $f$ é decrescente ao longo das linhas de fluxo do campo $-\gradiente$, então $0\leq E(\gamma)<\infty$.
	
	Um dos objetos de interesse desse capítulo são as soluções da equação de Floer que conectam pontos críticos do funcional $\funcionalH$. Será mostrado adiante que essas soluções residem em um espaço de funções cuja compacidade é dada em certas condições (veja o Teorema \ref{teorema_compacidade_gromov}). Além disso, tais soluções $u:\reta\to \lacocontrateis$, que podem ser vistas como as linhas do fluxo do campo $-\gradientefuncional$, conectam dois pontos críticos $\gamma, \gamma'\in \pontoscriticos{\funcionalH}$ se, e somente se, o funcional energia $E(u)$ é finito.
	
	Seja $\cilindrosLM$ o conjunto dos caminhos de classe $C^{1}$ em $\lacocontrateis$.
	
	\begin{definicao}
		(Funcional Energia) O funcional energia é a aplicação $E: \cilindrosLM \to \reta$ definida por
		$$
			E(u) = -\int_{\reta}u^{*}d\funcionalH.
		$$
	\end{definicao}
	
	\begin{observacao}
		No final da seção anterior foi realizada a identificação $\lacocontrateis \ni u(s) \mapsto u_{z}(s)\in M $, onde $u(s):\circulo\to M$ e $u_{z}(s) = u(s)(z)$.
		Com isso, dado $u \in \cilindrosLM$, tem-se que $E(u)$ depende explícitamente de $z \in \circulo$ e que $E(u):\circulo\to \reta$ é uma aplicação diferenciável.
	\end{observacao}
	
	Dado $Y \in T\reta$, tem-se que
	$$
	\begin{aligned}
		(u^{*}d\funcionalH)_{s}(Y) &= d_{u}\funcionalH(d_{s}u(Y))
		\\
		&=\produtointerno{\gradientefuncional}{Y\derivada{}{\lambda}u(\lambda)|_{\lambda=s}}_{u(s)}
		\\
		&=\produtointerno{\gradientefuncional}{-Y\gradientefuncional}_{u(s)}
		\\
		&=-Y\norma{\gradientefuncional(u(s))}^{2}.
	\end{aligned}
	$$
	Como $Y\in T\reta$ é arbitrário, então $(u^{*}d\funcionalH)_{s} = -\norma{\gradientefuncional(u(s))}^{2}$. Inserindo essa expressão no funcional energia, tem-se
	$$
	E(u)=-\int_{\reta}u^{*}d\funcionalH = \int_{\reta}\norma{\gradientefuncional(u(s))}^{2}.
	$$
	
	\begin{observacao}
		Seguem algumas propriedade do funcional energia:
		
		\begin{enumerate}
			\item Para todo $u \in \cilindrosLM$ tem-se que $E(u)\geq 0$.
			
			\item $E(u) = 0$ se, e somente se, $\derivadaparcial{}{s}u(s) = 0$. Isso reduz $u$ a uma solução do sistema Hamiltoniano. Logo $u$ é um ponto crítico de $\funcionalH$.
			
			\item Se $u \in \cilindrosLM$ conecta dois pontos $\gamma, \gamma' \in \pontoscriticos{\funcionalH}$, isto é, $\lim_{s\to -\infty}u(s)=\gamma'$ e $\lim_{s\to \infty}u(s)=\gamma$, então $E(u)=\funcionalH(\gamma') - \funcionalH(\gamma)<\infty$.
		\end{enumerate}
		
	\end{observacao}
	
 	\begin{definicao}
 		(Soluções de energia finita) O conjunto de todas as trajetórias de $\gradientefuncional$ tal que o funcional energia seja finito é denotado por $\energiafinitaM$ 
 	\end{definicao}
 	
 	\begin{observacao}
 		Como os elementos de $\lacocontrateis$ são caminhos fechados contráteis, então todo elemento $u$ de $\energiafinitaM$ é contrátil pois $u:\reta\to \lacocontrateis$.
 	\end{observacao}
 	
 	Grande parte da análise feita daqui em diante é baseada em operadores diferenciais definidos em $\energiafinitaM$. Com isso, a definição de uma topologia para esse conjunto é necessária. Para contornar dificuldades adicionais com essa escolha, é usado o fato de que a $2n$-variedade $M$ pode ser mergulhada no espaço euclidiano $\real{m}$, onde $m=4n+1$, o que é garantido pelo Teorema de Whitney \ref{teorema_whitney}. Como $M$ é um compacto, então $M \subset\real{m}$ é um fechado e limitado. Com isso, os elementos $\gamma\in \lacocontrateis$ e $u\in \energiafinitaM$ podem ser vistos como as seguintes aplicações $\gamma:\circulo \to \real{m}$ e $u:\retacartesianocirculo \to \real{m}$. Com essa identificação, tem-se que $\lacocontrateis\subset \aplicaoessuaves{\circulo}{M}\subset \aplicaoessuaves{\circulo}{\real{m}}$ e $\energiafinitaM \subset \aplicaoessuaves{\retacartesianocirculo}{M}\subset \aplicaoessuaves{\retacartesianocirculo}{\real{m}}$. 
	
	\begin{definicao}
		($\cktopologia{k}$) Seja $C^{r}(\real{n'}, \real{n})$ o conjunto das aplicações de $\real{n'}$ em $\real{n}$ de classe $C^{r}$. Então, dado $k\leq r$, definimos a $\cktopologia{k}$ de $C^{r}(\real{n'}, \real{n})$ como sendo a topologia gerada pela norma
		$$
		\norma{f} = max \{\norma{f}_{j}\}_{j=1}^{r},\; \text{tal que} \;\; \norma{f}_{j} = \sup_{p \in B(1)}\norma{f^{(j)}(p)},
		$$
		onde $B(1)$ é a bola unitária centrada na origem e $f^{(j)}(p)$ é a j-ésima diferenciação de $f$ no ponto $p$.
	\end{definicao}
	
	Daqui em diante assumiremos a $\cktopologia{\infty}$ para $\aplicaoessuaves{\circulo}{\real{m}}$ e a topologia da convegência uniforme para $\aplicaoessuavesloc{\retacartesianocirculo}{\real{m}}$. Além disso, assume-se que a topologia dos subconjuntos acima citados é a topologia induzida por essas normas, respectivamente.

	\begin{observacao}
		Em $\cite{palis_dynamical_systems}$ é demonstrado que, se $M$ é separável, então a $\cktopologia{\infty}$ é metrizável.
	\end{observacao}

	\vermelho{
	\begin{proposicao}
	Toda solução de classe $C^{1}$ da equação de Floer é de classe $C^{\infty}$. Além disso, em $\energiafinitaM$, as topologias $C^{0}_{loc}$, $C^{1}_{loc}$ e $C^{\infty}_{loc}$ são equivalentes.(não sei se é necessário incluir esse resultado)
	\end{proposicao}
	}
	
	A compacidade de $\energiafinitaM$ exerce um papel importante nos resultados desse capítulo. O teorema a seguir garante que, sob certas hipóteses, essa propriedade pode ser obtida, e sua demonstração pode ser encontrada em $\cite{audi_floer_homology}$.
	
	\begin{teorema}\label{teorema_compacidade_gromov}
		(Teorema da Compacidade de Gromov) Supondo a Condição de Aesfericidade $\ref{definicao_condicao_aesfericidade}$, então $\energiafinitaM$ é compacto em $C^{\infty}_{loc}(\retacartesianocirculo; M)$.
	\end{teorema}
	
	Temos o teorema a seguir tem sua importância destacada nas demais demonstração de outros resultados, além ser o equivalente ao Lema $\ref{lema_conjunto_limite_funcao_morse}$.
	
	\begin{teorema}
		Suponha que todas as trajetórias de $-\gradientefuncional \in \campossuaves{\lacocontrateis}$ sejam não-degeneradas. Então para cada $u \in \energiafinitaM$ existem dois pontos $\gamma, \gamma'\in \pontoscriticos{\funcionalH}$ tais que
		$$
		\lim_{s\to -\infty}u(s)=\gamma',\; \lim_{s\to \infty}u(s)=\gamma\;\;
		$$
		em $C^{\infty}(\circulo;M)$. Além disso, temos que
		$$
		\lim_{s\to \pm \infty}\derivadaparcial{}{s}u(s,t) = 0,
		$$
		converge uniformemente em $t\in \circulo$.
	\end{teorema}
	
	O lema a seguir caracterisza o conjunto dos pontos críticos do funcional Hamiltoniano e sua demonstração esta em $\cite{audi_floer_homology}$. 
	\begin{lema}
		Supondo que  trajetórias periódicas do campo Hamiltoniano $\campohamiltonianoabrev$ sejam não-degeneradas, então o conjunto $\pontoscriticos{\funcionalH}$ é finito. Logo, o conjunto das trajetórias de $\campohamiltonianoabrev$ é finito.
	\end{lema}

	\begin{comment}
		\begin{lema}
		(Floer) $\mathcal{M} = \bigcup \mathcal{M}(y,x)$, onde a união dos pares $y,x \in \lacocontrateis$, além disso, supondo a condição de aesfericidade, $\mathcal{M}$ será compacta.
		\end{lema}
		
		O lema abaixo é equivalente as condições de transversalidade em teoria finita de Morse, isto é, dada um função de Morse, o subconjunto de funções de Morse-Smale é um conjunto denso.
		
		\begin{lema}
		(Floer) Suponha que as soluções do sistema hamiltoniano sejam não-degenerados, então exite um conjunto denso $J_{reg} \subset C^{\infty}(End(TM))$ de estruturas quase-complexas tais que dado $J \in J_{reg}$ e cada par $y,x \in \lacocontrateis$ o espaço $\mathcal{M}(y,x)$ é uma variedade de dimensão finita.
		\end{lema}
		
		\textbf{Observação:} o lema acima afirma que, dada uma métrica gerada pelo par $\omega$ e J, teremos um conjunto denso de métricas geradas por outras estruturas complexas. Como o gradiente $\nabla f_{H}$ depende da métrica que adotamos então, como consequencia do lema, dado um gradiente sempre conseguimos construir um outro gradiente suficientemente proximo.
		
	
	A estratéria dada por Floer para a demonstração desse lema 
	é a utilização de operadores diferenciais lineares  definidos por 
	$$
	F_{u} \xi = \nabla_{s}\xi+J(u)(\nabla_{t}\xi - \nabla_{s}X_{H}(u,t))+\nabla_{\xi}J(u)\Big(\derivadaparcial{u}{t}-X_{H}(u,t)\Big),
	$$
	onde $\xi \in C^{\infty}(u^{*}TM)$, $u \in \mathcal{M}$ e $\nabla$ é a conexão riemanniana definida pela métrica anterior. Sendo que sua forma adjunta, de acordo com a mesma métrica, será dada por:
	$$
	F_{u}^{*} \beta = -\nabla_{s}\beta+J(u)(\nabla_{t}\beta - \nabla_{s}X_{H}(u,t))+\nabla_{\beta}J(u)\Big(\derivadaparcial{u}{t}-X_{H}(u,t)\Big).
	$$
	Podemos então definir 	os campos vetorias de quadrado integrável localmente como sendo a solução fraca da equação $F_{u}\xi = \beta$ se 
	$$
	\int_{\real{}} \int_{S{^{1}}} \iprod{F_{u}^{*}\beta}{\xi} dsdt= 
	\int_{\real{}} \int_{S{^{1}}} \iprod{\beta}{F_{u}\xi} dsdt.
	$$
	Defininindo o espaço de Hilbert $L^{2}(u) = \{\xi \in u^{*}TM: \int_{\real{}} \int_{S{^{1}}}|\xi|^{2} dsdt < \infty\}$ e o espaço $W^{1,2}(u) = \{\xi \in L^{2}(u):\nabla_{s} \xi,\nabla_{t} \xi \in L^{2}(u) \}$, podemos ver que $F_{u}: W^{1,2}(u) \to L^{2}(u)$ é um operador linear. \vermelho{Então a regularidade local dos operadores elípticos nos fornece a seguinte estimativa:}
	
	\begin{definicao}
		$L^{2}_{loc} = \{\xi \in u^{*}TM: \int_{K} |\xi(s,t)|^{2}dsdt <\infty, \; \text{compacto K} \subseteq M \}$ é é o conjunto de todos os campos vetoriais de quadrado integrável localmente.
	\end{definicao}
	
	\begin{lema}
		Sejam $\xi, \beta \in L^{2}_{loc}$, então $\nabla_{s}\xi, \nabla_{s}\xi \in L^{2}_{loc}$ (no sentido distribucional) e além disso
		$$
		\int_{-T}^{T}\int_{S^{1}}(|\xi|^{2}+|\nabla_{s}\xi|^{2}+|\nabla_{t}\xi|^{2})dsdt \leq c \int_{-T}^{T}\int_{S^{1}}(|\xi|^{2}+|F_{u}\xi|^{2})dsdt,
		$$
		onde c depende do parâmetro T e não de $\xi$.
	\end{lema}
	
	\begin{definicao}
		Um operador de Fredholm $F: A\to B$ é um operador linear limitado entre dois espaços de Banach com as dimensões finitas do $dim(Ker(F))< \infty$, $dim(Coker(F))< \infty$ e imagem fechada ($Im(F)=\overline{Im(F)}$). Chamamos de índice do operador de $ind (F)  = dim(Ker(F)) - dim(Coker(F))$.
	\end{definicao}
	
	\begin{lema}
		(Operador de Fredholm) Supondo que todas as soluções das equações de Hamilton sejam não-degeneradas, então dado $u \in \mathcal{M}(y,x)$ temos que $F_{u}$ é um operador de Fredholm. Além disso, $F_{u}$ é sobrejetor para quaisquer $J \in J_{reg}$ e $u \in \mathcal{M}(y,x)$.
	\end{lema}
	
	\textbf{Observação:} Para se definir o conceito de índice de Morse relativo vamos utilizar a definição de índice de $F_{u}$, o que dará origem a homologia de Floer. Sabemos que $\mathcal{M}(y,x)$ é uma variedade diferenciável, e que $F_{u}$ é sobrejetor, então existe uma vizinhaça de $u \in \mathcal{M}(y,x)$ que é difeomorfa a uma vizinhança de $0 \in Ker(F_{u})$, logo nessa vizinhança teremos $dim(\mathcal{M}(y,x)) = ind(F_{u})$.
	
	\begin{lema}\label{indice_ponto_critico} Sejam $z,y,x \in \lacocontrateis$ então $\mu(y,x) = ind(F_{u})$ é independente de $u \in \mathcal{M}(y,x)$. Além disso
		$$
		\mu(z,y) + \mu(y,x) = \mu(y,x), \; \mu(x,x) = 0.   
		$$
	\end{lema}
	Em Teoria de Morse clássica temos uma função de Morse $f:M\to \real{}$ e seus conjunto de pontos críticos não-degenerados $Crit(f)$, sendo que a cada ponto $x \in Crit(f)$ definimos uma aplicação $m:Crit(f) \to \inteiros$ tal que $m(x)$ é chamado índice de Morse de $x$.	O Lema $\ref{indice_ponto_critico}$ nos dá um candidato a índice dos pontos críticos pois, assim como o índice de Morse, $ind(F_{u}) \in \inteiros$ é um invariante topológico. Por outro lado vimos que $\mu(y,x) = ind(F_{u})$ é uma função que depende do par $(y, x)$ que define a variedade de conexão $\mathcal{M}(y, x)$. Então podemos afirmar que existe uma função $m : \lacocontrateis \to \inteiros$ definida a menos de um inteiro, tal que, dado $u \in \mathcal{M}(y,x)$ o índice do operador será 
	$$
	k = ind(F_{u}) = m(y)-m(x).
	$$
	Note que dessa definição surge uma abiguidade na função que determina o índice de cada ponto $y,x \in \lacocontrateis$ pois podemos escrever $m(y) = k+m(x)$, isto é, para se obter o índide de $y$ devemos ter o índice de $x$, e vice-versa. A seguir vamos definir uma função com as características necessárias para removermos essa ambiguidade. Tal função será chamada de índice de Maslov e a utilizaremos para graduarmos o complexo de cadeia, o que permitirá a construção da Homologia de Floer. 
	
	\end{comment}
		
	
	\section{As soluções da equação de Floer}

	Suponha que $H$ seja uma função Hamiltoniana dependente do tempo e $x^{-}, x^{+}$ sejam duas soluções 1-periódicas do sistema Hamiltoniano, isto é, $x^{-}, x^{+} \in \pontoscriticos{\funcionalH}$. Então, para toda estrutura complexa $J \in \estruturascomplexas{M}{\omega}$ podemos considerar $\energiafinitaMconectante \subset \energiafinitaM$ como sendo o conjunto das soluções da equação de Floer que conectam os pontos críticos $x^{-}$ e $x^{+}$. Vamos demonstrar que $\energiafinitaMconectante$ é uma variedade cuja dimensão esta relacionada com os índices de Maslov de cada uma das soluções $x^{-}$ e $x^{+}$ do seguinte modo $dim(\energiafinitaMconectante) = \mu(x^{-})-\mu(x^{+})$. Para isso vamos descrever o conjunto $\energiafinitaMconectante$ como sendo os zero de uma seção do fibrado vetorial na variedade de Banach $\caminhosexponenciaisconectantespadrao$, onde $\caminhosexponenciaisconectantespadrao$ são os caminhos de decaimento exponencial que conectam $x^{-}, x^{+}$ de modo que teremos $\energiafinitaMconectante \subset \caminhosexponenciaisconectantespadrao$. Aplicando a versão para dimensões infinitas do Teorema de Sard, chamado de Sard-Smale, mostraremos que $\energiafinitaMconectante$ é a pré-imagem de uma valor regular de uma determinada aplicação, sendo que para isso, deveremos mostrar que o diferencial dessa aplicação é um operador de Fredholm. Esse operador é chamado de operador de Fleor $\operadorflor$ e seu diferencial será denotador por $\diferencialfloer$. Mostraremos que, o índice do operador $\diferencialfloer$ é $\mu(x^{-}) - \mu(x^{+})$, portanto 
	$$
	dim(\energiafinitaMconectante)=ind(\diferencialfloer) <\infty.
	$$
	
	Seja $T:A\to B$ um operador linear entre espaços vetoriais. Sabse-se que $T$ é um isomorfimos se, e somente se, $Ker(T) = 0$ e $T(A) = B$, isto é, $B/T(A) = 0$. De forma equivalente, $T$ é um isomorfismo se, e somente se, $dim(Ker(T)) = 0$ e $dim(T(A)) = 0$. Suponha que $T$ sejá limitado e que existam subespaços fechados $V \subset A$ e $W \subset B$ tais que $A=V\oplus Ker(T)$ e $B=W\oplus T(A)$. Pode-se mostrar que $B/T(A) \cong W$ e $T:V \to T(A)$ é um isomorfismo (veja $\cite{abramovich}$). Nesse sentido, fica claro que as dimensões dos espaços $Ker(T)$ e $W$ são indicadores de "quanto o operador" se distancia de um isomorfismo. Com essa motivação define-se os operadores de Fredholm.
	
	
	\begin{definicao}\label{definicao_oeprador_fredholm}
		(Operador de Fredholm) Sejam $A, B$ espaços de Banach e $T: A\to B$ um operador linear limitado. O cokernel de $T$ é o quociente $Coker(T)=B/T(A)$. O operador $T$ é chamado de operador de Fredholm se $T(A)=\overline{T(A)}$ (imagem fechada) e $k(T) = dim(Ker(T)) < \infty$ e $c(T)=dim(Coker(T)) < \infty$. O índice do operador $T$ é o número inteiro $ind(T) = k(T) - c(T)$. 
	\end{definicao}
	
	Os resultados a seguir serão utilizados na construção e demonstração de estabilidade do operador de Fredholm adiante. Vamos denotar $\operadoresfredholm{A}{B}$ o conjunto dos operadores de Fredholm de $A$ em $B$ e $\operadoreslimitados{A}{B}$ o conjunto dos operadores limitados definidos nos mesmo espaços de Banach.
	
	\begin{teorema}\label{teorema_estabilidade_fredholm}
		(Estabilidade de Fredholm) Sejam  $T \in \operadoresfredholm{A}{B}$ e $\epsilon>0$. Dado $K \in \operadoreslimitados{A}{B}$ tal que $||K|| < \epsilon$ temos que $T+K \in \operadoresfredholm{A}{B}$. Além disso, $ind(T+K)=ind(T)$ e $k(T+K) \leq k(T)$.
	\end{teorema}
	
	\begin{definicao}
		(Mapa de Fredholm) Uma aplicação suave $f: A \to B$ é chamada de mapa de Fredholm se o diferencial $df(p): A \to B$ é um operador de Fredholm para todo $p \in A$.
	\end{definicao}
	
	
	Pela estabilidade dos operadores de Fredholm, podemos afirmar que o índice de $df(p) \in \operadoresfredholm{A}{B}$ não depende do ponto $p \in A$, com isso podemos definir o índice do mapa de Fredholm como $ind(f)=ind(df(p))$ que é constante e assim esta bem-definido.
	
	\begin{definicao}\label{definicao_caminhos_decaimentos_exponenciais}
		(Caminhos de decaimentos exponenciais) Sejam $x^{-}, x^{+} \in \pontoscriticos{\funcionalH}$. Denotamos o conjunto dos caminhos suaves com decaimento exponencial conectando $x^{-}$ a $x^{+}, $ por
		$$
		\caminhosdecaimentoexponencialpadrao = \{u \in \aplicaoessuavesreatacirculo: \lim_{s \to \mp} u(s,t) = x^{\mp}(t) \},
		$$
		onde 
		$$
		\normagrande{\derivadaparcial{u}{s}(s,t)} \leq Ke^{-\delta|s|},  \normagrande{\derivadaparcial{u}{t}(s,t) -X_{H}(u)} \leq Ke^{-\delta|s|}.
		$$
	\end{definicao}
	
	\begin{observacao}
		$\caminhosdecaimentoexponencialpadrao$ é não-vazio pois as soluções da equação de Floer existem e satisfazem tais condições, logo são elementos desse conjunto. A estimativa do módulo das derivadas parciais pode ser encontrada em $\cite{audi_floer_homology}$. Além disso, essas estimativas indicam que nos limites $s\to \mp \infty$ teremos as soluções das esquações do sistema Hamiltonio.
	\end{observacao}
	
	Um espaço de Banach é um espaço vetorial $B$ munido de uma norma e completo com na métrica definida por essa norma. Contudo, sabe-se que o espaço das funções suaves não é um espaço de Banach, mas sim um espaço de Frechet (uma generalização dos espaços de Banach e é munido de uma semi-norma), portanto $\caminhosdecaimentoexponencialpadrao$ não é um espaço de Banach. Isso é um impedimento pois necessitamos de um espaço normada para as construções seguintes.
	
	\begin{comment}
		\vermelho{(ESTA BEM DIÍCIL JUSTIFICAR ESSA ESCOLHA. ACREDITO QUE SEJA PORQUE AS SOLUÇÕES DAS EQUAÇÕES DE FLOER DEVAM SER DE CLASSE C1).}
		
	\end{comment}
	De acordo com a Definição  $\ref{definicao_espaco_Lp}$ o espaço $L^{p}(\Omega)$ é um espaço de Banach, o que esta demonstrado em $\cite{breazis_sobolev_spaces}$. Como a equação de Floer envolve derivadas de primeira ordem, o espaço de Banach com o qual trabalharemos será munido de uma norma que leva em consideração o comportamento dessas derivadas. Tal espaço normado é chamado por espaço de Sobolev.
	
	Uma segunda dificuldade técnica é que $\caminhosdecaimentoexponencialpadrao$ não tem estrutura de espaço vetorial. Para contornar esse problemas vamos adotar como espaço normado os espaços de Sobolev generalizados do que foi feito na Seção $\ref{secao_espaco_sobolev}$. 
	
	\begin{observacao}\label{observacao_escolha_espacos_sobolev}
		Pelo Teorema $\ref{teorema_sobolev}$ temos que, para um dado $\Omega \subset \real{n}$, a inclusão $\espacosobolev{\Omega} \hookrightarrow L^{q}(\Omega)$ é contínua para $1/q = 1/p - 1/n$. Lembrando que estamos tratando a variedade $M$ como um mergulho em um $\real{m}$ para algum $m>0$ suficientemente grande. Com isso, as aplicacões $u \in \energiafinitaM$ podem ser vistas como $u:\retacartesianocirculo\to M \subseteq \real{m}$. Assim, a topologia de $\espacosobolevcontradominio{\retacartesianocirculo}{M}$ é induzida pela topologia de $\espacosobolevcontradominio{\retacartesianocirculo}{\real{m}}$. Temos $\Omega = \retacartesianocirculo$, que é um cilíndro mergulhado em $\real{n}$ para algum $n\geq3$ o que implica em $ p\geq 2$. Essa condição é a restrição de escolha dos espaços de Sobolev com os quais trabalharemos.
	\end{observacao}
	
	\begin{definicao}
		(Exponenciais conectantes) Sejam $x^{-}, x^{+} \in \pontoscriticos{\funcionalH}$, $w \in \caminhosdecaimentoexponencialpadrao$ e $X \in \espacosobolev{\pullbackfibradotangenteM{w}}$. Denotamos por exponenciais conectantes  o conjunto $\caminhosexponenciaisconectantespadrao$ o das aplicações $exp(w,X):\retacartesianocirculo \to M$ tal que $exp(w,X)(s,t) = \aplicacaoexponencial{w(s,t)}{X(s,t)}$.
	\end{definicao}
	
	\begin{proposicao}
		$\energiafinitaMconectante \subset \caminhosdecaimentoexponencialpadrao \subset  \caminhosexponenciaisconectantespadrao$.
	\end{proposicao}
	\begin{prova}
		Seja $u \in \energiafinitaMconectante$, então pelas condições de contorno das soluções temos $u(s,t)  \to x^{\mp}(t)$ quando $s\to \mp \infty$, logo $\derivadaparcial{u}{s} \to 0$ e $\derivadaparcial{u}{t} \to X_{H}(x^{\mp})$ quando $s\to \mp \infty$, portanto $u \in \caminhosdecaimentoexponencialpadrao$. Suponha que $u \in \caminhosdecaimentoexponencialpadrao$ então tomando $X \in \espacosobolev{\pullbackfibradotangenteM{u}}$. Definindo $\gamma(s, t) =\aplicacaoexponencial{u(s, t)}{v(s, t)}$, podemos ver que $\gamma$ satisfaz as condições de decaímento exponencial, logo $\gamma \in \caminhosdecaimentoexponencialpadrao$, mas por construção temos que $\gamma \in \caminhosexponenciaisconectantespadrao$. Como cada $u \in \caminhosdecaimentoexponencialpadrao$ temos uma única $\gamma \in \caminhosexponenciaisconectantespadrao$, então $\caminhosdecaimentoexponencialpadrao \subset \caminhosexponenciaisconectantespadrao$. 
	\end{prova}
	
	Quando não houver riscos de ambiguidades adotaremos a notação de $\caminhosexponenciaisconectantesabrev$ em vez de $\caminhosexponenciaisconectantespadrao$.
	
Pelo Teorema $\ref{teorema_sobolev}$ temos que $\espacosobolevcontradominio{\retacartesianocirculo}{\real{2n}} \subset L^{\infty}(\retacartesianocirculo, \real{2n})$. Então dado $X \in \espacosobolevcontradominio{\retacartesianocirculo}{\real{2n}}$, existem constantes $A, B >0$ tais que $A\normaWp{X}\leq \norma{X}_{\infty} \leq B\normaWp{X}$, portanto, ambas as normas geram a mesma topologia. Como $L^{\infty}(\retacartesianocirculo, \real{2n})$ é um espaço de Banach, então pela inclusão e compatibilidade das normas, $\espacosobolevcontradominio{\retacartesianocirculo}{\real{2n}}$ é um espaço de Banach.
	
	\begin{proposicao}\label{proposicao_variedade_banach}
		Sejam $x^{-}, x^{+} \in \pontoscriticos{\funcionalH}$ soluções contráteis do sistema Hamiltoniano, então $\caminhosexponenciaisconectantespadrao$ é uma variedade de Banach.
	\end{proposicao}
	\begin{prova}
		Pela Proposição $\ref{proposicao_difeomorfismo_exponencial}$, dado $\epsilon > 0$, existe uma vizinhaça aberta $B_{\epsilon}(0)$ de seção nula $0 \in \espacosobolev{\pullbackfibradotangenteM{w}}$ tal que a restrição da aplicação exponencial $\aplicacaoexponencialgeral{w}:B_{\epsilon}(0) \to V$ é um difeomorfismo sobre um aberto $V \subset \pullbackfibradotangenteM{w}$. Para cada $X \in \espacosobolev{\pullbackfibradotangenteM{w}}$ temos a aplicação $\aplicacaoexponencial{w}{X}:\retacartesianocirculo \to M$ tal que $\aplicacaoexponencial{w}{X}(s,t) = \aplicacaoexponencial{w(s,t)}{X(s,t)}$, logo $\aplicacaoexponencial{w}{X} \in \caminhosexponenciaisconectantespadrao$. Portanto o par $(B_{\epsilon}(0), \aplicacaoexponencialgeral{w})$ de abertos e difeomorfismos é uma carta de $\caminhosexponenciaisconectantespadrao$, consequentemente a família indexada $\{(B_{\epsilon_{w}}(0), \aplicacaoexponencialgeral{w})\}_{w \in \caminhosexponenciaisconectantesabrev}$, forma um 1-atlas. Portanto, $\caminhosexponenciaisconectantespadrao$ é uma variedade de Banach.
	\end{prova}
	
	\section{A identificação  $\espacosobolev{\pullbackfibradotangenteM{u}} \mapsto \espacosobolevcontradominio{\retacartesianocirculo}{\real{k}}$}\label{secao_identificacao_pullback_fibrado}
	Em muitas situações  vamos nos deparar com a necessidade de efetuar diferenciação em variedades com estruturas diferenciais complexas tornando dificultanto muito as técnicas utilizadas na demonstração dos resultados desejado. Para contornar esse dificuldade adicional vamos utilizar o Teorema de Whitney, enunciado aqui, mas demonstrado em $\cite{guillemin_differential_topology}$, e com isso comparar campos vetoriais se torna trivial, já que estamos em um espaço euclidiano e os espaços tangentes estão todos identificados com o próprio $\real{k}$ (para algum k), além disso, as conexões afins (associadas ao transporte paralelo de campos vetoriais em variedades Riemannianas) são reduzidas a derivadas parciais. Veja uma breve descrição na Observação $\ref{observacao_conexao_afim}$.
	
	\begin{definicao}\label{definicao_mergulho_variedades}
		(Mergulho de variedades) Sejam $M, N$ m,n-variedade s diferenciávei, respectivamente. Uma aplicação diferenciável $\psi:M\to N$ é uma imersão se $D\psi_{p}:T_{p}M\to T_{\psi(p)}N$ é injetora para todo $p \in M$. Dizemos que $\psi$ é um mergulho se for uma imersão e um homeomorfismo sobre $\psi(M) \subseteq N$, na topologia induzida por $N$. 
	\end{definicao}
	
	\begin{teorema}\label{teorema_whitney}
		(Mergulho de Whitney) Toda variedade n-dimensional pode ser mergulhada em $\real{2n+1}$.
	\end{teorema}
	
	Por definição temos que, dado $w : \retacartesianocirculo \to M $ com $w \in C^{\infty}$ temos $\pullbackfibradotangenteM{w} = \{(p, X(p)) : p = w(s,t) \in M, \; X(s,t) \in T_{w(s,t)M} \}$. Pelo Teorema $\ref{teorema_whitney}$ (Whitney) temos que $\pullbackfibradotangenteM{w} \subset T\real{k}$, para algum $k \geq 2 dim(\pullbackfibradotangenteM{w}) +1$. Seja $\psi : \pullbackfibradotangenteM{w} \to \real{k}$ o mergulho, então a composição $\Psi=\psi \circ X:\retacartesianocirculo \to \real{k}$ esta bem-definida, logo $\Psi \in \espacosobolevcontradominio{\retacartesianocirculo}{\real{k}}$. 
	
	\section{O operador de Floer $\operadorflor$ e seu diferencial $\diferencialfloer$}
	
		
		\begin{definicao}
			(Operador de Floer)  Seja $\cilindrosLM = \{u:\retacartesianocirculo\to M : u \;\text{é} \;C^{\infty}\}$. O operador $\operadorflor: \cilindrosLM\to TM$ é definido por 
			$$
			\operadorflor = \derivadaparcial{}{s} + J(.)\derivadaparcial{}{t} -J(.) \campohamiltoniano{.},
			$$
			onde $\operadorflor(u) \in \pullbackfibradotangente{u}{M}$.
		\end{definicao}
	
	Sabemos que $\caminhosexponenciaisconectantesabrev$ é um espço de Banach, portanto é um espaço topológico com sua topologia gerada pela norma. Definindo $\fibradocaminhosexponenciais =\{(u, X): u \in \caminhosexponenciaisconectantesabrev,\; X \in L^{p}(\pullbackfibradotangenteM{u})\}$, temos que a tripla $
	\fibradocaminhosexponenciaisabrev= (\fibradocaminhosexponenciais, \caminhosexponenciaisconectantesabrev, \pi)
	$ é um fibrado vetorial sobre $\caminhosexponenciaisconectantesabrev$  onde $\pi: \fibradocaminhosexponenciais \to \caminhosexponenciaisconectantesabrev$ é tal que $\pi(u, X)=u$ é a projeção e $\pi^{-1}(u) = L^{p}(\pullbackfibradotangenteM{u})$ é uma fibra em $u$.
	
	\begin{definicao}
		(Operador de Floer) O operador de Floer é a aplicação $\operadorflor: \caminhosexponenciaisconectantesabrev \to \fibradocaminhosexponenciaisabrev$ tal que
		$$
		\operadorflorparametro{u} =\operadorflordefinicao{u}.
		$$
	\end{definicao}
	
	Seja $u \in \caminhosexponenciaisconectantesabrev$, então
	$$
	\begin{aligned}
	\normaLp{\operadorflorpadrao} &= \normagrandeLpdefinicao{\operadorflordefinicao{u}}{\retacartesianocirculo}
	\\
	&\leq \normaLpdefinicao{\normagrande{\derivadaparcial{u}{s}} +|J|\normagrande{\derivadaparcial{u}{t} - X_{H}(u)}}{\retacartesianocirculo}
	\\
	&\leq \normaLpdefinicao{(Ke^{-\delta|s|} +|J|Ke^{-\delta|s|})}{\retacartesianocirculo}
	\\
	&\leq G \int_{\retacartesianocirculo}e^{-\delta|s|p}
	\\
	&< \infty,
	\end{aligned}	
	$$ 
	portanto $\operadorflorparametro{\caminhosexponenciaisconectantesabrev} \subset L^{p}(\pullbackfibradotangenteMpadrao)$ e o operador de Floer esta bem-definido.
	
	Por hipótese $M$ é uma 2n-variedade Riemanniana, consequentemente, $\pullbackfibradotangenteM{w}$ também é uma variedade Riemannina para todo $u \in \caminhosexponenciaisconectantesabrev$. Portanto, para realizarmos a diferencição de $\operadorflor$ devemos comparar os campos vetoriais $\operadorflorparametro{w}$ a $\operadorflorparametro{w'}$ através do transporte paralelo descricao na Observação $\ref{observacao_transporte_paralelo}$, obrigando-nos a trabalhar com conexões afins, levando complexidade adicional à análise das características de $D\operadorflor_{w}$. Em termos do transporte paralelo teríamos a seguinte expressão:
	$$
	\nabla\operadorflorparametro{w}(\xi) = \nabla_{s}\xi + J(w)\nabla_{t}\xi + dJ(w)(\xi)(\xi - X_{H}(w)) - J(w)(dX_{H})_{w}(\xi).
	$$ 
	
	Vamos recorrer novamente o Teorema $\ref{teorema_whitney}$ (teorema de Whitney) para simplificar essa análise. Seja $\pullbackfibradotangenteM{w} \subset \real{k}$ o mergulho de Whitney, então todo $(s,t) \in \retacartesianocirculo$ temos $w(s,t) \in \real{k}$, logo  $ w(s,t)+\xi \in \real{k}$ para todo $\xi \in \real{k}$. Com isso:
	$$
	\begin{aligned}
	\operadorflorparametro{w+\xi}-\operadorflorparametro{w}
	&= \operadorflordefinicao{w+\xi}
	\\
	&\;\;\;\; -\derivadaparcial{w}{s} - J(w)\derivadaparcial{w}{t} + J(w)\campohamiltoniano{w}
	\\
	&=\derivadaparcial{\xi}{s}+J(w+\xi)\derivadaparcial{\xi}{t}+\bigparenteses{J(w+\xi)-J(w)}\derivadaparcial{w}{t}
	\\
	&\;\;\;\;-\bigparenteses{J(w+\xi)-J(w)}\campohamiltoniano{w} 
	-J(w+\xi)\bigparenteses{\campohamiltoniano{w+\xi}-\campohamiltoniano{w}}
	\\
	&=\derivadaparcial{\xi}{s} + J(w+\xi)\derivadaparcial{\xi}{t} + \bigparenteses{J(w+\xi)-J(w)}\bigparenteses{\derivadaparcial{w}{t} - X_{H}(w)}  
	\\
	&\;\;\;\; -J(w+\xi)\bigparenteses{X_{H}(w+\xi) - X_{H}(\xi)} 
	\end{aligned}
	$$
	Como os termos $w, \xi, J, X_{H}$ são de classe $C^{\infty}$, então o diferencial de $\operadorflor$ avaliado no ponto $w\in \caminhosexponenciaisconectantesabrev$ é o limite 
	$$
	\begin{aligned}
		\diferencialfloer_{w}(\xi)
		&=\lim_{\xi \to 0} \frac{	\operadorflorparametro{w+\xi}-\operadorflorparametro{w}}{\norma{\xi}}
		\\
		&= \derivadaparcial{\xi}{s} + J(w)\derivadaparcial{\xi}{t}+ dJ_{w}(\xi)\bigparenteses{\derivadaparcial{w}{t} - X_{H}(w)} - J(w) \diferencialhamiltoniano{w}(\xi)
		\\
		&= \underbrace{\Big( \derivadaparcial{}{s} + J(w)\derivadaparcial {}{t}\Big)}_{\overline{\partial}_{w}}(\xi)+ \underbrace{\Big(dJ_{w}(.)\bigparenteses{\derivadaparcial{w}{t} - X_{H}(w)} - J(w) \diferencialhamiltoniano{w}\Big)}_{S_{w}}(\xi)
		\\
		&= \operadorcauchyabrev{w} \xi + S_{w}\xi
		\\
		&= (\operadorcauchyabrev{w}  + S_{w})(\xi).
	\end{aligned}
	$$
	Portanto, temos $ D\operadorflor_{w}= \operadorcauchyabrev{w}  + S_{w}: \espacotangenteponto{w}{\caminhosexponenciaisconectantesabrev} \to \espacotangenteponto{\operadorflor(w)}{\fibradocaminhosexponenciaisabrev}$.
	
	\begin{observacao}
		Note que, ao mergulharmos a variedade $M$ em um espaço euclidiano com dimensão suficientemente grande, temos que a conexão afim se reduz a operação de parcial, isto é,  $\nabla=\partial$. Com isso, temos uma identidade entre a equação escrita em termos das conexões afins e a e equação obtida via mergulho em $\real{m}$. Portanto, $\nabla\operadorflor_{w} = \diferencialfloer_{w}$.
	\end{observacao}
	
	\section{$D\operadorflor$ é um operador de Fredholm}
	A estratégia não é demonstrar diretamente que o diferencial do operador de Floer é um operador de Fredholm, mas caracterizá-lo indiretamente, sendo que para isso utilizaremos o Lema $\ref{lema_desigualdade_operador_compacto}$ abaixo.
	
	\begin{definicao}\label{definicao_operador_compacto}
		(Operador Compacto) Sejam $A, B$ espaços normados. Então um operador linear $T:A\to B$ é chamdado de operador linear compactor se, para cada $W \subset A$ limitado, $T(W)$ é relativamente compacto, isto é, $\overline{T(W)} $ é um compacto.
	\end{definicao}
	
	\begin{observacao}\label{observacao_operador_compacto_limitado}
		Pode-se mostrar que todo operador compacto entre espaços de Banach é limitado. Veja $\cite{kreyszig_analise_funcional}$.
	\end{observacao}
	
	\begin{lema}\label{lema_desigualdade_operador_compacto}
		Sejam $A,B,C$ espaços de Banach e $D:A\to B$ um operador limitado e $K:A \to C$ um operador compacto. Supondo que exista uma constante $c>0$ tal que $\normasubscrito{\xi}{A} \leq c(\normasubscrito{D\xi}{B} + \normasubscrito{K\xi}{C})$, então $D(A) = \overline{D(A)}$ e $dim(Ker(D)) <\infty$.
	\end{lema}
	
	Para simplificar a notação, definiremos $\diferencialfloerabrev = D\operadorflor_{w}$ para um dado $w\in \caminhosexponenciaisconectantespadrao$.
	
	O seguinte lema afirma a injetividade do operador de Floer $\diferencialfloerabrev$, o que implicará diretamente que $\diferencialfloer$ é um operador compacto. Aqui vamos demonstrar apenas particularização $p=2$. O caso geral em que $p\geq 2$ pode ser encontrado em $\cite{salamon_lecture}$.
	\begin{lema}\label{lema_diferencial_floer_bijecao}
		Suponha que $S(s,t) = S(t)$, ou seja, não dependa de $s$, e que $det(id - \Psi(1)) \neq 0 $, onde $\Psi:[0,1] \to \gruposimpletico{2n}$ é solução do distema $\dot{\Psi}(t)=\estruturacomplexa S(t)\Psi(t)$. Então $\diferencialfloerabrev = \operadorcauchyabrev{w} + S_{w}: \espacosobolevretacirculo \to \espacoLpretacirculo$ é bijetor para $1 < p <\infty$.
	\end{lema}
	\begin{prova}
		Como $S$ não depende do parametro $s \in \real{}$, então podemos definir $A = \estruturacomplexa \derivadaparcialabrev{t} + S: \espacosobolevgeral{2}{\circulo;\real{2n}} \to \espacoLdois{\circulo;\real{2n}}$. 
		\begin{comment}
			\vermelho{O operador $A$ é auto-adjunto e ilimitado no espaço de Hilbert $H=\espacoLdois{\circulo;\real{2n}}$ (não sei demonstrar isso e acredito que possa ser retirado da demonstração).} 
		\end{comment}
		Note que $A\Psi(t) = 0$ se, e somente se, é solução do sistema $\dot{\Psi}(t)=\estruturacomplexa S(t)\Psi(t)$, isto é, $\Psi \in Ker(A)$. Vamos mostrar que $A$ é invertível, e com isso, decomporemos $\espacosobolevgeral{2}{\circulo;\real{2n}} $ em uma soma direta de seus auto-espaços.
		
		Por simplicidade denotaremos $W = \espacosobolevgeral{2}{\circulo;\real{2n}} $. Afirmar que $A$ é invertível é equivalente a determinar um único $\Psi \in W$ para um dado $\zeta \in H$ tal que $A\Psi(t) = \zeta(t)$. Caso $\zeta = 0$, então temos um sistema homogêneo cuja solução é da forma $\Psi(t) = R(t)\Psi_{0}$, onde $\Psi(0)=\Psi_{0}$ é a condição inicial que determina unicamente a solução, logo $R(0) = Id$. Pelo Teorema Fundamental de existência e unicidade das soluções de EDO $\cite{pontryagin_ode}$ podemos afirmar que, para um dado $\Psi_{0}$ existe um único $R$ tal que $\Psi$ seja solução do sistema, isto é, $\Psi_{0} \mapsto \Psi$ da forma $\Psi = R\Psi_{0}$ e, analogamente, $\Psi \mapsto \Psi_{0}$ da forma $\Psi_{0}=R^{-1}\Psi$. Aplicando $A$ na solução $R(t)\Psi_{0}$ temos um sistema de equações para $R$ dado por 
		$0=A\Psi(t) = A(R(t)\Psi_{0}) = (\estruturacomplexa\derivadaparcialabrev{t}+S)R(t)\Psi_{0}$, o que implica em $\dot{R}(t) = \estruturacomplexa SR(t)$ com $R(0) = Id$. 
		
		Para determinar a solução da equação mais geral ($\zeta\neq 0$), aplicaremos o método de variação de constantes 
		\begin{comment}
			\vermelho{(procurar uma referencia)} 
		\end{comment}
		que consiste em analisar uma solução mais geral da forma $\Psi(t) = R(t)\Psi_{0}(t)$ para $t \in [0,1]$. Aplicando o operador $A$ na solução $R(t)\Psi_{0}(t)$ e usando o fato de que $\dot{R}(t) = \estruturacomplexa SR(t)$ e $R(t)$ é invertível, teremos a seguinte condição sobre $\Psi_{0}(t)$
		$$
		\begin{aligned}
		\zeta(t) 
		&= A\Psi(t)
		\\ 
		&= (\estruturacomplexa\derivadaparcialabrev{t}+S)(R(t)\Psi_{0}(t)) 
		\\ 
		&= \estruturacomplexa\dot{R}(t)\Psi_{0}(t)+\estruturacomplexa R(t)\dot{\Psi}_{0}(t) + SR(t)\Psi_{0}(t)
		\\
		&= -SR(t)\Psi_{0}(t)+\estruturacomplexa R(t)\dot{\Psi}_{0}(t) + SR(t)\Psi_{0}(t)
		\\
		&= \estruturacomplexa R(t)\dot{\Psi}_{0}(t),
		\\
		&\therefore 
		\\
		\dot{\Psi}_{0}(t) &= -R^{-1}(t)\estruturacomplexa \zeta(t).
		\end{aligned}
		$$
		
		Integrando ambos os lados da equação, temos
		$$
		\Psi_{0}(t) = \Psi_{0}(0)-\int_{0}^{t} R^{-1}(\alpha)\estruturacomplexa \zeta(\alpha)
		$$
		e
		$$
		\Psi(t) = R(t)\Big(\Psi_{0}(0)-\int_{0}^{t} R^{-1}(\alpha)\estruturacomplexa \zeta(\alpha)\Big).
		$$
		
		Aplicando a condições de periodicidade $\Psi(1) = \Psi(0)$ temos
		$$
		\Psi_{0}(0)=\Psi(0)=\Psi(1)=R(1)\Big(\Psi_{0}(0)-\int_{0}^{1} R^{-1}(\alpha)\estruturacomplexa \zeta(\alpha)\Big)
		$$
		o que implica em
		
		$$
		(Id - R(1))\Psi_{0}(0) = -R(1)\int_{0}^{1} R^{-1}(\alpha)\estruturacomplexa \zeta(\alpha).
		$$
		
		Temos que $R \in Ker(A)$ e pela hipótese $det(Id - R(1))\neq 0$, portanto o operador $Id - R(1)$ é invertível. Com isso a condição inicial $\Psi_{0}(0)$ esta unicamente determinada por
		$$
		\Psi_{0}(0) = -(Id - R(1))^{-1}R(1)\int_{0}^{1} R^{-1}(\alpha)\estruturacomplexa \zeta(\alpha).
		$$
		
		O Teorema Funcamental de Existência e Unicidade das soluções de EDO (veja em $\cite{pontryagin_ode}$ ou $\cite{doering_ode}$) garante que existe $\Psi \in W$ solução do sistema $A\Psi = \zeta$ determinada unicamente pela condição inicial $\Psi(0) = \Psi_{0}(0)$. A equação anterior afirma que, para uma dada $\zeta \in H$, existe uma única $\Psi_{0}(0)$, logo $\zeta$ deteremina unicamente a solução $\Psi$ e podemos escrever $A^{-1}\zeta(t)=\Psi(t)$, ou seja, $A$ é invertível.
		
		Como $A$ é invertível, então $0 \notin \sigma(A)$, isto é, os auto-valores de $A$ são não-nulos. Sejam $E^{+}, E^{-}$ os auto-espaços de $A$ cujos auto-valores são positivos e negativos, respectivamente. Então temos a decomposição $W = \espacosobolevgeral{2}{\circulo;\real{2n}} = E^{+}\oplus E^{-}$. Definindo $A^{\pm}=A |_{E^{\pm}}$ e $P^{\pm}: W \to E^{\pm}$ as projeções ortogonais, $\real{}_{+} \ni s \mapsto exp(\mp s A^{\pm}) \in \mathcal{L}(W)$ e $K: \real{}\to \mathcal{L}(W)$.
		$$
		\funcaocond{K(s)}{exp(-s A^{+})P^{+}}{s \geq 0}{-exp(-s 	A^{-})P^{-}}{s<0}.
		$$
		
		Um semi-grupo fortemente contínuo sobre um espaço de Banach $B$ é uma aplicação $T:\real{}_{+}\to B$ tal que:
		
		\begin{enumerate}
			\item $T(0) = Id$,
			\item $T(s+t) = T(s)T(t), \;\forall s, t \in \real{}_{+}$
			\item $lim_{t \to 0^{+}}\norma{T(t)x - x } = 0, \; \forall x \in B $.
		\end{enumerate}
		Os dois primeiros  axiomas são algébricos, já o último é topológico e caracteriza a contínuidade do operador. Nesse sentido, a aplicação $exp(\mp A^{\pm}):\real{}_{+} \to \mathcal{L}(W)$ é um semi-grupo fortemente contínuo e $K$ é descontínua em $s=0$ pois $lim_{s\to 0^{+}}K(s) = P^{+}$ e $lim_{s\to 0^{-}} K(s)= -P^{-}$, mas é fortemente contínua para $s \in \real{}\backslash\{0\}$. Dado $s \in \real{}$ afirmo que $K(s)$ é um operador linear limitado. De fato, é linear pois é a composição dos operadores lineares $\pm exp(- sA^{\pm})$ e $P^{\pm}$. É limitado pois
		$$
		\begin{aligned}
		\norma{K(s)}_{\mathcal{L}(W)} &= \sup_{\norma{\xi}=1}\norma{K(s)\xi}
		\\
		&=\sup_{\norma{\xi}=1}\norma{\pm exp(- sA^{\pm})P^{\pm}\xi}
		\\
		&=\sup_{\norma{\xi}=1}\norma{exp(- sA^{\pm})\xi^{\pm}}
		\\
		&=\sup_{\norma{\xi}=1}\norma{exp(- s\lambda^{\pm})\xi^{\pm}}
		\\
		&\leq |exp(- s\lambda^{\pm})|\sup_{\norma{\xi}=1}\norma{\xi^{\pm}}
		\\
		&\leq exp(- s\lambda^{\pm})
		\\
		&\leq exp(- |s|\lambda),
		\end{aligned}
		$$
		onde fizemos $\xi^{\pm}$ os auto-vetores de $A^{\pm}$ com auto-valores $\lambda^{\pm}$ positivos e negativos, respectivamente. Na ultima desigualdade tivemos dois casos: o primeiro quando $s<0$ implicando em $- s\lambda^{\pm} = - s\lambda^{-} <0$, e o segundo quando $s>0$ implicando em $- s\lambda^{+}<0$. Definindo $\lambda = \inf \{|\alpha|: \alpha \in \sigma(A) \}$, temos a desigualdade desejada.
		
		Seja $Q: \espacoLdois{\real{};H} \to \espacosobolevgeral{2}{\retacartesianocirculo;\real{2n}} \cap \espacoLdois{\retacartesianocirculo;\real{2n}}$ tal que 
		$$
		Q(\eta(s))(t) = \int_{\real{}}K(s-\alpha)\eta(\alpha, t)d\alpha.
		$$
		
		Esse operador esta bem-definido pois, dado $\eta(s) \in \espacoLdois{\reta, H}$ temos
		$$
		\begin{aligned}
		\norma{Q(\eta(s))(t)}_{\espacoLdois{\retacartesianocirculo}} &= \normagrande{\int_{\real{}}K(s-\alpha)\eta(\alpha, t)d\alpha}_{\espacoLdois{\retacartesianocirculo}}
		\\
		&\leq
		\int_{\reta}\norma{K(s-\alpha)\eta(\alpha, t)}_{\espacoLdois{\retacartesianocirculo}}d\alpha
		\\
		&\leq
		\int_{\reta}\norma{K(s-\alpha)}\norma{\eta(\alpha, t)}_{\espacoLdois{\retacartesianocirculo}}d\alpha
		\\
		&\leq
		\int_{\reta}e^{-\delta|s-\alpha|}  \underbrace{\norma{\eta(\alpha, t)}_{\espacoLdois{\retacartesianocirculo}}}_{N < \infty}d\alpha
		\\
		&\leq
		N\int_{\reta}e^{-\delta|s-\alpha|} d\alpha
		\\
		&< \infty,	
		\end{aligned}
		$$
		
		portanto $Q(\eta(s)) \in \espacoLdois{\retacartesianocirculo;\real{2n}}$. Vamos mostrar que $Q(\eta(s)) \in \espacosobolevgeral{2}{\retacartesianocirculo;\real{2n}}$ através do cálculo
		$$
		\begin{aligned}
		Q(\eta(s)) 
		&= \int_{\reta}K(s-\alpha)\eta(\alpha)
		\\
		&=\int_{-\infty}^{s}exp(-(s-\alpha)A^{+})P^{+}\eta(\alpha)-\int_{s}^{\infty}exp(-(s-\alpha)A^{-})P^{-}\eta(\alpha)
		\\
		&=\underbrace{\int_{-\infty}^{s}exp(-(s-\alpha)A^{+})\eta^{+}(\alpha)}_{\xi^{+}} +\underbrace{\int_{s}^{\infty}-exp(-(s-\alpha)A^{-})\eta^{-}(\alpha)}_{\xi^{-}}
		\\
		&= \xi^{+}(s)+\xi^{-}(s).
		\end{aligned}
		$$
		
		Definindo $\xi = \xi^{+} + \xi^{-}$ e realizando a diferenciação de $\xi^{+}$ temos
		$$
		\begin{aligned}
		\derivada{\xi^{+}}{s} 
		&= \derivada{}{s}\int_{-\infty}^{s}exp(-(s-\alpha)A^{+})\eta^{+}(\alpha)
		\\
		&= exp(-0A^{+})\eta^{+}(s)+\int_{-\infty}^{s} \derivada{}{s} exp(-(s-\alpha)A^{+})\eta^{+}(\alpha) 
		\\
		&= \eta^{+}(s)-A^{+}\xi^{+}.
		\end{aligned}
		$$
		Realizando o mesmo cáculo para $\xi^{-}$ teremos 
		$$
		\eta^{\pm}(s) =\derivada{\xi^{\pm}}{s}+A^{\pm}\xi^{\pm}
		$$
		o que implica que
		$$
		\eta(s) =\derivada{\xi}{s}+A\xi = (\derivadaparcialabrev{s}+\estruturacomplexa \derivadaparcialabrev{t} + S)\xi(s) = \diferencialfloerabrev \xi(s) = (\diferencialfloerabrev \circ Q)\eta(s).
		$$
		Como $\eta(s) \in \espacoLdois{\reta; H}$ é arbitrário, então mostramos que $\diferencialfloerabrev \circ Q = Id$. Com isso, temos que a imagem de $Q$ é o domínio de $\diferencialfloerabrev$, logo $Q(\eta(s)) \in \espacosobolevgeral{2}{\retacartesianocirculo;\real{2n}}$. Vamos mostrar que $Q\circ \diferencialfloerabrev = Id$, sendo que para isso vamos calcular diretamente $Q(\diferencialfloerabrev\eta)=Q((\derivadaparcialabrev{s}+A)\eta) = \eta$ para um dado $\eta \in \espacosobolevgeral{2}{\retacartesianocirculo;\real{2n}}$. Primeiramente, calculemos 
		$$
		\begin{aligned}
		Q(\diferencialfloerabrev \eta^{+}(s))
		&=Q((\derivadaparcialabrev{s}+A)\eta^{+}(s))
		\\
		&=Q(\derivadaparcialabrev{s}\eta^{+}(s)) +Q(A\eta^{+}(s))
		\\
		&=Q(\derivadaparcialabrev{s}\eta^{+}(s))  + \int_{-\infty}^{s}exp(-(s-\alpha)A^{+})A\eta^{+}(\alpha)
		\\
		&=Q(\derivadaparcialabrev{s}\eta^{+}(s))  - \int_{-\infty}^{s}\derivadaparcial{}{s}\Big(exp(-(s-\alpha)A^{+})\eta^{+}(\alpha)\Big)
		\\
		&=Q(\derivadaparcialabrev{s}\eta^{+}(s)) - \derivadaparcial{}{s}\int_{-\infty}^{s} exp(-(s-\alpha)A^{+})\eta^{+}(\alpha) + \eta^{+}(s)
		\\
		&=Q(\derivadaparcialabrev{s}\eta^{+}(s)) - \derivadaparcial{}{s}\int_{-\infty}^{0} exp(\lambda A^{+})\eta^{+}(s+\lambda) + \eta^{+}(s)
		\\
		&=Q(\derivadaparcialabrev{s}\eta^{+}(s)) - \int_{-\infty}^{0} exp(\lambda A^{+})\derivadaparcial{}{s}\eta^{+}(s+\lambda) + \eta^{+}(s)
		\\
		&=Q(\derivadaparcialabrev{s}\eta^{+}(s)) - Q(\derivadaparcialabrev{s}\eta^{+}(s)) + \eta^{+}(s)
		\\
		&=\eta^{+}(s).
		\end{aligned}
		$$
		Analogamente teremos $Q(\diferencialfloerabrev \eta^{-}(s)) =\eta^{-}(s)$. Portanto, $Q\circ \diferencialfloerabrev = Id$ pois
		$$
		Q(\diferencialfloerabrev \eta(s)) = Q(\diferencialfloerabrev (\eta^{+}(s)+\eta^{-}(s)))=Q(\diferencialfloerabrev \eta^{+}(s))+Q(\diferencialfloerabrev \eta^{-}(s)) = \eta^{+}(s)+\eta^{-}(s) = \eta(s).
		$$
	\end{prova}
	
	Vamos utilizar a regularidade elíptica para garantir a continuidade do diferencial do operador de Floer $\diferencialfloerabrev$ cuja demonstração pode ser encontrada em $\cite{audi_floer_homology}$.
	
	\begin{teorema}
		(Regularidade Elíptica) Seja $p> 1$. Se $\xi \in \espacoLpcontradominio{\retacartesianocirculo}{\real{2n}}$ é uma solução fraca (no sentido distribuicional) da equação $\diferencialfloerabrev\xi =0$, então $\xi \in \espacosobolevcontradominio{\retacartesianocirculo}{\real{2n}}$ e $\xi \in C^{\infty}$.
	\end{teorema} 
	
	
	Foi mostrado na Seção $\ref{secao_identificacao_pullback_fibrado}$ que, dado $u \in \caminhosexponenciaisconectantespadrao$, temos as identificações entre os espaços  $\espacoLp{\pullbackfibradotangenteM{u}} \mapsto \espacoLpcontradominio{\retacartesianocirculo}{\real{2n}}$ e $\espacosobolev{\pullbackfibradotangenteM{u}} \mapsto \espacosobolevcontradominio{\retacartesianocirculo}{\real{2n}}$. Sendo que essa identificação foi utilizada para simplificar cálculos e tonar a exposição do assunto menos densa. Na demosntração a seguir esse fato será crucial pois, para aplicarmos o Teorema $\ref{teorema_sobolev}$, necessitamos restrigir o domínio das funções a conjuntos limitados com borda de classe $C^{1}$. Note que $\retacartesianocirculo$ não é limitado. Pelo Teorema $\ref{teorema_whitney}$, temos o mergulho $\caminhosexponenciaisconectantespadrao \subset M\subset \real{k}$, o que já foi feito na Seção $\ref{secao_identificacao_pullback_fibrado}$. Podemos afirmar que $\caminhosexponenciaisconectantespadrao$ é um limitado, pois é um subconjunto da variedade compacta $M$ mergulhada em $\real{k}$. Além disso, seu bordo é dado por $\partial \caminhosexponenciaisconectantesabrev =x^{+}(\circulo) \cup x^{-}(\circulo)$.
	\vermelho{(aqui devemos ter uma figura ilustrando o bordo ...!!!)}
	
	A seguir enunciaremos alguns resultados importantes de análise funcional os quais serão utilizados diretamente na demonstração da propriedade de Fredholm do operador $\diferencialfloerabrev$. São eles: o Teorema da Aplicação Aberta, o Teorema da Representação de Reiz e o Teorema de Han-Banach. Todos podem ser encontrados em $\cite{kreyszig_analise_funcional}$. Por fim, o seguinte lema, que pode ser encontrado em $\cite{audi_floer_homology}$, também será utilizado na demonstração.
	
	\begin{lema}\label{lema_desigualdade_inclusao_sobolev}
		Sejam $p \in \mathbb{N}$ tal que $p\geq 1$. Se $\xi \in \espacosobolevcontradominio{\retacartesianocirculo}{\real{2n}}$, então existe uma constante $c>0$ tal que $\normaWp{\xi}\leq c(\normaLp{\diferencialfloer \xi} +\normaLp{\xi})$.
	\end{lema}
	
	\begin{teorema}\label{teorema_aplicacao_aberta}
		(Teorema da aplicação aberta) Seja $T : A \to E$ uma aplicação linear limitada e sobrejetora entre os espaços de Banach $A$ e $E$. Se $T$ é uma bijeção, então $T^{-1}$ é contínua e limitada.
	\end{teorema}
	
	\begin{observacao}\label{observacao_teorema_aplicacao_aberta}
		Como consequência do teorema anterior pode-se afirma que, dada a bola aberta centrada na origem $B_{A}(0,1)$, existe uma constante $c>0$ tal que $B_{E}(0,c)\subset T(B_{A}(0,1))$
	\end{observacao}
	
	\begin{teorema}
		(Teorema da representação de Reisz) Seja $f:H \to \reta$ um funcional linear limitado definido em um espaço de Hilbert $H$. Então $f$ pode ser representado em termos de produto interno, isto é, $f(x) = \produtointerno{x}{z}$, onde $z$ depende de $f$, é unicamente determinada por $f$ e sua norma é $\norma{z} = \norma{f}$.
	\end{teorema}
	
	Seja $f: B \to \reta$ um funcional linear definido no espaço vetorial $B$. Dizemos que $f$ é sublinear se $f(x+y)\leq f(x)+f(y)$ para todos $x,y \in B$.
	
	\begin{teorema}
		(Teorema de Hahn-Banach) Seja $f:B \to \reta$ um funcional sublinear definido no espço vetorial $B$. Suponha que $g:Z \subset B \to \reta$ seja um funcional linear tal que $g(x)\leq f(x)$ para todo $x \in Z$. Então $g$ tem uma única extensão linear $g': B \to \reta$ tal que $g'(x)\leq f(x)$ para todo $x \in B$ e $g=g'|_{Z}$.
	\end{teorema}
	
	\begin{teorema}
		(Propriedade de Fredholm de $\diferencialfloer$) Suponha que $det(Id - \Psi^{\pm}(1)) \neq 0$. Então o operador $\diferencialfloer: \espacosobolevcontradominio{\retacartesianocirculo}{\real{2n}} \to \espacoLpcontradominio{\retacartesianocirculo}{\real{2n}}$ é um operador de Fredholm para $1<p<\infty$. Além disso, seu índice de Fredholm é dado por
		$$
		ind(\diferencialfloer) = \mu(\Psi^{+}) - \mu(\Psi^{-}).
		$$
	\end{teorema}
	
	\chapter{Homologia de Floer}
	A Teoria de Morse clássica é uma estrátegia de se estudar a topologia de variedades diferenciáveis de dimensão finita através da análise do comportamento de funções suaves definidas na dada variedade, chamadas funções de Morse. Tal estratégia resume-se em determinar os pontos críticos dessas funções, e a eles associar invariantes topológicos (os índices de Morse). Por fim, calcula-se a homologia da variedade.
	
	Em 1966 em um congresso em Moscou, o matemático russo Vladimir Igorevich Arnold, nos estudos de sistemas Hamiltonianos e topologia do 2-toro, formulou uma conjectura a respeito do número de pontos fixos que um simplectomorfismo definido em uma variedade simplética possui e o número de pontos crísticos de uma função de Morse, que deu origem a seguinte generalização:
	
	\textit{\textbf{(Conjectura de Arnold):} Sejam $(M, \omega)$ uma variedade simplética 2n-dimensioanl e $\psi : M \to M $ um simplectomorfismo Hamiltoniano, então $\psi$ deve ter tantos pontos fixos quanto uma função suave em $M$ deve ter de ponstos críticos. Se os pontos fixos forem não-degenerados, então os número de pontos fixos é, no mínimo, o mesmo número de pontos críticos de uma função de Morse em $M$.}
	
	Foi no contexto de sua demonstração que nasceu a Homologia de Floer, quando na tentativa de se construir uma técnica analoga a  Morse clássica que se deparou com algumas barreiras técnicas, tais como: variedades infinitas e a definição de um análogo ao índice de Morse dos pontos críticos não-degenerados nessas variedades. Floer utiliza como função de Morse um funcional definido nesse espaço, que é um espaço de funções, e estuda seus pontos críticos e linhas de fluxo de seu gradiente para a construção da homologia da variedade simplética. Ao final, mostra que essa homologia é isomorfa a homologia singular da variedade.
	
	\section{Definições}
	Seja $C = C(M, \inteiros_{2})$ o espaço vetorial sobre o corpo $\inteiros_{2}$ e gerado pelos elementos de $\lacocontrateis$. Esse espaço é graduado pela função $m$, de modo que
	$$
	C = \bigoplus_{k}C_{k}, \; C_{k}(M, \inteiros_{2}) = \text{span}\{x \in \lacocontrateis:m(x)=k \}.
	$$
	Segue-se da compacidade e da estruta de variedade diferenciável que $\mathcal{M}(y,x)$ possui um número finito de elementos sempre que $ m(y)-m(x)=1 = dim(\mathcal{M}(y,x))$. Com esse fato podemos construir o operador bordo $\partial : C_{k+1} \to C_{k}$ tal que
	$$
	\partial y = \sum_{m(x)=k} n_{k}(x) x,	
	$$
	onde $n_{k}(x)$ é o número de componentes de $\mathcal{M}(y,x)$ módulo 2. Posteriormente Floer demonstra que $\partial^{2}=0$, assim $(C,\partial)$ define um complexo de cadeia e a homologia
	$$
	HF_{*}(M, \inteiros_{2}) = \frac{Ker(\partial)}{Im(\partial)}
	$$
	é chamada homologia de Floer.
	
	\textbf{Observação:} Note que ao se definir o complexo de cadeia não fizemos referência a função Hamiltoniana nem a estrutura quase-complexa J escolhidas, isso porque a homologia de Floer não depende de tais escolhas, seguindo o teorema:
	
	\begin{teorema}
		Sejam $(H,J)$ e $(H',J')$ pares regulares, isto é, $J, J' \in J_{reg}$, respeticamente, então existe um homomorfismo de cadeias induzindo um isomorfismo nas homologias de Floer
		$$
		HF_{*}(M,\inteiros_{2};H,J) \cong HF_{*}(M,\inteiros_{2};H',J'). 
		$$
		Além disso, existe um isomorfismo natural entre a homologia de Floer e a homologia singular de $M$
		$$
		HF_{*}(M,\inteiros_{2};H,J) \cong H_{*}(M;\inteiros_{2}). 
		$$
		
	\end{teorema}
	
	\appendix
	
	\chapter{Complexificação de espaços vetoriais}\label{apendice_complexificacao_espacos_vetoriais}
	
	Seja $V$ um n-espaço vetorial real. Desejamos construir um espaço vetorial complexo a partir de $V$ e uma forma de realizarmos essa construção é tomando $v \in V $ e $a+ib \in \complexo{}$ e efetuando a multiplicação $(a+ib)v = av+ibv$. Porém, a multiplicação por $i$ não tem sentido nesse contexto, já que estamos tratando de um espaço vetorial real. Podemos contornar essa dificuldade tratando a multiplicação formal $(a+ib)v = av+ibv$ como o par ordenado $(av, bv)$, resultando na seguinte definição de complexificação:
	
	\begin{definicao}
		(Complexificação do n-espaço vetorial real V) A complexificação do n-espaço vetorial real $V$ é o n-espaço vetorial complexo $\complexificacao{V}=V\oplus V$ munido da multiplicação por escalares $(a+ib)(v_{1}, v_{2}) = (av_{1}-bv_{2}, bv_{1}+ av_{2})$, onde $a+ib \in \complexo{}$ e $(v_{1}, v_{2})\in \complexificacao{V}$.
	\end{definicao}
	
	\begin{observacao}
		Essa definição tem como motivação a seguinte situação: supondo $V$ complexo, então $V \ni (a+ib)v = (a+ib)(v_{1}+iv_{2}) = av_{1}-bv_{2}+i(bv_{2}+av_{1}) \mapsto (av_{1}-bv_{2}, bv_{1}+ av_{2}) \in \complexificacao{V}$.
	\end{observacao}
	
	É possível mostrar que $\complexificacao{V}$ é um espaço vetorial sobre $\complexo{}$. De fato, basta mostrarmos que esse espaço é fechado pela operação de adição e multiplicação por escalares, pois as outras operaçoes são triviais. Dados $a+ib \in \complexo{}$ e $v,u \in \complexificacao{V}$, então $(a+ib )v+u = (av_{1}-bv_{2}, bv_{1}+ av_{2}) + (u_{1}, u_{2}) = (av_{1}-bv_{2} + u_{1}, bv_{1}+ av_{2} + u_{2}) \in \complexificacao{V}$. As demais propriedades dos axiomas de espaços vetoriais resultam imediatamente dessa.
	
	\begin{exemplo}\label{exemplo_conjugado_reta_real}
		(Complexificação de $\reta$) Definindo $V = \reta$ temos que $\complexificacao{\reta}$ é $\reta\oplus \reta$ munido da operação $(a+ib)(x,y) = (ax-by, bx+ay)$. Assim, a identificação $(a+ib)(x, y ) = (ax-by, bx+ay) \mapsto ax-by+i(bx+ay) \in \complexo{}$ nos dá o isomorfimo $\complexificacao{\reta} \ni (x,y)\mapsto x +iy\in \complexo{}$.
	\end{exemplo}
	
	\begin{exemplo}\label{exemplo_complexificacao_matrizes}
		Analogamente ao exemplo anterior, temos que as aplicações $\complexificacao{\real{n}} \ni (x,y) \mapsto x+iy \in \complexo{n}$ e  $\complexificacao{\matrizquadreal{n}} \ni (x,y) \mapsto x+iy \in \matrizquadcomplexa{n}$ também são isomorfismos.
	\end{exemplo}
	
	Notemos que os espaços vetoriais $V\oplus\{0\}$ e $\{0\}\oplus V$ são isomorfos a $V$. Com isso, a inclusão $i_{V} :V \hookrightarrow \complexificacao{V}$ definida por $i(v) = (v, 0)$ é chamada de mergulho padrão de $V$ em $\complexificacao{V}$.
	
	Complexificamos os espaços vetoriais reais, porém, as aplicações lineares entre tais espaços vetoriais são outros tipos de objetos a serem complexificados. O teorema a seguir afirma a unicidade da complexificação de aplicações lineares entre as complexificações de espaços vetoriais reais.
	
	\begin{teorema}
		(Complexificação de aplicações lineares) Seja $\varphi : V \to W$ uma aplicação
		$\reta$-linear entre espaços vetoriais reais de dimensão finita. Então existe uma única aplicação linear $\complexificacao{\varphi}:\complexificacao{V} \to \complexificacao{W}$ tal que $i_{W}\circ \varphi = \complexificacao{\varphi} \circ i_{V}$, isto é, o diagrama abaixo é comutativo.
		$$
		\xymatrix{
			V\ar[d]_{i_{V}}\ar[r]^{\varphi} & W\ar[d]^{i_{W}} 
			\\
			\complexificacao{V} \ar[r]_{\complexificacao{\varphi}}&\complexificacao{W} 
		}
		$$
	\end{teorema}
	\begin{prova}
		Defina $\complexificacao{\varphi}:\complexificacao{V} \to \complexificacao{W}$ por $\complexificacao{\varphi}(v_{1}, v_{2}) = (\varphi v_{1}, \varphi v_{2})$. Como $\varphi v_{1}, \varphi v_{2} \in W$, então $(\varphi v_{1}, \varphi v_{2}) \in W\oplus W = \complexificacao{W}$. Além disso, dados $\lambda = (a+ib)\in \complexo{}$ e $(v_{1}, v_{2}), (u_{1}, u_{2})\in \complexificacao{V}$ temos 
		$$
		\begin{aligned}
		\complexificacao{\varphi} (\lambda(v_{1}, v_{2}) + (u_{1}, u_{2})) 
		&= (\varphi(a v_{1} - bv_{2} + u_{1}), \varphi(b v_{1}+a v_{2} + u_{2}))
		\\
		&=(a\varphi v_{1} - b\varphi v_{2} + \varphi u_{1}, b\varphi v_{1}+a\varphi v_{2} + \varphi u_{2})
		\\
		&=(a\varphi v_{1} - b\varphi v_{2} , b\varphi v_{1}+a\varphi v_{2} )+(\varphi u_{1},\varphi u_{2})
		\\
		&=\lambda(\varphi v_{1},\varphi v_{2})+(\varphi u_{1},\varphi u_{2})
		\\
		&=\lambda 	\complexificacao{\varphi} (v_{1},v_{2})+\complexificacao{\varphi} (u_{1},u_{2}).
		\end{aligned}
		$$
		Portanto $\complexificacao{\varphi}$ é $\complexo{}$-linear. Dado $v \in V$ temos $(i_{W}\circ \varphi)(v) = (\varphi v,0) = \complexificacao{\varphi}(v,0) = (\complexificacao{\varphi} \circ i_{V})(v)$, logo $i_{W}\circ \varphi =\complexificacao{\varphi} \circ i_{V}$. Por fim, suponha que exista $\complexificacao{\varphi'} $ definida por $\complexificacao{\varphi'}(v_{1}, v_{2}) = (\varphi' v_{1}, \varphi' v_{2})$ e tal que $i_{W}\circ \varphi =\complexificacao{\varphi'} \circ i_{V}$, onde $\varphi':V\to W$ é uma aplicação linear. Então para todo $v \in V$ temos $(i_{W}\circ \varphi )(v)=(\complexificacao{\varphi} \circ i_{V})(v) = (\complexificacao{\varphi'} \circ i_{V})(v)$, o que implica em $(\varphi v, 0) = (\varphi' v, 0)$. Como $v \in V$ é arbitrário, então $\varphi = \varphi'$, logo $\complexificacao{\varphi} = \complexificacao{\varphi'}$, demonstrando a unicidade. 
	\end{prova}
	
	Considerando o corpo dos complexos como um espaço vetorial sobre $\reta$, mostraremos que o produto tensorial $\complexificacaotensorial{V}$ é equivalente a construção da complexificação feita anteriormente. Utilizaremos esta construção na complexificação do 2n-espaço vetorial simplético $(V, \omega)$.
	
	\begin{teorema}\label{teorema_isomorfismo_complexificacao}
		Seja $V$ um espaço vetorial real de dimensão finita. A aplicação $f_{V}: \complexificacao{V} \to \complexificacaotensorial{V}$ definida por  $f_{V}(v_{1},v_{2}) = 1\otimes_{\reta} v_{1} + i \otimes_{\reta} v_{2}$ é um isomorfismo entre os espaços vetoriais complexos que faz o diagrama
		$$
		\xymatrix{
			& V\ar[dl]_{i_{V}}\ar[rd]^{j} &
			\\
			\complexificacao{V}\ar[rr]_{f_{V}}& & \complexificacaotensorial{V},
		}
		$$
		comutar, onde $i_{V}(v) = (v,0)$ (é o mergulho padrão) e $j(v) = 1\otimes_{\reta} v$.
	\end{teorema}
	\begin{prova}
		Dados $z=(a+ib) \in \complexo{}$ e $v,u \in \complexificacao{V}$ temos
		$$
		\begin{aligned}
		f_{V}(zv+u) &= 1\otimes_{\reta} (a v_{1} - bv_{2} + u_{1}) + i \otimes_{\reta} (bv_{1}+av_{2} + u_{2})
		\\
		&=1\otimes_{\reta} (a v_{1} - bv_{2}) + i \otimes_{\reta} (bv_{1}+av_{2})+1\otimes_{\reta} u_{1} + i \otimes_{\reta} u_{2}
		\\
		&=(a+ib)\otimes_{\reta}v_{1} + (-b+ia)\otimes_{\reta} v_{2}+f_{V}(u)
		\\
		&=(a+ib)\otimes_{\reta}v_{1} + (a+ib)i\otimes_{\reta} v_{2}+f_{V}(u)
		\\
		&=zf_{V}(v)+f_{V}(u).
		\end{aligned}
		$$
		Portanto $f_{V}$ é $\complexo{}$-linear. Além disso, é injetora pois $f_{V}(u) = 0$ se, e somente se, $u = 0$. Tomando um elemento $w = (a+ib)\otimes_{\reta}v \in \complexificacaotensorial{V} $ e definindo $u = (av, bv) \in \complexificacao{V}$ temos que $f_{V}(u) = \complexificacaoelemento{1}{av} +\complexificacaoelemento{i}{bv} = (a+ib)\otimes_{\reta} v = w$. Portanto $f_{V}$ é sobrejetora. Logo é um isomorfismo.
		
		Para todo $v \in V$ temos que $(f_{V}\circ i_{V})(v) = f_{V}(v, 0) = 1\otimes_{\reta} v = j(v)$. Portanto o diagrama é comutativo.
	\end{prova}
	
	O conjugado de cada $z=(a+ib) \in \complexo{}$ é definido por $\overline{z} = a-ib$. Analogamente, o conjugado de $\complexificacaoelemento{z}{v} \in \complexificacaotensorial{V}$ é definido por $\complexificacaoelemento{\overline{z}}{v} \in \complexificacaotensorial{V}$. Definindo $v_{1} =av$ e $v_{2} = -bv$ temos $f_{V}(v_{1}, v_{2}) =\complexificacaoelemento{1}{v_{1}} +\complexificacaoelemento{i}{v_{2}} = \complexificacaoelemento{a}{v} -\complexificacaoelemento{ib}{v}  =\complexificacaoelemento{\overline{z}}{v}  $. Portanto, temos a identificação entre os conjugados $ \complexificacao{V} \ni (av, -bv) \mapsto \complexificacaoelemento{\overline{z}}{v} \in \complexificacaotensorial{V}$.
	
	Sejam $k\geq 1$ um inteiro e $\{V_{j}\}_{j=1}^{k}$ um conjunto de espaços vetoriais de dimensão finita sobre um corpo $\mathbb{K}$. Sabe-se que o produto tensorial $\produtotensorial{V}{k}$ é um espaço vetorial de dimensão finita sobre $\mathbb{K}$. Além disso, sabe-se que, dado um $W$ espaço vetorial também sobre $\mathbb{K}$, tem-se que $W\otimes_{\mathbb{K}}(\produtotensorial{V}{k}) \cong W\otimes_{\mathbb{K}} \produtotensorial{V}{k}$. 
	
	Daqui em diante fixaremos a notação $\complexificado{V} = \complexificacaotensorial{V}$ para a complexificação de $V$ via produto tensorial.
	
	\begin{observacao}
		Visto como espaço vetorial complexo, temos que $\{1\}$ é uma base de $\complexo{}$, logo temos que $\{1^{*}\}$ é uma base do dual $\complexo{*}$, tal que $1^{*}(1) = 1$. Com isso, dado $z = (a+ib).1 \in \complexo{}$, temos $1^{*}(z) = 1^{*}((a+ib).1) = (a+ib)1^{*}(1) = a+ib$.
	\end{observacao}
	
	Sejam $B = \colecaofinita{e}{n}$ uma base do n-espaço vetorial real $V$ e $B^{*} = \colecaofinita{e^{*}}{n}$ a base de seu dual $V^{*}$ tal que $e^{*}_{j}(e_{k}) =\delta_{jk}$. Definindo $1 \otimes B = \colecaofinita{\complexificacaoelemento{1}{e}}{n}$ e
	$1^{*} \otimes B^{*} = \colecaofinita{\complexificacaoelemento{1^{*}}{e^{*}}}{n}$, temos que $1\otimes B$ gera $\complexificado{V}$. De fato, todo $v \in \complexificado{V}$ pode ser escrito como 
	$$
	\begin{aligned}
	v &= \sum_{j=1}^{n} \complexificacaoelemento{z_{j}}{v_{j}}
	\\
	&= \sum_{j=1}^{n} \complexificacaoelemento{z_{j}}{\bigparenteses{\sum_{i=1}^{n}\alpha_{ji}e_{i}}} 
	\\
	&= \sum_{j,i=1}^{n}\complexificacaoelemento{z_{j}\alpha_{ji}}{e_{i}}
	\\
	&= \sum_{i=1}^{n}\complexificacaoelemento{z'_{i}}{e_{i}}
	\\
	&= \sum_{i=1}^{n}z'_{i}(\complexificacaoelemento{1}{e_{i}}),
	\end{aligned}
	$$
	onde $z'_{i} = \sum_{j=1}^{n}z_{j}\alpha_{ji}$ e $ z_{j}$ estão em $\complexo{}$ e $v_{j} \in V$.
	
	
	\begin{observacao}
		Os elementos $v \in \complexificado{V}$ do tipo $v = \complexificacaoelemento{z}{\alpha}$ são chamados de vetores simples. Vimos anteriormente que os vetores mais gerais são combinações lineares dos vetores simples. Por esse fato, as aplicações lineares e multilineares estudadas nessa seção serão avaliadas apenas nos tensores simples.
	\end{observacao}
	
	Para simplificar a notação, denotando os elementos das bases $1\otimes B$ e $1^{*}\otimes B^{*}$ por $\{\textbf{a}_{j}\} $ e $\{\textbf{a}^{*}_{j}\}$, respectivamente.
	
	\begin{proposicao}\label{proposicao_base_complexificada}
		$1\otimes B$ e $1^{*}\otimes B^{*}$ são bases de $\complexificado{V}$ e $\complexificado{V}^{*}$, respectivamente, tais que $\textbf{a}^{*}_{j}(\textbf{a}_{k})=\delta_{ik}$.
	\end{proposicao}
	\begin{prova}
		Por construção $1\otimes B$ gera $\complexificado{V}$. Afirmo que $B$ é linearmente independente pois, $0=\sum_{j}z_{j}\textbf{a}_{j} = \sum_{j}z_{j}(1\otimes e_{j}) = \sum_{j}(b_{j} +ic_{j})(1\otimes e_{j}) = 1\otimes \sum_{j}b_{j}e_{j}+i\otimes \sum_{j}c_{j}e_{j}$, o que implica que $\sum_{j}b_{j}e_{j}=\sum_{j}c_{j}e_{j} = 0$. Logo, $b_{j}=c_{j}=0$  e $z_{j} = 0$ para $1\leq j\leq k$. O isomorfismo $\complexificado{V} \mapsto \complexificado{V}^{*}$ garante que $B^{*}$ é base de $\complexificado{V}^{*}$. Por fim, temos que $\textbf{a}^{*}_{j}(\textbf{a}_{k})=(\complexificacaoelemento{1^{*}}{e^{*}_{j}})(\complexificacaoelemento{1}{e_{k}}) = 1^{*}(1)e^{*}_{j}(e_{k}) = \delta_{ik}$.
	\end{prova}
	
	Como o produto tensorial real $\otimes_{j=1}^{k}V_{j}$ é um espaço vetorial real, então sua complexificação será o espaço vetorial complexo $\complexificacaotensorial{(\produtotensorialreal{k}{V})} \cong \complexificacaotensorial{V_{1} \otimes_{\reta} \dots \otimes_{\reta} V_{k}}$, pelo isomorfismo anteriormente citado. Contudo, pode-se se mostrar que essa complexificação é isomorfa ao produto tensorial sobre os complexos $\bigotimes^{k}_{j=1}\complexificado{V}$, a qual denotamos por $\complexificado{V}^{k\otimes}$. 
	
	Na seção seguinte vamos utilizar o produto tensorial complexificado $\complexificado{V}^{k\otimes}$ pois ela facilita a demonstração de algumas propriedades da complexificação da forma simplética, que é um tensor de ordem 2.
	
	\begin{exemplo}\label{exemplo_complexificacao_tensorial}
		Sejam $V$ um n-espaço vetorial real e $B = \colecaofinita{e}{n}$ uma base ortonormal. Supondo que $\complexificado{V}$ seja a complexificação de $V$, então pela Proposição $\ref{proposicao_base_complexificada}$ temos que $1\otimes B^{*}$ é uma base de $\complexificado{V}^{*}$. Com isso, dado $T \in \produtotensorialdual$ podemos escrever $T = \sum_{i,j = 1}^{n} T^{ij}\textbf{a}^{*}_{i}\otimes \textbf{a}^{*}_{j}$, onde $T^{ij} \in \complexo{}$. Dados $v, u\in \complexificado{V}$ tais que $v=\complexificacaoelemento{x}{\alpha}$ e $u=\complexificacaoelemento{y}{\beta}$, temos
		$$
		\begin{aligned}
		T(v,u) &= \sum_{i,j = 1}^{n}
		T^{ij}\textbf{a}^{*}_{i}\otimes \textbf{a}^{*}_{j}(v,u)
		\\
		&= \sum_{i,j = 1}^{n}
		T^{ij}\textbf{a}^{*}_{i}(v)\textbf{a}^{*}_{j}(u)
		\\
		&= \sum_{i,j = 1}^{n}
		T^{ij}(\complexificacaoelemento{1^{*}}{e^{*}_{i}})(\complexificacaoelemento{x}{\alpha})(\complexificacaoelemento{1^{*}}{e^{*}_{j}})(\complexificacaoelemento{y}{\beta})
		\\
		&= \sum_{i,j = 1}^{n}
		T^{ij}1^{*}(x)e^{*}_{i}(\alpha)1^{*}(y)e^{*}_{i}(\beta)
		\\
		&= \sum_{i,j = 1}^{n}
		xyT^{ij}\alpha_{i}\beta_{j},
		\end{aligned}
		$$ 
		onde $\alpha_{j}$ e $\beta_{j}$ são as $j$-ésimas coordenadas de $\alpha, \beta \in V$. 
	\end{exemplo}
	
	Para estudarmos algumas características dos auto-espaços do grupo simplético precisaremos decompor esse espaço vetorial em soma direta de determinados  auto-espaços. Tal composição será garantida pelo Teorema $\ref{teorema_espectral_jordan}$, que depende de um espaço vetorial complexo como hipótese. Como consequencia desse fato, precisaremos de um espaço vetorial simplético complexo e que seja gerado a partir de um dado espaço vetorial simplético.
	
	Nessa seção, vimos que uma das construções da complexificação de $V$ é o espaço vetorial complexo $\complexificado{V}$. Dado $(V, \omega)$ um 2n-espaço vetorial simplético real, temos que $\omega \in V^{*}\otimes V^{*}$. Sejam $B = \{e_{j}\}_{j=1}^{2n}$ uma base de $V$ e $\omega^{ij} = \omega(e_{i}, e_{j})$ as componentes da forma simplética nessa base. Dessa definição, temos que $\omega^{ij} = -\omega^{ji}$.
	
	Utilizando as notações da Proposição $\ref{proposicao_base_complexificada}$, defina $\Omega=\sum_{i,j}\omega^{ij}\textbf{a}^{*}_{i}\otimes \textbf{a}^{*}_{j} \in \produtotensorialdual$. Temos que
	
	\begin{proposicao} $(\complexificado{V}, \Omega)$ é um 2n-espaço vetorial simplético sobre $\complexo{}$.
	\end{proposicao}\label{proposicao_complexificacao_espaco_simpletico}
	\begin{prova}
		Pelo Teorema $\ref{teorema_isomorfismo_complexificacao}$, temos que $\complexificado{V}$ é um 2n-espaço vetorial complexo. Por construção $\Omega$ é uma aplicação $\complexo{}$-bilinear. Como $\omega^{ij}  = -\omega^{ji} $, temos que $\Omega$ é anti-simétrica. Afirmo que $\Omega$ é não degenerada. De fato, tomando $v, u \in \complexificado{V}$ tal que $v= \complexificacaoelemento{x}{\alpha}$ e $u=\complexificacaoelemento{y}{\beta} $, temos, pelo Exemplo $\ref{exemplo_complexificacao_tensorial}$, que 
		$$
		\begin{aligned}
		\Omega(v,u) &= \sum_{i,j = 1}^{2n}
		\omega^{ij}\textbf{a}^{*}_{i}\otimes \textbf{a}^{*}_{j}(v,u)
		\\
		&= \sum_{i,j = 1}^{2n}
		xy\omega^{ij}\alpha_{i}\beta_{j}
		\\
		&= xy\sum_{i,j = 1}^{2n}
		\formaSimpletica{\alpha_{i}e_{i}}{\alpha_{j}e_{j}}
		\\
		&= xy\formaSimpletica{\alpha}{\beta}.
		\end{aligned}
		$$
		Como $\omega$ é não-degenerada, então $\Omega$ é não-degenerada. Portanto, $(\complexificado{V}, \Omega)$ é um 2n-espaço vetorial simplético sobre $\complexo{}$.
	\end{prova}
	
	\begin{observacao}
		Note que $\complexificado{V}_{0} = \{\complexificacaoelemento{1}{v} \in \complexificado{V} \}$ é um subespaço vetorial de $\complexificado{V}$ isomorfo a $\real{2n}$. Além disso, $\Omega|_{\complexificado{V}_{0}} = \omega$ e $\Omega$ é uma extensão contínua de $\omega$.
	\end{observacao}
	
	\begin{proposicao}
		A complexificação $\Omega$ de $\omega$ pode ser decomposta em $\Omega = g +i k$, onde $g,k$ são ambas formas simpléticas definidas em $V$.
	\end{proposicao}
	\begin{prova}
		
		O Teorema $\ref{teorema_existencia_base_simpletica}$ garante que $\complexificado{V}$ possui uma base simplética $B$. Pela Observação $\ref{observacao_existencia_base_simpletica}$ temos que a representação matricial de $\Omega$ nessa base simplética é $\estruturacomplexa$. Suponha que $V, U \in \matrizquadcomplexa{2n}$ sejam as representações matriciais de $v,u \in \complexificado{V}$ nessa base. Então podemos escrever $$
		\begin{aligned}
		\formaSimpleticaExtendida{v}{u} &= V^{*}\estruturacomplexa^{*} U 
		\\
		&= (V_{1}+iV_{2})^{*}\estruturacomplexa^{*} (U_{1}+iU_{2})
		\\
		&= \underbrace{V_{1}^{t}\estruturacomplexa^{t} U_{1}}_{\formaSimpletica{v_{1}}{u_{1}}} -V_{2}^{t}\estruturacomplexa^{t} U_{2}+i(\underbrace{ V_{1}^{t}\estruturacomplexa^{t} U_{2}}_{\formaSimpletica{v_{1}}{u_{2}}} - V_{2}^{t}\estruturacomplexa^{t} U_{1})
		\\
		&= \underbrace{\formaSimpletica{v_{1}}{u_{1}} -\formaSimpletica{v_{2}}{u_{2}}}_{g(v,u)}+i(\underbrace{ \formaSimpletica{v_{1}}{u_{2}}- \formaSimpletica{v_{2}}{u_{1}}}_{k(v,u)})
		\\
		&= g(v,u)+ik(v,u),
		\end{aligned}
		$$
		onde fizemos $v_{j}, u_{j}\in V$ e $V_{j}, U_{j}$ sendo a representação matricial de cada um deles para $j \in \{1,2\}$.
		Temos que $g,k:\complexificado{V}\times \complexificado{V} \to \reta$ são aplicações $\complexo{}$-bilineares e anti-simétricas pois são combinações lineares de aplicações $\complexo{}$-bilineares anti-simétricas. Pelo mesmo argumento, $g,k$ são não-degeneradas.
	\end{prova}
	
	\begin{proposicao}\label{proposicao_base_simpletica_conjugada}
		Seja $B=\{\textbf{e}, \textbf{f} \}$ uma base simplética de $\complexificado{V}$. Então $\overline{B}=\{\overline{\textbf{e}}, \overline{\textbf{f}}\}$ é uma base simplética de $\complexificado{V}$.
	\end{proposicao}
	\begin{prova}
		Temos que a aplicação de conjugação $\complexificado{V} \ni z\otimes v \mapsto \overline{z}\otimes v \in \complexificado{V}$ é um isomorfismo. Logo, leva base em base. Como $B$ é uma base simplética, temos $\Omega(\textbf{e}_{i}, \textbf{e}_{j}) = \Omega(\textbf{f}_{i}, \textbf{f}_{j}) =0$ e $\Omega(\textbf{e}_{i}, \textbf{f}_{j}) = \delta_{ij}$. Com isso, fazendo $\textbf{e}_{i} = \complexificacaoelemento{z_{i}}{v_{i}}$, e pela Proposição $\ref{proposicao_complexificacao_espaco_simpletico}$ temos $\Omega(\overline{\textbf{e}}_{i}, \overline{\textbf{e}}_{j}) =	\overline{z_{i}}\overline{z_{j}}\formaSimpletica{v_{i}}{v_{j}}=\overline{z_{i}z_{j}\formaSimpletica{v_{i}}{v_{j}}} = \overline{\Omega(\textbf{e}_{i}, \textbf{e}_{j})} = 0$. Com argumento análogo pode-se ver que $\Omega(\overline{\textbf{e}}_{i}, \overline{\textbf{e}}_{j}) =0$ e $\Omega(\overline{\textbf{e}}_{i}, \overline{\textbf{e}}_{j}) =\delta_{ij}$. Logo, $\overline{B}$ é uma base simplética.
	\end{prova}
	
	Um operador linear $T : \complexificado{V} \to \complexificado{V}$ pode ser escrito como $T=\complexificacaoelemento{\lambda}{A}$, tal que $T(v) = \complexificacaoelemento{\lambda}{A}(\complexificacaoelemento{z}{\alpha}) = \complexificacaoelemento{\lambda z}{A\alpha}$, onde $\lambda \in \complexo{}$ e $A$ é um operador linear $A:V \to V$. Com isso, $T$ esta bem-definido.
	
	\begin{definicao}
		(Transformação simplética) Seja $(\complexificado{V}, \Omega)$ um 2n-espaço vetorial simplético sobre $\complexo{}$. Um operador linear $T: \complexificado{V} \to \complexificado{V}$ é uma transformação simplética se 
		$$
		\formaSimpleticaExtendida{Tu}{Tv} = \formaSimpleticaExtendida{u}{v}
		$$ para todo $u,v\in \complexificado{V}$.
	\end{definicao}
	
	Suponha que $T=\complexificacaoelemento{1}{A}$ para algum operador linear $A:V \to V$. Afirmo que se $T$ é uma transformação simplética, então $A$ é uma transformação simplética. De fato, tomando $u = \complexificacaoelemento{x}{\alpha}$ e $v=\complexificacaoelemento{y}{\beta}$ em $\complexificado{V}$ temos que $Tu = \complexificacaoelemento{x}{A\alpha}$ e $Tv= \complexificacaoelemento{y}{A\beta}$
	$$
	\begin{aligned}
	\Omega(Tv,Tu)
	&= \sum_{i,j = 1}^{2n}
	\omega^{ij}\textbf{a}^{*}_{i}\otimes \textbf{a}^{*}_{j}(Tv,Tu)
	\\
	&= 
	\sum_{i,j = 1}^{2n}
	\omega^{ij}\textbf{a}^{*}_{i}\otimes \textbf{a}^{*}_{j}(\complexificacaoelemento{x}{A\alpha},\complexificacaoelemento{y}{A\beta})
	\\
	&= \sum_{i,j = 1}^{2n}
	xy\omega^{ij}e^{*}_{i}(Au)e^{*}_{j}(Av)
	\\
	&= xy
	\formaSimpletica{A\alpha}{A\beta}.
	\end{aligned}
	$$  
	Temos que $\formaSimpleticaExtendida{Tu}{Tv} = \formaSimpleticaExtendida{u}{v}$ o que implica que $xy
	\formaSimpletica{A\alpha}{A\beta} = xy
	\formaSimpletica{\alpha}{\beta}$. Como $\alpha, \beta \in V$ são arbitrários, então $A$ é uma transformação simplética. A recíproca é imediata.
	
	Adotando a notação da Convensão \ref{convensao_base_simpletica}, pode-se enunciar o seguinte resultado:
	
	\begin{proposicao}\label{proposicao_forma_simpletica_vetor_conjugado}
		Seja $\{\textbf{e}, \textbf{f}\}$ uma base simplética de $\complexificado{V}$. Então para um dado $v = v_{(1)}\textbf{e}+v_{(2)}\textbf{f} \in \complexificado{V}$ tem-se que $\formaSimpleticaExtendida{\overline{v}}{v}
		=i2\parteImaginaria{\produtointerno{\overline{v}_{(1)}}{v_{(2)}}} \in i\reta$.
	\end{proposicao}
	\begin{prova}
		Seja $\{\textbf{e}, \textbf{f} \}$ uma base simplética de $\complexificado{V}$. Tomando $v = v_{(1)}\textbf{e}+v_{(2)}\textbf{f}$, tem-se
		$$
		\begin{aligned}
		\formaSimpleticaExtendida{\overline{v}}{v} 
		&= \produtointerno{\overline{v}_{(1)}}{v_{(2)}}- \produtointerno{\overline{v}_{(2)}}{v_{(1)}}
		\\
		&= \produtointerno{\overline{v}_{(1)}}{v_{(2)}}- \overline{\produtointerno{v_{(2)}}{\overline{v_{(1)}}}}
		\\
		&= \produtointerno{\overline{v}_{(1)}}{v_{(2)}}-\overline{ \produtointerno{\overline{v}_{(1)}}{v_{(2)}}}
		\\
		&=i2\parteImaginaria{\produtointerno{\overline{v}_{(1)}}{v_{(2)}}}\in i\reta,
		\end{aligned}
		$$
		onde usamos a Observação \ref{observacao_convensao_base_simpletica}.
	\end{prova}
	
	\begin{definicao}
		(Grupo simplético complexo) O grupo simplético complexo $\gruposimpletico{\complexificado{V}} \subset \generalgroupcomplexo{2n}$ de $\complexificado{V}$ é o conjunto das matrizes associadas as transformações simpléticas definidas em $\complexificado{V}$.
	\end{definicao}
	
	\chapter{Preliminares de Álgebra}
	
	Nessa seção, apresentaremos alguns resultados preliminares necessários para o estudo da topologia de $\gruposimpletico{2n}$ e do conjunto das estruturas complexas $\estruturascomplexas{V}{\omega}$.
	
	
	Sejam $V$ um n-espaço vetorial real, $B= \{e_{j}\}_{j=1}^{n}$ uma base ordenada e $f:V\times V\to \reta$ uma aplicação bilinear. Definindo $A$ como a matriz de $f$ na base $B$, isto é, $A_{ij} = f(e_{i}, e_{j})$, pode-se mostrar que $f$ é positiva-definida se, e somente se, a função $g_{A}: M_{n\times 1}(\reta)\times M_{n\times 1}(\reta) \to \reta$, definida por $
	g_{A}(U,V) = U^{t}AV$, é positiva-definida.
	
	\begin{definicao}
		(k-subdeterminante) Seja $A \in \matrizquadreal{n}$. O seu k-subdeterminante é
		$$
		det_{k}(A) =
		det \left(
		\begin{array}{ccc}
		A_{11} & \dots & A_{1k}
		\\
		\vdots & \ddots & \vdots
		\\
		A_{k1} & \dots & A_{kk}
		\end{array}
		\right),\;\; 1\leq k \leq n.
		$$
	\end{definicao}
	
	\begin{teorema}\label{teorema_matriz_positiva_definida}
		Sejam $V$ um n-espaço vetorial real, $f: V\times V\to \reta$ uma aplicação bilinear e $A$ a matriz de $f$ na base ordenada $B$. Então $f$ é positiva-definida se, e somente se, $A=A^{t}$ e $det_{k}(A)>0$ para $1\leq k\leq n$.
	\end{teorema}
	
	\begin{observacao}\label{observacao_matriz_positiva_definida}
		Do teorema anterior pode-se afirmar que, se $A$ for diagonalizável, todos os seus auto-valores serão positivos.
	\end{observacao}
	
	Note a relação $f(u, v) = \sum_{i, j}u_{i}v_{j}f(e_{i}, e_{j}) = \sum_{i, j}u_{i}v_{j}A_{ij} = U^{t}AV = g_{A}(U, V)$, onde $U, V$ são as representações matriciais de $u$ e $v$, respectivamente. Assim, temos a seguinte definição.
	
	\begin{definicao}\label{definicao_matriz_positiva_definida}
		(Matriz positiva-definida) Se uma matriz $A \in \matrizquadreal{n}$ é tal que $g_{A}(U, V)\geq 0$ para todos $U,V \in M_{n\times 1}$, então diremos que $A$ é positiva-definida. Denotaremos por $\matrizsimetricapositiva{n}$ o conjunto de todas as matrizes em $\matrizquadreal{n}$ positivas-definidas.
	\end{definicao}
	
	Supondo que $\espectrooperador{A}$ seja o espectro de $A$, então $\autoespaco{\lambda}$ é o auto-espaço de $A$ associado a $\lambda \in \espectrooperador{A}$.
	
	\begin{definicao}\label{definicao_potenciacao_matriz}
		(Potênciação de matriz) Sejam $V$ um n-espaço vetorial real, $A:V \to V$ um operador linear diagonalizável. Dado $\alpha \in \real{}$ o operador linear $A^{\alpha}:V \to V$ é o operador definido por $A^{\alpha} = B^{-1}diag\{ \lambda_{1}^{\alpha} , \dots, \lambda_{2n}^{\alpha} \}B$, onde $B$ é a matriz tal que $A= B^{-1}diag\{ \lambda_{1}, \dots, \lambda_{2n}\}B$.
	\end{definicao}
	
	\begin{observacao}
		Note que, para um dado $v \in E_{\lambda}$, teremos $A^{\alpha}v = \lambda^{\alpha}v$, logo $v \in E_{\lambda^{\alpha}}$. Portanto, $E_{\lambda}\subset E_{\lambda^{\alpha}}$.
	\end{observacao}
	
	\begin{observacao}\label{observacao_transposta_potenciacao_matriz}
		Denotando $D = diag\{ \lambda_{1}^{\alpha} , \dots, \lambda_{2n}^{\alpha} \}$, temos da definição que
		$$
		(A^{\alpha})^{t} = B^{t}D^{t}(B^{-1})^{t} = B^{t}D(B^{t})^{-1} = (A^{t})^{\alpha}.
		$$
	\end{observacao}
	
	\begin{definicao}
		(Matriz ortogonal) O conjunto $\matrizortogonal{n} =\{ A \in \generalgroupreal{n}: AA^{t}=Id \}$ é denominado conjunto das matrizes ortogonais.
	\end{definicao}
	
	\begin{observacao}
		O conjunto das matrizes ortogonais forma um grupo com a operação de multiplicação de matrizes.
	\end{observacao}
	
	O resultado seguinte é uma caracterização de matrizes normais  utilizando matrizes ortogonais e matrizes positivas definidas. Sua demonstração pode ser encontrada em $\cite{hoffman_kunze}$.
	
	\begin{lema}\label{lema_caracterizacao_matriz_normal}
		(Caracterização matriz normal) Seja $A\in \matrizquadreal{n}$ uma matriz normal, isto é, $A^{t}A=AA^{t}$. Então existem uma matriz diagonal $D \in \matrizquadreal{n}$ positiva-definida e uma matriz $O\in \matrizortogonal{n}$ tais que $A=ODO^{t}$. Nesse caso, seus auto-valores são positivos. 
	\end{lema}
	
	\begin{observacao}\label{observacao_caracterizacao_matriz_normal}
		No caso de uma matriz complexa $A \in M_{n\times n}(\mathbb{C})$ teremos a decomposição $A=U^{*}DU$, onde $U \in \matrizunitaria{n}$.
	\end{observacao}
	
	\begin{lema}\label{lema_raiz_matriz_normal}
		(Raíz de matriz normal) Sejam $V$ um n-espaço vetorial e $A\in \matrizquadreal{n}$ uma matriz normal, então existe uma única $P\in \matrizsimetricapositiva{n}$ tal que $A=P^{2}$. 
	\end{lema}
	\begin{prova}
		Como $A$ é normal, então pelo Lema $\ref{lema_caracterizacao_matriz_normal}$ podemos escrever $A=ODO^{t}$, onde $D=(D_{ii})$ é uma matriz diagonal com entradas positivas e $O\in \matrizortogonal{n}$. Logo podemos definir $C \in \generalgroupreal{n}$ como sendo $C = (\sqrt{D_{ii}})$, o que implica em $C^{2} = D$. Definindo $P = OCO^{t}$ teremos $P^{2} = OCO^{t}OCO^{t} = OC^{2}O^{t} = ODO^{t}=A$. Temos que $P^{t} = (OCO^{t})^{t} = OCO^{t} = P$, pois $C$ é diagonal. Além disso, $P$ é semelhante a uma matriz diagonal, então pelo Teorema $\ref{teorema_matriz_positiva_definida}$ temos que $P \in \matrizsimetricapositiva{n}$. A unicidade vem do fato de que $C^{2} = D$ é única, logo $P$ é única.
	\end{prova}
	
	\begin{observacao}\label{observacao_raiz_matriz_normal}
		A matriz do Lema $\ref{lema_raiz_matriz_normal}$ é chamada raíz de $A$ e é denotada por $P=\sqrt{A}$.
	\end{observacao}
	
	O seguinte teorema é de grande importância pois é fundamental na caracterização do grupo simplético e investigação de sua topologia. Um caso mais geral que pode ser encontrado em $\cite{hoffman_kunze}$.
	
	\begin{teorema}\label{teorema_decomposicao_polar}
		(Decomposição polar) Se $(V, \produtointerno{}{})$ é um n-espaço vetorial real com um produto interno positivo-definido e $A \in \generalgroupreal{n}$, então podemos escrever $A=PO$ onde $P \in  \matrizsimetricapositiva{n}$ e $O \in \matrizortogonal{n}$. Além disso, essa decomposição é única.
	\end{teorema}
	\begin{prova}
		Por construção temos que $AA^{t}$ é normal. Então pelo Lema $\ref{lema_raiz_matriz_normal}$ existe uma única $P \in \matrizsimetricapositiva{n}$ tal que $P^{2} = AA^{t}$, isto é, $P = \sqrt{AA^{t}}$. Como $P$ é invertível, podemos definir $O = P^{-1}A$. Vejamos que $OO^{t} = P^{-1}AA^{t}(P^{-1})^{t} = P^{-1}AA^{t}(P^{t})^{-1} = P^{-1}AA^{t}P^{-1} = P^{-1}P^{2}P^{-1} = Id$, logo $O \in \matrizortogonal{n}$. Pela unicidade de $P$ temos que $O=P^{-1}A$ é única. Portanto $A=PO$, onde $P \in \matrizsimetricapositiva{n}$ e $O \in \matrizortogonal{n}$ são únicas.
	\end{prova}
	
	\begin{corolario}\label{corolario_decomposicao_matriz_antisimetrica}
		(Decomposição polar anti-simétrica) Considere as hipóteses no Teorema $\ref{teorema_decomposicao_polar}$ e tome $A \in \generalgroupreal{n}$ uma matriz anti-simétrica. Então $A = PO$, onde $P \in \matrizsimetricapositiva{n}$, $O \in \matrizortogonal{n}$ com  $O^{2} = -Id$ e $O^{t} = -O$.
	\end{corolario}
	\begin{prova}
		Temos que $AA^{t}$ é uma matriz normal, logo pelo Lema $\ref{lema_caracterizacao_matriz_normal}$ é diagonalizável e podemos escrever $AA^{t} = B^{-1}diag\{\lambda_{1}, \dots, \lambda_{n}\}B$. Logo temos 
		$$
		(AA^{t})^{1/2} =B^{-1}diag\{\lambda_{1}^{1/2}, \dots, \lambda_{n}^{1/2}\}B.
		$$
		Portanto, $P=(AA^{t})^{1/2}$ é positiva-definida, logo $P=P^{t}$. Sejam $D_{P}$ e $D_{A}$ as matrizes diagonais semelhantes a $P$ e $A$, respectivamente. Afirmo que $PA = AP$. De fato, $PA = B^{-1}D_{P}B B^{-1}D_{A}B = B^{-1}D_{A} D_{P}B = B^{-1}D_{A}B B^{-1}D_{P}B = AP$. Definindo $O = P^{-1}A$ teremos $OO^{t} = P^{-1}A(P^{-1}A)^{t} = P^{-1}AA^{t}(P^{t})^{-1} = P^{-1}AA^{t}P^{-1} = P^{-1}P^{2}P^{-1}  = Id$, portanto $O \in \matrizortogonal{n}$. Além disso, $O^{2} = P^{-1}AP^{-1}A = P^{-2}A^{2} = (AA^{t})^{-1}A^{2} = -Id$, onde usamos a anti-simetria e o fato de que $P^{-1}$ comuta com $A$. Pelo Teorema $\ref{teorema_decomposicao_polar}$ a decomposição $A=PO$ é única.
	\end{prova}
	
	Seja $T:V\to V$ um operador linear e $V$ um n-espaço vetorial sobre os complexos. Suponha que $A \in \matrizquadcomplexa{n}$ seja sua representação matricial. Mesmo que $A$ seja invertível, isso não garante que $T$ seja diagnalizável. Contudo, sob algumas hipósteses, podemos escrever $A$ de um modo que se assemelha a uma matriz diagonal, e com isso, decompormos $V$ como soma direta de subespaços vetoriais construidos a partir de $T$. Essa construção é o que chamamos de forma canônica de Jordan e será utilizada na análise dos auto-valores do grupo simplético.
	
	A diagonalização de um dado operador linear $T:V\to V$ nem sempre é garantida, consequentemente, não teremos a decomposição do espaço vetorial em auto-espaços $\autoespaco{\lambda}$, onde $\lambda\in \espectrooperador{T}$. Para contornar esse dificuldade técnica vamos utilizar a forma canônica de Jordan de $T$ e construir uma generalização de auto-espaços desse operador.
	
	\begin{teorema}\label{teorema_espectral_jordan}
		(Teorema Espectral) Sejam $T:V \to V$ um operador linear e $V$ um n-espaço vetorial complexo. Suponha que o polinômio característico de $T$ seja
		$$
		p_{T}(\lambda) = (\lambda - \lambda_{1})^{n_{1}}\dots (\lambda - \lambda_{k})^{n_{k}}
		$$
		em que $\lambda_{j} \in \sigma(T)$ sejam todos distintos. Então existem subespaços $V_{j} \subset V$ com $1\leq j \leq k$ invariantes por $T$ tais que 
		$$
		V = E_{1}\oplus \dots \oplus E_{k}.
		$$
		Além disso, $dim(E_{j}) = n_{j}$,  o polinômio de $T|_{E_{j}}$ é $m_{j}(\lambda) = (\lambda - \lambda_{j})^{m_{j}}$ e $E_{j} = Ker(T-\lambda_{j})^{m_{j}}$, em que $1\leq m_{j}\leq n_{j}$ é o comprimento do maior bloco de Jordan associado ao auto-valor $\lambda_{j}$.
	\end{teorema}
	
	\begin{definicao}\label{definicao_autoespaco_generalizado}
		(Auto-espaços generalizados) Sejam $A$ como no teorema anterior e $\lambda \in \sigma(A)$. O $\lambda-$auto-espaço de generalizado de $A$ é o conjunto $E_{\lambda} = \bigcup_{r \in \mathbb{N}} Ker(A - \lambda)^{r}$.
	\end{definicao}
	
	\chapter{Variedades Riemannianas}\label{apendice_variedade_riemanniana}
	
	Parte das construções adiante utilizarão a definição de aplicação exponencial para exibir um atlas, e consequentemente, uma estrutura de variedade, para espaços de funções. Já as definições de métricas riemannianas e conexões afins aparecerão em um determinado momento, mas apenas sua citaçao será feita, portanto, caso tenha familiaridade com esse conceitos, o capítulo não se faz necessário.
	
	\begin{definicao}\label{definicao_variedade_riemanniana}
		(Variedade Riemanniana) Sejam $M$ uma n-variedade diferenciável e $g:T_{p}M \times T_{p}M \to \real{}$, um produto interno positivo-definido para todo $p \in M$, então o par $g$ é chamada de métrica Riemanniana e o par $(M, g)$ é chamado de n-variedade Riemanniana.
	\end{definicao}
	
	O conceito de conexão afim esta intimamente relacionado com a forma de comparar um campo vetorial avaliado em pontos distintos da variedade. Uma das estratégias de se efetuar essa comparação é chamada de transporte paralelo e uma boa discussão pode ser encontrada em \cite{nakahara}. Por fim, veremos que a conexão afim é uma generalização do conceito de diferenciação de uma aplicação definida no espaço euclidiano.
	
	\begin{definicao}
		(Conexão afim) Uma conexão afim $\nabla$ em uma n-variedade diferenciável é a aplicação $\nabla: \campossuaves{M} \times \campossuaves{M} \to \campossuaves{M}$ tal que, dadas $f,h \in \funcoessuaves{M}$ e $X,Y,Z \in \campossuaves{M}$:
		\begin{enumerate}
			\item $\nabla_{fX+hY}Z = f\nabla_{X}Z+h\nabla_{Y}Z$
			\item $\nabla_{X}(Y+Z) = \nabla_{X}Y+ \nabla_{X}Z$
			\item $\nabla_{X}(fY) = X(f)Y+f\nabla_{X}Y$.
		\end{enumerate}
	\end{definicao}
	
	\begin{observacao}\label{observacao_conexao_afim}
		Seja $\{\partial_{j}(p)\}$ uma base ortonormal de $T_{p}M$ onde $\partial_{j} = \partial/\partial x_{j}$. Pode-se mostrar que, dados $X=\sum X_{j}\partial_{j}, Y=\sum Y_{j}\partial_{j} \in T_{p}M$, temos:
		$$
		\begin{aligned}
		\nabla_{X}Y &= 
		\sum_{k} \Big( \sum_{ij} X(Y_{k}) + X_{i}Y_{j} \Gamma^{k}_{ij}\Big)\partial_{k} 
		\\
		&= 
		\sum_{k} \Big( \sum_{ij} X_{i} (\partial_{i}(Y_{k}) + Y_{j} \Gamma^{k}_{ij})\Big)\partial_{k} 
		\\
		&= \sum_{k} (\nabla_{X}Y)^{k}\partial_{k}.
		\end{aligned} 
		$$
		
		Temos o operador linear $\nabla_{\partial_{i}}: \campossuaves{M} \to \campossuaves{M}$ tal que $\nabla_{\partial_{i}}Y = \sum_{j}  (\partial_{i}(Y_{k}) + Y_{j} \Gamma^{k}_{ij})\partial_{k} $. Pode-se mostrar que no caso em que a n-variedade seja o $\real{n}$, então $\Gamma_{ij}^{k}=0$ para todos $1\leq i,j,k \leq n$. Suponha que $\campossuaves{\real{n}} \ni Y = f\partial_{j}$ para algum $0 \leq j \leq n$ e todas as outras componentes nula, então $\nabla_{\partial_{i}}Y = \partial_{i}(f)\partial_{j} $, isto é, a conexão afim se reduz a derivada direcional em $\real{n}$.
	\end{observacao}
	
	\begin{observacao}\label{observacao_transporte_paralelo}
		(Transporte paralelo) Um meio de comparar campos vetoriais em espaços tangentes distintos de $M$ é através do transporte paralelo. Suponha que $\gamma:[0,1] \to N$ uma curva suave e $Y \in TN$. Então o transporte paralelo de $Y \in TM$ ao longo de $\gamma$ é dado por 
		$\nabla_{\frac{d\gamma}{dt}}Y$. Mais detalhes sobre podem ser encontrados em $\cite{nakahara}$.
	\end{observacao}
	
	Dizemos um campo $X \in \campossuaves{M}$ é paralelamente transportado ao longo de uma curva $\gamma:\real{} \to M$ se $\nabla_{\gamma'}X=0$. Uma curva na variedade é chamada de geodésica se o seu campo de velocidades $v(t) = \gamma'(t)$ é paralelamente transportado ao logo dela, isto é, $\nabla_{v}v=0$, o que implica que $\norma{v}$ constante.
	
	Seja $\gamma:[0,1] \to M$ a geodésica definida por $\gamma(t,p,v)$ tal que $\gamma(0,p,v) = p$ e $\gamma'(0,p,v) = v(p)$ com $\norma{v}$ constante. Então o comprimento de arco de $\gamma$ é definido por 
	$$
	L(\gamma) =  \int_{0}^{1}\norma{\gamma'}dt = \int_{0}^{1}\norma{v}dt = \norma{v},
	$$
	logo $\gamma([0,1]) \subset M$ é um arco iniciado em $p \in M$ cujo comprimento é $\norma{v}$.
	\begin{definicao}\label{definicao_aplicacao_exponencial}
		(Aplicação exponencial) Definimos $exp:U \subset TM \to M$ tal que $exp(p,v) = \gamma(1, p, v)$, onde $\gamma$ é a geodésica definida anteriormente. Essa aplicação é chamada aplicação exponencial.
	\end{definicao}
	
	Restringindo a aplicação expoencial a um ponto $p \in M$ arbitrário temos $exp_{p}:T_{p}M \to M$, isto é, $exp_{p}(v)$ é um ponto de $M$ conectado a $p$ por uma geodésica cujo comprimento de arco é $\norma{v}$. 
	
	A proposição a seguir, que pode ser encontrada em $\cite{manfredo_riemannian_geo}$, será utilizada na definição de uma aplicação exponencial para construirmos estruturas de variedade de dimensão infinita (variedade de Banach).
	
	\begin{proposicao}\label{proposicao_difeomorfismo_exponencial}
		(Difeomorfismo exponencial) Seja  $exp_{p}:T_{p}M \to M$ a aplicação exponencial anterioemente definida. Então existem $\epsilon>0$ e uma vizinhança aberta $B_{\epsilon}(0)$ do vetor nulo $0 \in T_{p}M$ tal que a restrição $exp_{p}:B_{\epsilon}(0) \to W \subset M$ é um difeomorfismo sobre algum aberto $W \subset M$.
	\end{proposicao}
	
	\begin{definicao}\label{definicao_gradiente_hessiana}
		(campo gradiente e a Hessiana) Sejam $M$ uma n-variedade diferenciável Riemanniana e $f\in \funcoessuaves{M}$. Então o gradiente de $f$ é definido como sendo o campo vetorial $\gradiente \in \campossuaves{M}$ tal que $df_{p}(v) = g(\gradiente, v)$ para todo $v \in T_{p}M$. A Hessiana de $f$ e a aplicação bilinear $H_{p}(f): T_{p}M\times T_{p}M \to \reta$ dada pelo diferencial do gradiente de $f$, isto é, $H_{p}(f) = d\gradiente(p)$.
	\end{definicao}
	
	
	\chapter{Variedades de Banach}\label{apendice_variedades_banach}
	
	Uma variedade de Banach é uma extensão da definição de variedades em dimensão finita para dimensões infinitas. Mais precisamente, tem-se um espaço topológico em que cada ponto possui uma vizinhaça que é homeomorfa a um aberto de um espaço de Banach. Parte desse texto estudo é apoiado em operadores definidos em espaços de funções, que podem não ter dimensão finita. Contudo, pode-se mostrar que alguns desses conjuntos possuem uma estrutura de variedades de Banach, e com isso, defini-se uma noção de espaço tangente nos mesmos. 
	
	\section{Distribuições - Motivação e Definições}
	Muitos problemas de análise clássica tiveram suas origens na modelagem de sistemas físicos, tais como, a dinâmica de uma corda vibrante, a propagação do calor em um meio condutor, a descrição de um oscilador com uma força externa, etc. Contudo, em alguns desses modelos surgiram objetos matemáticos cuja definição não fora sistematizada. Sabe-se que uma distribuição de carga elétrica é a fonte de um campo elétrico. Sendo esse campo elétrico a solução de uma das equações de Maxwell expressa por
	$$
	\nabla.E(x) = \rho(x),
	$$
	onde $E:\real{3} \to \real{3}$ é o campo elétrico, $\rho:\real{3}\to \reta$ é a densidade de carga, sendo ambos de classe $C^{\infty}$. O problema surge quando se tem uma distribuição puntual de carga, isto é, quando a densidade de carga dada por $\rho(V) = q/V$ é tomada quando $\lim_{V \to 0}\rho(V)$, onde $q$ é a carga total do sistema e $V$ o volume que a envolve. Essa situação é comum na Física e para modelá-la foi introduzido o símbolo
	$$
	\rho(x) = \lim_{V\to 0} \frac{q}{V} = q\delta(x - x_{0}),
	$$
	onde $x,x_{0} \in \real{3}$, $x_{0}$ é o ponto onde esta localizada a carga $q$ e 
	$$
	\funcaocond{\delta(x - x_{0})}{\infty}{x=x_{0}}{0}{x\neq x_{0}}\;\; \text{tal que} \;\; \int_{\real{3}} \delta(x) = 1.
	$$
	Note que o símbolo $\delta$ não esta bem-definido em $x_{0}$, portanto não é uma função e o problema 
	$$
	\nabla.E(x) = q\delta(x),
	$$
	não esta bem formulado matematicamente. 
	
	Em teoria de distribuição essa EDP terá uma solução em um sentido mais geral, o qual é chamada de solução distribuicional ou solução fraca. Aqui serão apresentadas apenas algumas definições desse formalismo com o objetivo principal de definirmos os espaços de Sobolev.
	
	Sejam $f:\Omega\to \reta$ uma função e $\Omega \subseteq \real{n}$ um aberto. O suporte de $f$ é o definido por $supp(f) = \overline{\{ x\in \Omega: f(x)\neq 0 \}}$. Denotaremos por $\funcoesteste=\funcoesdiferenciaveissupp{k}{\Omega} \subset \funcoesdiferenciaveis{k}{\Omega}$ o conjunto de todas as funções de classe $C^{k}$ com suporte compacto e definidas em $\Omega$. Os elementos de tal conjunto são chamados de funções de teste.
	
	\begin{observacao}
		É possível se mostrar que o espaço das funções de teste $\funcoesteste$ forma um espaço vetorial real, veja em $\cite{friedlander}$.
	\end{observacao}
	
	Seja $l^{p}(\Omega)$ o conjunto das funções cujos módulos são Riemann integráveis em $\Omega$, isto é, $\int_{\Omega}|f(x)| \in \reta$. É possível mostrar que $l^{p}(\Omega)$ é um espaço vetorial real e, munido da norma
	$$
	\norma{f}=\normaLpdefinicao{|f(x)|}{\Omega},
	$$
	forma um espaço de Banach. Porém, para os resultados a seguir utiliza-se o seguinte quociente:
	
	\begin{definicao}\label{definicao_espaco_Lp}
		(Espaços $\espacoLpGeral{p}{\Omega}$ e $\espacoLpcomp{\Omega}$) Sejam $f,g \in l^{p}(\Omega)$ e $\Omega(f,g) = \{x\in \Omega: f(x) \neq g(x)\}$. Diremos que $f \sim g$ se $\Omega(f,g)$ possui medida nula. Assim, o quociente $l^{p}(\Omega)/\sim$ será denotado por $\espacoLpGeral{p}{\Omega}$.
	\end{definicao}
	
	\begin{definicao}
		(Espaços $\espacoLpcomp{\Omega}$) O conjunto $\espacoLpcomp{\Omega} = \{f:\Omega\to \reta: f \in \espacoLp{K}, \forall K \subset \Omega \;\text{compacto}\}$ é chamado espaço das funções localmente p-integráveis.
	\end{definicao}
	
	\begin{observacao}
		No caso em quem $p=2$ teremos que $L^{2}$ é um espaço de Hilbert, o que esta demonstrado em $\cite{kreyszig_analise_funcional}$, e um espaço de Hilbert é um espaço vetorial munido de um produto interno positivo-definido e completo na métrica gerada por esse produto interno.
	\end{observacao}
	
	Dada $f \in \funcoesdiferenciaveis{k}{\Omega}$, denote a $j$-ésima derivada parcial $\partial f(x)/\partial x_{j}$ por $\derivadaparcialabrev{j}f(x) $. Um q-multi-índice é uma q-tupla $\alpha = (\alpha_{1}, \dots, \alpha_{q})$, onde $\alpha_{i} \in \mathbb{N}$ para $0\leq i \leq q$ e seu comprimento é definido por $|\alpha| = \sum_{i}\alpha_{i}$. Sendo $\beta$ um q-multi-índice, então $\alpha+\beta=(\alpha_{1}+\beta_{1}, \dots, \alpha_{q}+\beta_{q})$ e $\alpha\leq \beta$ se $\alpha_{i}\leq\beta_{i}$ para todo $0\leq i \leq q$. O operador multi-diferencial é definido por $\partial^{\alpha} f(x) = \partial^{|\alpha|}f(x)/\partial{x^{\alpha_{1}}_{1}} \dots \partial{x^{\alpha_{q}}_{q}}$, onde $|\alpha|\leq k$.
	
	\begin{definicao}
		(Espaço das distribuições) Seja $\Omega \subseteq \real{n}$ um aberto. Uma forma linear $f:\funcoesteste \to \reta$ é chamada de distribuição se, para cada compacto $K\subset \Omega$, existe $c\geq 0$ e $m \in \mathbb{N}$ tal que 
		$$
		|\produtointerno{f}{\phi}| \leq c \sum_{|\alpha|\leq m} \sup|\partial^{\alpha}\phi|,
		$$
		para todo $\phi \in \funcoesdiferenciaveissupp{\infty}{\Omega}$ com $supp (\phi) \subseteq K$. O espaço das distribuições é denotado por $\distribuicoes$.
	\end{definicao}
	
	Sejam $f \in \funcoesdiferenciaveis{0}{\Omega}$ e $\phi \in \funcoesteste$. Defina $\produtointerno{f}{\phi}=\int_{\Omega}f(x)\phi(x)$ e $K = supp(\phi) \subset \Omega$ compacto. Como $f$ é contínua, então a sua restrição a $K$ é limitada, logo $|\produtointerno{f}{\phi}|\leq sup(|\phi|)\int_{K}|f(x)|<\infty$ e $f \in \distribuicoes$. Além disso, é possível se mostrar que, se $\produtointerno{f}{\phi}= 0$, então $f=0$. Dessa forma, a aplicação $\funcoesdiferenciaveis{0}{\Omega}\ni f \mapsto \produtointerno{f}{.} \in \distribuicoes$ é injetora. Note que a condição para essa identificação foi $\int_{K}|f(x)|<\infty$, o que implica na condição de integrabilidade. Com isso, pode-se generalizar para o caso em que $f \in \espacoLpcomp{\Omega}$, de modo que ainda se tenha a injetividade $\espacoLpcomp{\Omega} \ni f \mapsto \produtointerno{f}{.} \in \distribuicoes$.
	
	Por fim, pode-se efetuar a diferenciação das distribuições. Em $\cite{friedlander}$ mostra-se que, dados $f \in \distribuicoes$ e $\alpha$ um k-multi-índice, então $\produtointerno{\partial^{\alpha}f}{\phi} = (-1)^{|\alpha|}\produtointerno{f}{\partial^{\alpha}\phi}$.
	
	\begin{definicao}
		(Delta de Dirac) A distribuição $\delta \in \distribuicoes$, chamada de Delta de Dirac, é definida por $\produtointerno{\delta}{\phi} = \phi(0)$, para todo $\phi \in \funcoesteste$.
	\end{definicao}
	
	\begin{exemplo}
		(Função de Heaviside) Seja $\theta:\Omega = \reta \to \reta$, a de função de Heaviside, definida por
		$$
		\funcaocond{\theta(x)}{1}{x\geq0}{0}{x<0}.
		$$
		Afirmo que o diferencial de $\theta$ é o delta de Dirac. De fato,
		$$
		\produtointerno{\partial\theta}{\phi} = -		\produtointerno{\theta}{\partial\phi} = -\int_{\reta_{+}}\partial \phi(x) = -\phi(0)\Big|^{\infty}_{0} = \phi(0) = \produtointerno{\delta}{\phi}.
		$$
		Como $\phi \in \funcoesteste$ é arbitrária, então $\partial \theta = \delta \in \distribuicoes$.
	\end{exemplo}
	
	\begin{observacao}
		No exemplo anterior determinamos que a diferencial da função de Heaviside é o delta de Dirac e escrevemos $\partial \theta = \delta$. Note que essa não é uma identidade no sentido usual, mas no sentido distribucional, ou seja, é uma identificação que tem sentido sob a operação de integração. Mais precisamente, dados $f,g \in \distribuicoes$ tais que $\produtointerno{f}{\phi}=\produtointerno{g}{\phi}$ para todo $\phi \in \funcoesteste$, então escreveremos $f=g$ no sentido distribucional ou sentido fraco. Com isso, sistematizamos o problema da física $\produtointerno{\nabla.E}{\phi} = \produtointerno{q\delta}{\phi}$, logo $\nabla.E = q\delta$, no sentido fraco.
	\end{observacao}
	
	Sejam $\Omega, \Omega'\subset \real{n}$ subconjuntos abertos. Um homomorfismo $h : \funcoesdiferenciaveissupp{\infty}{\Omega} \to \funcoesdiferenciaveissupp{\infty}{\Omega'}$ é chamado contínuo se, para todo compacto $K \subset \Omega$ existe um compacto $K' \subset \Omega'$ tal que $supp(h \circ \phi ) \subset K'$ se $supp(\phi) \subset K$, e também, para todo compacto $K \subset \Omega$ e todo $p \in \mathbb{N}$ existem $c = c(K,p), m = m(k,p) \in \mathbb{N}$ tais que 
	$$
	\sum_{|\alpha|\leq p}|\partial^{\alpha}(h\phi)|\leq c			\sum_{|\beta|\leq p}|\partial^{\beta}(\phi)|, \text{onde }\; \phi \in \funcoesdiferenciaveissupp{\infty}{K}.
	$$
	
	\begin{definicao}\label{definicao_distribuicao_adjunta}
		(Distribuição adjunta) Sejam $\Omega, \Omega'\subseteq \real{n}$ dois abertos. O adjunto $h^{*}: \distribuicoesgeral{\Sigma} \to \distribuicoesgeral{\Omega}$é definida por
		$$
		\produtointerno{h^{*}\circ f}{\phi} = 		\produtointerno{f}{h\circ\phi}.
		$$
	\end{definicao}\label{definicao_espalo_sobolev}
	
	\section{Espaço de Sobolev $W^{1,p}$}\label{secao_espaco_sobolev}
	\begin{definicao}\label{definicao_espaco_sobolev}
		(Espaço de Sobolev) Seja $\Omega \subset \real{n}$ um aberto limitado com bordo suave e $p \in \mathbb{N}$ tal que $p\geq 1$. Os espaços de Sobolev são os conjuntos
		$$
		W^{k,p} (\Omega)= \{ f\in \espacoLp{\Omega}: \exists g \in \espacoLp{\Omega} \; e \;\produtointerno{f}{\partial^{\alpha}\phi} = (-1)^{|\alpha|}\produtointerno{g}{\phi}\; \forall \phi \in \funcoesteste \; e\; |\alpha|\leq k\}.
		$$
		Quando existir uma $g \in \espacoLpGeral{p}{\Omega}$ diremos que $g = \partial^{\alpha}f$.
	\end{definicao}
	
	\begin{observacao}\label{observacao_espacos_sobolev_distribuicao}
		Note que, $W^{k,p}$ pode ser visto como um espaço de distribuições.
	\end{observacao}
	
	\begin{definicao}\label{definicao_espalo_sobolev_generalizado}
		(Espaço de Sobolev generalizado) Seja $\Omega\subset \real{n}$ um aberto. Então $\espacosobolevcontradominio{\Omega}{\real{k}}$ é o conjunto das aplicações  $f:\Omega \to \real{k}$ onde $f(x) = (f_{1}(x), \dots, f_{k}(x))$ tais que $f_{j} \in \espacosobolev{\Omega} $ para $1\leq j \leq k$.
	\end{definicao}
	
	\begin{exemplo}
		(O espaço $\espacosobolevgeral{p}{\Omega}$) Sejam $\Omega \{(x_{1}, \dots, x_{n})=x \in \real{n}: \norma{x}<1 \}$ e $f:\Omega\to \reta$ definida por $f(x) = |x_{1}|$. Como $f$ é limitada em $\Omega$, então é localmente compacta, logo $f \in \espacoLp{\Omega}$. Defina $g:\Omega\to \real{n}$ por
		$$
		\funcaocond{g(x)}{(\pm 1, 0, \dots, 0)}{\pm x_{1}>0}{(0, 0, \dots, 0)}{x_{1} = 0}.
		$$
		Como $\Omega$ é limitado, então $g_{j}\in \espacoLp{\Omega}$. Dado $\phi \in \funcoesteste$ tem-se que
		$$
		\begin{aligned}
		\produtointerno{g_{1}}{\phi} 
		&= \int_{\Omega}g_{1}(x)\phi(x)
		\\
		&=\int_{\Omega^{+}}\phi(x)+\int_{\Omega^{-}}(-1)\phi(x)+\int_{\{0\}}0\phi(x)
		\\ &=\int_{\Omega^{+}}\derivadaparcial{}{x_{1}}(x_{1}\phi(x)) - \int_{\Omega^{+}}x_{1}\derivadaparcial{\phi(x)}{x_{1}}
		\\
		&+\int_{\Omega^{-}}\derivadaparcial{}{x_{1}}(-x_{1}\phi(x)) + \int_{\Omega^{-}}x_{1}\derivadaparcial{\phi(x)}{x_{1}}
		\\ 
		&=
		\int_{\Omega^{+}\cup \Omega^{-}}\derivadaparcial{}{x_{1}}(|x_{1}|\phi(x)) - \int_{\Omega^{+}\cup \Omega^{-}}|x_{1}|\derivadaparcial{\phi(x)}{x_{1}}
		\\
		&=
		\underbrace{|x_{1}|\phi(x)\Big|_{\partial(\Omega^{+}\cup \Omega^{-})}}_{=0} - \int_{\Omega}|x_{1}|\derivadaparcial{\phi(x)}{x_{1}}
		\\
		&=
		-\produtointerno{f}{\derivadaparcialabrev{1}\phi},
		\end{aligned}
		$$
		onde $\Omega^{\pm} = \{x \in \Omega: \pm x_{1} >0\}$. Além disso, foi usados os fatos de que $supp(\phi) $ é um compacto no interior de $\Omega$ e $\phi$ restrita a $\partial \Omega$ é nula. Pela Definição \ref{definicao_espaco_sobolev} tem-se que $g_{1} = \derivadaparcialabrev{1}f$ (no sentido fraco). Os casos em que $2\leq j\leq n$ são triviais pois $g_{j} = 0$. Portanto $f\in \espacosobolev{\Omega}$.
	\end{exemplo}
	
	\begin{observacao}
		De acordo com a proposição a seguir, e demonstrada em $\cite{breazis_sobolev_spaces}$, os espaços de Sobolev $W^{1,p}$ munidos da norma $\normaWp{f} = \normaLp{f}+\sum_{j}\normaLp{\partial_{j} f}$, formam uma espaço de Banach.
	\end{observacao}
	
	\begin{proposicao}
		Os espaços de Sobolev $W^{1,p}$ são espaços de Banach separáveis para $1\leq p \leq \infty$.
	\end{proposicao}
	
	Quando $\Omega\subset\real{n}$ e $n=1$, tem-se que $\espacosobolev{\Omega} \subset C^{0}(\overline{\Omega})$, isto é, os elementos de $\espacosobolev{\Omega}$ são funções contínuas para $p \geq 1$. Por outro lado, quando $n>1$ teremos $\espacosobolev{\Omega} \subset C^{0}(\overline{\Omega})$ apenas para $p>n$. Contudo, o Teorema de mergulho de Sobolev (ou Teorema de Sobolev) garante a inclusão em $L^{p}$ e uma demonstração pode ser encontrada em $\cite{breazis_sobolev_spaces}$.	
	
	\begin{teorema}\label{teorema_sobolev}
		(Rellich-Kondrachov ou Mergulho de Sobolev) Sejam $\Omega \subseteq \real{n}$ e $1 \leq p < n$. Então $\espacosobolev{\Omega} \hookrightarrow L^{q}(\Omega)$ é uma inclusão contínua, onde $1/q = 1/p -1/n$. Além disso, se $\Omega$ for limitado com bordo $\partial\Omega$ de classe $C^{1}$, então a injeção é um operador compacto para $1/q>1/p -1/n$.
	\end{teorema}
	
	\begin{observacao}
		O teorema implica que existe uma constate $C>0$ tal que 
		$$
		\normaLgGeral{\xi}{q}{\Omega} \leq C \normaWpGeralDominio{\xi}{p}{\Omega},
		$$
		para todo $\xi\in \espacosobolev{\Omega}$.
	\end{observacao}
	
	
	\begin{definicao}
		(Variedade de Banach) Seja N um espaço topologico de Hausdorff. Uma família indexada $\{(U_{i}, \phi_{i})\}_{i \in I}$, cujos pares $(U_{i}, \phi_{i})$ são chamados de cartas, é um k-atlas de N se:
		\begin{enumerate}
			\item $N=\bigcup_{i\in I} U_{i}$
			\item $\phi_{i}:U_{i} \to \phi_{i}(U_{i})$ é um homeomorfismo
			\item $\phi_{i}\circ \phi_{j}^{-1}: \phi_{j}(U_{j}\cap U_{i}) \to \phi_{i}(U_{j}\cap U_{i}) $ é uma aplicação de classe $C^{k}$
		\end{enumerate}
	\end{definicao}
	
	Foi demonstrado \vermelho{(??? Fazer um exemplo no capitulo de fibrados)} que todo fibrado vetorial cujo espaço base é $\circulo$ é trivializável. Portanto, dado $\gamma\in\lacocontrateis$, tem-se que $\pullbackfibradotangenteM{\gamma} \cong \circulo\times \real{2n}$. Suponha que $h:\pullbackfibradotangenteM{\gamma} \to \circulo\times \real{2n}$ é uma trivialização e que, de acordo com a identifição realizada na Observação \ref{observacao_identificacao_secao_campo_vetorial}, $X \in \campossuaves{M}$ é uma seção de $TM$. 
	
	Seja $hX$ uma seção do fibrado trivial $\circulo\times \real{2n}$ associada a seção $X$ de $\pullbackfibradotangenteM{\gamma}$, e defina 
	$$
	\espacosobolev{\pullbackfibradotangenteM{\gamma}} = \{Y \in \campossuaves{M} : hY\in \espacosobolevcontradominio{\circulo}{\real{2n}}\},
	$$ 
	onde $W^{1,p}$ são os espaços de Sobolev definidos na Seção \ref{secao_espaco_sobolev}.
	
	Suponha que $h':\pullbackfibradotangenteM{\gamma} \to \circulo\times \real{2n}$ é uma outra trivialização. Então existe uma aplicação $f:\circulo\to \generalgroupreal{2n}$ tal que $(h'Y)(z) = f(z)(hY)(z)$. De fato, sejam $B'(z)$ e $B(z)$ 
	bases de $\real{2n}$ em $z\in \circulo$, respectivamente. Defina as aplicações $A:\circulo \to \generalgroupreal{2n}$ por $B'(z)=A(z)B(z)$ e $f_{A}:\circulo\times \real{2n}\to \circulo\times \real{2n}$ por $f_{A}(x, X(x))=(x, A(x)X(x))$.  Sem perda de generalidade, pode-se supor que $(h'Y)(z) = (z, Y'(z))$ e $(hY)(z) = (z, Y(z))$ e que $Y'(z)$ e $Y(z)$ estão nas bases $B'(z)$ e $B(z)$, respectivamente. Com isso, tem-se que $f_{A}((hY)(z)) = f_{A}(z, Y(z)) = (z, A(z)Y(z)) = (h'Y)(z)$. Portanto, $h'Y = f_{A}\circ hY$ e as trivializações $h$ e $h'$ definem as mesmas secões no fibrado $\circulo\times \real{2n}$, a menos de um isomorfismo.
	
	Desse resultado, pode-se afirmar que, para todo $Y \in \espacosobolev{\pullbackfibradotangente{\gamma}{M}}$, tem-se
	$$
	\normaWp{h'Y} \leq c(A)\normaWp{hY},
	$$
	onde $c(A) = \norma{A}\geq 0$.
	Logo, $h$ e $h'$ definem a mesma topologia para $\espacosobolev{\pullbackfibradotangenteM{\gamma}}$.
	
	Para a construção de um sistema local de cartas em $\lacocontrateis$ será utilizada a aplicação exponencial definida nos espaços tangentes de variedades Riemannianas. Uma breve apresentação pode ser vista no Apêndice \ref{apendice_variedade_riemanniana}.
	
	Seja $D \subset TM$ uma vizinhança da seção nula de $TM$. Então a aplicação exponencial $\exp_{\gamma}: \espacosobolev{\gamma^{*}D} \to C^{0}(\circulo;M)$ é definida por $\exp_{\gamma}(Y)(t) = \exp_{\gamma(t)}(Y(t))$.
	Tome a coleção $\{ (\espacosobolev{\gamma^{*}D}, \exp_{\gamma}) \}_{\gamma \in \lacocontrateis}$ e defina $\caminhosexponenciaisSobolev$ como sendo o conjunto das aplicações $\beta:\circulo\to M$ tais que existem $\gamma\in \lacocontrateis$ e $Y \in \espacosobolev{\gamma^{*}D}$ com $\beta(t)=\exp_{\gamma(t)}Y(t)$.
	
	\begin{proposicao}
		O conjunto $\caminhosexponenciaisSobolev$ é uma variedade de Banach de classe $C^{\infty}$. Além disso, 
		$$
		\lacocontrateis = C^{\infty}(\circulo;M) \subset  \caminhosexponenciaisSobolev\subset C^{0}(\circulo;M),
		$$
		onde cada um dos conjuntos da sequência é denso em seu sucessor.
	\end{proposicao}
	\begin{prova}
		Afirmo que $\caminhosexponenciaisSobolev$ satisfaz os axiomas das variedades de Banach. De fato,
		\begin{enumerate}
			\item Para todo $\beta \in \caminhosexponenciaisSobolev $ existem $\gamma_{\beta} \in \lacocontrateis$ e $Y_{\beta} \in \espacosobolev{\gamma_{\beta}^{*}D}$ tal que $\beta(t)=\exp_{\gamma_{\beta}(t)}Y_{\beta}(t)$. A Proposição \ref{proposicao_difeomorfismo_exponencial} garante a existência de uma vizinhança aberta $D_{\beta} \subset TM$ da seção nula de $TM$ onde a aplicação exponencial é injetora. Denotando $\espacosobolev{\gamma_{\beta}^{*}D_{\beta}}$ por $W_{\beta}$, tem-se que
			$$
			\caminhosexponenciaisSobolev = \bigcup_{\beta \in \caminhosexponenciaisSobolev } \exp_{\gamma_{\beta}}(W_{\beta}).
			$$
			
			\item Pela escolha de $D_{\beta}$ do item anterior e pela Proposição \ref{proposicao_difeomorfismo_exponencial}, pode-se afirmar que $\exp_{\gamma_{\beta}}:W_{\beta} \to \exp_{\gamma_{\beta}}(W_{\beta})$ é um difeomorfismo. Logo é um homeomorfismo.
			
			\item Como $\exp_{\gamma_{\beta}}$ e $\exp_{\gamma_{\alpha}}$ são difeomorfismos sobre suas imagens, então a composição $\exp_{\gamma_{\beta}}^{-1} \circ \exp_{\gamma_{\alpha}} :W_{\beta} \cap W_{\alpha} \to W_{\beta} \cap W_{\alpha}$ é um difeomorfismo. Além disso, como as aplicações exponenciais são de classe $C^{\infty}$, então a composição anterior também o é.
		\end{enumerate}
		
		Portanto $\caminhosexponenciaisSobolev$ é uma variedade de Banach de classe $C^{\infty}$.
		
		Por fim, a demonstração da cadeia de inclusão dos conjuntos e suas densidades podem ser encontrada em \cite{matthias_morse_homology}.
	\end{prova}
	
	
	\begin{thebibliography}{9}
		\bibitem{abramovich}
		Abramovich, Y. A.; Aliprantis, C. D.
		\emph{An Invitation to Operator Theory},
		American Mathematical Society, 2002.
		
		\bibitem{amyia_diff_topology}
		Amyia, Mukherjee:
		\emph{Topics in Differential Topology},
		Texts and Reading in Mathematics 34,
		2005.
		
		\bibitem{audi_floer_homology}
		Audi, Michèlle; Damian, Mihai:
		\emph{Morse Theory and Floer Homology},
		Springer, first edition,
		2010.
		
		\bibitem{banyaga_morse_homology}
		Banyaga, Augustin; Hurtubise, David
		\emph{Lectures on Morse Homology},
		Springer Science + Business Media, first edition,
		2004.
		
		\bibitem{breazis_sobolev_spaces}
		Brezis, Haim:
		\emph{Functional Analysis, Sobolev Spaces and Partial Differential Equantions},
		Springer, first edition,
		2011.
		
		\bibitem{cappell_maslov_index_equivalencia}
		Cappell, Sylvain E.; Lee, Ronnie; Miller, Edward Y.
		\emph{On Maslov Index}, Communication on Pure and Applied Mathematics, Vol. XLVII, 121-186 (1994).
		
		\bibitem{manfredo_riemannian_geo}
		Do Carmo, Manfredo P.:
		\emph{Riemannian Geometry},
		Birkhauser, 2nd edition,
		1992.
		
		\bibitem{doering_ode}
		Doering, Claus I.; Lopes, Artur O.:
		\emph{Equações Diferenciais Ordinárias - Coleção Matemática Universitária},
		IMPA, terceira edição,
		2008.
		
		\bibitem{guillemin_differential_topology}
		Guillemin, Victor; Pollack, Alan:
		\emph{Differential Topolgy},
		Prentice-Hall,
		1974.	
		
		\bibitem{hoffman_kunze}
		Hoffman, Kenneth; Kunze, Ray
		\emph{Linear Algebra},
		John Wiley and Sons, 2nd edition, 1971.
		
		\bibitem{kreyszig_analise_funcional}
		Kreyszig, Erwin
		\emph{Introduction to Functional Analysis with Applications},
		John Wiley and Sons, 1978.
		
		\bibitem{friedlander}
		Friedlander, F. G.; Joshi, M.
		\emph{Introduction to Theory of Distributions},
		Cambridge University Press, 2nd edition
		1998.
		
		\bibitem{elon_grupo_fundamental}
		Lima, Elon Lages:
		\emph{Fundamental Group and Covering Spaces},
		A K Peters, 2003.
		
		\bibitem{massey}
		Massey, William S.:
		\emph{A Basic Course in Algebraic Topology},
		Springer-Verlag, first edition,
		1991.
		
		\bibitem{milnor}
		Milnor, J.:
		\emph{Morse Theory},
		Princeton University Press, 1963.

		\bibitem{munkres_topology}
		Munkres, James R.:
		\emph{Topology},
		Prentice Hall Inc., 2000.
				
		\bibitem{nakahara}
		Nakahara, Mikio:
		\emph{Geometry, Topology and Physics},
		Graduate Student Series in Physics, 2nd edition,
		2003.
		
		\bibitem{nussenzveig}
		Nussenzveig, Moyses:
		\emph{Mecânica - Curso de Física Básica Vol 1},
		Edgard Blucher, 4 edição,
		2002.
		
		\bibitem{palis_dynamical_systems}
		Palis, Jaboc; Melo, Welington:
		\emph{Geometric Theory od Dynamical Systems},
		Springer-Verlag,
		1982.
		
		\bibitem{pontryagin_ode}
		Pontryagin, L.S.:
		\emph{Ordinary Differential Equations},
		Addison-Wesley Publishing Company,
		1962.
		
		\bibitem{salamon_lecture}
		Salamon, Dietmar:
		\emph{Lectures on Floer Homology},
		University of Warnick,
		1997.
		
		\bibitem{matthias_morse_homology}
		Schwarz, Matthias
		\emph{Morse Homology},
		Birkhauser, Boston
		1993.
		
		\bibitem{witten_supersymmetry_morse}
		Witten, Edward:
		\emph{Supersymmetry and Morse Theory},
		J. Differential Geometry 17,
		1982.
	\end{thebibliography}
	
\end{document}