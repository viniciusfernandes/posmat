\documentclass[12pt]{book}
\usepackage{graphicx}
\usepackage{indentfirst}
\usepackage[utf8]{inputenc}
\usepackage{amssymb}
\usepackage{enumitem}
\usepackage{color}
\usepackage[fleqn]{amsmath}
\usepackage[a4paper, margin=1.0in]{geometry}

\newcommand{\cohomologia}[2]{H^{#1}(#2)}
\newcommand{\cohomologiadual}[2]{H^{#1}(#2)^{*}}
\newcommand{\cohomologiacompac}[2]{H^{#1}_{c}(#2)}
\newcommand{\cohomologiacompacdual}[2]{H^{#1}_{c}(#2)^{*}}
\newcommand{\real}[1]{\mathbb{R}^{#1}}
\newcommand{\tese}[3]{\vspace{2mm} \textit{\textbf{#1}}: \textit{#2} \par $\square$ #3 \par $\blacksquare$}
\newcommand{\innerprod}[2]{\langle #1, #2 \rangle}
\begin{document}
	
	\title{Demonstrações Artigo Joa Weber}
	
	\author{Vinicius Fernades}
	
	\maketitle
	
	
	\vspace{8mm}
	\chapter{Teoria de Morse}
	
	\section{Pontos Críticos}
	
	\tese{Proposição}{Seja M uma variedade fechada e $f$ uma função de Morse, então se $p \in Crit(f)$ teremos $\nabla f(p)=0$ e $\exists q \in M$ tal que $p \in \omega(q)$ ou $p \in \alpha(q)$ , isto é, $p$ é uma singularidade e é um ponto limite.}{Tomando $p \in Crit(f)$, então $\forall v \in T_{p}M \Rightarrow Df(p)(v) = \innerprod{\nabla f(p)}{v} = 0$, portanto, para $v = -\nabla f(p)$ teremos $-\innerprod{\nabla f(p)}{\nabla f(p)} = 0 \iff \nabla f(p) = 0$, logo, $p$ é uma singularidade do campo gradiente. Com isso, podemos afirmar que existe um ponto $q \in M$ tal que a órbita $\mathcal{O}(q)$ tenha como um dos pontos limites o dado $p \in Crit(f)$ pois caso contrário, o ponto $p$ não será uma singularidade. Como $p \in Crit(f)$ é um ponto limite, então $p \in \omega(q)$ ou $p \in \alpha(q)$, como desejávamos.}
	
	\tese{Lema}{Seja M uma variedade fechada, $f$ uma função de Morse e $X =-\nabla f$, então $\alpha(p), \omega(p)$ consistem de um único ponto crítico de $f$ para qualquer $p \in M$.}
	{Como $M$ é uma variedade fechada e $f$ é uma função de Morse, então $Crit(f) = \{p_{i} \in M: 1\leq i \leq k \}$ é um conjunto finito de pontos isolados. Pela proposição anterior, dado $p_{i} \in Crit(f), \exists q_{i} \in M$ tal que $p_{i} \in \omega(q_{i})$ ou $p_{i} \in \alpha(q_{i})$. Com isso, podemos afirmar que $Crit(f) \subseteq \bigcup_{i=1}^{k}\omega(q_{i}) \cup \alpha(q_{i})$. Sabemos que os conjuntos limite $\omega(q_{i})$ e $\alpha(q_{i})$ consistem de singularidades do campo gradiente $-\nabla f$, logo $\forall p \in \bigcup_{i=1}^{k}\omega(q_{i}) \cup \alpha(q_{i})$ teremos $\nabla f(p) = 0 \Rightarrow Df(p)(v) = \innerprod{\nabla f(p)}{v} = 0 \therefore p \in Crit(f)$, e como $p$ é arbitrário, então $\bigcup_{i=1}^{k}\omega(q_{i}) \cup \alpha(q_{i}) \subseteq Crit(f)$, logo $Crit(f) = \bigcup_{i=1}^{k}\omega(q_{i}) \cup \alpha(q_{i})$. Com esse resultado fica evidente que $\omega(q_{i}), \alpha(q_{i}) \subset Crit(f)$. Como os conjuntos limites são conjuntos finitos de pontos isolados vamos supor que $\omega(q_{i}) = \{r_{j} \in Crit(f): 0\leq j \leq m\}$, logo pela topologia induzinda, os abertos de cada um deles serão os conjuntos unitários e teremos $\omega(q_{i}) = \bigcup_{j=1}^{k} \{r_{j}\} $, e como os conjuntos limite são conexos (vide resultado do Palis), essa união disjunta contradiz a conexidade, portanto $\omega(q_{i}) = \{r_{i}\}$. Os mesmos argumentos valem para o conjunto $\alpha(q_{j})$. Portanto, ambos os conjuntos limite devem conter apenas um ponto crítico, como desejávamos.}
\end{document}