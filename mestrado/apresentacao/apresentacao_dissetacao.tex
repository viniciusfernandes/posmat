\documentclass{beamer}
\mode<presentation>{
	\usetheme{Madrid}
	\setbeamercovered{transparent}
}

% PACOTES PARA DIAGRAMAS %
\usepackage[all,cmtip]{xy}
% PACOTES PARA DIAGRAMAS %

% QUALIFICADORES DOS RESULTADOS: TEOREMAS, LEMAS, COROLARIOS E PROVA %
\newtheorem{teorema}{Teorema}[section]
\newtheorem{corolario}[teorema]{Corol�rio}
\newtheorem{lema}[teorema]{Lema}
\newtheorem{definicao}[teorema]{Defini��o}
\newtheorem{exemplo}[teorema]{Exemplo}
\newtheorem{observacao}[teorema]{Observa��o}
\newtheorem{proposicao}[teorema]{Proposi��o}
\newenvironment{prova}[1]{$\square$ #1}{\hfill$\blacksquare$}
% QUALIFICADORES DOS RESULTADOS: TEOREMAS, LEMAS, COROLARIOS E PROVA %

% AQUI ESTAO OS COMANDOS %
\newcommand{\aplicacaoexponencial}[2]{exp_{#1}(#2)}
\newcommand{\aplicacaoexponencialgeral}[1]{exp_{#1}}
\newcommand{\aplicaoessuaves}[2]{C^{\infty}(#1, #2)}
\newcommand{\aplicaoessuavesreatacirculo}{C^{\infty}(\retacartesianocirculo, M)}
\newcommand{\autoespaco}[1]{E_{#1}}
\newcommand{\bigmodulo}[1]{\Bigm\lvert #1 \Bigm\lvert }
\newcommand{\bigparenteses}[1]{\Big( #1 \Big) }
\newcommand{\bordo}[1]{\partial_{#1}}
\newcommand{\bordorel}[1]{\overline{\partial}_{#1}}
\newcommand{\cadeia}[2]{C_{#1}(#2; A)}
\newcommand{\caminhosfechadoscirculo}[2]{L([#1,#2], S^{1})}
\newcommand{\caminhosfechadosSp}[2]{L([#1,#2], \gruposimpletico{2n})}
\newcommand{\caminhosdecaimentoexponencial}[2]{C^{\infty}_{\searrow}(#1, #2)}
\newcommand{\caminhosdecaimentoexponencialpadrao}{\caminhosdecaimentoexponencial{x^{-}}{x^{+}}}
\newcommand{\caminhosexponenciaisconectantesabrev}{\mathcal{P}}
\newcommand{\caminhosexponenciaisconectantes}[2]{\mathcal{P}^{1,p}(#1, #2)}
\newcommand{\caminhosexponenciaisconectantespadrao}{\caminhosexponenciaisconectantes{x^{-}}{x^{+}}}
\newcommand{\caminhos}{\mathcal{L}}
\newcommand{\caminhosfechados}[1]{\caminhos^{o}(#1)}
\newcommand{\caminhoslagrangianos}[3]{\caminhos_{#3}(#1,#2)}
\newcommand{\caminhoslagrangianosV}[2]{\caminhoslagrangianos{#1}{#2}{V}}
\newcommand{\caminhospontobase}[1]{\caminhos_{#1}}
\newcommand{\caminhossempontobase}[1]{\caminhos(#1)}
\newcommand{\caminhosNaoDegeneradosSp}{\caminhos^{*}(\gruposimpletico{2n})}
\newcommand{\caminhospontobasegeral}[2]{\caminhos_{#1}(#2)}
\newcommand{\caminhossuavesconectantes}[2]{\caminhos(#1, #2)}
\newcommand{\caminhossubespacoslagrangianos}[2]{L[#1,#2]}
\newcommand{\campogradiente}{\mathcal{X}}
\newcommand{\campogradientefuncional}{\mathcal{X}_{\mathcal{A}}}
\newcommand{\campohamiltoniano}[1]{X_{H}(#1)}
\newcommand{\campohamiltonianoabrev}{X_{H}}
\newcommand{\campossuaves}[1]{\mathfrak{X}(#1)}
\newcommand{\celula}[2]{D^{#1}_{#2}}
\newcommand{\celulabordo}[2]{\partial D^{#1}_{#2}}
\newcommand{\circulo}{S^{1}}
\newcommand{\circulovariedade}{\circulo\times M}
\newcommand{\cktopologia}[1]{\mathcal{C}^{#1}\text{-topologia}}
\newcommand{\classe}[1]{[#1]}
\newcommand{\cohomologia}[2]{H^{#1}(#2)}
\newcommand{\cohomologiadual}[2]{H^{#1}(#2)^{*}}
\newcommand{\cohomologiacompac}[2]{H^{#1}_{c}(#2)}
\newcommand{\cohomologiacompacdual}[2]{H^{#1}_{c}(#2)^{*}}
\newcommand{\colecao}[1]{\{#1_{k} \}_{k\in \inteiros}}
\newcommand{\colecaoabrev}[1]{\{#1 \}_{k\in \inteiros}}
\newcommand{\colecaofinita}[2]{\{#1_{j} \}_{j=1}^{#2}}
\newcommand{\colecaofinitaabrev}[2]{\{#1 \}_{j=1}^{#2}}
\newcommand{\complementar}[2]{#1 \backslash #2}
\newcommand{\complexificacao}[1]{#1_{\complexo{}}}
\newcommand{\complexificacaotensorial}[1]{\complexo{}\otimes_{\reta} #1}
\newcommand{\complexificado}[1]{\mathcal{V}}
\newcommand{\complexificacaoelemento}[2]{#1\otimes_{\reta} #2}
\newcommand{\complexo}[1]{\mathbb{C}^{#1}}
\newcommand{\diferencialfloer}{D\operadorflor}
\newcommand{\diferencialfloerabrev}{\mathcal{D}}
\newcommand{\derivada}[2]{\frac{d #1}{d #2}}
\newcommand{\derivadaparcial}[2]{\frac{\partial #1}{\partial #2}}
\newcommand{\derivadaparcialabrev}[1]{\partial_{#1}}
\newcommand{\diferencialhamiltoniano}[1]{(dX_{H})_{#1}}
\newcommand{\distribuicoes}{\distribuicoesgeral{\Omega}}
\newcommand{\distribuicoesgeral}[1]{\mathcal{D'}(#1)}
\newcommand{\energiafinitaM}{\mathcal{E}M}
\newcommand{\energiafinitaMconectante}{\energiafinitaM(x^{-}, x^{+})}
\newcommand{\espacoLdois}[1]{L^{2}(#1)}
\newcommand{\espacoLp}[1]{L^{p}(#1)}
\newcommand{\espacoLpcomp}[1]{L^{p}_{loc}(#1)}
\newcommand{\espacoLpcontradominio}[2]{L^{p}(#1;#2)}
\newcommand{\espacoLpGeral}[2]{L^{#1}(#2)}
\newcommand{\espacoLpretacirculo}{\espacoLpcontradominio{\retacartesianocirculo}{\real{2n}}}
\newcommand{\espacomoduli}[2]{\mathcal{M}_{#1#2}}
\newcommand{\espacoSimpleticoOrtogonal}[1]{#1^{\omega}}
\newcommand{\espacosobolev}[1]{W^{1,p}(#1)}
\newcommand{\espacosobolevcontradominio}[2]{W^{1,p}(#1;#2)}
\newcommand{\espacosobolevretacirculo}{\espacosobolevcontradominio{\retacartesianocirculo}{\real{2n}}}
\newcommand{\espacosobolevgeral}[2]{W^{1,#1}(#2)}
\newcommand{\espacotangente}[1]{\espacotangenteponto{p}{#1}}
\newcommand{\espacotangentevariedadeestavel}{T^{s}_{p}M}
\newcommand{\espacotangentevariedadeinstavel}{T^{u}_{p}M}
\newcommand{\espacotangenteponto}[2]{T_{#1}#2}
\newcommand{\espacotangentevariedade}{\espacotangenteponto{p}{M}}
\newcommand{\espectrooperador}[1]{\sigma(#1)}
\newcommand{\estruturacomplexa}{J_{0}}
\newcommand{\estruturascomplexas}[2]{\mathcal{J}(#1, #2)}
\newcommand{\estruturascomplexaspadrao}{\mathcal{J}(V, \omega)}
\newcommand{\fibradocaminhosexponenciais}{\mathcal{E}(x^{-}, x^{+})}
\newcommand{\fibradocaminhosexponenciaisabrev}{\mathcal{E}}
\newcommand{\formaSimpletica}[2]{\omega(#1, #2)}
\newcommand{\formaSimpleticaabrev}{\omega_{0}}
\newcommand{\formaSimpleticaExtendida}[2]{\Omega(#1, #2)}
\newcommand{\formaSimpleticaPadrao}[2]{\omega_{0}(#1, #2)}
\newcommand{\funcionalH}{\mathcal{A}_{H}}
\newcommand{\funcionalHponto}[1]{\mathcal{A}_{H}(#1)}
\newcommand{\funcoesdiferenciaveis}[2]{C^{#1}(#2)}
\newcommand{\funcoesdiferenciaveissupp}[2]{C^{#1}_{c}(#2)}
\newcommand{\funcoesmorse}[1]{\mathcal{M}_{o}(#1)}
\newcommand{\funcoesmorsesmale}[1]{\mathcal{M}^{S}_{o}(#1)}
\newcommand{\funcoessuaves}[1]{C^{\infty}(#1, \real{})}
\newcommand{\funcoesteste}{\mathcal{D}(\Omega)}
\newcommand{\generalgroup}[2]{GL(#1, #2)}
\newcommand{\generalgroupreal}[1]{\generalgroup{#1}{\real{}}}
\newcommand{\generalgroupcomplexo}[1]{\generalgroup{#1}{\complexo{}}}
\newcommand{\gradiente}{\nabla f}
\newcommand{\gradientefuncional}{\nabla \funcionalH}
\newcommand{\grupofundamental}[1]{\pi_{1}(#1)}
\newcommand{\grupofundamentalpontobase}[2]{\pi_{1}(#1; #2)}
\newcommand{\gruposimpletico}[1]{Sp(#1)}
\newcommand{\gruposimpleticocomplexo}[1]{Sp(#1; \complexo{})}
\newcommand{\gruposimpleticoreal}[1]{Sp(#1;\reta)}
\newcommand{\gruposimpleticoespecial}[1]{Sp^{1}(#1)}
\newcommand{\gruposimpleticonaodegenerado}[1]{Sp^{#1}(2n)}
\newcommand{\gruposimpleticopositivo}[1]{Sp_{+}(#1)}
\newcommand{\hessiana}{H_{p}(f)}
\newcommand{\homologia}[2]{H_{#1}(#2;A)}
\newcommand{\homologiaabrev}[2]{H_{#1}(#2)}
\newcommand{\homologiarel}[3]{H_{#1}(#2,#3)}
\newcommand{\homologiarelcel}[3]{H_{#1}(D^{#2}_{#3}, \partial D^{#2}_{#3})}
\newcommand{\homologiarelskele}[3]{H_{#1}(X^{(#2)}, X^{(#3)})}
\newcommand{\homologiarelskelesimpl}[2]{H_{#1}(X^{(#2)}, X^{(#2-1)})}
\newcommand{\imagembordo}[2]{B_{#1}(#2;A)}
\newcommand{\imagembordoabrev}[2]{B_{#1}(#2)}
\newcommand{\induzida}[1]{#1_{\#}}
\newcommand{\inteiros}{\mathbb{Z}}
\newcommand{\inteirospos}{\inteiros_{+}}
\newcommand{\iprod}[2]{\langle #1, #2 \rangle}
\newcommand{\intervalo}{[0,1]}
\newcommand{\kernelbordo}[2]{Z_{#1}(#2;A)}
\newcommand{\kernelbordoabrev}[2]{Z_{#1}(#2)}
\newcommand{\liederivada}[1]{\mathcal{L}_{#1}}
\newcommand{\operadorflor}{\mathcal{F}}
\newcommand{\operadorflordefinicao}[1]{\derivadaparcial{#1}{s} + J(#1)\derivadaparcial{#1}{t} - J(#1)X_{H}(#1)}
\newcommand{\operadorflorparametro}[1]{\mathcal{F}(#1)}
\newcommand{\operadorflorpadrao}{\operadorflorparametro{u}}
\newcommand{\matrizantisimetrica}[1]{Asym(#1)}
\newcommand{\matrizortogonal}[1]{O(#1)}
\newcommand{\matrizquadcomplexa}[1]{M_{#1 \times #1}(\complexo{})}
\newcommand{\matrizquadreal}[1]{M_{#1 \times #1}(\real{})}
\newcommand{\matrizsimetrica}[1]{Sym(#1)}
\newcommand{\matrizSimpleticaOrtogonal}{\mathcal{U}}
\newcommand{\matrizsimetricapositiva}[1]{Sym_{+}(#1)}
\newcommand{\matrizsimpleticapositiva}[1]{Sp_{+}(#1)}
\newcommand{\matrizunitaria}[1]{U(#1)}
\newcommand{\norma}[1]{||#1||}
\newcommand{\normagrande}[1]{\Big|\Big|#1\Big|\Big|}
\newcommand{\normaLp}[1]{||#1||_{L^{p}}}
\newcommand{\normaLpdefinicao}[2]{ \Big(\int_{#2}#1^{p}\Big)^{1/p}}
\newcommand{\normaLpDominio}[2]{||#1||_{L^{p}(#2)}}
\newcommand{\normaLgGeral}[3]{\norma{#1}_{\espacoLpGeral{#2}{#3}}}
\newcommand{\normapequenaLpdefinicao}[2]{ \normaLpdefinicao{\norma{#1}}{#2}}
\newcommand{\normagrandeLpdefinicao}[2]{ \normaLpdefinicao{\normagrande{#1}}{#2}}
\newcommand{\normasubscrito}[2]{\norma{#1}_{#2}}
\newcommand{\normaWp}[1]{||#1||_{W^{1,p}}}
\newcommand{\normaWpgeral}[2]{||#1||_{W^{1,#2}}}
\newcommand{\normaWpGeralDominio}[3]{\norma{#1}_{W^{1,#2}(#3)}}
\newcommand{\normaWpDominio}[2]{||#1||_{W^{1,p}(#2)}}
\newcommand{\operadorcauchyabrev}[1]{\overline{\partial}_{#1}}
\newcommand{\operadorfloer}{\mathfrak{F}}
\newcommand{\operadoresfredholm}[2]{\mathcal{F}r(#1, #2)}
\newcommand{\operadoreslimitados}[2]{\mathcal{B}(#1, #2)}
\newcommand{\orbitaponto}[1]{\mathcal{O}(#1)}
\newcommand{\orbitasconectantes}[2]{\mathcal{M}(#1, #2)}
\newcommand{\orbitasconectantespadrao}{\mathcal{M}(x^{-}, x^{+})}
\newcommand{\parteImaginaria}[1]{\mathcal{I}m(#1)}
\newcommand{\parteReal}[1]{\mathcal{R}e(#1)}
\newcommand{\pontoscriticos}[1]{\textit{Cr}(#1)}
\newcommand{\produtointerno}[2]{\langle #1, #2 \rangle}
\newcommand{\produtointernoabrev}{\langle ., .\rangle}
\newcommand{\produtosinternos}[1]{Riem(#1)}
\newcommand{\produtotensorial}[2]{ #1_{1} \otimes_{\mathbb{K}} \dots \otimes_{\mathbb{K}} #1_{#2}}
\newcommand{\produtotensorialabrev}[2]{#1\otimes #2}
\newcommand{\produtotensorialdual}{\produtotensorialabrev{\complexificado{V}^{*}}{\complexificado{V}^{*}}}
\newcommand{\produtotensorialreal}[2]{\bigotimes_{j=1}^{#1} #2_{j}}
\newcommand{\pullbackfibradotangente}[2]{#1^{*}T#2}
\newcommand{\pullbackfibradotangenteM}[1]{\pullbackfibradotangente{#1}{M}}
\newcommand{\pullbackfibradotangenteMpadrao}{\pullbackfibradotangente{u}{M}}
\newcommand{\retacartesianocirculo}{\real{} \times \circulo}
\newcommand{\retacartesianovariedade}{\real{} \times M}
\newcommand{\real}[1]{\mathbb{R}^{#1}}
\newcommand{\realprojetivo}[1]{\mathbb{R}P^{#1}}
\newcommand{\reta}{\real{}}
\newcommand{\subespacoslagrangianos}{L(V)}
\newcommand{\lacocontrateis}{\mathcal{L}M}
\newcommand{\somadir}[1]{\bigoplus \limits_{#1}}
\newcommand{\cilindrosLM}{\mathcal{C}M}
\newcommand{\morsefunc}[1]{\mathcal{M}o(#1)}
\newcommand{\skeleton}[1]{X^{(#1)}}
\newcommand{\variedadeconectante}{\variedadeconectantepontos{p}{q}}
\newcommand{\variedadeconectantepontos}[2]{W_{#1#2}}
\newcommand{\variedadeestavel}[1]{W^{s}(#1)}
\newcommand{\variedadeinstavel}[1]{W^{u}(#1)}
\newcommand{\vermelho}[1]{{\color{red}#1}}
% AQUI ESTAO OS COMANDOS %


\definecolor{azul}{rgb}{0,0,0.8}

\usepackage[portuguese]{babel}
\usepackage[utf8]{inputenc}

\usepackage{times}
\usepackage[T1]{fontenc}
\usepackage[all]{xy}


\setbeamercovered{highly dynamic}

\title[Short title]{Main Topic } % The short title appears at the bottom of every slide, the full title is only on the title page

\author{Vinicius Fernandes}
\title{O Índice de Maslov e suas Aplicações em Topologia Simplética: a homologia de Floer e a Conjectura de Arnold.}
\institute{Universidade Federal do ABC}
\date{18 de Junho de 2018}
\begin{document}
	
	\begin{frame}
		\titlepage % Print the title page as the first slide
	\end{frame}
	
	\begin{frame}
		\frametitle{
			A motivação.}
		\textbf{Conjectura de Arnold (V1):} \textit{Seja $\phi$ uma aplicação Hamiltoniana em uma 2n-variedade compacta simplética $(M, \omega)$. Então, o número de pontos fixos de $\phi$ é, no mínimo, o número de pontos críticos de uma função Hamiltoniana em $M$.}
		\\~\\
		
		\pause	
		\textbf{Conjectura de Arnold (V2):} \textit	{Seja $(M,\omega)$ uma 2n-variedade compacta e simplética. Defina  $H:M\times \real{} \to \reta$  uma função hamiltonia 1-periódica e suponha que as soluções 1-periódicas do sistema hamiltoniano sejam não-degeneradas. Então o número de soluções $\mathcal{N}$ desse sistema será limitado interiormente pela soma dos números de Betti de M, isto é:
			$$
			\mathcal{N}\geq \sum_{i=0}^{2n}\beta_{i}(M),
			$$
			onde $\beta_{i}(M)$ é a dimensão do i-ésimo grupo de homologia singular de $M$.}
	\end{frame}
	
	%-------------------------------------------------
	% FRAME
	%-------------------------------------------------
	\begin{frame}
		
		\begin{definicao}[Colagem de célula]
			Sejam $X$ um espaço topológico, $D^{k}$ o n-disco e $f_{\partial}:\partial D^{k} \to X$ contínua. Denotaremos a colagem da $k$-célula por $X\cup_{f_{\partial}}D^{k}$.
		\end{definicao}
		
		\begin{definicao}[CW-complexo]
			Dizemos que um espaço topológico $X$ tem uma CW-estrutura se existem uma sequência de espaços
			$$
			\skeleton{0} \subseteq \skeleton{1} \subseteq \dots \subseteq X = \bigcup \limits_{n\in \mathbb{N}} \skeleton{n}
			$$ 
			tais que:
			\begin{itemize}
				\item $\skeleton{0}$ é um conjunto discreto de pontos.
				
				\item $\skeleton{n+1}$ é obtido anexando $(n+1)-$células a $\skeleton{n}$.
				
				\item $X$ tem uma topologia fraca, ou seja, um dado $A \subseteq X$ é dito um aberto se, e somente se, $A \cap \skeleton{n}$ for um aberto em $\skeleton{n}$ para todo $n \in \mathbb{N}$.
			\end{itemize}
		\end{definicao}
	\end{frame}
	
	
	%-------------------------------------------------
	% FRAME
	%-------------------------------------------------
	\begin{frame}
		\begin{exemplo}[2-esfera]
			Fixemos um ponto-base $p \in S^{2}$ e definamos o $0-$esqueleto $\skeleton{0}=\{p\}$. Anexando uma $n-$célula a $\skeleton{0}$ teremos $f_{\partial}: \partial D^{2} \to \skeleton{0}$, isto é, $S^{2} \approx \skeleton{2} = \{p\}\cup_{f_{\partial}} \celula{n}{}$.
		\end{exemplo}
		
	\end{frame}
	
	%-------------------------------------------------
	% FRAME
	%-------------------------------------------------
	\begin{frame}
		
		\begin{lema}[Homologia celular]
			$\mathcal{C}_{k}(X)$ é um $A-$módulo livre e finitamente gerado pelas $k-$células de $X$ tal que 
			$$
			\mathcal{C}_{k}(X) \cong 
			\homologiarelskelesimpl{k}{k} \cong \somadir{\sigma} A.
			$$
		\end{lema}
		
		\begin{definicao}[Grau homológico]
			Sejam $f: S^{n} \to S^{n}$ é contínua e $[S^{n}] \in \homologia{n}{S^{n}}$ o gerador, então $f_{*}[S^{n}] = deg(f)[S^{n}]$, onde $deg(f) \in A$.
		\end{definicao}
		
		\pause
		\begin{teorema}[CW-bordo]
			É o homomorfismo $\partial_{k}:\mathcal{C}_{k}(X)\to \mathcal{C}_{k-1}(X)$ definido por
			$$
			\partial_{k}(\sigma) = \sum_{\beta}[\beta:\sigma]\beta,
			$$
			onde $[\beta:\sigma]$ é o grau homológico da aplicação $p_{\beta} \circ f_{\partial\sigma}:\celulabordo{n}{\sigma} \to S^{n-1}_{\sigma}$.
		\end{teorema}
		
	\end{frame}

	\begin{frame}
		\begin{teorema}[CW-homologia]
			Seja $X$ um CW-complexo, então existe uma identificação natural entre a CW-homologia $\mathcal{C}_{*}(X)$ e a homologia singular $\homologia{*}{X}$, isto é 
			$$
			\homologia{k}{X} \cong \homologia{k}{\mathcal{C}_{*}(X)}\; \forall k \in \inteiros.
			$$
		\end{teorema}
		
	\end{frame}

	%-------------------------------------------------
	% FRAME
	%-------------------------------------------------
	\begin{frame}
		\frametitle{Teoria de Morse}
		\begin{definicao}[Função de Morse]
			Sejam $M$ uma n-variedade fechada, $f \in \funcoessuaves{M}$ e $\pontoscriticos{f} = \{p \in M: df_{p} = 0\}$ o conjuntos dos pontos críticos de $f$. Dizemos que $f$ é uma função de Morse se a hessiana $H_{p}(f)$ é não-degenerada para todo $p \in \pontoscriticos{f}$. O conjunto das funções de Morse definidas em $M$ será denotado por $\funcoesmorse{M}$. 
		\end{definicao}
		\begin{lema}[Lema de Morse]
			Sejam $f \in \funcoesmorse{M}$ e $p \in \pontoscriticos{f}$. Então existe uma carta $\{U, \phi\}$ de $p$ com $\phi(p)=0 \in \real{n}$ tal que 
			$$
			(f\circ \phi^{-1})(x_{1}, \dots, x_{n}) = f(p)-x_{1}^{2}-\dots -x^{2}_{\lambda_{p}}+x^{2}_{\lambda_{p}+1}+\dots + x^{2}_{n}.
			$$
		\end{lema}
	\end{frame}
	
	%-------------------------------------------------
	% FRAME
	%-------------------------------------------------
	\begin{frame}
		
		\begin{definicao}
			O inteiro $\lambda_{p}$ no Lema anterior é chamado índice de Morse.
		\end{definicao}
		
		\pause
		\begin{teorema}
			Seja $g\in \funcoessuaves{M}$. Então existe $f \in \funcoesmorse{M}$ suficientemente próxima a $g$, isto é, $\funcoesmorse{M}$ é denso em $\funcoessuaves{M}$.
		\end{teorema}
	\end{frame}
	
	\begin{frame}
			\begin{itemize}
				\item $M^{a}= f^{-1}((-\infty, a]) = \{p \in M: f(p)\leq a\}$ o conjunto de nível $a$. 
				
				\item Dados $a\leq b \in \reta$, então $M^{a} \subseteq M^{b}$.
			\end{itemize}
			
			\pause
			\begin{teorema}
				Sejam $f \in \funcoesmorse{M}$ e $a<b \in \reta$ tais que $f^{-1}([a,b])\subset M$ é um compacto e não contém pontos críticos de $f$. Então $M^{a}$ é difeomorfo a $M^{b}$. Além disso, $M^{a}$ é um retrato de deformação de $M^{b}$, de modo que a inclusão  $M^{a} \hookrightarrow M^{b}$ é uma equivalência homotópica.
			\end{teorema}
			
	\end{frame}
	
	%-------------------------------------------------
	% FRAME
	%-------------------------------------------------
	\begin{frame}
		O seguinte teorema afirma que, fixando $a \in \reta$ e variando $t \in \reta$, se $f^{-1}([a,t]) \cap \pontoscriticos{f} \neq \emptyset$, os conjuntos de nível $M^{a}$ e $M^{t}$ não podem ser deformados um no outro.
		
		\begin{teorema}
			Sejam $f\in \funcoesmorse{M}$ e $p\in \pontoscriticos{f}$ com índice $\lambda_{p}$ tal que $f(p) = c$. Suponhamos que $f^{-1}([c-\epsilon,c+\epsilon])$ seja compacto e $f^{-1}([c-\epsilon,c+\epsilon])\cap \pontoscriticos{f} = \{p\}$ para algum $\epsilon>0$. Então o conjunto de nível $M^{c+\epsilon}$ tem o mesmo tipo de homotopia de $M^{c+\epsilon}$ com uma $\lambda_{p}$-célula colada, isto é, $M^{c+\epsilon} \simeq M^{c-\epsilon}\cup_{f_{\partial}} D^{\lambda_{p}}$.
		\end{teorema}
		
		\pause
		\begin{observacao}
			O teorema anterior tem como hipótese a exitência de apenas um ponto crítico em $f^{-1}([c-\epsilon,c+\epsilon])$. No caso em que $f^{-1}([c-\epsilon,c+\epsilon]) \cap \pontoscriticos{f} = \{p_{j}\}_{j=1}^{r}$, teremos $M^{c+\epsilon} \simeq M^{c-\epsilon}\cup_{f_{\partial_{1}}} D^{\lambda_{1}}\dots  \cup_{f_{\partial_{r}}} D^{\lambda_{r}}$.
		\end{observacao}
	\end{frame}
	
	%-------------------------------------------------
	% FRAME
	%-------------------------------------------------
	\begin{frame}
		
		\frametitle{Fluxos Gradiente e Variedades de Conexão}
		\begin{tiny}
			Para construirmos o operador bordo vamos olhar com os pontos críticos estão conectados.
		\end{tiny}
		
		\begin{definicao}[Gradiente negativo]
				$$
				\derivada{\gamma(t)}{t} = -\gradiente(\gamma(t)), \; \gamma(0) = p.
				$$
		\end{definicao}
		
		$$
		\derivada{}{t}(f \circ \gamma)(t) = g(\gradiente(\gamma(t)), \dot{\gamma}(t)) 
		=-\norma{\gradiente(\gamma(t))}^{2}
		\leq 0.
		$$
		\begin{tiny}
			Isso mostra que o fluxo sai de um ponto crítico de 
			índice maior para um ponto crítico de índice menor.
		\end{tiny}
		
		\pause
		\begin{definicao}[Variedades instáveis/estáveis]
			As variedades instável e estável de um ponto $p \in \pontoscriticos{f}$ são os conjutos $\variedadeinstavel{p} = \{q\in M: \phi_{t}(q) \to p,\; t\to -\infty\}$ e $\variedadeestavel{p} = \{q\in M: \phi_{t}(q) \to p,\; t\to \infty\}$.
		\end{definicao}
		
	\end{frame}
	
	\begin{frame}
		
		\begin{teorema}[Teorema da variedade instável/estável]
			Sejam $f \in \funcoesmorse{M}$ e $p \in \pontoscriticos{f}$. Então temos a decomposição $\espacotangentevariedade=\espacotangentevariedadeinstavel\oplus\espacotangentevariedadeestavel$. Além disso, existem mergulhos suaves e sobrejetores $\espacotangentevariedadeinstavel \hookrightarrow \variedadeinstavel{p} \subseteq M$ e $\espacotangentevariedadeestavel \hookrightarrow \variedadeestavel{p} \subseteq M$. Com isso, $\variedadeinstavel{p}$ e $\variedadeestavel{p}$ são subvariedades sem bordo com dimensão $\lambda_{p}$ e $n-\lambda_{p}$, respectivamente.
		\end{teorema}
	
		\begin{observacao}
			$\variedadeinstavel{p}$ e $\variedadeestavel{p}$ são homotopicamente equivalentes a discos.
		\end{observacao}
		
		\begin{tiny}
			Com isso, são orientáveis e podemos definir o número de intersecção.
		\end{tiny}
	\end{frame}
	
	\begin{frame}
		
		\begin{definicao}[Variedade Conectante e o Espaço Moduli]
			A variedade conectante de $p$ e $q$ é $\variedadeconectantepontos{p}{q} = \variedadeinstavel{p}\cap \variedadeestavel{q}$. Tomando $c \in (f(a), f(p)) \subset \reta$ um valor regular de $f$, o espaço moduli de $p$ e $q$ é $\espacomoduli{p}{q} = \variedadeconectantepontos{p}{q}\cap f^{-1}(c)$.
		\end{definicao}
		
		\begin{figure}
			\centering
			\textbf{}\par
			\includegraphics[width=4cm,height=2.25cm]{../imagem/pontos_cristicos_morse_smale}
		\end{figure}
		
		\begin{definicao}[Funções de Morse-Smale]
			Dizemos que o gradiente de $f \in \funcoesmorse{M}$ satisfazem a condição de Morse-Smale se $\variedadeinstavel{p}\pitchfork \variedadeestavel{q}$ para todos $p,q \in \pontoscriticos{f}$. O conjuntos dessas funções é denotado por $\funcoesmorsesmale{M}$.
		\end{definicao}
	
	\end{frame}
	
	%-------------------------------------------------
	% FRAME
	%-------------------------------------------------
	\begin{frame}
		\begin{proposicao}
			Sejam $f \in \funcoesmorsesmale{M}$ e $p,q \in \pontoscriticos{f}$.
			\begin{itemize}
				\item Se $\lambda_{p}<\lambda_{q}$, então $\variedadeconectantepontos{p}{q} = \emptyset$,
				
				\item $\variedadeconectantepontos{p}{p} = \{p\}$
				
				\item Se $\lambda_{p} = \lambda_{q}$ e $p\neq q$, então $\variedadeconectantepontos{p}{q} = \emptyset$,
				
				\item Se $p \neq q$ tal que $\variedadeconectantepontos{p}{q} \neq \emptyset$, então $\lambda_{p}>\lambda_{q}$.
			\end{itemize}
		\end{proposicao}
		
		Com isso, as óbitas não-singulares do fluxo do gradiente negativo dessa função sempre partem de um ponto crítico para um outro ponto crítico de índice inferior.

		\begin{tiny}
			Como as variedades instáveis/estáveis são homotópicas a discos, então são contráteis, logo orientáveis.
		\end{tiny}		
	\end{frame}
	
	%-------------------------------------------------
	% FRAME
	%-------------------------------------------------
	\begin{frame}
		
		As variedades instável/estável dos pontos críticos são contráteis, logo são orientáveis.
		
		\begin{definicao}[Número de intersecção]
			Dado $x \in \espacomoduli{p}{q}$ tal que $\lambda_{p}-\lambda_{q} = 1$
			
			\begin{enumerate}
				\item $o(\orbitaponto{x})_{ind} = n_{x}o(\phi(x))$, $n_{x}$ é o sinal característico da órbita de $x$.
				
				\item $n(p,q) = \sum_{x \in \espacomoduli{p}{q} }n_{x}$ é o número de intersecçao de $p$ e $q$.
			\end{enumerate}
			
		\end{definicao}
		\begin{figure}
			\centering
			\textbf{}\par
			\includegraphics[width=4cm,height=2.25cm]{../imagem/pontos_cristicos_morse_smale}
		\end{figure}
	\end{frame}
	
	\begin{frame}
		\frametitle{O complexo de Morse-Smale-Witten}
		\begin{definicao}[Complexo de Morse-Smale-Witten]
			$C_{k}(f)$ o grupo abeliano livremente gerado pelos pontos críticos de índice $k$ 
			\\~\\
			$\partial_{k}: C_{k}(f)\to C_{k-1}(f)$ definido em cada gerador p de $C_{k}$
			$$
			\partial_{k}\gerador{p}=\sum_{q \in Cr_{k-1}(f)}n(p,q)\gerador{q}
			$$
			\\~\\
			O par $(C_{*}(f), \partial_{*})$ é o complexo de cadeias de Morse-Smale-Witten  da função $f$.
		\end{definicao}
		
		
		\begin{teorema}[Homologia de Morse]
			 A homologia $H_{*}((C_{*}(f), \partial_{*})) $ é isomorfa a homologia singular $H_{*}(M, \inteiros)$.
		\end{teorema}
	\end{frame}
	
	
	
	\begin{frame}
		
		\frametitle{Exemplo Complexo Morse-Witten}
		
		\begin{minipage}[t]{0.5\linewidth}
			
			\begin{itemize}
				\item Seja $F:\real{3} \to \reta$ definida por $F(x, y, z) = z$. Temos que $\pontoscriticos{f} = \{p,q,r,s\}$ e $\lambda_{q} = 0$, $\lambda_{p} = 1$ e $\lambda_{r}=\lambda_{s} = 2$. 
				
				\item $\espacomoduli{p}{q}=\{x_{1}, x_{2}\}$, $\espacomoduli{r}{p} = \{ x_{3}\}$, $\espacomoduli{s}{p} = \{x_{4} \}$.
				
				\item Pela orientação escolhida para a variedade instável $\variedadeinstavel{p}$, temos $n_{x_{1}}=1$, $n_{x_{2}} = -1$. E a orientação escolhida para $\variedadeinstavel{r}$ e $\variedadeinstavel{s}$ nos dá $n_{x_{3}} =n_{x_{4}}= 1$. Portanto 
				$$
				n(p,q) = n_{x_{1}}+n_{x_{2}} = 0,\; n(r,p) = n(s,p)=n_{x_{3}}= 1. 
				$$			
			\end{itemize}
		\end{minipage}
		\hfill%
			\begin{minipage}[t]{0.4\linewidth}
				\begin{figure}
					\centering
					\textbf{}\par
					\includegraphics[width=5cm,height=4cm]{../imagem/pontos_cristicos_morse_smale}
				\end{figure}
				
			\end{minipage}
	
	\end{frame}
	
	\begin{frame}
		
		\begin{minipage}[t]{0.5\linewidth}
			\begin{itemize}
				\item As $k$-cadeias do complexo são
				$$
				\begin{aligned}
				C_{0}(f) &= \inteiros\gerador{q}, 
				\\
				C_{1}(f) &= \inteiros\gerador{p},
				\\
				C_{2}(f) &= \inteiros\gerador{r}\oplus\inteiros\gerador{s}.
				\end{aligned}
				$$
				
				\item Os $k$-operadores bordo são
				$$
				\begin{aligned}
				\bordo{0}\gerador{q} &=0, 
				\\
				\bordo{1}\gerador{p} &=n(p,q)\gerador{q} = 0,
				\\
				\bordo{2}\gerador{r}&=\bordo{2}\gerador{s} = n(r,p)\gerador{p} = \gerador{p}.
				\end{aligned}
				$$
				
			\end{itemize}
		\end{minipage}
		\hfill%
		\begin{minipage}[t]{0.4\linewidth}
			\begin{figure}
				\centering
				\textbf{}\par
				\includegraphics[width=5cm,height=4cm]{../imagem/pontos_cristicos_morse_smale}
			\end{figure}
			
		\end{minipage}
		
	\end{frame}
	

	%-------------------------------------------------
	% FRAME
	%-------------------------------------------------
	\begin{frame}
		\frametitle{Espaço Vetorial Simplético}
		\begin{definicao}[Espaço vetorial simplético]
			Sejam $V$ um 2n-espaço vetorial real e uma forma bilinear anti-simétrica $\omega$ em $\Lambda^{2}(V, \real{})$ tal que $\omega(u,v) = 0 \; \forall v \in V \Rightarrow u=0$. Então dizemos que $\omega$ é não-degenerada e o par $(V, \omega)$ é chamado de 2n-espaço vetorial simplético.
		\end{definicao}
		\begin{definicao}[Base simplética]
			Seja $(V, \omega)$ um 2n-espaço vetorial simplético, então uma base simplética é uma base $\{ e_{1},\dots, e_{n},f_{1},\dots f_{n}\}$ de $V$ tal que valem as relações:
			$$
			\omega(e_{i}, e_{j}) = \omega(f_{i}, f_{j}) = 0, \; \omega(e_{i}, f_{j}) = \delta_{ij}.
			$$
		\end{definicao}
		
	\end{frame}
	
	\begin{frame}
		
		\begin{teorema}[Existência de base simplética]
			Todo espaço vetorial simplético de dimensão finita possui uma base simplética.
		\end{teorema}
		
		\begin{observacao}
			Na base simplética a representação matricial de $\omega$ é $\left(
			\begin{array}{cc}
			0 & -Id
			\\
			Id & 0
			\end{array}
			\right) $.
		\end{observacao}
	\end{frame}
	
	%-------------------------------------------------
	% FRAME
	%-------------------------------------------------
	\begin{frame}
	\begin{exemplo}
			\begin{enumerate}
				\item $V = \real{2}$, $\{e_{x}, e_{y}\}$ uma base de $V$ e $\omega=dx \wedge dy$. Então 
				$$
				\begin{aligned}
					\omega(e_{x}, e_{y}) &=dx(e_{x}) dy(e_{y}) - dx(e_{y}) dy(e_{x}) = 1-0= 1 \; e\; 
					\\
					\omega(e_{x}, e_{x}) &= \omega(e_{y}, e_{y}) = 0	
				\end{aligned}
				$$ 
				
				\item Dado $v \in V$ tal que $\formaSimpletica{v}{u}=0 \;\;\forall u \in V$, temos que 
				$$
				\begin{aligned}
					\omega(v, u) &= \omega(v_{x}e_{x}+v_{y}e_{y}, u_{x}e_{x}+u_{y}e_{y})
					\\
					&=
					v_{x}u_{y}\omega(e_{x}, e_{y}) +v_{y}u_{x}\omega(e_{y}, e_{x})
					\\
					&= v_{x}u_{y} -v_{y}u_{x} = 0
				\end{aligned}
				$$
				
				$u_{x} = 1$ e $u_{y} = 0 \Rightarrow v_{y} = 0$ e $u_{x} = 0$ e $u_{y} = 1 \Rightarrow v_{x} = 0$, logo $v=0$.
				
			\end{enumerate}			
		\end{exemplo}
	\end{frame}
	
	%-------------------------------------------------
	% FRAME
	%-------------------------------------------------
	\begin{frame}
		\frametitle{Estruturas complexas}
		
		\begin{definicao}[Estrutura Complexa]
			\begin{itemize}
				\item $J: V \to V$ é um endomorfismo tal que $J^{2} = -Id$. 
				
				\item $J$ é $\omega$-compatível se $g(u,v):=\omega(u, Jv)$ é um produto interno. 
				
				\item $\estruturascomplexaspadrao$ o conjunto de todas as estruturas complexas $\omega$-compatíveis.
				
			\end{itemize}
		\end{definicao}
		Fixaremos a notação
			$
			\estruturacomplexa=
			\left(
			\begin{array}{cc}
			0 & -Id
			\\
			Id & 0
			\end{array}
			\right) \in \estruturascomplexaspadrao .
			$
	\end{frame}
	
	\begin{frame}
		
		\begin{definicao}
			$\produtosinternos{V}$ é o conjunto de todos os produtos internos positivos-definidos em $V$.
		\end{definicao}
		
		
		\begin{proposicao}
			Se $V$ é um n-espaço vetorial real, então $\produtosinternos{V}$ é contrátil.
		\end{proposicao}
		
		\begin{proposicao}
			$\estruturascomplexaspadrao$ é homeomorfo a $\produtosinternos{V}$, logo é contrátil.
		\end{proposicao}
	\end{frame}
	
	%-------------------------------------------------
	% FRAME
	%-------------------------------------------------
	\begin{frame}
		\frametitle{Grupo Simplético}
		
		\begin{definicao}[Transformação simplética]
			Seja $(V, \omega)$ um 2n-espaço vetorial simplético sobre $\reta$. Um operador linear $T: V \to V$ é uma transformação simplética se 
			$$
			\formaSimpletica{Tu}{Tv} = \formaSimpletica{u}{v}
			$$ para todo $u,v\in V$.
		\end{definicao}
		
		\begin{definicao}[Grupo simplético]
			O grupo simplético $\gruposimpletico{2n} \subset \generalgroupreal{2n}$ de $V$ é o conjunto das matrizes associadas as transformações simpléticas definidas em $V$.
		\end{definicao}	
		
	\end{frame}
	
	\begin{frame}
		\begin{proposicao}
			$\gruposimpletico{2n}$ é um grupo com a operação de multiplicação de matrizes.
		\end{proposicao}
		
		\begin{lema}[Caracterização de $Sp(2n)$]
			 Considere $V = \real{2n}$ e $\estruturacomplexa \in \estruturascomplexaspadrao$. Então $A\in Sp(2n)$ se, e somente se, $A^{t}\estruturacomplexa A = \estruturacomplexa$.
		\end{lema}
	\end{frame}
	
	\begin{frame}
		Realizando a identificação
		$$
		\generalgroupcomplexo{n}\leftrightarrow \generalgroupreal{2n}
		$$
		$$
		B+iC \leftrightarrow
		\left(
		\begin{array}{cc}
		B & -C
		\\
		C & B
		\end{array}
		\right)
		$$  
		\begin{lema}
			$F|_{\matrizunitaria{n}}: \matrizunitaria{n} \to \matrizSimpleticaOrtogonal $, onde $\matrizSimpleticaOrtogonal  = \gruposimpletico{2n}\cap \matrizortogonal{2n}$ é um isomorfismo. Além disso, dado $A \in \matrizSimpleticaOrtogonal $ temos $A\estruturacomplexa=\estruturacomplexa A$.
		\end{lema}
	\end{frame}
		
	\begin{frame}
		\frametitle{A topologia do grupo simplético}
		
		\begin{enumerate}
			\item $\matrizSimpleticaOrtogonal$ é conexo
			
			\item $\gruposimpletico{2n}$ é conexo
			
			\item $\gruposimpletico{2n}/\matrizSimpleticaOrtogonal$ é contrátil.
			
			\item $\grupofundamental{\matrizSimpleticaOrtogonal} \cong \inteiros$.
			
		\end{enumerate}
		
		\begin{teorema}
			$\grupofundamental{\gruposimpletico{2n}} \cong \inteiros$.
			
		\end{teorema}
	\end{frame}
	
	\begin{frame}
		Definamos os conjuntos $\gruposimpleticonaodegenerado{*} = \{ A \in \gruposimpletico{2n}: det(Id-A)\neq 0 \}$, $\gruposimpleticonaodegenerado{+} = \{ A \in \gruposimpletico{2n}: det(Id-A)> 0 \}$ e  $\gruposimpleticonaodegenerado{-} = \{ A \in \gruposimpletico{2n}: det(Id-A)< 0 \}$.
		
		\begin{lema}
			$\gruposimpleticonaodegenerado{*} = \gruposimpleticonaodegenerado{+}\cup \gruposimpleticonaodegenerado{-}$, onde $\gruposimpleticonaodegenerado{\pm}$ são as duas componentes conexas por caminhos, portanto conexas.
		\end{lema}
		
	\end{frame}
	
	\begin{frame}
		\frametitle{Aplicação $\rho$}
		
		O primeiro passo é construir uma aplicação contínua $\rho: \gruposimpletico{2n}\to \circulo$ que será a extenção contínua da aplicação determinante $det:\matrizSimpleticaOrtogonal\to \circulo$ em $\matrizSimpleticaOrtogonal$ e que satisfaz as seguintes propriedades:
		
			\begin{enumerate}
				\item \textbf{Naturalidade:}  Se $T:V_{1} \to V_{2}$ é um isomorfismo simplético, isto é, $T^{*}\omega_{2} = \omega_{1}, $então 
				$$
				\rho(TAT^{-1}) = \rho(A)
				$$
				para toda $A\in \gruposimpletico{V_{1}}$.
				
				\item \textbf{Produto:} Se $(V,\omega) = (V_{1}\times V_{2},\omega_{1}\times \omega_{2})$, então
				$$
				\rho(A) = \rho(A_{1})\rho(A_{2})
				$$
				para $A\in \gruposimpletico{V}$ definida por $A(v_{1}, v_{2})=(A_{1}v_{1}, A_{2}v_{2})$, onde $A_{i} \in \gruposimpletico{V_{i}}$.
				
			
			\end{enumerate}
		
	\end{frame}
	
	\begin{frame}
		\begin{enumerate}
			
			\item \textbf{Deteminante:} Se $A\in \matrizSimpleticaOrtogonal$, então 
			$$
			\rho(A) = det(X+iY), \text{onde} \;	
			A=\left(
			\begin{array}{cc}
			X & -Y					\\
			Y & X
			\end{array}
			\right).
			$$
			Além disso, a aplicação indizida $\rho_{*}: \grupofundamental{\gruposimpletico{2n}} \to \grupofundamental{\circulo} \cong \inteiros$ é um isomorfismo.
			
			\item \textbf{Normalização:} Se $A \in \gruposimpletico{2n}$ com $\sigma(A)\cap \circulo = \emptyset$, então $\rho(A) = \pm 1$.
		\end{enumerate}
	\end{frame}
	
	\begin{frame}
		\frametitle{Índice de Maslov}
		\begin{definicao}
			Sejam $W^{\pm} \in Sp(2n)^{\pm}$ duas matrizes fixas. Para cada curva $\gamma:[0,1] \to Sp(2n)$ escolhamos o levantamento $\alpha:[0,1] \to \real{}$ de $\rho\circ \gamma$ tal que o diagrama abaixo comute
			$$
			\xymatrix{
				& & \real{}\ar[d]\ar[d]^{\text{exp}}
				\\
				[0,1 ]\ar[urr]^{\alpha} \ar[r]_{\gamma} & Sp(2n) \ar[r]_{\rho} & S^{1}
			}
			$$	
			Definamos $\varDelta(\gamma) = (\alpha(0) - \alpha(1))/\pi$. Dada $A \in Sp(2n)^{*}$ e uma determinada $\gamma_{A}:[0,1] \to Sp(2n)^{*}$ com $\gamma_{A}(0) = A$ e $\gamma_{A}(1) = W^{\pm}$, o número $\varDelta(\gamma_{A}) \in \real{}$ fixo.
			Seja $\psi:[0,1] \to Sp(2n)$ com $\psi(0)=Id$ e $\psi(1)=A$, então o índice de Maslov da curva $\psi$ é dado por $\mu(\psi) = \varDelta(\psi) - \varDelta(\gamma_{A})$.
			
		\end{definicao}
		
	\end{frame}
	
	\begin{frame}
			\frametitle{Índice de Maslov - Exemplo}
			\begin{figure}[!h]
				\centering
				\includegraphics[width=10cm,height=3.5cm]{../imagem/caminhos_especiais_em_Sp2n.pdf}
			\end{figure}
	\end{frame}
	
	\begin{frame}			
		\frametitle{Índice de Maslov - Propriedades}
		\begin{teorema}
			Se $\psi \in \dominioMaslov$, então o índice de Maslov satisfaz as seguintes propriedades:
			\begin{enumerate}
				\item $\mu(\psi) \in \inteiros$.
				
				\item \textbf{(Naturalidade)}\label{item_naturalidade_maslov} Se $\phi\in \caminhossempontobase{\gruposimpletico{2n}}$, então o caminho $\phi\psi\phi^{-1}: \intervalo \to \gruposimpletico{2n}$ definido por $(\phi\psi\phi^{-1})(t) = \phi(t)\psi(t)\phi^{-1}(t)$ é um elemento de $\caminhosespeciais{\gruposimpletico{2n}}$ e $\mu(\phi\psi\phi^{-1}) = \mu(\psi)$.
				
				\item \textbf{(Homotopia)} \label{item_homotopia_caminhos_teorema_indice_maslov} Dois caminhos $\psi_{1}, \psi_{2}\in \dominioMaslov $ com $\psi_{1}(0) = \psi_{2}(0)$ e $\psi_{1}(1) = \psi_{2}(1)$ são homotópicos se, e somente se, $\mu(\psi_{1}) = \mu(\psi_{2})$.
				
				\item \textbf{(Nulidade)} se $\espectrooperador{\psi(t)}\cap \circulo = \emptyset$ para todo $t\neq 0$, então $\mu(\psi) = 0$.
				
					\item \textbf{(Produto)} Se $n=p+q$, então  $\mu:\caminhosespeciais{\gruposimpletico{2p}\times \gruposimpletico{2q}} \to \inteiros$ é dado por $\mu((\psi_{p}, \psi_{q})) = \mu(\psi_{p})+\mu(\psi_{q})$.
			\end{enumerate}
		\end{teorema}
	\end{frame}
	
	\begin{frame}
		\frametitle{Índice de Maslov - Propriedades}
			\begin{teorema}\label{teorema_indice_maslov}
				Se $\psi \in \dominioMaslov$, então o índice de Maslov satisfaz as seguintes propriedades:
				\begin{enumerate}
				\item \textbf{(Determinante)} O sinal de $ \det(\psi(1)-Id)$ é $(-1)^{\mu(\psi)-n}$.
					
					\item \textbf{(Assinatura)}\label{item_assinatura_maslov} Se $S \in GL(2n)$ é uma matriz simétrica com a norma $||S|| < 2\pi$ e se $\psi(t) = exp(t\estruturacomplexa S)$, então 
					$$
					\mu(\psi) = Ind(S) - n,
					$$
					onde $Ind(S)$ é o número de auto-valores negativos de $S$ contatos com multiplicidade.
					
					\item \textbf{(Inversa)} $\mu(\psi^{-1}) = -\mu(\psi)$.
					
				\end{enumerate}
			\end{teorema}
	\end{frame}
	
	\begin{frame}
		
		\frametitle{Índice de Conley-Zehnder}
		\begin{enumerate}
			\item $A_{x}:\intervalo \to \gruposimpletico{2n}$ definida por $A_{x}(t) = D_{x(0)}\psi_{t}$
			
			\item $A_{x}(0) =Id$ e $\det(D_{x(0)}\psi_{1}-Id) \neq 0 \Rightarrow A_{x}(1)\in \gruposimpleticonaodegenerado{*}$ 
		\end{enumerate}
		
		O índice de Conley-Zehnder $\iconleyabrev:\pontoscriticos{\funcionalH} \to \inteiros$ é dado por 
			$$
			\iconley{x} = \mu(A_{x}).
			$$
		
		
		\begin{corolario}
			Sejam $(M,\omega)$ uma 2n-variedade simplética, $H:M\to \reta$ uma função Hamiltoniana autônoma e $x \in \pontoscriticos{H}$. Se $\norma{\hessianaponto{x}{H}}<2\pi$ (assumindo a $\cktopologia{2}$), então o índice de Conley-Zender $\iconley{x}$ de $x$ como uma solução periódica do sistema Hamiltoniano e seu índice $\ind(x)$ como ponto crítico da função $H$ são relacionados por
			$$
			\iconley{x} = \ind(x)-n.
			$$
		\end{corolario}
	\end{frame}
	
	
	\begin{frame}
		\frametitle{Funcional de Ação}
		\begin{definicao}
			
			Uma estrutura simplética em uma 2n-variedade diferenciável $M$ é uma 2-forma não-degenerada $\omega\in \Omega^{2}(M)$ tal que $d\omega=0$. O par $(M, \omega)$ é uma 2n-variedade simplética.
		\end{definicao}
		
		O conjunto das aplicações $x:\circulo \to M$ de classe $C^{\infty}$ e contráteis é denotado por $\lacocontrateis$.
		
		\begin{definicao}
			O funcional de ação é a aplicação $\funcionalH: \lacocontrateis\to \reta$ definida por
			$$
			\funcionalHponto{x} = -\int_{D^{2}}u^{*}\omega + \int_{0}^{1}H(x(t), t)dt,
			$$
			onde $D^{2} \subset \mathbb{C}$ é o disco fechado e $u:D^{2}\to M$ é tal que $u(e^{i2\pi t})=x(t)$, ou seja, $u$ é uma extensão de $x$ para o disco.
			
		\end{definicao}
	\end{frame}
	
	\begin{frame}
		\frametitle{Condição de Aesfericidade}
			\begin{enumerate}
				
				\item (Condição de Aesfericidade) Seja $u:S^{2} \to M$ uma aplicação suave, então a condição de aesfericidade é dada por 
				$$
				\int_{S^{2}} u^{*}\omega = 0.
				$$
				
				\item (Condição de Trivialização) 
				Para cada aplicação $w:S^{2}\to M$ de classe $C^{\infty}$ existe uma trivialização simplética do fibrado $\pullbackfibradotangenteM{w}$.
				
			\end{enumerate}
	\end{frame}
	
	\begin{frame}
		\frametitle{Diferencial do Funcional de Ação}
			$$
			\funcionalHponto{\tilde{x}} = -\int_{D^{2}}\tilde{u}^{*}\omega + \int_{0}^{1}H(\tilde{x}(s,t),t)dt.
			$$
				\begin{figure}[!h]
					\centering
					\includegraphics[width=4cm,height=3cm]{../imagem/extensao_solucao.pdf}
					\caption{As aplicações $x$ e $u$ e suas variações $\tilde{x}$ e $\tilde{u}$.}
					\label{figura_extensao_solucao}
				\end{figure}
		\begin{proposicao}[Diferencial do funcional de ação]
			$$
		D_{x}\funcionalH(Y) = \int_{[0,1]} \omega(\dot{x} - \campohamiltonianoabrev, Y)dt.
		$$
		\end{proposicao}
		
	\end{frame}
	
	\begin{frame}
				\frametitle{Diferencial do Funcional de Ação}
		\begin{enumerate}
			\item 	Se $x \in \pontoscriticos{\funcionalH}$, então
			$$
			D_{x}\funcionalH(Y) = 0 \Rightarrow \omega(\dot{x} - \campohamiltonianoabrev, Y)=0,\; \forall Y \in \espacotangenteponto{x}{\lacocontrateis}
			$$
			$$
			\dot{x}(t) =\campohamiltonianoabrev(x(t), t).
			$$
			Portanto, $x \in \pontoscriticos{\funcionalH}$ se, e somente se, $x$ é solução 1-periódica do sistema Hamiltoniano. 
			
			\item Seja o produto interno $ \produtointerno{X}{Y}_{x} = \int_{0}^{1}g_{x(t)}(X, Y)dt$. 
			
			O gradiente do funcional de ação é 
			$$
			D_{x}\funcionalH(Y) = \produtointerno{\gradientefuncional(x)}{Y}_{x}
			$$
			para todo $x \in \lacocontrateis$ e $Y \in \espacotangenteponto{x}{\lacocontrateis}$.
		\end{enumerate}
		
	\end{frame}
	
	\begin{frame}
		\frametitle{Gradiente do Funcional de Ação}
		\begin{footnotesize}
		$$
		\begin{aligned}
		\int_{[0,1]} \omega(\dot{x} - \campohamiltonianoabrev, Y)dt&=
		D_{x}\funcionalH(Y)
		\\ 
		&= \iprod{\gradientefuncional}{Y}_{x}
		\\
		&= \int_{[0,1]}g_{x(t)}(\gradientefuncional, Y)dt
		\\
		&=\int_{[0,1]} \omega_{x(t)}(\gradientefuncional, JY)dt
		\\
		&=\int_{[0,1]} \omega_{x(t)}(-J\gradientefuncional, Y)dt.
		\end{aligned}
		$$
				
			
		Logo
		$$
		\int_{[0,1]} \omega(\dot{x} - \campohamiltonianoabrev + J\gradientefuncional, Y)dt = 0.
		$$
		\begin{block}{O gradiente do funcional é dado por:}
			$$
			\gradientefuncional(x(t))= J_{x(t)}\dot{x}(t)-J_{x(t)}\campohamiltoniano{x(t), t}.
			$$
		\end{block}
		\end{footnotesize}
	\end{frame}
	
	\begin{frame}
		\frametitle{Equação de Floer}
		\begin{footnotesize}
	Suponha que $u :\reta\to \lacocontrateis$ seja uma solução do sistema 
	$$
	\derivada{}{s}u(s)|_{s=0} = -\gradientefuncional(x) \;\;\text{e}\;\;u(0)=x.
	$$
	Tais soluções são as trajetórias em $\lacocontrateis$ do campo gradiente negativo $-\gradientefuncional$. Tais soluções ser vistas como aplicações $u:\retacartesianocirculo \to M$ de classe $C^{1}$.
	\begin{figure}[!h]
		\centering
		\includegraphics[width=4.5cm,height=2.5cm]{../imagem/caminho_espaco_laco.pdf}
	\end{figure} 
		
		Com isso, tem-se a
		\begin{block}{Equação de Floer}
			$$
			\operadorFloerDefParametros{u}=0.
			$$
		\end{block}
		\end{footnotesize}
	\end{frame}
	
	\begin{frame}
		\frametitle{Funcional Energia}
		
		Seja $\cilindrosLM$ o conjunto dos caminhos de classe $C^{1}$ em $\lacocontrateis$. O funcional energia é a aplicação $E: \cilindrosLM \to \reta$ definida por $E(u)
		=-\frac{1}{2}\int_{\reta}u^{*}d\funcionalH$.
		$$
		\begin{aligned}
			E(u) &= -\frac{1}{2}\int_{\reta}-\norma{\gradientefuncional(u(s))}^{2}
			\\
			&= \frac{1}{2}\int_{\reta}\bigparenteses{ \int_{\circulo} \bigparenteses{\normagrande{\derivadaparcial{u}{s}(s,t)}^{2} + \normagrande{\derivadaparcial{u}{t}(s,t) - \campohamiltoniano{u(s,t)}}^{2}} dt }ds
		\end{aligned}
		$$
		
		\begin{enumerate}
			\item $E(u)\geq0$
			
			\item Se $u$ é uma solução da equação de Floer, então
			$$
			E(u)=\int_{\retacartesianocirculo}\normagrande{\derivadaparcial{u}{s}(s,t)}^{2} =0 \Leftrightarrow u \in \pontoscriticos{\funcionalH}.
			$$
			
			\item Se $\lim_{s\to \pm\infty}u(s)=x^{\pm}$, então $E(u)=\funcionalH(x^{-}) - \funcionalH(x^{+})<\infty$.
		\end{enumerate}
	\end{frame}

	\begin{frame}
		\frametitle{Espaço de trajetórias}
		O conjunto das soluções contráteis e de energia finita da equações de Floer é denotado por $\energiafinitaM$.
		
		\begin{teorema}\label{teorema_limite_solucoes_energia_finita}
			Suponha que todas as trajetórias 1-periódicas do campo Hamiltoniano $\campohamiltonianoabrev$ sejam não-degeneradas. Então para cada $u \in \energiafinitaM$ existem dois pontos $x^{-}, x^{+}\in \pontoscriticos{\funcionalH}$ tais que
			$$
			\lim_{s\to -\infty}u(s)=x^{-},\; \lim_{s\to \infty}u(s)=x^{+}\;\;
			$$
			em $C^{\infty}(\circulo;M)$. Além disso, 
			$$
			\lim_{s\to \pm \infty}\derivadaparcial{u}{s}(s,t) = 0,
			$$
			converge uniformemente em $t\in \circulo$.
		\end{teorema}
		
	\end{frame}
	
	\begin{frame}
	\frametitle{Espaço de trajetórias}		
	Sejam $x^{-}, x^{+}\in \pontoscriticos{\funcionalH}$. O conjunto dos caminhos $u \in \energiafinitaM$ tais $\lim\limits_{s\to \pm \infty}u(s) = x^{\pm}$ é denotado por $\energiafinitaMconectante$.
			
			\begin{observacao}\label{observacao_uniao_orbitas_conectantes}
				Uma consequência imediata do teorema anterior é que 
				$$
				\energiafinitaM = \bigcup_{x^{-}, x^{+} \in \pontoscriticos{\funcionalH}} \energiafinitaMconectante.
				$$
			\end{observacao}
			
	\end{frame}
	
	\begin{frame}
		\frametitle{ESpaço de Trajetórias - Propriedades}
		\begin{itemize}
			\item $\energiafinitaMconectante$ é uma variedade de Banach
			
			\item $\dim(\energiafinitaMconectante)=\iconley{x^{-}}-\iconley{x^{+}}$
			
			
		\end{itemize}
		Fixe $\alpha \in \reta$ e defina $\tilde{u} = u*\alpha$ por $\tilde{u}(s,t) = u(s+\alpha,t)$.  Defina o quociente
		$$
		\energiafinitaM(x^{-},x^{+}, H, J)/\reta=\{u*\reta: u\in \energiafinitaM(x^{-},x^{+}, H, J)\},
		$$
		onde $u*\reta = \{u*\alpha:\alpha \in \reta\}$.
		
		
		Denote 
		$$
		\quocientetrajetorias = \energiafinitaM(x^{-},x^{+}, H, J)/\reta.
		$$ 
		\begin{proposicao}
		O conjunto $\quocientetrajetorias$ é finito.
		\end{proposicao}
	\end{frame}
	
	\begin{frame}
		\frametitle{Complexo de Floer}
		\begin{footnotesize}
				
			\begin{itemize}
				\item Seja $
				\pontoscriticosordem{k}{\funcionalH} = \{ y \in \pontoscriticos{\funcionalH}: \iconley{y} \ =	 k\}$
				
				\item Considere $C^{F}_{k}(M, \inteiros_{2}; H,J)$ o espaço vetorial sobre o corpo $\inteiros_{2}$, cujos elementos são gerados por $x \in \pontoscriticosordem{k}{\funcionalH} $.
				
				\item Denote $C^{F}_{k}(M, \inteiros_{2}; H,J)$  por $C^{F}_{k}(H,J)$.
				
				\item O número de elementos de $\quocientetrajetorias$ módulo 2 é denotado por $n(x^{-},x^{+})$.
				
				\item 
				Considere o homomorfismo $\bordo{k}^{F}: C^{F}_{k}(H,J)\to C^{F}_{k-1}(H,J)$ de grau -1 definido em cada gerador por
				$$
				\bordo{k}^{F}(\gerador{x}) = \sum_{\iconley{y}=k-1}n(x,y)\gerador{y}
				$$
				e estendido por linearidade. 
				
				\item Tem-se que $\bordo{k-1}^{F}\circ\bordo{k}^{F} = 0$ e o par $(C^{F}_{*}(H,J), \partial^{F})$ é um complexo de cadeias.
			\end{itemize}
		
		\end{footnotesize}
		
	\end{frame}
\end{document}