\documentclass[12pt]{book}
\usepackage{graphicx}
\usepackage{indentfirst}
\usepackage[utf8]{inputenc}
\usepackage{amssymb}
\usepackage{enumitem}
\usepackage{color}
\usepackage[fleqn]{amsmath}
\usepackage[a4paper, margin=1.0in]{geometry}

\usepackage{amsthm, amssymb, amsfonts, amsmath}
\usepackage{graphicx}
\usepackage{tikz}
\usetikzlibrary{calc,shapes}
% \usepackage{enumitem}
\usepackage{mathtools}
\usepackage{mathrsfs}
\usepackage{tikz-cd}
\usepackage[all,cmtip]{xy}

\newtheorem{teorema}{Teorema}[section]
\newtheorem{lema}[teorema]{Lema}
\newtheorem{definicao}[teorema]{Definição}
\newtheorem{exemplo}[teorema]{Exemplo}

\newenvironment{prova}{$\square$}{\hfill$\blacksquare$}

\newcommand{\celula}[2]{D^{#1}_{#2}}
\newcommand{\celulabordo}[2]{\partial D^{#1}_{#2}}
\newcommand{\classe}[1]{[#1]}
\newcommand{\cohomologia}[2]{H^{#1}(#2)}
\newcommand{\cohomologiadual}[2]{H^{#1}(#2)^{*}}
\newcommand{\cohomologiacompac}[2]{H^{#1}_{c}(#2)}
\newcommand{\cohomologiacompacdual}[2]{H^{#1}_{c}(#2)^{*}}
\newcommand{\funcaocond}[5]{
	#1 = 
	\left\{
	\begin{array}{cc}
	#2, & #3\\
	#4, & #5\\
	\end{array}
	\right.
	}
\newcommand{\skeleton}[1]{X^{(#1)}}
\newcommand{\homologia}[2]{H_{#1}(#2)}
\newcommand{\homologiarel}[3]{H_{#1}(#2,#3)}
\newcommand{\homologiarelcel}[3]{H_{#1}(D^{#2}_{#3}, \partial D^{#2}_{#3})}
\newcommand{\homologiarelskele}[3]{H_{#1}(X^{(#2)}, X^{(#3)})}
\newcommand{\homologiarelskelesimpl}[2]{H_{#1}(X^{(#2)}, X^{(#2-1)})}
\newcommand{\real}[1]{\mathbb{R}^{#1}}
\newcommand{\realprojetivo}[1]{\mathbb{R}P^{#1}}
\newcommand{\somadir}[1]{\bigoplus \limits_{#1}}

\newcommand{\tese}[3]{\vspace{2mm} \textit{\textbf{#1}}: \textit{#2} \par $\square$ #3 \par $\blacksquare$}
\newcommand{\innerprod}[2]{\langle #1, #2 \rangle}
\newcommand{\morsefunc}[1]{\mathcal{M}o(#1)}
\newcommand{\pontocritico}[1]{\textit{Crit}(#1)}
\newcommand{\suavefunc}[1]{C^{\infty}(#1, \real{})}
\begin{document}
	
	\title{Homologia de Morse}
	
	\author{Vinicius Fernades}
	
	\maketitle
	
	\chapter{CW-Homologia}
	\section{Homologia singular}
	\vspace{2mm} Para cada $k \in \mathbb{N}$ temo 
	\begin{definicao}
		(Sequência da triplas) Sejam $X, Y, Z$ espaços topológicos tais que $Z \subseteq Y \subseteq X$. Então a sequência da tripla $(X,Y,Z)$ é dada por:
			\[
			\xymatrix{
				\dots \ar[r] & \homologiarel{k}{Y}{Z} \ar[r]^{i_{*}} & \homologiarel{k}{X}{Z} \ar[r]^{j_{*}} & \homologiarel{k}{X}{Y} \ar[r]^{\delta_{*}} & \homologiarel{k-1}{Y}{Z} \ar[r] & \dots
			}
			\]
			onde $i_{*},\;j_{*}$ são inclusões e $\delta_{*}$ o homomorfismo de conexão.
	\end{definicao}
	
	\section{CW-Homologia}
	
	\begin{definicao}
		(Célula anexada) Sejam $X$ um espaço topológico, $D^{n}=\{x\in \mathbb{R}^{n} : ||x|| \leq 1\}$ e $S^{n-1} = \partial D^{n}=\{x\in \mathbb{R}^{n} : ||x|| = 1\}$. Se $f_{\partial}:S^{n-1} \to X$ é uma função contínua, denotaremos por $X\cup_{f_{\partial}}D^{n}$ o espaço quociente da união disjunta $X \coprod D^{n}$ onde $x \in \partial D^{x} = S^{n-1}$ é identificado com $f_{\partial}(x) \in X$. Diremos que $X\cup_{f_{\partial}}D^{n}$ é obtido a partir de $X$ anexando uma $n-$célula e $f_{\partial}$ é chamado de mapa de anexação.
 	\end{definicao}
	
	\begin{definicao}
		Dizemos que um espaço topológico $X$ tem uma CW-estrutura se existem $\skeleton{n}$ espaços tais com 
		$$
		\skeleton{0} \subseteq \skeleton{1} \subseteq \dots \subseteq X = \bigcup \limits_{n\in \mathbb{N}} \skeleton{n}
		$$
		tal que:
		\begin{enumerate}
			\item $\skeleton{0}$ é um conjunto discreto de pontos
			\item $\skeleton{n+1}$ é obtido anexando $(n+1)-$células a $\skeleton{n}$
			\item $X$ tem uma topologia fraca, no sentido de que, um dado $A \subseteq X$ é dito um aberto se, e somente se, $A \cap \skeleton{n}$ for um aberto em $\skeleton{n}$ para todo $n \in \mathbb{N}$
		\end{enumerate}
	\end{definicao}
	
	\vspace{2mm} Um espaço $X$ com uma CW-estrutura é chamado de CW-complexo e os subespaços $\skeleton{n}$ são chamado $n-$esqueleto do CW-complexo $X$. Um mapa de anexação $f_{\partial}:S^{n-1} \to \skeleton{n-1}$ extende a um mapa $f:D^{n} \to \skeleton{n}$ chamado mapa caraterístico. Chamaremos a imagem de $D^{n}$ pelo mapa $f$ por célula fechada em $X$, e a imagem de $D^{n} - \partial D^{n}$ por célula aberta em $X$.
	
	\subsection{Exemplos de CW-complexos}
	\begin{exemplo}
		(n-esféra) Vamos exibir uma estrutura de CW-complexo para $S^{n}$. Fixemos um ponto-base $p \in S^{n}$ e definamos o $0-$esqueleto $\skeleton{0}=\{p\}$. Anexando uma $n-$célula a $\skeleton{0}$ teremos $f_{\partial}: \partial D^{n} \to \skeleton{0}$, isto é, $S^{n} \approx \skeleton{n} = \{p\}\cup_{f_{\partial}} \celula{n}{}$.
	\end{exemplo}
	
	\begin{exemplo}
		(2-toro) Vamos exibir uma estrutura de CW-complexo para $T^{2}$. Identificando toro com o quadrado cujo os lados opostos serão identificados preservando a orientação, então todos os vertices do quadrado serão identificados com um único ponto um ponto $p \in T^{2}$. Definamos o $0-$esqueleto como sendo $\skeleton{0} = \{p\}$. Cada uma das faces horizontais representam o mesmo um $S^{1}$ no toro, isso é o mesmo que anexarmos uma 1-célula ao 0-esqueleto, ou seja, $\skeleton{0}\cup_{f_{1\partial}}\celula{1}{1}$. Analogamente, as faces verticais também representam o mesmo $S^{1}$ no toro, o que indica que devemos anexar uma outra 1-célula a espaço anexado anteriormente, isto é, $\skeleton{0}\cup_{f_{1\partial}}\celula{1}{1}\cup_{f_{2\partial}}\celula{1}{2}$. Por fim, temos que anexar um 2-célula para cobrir o interior do quadrado, então $T^{2} =\skeleton{2} = \skeleton{0}\cup_{f_{1\partial}}\celula{1}{1}\cup_{f_{2\partial}}\celula{1}{2}\cup_{f_{3\partial}}\celula{2}{3}$.
	\end{exemplo}
	
	\begin{exemplo}
		(n-plano projetivo) Vamos exibir uma estrutura de CW-complexo para $\realprojetivo{n}$. Para $n=0$ temos que $\realprojetivo{0} = \{\classe{p}\}$, para um determinado $p \in \real{}$. Já para $n=1$, sabemos que existe um homeomorfismo $\realprojetivo{1} \approx S^{1}_{\sim}$ onde $S^{1}_{\sim} = \{\classe{p}: p \in S^{1},\; p \sim -p\}$. Note esse espqço quociente já possui, naturalmente, uma CW-estrutura pois, na passagem ao quociente, identificamos todos os pontos do equador com um único ponto desse conjunto, e digamos que seja o ponto $p_{0} = (1,0)$, sem perda de generalidade. Isso quer dizer que ao anexarmos um 1-célula ao ponto $p_{0}$ teremos $S^{1}_{\sim} \approx \{[p_{0}]\} \cup_{f}D^{1} \approx \realprojetivo{0}\cup_{f}D^{1} $ pois $\realprojetivo{0} \approx \{[p_{0}]\}$, portanto $ \realprojetivo{1} \approx \realprojetivo{0}\cup_{f}D^{1}$. Repetindo o procedimento anterior para $\realprojetivo{n} \approx S^{n}_{\sim} $ onde $p_{0} = (1,0,\dots, 0)$, teremos $\realprojetivo{n} \approx \realprojetivo{n-1} \cup_{f_{\partial}}D^{n}$. Assim, temos a CW-estrutura $\realprojetivo{j-1} \subseteq \realprojetivo{j}$ e $\realprojetivo{j} = \skeleton{j} \approx \skeleton{j-1}\cup_{f_{j\partial}}D^{j}$ para $1\leq j \leq n$.
	\end{exemplo}
	
	\subsection{Homologia Celular}
	
	\begin{lema}\label{homologiacelular}
		(Homologia celular relativa) Sejam $\Lambda$ um anel comutativo com unidade e $X$ um CW-complexo, então
		$$
		\homologiarelskelesimpl{k}{n} \cong 
		\left\{
		\begin{array}{cc}
		\mathcal{C}_{n}(X), & k = n\\
		0, & k\neq n\\
		\end{array}
		\right.,
		$$
		onde $\mathcal{C}_{n}(X)$ é um $\Lambda-$módulo livre e finitamente gerado pelas $n-$células de $X$. Além disso,
		$$
		\somadir{\sigma} \homologiarelcel{n}{n}{\sigma} \cong \somadir{\sigma} \Lambda
		$$
		tal que 
		$$
		\somadir{\sigma}f_{\sigma*}: \somadir{\sigma} \homologiarelcel{n}{n}{\sigma} \to \homologiarelskelesimpl{n}{n}
		$$
		denota o isomorfismo descrito.
	\end{lema}
	\prova{Por definição, temos  $\skeleton{n} = \skeleton{n-1} \bigcup_{f_{\partial \sigma} } \celula{n}{\sigma}$ e também $\celula{n}{\sigma} \subset \skeleton{n}$, onde $f_{\sigma}:\celula{n}{\sigma} \to \skeleton{n}$ é a aplicação característica. Sejam $C_{\sigma}$ e $A_{\sigma}$ discos fechado e abertos, respectivamente, contendo o hemisfério norte de $f_{\sigma}(\celula{n}{\sigma})$. Definindo $N_{\sigma} = f_{\sigma}(\celula{n}{\sigma}) - C_{\sigma}$, $M_{\sigma} = f_{\sigma}(\celula{n}{\sigma}) - A_{\sigma}$ tal que $\overline{N_{\sigma}} \subset M_{\sigma}$, e considerando $U = \skeleton{n} - \bigcup C_{\sigma}$ $Y = \skeleton{n} - \bigcup A_{\sigma}$ temos $U \subseteq Y$. Note que $\skeleton{n-1}$ é um retrato de deformação de $Y$, logo, pela invariância homotópica temos $\homologiarel{k}{\skeleton{n}}{\skeleton{n-1}} \cong  \homologiarel{k}{\skeleton{n}}{Y}$. Como $\skeleton{n} - U = \bigcup C_{\sigma}$ e $Y - U $ é homotópico a $\bigcup S^{n}_{\sigma}$, pelo teorema da excisão $\homologiarel{k}{\skeleton{n} - U}{Y- U} \cong \homologiarel{k}{\skeleton{n}}{Y}$, portanto $\homologiarel{k}{\skeleton{n} - U}{Y- U} = \homologiarel{k}{\bigcup C_{\sigma}}{\bigcup S^{n}_{\sigma}} \cong \homologia{k}{\bigcup (\celula{n}{\sigma}, \celulabordo{n}{\sigma})} \cong \somadir{\sigma} \homologiarelcel{k}{n}{\sigma}$, pois $C_{\sigma} \approx \celula{n}{\sigma}$ e $S^{n}_{\sigma} = \celulabordo{n}{\sigma}$. Enfim, temos o diagrama:
		\[
		\xymatrix{
			\somadir{\sigma} \homologiarelcel{k}{n}{\sigma} \ar[r]^{id_{*}} \ar[d]^{\cong} & 
			\somadir{\sigma} \homologiarelcel{k}{n}{\sigma} \ar[r]^{id_{*}} \ar[d]^{\somadir{\sigma}f_{\sigma*}} & 
			\somadir{\sigma} \homologiarelcel{k}{n}{\sigma} \ar[d]^{\somadir{\sigma}f_{\sigma*}} 
			\\
			\homologiarel{k}{\skeleton{n} - U}{Y- U} \ar[r]^{\cong} & \homologiarel{k}{\skeleton{n}}{Y} \ar[r]^{\cong} & 
			\homologiarelskelesimpl{k}{n}.
		}
		\]
		Por fim, sabemos que $\homologiarelcel{k}{n}{\sigma} \cong \Lambda$ para $k=n$ e é trivial para $k\neq n$, então pela sequência anterior temos $\homologiarelskelesimpl{k}{n} \cong \somadir{\sigma}\homologiarelcel{k}{n}{\sigma} \cong \somadir{\sigma} \Lambda$ se $k=n$ e é trivial caso $k\neq n$, como desejávamos.}

	\begin{definicao}
		(Aplicação de Pares) Seja $X = \skeleton{n}$ um CW-complexo e tome $p \in \skeleton{n-1}$ como um ponto-base. Identificaremos um dado $q \in \skeleton{n}$ com o ponto-base se $q \in \skeleton{n-1}$, e diremos que $q \sim p$. Definimos $\skeleton{n}/\skeleton{n-1} = \{[q]: q \in \skeleton{n}, \; q \sim p\}$ e a aplicação quociente $\pi : \skeleton{n} \to \skeleton{n}/\skeleton{n-1}$ por:
		$$
		\pi(q) = 
		\left\{
		\begin{array}{cc}
		\classe{p}, & q \in \skeleton{n-1}\\
		\classe{q}, & q \notin \skeleton{n-1}\\
		\end{array}
		\right..
		$$
		Seja $\bigvee_{\sigma} S^{n}_{\sigma}$ o buquet de $n-$esféras com o ponto-base $p$, então $\skeleton{n}/\skeleton{n-1} \approx \bigvee_{\sigma} S^{n}_{\sigma}$. Agora, definindo $s_{\sigma} : \skeleton{n}/\skeleton{n-1} \to S^{n}_{\sigma}$ por 
		$$
		\funcaocond{s_{\sigma}([q])}{q}{q \in \celula{n}{\sigma}}{p}{q \notin \celula{n}{\sigma}}
		$$
		chamamos de $\sigma-$aplicação de pares $p_{\sigma} = s_{\sigma} \circ \pi : (\skeleton{n}, \skeleton{n-1}) \to (S^{n}_{\sigma}, \{p\})$.
	\end{definicao}
	
	\begin{definicao}
		Denotaremos $\Psi_{n}:\mathcal{C}_{k}(X) \to \homologiarelskelesimpl{k}{n}$ como sendo o isomorfismo anterior dado por 
		$$
		\Psi(\sum_{\sigma} n_{\sigma} \sigma) = \sum_{\sigma} n_{\sigma} f_{\sigma *}[D^{n}],
		$$
		onde $[D^{n}]$ é um gerador do grupo $\homologiarelcel{n}{n}{}$.
	\end{definicao}

	\begin{lema}
		(Inversa de $\Psi_{n}$) A aplicação inversa $\Phi_{n} : \homologiarelskelesimpl{n}{n} \to \mathcal{C}_{n}(X)$ do isomorfismo definido anteriormente é dada por
		$$
		\Phi_{n}(\alpha) = \sum_{\sigma} \phi_{n}(p_{\sigma *}\alpha)\sigma,
		$$
		onde $\phi_{n}: \homologiarel{n}{S^{n}}{\{p\}} \to \Lambda$ é o único homomorfismo tal que $\phi_{n}([S^{n}])=1$ e $[S^{n}]$ é a classe fundamental do par $(S^{n}, \{p\})$.
	\end{lema}
	\prova{Mostremos a unicidade do homomorfismo. Sabemos que o grupo $\homologiarel{n}{S^{n}}{\{p\}}$ tem como geradores as classes $\{[S^{n}], [0]\}$ e como $\phi_{n}$ é homomorfismo, então $\phi_{n}([0]) = 0$. Definindo $\phi_{n}([S^{n}]) = 1$ e supondo que exista outro homomorfismo tal que $\phi_{n}^{'} ([S^{n}]) = 1$, então ambos homomorfismos coincidem quando avaliados nos geradores, logo $\phi_{n}^{'}=\phi_{n}$ o que é uma contradição, portanto $\phi_{n}$ é único. Sabemos o que $\Psi_{n}$ é um isomorfismo, então existe uma única aplicação $\Phi_{n}$ tal que $\Phi_{n} \circ \Psi_{n} = 1$, com isso, tomemos $\sigma$ uma $n-$célula geradora de $\mathcal{C}_{n}(X)$, então
		$$
		\begin{aligned}
		\Phi_{n}(\Psi_{n}(\sigma)) 
		&= \sum_{\beta}\phi_{n}(p_{\beta *}\Psi_{n}(\sigma))\beta
		\\
		&= \sum_{\beta}\phi_{n}(p_{\beta *}f_{\partial\sigma *}[D^{n}])\beta
		\\
		&= \sum_{\beta}\phi_{n}((p_{\beta}\circ f_{\partial\sigma})_{*}[D^{n}])\beta
		\\
		&= \phi_{n}((p_{\sigma}\circ f_{\partial\sigma})_{*}[D^{n}])\sigma
		\\
		&= \phi_{n}([S^{n}])\sigma
		\\
		&= \sigma	
		\end{aligned},
		$$
		e como $\sigma \in \mathcal{C}_{n}(X)$ é arbitrário, então $\Phi_{n} \circ \Psi_{n} = 1$, como desejávamos.}
	
	\begin{definicao}
		(Grau homológico) Seja $f: S^{n} \to S^{n}$ uma aplicação contínua e $f_{*}: \homologia{n}{S^{n}} \to \homologia{n}{S^{n}}$ o homomorfismo induzindo. Seja $[S^{n}] \in \homologia{n}{S^{n}}$ o gerador não-trivial desse grupo, então $f_{*}[S^{n}] = k[S^{n}]$ para algum $k \in \Lambda$. Denominamos por grau da aplicação $deg(f) = k$, com $k$ definido anteriormente.
	\end{definicao}

	\begin{definicao}
		(Aplicação CW-bordo) Tome a tripla $(\skeleton{n}, \skeleton{n-1}, \skeleton{n-2})$ e a composição abaixo
		\[
		\xymatrix{
			\mathcal{C}_{n}(X) \ar[r]^{\Psi_{n}\qquad} &
			\homologiarelskelesimpl{n}{n} \ar[r]^{\delta_{*}} & 
			\homologiarelskele{n-1}{n-1}{n-2} \ar[r]^{\qquad \Phi_{n-1}}&
			\mathcal{C}_{n-1}(X)
		}
		\]
		onde $\delta_{*}$ é o homomorfismo de conexão. Denominamos por operador CW-bordo o homomorfismo $\partial_{n} = \Phi_{n-1} \circ \delta_{*} \circ \Psi_{n} : \mathcal{C}_{n}(X) \to \mathcal{C}_{n-1}(X)$.
	\end{definicao}
	
	\begin{teorema}
		Teorema (CW-bordo) A aplicação CW-bordo é um homomorfismo tal que $\partial_{n-1}\circ\partial_{n} = 0$ e é dado por:
		$$
		\partial_{n}(\sigma) = \sum_{\beta}[\beta:\sigma]\beta,
		$$
		onde $[\beta:\sigma]$ é o grau da aplicação $p_{\beta} \circ f_{\partial\sigma}:\celulabordo{n}{\sigma} \to S^{n-1}_{\sigma}$.
	\end{teorema}
	
	\prova{Por definição temos $\partial_{n} = \Phi_{n-1} \circ \delta_{*} \circ \Psi_{n}$, logo é um homomorfismo pois é a composição de homomorfismos.
	
	 Consideremos a sequência vertical exata longa do par $(\skeleton{n-1}, \skeleton{n-2})$
	\[
	\xymatrix{
		& \homologia{n-2}{\skeleton{n-2}}\ar[rd]^{j_{*}}
		\\
		\homologiarelskele{n}{n}{n-1} \ar[r]^{\delta_{*} \qquad}\ar[rd]_{\delta_{n}} &
		\homologiarelskele{n-1}{n-1}{n-2} \ar[u]^{\delta_{n-1}} \ar[r]^{ \delta_{*}}&
		\homologiarelskele{n-2}{n-2}{n-3}
		\\
		& \homologia{n-1}{\skeleton{n-1}}\ar[u]^{j_{*}}
	}
	\]	
	
	\vspace{2mm} Note que $\delta_{*} \circ \delta_{*} = j_{*} \circ \delta_{n-1} \circ j_{*} \circ \delta_{n}$. Pela exatidão da sequência vertical temos $Im(j_{*}) = Ker(\delta_{n-1})$, logo $\delta_{*}^{2}=0$. Com isso, temos o composição do CW-bordo 
	$\partial_{n-1}\circ \partial_{n} = \Phi_{n-2} \circ \delta_{*} \circ \Psi_{n-1} \circ \Phi_{n-1} \circ \delta_{*} \circ \Psi_{n} = \Phi_{n-2} \circ \delta_{*}^{2} \circ \Psi_{n} =0$, pois $\Psi_{n-1} \circ \Phi_{n-1}=1$.
	
	\vspace{2mm} Por definição temos $f_{\partial\sigma}: \celulabordo{n}{\sigma} \to \skeleton{n-1}$, assim temos o homomorfismo induzido $f_{\partial\sigma*}: \homologia{n-1}{\celulabordo{n}{\sigma} }\to \homologia{n-1}{\skeleton{n-1}}$. Analogamente, temos o homomorfismo $f_{\sigma*}:\homologiarelcel{n}{n}{\sigma} \to \homologiarelskelesimpl{n}{n}$ e o homomorfismo conectante $\delta_{n} : \homologiarelcel{n}{n}{\sigma} \to \homologia{n-1}{\celulabordo{n}{\sigma}}$ de modo que, tomanto $[\celula{n}{\sigma}] \in \homologiarelcel{n}{n}{\sigma}$ como um elemento gerador, então $\delta_{n}\circ f_{\sigma*}[\celula{n}{\sigma}] \in \homologia{n-1}{\skeleton{n-1}}$ é um elemento gerador, por outro lado $f_{\partial\sigma*}\circ \delta_{n}[\celula{n}{\sigma}] \in \homologia{n-1}{\skeleton{n-1}}$ também é um elemento gerador, logo $f_{\partial\sigma*}\circ \delta_{n}[\celula{n}{\sigma}] = \lambda \delta_{n}\circ f_{\sigma*}[\celula{n}{\sigma}]$, para algum $\lambda \in \Lambda$, mas sempre podemos escolher um mapa caracteristico $f_{\sigma *}$ tal que $\lambda = 1$, portanto $f_{\partial\sigma*}\circ \delta_{n} = \delta_{n}\circ f_{\partial\sigma*}$, e como $\delta_{*} = j_{*}\circ\delta_{n} \Rightarrow f_{\partial\sigma*}\circ \delta_{*} = \delta_{*}\circ f_{\partial\sigma*}$. Assim, temos o operador CW-bordo
	$$
	\begin{aligned}
	\partial_{n}(\sigma) &= \Phi_{n-1}\circ\delta_{*}\circ\Psi_{n}(\sigma)
	\\
	&= \Phi_{n-1}\circ\delta_{*}\circ f_{\sigma*}([\celula{n}{\sigma}])
	\\
	&= \Phi_{n-1}\circ f_{\partial\sigma*}\circ\delta_{*}([\celula{n}{\sigma}])
	\\
	&= \Phi_{n-1}\circ f_{\partial\sigma*}\circ (j_{*}\circ \delta_{n}) ([\celula{n}{\sigma}])
	\\
	&= \Phi_{n-1} \circ f_{\partial\sigma*}([\celulabordo{n}{\sigma}])
	\\
	&= \sum_{\beta} \phi_{n-1}(p_{\beta*}\circ f_{\partial\sigma*}[\celulabordo{n}{\sigma}])\beta
	\\
	&= \sum_{\beta} \phi_{n-1}((p_{\beta}\circ f_{\partial\sigma})_{*}[S^{n-1}])\beta
	\\
	&= \sum_{\beta} \phi_{n-1}(deg(p_{\beta}\circ f_{\partial\sigma})[S^{n-1}])\beta
	\\
	&= \sum_{\beta} deg(p_{\beta}\circ f_{\partial\sigma})\phi_{n-1}([S^{n-1}])\beta
	\\
	&= \sum_{\beta} deg(p_{\beta}\circ f_{\partial\sigma})\beta,
	\end{aligned}
	$$
	como desejávamos.}
	
	\begin{teorema}
		Teorema (CW-homologia) Existe uma identificação natural entre a homologia do complexo $\mathcal{C}_{*}(X)$ e a homologia singular $\homologia{*}{X}$, isto é 
		$$
		\homologia{k}{X} \cong \homologia{k}{\mathcal{C}_{*}(X)}\; \forall k \in \mathbb{Z}.
		$$
	\end{teorema}
	\begin{proof}
		Para a demonstração desse resultado vamos considerar a sequência 
		\[
		\xymatrix{
			\mathcal{C}_{k+1}(X) \ar[r]^{\partial_{k+1}} & \mathcal{C}_{k}(X) \ar[r]^{\partial_{k}} & \mathcal{C}_{k-1}(X) &
		}
		\]
		e por definição temos $\homologia{k}{\mathcal{C}_{*}(X)} = Ker(\partial_{k})/Im(\partial_{k+1})$, então provaremos que $Ker(\partial_{k})/Im(\partial_{k+1}) \cong \homologia{k}{X}$ para que concluir a equivalência entre a CW-homologia e a homologia singular do espaço $X$.
		
		\vspace{2mm} Tomemos a sequência longa exata vertical da tripla $(\skeleton{k+1}, \skeleton{k-1}, \skeleton{k-2})$ e a sequência longa exata horizontal da tripla $(\skeleton{k+1}, \skeleton{k}, \skeleton{k-1})$  no diagrama abaixo:
		\[
		\xymatrix{
			& & \homologiarelskele{k}{k-1}{k-2}= 0 \ar[d]^{i_{*}} &
			\\
			& & \homologiarelskele{k}{k+1}{k-2} \ar[d]^{j_{*}} &
			\\
			\homologiarelskele{k+1}{k+1}{k} \ar[r]^{\quad\delta_{1*}} &		\homologiarelskele{k}{k}{k-1} \ar[r]^{i_{*}} \ar[rd]^{\delta_{2*}} & \homologiarelskele{k}{k+1}{k-1} \ar[r] \ar[d]^{\delta_{3*}} & 0
			\\
			& & \homologiarelskele{k-1}{k-1}{k-2} &
		}
		\]	
		onde $i_{*}, \; j_{*}$ e $\delta_{*}$ são as inclusões induzidas e o homomorfismo conectante, respectivamente. Seja $\classe{\alpha} \in \homologiarelskelesimpl{k}{k}$, então $\delta_{3*}\circ i_{*}\classe{\alpha} = \delta_{2*}\classe{\alpha}$, pois $i_{*}$ é um monomorfismo e pela construção do homomorfismo conectante, caso contrário não conectaríamos as homologias, logo temos a comutatividade do triângulo do diagrama.
		
		\vspace{2mm} Vamos agora caracterizar $Ker(\delta_{2*})$. Dado $[\alpha] \in Ker(\delta_{2*})$, então $\delta_{3*}\circ i_{*}\classe{\alpha} = \delta_{2*}\classe{\alpha} = 0$. Como $i_{*}$ é um epimorfismo e $j_{*}$ é monomorfismo, então existe um único $\classe{\beta} \in \homologiarelskele{k}{k+1}{k-2}$ tal que $i_{*} \classe{\alpha} = j_{*} \classe{\beta}$. 
		
		\vspace{2mm} Afirmo que $\phi: Ker(\delta_{2*}) \to \homologiarelskele{k}{k+1}{k-2}$ dado por $\phi(\classe{\alpha}) = \classe{\beta}$ é um epimorfismo. Seja $\classe{\beta'} \in \homologiarelskele{k}{k+1}{k-2}$, então existe um $\classe{\alpha'} \in \homologiarelskele{k}{k}{k-1}$ tal que $i_{*} \classe{\alpha'} = j_{*} \classe{\beta'}$, com isso $\delta_{2*}\classe{\alpha'} = \delta_{3*}\circ i_{*}\classe{\alpha'} = \delta_{3*}\circ j_{*}\classe{\beta'} = 0$ pois $Im(j_{*}) = Ker(\delta_{3*})$, logo $\classe{\alpha'} \in Ker(\delta_{2*})$ e $\phi$ é epimorfismo.
		
		\vspace{2mm} Analogamente temos $Im(\delta_{1*}) = Ker(i_{*}) \Rightarrow i_{*}\circ\delta_{1*}=0$. Como $\phi$ é sobrejetor, existe $\classe{\alpha} \in Ker(\delta_{2*})$ tal que $\phi(\classe{\alpha}) = 0$, o que implica em $i_{*}{\classe{\alpha}} = j_{*}\classe{0} = 0$, pois $j_{*}$ é monomorfismo, portanto $\classe{\alpha} \in Ker(i_{*})$ e pela exatidão, $\classe{\alpha} \in Im(\delta_{1*})$. Com isso podemos concluir que $Ker(\phi) = Im(\delta_{1*})$, e pelo teorema fundamental do isomorfismo de grupos, temos que $Ker(\delta_{2*})/Ker(\phi) \cong \homologiarelskele{k}{k+1}{k-2}$, ou seja, $Ker(\delta_{2*})/Im(\delta_{1*}) \cong \homologiarelskele{k}{k+1}{k-2}$.
		
		\vspace{2mm} Sem perda de generalidade, vamos assumir que $X$ seja um CW-complexo de ordem $n$, isto é, $X= \skeleton{n}$, então temos $X= \skeleton{n} \supseteq \skeleton{n-1} \supseteq \dots \supseteq \skeleton{0} \supseteq \skeleton{-1}$, onde $\skeleton{-1} = \emptyset$. Com isso, podemos escrever a sequência de homomorfismos de inclusão de pares:
		\[
		\xymatrix{
			\homologia{k}{X} = \homologiarel{k}{\skeleton{n}}{\skeleton{-1}}\ar[r]& \homologiarel{k}{\skeleton{n}}{\skeleton{0}} \ar[r] & \dots \ar[r] & \homologiarel{k}{\skeleton{n}}{\skeleton{k-2}}
		}
		\]
		
		de modo que, para cada $i-$ésimo termo $\homologiarelskele{k}{n}{i}$ no centro do diagrama abaixo, teremos a sequência exata de triplas nas verticais $(\skeleton{n}, \skeleton{i}, \skeleton{i-1})$, com $0\leq i\leq k-2$ com $h_{i}$ sendo os homomorfismos de inclusão:
		\[
		\xymatrix{
			\homologiarel{k}{\skeleton{i-1}}{\skeleton{i-2}}=0 \ar[d] & \homologiarel{k}{\skeleton{i}}{\skeleton{i-1}}=0 \ar[d] & \homologiarel{k}{\skeleton{i+1}}{\skeleton{i}}=0 \ar[d] &	
			\\
			\homologiarel{k}{\skeleton{n}}{\skeleton{i-2}} \ar[d]^{h_{i-1}} & \homologiarel{k}{\skeleton{n}}{\skeleton{i-1}} \ar[d]^{h_{i}} & \homologiarel{k}{\skeleton{n}}{\skeleton{i}} \ar[d]^{h_{i+1}}
			\\
			\homologiarel{k}{\skeleton{n}}{\skeleton{i-1}} \ar[r]\ar[d]^{\delta_{(i-1)*}}& \homologiarel{k}{\skeleton{n}}{\skeleton{i}} \ar[r] \ar[d]^{\delta_{i*}} &  \homologiarel{k}{\skeleton{n}}{\skeleton{i+1}} \ar[d]^{\delta_{(i+1)*}} 
			\\
			\homologiarelskele{k-1}{i-1}{i-2}=0& \homologiarel{k-1}{\skeleton{i}}{\skeleton{i-1}}=0 &  \homologiarel{k}{\skeleton{i+1}}{\skeleton{i}}=0. &		
		}
		\]
		Do lema \ref{homologiacelular} temos que $\homologiarelskele{k}{i}{i-1} =0$ caso $k \neq i$, logo os grupos naz extemidades verticais do diagrama serão os triviais. Pela exatidão das sequências verticais temos $Im(h_{i}) = Ker(\delta_{i*})$, mas como $Im(\delta_{i*}) = 0 \Rightarrow Ker(\delta_{i*}) = \homologiarel{k}{\skeleton{n}}{\skeleton{i}}$, logo $h_{i}$ é um epimorfismo, portanto um isomorfismo para $0\leq i \leq k-2$, isto é, $\homologiarel{k}{\skeleton{n}}{\skeleton{i-1}} \cong \homologiarel{k}{\skeleton{n}}{\skeleton{i}}$, o que nos pertmite escrever a cadeia de isomorfismos 
		
		$$
		\begin{aligned}
		\homologia{k}{X} &= \homologiarel{k}{\skeleton{n}}{\skeleton{-1}} 
		\\
		&\cong  \homologiarel{k}{\skeleton{n}}{\skeleton{0}} \cong \dots \cong  \homologiarel{k}{\skeleton{n}}{\skeleton{i}} \cong  \dots \cong \homologiarel{k}{\skeleton{n}}{\skeleton{k-2}},
		\end{aligned}
		$$
		logo $\homologia{k}{X} \cong \homologiarel{k}{\skeleton{n}}{\skeleton{i}}$ para $0 \leq i \leq k-2$.
		\vspace{2mm} Por fim, podemos repetir a mesma construção anterior para $0 \leq j \leq n$ da sequência abaixo:
		\[
		\xymatrix{
			\homologiarel{k}{\skeleton{j}}{\skeleton{-1}}\ar[r]& \homologiarel{k}{\skeleton{j}}{\skeleton{0}} \ar[r] & \dots \ar[r] & \homologiarel{k}{\skeleton{j}}{\skeleton{k-2}}
		}
		\]
		e concluiremos $\homologiarel{k}{\skeleton{j}}{\skeleton{i-1}} \cong \homologiarel{k}{\skeleton{j}}{\skeleton{i}}$ onde $0 \leq i \leq k-2$.
		
		\vspace{2mm} Assim, $\homologia{k}{X} \cong \homologiarel{k}{\skeleton{k+1}}{\skeleton{k-2}} \cong Ker(\delta_{2*})/Im(\delta_{1*})$. Mas como $\partial_{k} = \Phi_{n-1}\circ\delta_{*}\circ\Psi_{n}$, então $Ker(\delta_{2*}) \cong Ker(\partial_{k})$ e $Im(\delta_{1*}) \cong Im(\partial_{k+1})$, logo $Ker(\delta_{2*})/Im(\delta_{1*}) \cong Ker(\partial_{k})/Im(\partial_{k+1}) = \homologia{k}{\mathcal{C}_{*}}$, logo $\homologia{k}{X} \cong \homologia{k}{\mathcal{C}_{*}}$ que é a equivalência entre as homologias, como desejávamos.
	\end{proof}
	
	\section{Exemplos}
	\tese{Homologia da n-esféra}{$\homologia{*}{S^{n}}$}{Vamos exibir uma estrutura de CW-complexo para $S^{n}$, para isso tomemos um ponto $p \in S^{n}$ e definindo o $0-$skeleton $\skeleton{0}=\{p\}$, agora anexando uma $n-$célula a $\skeleton{0}$ onde $f_{\partial}: \celulabordo{n}{} \to \skeleton{0}$, isto é, $\skeleton{n} = \{p\}\cup_{f_{\partial}} \celula{n}{}$. Pelo teorema da CW-homologia temos que $\homologia{k}{S^{n}} \cong \homologiarelskelesimpl{k}{k}$, de onde temos apenas $k \in \{0,n\}$, logo $\homologia{0}{S^{n}} \cong \homologiarelskelesimpl{0}{0} \cong \Lambda$, e analogamente, $\homologia{n}{S^{n}} \cong \homologiarelskelesimpl{n}{n} \cong \Lambda$ e $\homologia{k}{S^{n}} \cong \homologiarelskelesimpl{k}{j} =0$ caso $k \neq j$, como desejávamos.}
	
	\tese{Homologia do 2-toro}{$\homologia{*}{T^{2}}$}{
	Vamos exibir uma estrutura de CW-complexo para $T^{2}$, para isso tomemos a identificação do toro com o quadrado cujo os lados opostos serão identificados, assim os vertices do quadrado serão um ponto $p \in T^{2}$ e definindo o $0-$skeleton $\skeleton{0} = \{p\}$, agora vamos anexar às faces do quadrado duas $1-$células, isto é, $\skeleton{1} = \skeleton{0}\cup_{f_{1\partial}}\celula{1}{1}\cup_{f_{2\partial}}\celula{1}{2}$, e por fim, cobrir o centro do quandrado anexando um $2-$célula, com isso, $\skeleton{2} = \skeleton{1}\cup_{f_{3\partial}}\celula{2}{3}$, então
	$$
	T^{2} =\skeleton{2} = \skeleton{0}\cup_{f_{1\partial}}\celula{1}{1}\cup_{f_{2\partial}}\celula{1}{2}\cup_{f_{3\partial}}\celula{2}{3}.
	$$
	Teremos os grupos de homologia não-triviais:
	$$
	\begin{aligned}
	\homologia{0}{T^{2}} &\cong \homologiarelskele{0}{0}{-1} \cong \Lambda,
	\\
	\homologia{1}{T^{2}} &\cong \homologiarelskele{1}{1}{0} \cong \somadir{i=1,2}\homologiarelcel{1}{1}{i} \cong \somadir{i=1,2}\Lambda
	\\
	\homologia{2}{T^{2}} &\cong \homologiarelskele{2}{2}{1} \cong \Lambda.
	\end{aligned}
	$$
	}
	
	\tese{Homologia do n-plano projetivo real}{$\homologia{*}{\realprojetivo{n}}$}{
	Para $n=0$ temos que $\realprojetivo{0} = \{\classe{p}\}$, para um determinado $p \in \real{}$. Já para $n=1$, sabemos que existe um homeomorfismo $\realprojetivo{1} \approx S^{1}_{\sim}$ onde $S^{1}_{\sim} = \{\classe{p}: p \in S^{1},\; p \sim -p\}$, contudo, notemos que o conjunto $E^{1} = \{p \in S^{1}: p = (x, 0)\}$ serão levados em um único ponto realizarmos a identificação, isto é, $E^{1}_{\sim} = \{\classe{p_{0}}\}$, onde $p_{0} = (1,0)$, sem perda de generalidade. Por outro lado, essa identificação contém uma estrutura de CW-complexo, pois $S^{1}_{\sim}$ é o mesmo que anexar uma $1-$célula a um ponto, isto é, $S^{1}_{\sim} \approx \{p_{0}\} \cup_{f_{1}}D^{1}$, mas como $\realprojetivo{0} =\{\classe{p}\} \approx p_{0} \Rightarrow S^{1}_{\sim} \approx \realprojetivo{0} \cup_{f_{1\partial}}D^{1}$, logo $\realprojetivo{1} \approx \realprojetivo{0} \cup_{f_{1}}D^{1}$. Repetindo o procedimento anterior para $\realprojetivo{n} \approx S^{n}_{\sim} $, $E^{n} = \{p \in S^{n}: p = (x_{1}, \dots, x_{n}, 0)\}$ e $E^{n}_{\sim} = \{\classe{p_{0}}\}$, onde $p_{0} = (1,0,\dots, 0)$, teremos $\realprojetivo{n} \approx \realprojetivo{n-1} \cup_{f_{n\partial}}D^{n}$. Assim, temos a estrutura de CW-complexo $\realprojetivo{j-1} \subseteq \realprojetivo{j}$  e $\realprojetivo{j} = \skeleton{j} \approx \skeleton{j-1}\cup_{f_{j\partial}}D^{j}$ para $1\leq j \leq n$. Calculando a homologia:
	$$
	\begin{aligned}
	\homologia{0}{\realprojetivo{n}} &\cong \homologiarelskelesimpl{0}{0} \cong \Lambda
	\\
	\homologia{1}{\realprojetivo{n}} &\cong \homologiarelskelesimpl{1}{1} \cong \Lambda
	\\
	\vdots
	\\
	\homologia{n}{\realprojetivo{n}} &\cong \homologiarelskelesimpl{n}{n} \cong \Lambda.
	\end{aligned}
	$$
	}
	\chapter{Teoria de Morse}
	\section{Pontos Críticos}
	
	\tese{Proposição}{Seja M uma variedade fechada e $f$ uma função de Morse, então se $p \in Crit(f)$ teremos $\nabla f(p)=0$ e $\exists q \in M$ tal que $p \in \omega(q)$ ou $p \in \alpha(q)$ , isto é, $p$ é uma singularidade e é um ponto limite.}{Tomando $p \in Crit(f)$, então $\forall v \in T_{p}M \Rightarrow Df(p)(v) = \innerprod{\nabla f(p)}{v} = 0$, portanto, para $v = -\nabla f(p)$ teremos $-\innerprod{\nabla f(p)}{\nabla f(p)} = 0 \iff \nabla f(p) = 0$, logo, $p$ é uma singularidade do campo gradiente. Com isso, podemos afirmar que existe um ponto $q \in M$ tal que a órbita $\mathcal{O}(q)$ tenha como um dos pontos limites o dado $p \in Crit(f)$ pois caso contrário, o ponto $p$ não será uma singularidade. Como $p \in Crit(f)$ é um ponto limite, então $p \in \omega(q)$ ou $p \in \alpha(q)$, como desejávamos.}
	
	\tese{Lema}{Seja M uma variedade fechada, $f$ uma função de Morse e $X =-\nabla f$, então $\alpha(p), \omega(p)$ consistem de um único ponto crítico de $f$ para qualquer $p \in M$.}
	{Como $M$ é uma variedade fechada e $f$ é uma função de Morse, então $Crit(f) = \{p_{i} \in M: 1\leq i \leq k \}$ é um conjunto finito de pontos isolados. Pela proposição anterior, dado $p_{i} \in Crit(f), \exists q_{i} \in M$ tal que $p_{i} \in \omega(q_{i})$ ou $p_{i} \in \alpha(q_{i})$. Com isso, podemos afirmar que $Crit(f) \subseteq \bigcup_{i=1}^{k}\omega(q_{i}) \cup \alpha(q_{i})$. Sabemos que os conjuntos limite $\omega(q_{i})$ e $\alpha(q_{i})$ consistem de singularidades do campo gradiente $-\nabla f$, logo $\forall p \in \bigcup_{i=1}^{k}\omega(q_{i}) \cup \alpha(q_{i})$ teremos $\nabla f(p) = 0 \Rightarrow Df(p)(v) = \innerprod{\nabla f(p)}{v} = 0 \therefore p \in Crit(f)$, e como $p$ é arbitrário, então $\bigcup_{i=1}^{k}\omega(q_{i}) \cup \alpha(q_{i}) \subseteq Crit(f)$, logo $Crit(f) = \bigcup_{i=1}^{k}\omega(q_{i}) \cup \alpha(q_{i})$. Com esse resultado fica evidente que $\omega(q_{i}), \alpha(q_{i}) \subset Crit(f)$. Como os conjuntos limites são conjuntos finitos de pontos isolados vamos supor que $\omega(q_{i}) = \{r_{j} \in Crit(f): 0\leq j \leq m\}$, logo pela topologia induzinda, os abertos de cada um deles serão os conjuntos unitários e teremos $\omega(q_{i}) = \bigcup_{j=1}^{k} \{r_{j}\} $, e como os conjuntos limite são conexos (vide resultado do Palis), essa união disjunta contradiz a conexidade, portanto $\omega(q_{i}) = \{r_{i}\}$. Os mesmos argumentos valem para o conjunto $\alpha(q_{j})$. Portanto, ambos os conjuntos limite devem conter apenas um ponto crítico, como desejávamos.}

	\begin{definicao}
		(Funções de Morse) $\morsefunc{M} \subset \suavefunc{M}$ é o conjunto de todas as funções de Morse definidas na variedade $M$.
	\end{definicao}
	
	\begin{definicao}
		(Norma $C^{r}$) Sabemos o que espaço das fuções infinitamente diferenciáveis $\suavefunc{M}$ forma um espaço vetorial, além disso, como temos uma variedade compacta, dada uma cobertura, podemos escolher uma subconvertura finita por abertos $\{V_{i}\}_{i=1}^{k}$e um atlas finito $\{(U_{i}, \phi_{i})\}_{i=1}^{k}$ tal que $V_{i} \subset \phi_{i}(U_{i})$ com $\phi_{i}^{-1}(\phi_{i}(U_{i})) = B(0,2)$ e $\phi_{i}^{-1}(V_{i}) = B(0,1)$. Dado $f \in \suavefunc{M}$ definimos $f_{i} := f|_{\phi_{i}(U_{i})}$ e a \textit{r-norma} como sendo 
		$$
		||f||_{r} := \max_{i} \sup\{||f_{i}(p)||, ||D^{1}f_{i}(p)||, \dots, ||D^{r}f_{i}(p)||: p \in B(1)\}.
		$$
		Pode-se verificar que $||.||_{1}: \suavefunc{M} \to \real{}$ é uma norma, além disso, esse espaço é completo nessa norma, portanto é um \textit{espaço de Banach}.
	\end{definicao}
	
	\begin{definicao}
		Dados $f, g \in \suavefunc{M}$, dizemos que $f$ é uma $C^{1}$-aproximação de $g$ quando existe um $\epsilon \in \real{+}$ tal que $||f-g||_{1} < \epsilon$.
	\end{definicao}
	
	\begin{teorema}
		(Transversalidade de Morse-Smale) Seja M uma variedade fechada e $f \in \morsefunc{M}$, então existe uma $h \neq f \in \morsefunc{M}$ tal que $\nabla f$ pode ser $C^{1}$-aproximado por $\nabla h$ e que satisfaz as condições de Morse-Smale.
	\end{teorema}
	\begin{prova}
		Seja $p \in \pontocritico{f}$ e tomemos uma vizinhança aberta $U \ni p$ tal que esse ponto seja o único ponto crítico pertencente a ela. Como o espaço das funções de Morse $\morsefunc{M} \subset C^{\infty}(M, \real{})$ é um conjunto denso na topologia $C^{1}$, então dado $\epsilon > 0 \in \real{}$ existe uma $h \neq f \in \morsefunc{M}$ tal que $||f-h||_{1} < \epsilon$. Sejam $\gamma, \alpha : \real{} \to M$ as curvas integrais dos campos $-\nabla f$ e $-\nabla h$, respectivamente, de modo que $\gamma'(p) = -\nabla f(p), \alpha'(p) = -\nabla h(p)$ com $\gamma(p) = \alpha(p)$, então
	\end{prova}

	\section{Desigualdades de Morse}
	\tese{Teorema de Euler-Poincaré}{Seja $(C_{*}, \partial_{*})$ um complexo de cadeia finitamente gerado com $C_{k} = 0$ para $k>m>0$ e algum $m \in \mathbb{Z}$. Definindo $c_{k} = dim(C_{k})$ e $b_{k} = dim(\homologia{k}{C_{*}})$ para $0 \leq k \leq m$, então
	$$
	\sum_{k=0}^{m}(-1)^{k}c_{k} = \sum_{k=0}^{m}(-1)^{k}b_{k}.
	$$}{Temos que $Ker(\partial_{k}) \subseteq C_{k}(X)$ e $Im(\partial_{k}) \subseteq C_{k-1}(X)$ são submódulos, então podemos construir a sequência exata
	\[
	\xymatrix{
		0 \ar[r] & Ker(\partial_{k}) \ar[r]^{i_{*}} &  C_{k}(X)  \ar[r]^{\partial_{k}} & Im(\partial_{k}) \ar[r] & 0,
	}
	\]
	onde $i_{*}$ é a inclusão e $\partial_{k}$ é um epimorfismo, logo pelo teorema fundamental de isomorfismo de grupos, temos que $ C_{k}(X)/Ker(\partial_{k}) \cong Im(\partial_{k})$, portanto $dim(C_{k}(X)/Ker(\partial_{k})) = dim(C_{k}(X)) - dim(Ker(\partial_{k})) = dim(Im(\partial_{k})) \Rightarrow c_{k} = dim(Ker(\partial_{k})) + dim(Im(\partial_{k}))$. Análogamente, temos a sequência exata
	\[
	\xymatrix{
		0 \ar[r] & Im(\partial_{k+1}) \ar[r] &  Ker(\partial_{k})  \ar[r]^{j_{*}} & \homologia{k}{X}\ar[r] & 0,
	}
	\]
	onde $j_{*}$ é um epimorfismo pois dado $\classe{\alpha} \in \homologia{k}{X} = Ker(\partial_{k})/Im(\partial_{k+1})$ podemos tomar $\alpha \in Ker(\partial_{k})$ tal que $j_{*}(\alpha) = \classe{\alpha}$, consequentemente, pelo teorema fundamental do isomorfismo de grupos, temos que $Ker(\partial_{k})/Ker(j_{*}) \cong \homologia{k}{X}$, mas $Ker(j_{*}) = Im(\partial_{k+1})$, pela definição de homologia, então $dim(Ker(\partial_{k})/Ker(j_{*})) = dim(Ker(\partial_{k})) - dim(Ker(j_{*})) = dim(\homologia{k}{X}) \Rightarrow dim(Ker(\partial_{k})) = dim(Im(\partial_{k+1}))+ b_{k}$. Igualando as expressões para $dim(Ker(\partial_{k}))$:
	$$
	dim(Ker(\partial_{k})) = c_{k} - dim(Im(\partial_{k})) = dim(Im(\partial_{k+1}))+ b_{k} = dim(Ker(\partial_{k})).
	$$
	Como vale para $0 \leq k \leq m$, então:
	$$
	\begin{aligned}
		\sum_{k=0}^{m}	(-1)^{k}(c_{k} - dim(Im(\partial_{k}))) &= \sum_{k=0}^{m}	(-1)^{k}( dim(Im(\partial_{k+1}))+ b_{k})
		\\
		\sum_{k=0}^{m}	(-1)^{k}c_{k} - \sum_{k=0}^{m}(-1)^{k} b_{k} &= \sum_{k=0}^{m}	(-1)^{k} (dim(Im(\partial_{k+1})) -  dim(Im(\partial_{k})))
		\\
		&= 0,
	\end{aligned}
	$$
	onde teremos somas alternadas no ultimo termo que se cancelarão, restando apenas $dim(Im(\partial_{m+1})) - dim(Im(\partial_{0}))$, mas ambos são identicamente nulos, pois $\partial_{m+1} = 0$ e $\partial_{0} = 0$.
	}
	
\end{document}