\documentclass[12pt]{book}
\usepackage{graphicx}
\usepackage{indentfirst}
\usepackage[utf8]{inputenc}
\usepackage{amssymb}
\usepackage{enumitem}
\usepackage{color}
\usepackage[fleqn]{amsmath}
\usepackage[a4paper, margin=1.0in]{geometry}

\usepackage{amsthm, amssymb, amsfonts, amsmath}
\usepackage{graphicx}
\usepackage{tikz}
\usetikzlibrary{calc,shapes}
% \usepackage{enumitem}
\usepackage{mathtools}
\usepackage{mathrsfs}
\usepackage{tikz-cd}
\usepackage[all,cmtip]{xy}


\newcommand{\celula}[2]{D^{#1}_{#2}}
\newcommand{\celulafront}[2]{\partial D^{#1}_{#2}}
\newcommand{\cohomologia}[2]{H^{#1}(#2)}
\newcommand{\cohomologiadual}[2]{H^{#1}(#2)^{*}}
\newcommand{\cohomologiacompac}[2]{H^{#1}_{c}(#2)}
\newcommand{\cohomologiacompacdual}[2]{H^{#1}_{c}(#2)^{*}}
\newcommand{\funcaocond}[5]{
	#1 = 
	\left\{
	\begin{array}{cc}
	#2, & #3\\
	#4, & #5\\
	\end{array}
	\right.
	}
\newcommand{\skeleton}[1]{X^{(#1)}}
\newcommand{\homologia}[2]{H_{#1}(#2)}
\newcommand{\homologiarel}[3]{H_{#1}(#2,#3)}
\newcommand{\homologiarelcel}[3]{H_{#1}(D^{#2}_{#3}, \partial D^{#2}_{#3})}
\newcommand{\homologiarelskele}[3]{H_{#1}(X^{(#2)}, X^{(#3)})}
\newcommand{\homologiarelskelesimpl}[2]{H_{#1}(X^{(#2)}, X^{(#2-1)})}
\newcommand{\real}[1]{\mathbb{R}^{#1}}
\newcommand{\realprojetivo}[1]{\mathbb{R}P^{#1}}
\newcommand{\somadir}[1]{\bigoplus \limits_{#1}}
\newcommand{\definicao}[1]{\vspace{2mm} \textbf{Definição:}{ #1}}
\newcommand{\definicaonomeada}[2]{\vspace{2mm} \textbf{Definição (#1):}{ #2}}
\newcommand{\tese}[3]{\vspace{2mm} \textit{\textbf{#1}}: \textit{#2} \par $\square$ #3 \par $\blacksquare$}
\newcommand{\innerprod}[2]{\langle #1, #2 \rangle}
\newcommand{\morsefunc}[1]{\mathcal{M}o(#1)}
\newcommand{\pontocritico}[1]{\textit{Crit}(#1)}
\newcommand{\suavefunc}[1]{C^{\infty}(#1, \real{})}
\begin{document}
	
	\title{Homologia de Morse}
	
	\author{Vinicius Fernades}
	
	\maketitle
	
	\chapter{CW-Complexos}
	\section{CW-Homologia}
	\tese{Lema}{Seja $\Lambda$ um anel comutativo com unidade e $X$ um CW-complexo e $A \subseteq X$, então 
	$$
	\homologiarelskelesimpl{k}{n} \cong 
	\left\{
		\begin{array}{cc}
		C_{n}(X, A;\Lambda) , & k = n\\
		0, & k\neq n\\
		\end{array}
	\right.,
	$$
	onde 
	$$
	C_{n}(X, A;\Lambda) \cong \somadir{\sigma} \homologiarelcel{n}{n}{\sigma} \cong \somadir{\sigma} \Lambda
	$$
	é um $\Lambda-$módulo livre gerado por $n-$células de $X$ que não estão em $A$, além disso:
	$$
	\somadir{\sigma}f_{\sigma*}: \somadir{\sigma} \homologiarelcel{n}{n}{\sigma} \to \homologiarelskelesimpl{n}{n}
	$$
	é um isomorfismo.}{Por definição, temos  $\skeleton{n} = \skeleton{n-1} \bigcup_{f_{\partial \sigma} } \celula{n}{\sigma}$ e também $\celula{n}{\sigma} \subset \skeleton{n}$ pelo aplicação característica $f_{\sigma} : \celula{n}{\sigma} \to \skeleton{n}$, e sejam $C_{\sigma}$ e $A_{\sigma}$ discos fechado e abertos, respectivamente, do hemisfério norte de cada $f_{\sigma}(\celula{n}{\sigma})$, e definindo $N_{\sigma} = f_{\sigma}(\celula{n}{\sigma}) - C_{\sigma}$, $M_{\sigma} = f_{\sigma}(\celula{n}{\sigma}) - A_{\sigma}$ com $\overline{N_{\sigma}} \subset M_{\sigma}$. $U = \skeleton{n} - \bigcup C_{\sigma}$ e $Y = \skeleton{n} - \bigcup A_{\sigma}$, consequentemente, $U \subseteq Y$. Note que $\skeleton{n-1}$ é um retrato por deformação de $Y$, logo, pela invariância homotópica temos $\homologiarel{k}{\skeleton{n}}{\skeleton{n-1}} \cong  \homologiarel{k}{\skeleton{n}}{Y}$. Vejamos que $\skeleton{n} - U = \bigcup C_{\sigma}$ e que $Y - U $ é homotópico a $\bigcup S^{n}_{\sigma}$, assim, pelo teorema da excisão $\homologiarel{k}{\skeleton{n} - U}{Y- U} \cong \homologiarel{k}{\skeleton{n}}{Y}$, e pela homotopia citada anteriormente, $\homologiarel{k}{\skeleton{n} - U}{Y- U} = \homologiarel{k}{\bigcup C_{\sigma}}{\bigcup S^{n}_{\sigma}} \cong \homologia{k}{\bigcup (\celula{n}{\sigma}, \celulafront{n}{\sigma})} \cong \somadir{\sigma} \homologiarelcel{k}{n}{\sigma}$, pois $C_{\sigma} \simeq \celula{n}{\sigma}$ e $S^{n}_{\sigma} = \celulafront{n}{\sigma}$. Enfim, temos sequência:
	\[
	\xymatrix{
		\somadir{\sigma} \homologiarelcel{k}{n}{\sigma} \ar[r]^{id_{*}} \ar[d]^{\cong} & 
		\somadir{\sigma} \homologiarelcel{k}{n}{\sigma} \ar[r]^{id_{*}} \ar[d]^{\somadir{\sigma}f_{\sigma*}} & 
		\somadir{\sigma} \homologiarelcel{k}{n}{\sigma} \ar[d]^{\somadir{\sigma}f_{\sigma*}} 
		\\
		\homologiarel{k}{\skeleton{n} - U}{Y- U} \ar[r]^{\cong} & \homologiarel{k}{\skeleton{n}}{Y} \ar[r]^{\cong} & 
		\homologiarelskelesimpl{k}{n}.
	}
	\]
	Por fim, sabemos que $\homologiarelcel{k}{n}{\sigma} \cong \Lambda$ para $k=n$ e é trivial para $k\neq n$, então pela sequência anterior temos $\homologiarelskelesimpl{k}{n} \cong \somadir{\sigma}\homologiarelcel{k}{n}{\sigma} \cong \somadir{\sigma} \Lambda$ se $k=n$ e é trivial caso $k\neq n$, como desejávamos.}

	\definicaonomeada{aplicação de pares}{Seja $X = \skeleton{n}$ um CW-complexo e fixando um ponto $p \in \skeleton{n-1}$, realizamos o quociente $\pi : \skeleton{n} \to \skeleton{n}/\skeleton{n-1}$ tal que 	
	$$
		\pi(q) = 
		\left\{
		\begin{array}{cc}
		p, & q \in \skeleton{n-1}\\
		q, & q \notin \skeleton{n-1}\\
		\end{array}
		\right..
	$$
	Seja $\bigvee_{\sigma} S^{n}_{\sigma}$ o buquet de $n-$esfperas, então $\skeleton{n}/\skeleton{n-1} \simeq \bigvee_{\sigma} S^{n}_{\sigma}$. Agora, definindo $s_{\sigma} : \skeleton{n}/\skeleton{n-1} \to S^{n}_{\sigma}$ tal que 
	$$
	\funcaocond{s_{\sigma}(q)}{q}{q \in \celula{n}{\sigma}}{p}{q \notin \celula{n}{\sigma}}
	$$
	chamamos de $\sigma-$aplicação de pares $p_{\sigma} = s_{\sigma} \circ \pi : (\skeleton{n}, \skeleton{n-1}) \to (S^{n}_{\sigma}, \{p\})$.}
	
	\definicao{Denotaremos $\Psi_{n}:\homologiarelskelesimpl{k}{n} \to C_{k}(X,A)$ como o isomorfismo anterior dado por 
	$$
	\Psi(\sum_{\sigma} n_{\sigma} \sigma) = \sum_{\sigma} n_{\sigma} f_{\sigma *}[D^{n}],
	$$
	onde $[D^{n}]$ é um gerador do grupo $\homologiarelcel{n}{n}{}$.}

	\tese{Proposição (inversa do isomorfimo)}{A aplicação inversa $\Phi_{n} : \homologiarelskelesimpl{n}{n} \to C_{n}(X,A)$ do isomorfismo definido anteriormente é dada por
	$$
	\Phi_{n}(\alpha) = \sum_{\sigma} \phi_{n}(p_{\sigma *}\alpha)\sigma,
	$$
	onde $\phi_{n}: \homologiarel{n}{S^{n}}{\{p\}} \to \Lambda$ é o único homomorfismo tal que $\phi_{n}([S^{n}])=1$ e $[S^{n}]$ é a classe fundamental do par $(S^{n}, \{p\})$.}{O grupo $\homologiarel{n}{S^{n}}{\{p\}}$ tem como geradores as classes $\{[S^{n}], [0]\}$ e como $\phi_{n}$ é homomorfismo, então $\phi_{n}([0]) = 0$. Definindo $\phi_{n}([S^{n}]) = 1$ e supondo que exista outro homomorfismo tal que $\phi_{n}^{'} ([S^{n}]) = 1$, então ambos homomorfismos coincidem quando avaliados nos geradores, logo $\phi_{n}^{'}=\phi_{n}$ o que é uma contradição, portanto $\phi_{n}$ é único. Sabemos o que $\Psi_{n}$ é um isomorfismo, então existe uma única aplicação $\Phi_{n}$ tal que $\Phi_{n} \circ \Psi_{n} = 1$, com isso, tomemos $\sigma$ uma $n-$célula geradora de $C_{n}(X,A)$, então
	$$
	\begin{aligned}
	\Phi_{n}(\Psi_{n}(\sigma)) 
	&= \sum_{\beta}\phi_{n}(p_{\beta *}\Psi_{n}(\sigma))\beta
	\\
	&= \sum_{\beta}\phi_{n}(p_{\beta *}f_{\sigma *}[D^{n}])\beta
	\\
	&= \sum_{\beta}\phi_{n}((p_{\beta}\circ f_{\sigma})_{*}[D^{n}])\beta
	\\
	&= \phi_{n}((p_{\sigma}\circ f_{\sigma})_{*}[D^{n}])\sigma
	\\
	&= \phi_{n}([S^{n}])\sigma
	\\
	&= \sigma	
	\end{aligned},
	$$
	e como $\sigma \in C_{n}(X,A)$ é arbitrário, então $\Phi_{n} \circ \Psi_{n} = 1$, como desejávamos.
	}
	
	\definicaonomeada{operador CW-bordo}{Tome a composição abaixo
	\[
	\xymatrix{
		C_{n}(X,A) \ar[r]^{\Psi_{n}\qquad} &
		\homologiarelskelesimpl{n}{n} \ar[r]^{\delta_{*}} & 
		\homologiarelskele{n-1}{n-1}{n-2} \ar[r]^{\qquad \Phi_{n-1}}&
		C_{n-1}(X,A)
		}
	\]
	então $\partial_{n} = \Phi_{n-1} \circ \delta_{*} \circ \Psi_{n}$ tal que $\partial_{n}: C_{n}(X,A) \to C_{n-1}(X,A)$ é denomidado de operador CW-bordo.}

	\tese{Teorema (CW-bordo)}{O operador  CW-bordo é um homomorfismo, $\partial_{n-1}\circ\partial_{n} = 0$ e é dado por:
	$$
	\partial_{n}(\sigma) = \sum_{\beta}[\beta:\sigma]\beta,
	$$
	onde $[\beta:\sigma]$ é o grau da aplicação $p_{\beta} \circ f_{\partial\sigma}:\celulafront{n}{\sigma} \to S^{n-1}_{\sigma}$.}{Por definição temos $\partial_{n} = \Phi_{n-1} \circ \delta_{*} \circ \Psi_{n}$, e sejam $\alpha,\beta \in C_{n}(X,A; \Lambda)$ e $\lambda \in \Lambda$, então
	$$ 
	\begin{aligned}
	\partial_{n}(\lambda \alpha+\beta) &= (\Phi_{n-1} \circ \delta_{*} \circ \Psi_{n})(\lambda \alpha+\beta)
	\\
	&= (\Phi_{n-1} \circ \delta_{*}) (\lambda\Psi_{n}(\alpha) + \Psi_{n}(\beta))
	\\
	&= \Phi_{n-1}(\lambda\delta_{*}(\Psi_{n}(\alpha)) + \delta_{*}(\Psi_{n}(\beta)))
	\\
	&= \lambda \Phi_{n-1}(\delta_{*}(\Psi_{n}(\alpha))) + \Phi_{n-1}(\delta_{*}(\Psi_{n}(\beta)))
	\\
	&=\lambda (\Phi_{n-1}\circ\delta_{*} \circ\Psi_{n})(\alpha) + (\Phi_{n-1}\circ\delta_{*} \circ\Psi_{n})(\beta)
	\\
	&= \lambda\partial_{n}\alpha +\partial_{n}\beta
	\end{aligned},
	$$
	logo é um homomorfismo.
	
	 Temos a sequência exata longa de pares
	\[
	\xymatrix{
		\dots \ar[r]^{i_{*}} &
		\homologia{n-1}{\skeleton{n-1}}\ar[r]^{j_{*}\qquad} &
		\homologiarel{n-1}{\skeleton{n-1}}{\skeleton{n-2}} \ar[r]^{\qquad \delta_{n-1}}&
		\homologia{n-1}{\skeleton{n-2}} \ar[r] & \dots
	}
	\]	
	e como é uma sequência exata temos $Im(j_{*})=ker(\delta_{n-1})$ e $\delta_{*} = j_{*}\circ \delta_{n-1}$, logo dado $\sigma \in \homologiarelskelesimpl{n}{n}$ então $\delta_{*}^{2}\sigma = (j_{*}\circ \delta_{n-1})\circ(j_{*}\circ \delta_{n-1})\sigma = j_{*}( \delta_{n-1}\beta) = 0$, pois $\beta = j_{*}(\delta_{n-1}\sigma) \in Im(j_{*})$, portanto $\delta_{*}^{2} = 0$. Por definição, temos que 
	$$
	\begin{aligned}
		\partial_{n-1}\circ \partial_{n} 
		&= \Phi_{n-2} \circ \delta_{*} \circ \Psi_{n-1} \circ \Phi_{n-1} \circ \delta_{*} \circ \Psi_{n}
		\\
		&=\Phi_{n-2} \circ \delta_{*} \circ  \delta_{*} \circ \Psi_{n}
		\\
		&=0.
	\end{aligned}
	$$
	Por definição temos $f_{\partial\sigma}: \celulafront{n}{\sigma} \to \skeleton{n-1}$, assim temos o homomorfismo induzido $f_{\partial\sigma*}: \homologia{n-1}{\celulafront{n}{\sigma} }\to \homologia{n-1}{\skeleton{n-1}}$, analogamente, temos o homomorfismo $f_{\sigma*}:\homologiarelcel{n}{n}{\sigma} \to \homologiarelskelesimpl{n}{n}$ e o homomorfismo conectante $\delta_{n} : \homologiarelcel{n}{n}{\sigma} \to \homologia{n-1}{\celulafront{n}{\sigma}}$ de modo que, tomanto $[\celula{n}{\sigma}] \in \homologiarelcel{n}{n}{\sigma}$ como um elemento gerador, então $\delta_{n}\circ f_{\sigma*}[\celula{n}{\sigma}] \in \homologia{n-1}{\skeleton{n-1}}$ é um elemento gerador, por outro lado $f_{\partial\sigma*}\circ \delta_{n}[\celula{n}{\sigma}] \in \homologia{n-1}{\skeleton{n-1}}$ também é um elemento gerador, logo $f_{\partial\sigma*}\circ \delta_{n}[\celula{n}{\sigma}] = \lambda \delta_{n}\circ f_{\sigma*}[\celula{n}{\sigma}]$, para algum $\lambda \in \Lambda$, mas como os geradores são únicos, então $\lambda = 1 \in \Lambda$, portanto $f_{\partial\sigma*}\circ \delta_{n} = \delta_{n}\circ f_{\partial\sigma*}$, e como $\delta_{*} = j_{*}\circ\delta_{n} \Rightarrow f_{\partial\sigma*}\circ \delta_{*} = \delta_{*}\circ f_{\partial\sigma*}$. Assim, temos o operador CW-bordo
	$$
	\begin{aligned}
	\partial_{n}(\sigma) &= \Phi_{n-1}\circ\delta_{*}\circ\Psi_{n}(\sigma)
	\\
	&= \Phi_{n-1}\circ\delta_{*}\circ f_{\sigma*}([\celula{n}{\sigma}])
	\\
	&= \Phi_{n-1}\circ f_{\sigma*}\circ\delta_{*}([\celula{n}{\sigma}])
	\\
	&= \Phi_{n-1}\circ f_{\sigma*}\circ (j_{*}\circ \delta_{n}) ([\celula{n}{\sigma}])
	\\
	&= \Phi_{n-1} \circ f_{\sigma*}([\celulafront{n}{\sigma}])
	\\
	&= \sum_{\beta} \phi_{n-1}(p_{\beta*}\circ f_{\sigma*}[\celulafront{n}{\sigma}])\beta
	\\
	&= \sum_{\beta} \phi_{n-1}((p_{\beta}\circ f_{\sigma})_{*}[\celulafront{n}{\sigma}])\beta
	\\
	&= \sum_{\beta} \phi_{n-1}(deg(p_{\beta}\circ f_{\sigma})[S^{n-1}])\beta
	\\
	&= \sum_{\beta} deg(p_{\beta}\circ f_{\sigma})\phi_{n-1}([S^{n-1}])\beta
	\\
	&= \sum_{\beta} deg(p_{\beta}\circ f_{\sigma})\beta,
	\end{aligned}
	$$
	como desejávamos.}
	
	\tese{Teorema (CW-Homologia)}{Existe uma identificação natural entre a homologia do complexo $C_{*}(X,A; \Lambda)$ e a homologia singular $\homologiarel{*}{X}{A}$, isto é, $\homologia{k}{C_{*}(X,A; \Lambda)} \cong \homologiarel{k}{X}{A}\; \forall k \in \mathbb{Z}$}{arg3}
	\section{Exemplos}
	\tese{Proposição}{Seja $n \in \mathbb{N}$ e $\realprojetivo{n}$ o n-plano projetivo, então existe um aplicação de anexamento $f: \partial{D^{n}} \to \realprojetivo{n-1}$ tal que $\realprojetivo{n} \approx \realprojetivo{n-1} \cup_{f} D^{n}$}{Podemos definir $S^{n-1} =\{\frac{p}{||p||}: p  \in \real{n}\backslash\{0\}\}$, e tomando $[q] \in \realprojetivo{n-1}$ podemos escrever $q \in \real{n}\backslash\{0\} \Rightarrow q/||q|| = \lambda q \therefore q/||q|| \in [q]$, logo temos a aplicação  $f:S^{n-1} \to \realprojetivo{n-1}; \; f(p/||p||) = [p]$ que é contínua, portanto temos a célula anexada ao (n-1)-plano projetivo $\realprojetivo{n-1}\cup_{g}D^{n}$. Por definição $S^{n} = \{p \in \real{n+1}: ||p||^{2} = x_{1}^{2}+x_{2}^{2}+...+x_{n+1}^{2}=1\}$, assim, realizando o quociente $S^{n}/\sim = \{p \in S^{n}: p \sim -p\}$ e seja $A^{n} = \{[p] \in S^{n}/\sim:  p = (x_{1},x_{2},...,x_{n}, 0)\}$, isto é, $||p||^{2} = x_{1}^{2}+x_{2}^{2}+...+x_{n+1}^{2}=1$ o que implica em $[p] \in S^{n}/\sim$, e como temos $\realprojetivo{n-1} \approx S^{n}/\sim $ então podemos definir a identificação $g:A^{n} \to \realprojetivo{n-1}$, que por abuso diremos $[p] \longmapsto [p] \in \realprojetivo{n-1}$. essa identificação é injetora pois dados $[p] \neq [q] \in A^{n}$ e supondo que $g([p]) = g([q])$ implica que $[p] = [q] \in \realprojetivo{n-1}$, desse modo teremos $p = \lambda q$ com $\lambda \in \real{}\backslash\{0\}$, isto é, $[p]=[q] \in A^{n}$, contradizendo a hipótese de que $[p]\neq[q] \in A^{n}$. Por construção, a sobrejeção é imediata, portanto é uma bijeção. Além disso, ambas $g$ e $g^{-1}$ são contínuas, logo temos um homeomorfismo. Seja $B^{n} = \{[p] \in S^{n}/\sim:  p = (x_{1},x_{2},..., x_{n+1}), x^{n+1} \neq 0\}$, logo dado $[p] \in B^{n}$ teremos $x_{1}^{2}+x_{2}^{2}+...+x_{n}^{2}+x_{n+1}^{2}=1 \Rightarrow x_{1}^{2}+x_{2}^{2}+...+x_{n}^{2} < 1$, isto é, $p \in D^{n}$, definindo assim a $h : B^{n} \to D^{n}$ tal que $h([p]) = (|x_{1}|,\dots, |x_{n}|)$, esta bem-definida, assim teremos que $Im(h) = \{(x_{1},\dots, x_{n}) \in D^{n}: 0 \leq x_{j} < 1 \} \subset D^{1}$, que é homeomorfo ao inteior de $D^{n}$, isto é, $Im(h) \approx Int(D^{n})$. }
	

	
	
	\chapter{Teoria de Morse}
	\section{Pontos Críticos}
	
	\tese{Proposição}{Seja M uma variedade fechada e $f$ uma função de Morse, então se $p \in Crit(f)$ teremos $\nabla f(p)=0$ e $\exists q \in M$ tal que $p \in \omega(q)$ ou $p \in \alpha(q)$ , isto é, $p$ é uma singularidade e é um ponto limite.}{Tomando $p \in Crit(f)$, então $\forall v \in T_{p}M \Rightarrow Df(p)(v) = \innerprod{\nabla f(p)}{v} = 0$, portanto, para $v = -\nabla f(p)$ teremos $-\innerprod{\nabla f(p)}{\nabla f(p)} = 0 \iff \nabla f(p) = 0$, logo, $p$ é uma singularidade do campo gradiente. Com isso, podemos afirmar que existe um ponto $q \in M$ tal que a órbita $\mathcal{O}(q)$ tenha como um dos pontos limites o dado $p \in Crit(f)$ pois caso contrário, o ponto $p$ não será uma singularidade. Como $p \in Crit(f)$ é um ponto limite, então $p \in \omega(q)$ ou $p \in \alpha(q)$, como desejávamos.}
	
	\tese{Lema}{Seja M uma variedade fechada, $f$ uma função de Morse e $X =-\nabla f$, então $\alpha(p), \omega(p)$ consistem de um único ponto crítico de $f$ para qualquer $p \in M$.}
	{Como $M$ é uma variedade fechada e $f$ é uma função de Morse, então $Crit(f) = \{p_{i} \in M: 1\leq i \leq k \}$ é um conjunto finito de pontos isolados. Pela proposição anterior, dado $p_{i} \in Crit(f), \exists q_{i} \in M$ tal que $p_{i} \in \omega(q_{i})$ ou $p_{i} \in \alpha(q_{i})$. Com isso, podemos afirmar que $Crit(f) \subseteq \bigcup_{i=1}^{k}\omega(q_{i}) \cup \alpha(q_{i})$. Sabemos que os conjuntos limite $\omega(q_{i})$ e $\alpha(q_{i})$ consistem de singularidades do campo gradiente $-\nabla f$, logo $\forall p \in \bigcup_{i=1}^{k}\omega(q_{i}) \cup \alpha(q_{i})$ teremos $\nabla f(p) = 0 \Rightarrow Df(p)(v) = \innerprod{\nabla f(p)}{v} = 0 \therefore p \in Crit(f)$, e como $p$ é arbitrário, então $\bigcup_{i=1}^{k}\omega(q_{i}) \cup \alpha(q_{i}) \subseteq Crit(f)$, logo $Crit(f) = \bigcup_{i=1}^{k}\omega(q_{i}) \cup \alpha(q_{i})$. Com esse resultado fica evidente que $\omega(q_{i}), \alpha(q_{i}) \subset Crit(f)$. Como os conjuntos limites são conjuntos finitos de pontos isolados vamos supor que $\omega(q_{i}) = \{r_{j} \in Crit(f): 0\leq j \leq m\}$, logo pela topologia induzinda, os abertos de cada um deles serão os conjuntos unitários e teremos $\omega(q_{i}) = \bigcup_{j=1}^{k} \{r_{j}\} $, e como os conjuntos limite são conexos (vide resultado do Palis), essa união disjunta contradiz a conexidade, portanto $\omega(q_{i}) = \{r_{i}\}$. Os mesmos argumentos valem para o conjunto $\alpha(q_{j})$. Portanto, ambos os conjuntos limite devem conter apenas um ponto crítico, como desejávamos.}
	
	\definicaonomeada{funções de Morse}{$\morsefunc{M} \subset \suavefunc{M}$ é o conjunto de todas as funções de Morse definidas na variedade $M$.}
	
	\definicaonomeada{norma $C^{r}$}{Sabemos o que espaço das fuções infinitamente diferenciáveis $\suavefunc{M}$ forma um espaço vetorial, além disso, como temos uma variedade compacta, dada uma cobertura, podemos escolher uma subconvertura finita por abertos $\{V_{i}\}_{i=1}^{k}$e um atlas finito $\{(U_{i}, \phi_{i})\}_{i=1}^{k}$ tal que $V_{i} \subset \phi_{i}(U_{i})$ com $\phi_{i}^{-1}(\phi_{i}(U_{i})) = B(0,2)$ e $\phi_{i}^{-1}(V_{i}) = B(0,1)$. Dado $f \in \suavefunc{M}$ definimos $f_{i} := f|_{\phi_{i}(U_{i})}$ e a \textit{r-norma} como sendo 
	$$
	||f||_{r} := \max_{i} \sup\{||f_{i}(p)||, ||D^{1}f_{i}(p)||, \dots, ||D^{r}f_{i}(p)||: p \in B(1)\}.
	$$
	Pode-se verificar que $||.||_{1}: \suavefunc{M} \to \real{}$ é uma norma, além disso, esse espaço é completo nessa norma, portanto é um \textit{espaço de Banach}.}

	\definicao{Dados $f, g \in \suavefunc{M}$, dizemos que $f$ é uma $C^{1}$-aproximação de $g$ quando existe um $\epsilon \in \real{+}$ tal que $||f-g||_{1} < \epsilon$.}
	
	\tese{Teorema (transversalidade de Morse-Smale)}{Seja M uma variedade fechada e $f \in \morsefunc{M}$, então existe uma $h \neq f \in \morsefunc{M}$ tal que $\nabla f$ pode ser $C^{1}$-aproximado por $\nabla h$ e que satisfaz as condições de Morse-Smale.}{
		
	Seja $p \in \pontocritico{f}$ e tomemos uma vizinhança aberta $U \ni p$ tal que esse ponto seja o único ponto crítico pertencente a ela. Como o espaço das funções de Morse $\morsefunc{M} \subset C^{\infty}(M, \real{})$ é um conjunto denso na topologia $C^{1}$, então dado $\epsilon > 0 \in \real{}$ existe uma $h \neq f \in \morsefunc{M}$ tal que $||f-h||_{1} < \epsilon$. Sejam $\gamma, \alpha : \real{} \to M$ as curvas integrais dos campos $-\nabla f$ e $-\nabla h$, respectivamente, de modo que $\gamma'(p) = -\nabla f(p), \alpha'(p) = -\nabla h(p)$ com $\gamma(p) = \alpha(p)$, então }
\end{document}