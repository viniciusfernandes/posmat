\documentclass{article}
\usepackage{graphicx}
\usepackage{indentfirst}
\usepackage[utf8]{inputenc}
\usepackage{amssymb}
\usepackage{enumitem}
\usepackage{color}
\usepackage[fleqn]{amsmath}
\usepackage[a4paper, margin=0.5in]{geometry}
\begin{document}
	
	\title{Lista 6 para Entregar}
	\author{Vinicius Fernandes}
	
	\maketitle
	
	\begin{enumerate}
		
		\item \textbf{Solução:} 
			\begin{enumerate}
				\item Sabemos que $\int_{\mathbb{R}} e^{-x^{2}} = \sqrt{\pi}$, então definindo $\phi_{n}(x) = \frac{ne^{-(nx)^{2}}}{\sqrt{\pi}}$:
				$$
				\int_{\mathbb{R}}\phi_{n}(x)dx= 				\int_{\mathbb{R}} \frac{ne^{-(nx)^{2}}}{\sqrt{\pi}} dx = \int_{\mathbb{R}} \frac{ne^{-y^{2}}}{\sqrt{\pi}} \frac{1}{n} dy = \frac{1}{\sqrt{\pi}} \int_{\mathbb{R}} e^{-y^{2}} dy = \frac{1}{\sqrt{\pi}}  \sqrt{\pi} = 1,
				$$
				portanto $\int_{\mathbb{R}}\phi_{n}(x) = 1$ para qualquer $n \in \mathbb{N}$.
				
				\item Por definição, se $f \in \mathcal{S}^{'}(\mathbb{R})$ e $(f_{n}) \subset \mathcal{S}^{'}(\mathbb{R})$ é uma sequência de sitribuições, então dizemos que $f_{n} \rightharpoonup$ f se $\lim_{n \to \infty} \int_{\mathbb{R}}f_{n}(x)g(x) = \int_{\mathbb{R}}f(x)g(x)$ para qualquer $g \in \mathcal{S}(\mathbb{R})$. Lembremos que por definição temos a distribuição delta $\delta_{x_{0}} \in \mathcal{S}^{'}(\mathbb{R})$ tal que $\int_{\mathbb{R}}\delta_{x_0}(x)g(x) = g(x_{0})$, e no caso em que $g(x) = 1$ isso implica que $\int_{\mathbb{R}}\delta_{x_0}(x)g(x) = \int_{\mathbb{R}}\delta_{x_0}(x) = 1, \forall x_{0} \in \mathbb{R}$.
				
				
				Vamos construir duas sequências $(f_{n}), (g_{n}) \in L^{1}$ tal que $g_{n}(x) \leq \phi_{n}(x) \leq f_{n}(x), \forall n \in \mathbb{N}$ de modo que
				$$
				\underbrace{ \lim_{n \to \infty} \int_{\mathbb{R}} g_{n}(x)\varphi(x) }_{= \varphi(0)}\leq \lim_{n \to \infty} \int_{\mathbb{R}} \phi_{n}(x)\varphi(x) \leq \underbrace{ \lim_{n \to \infty} \int_{\mathbb{R}} f_{n}(x)\varphi(x) }_{= \varphi(0)}.
				$$
				Pois bem, definindo
				$$
				f_{n}(x) = 
				\left\{
				\begin{array}{cc}
				\phi_{n}(0) &, |x| \leq 1/n \\
				\phi_{n}(x) &, |x| > 1/n \\
				\end{array}
				\right.
				,
				g_{n}(x) = 
				\left\{
				\begin{array}{cc}
				\phi_{n}(1/n) &, |x| \leq 1/n \\
				\phi_{n}(x) &, |x| > 1/n \\
				\end{array}
				\right.
				.
				$$
				Notemos agora que se $|x| > 1/n$ então seque da definição da sequência de funções que $g_{n}(x) = \phi_{n}(x) = f_{n}(x)$, o que nos permite escrever $g_{n}(x) \leq \phi_{n}(x) \leq f_{n}(x)$. Por outro lado, se $|x| \leq 1/n$ teremos $\phi_{n}(x) \leq f_{n}(0) = \phi_{n}(0)$ pois $x =0 $ é um ponto de máximo de $\phi_{n}$, além disso, $\phi_{n}$ é monótona crescente para $x \in [-1/n, 0]$, portanto $\phi_{n}(1/n) \leq \phi_{n}(x)$, e como ela atinge o máximo em $x=0$ e é monótona decrescente para $x \in [0, 1/n]$, então $\phi_{n}(1/n) \leq \phi_{n}(x)$, portanto vale $\phi_{n}(1/n) \leq \phi_{n}(x)$ para $|x| \leq -1/n$, consequentemente temos $g_{n}(x) = \phi_{n}(1/n) \leq \phi_{n}(x) \leq \phi_{n}(0) = f_{n}(x), \forall |x| \leq 1/n$. Conclusão, $g_{n}(x) \leq \phi_{n}(x) \leq f_{n}(x), \forall n \in \mathbb{N}$ e $x \in \mathbb{R}$.
				
				Definindo $I_n = \{|x| \leq 1: x \in \mathbb{R} \}$ e $J_n = \{|x| > 1: x \in \mathbb{R} \}$, assim podemos escrever $\mathbb{R} = I_{n} \cup J_{n}$, e vejamos agora o resultado da estimativa:
				$$
				\begin{aligned}
				\int_{\mathbb{R}} g_{n}(x)\varphi(x) 
				\leq & \int_{\mathbb{R}} \phi_{n}(x)\varphi(x) \leq \int_{\mathbb{R}} f_{n}(x)\varphi(x) \\
				\int_{I_{n}} + \int_{J_{n}} g_{n}(x)\varphi(x) 
				\leq & \int_{I_{n}} + \int_{J_{n}} \phi_{n}(x)\varphi(x) \leq \int_{I_{n}} + \int_{J_{n}} f_{n}(x)\varphi(x),\\
				\int_{I_{n}} g_{n}(x)\varphi(x) + \int_{J_{n}} \phi_{n}(x)\varphi(x) 
				\leq & \int_{I_{n}} \phi_{n}(x)\varphi(x) + \int_{J_{n}} \phi_{n}(x)\varphi(x) \leq \int_{I_{n}} f_{n}(x)\varphi(x) + \int_{J_{n}} \phi_{n}(x)\varphi(x),\\
				\int_{I_{n}} g_{n}(x)\varphi(x) 
				\leq & \int_{I_{n}} \phi_{n}(x)\varphi(x) \leq \int_{I_{n}} f_{n}(x)\varphi(x), \\
				\end{aligned}
				$$
				note que as integrais no intervalo $J_{n}$ são identicas em todos os lados da desigualdade, por isso foram eliminadas da estimativa. 
				
				Sabemos que $\phi$ é infinitamente diferenciável na reta, então é contínua. Como o intervalo $I_{n}$ é um compacto, então $\phi$ restrita a esse intervalo possui um mínimo é um máximo, sendo assim, definimos $a_{n}, b_{n}$ como sendo esses pontos de mínimo e máximo de $\phi$, respectivamente. Com isso, temos duas sequências $(a_{n}), (b_{n}) \subset I_{n}$ tal que $a_{n} \to 0$ e $b_{n} \to 0$ pois isso decorre do fato de que $I_{n} \to \{0\}$. Agora vamos estudar o comportamento do limite das extremidades da estimativa.
				$$
				\begin{aligned}
				\varphi(a_{n}) \int_{I_{n}} f_{n}(x) 
				\leq & \int_{I_{n}} f_{n}(x)\varphi(x) \leq \varphi(b_{n}) \int_{I_{n}} f_{n}(x), \\
				\lim_{n \to \infty} \varphi(a_{n}) \int_{I_{n}} f_{n}(x) 
				\leq & \lim_{n \to \infty}\int_{I_{n}} f_{n}(x)\varphi(x) \leq  \lim_{n \to \infty} \varphi(b_{n}) \int_{I_{n}} f_{n}(x) \\
				\lim_{n \to \infty} \varphi(a_{n}) \int_{I_{n}} \phi_{n}(0)
				\leq & \lim_{n \to \infty}\int_{I_{n}} f_{n}(x)\varphi(x) \leq  \lim_{n \to \infty} \varphi(b_{n}) \int_{I_{n}} \phi_{n}(0), \\
				\lim_{n \to \infty} \varphi(a_{n}) \phi_{n}(0) \frac{2}{n}
				\leq & \lim_{n \to \infty}\int_{I_{n}} f_{n}(x)\varphi(x) \leq  \lim_{n \to \infty} \varphi(b_{n}) \phi_{n}(0) \frac{2}{n}, \\
				\lim_{n \to \infty} \varphi(a_{n}) \frac{n}{\sqrt{\pi}} \frac{2}{n}
				\leq & \lim_{n \to \infty}\int_{I_{n}} f_{n}(x)\varphi(x) \leq  \lim_{n \to \infty} \varphi(b_{n}) \frac{n}{\sqrt{\pi}} \frac{2}{n}, \\
				\frac{2}{\sqrt{\pi}} \varphi(0)
				\leq & \lim_{n \to \infty}\int_{I_{n}} f_{n}(x)\varphi(x) \leq  \frac{2}{\sqrt{\pi}} \varphi(0), \\
				\therefore & \lim_{n \to \infty}\int_{I_{n}} f_{n}(x)\varphi(x) = \frac{2}{\sqrt{\pi}} \varphi(0).
				\end{aligned}
				$$
				Analogamente para a outra extremidade:
				$$
				\begin{aligned}
				\varphi(a_{n}) \int_{I_{n}} g_{n}(x) 
				\leq & \int_{I_{n}} g_{n}(x)\varphi(x) \leq \varphi(b_{n}) \int_{I_{n}} g_{n}(x), \\
				\lim_{n \to \infty} \varphi(a_{n}) \int_{I_{n}} g_{n}(x) 
				\leq & \lim_{n \to \infty}\int_{I_{n}} g_{n}(x)\varphi(x) \leq  \lim_{n \to \infty} \varphi(b_{n}) \int_{I_{n}} g_{n}(x) \\
				\lim_{n \to \infty} \varphi(a_{n}) \int_{I_{n}} \phi_{n}(1/n)
				\leq & \lim_{n \to \infty}\int_{I_{n}} g_{n}(x)\varphi(x) \leq  \lim_{n \to \infty} \varphi(b_{n}) \int_{I_{n}} \phi_{n}(1/n), \\
				\lim_{n \to \infty} \varphi(a_{n}) \phi_{n}(1/n) \frac{2}{n}
				\leq & \lim_{n \to \infty}\int_{I_{n}} g_{n}(x)\varphi(x) \leq  \lim_{n \to \infty} \varphi(b_{n}) \phi_{n}(1/n) \frac{2}{n}, \\
				\lim_{n \to \infty} \varphi(a_{n}) \frac{ne^{-1}}{\sqrt{\pi}} \frac{2}{n}
				\leq & \lim_{n \to \infty}\int_{I_{n}} g_{n}(x)\varphi(x) \leq  \lim_{n \to \infty} \varphi(b_{n}) \frac{ne^{-1}}{\sqrt{\pi}} \frac{2}{n}, \\
				\frac{2e^{-1}}{\sqrt{\pi}} \varphi(0)
				\leq & \lim_{n \to \infty}\int_{I_{n}} g_{n}(x)\varphi(x) \leq  \frac{2e^{-1}}{\sqrt{\pi}} \varphi(0), \\
				\therefore & \lim_{n \to \infty}\int_{I_{n}} g_{n}(x)\varphi(x) = \frac{2e^{-1}}{\sqrt{\pi}} \varphi(0).
				\end{aligned}
				$$
				
			\end{enumerate}	
		
		\item \textbf{Solução:}
			\begin{enumerate}
				\item Efetuando a integração:
				$$
				\begin{aligned}
				\int_{\mathbb{R}}\phi_{n}(x) 
				= & \int_{|x| < 1/n}\phi_{n}(x) + \int_{|x| \geq 1/n}\phi_{n}(x) \\
				= & \int_{|x| < 1/n}\frac{n}{2} = \frac{n}{2} \Big|^{1/n}_{-1/n} = \frac{n}{2} (1/n + 1/n) = 1\\
				\int_{\mathbb{R}}\psi_{n}(x)
				= & \int_{\mathbb{R}}\frac{n}{\pi} \frac{1}{1+ (nx)^{2}} \\
				= & \int_{\mathbb{R}}\frac{n}{\pi} \frac{dx}{1+ (nx)^{2}} = \int_{\mathbb{R}}\frac{1}{\pi} \frac{dy}{1+ y^{2}} \\
				= & \frac{1}{\pi} \arctan(x) \Big|^{\infty}_{-\infty} = \frac{1}{\pi} (\frac{\pi}{2}+ \frac{\pi}{2}) = 1.
				\end{aligned}
				$$
				
					
				\item Seja $g \in \mathcal{S}(\mathbb{R})$, e desse modo 
				podemos afirmar que $g$ é contínua pois é infinitamente diferenciável na reta. Como o intervalo $I_{n} = \{|x| \leq 1/n: x \in \mathbb{R} \}$ é um compacto, então $\phi$ restrita a esse intervalo possui um mínimo é um máximo, sendo assim, definimos $a_{n}, b_{n}$ como sendo esses pontos de mínimo e máximo de $\phi$, respectivamente. Com isso, temos duas sequências $(a_{n}), (b_{n}) \subset \mathbb{R}$ tal que $a_{n} \to 0$ e $b_{n} \to 0$ pois isso decorre do fato de que $I_{n} \to \{0\}$. Agora vamos estudar o comportamento do limite das extremidades da estimativa. Definindo $J_{n} = \mathbb{R}\backslash I_{n}$ podemos escrever $\mathbb{R} = I_{n} \cup J_{n}$, então teremos a desigualdade $g(a_{n}) \leq g(x) \leq g(b_{n}), \forall n \in \mathbb{N} \; e \; x \in I_{n}$, além disso, podemos reduzir a seguinte integração $\int_{\mathbb{R}} \phi_{n}(x)g(x) = \int_{I_{n}} \phi_{n}(x)g(x)$ pois $\phi_{n}(x)$ quando $x \in J_{n}$, portanto:
				$$
				\int_{I_{n}} \phi_{n}(x) g(x) \leq \int_{I_{n}} \phi_{n}(x) g(b_{n}) \; \text{e} \;
				\int_{I_{n}} \phi_{n}(x) g(a_{n}) \leq \int_{I_{n}} \phi_{n}(x)g(x),
				$$
				logo
				$$
				\int_{I_{n}} \phi_{n}(x) g(a_{n}) 
				\leq \int_{I_{n}} \phi_{n}(x)g(x) \leq \int_{I_{n}} \phi_{n}(x) g(b_{n}),
				$$
				tomando o limite de ambos os lados da desigualdade
				$$
				\lim_{n\to \infty} g(a_{n}) \underbrace{ \int_{I_{n}} \phi_{n}(x) }_{=1}\leq \lim_{n\to \infty} \int_{I_{n}}\phi_{n}(x)g(x) \leq  \lim_{n\to \infty}  g(b_{n}) \underbrace{ \int_{I_{n}} \phi_{n}(x) }_{=1},
				$$
				$$
				\lim_{n\to \infty} g(a_{n}) \leq \lim_{n\to \infty} \int_{I_{n}}\phi_{n}(x)g(x) \leq  \lim_{n\to \infty}  g(b_{n}) ,
				$$
				$$
				\underbrace{ g(\lim_{n\to \infty} a_{n}}_{continuidade}) \leq \lim_{n\to \infty} \int_{I_{n}}\phi_{n}(x)g(x) \leq \underbrace{ g(\lim_{n\to \infty} b_{n}) }_{continuidade} ,
				$$
				$$
				g(0) \leq \lim_{n\to \infty} \int_{I_{n}}\phi_{n}(x)g(x) \leq  g(0) ,
				$$
				portanto
				$$
				 \lim_{n\to \infty} \int_{\mathbb{R}} \phi_{n}(x)g(x) = \lim_{n\to \infty} \int_{I_{n}}\phi_{n}(x)g(x) = g(0) \; \text{e} \; \int_{I_{n}}\delta(x)g(x) = g(0),
				$$
				o que implica que $\phi_{n} \rightharpoonup \delta$.
				
				Adotando novamente as definições acima para a função $g \in \mathcal{S}(\mathbb{R})$, os intervalos $I_{n}, J_{n}$ e as sequências $(a_{n}), (b_{n})$ como sendo os pontos de mínimo e máximo de $g$ restrita a $I_{n}$, respectivamente. Definindo agora as funções
				$$
				f_{n}(x) = 
				\left\{
				\begin{array}{cc}
				2\psi_{n}(0) &, x \in I_{n}\\
				\psi_{n}(x) &, x \notin I_{n} \\
				\end{array}
				\right.
				,
				g_{n}(x) = 
				\left\{
				\begin{array}{cc}
				\psi_{n}(1/n) &, x \in I_{n} \\
				\psi_{n}(x) &, x \notin I_{n}\\
				\end{array}
				\right.
				.
				$$
				Podemos ver que $\forall x \notin I_{n}$ então $g_{n}(x) = \psi_{n}(x) = f_{n}(x)$, logo podemos escrever $g_{n}(x) \leq  \psi_{n}(x) \leq f_{n}(x)$. Analogamente, se $x \in I_{n}$ temos que $\phi_{n}(x) \leq \phi_{n}(0) \leq 2\phi_{n}(0) = f_{n}(x)$ pois $x=0$ é um ponto de máximo de $\phi_{n}$, e também temos $g_{n}(x) = \phi_{n}(1/n) \leq \phi_{n}(x)$,  pois para $x \in [-1/n, 0]$ é monótona crescente, atinge o máximo na origem e para $x \in [0, 1/n]$ é monótona decrescente, portanto vale a desigualdade $g_{n}(x) \leq \psi_{n}(x) \leq f_{n}(x), \forall x \in I_{n}$. Podemos escrever a estimativa:
				$$
				\begin{aligned}
				\int_{\mathbb{R}} g_{n}(x)g(x) 
				\leq & \int_{\mathbb{R}} \phi_{n}(x)g(x) \leq \int_{\mathbb{R}} f_{n}(x)g(x), \\
				\int_{I_{n}} + \int_{J_{n}}g_{n}(x)g(x) 
				\leq & \int_{\mathbb{R}} \phi_{n}(x)g(x) \leq \int_{I_{n}} + \int_{J_{n}} f_{n}(x)g(x). \\
				\end{aligned}
				$$
				Vamos analisar o comportamento das extremidades da desigualdade anterior. Primeiramente, vamos definir $K = \overline{supp(g) \cap J_{n}} \subseteq supp(g)$ que é um compacto pois é subconjunto fechado de um compacto. Agora notemos que:
				$$
				\begin{aligned}
				\int_{J_{n}} f_{n}(x)g(x) = & \int_{J_{n}} \phi_{n}(x) g(x)=  \int_{K} \phi_{n}(x) \underbrace{g(x)}_{limitada} \\
				\leq & ||g||_{\infty} \int_{K} \phi_{n}(x) = ||g||_{\infty} \int_{[1/n, x]} \frac{ndy}{\pi(1+(ny)^{2})} \\
				= & ||g||_{\infty} \int_{[1, nx]} \frac{dz}{\pi(1+z^{2})} = \frac{||g||_{\infty} }{\pi} \arctan(z)\Big|^{nx}_{1} \\
				= & \frac{||g||_{\infty} }{\pi}
				\end{aligned}
				$$
			\end{enumerate}
	\end{enumerate}
	
	
\end{document}