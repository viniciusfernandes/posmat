\documentclass{article}
\usepackage{graphicx}
\usepackage{indentfirst}
\usepackage[utf8]{inputenc}
\usepackage{amssymb}
\usepackage{enumitem}
\usepackage{color}
\usepackage[fleqn]{amsmath}
\usepackage[a4paper, margin=0.5in]{geometry}
\begin{document}
	
	\title{Lista 6 para Entregar}
	\author{Vinicius Fernandes}
	
	\maketitle
	
	\begin{enumerate}
		\item[FATO-1:] 	Sabemos que toda $\varphi \in \mathcal{S}(\mathbb{R})$ pode ser escrita como $\varphi(x) = \varphi(0) + x \psi(x)$, onde $\psi \in C^{\infty}$ e que por definição $x^{\alpha}D^{\beta}\varphi \in \mathcal{S}(\mathbb{R})$, portanto $\lim_{x \to \infty} x^{\alpha}D^{\beta}\varphi(x) = 0$ (pois essas derivadas devem se anular fora de um conjunto limitado), em outras palavras, existe $x_{0} \in supp(\varphi)$ e $r >0$ tal que $supp(\varphi) \subset B(x_{0}, r)$, e assim, para qualquer $y \in \mathbb{R} \backslash B(x_{0}, r)$ teremos $x^{\alpha}D^{\beta}\varphi(x) = 0$. Aplicando essa afirmação na seguinte relação $D^{2}\varphi(x) = 2D^{1}\psi(x) + x D^{2}\psi(x)$  para $y \in \mathbb{R} \backslash B(x_{0}, r)$ teremos $D^{2}\varphi(y)=0 = 2D^{1}\psi(y) + y D^{2}\psi(y)$ e redefinindo $\lambda(y) = D^{1}\psi(y)$ chegamos na equação $0 = 2\lambda(y)+y\lambda'(y)$, ou ainda $\lambda'(y)/\lambda(y) = -2/y$, que integrando ambos os lados teremos $\lambda(y) = 1/y^{2}$. Portanto, $\lim_{y \to \infty} |D^{1}\psi(y)| = \lim_{y \to \infty} |\lambda(y)| = 0$ e podemos afirmar que $D^{1}\psi$ é limitada na reta.
		
		\item[FATO-2] Por definição, se $f \in \mathcal{S}^{'}(\mathbb{R})$ e $(f_{n}) \subset \mathcal{S}^{'}(\mathbb{R})$ é uma sequência de sitribuições, então dizemos que $f_{n} \rightharpoonup$ f se $\lim_{n \to \infty} \int_{\mathbb{R}}f_{n}(x)g(x) = \int_{\mathbb{R}}f(x)g(x)$ para qualquer $g \in \mathcal{S}(\mathbb{R})$. Lembremos que por definição temos a distribuição delta $\delta_{x_{0}} \in \mathcal{S}^{'}(\mathbb{R})$ tal que $\int_{\mathbb{R}}\delta_{x_0}(x)g(x) = g(x_{0})$, e no caso em que $g(x) = 1$ isso implica que $\int_{\mathbb{R}}\delta_{x_0}(x)g(x) = \int_{\mathbb{R}}\delta_{x_0}(x) = 1, \forall x_{0} \in \mathbb{R}$.
			
		\item \textbf{Solução:} 
			\begin{enumerate}
				\item Sabemos que $\int_{\mathbb{R}} e^{-x^{2}} = \sqrt{\pi}$, então definindo $\phi_{n}(x) = \frac{ne^{-(nx)^{2}}}{\sqrt{\pi}}$:
				$$
				\int_{\mathbb{R}}\phi_{n}(x)dx= 				\int_{\mathbb{R}} \frac{ne^{-(nx)^{2}}}{\sqrt{\pi}} dx = \int_{\mathbb{R}} \frac{ne^{-y^{2}}}{\sqrt{\pi}} \frac{1}{n} dy = \frac{1}{\sqrt{\pi}} \int_{\mathbb{R}} e^{-y^{2}} dy = \frac{1}{\sqrt{\pi}}  \sqrt{\pi} = 1,
				$$
				portanto $\int_{\mathbb{R}}\phi_{n}(x) = 1$ para qualquer $n \in \mathbb{N}$.
				
				Utiliazando o FATO-1, escreveremos $\varphi$ do modo mencionado e realizaremos a seguinte integração:
				$$
				\begin{aligned}
				\int_{\mathbb{R}} \phi_{n}(x)\varphi(x) 
				= &\int_{\mathbb{R}} \phi_{n}(x)(\varphi(0) + x\psi(x)) \\
				= & \varphi(0) \underbrace{\int_{\mathbb{R}} \frac{ne^{-(nx)^{2}}}{\sqrt{\pi}} }_{=1} + \int_{\mathbb{R}} \underbrace{ \frac{ne^{-(nx)^{2}}}{\sqrt{\pi}} x\psi(x) }_{I_{n}(x)},
				\end{aligned}
				$$
				que integrando por partes, teremos:
				$$
				\Big(\frac{ne^{-(nx)^{2}}}{\sqrt{\pi}} \psi(x)\Big)' = \frac{-2n^{3}e^{-(nx)^{2}}}{\sqrt{\pi}} x\psi(x) + \frac{ne^{-(nx)^{2}}}{\sqrt{\pi}} \psi'(x), 
				$$
				logo
				$$
				I_{n}(x) = \frac{-1}{2n^{2}}	\Big( \Big( \frac{ne^{-(nx)^{2}}}{\sqrt{\pi}} \psi(x)\Big)' - \frac{ne^{-(nx)^{2}}}{\sqrt{\pi}} \psi'(x) \Big), 
				$$
				que integrando ambos os lados da igualdade:
				$$
				\begin{aligned}
				\int_{\mathbb{R}} I_{n}(x) 
				= & \frac{-1}{2n^{2}}	\Big(  \underbrace{ \frac{ne^{-(nx)^{2}}}{\sqrt{\pi}} \psi(x) }_{\to 0} \Big|^{\infty}_{-\infty} - \int_{\mathbb{R}} \frac{ne^{-(nx)^{2}}}{\sqrt{\pi}} \psi'(x) \Big) \\
				= & \frac{-1}{2n^{2}}	\Big(  0 - \int_{\mathbb{R}} \frac{ne^{-(nx)^{2}}}{\sqrt{\pi}} \underbrace{ \psi'(x) }_{limitada} \Big) \\
				\leq & \frac{1}{2n^{2}} ||\psi'||_{\infty} \underbrace{ \int_{\mathbb{R}} \frac{ne^{-(nx)^{2}}}{\sqrt{\pi}} }_{=1} \\
				= & \frac{1}{2n^{2}} ||\psi'||_{\infty}, \\
				\therefore & \lim_{n \to \infty} \int_{\mathbb{R}} I_{n}(x) = \lim_{n \to \infty}  \frac{1}{2n^{2}} ||\psi'||_{\infty} = 0.
				\end{aligned}
				$$
				e tomando o limite na integral inicial
				$$
				\lim_{n \to \infty} \int_{\mathbb{R}} \phi_{n}(x)\varphi(x) 
				= \varphi(0) + \underbrace{ \lim_{n \to \infty} \int_{\mathbb{R}} I_{n}(x) }_{=0}= \varphi(0),
				$$
				e pelo FATO-2 temos $\int_{\mathbb{R}} \delta(x)\varphi(x) = \varphi(0)$, isso implica que $\phi_{n} \rightharpoonup \delta$, como desejávamos.
				
			\end{enumerate}	
		
		\item \textbf{Solução:}
			\begin{enumerate}
				\item Efetuando a integração:
				$$
				\begin{aligned}
				\int_{\mathbb{R}}\phi_{n}(x) 
				= & \int_{|x| < 1/n}\phi_{n}(x) + \int_{|x| \geq 1/n}\phi_{n}(x) \\
				= & \int_{|x| < 1/n}\frac{n}{2} = \frac{n}{2} \Big|^{1/n}_{-1/n} = \frac{n}{2} (1/n + 1/n) = 1\\
				\int_{\mathbb{R}}\psi_{n}(x)
				= & \int_{\mathbb{R}}\frac{n}{\pi} \frac{1}{1+ (nx)^{2}} \\
				= & \int_{\mathbb{R}}\frac{n}{\pi} \frac{dx}{1+ (nx)^{2}} = \int_{\mathbb{R}}\frac{1}{\pi} \frac{dy}{1+ y^{2}} \\
				= & \frac{1}{\pi} \arctan(x) \Big|^{\infty}_{-\infty} = \frac{1}{\pi} (\frac{\pi}{2}+ \frac{\pi}{2}) = 1.
				\end{aligned}
				$$
				
					
				\item Seja $g \in \mathcal{S}(\mathbb{R})$, e desse modo 
				podemos afirmar que $g$ é contínua pois é infinitamente diferenciável na reta. Como o intervalo $I_{n} = \{|x| \leq 1/n: x \in \mathbb{R} \}$ é um compacto, então $\phi$ restrita a esse intervalo possui um mínimo e um máximo, sendo assim, definimos $a_{n}, b_{n}$ como sendo esses pontos de mínimo e máximo de $\phi$, respectivamente. Com isso, temos duas sequências $(a_{n}), (b_{n}) \subset \mathbb{R}$ tal que $a_{n} \to 0$ e $b_{n} \to 0$ pois isso decorre do fato de que $I_{n} \to \{0\}$. Agora vamos estudar o comportamento do limite das extremidades da estimativa. Definindo $J_{n} = \mathbb{R}\backslash I_{n}$ podemos escrever $\mathbb{R} = I_{n} \cup J_{n}$, então teremos a desigualdade $g(a_{n}) \leq g(x) \leq g(b_{n}), \forall n \in \mathbb{N} \; e \; x \in I_{n}$, além disso, podemos reduzir a seguinte integração $\int_{\mathbb{R}} \phi_{n}(x)g(x) = \int_{I_{n}} \phi_{n}(x)g(x)$ pois $\phi_{n}(x)$ quando $x \in J_{n}$, portanto:
				$$
				\int_{I_{n}} \phi_{n}(x) g(x) \leq \int_{I_{n}} \phi_{n}(x) g(b_{n}) \; \text{e} \;
				\int_{I_{n}} \phi_{n}(x) g(a_{n}) \leq \int_{I_{n}} \phi_{n}(x)g(x),
				$$
				logo
				$$
				\int_{I_{n}} \phi_{n}(x) g(a_{n}) 
				\leq \int_{I_{n}} \phi_{n}(x)g(x) \leq \int_{I_{n}} \phi_{n}(x) g(b_{n}),
				$$
				tomando o limite de ambos os lados da desigualdade
				$$
				\lim_{n\to \infty} g(a_{n}) \underbrace{ \int_{I_{n}} \phi_{n}(x) }_{=1}\leq \lim_{n\to \infty} \int_{I_{n}}\phi_{n}(x)g(x) \leq  \lim_{n\to \infty}  g(b_{n}) \underbrace{ \int_{I_{n}} \phi_{n}(x) }_{=1},
				$$
				$$
				\lim_{n\to \infty} g(a_{n}) \leq \lim_{n\to \infty} \int_{I_{n}}\phi_{n}(x)g(x) \leq  \lim_{n\to \infty}  g(b_{n}) ,
				$$
				$$
				\underbrace{ g(\lim_{n\to \infty} a_{n}}_{continuidade}) \leq \lim_{n\to \infty} \int_{I_{n}}\phi_{n}(x)g(x) \leq \underbrace{ g(\lim_{n\to \infty} b_{n}) }_{continuidade} ,
				$$
				$$
				g(0) \leq \lim_{n\to \infty} \int_{I_{n}}\phi_{n}(x)g(x) \leq  g(0) ,
				$$
				portanto
				$$
				 \lim_{n\to \infty} \int_{\mathbb{R}} \phi_{n}(x)g(x) = \lim_{n\to \infty} \int_{I_{n}}\phi_{n}(x)g(x) = g(0),
				$$
				e pelo FATO-2 temos $\int_{I_{n}}\delta(x)g(x) = g(0)$ o que implica que $\phi_{n} \rightharpoonup \delta$.
				
				
				
				Seja $\varphi \in \mathcal{S}(\mathbb{R})$ e a escrevamos de acordo com as condições do FATO-1, então $\varphi(x) = \varphi(0) + x\psi(x)$. Realizando a seguinte integração onde definiremos $K = supp(\varphi)$:
				$$
				\begin{aligned}
				\int_{\mathbb{R}} \psi_{n}(x)\varphi(x) 
				= &\int_{\mathbb{R}} \psi_{n}(x)(\varphi(0) + x\psi(x)) \\
				= &\int_{K} \psi_{n}(x)(\varphi(0) + x\psi(x)) \\
				= & \varphi(0) \underbrace{ \int_{K} \psi_{n}(x) } _{=1}  +  \int_{\mathbb{R}} \psi_{n}(x) x\psi(x) \\
				= & \varphi(0)  +  \int_{\mathbb{R}} \frac{n}{\pi(1+(nx)^{2})} \underbrace{ x\psi(x) }_{supp \; compact} \\
				= & \varphi(0)  +  \int_{supp(x\psi)} \underbrace{ \frac{y}{n\pi(1+y^{2})} \psi(y/n). }_{I_{n}(y)}
				\end{aligned}
				$$
				Realizando uma integração por partes:
				$$
				\Big( \frac{1}{n\pi}\arctan(y^{2}) \psi(y/n) \Big)' = \underbrace{ \frac{2y}{n\pi(1+y^{2})} \psi(y/n) }_{2I_{n}(y)} + \frac{1}{n^{2}\pi}\arctan(y^{2}) \psi'(x),
				$$
				logo
				$$
				2I_{n}(y) = \Big( \frac{1}{n\pi}\arctan(y^{2}) \psi(y/n) \Big)' - \frac{1}{n^{2}\pi}\arctan(y^{2}) \psi'(x)
				$$
				e integrando ambos os lados da igualdade
				$$
				\begin{aligned}
				\int_{\mathbb{R}} 2I_{n}(y) = & \int_{\mathbb{R}} \Big( \frac{1}{n\pi}\arctan(y^{2}) \psi(y/n) \Big)' - \int_{\mathbb{R}} \frac{1}{n^{2}\pi}\arctan(y^{2}) \psi'(x) \\
				= &  \frac{1}{n\pi} \underbrace{ \arctan(y^{2}) \psi(y/n) \Big|^{\infty}_{-\infty} }_{A < \infty}- \int_{\mathbb{R}} \frac{1}{n^{2}\pi}\arctan(y^{2}) \underbrace{ \psi'(x) }_{FATO-1}\\
				\leq &  \frac{A}{n\pi} - \int_{\mathbb{R}} \frac{1}{n^{2}\pi}\arctan(y^{2}) ||\psi'||_{\infty} \\
				= &  \frac{A}{n\pi} - ||\psi'||_{\infty} \int_{\mathbb{R}} \frac{1}{n^{2}\pi}\arctan(y^{2}) \\
				\end{aligned}
				$$
				
			\end{enumerate}
	\end{enumerate}
	
	
\end{document}