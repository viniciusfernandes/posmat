\documentclass{article}
\usepackage{graphicx}
\usepackage{indentfirst}
\usepackage[utf8]{inputenc}
\usepackage{amssymb}
\usepackage{enumitem}
\usepackage{color}
\usepackage[fleqn]{amsmath}
\usepackage[a4paper, margin=0.5in]{geometry}
\begin{document}
	
	\title{Lista 5 para Entregar}
	\author{Vinicius Fernandes}
	
	\maketitle
	
	\begin{enumerate}
		
		\item \textbf{Solução:} Vamos omitir o aberto $\Omega$ das notações, por simplicidade. Definamos $\psi(x) := \frac{\phi(x) - \phi(0)}{x}$ e assim $\psi:\Omega\backslash\{0\} \to \mathbb{R}$ esta bem-definida. Contudo, tomando o limite $\lim_{x \to 0} \psi(x) = \lim_{x \to 0} \frac{\phi(x) - \phi(0)}{x} = \phi'(0)$ que existe pois $\phi$ é infinitamente diferenciável em $\Omega$, portanto $\psi(0) = \phi'(0)$, consequentemente $\psi$ pode ser avaliada em $x=0$ e $\psi:\Omega \to \mathbb{R}$ esta bem-definida. 
		
		Pela própria definição temos que $\psi$ é o produto de $f(x)$ e $1/x$ que são funções infinitamente diferenciáveis em $\Omega \backslash\{0\}$, portanto $\psi$ é infinitamente diferenciável em $\Omega \backslash\{0\}$. Vejamos agora o comportamento das derivadas de $\psi$ em $x=0$. Aplicando a definição de derivação e utilizando a hipótese de que $\phi^{(n)}(0)$ existe para qualquer $n \in \mathbb{N}$, teremos:
		$$
		\begin{aligned}
		\psi^{(0)}(0) 
		= & \lim_{x \to 0}\frac{\phi^{(0)}(x) - \phi^{(0)}(0)}{x} = \phi^{(1)}(0) \\
		\psi^{(1)}(0) 
		= & \lim_{x \to 0}\frac{\psi^{(0)}(x) - \psi^{(0)}(0)}{x} = \lim_{x \to 0}\frac{\frac{\phi^{(1)}(x) - \phi^{(1)}(0)}{x} - \phi^{(1)}(0)}{x} = \phi^{(2)}(0) \\
		\vdots & \\
		\psi^{(n)}(0) 
		= & \lim_{x \to 0}\frac{\psi^{(n-1)}(x) - \psi^{(n-1)}(0)}{x} = \lim_{x \to 0}\frac{\frac{\phi^{(n)}(x) - \phi^{(n)}(0)}{x} - \phi^{(n)}(0)}{x} = \phi^{(n+1)}(0) \\
		\end{aligned}
		$$
		portanto, $\psi^{(n)}(0)$ existe e é dado por $\psi^{(n)}(0) = \phi^{(n+1)}(0)$. Com isso podemos afirmar que $\psi:\Omega \to \mathbb{R}$ é tal que $\psi \in C^{0}$. 
		
		Necessitamos mostrar que $\psi \in C^{\infty}_{0}$. Pela definição $\psi(x) = \frac{\phi(x) - \phi(o)}{x}$, consequentemente $\psi(x) \neq 0 \iff \phi(x) \neq \phi(0)$, isto é, $supp(\psi) = \overline{ \{x \in \Omega: \phi(x) \neq \phi(0)\} }$. Vejamos os casos:
		\begin{enumerate}
			\item Caso $\phi$ constante e $\phi \neq 0$, então $\psi = 0$ e $supp(f) = \emptyset$ que é um compacto, portanto $\psi \in C^{\infty}_{0}$.
			
			\item Caso $\phi$ não-constante, então teremos os subcasos:
				\begin{enumerate}
					\item Supondo $\phi(0) = 0$ teremos $supp(\psi) = \overline{ \{x \in \Omega: \phi(x) \neq 0\} } = supp(\phi)$ que é um compacto, portanto $\psi \in C^{\infty}_{0}$.
					
					\item Supondo $\phi(0) \neq 0$ vamos mostrar que $supp(\psi) \neq \emptyset$. Usando o fato de que $\phi$ é contínua podemos afirmar que existem $\epsilon > 0$, uma bola $B(0, \epsilon)$ e um $y \in B(0, \epsilon)$ tal que $\phi(y) \neq \phi(0)$ (pois $\phi$ é não-constante), portanto $y \in supp(\psi)$, logo $supp(\psi) \neq \emptyset$. Tome agora $x \in supp(\psi)$, então $\phi(x) \neq \phi(0)$, mas como $\phi(0) \neq 0$ então $\phi(x) \neq 0$ , assim temos que $x \in supp(\phi)$, portanto $supp(\psi) \subseteq supp(\phi)$ e consequentemente $supp(\psi)$ é limitado pois é um subconjunto de um compacto da reta, além disso, como $supp(\psi)$ é um conjunto fechado, então é fechado e limitado, portanto é um compacto e $\psi \in C^{\infty}_{0}$.
				\end{enumerate}
		\end{enumerate}
		
		Conclusão: $\psi \in C^{\infty}_{0}$ e pela definição dessa função podemos escrever $\phi(x) = \phi(0) + x \psi(x)$.
		
		\item \textbf{Solução:} Vamos omitir o aberto $\Omega$ das notações, por simplicidade.
		\begin{enumerate}
			\item Devemos mostrar que $L^{P} \subsetneq L^{p}_{loc}$ para $1 \leq p \leq \infty$, ou seja, $\exists f \in L^{p}_{loc}; f \notin L^{p}$. 
			
			Primeiro para $1 \leq p < \infty$, tomemos $f \in L^{p}$ e com isso $||f||_{p} < \infty$, o que implica que para qualquer compacto $K \subseteq \Omega$ temos 
			$$
			||f\chi_{K}||_{p} = \Big( \int_{\Omega}|f(x)\chi_{K} (x)|^{p} \Big)^{1/p}= \Big( \int_{K}|f(x)|^{p} \Big)^{1/p} \leq \Big( \int_{\Omega}|f(x)|^{p} \Big)^{1/p} \ = ||f||_{p} < \infty,
			$$
			logo $f \in L^{p}_{loc} \Rightarrow L^{p} \subseteq L^{p}_{loc}$. Por outro lado, fazendo $\Omega = \mathbb{R}$ e definindo a função constante $f(x) = 1$. Tomando um compacto qualquer $K \subset \Omega$ teremos que 
			$$
			||f \chi_{K}||_{p} = \Big( \int_{\Omega} |f(x)\chi_{K}(x)|^{p} \Big)^{1/p} = \Big( \int_{K} 1 \Big)^{1/p} < \infty,
			$$
			contudo, 
			$||f||_{p} = \infty$, portanto $f \notin L^{p} \Rightarrow L^{p} \subsetneq L^{p}_{loc}$.
			
			Resta-nos mostrar o caso em que $p=\infty$. Tomemos $f \in L^{\infty}$ e com isso $||f||_{\infty} < \infty$, o que implica que para qualquer compacto $K \subseteq \Omega$ temos 
			$$
			||f\chi_{K}||_{\infty} = \sup_{x \in \Omega}|f(x)\chi_{K} (x)|= \sup_{x \in K}|f(x)| \leq \sup_{x \in \Omega}|f(x)| = ||f||_{\infty} < \infty,
			$$
			logo $f \in L^{\infty}_{loc} \Rightarrow L^{\infty} \subseteq L^{\infty}_{loc}$. Por outro lado, fazendo $\Omega = \mathbb{R}$ e definindo a função constante $f(x) = 1$. Tomando um compacto qualquer $K \subset \Omega$ teremos que 
			$$
			||f \chi_{K}||_{\infty} = \sup_{x \in \Omega}|f(x)\chi_{K} (x)| = \sup_{x \in K}|1| = 1 < \infty,
			$$
			contudo, 
			$||f||_{\infty} = \infty$, portanto $f \notin L^{p} \Rightarrow L^{p} \subsetneq L^{p}_{loc}$.
			
			Conclusão, $L^{p}$ esta contido propriamente em $L^{p}_{loc}$ para $1 \leq p \leq \infty$.
			
			\item Supondo $\Omega$ limitado, então existe uma bola fechada centrada em algum ponto $x_{0} \in \Omega$ tal que $\Omega \subset B(x_{0}, r)$, assim podemos definir $\omega := \int_{\Omega} 1 < \int_{B(x_0, r)} 1 < \infty$. Note que para $1 \leq p \leq q \leq \infty$ temos:
			$$
			||f||_{q} = \Big( \int_{\Omega} |f(x)|^{q} \Big)^{1/q} \leq \Big( \int_{\Omega} (\sup_{y \in \Omega}|f(y)|)^{q} \Big)^{1/q} = \sup_{y \in \Omega}|f(y)|\int_{\Omega} 1 = ||f||_{\infty}\omega,
			$$
			lembrando que pela limitação de $\Omega$ temos $\omega < \infty$. Com isso, caso $||f||_{\infty} < \infty$ então teremos pela desigualdade anterior $||f||_{q} \leq ||f||_{\infty}\omega < \infty \Rightarrow L^{\infty} \subseteq L^{q}$. Vamos mostrar agora que se $p \leq q$ então $L^{q} \subseteq L^{p}$. Com efeito, seja $f \in L^{q}$ e definamos $\Omega_{1} := \{ |f(x)| \leq 1: x \in \Omega \}$,  $\Omega_{2} := \Omega\backslash\Omega_{1}$ assim $\Omega = \Omega_{1} \cup \Omega_{2}$. Integrando:
			$$
			\begin{aligned}
			\int_{\Omega}|f(x)|^{p}
			= & \int_{\Omega_{1}} \underbrace{|f(x)|^{p}}_{\leq 1} + \int_{\Omega_{2}}|f(x)|^{p} \\
			\leq & \underbrace{ \int_{\Omega_{1}} 1 }_{< \infty}  + \int_{\Omega_{2}}|f(x)|^{p} \\
			= & \omega_{1} + \int_{\Omega_{2}} |f(x)|^{p}, \; \text{como} \; |f(x)|>1 \; e \; p\leq q \; \text{teremos}\\ 
			\leq & \omega_{1} + \int_{\Omega_{2}}|f(x)|^{q} = \underbrace{ \omega_{1} }_{< \infty }+ \underbrace{ ||f||_{q}^{q} }_{< \infty} < \infty \\
			\therefore & \; ||f||_{p}^{p} = \int_{\Omega}|f(x)|^{p} < \infty.
			\end{aligned}
			$$
			logo, $||f||_{p} <\infty \Rightarrow f \in L^{p}$, portanto $L^{q} \subseteq L^{p}$. Temos assim o encadeamento $L^{\infty} \subseteq L^{q} \subseteq L^{p}$. 
			
			No item anterior vimos que se $1 \leq p \leq \infty$ temos $L^{p} \subseteq L^{p}_{loc}$, então $L^{\infty} \subseteq L^{q} \subseteq L^{p} \subseteq L^{p}_{loc}$. Agora resta-nos mostrar que $L^{p}_{loc} \subseteq L^{1}_{loc}$ para $1 < p < \infty$. Pelo encadeamento, sabemos que $L^{q}\subseteq L^{p}_{loc}$, então tomemos $f \in L^{q}$, $K \subset \Omega$ um compacto qualquer e $\chi_{K}$ a função característica em $K$. Pela desigualdade de Holder $||f\chi_{K}||_{1} \leq ||f\chi_{K}||_{p}||f\chi_{K}||_{q}$, e pela escolha de $f$ teremos que $||f\chi_{K}||_{p} < \infty$ e $||f\chi_{K}||_{q} \leq ||f||_{q} < \infty$, que voltando na desigualdade teremos $||f\chi_{K}||_{1} < \infty \Rightarrow f \in L^{1}_{loc}$, logo $L^{p}_{loc} \subseteq L^{1}_{loc}$. 
			
			Conclusão: $L^{\infty} \subseteq L^{q} \subseteq L^{p} \subseteq L^{p}_{loc} \subseteq L^{1}_{loc}$, como desejávamos.
		\end{enumerate}
		
		\item \textbf{Solução:} Seja $f \in \mathcal{S}(\Omega)$ e aplicando a definição de continuidade em $f$, temos que dado $\epsilon >0 \; \exists \; \delta >0$ tal que $||f - g||_{\infty} < \delta, \forall g \in \mathcal{S}(\Omega) \Rightarrow ||\delta_{x_0}(f) - \delta_{x_0}(g)|| < \epsilon$. Tomando $\delta = \epsilon/2$ teremos $$
		||\delta_{x_0}(f) - \delta_{x_0}(g)|| = ||f(x_{0}) - g(x_{0})|| \leq \sup_{x \in \Omega}|f(x) -g(x)| = ||f-g||_{\infty} < \delta = \epsilon/2 < \epsilon,
		$$ 
		portanto $||\delta_{x_0}(f) - \delta_{x_0}(g)||<\epsilon$ e como $f$ é arbitrária, então $\delta_{x_{0}} : \mathcal{S}(\Omega) \to \mathbb{R}$ é contínuo. Resta-nos mostrar a linearidade. Sejam $f, g \in \mathcal{S}(\Omega)$ e $\lambda \in \mathbb{R}$, e já sabemos da lista anterior que $f+\lambda g \in \mathcal{S}(\Omega)$, então $\delta_{x_{0}}(f+\lambda g) = (f+\lambda g)(x_{0}) = f(x_{0})+\lambda g(x_{0}) = \delta_{x_{0}}(f) + \lambda \delta_{x_{0}}(g)$, portanto é um operador linear. Conclusão: $\delta_{x_{0}}$ é um operador contínuo e linear, como desejávamos.
		
		\item \textbf{Solução:} Verifiquemos o resultado para o caso em que $n =1$ e tomemos $k \in \mathbb{N}_{0}$ como um multiíndice e $f, g \in C^{\infty}$. Vamos convencionar a derivada parcial $\partial/\partial {x_{j}} = \partial_{j}$ e nessas condições temos que $D^{|k|} = \partial^{k}_{1}$ o que nos leva a fórmula de Leibnitz para derivação:
		$$
		\begin{aligned}
		D^{|k|} (fg)
		= & \partial^{k}_{1} (fg) = \sum_{j=0}^{k} \frac{k!}{j!(k-j)!}\partial_{1}^{j}f \partial_{1}^{k-j}g
		\end{aligned}
		$$
	\end{enumerate}
	
\end{document}