\documentclass{article}
\usepackage{graphicx}
\usepackage{indentfirst}
\usepackage[utf8]{inputenc}
\usepackage{amssymb}
\usepackage{enumitem}
\usepackage{color}
\usepackage[a4paper, margin=0.5in]{geometry}
\begin{document}
	
	\title{Lista 2 para Entregar}
	\author{Vinicius Fernades}
	
	\maketitle
	
	\begin{enumerate}
		\item Seja $X$ um subconjunto da reta, $\mathcal{L}(X)$ o conjunto das funções $f:X \to \mathbb{R}$ integráveis, $\sim$ a relação $f \sim g$ se, e somente se, o conjunto em que $f$ difere de $g$ possui medida nula, e $\ell^{p}$ o espaço formado pelas sequências $(x_n)_{n \in \mathbb{N}}$ tais que
		$$
		\sum \limits_{j=1}^{\infty} |x_j|^{p} < \infty.
		$$
		\begin{enumerate}
			\item Mostre que $\sim$ é uma relação de equivalência.
			
			\textbf{Solução:} Sejam $f, g, h \in \mathcal{L}(X)$, então devemos mostrar a reflexividade, comutatividade e transitividade:
			\begin{enumerate}
				\item (\textit{Reflexividade}) Como $f$ é integrável, então o conjunto em $f(t) \neq f(t)$ é vazio, que tem medida nula, portanto $f \sim f$.
				
				\item (\textit{Comutatividade}) Supondo $f \sim g$, então $f(t) \neq g(t)$ em um conjunto de medida nula, mas $g(t) \neq f(t)$ nesse mesmo conjunto, portanto $g \sim f$. 
				
				\item (\textit{Transitividade}) Supondo $f \sim g$ e $g \sim h$, então sejam os conjuntos de medida nula $A=\{t \in X: f(t) \neq g(t)\}$ e $B=\{t \in X: g(t) \neq h(t)\}$, então: se $f = g \Rightarrow A=B$ e teremos que $f \sim h$. Caso exista um $t_0 \in X; f(t_0) = g(t_0)$ e $f(t_0) \neq h(t_0) \Rightarrow t_0 \in B$, mas caso  $f(t_0) \neq g(t_0)$ e $f(t_0) \neq h(t_0) \Rightarrow t_0 \in A$, assim, os ponto em que $f$ difere de $h$ estão em $A$ ou $B$, ou seja, $A\cup B=\{ t \in X: f(t) \neq h(t) \}$ é de medida nula e portanto $f \sim h$.  
			\end{enumerate}
			Portanto $\sim$ é uma relação de equivalência, como desejávamos. 
			
			
			\item Mostre que $L_{p}(X) := \mathcal{L}(X)/\sim$ é um espaço vetorial
			
			\textbf{Solução:} Seja $f, g, h \in L_{p}(X)$ e $\lambda, \delta \in \mathbb{R}$. Sabemos que $\int_{X}f  + \int_{X} g = \int_{X}(f+g)$ pois a soma de funções integráveis é uma função integrável, portanto $f+g \in L_{p}(X)$. Então:
				\begin{enumerate}
					\item (\textit{Comutatividade}) Já vimos que $L_{p}(X)$ é fechado pela operação de adição, assim podemos supor que $f +g \in L_{p}(X)$, então 
					$$
					\int_{X}(f+g) = \int_{X}f+ \int_{X}g =  \int_{X}g+ \int_{X}f =
					$$
					$$
					= \int_{X}(g+ f)  \Rightarrow f+g = g+f \in L_{p}(X)
					$$
					
					\item (\textit{Associatividade}) Supondo $g+h \in L_{p}$
					$$
					\int_{X} (f+(g+h)) = \int_{X} f+ \int_{X}(g+h) = \int_{X} f+ \int_{X} g + \int_{X}h = 
					$$
					$$
					=\int_{X} (f+g) +\int_{X} h = \int_{X}((f+g)+h) \Rightarrow f+(g+h)=(f+g)+h \in L_{p}(X)
					$$
					
					\item (\textit{Elemento neutro}) Claro que a função identicamente nula $0 \in L_{p}(X) $, pois é integrável. Além disso $\int_{X}(0 + f) = \int_{X} f(t) \Rightarrow 0 + f = f$.
					 
					\item (\textit{Elemento inverso}) Sabemos que se $f \in L_{p}(X)$, então 
					$$
					\int_{X}(f + (-f)) = \int_{X} f +\int_{X}(-f)  = 0 \Rightarrow \int_{X}(-f) = - \int_{X}f, 
					$$
					portanto $-f \in L_{p}(X)$.
					
					\item $\lambda(\delta f) = (\lambda \delta) f \in L_{p}(X) $ pois  $\int_{X} \lambda(\delta f) = \int_{X} (\lambda\delta)f$ que é uma função integrável.
					
					\item (\textit{Distributividade}) Supondo $\lambda (f+g) \in L_{p}(X) $, então
					$$
					\int_{X} \lambda (f+g) = \int_{X} \lambda f+ \lambda g \Rightarrow \lambda (f+g)  = \lambda f + \lambda g \in L_{p}(X)
					$$
				\end{enumerate}
				Conclusão: $L_{p}(X)$ é um espaço vetorial, como desejávamos.
				
				\item Mostre também que $(L_{p}(X), ||.||_{p}); p \geq 1$ é um espaço normado, onde
					$$
					||f||_{p} := \Big (\int_{X} |f(x)|^{p} \Big )^{1/p}
					$$
					\begin{enumerate}
						\item $||f||_{p} \geq 0$, pois $|f(x)|^{p} \geq 0; \forall x \in X \Rightarrow \int_{X} |f(x)|^{p} \geq 0$, e $\forall p \geq 1$. Agora, $||f||_{p} = 0 \iff f \sim 0$. $(\Leftarrow)$ Suponha que $f \sim 0$, então $\int_{X} |f(x)|^{p} = \int_{X} 0 = 0 \Rightarrow ||f||_{p} = 0$. $(\Rightarrow)$ Supondo que $||f||_{p} = 0$, então por definição 
						$$
						\Big (\int_{X} |f(x)|^{p} \Big )^{1/p} = 0 \Rightarrow |f|^{p} =0 \in  L_{p}(X), 
						$$
						ou seja, $f$ difere da função nula apenas em um um conjunto de medida nula, isto é, $f = 0 \in  L_{p}(X)$.
						
						\item $||\lambda f||_{p} = \Big ( \int_{X} |\lambda f(x)|^{p} \Big )^{1/p} = \Big ( \int_{X} |\lambda|^{p}| f(x)|^{p} \Big )^{1/p} = |\lambda|||f||_{p}$.
						
						\item Vamos demonstrar a desigualdade de Minkowski, que é uma generalização da desiguldade triangular. Suponha a desigualdade de Holder $\int_{X} |f(x) g(x)| \leq ||f||_{p} ||g||_{q}$, onde $1/p + 1/q = 1$, então:
						$$
						||f+g||_{p}^{p} =\int_{X} |f(x) + g(x)|^{p} = \int_{X} |f(x) + g(x)||f(x) + g(x)|^{p-1} \leq \int_{X} (|f(x)| + |g(x)|)|f(x) + g(x)|^{p-1}
						$$
						$$
						 = \int_{X} |f(x)||f(x) + g(x)|^{p-1} + \int_{X}|g(x)||f(x) + g(x)|^{p-1} \leq (||f||_{p} + ||g||_{p})\Big( \int_{X} |f(x) + g(x)|^{(p-1)q} \Big)^{1/q} = 
						$$
						$$
						=(||f||_{p} + ||g||_{p})\Big( \int_{X} |f(x) + g(x)|^{p} \Big)^{-1+1/p} = (||f||_{p} + ||g||_{p})||f + g||_{p}^{p-1} \Rightarrow
						$$
						$$
						\Rightarrow ||f+g||_{p} \leq ||f||_{p} + ||g||_{p}. 
						$$
						Que é a desigualdade triangular, portanto $||.||_{p}$ é uma norma, e como $L_{p}(X)$ é um espaço vetorial, como foi provado anteriormente, então $(L_{p}(X), ||.||_{p})$ é um espaço vetorial normado, como desejávamos.
					\end{enumerate}
					
				\item (Desigualdade de Holder para sequências) Prove que se $1/p + 1/q = 1$ e $x, y \in \ell^{p}$, então:
				$$
				\sum \limits_{i=1}^{\infty} |x_{i}y_{i}| \leq \Big(\sum \limits_{i=1}^{\infty} |x_{i}|^{p}\Big)^{1/p} \Big(\sum \limits_{i=1}^{\infty}|y_{i}|^{q}\Big)^{1/q}.
				$$
				
				\textbf{Solução:}
				
				\item Prove que $(\ell^{p}, ||.||_p)$ é um espaço normado, onde
				$$
				||x||_{p} := \Big(\sum \limits_{i=1}^{\infty} |x_{i}|^{p}\Big)^{1/p}
				$$
				
				\textbf{Solução:} Seja a função $f: \ell^{p} \times \mathbb{R} \to \mathbb{R}$, tal que fixando $f_{x} := \mathbb{R} \to \mathbb{R}$ onde $f_{x}(t) := f(x, t)$. Então, definindo:
				$$
				f_{x}(t) := \left\{
				\begin{array}{cc}
				0, & t \notin [i, i+1)\\
				x_{i}, & t \in [i, i+1) \\
				\end{array}
				\right.
				$$
				Note que $\forall x \in \ell^{p}$ temos que $x \mapsto f_{x}$ esta bem-definida. Afirmo que $f_{x} \in L_{p}(X)$. Com efeito, vejamos que pela definição da função degrau, a integral será reduzida a uma somatória nos intervalos $t \in [i, i+1)$: 
				$$
				||f_{x}||_{p} = \Big(\int_{X}|f_x(t)|^{p}\Big)^{1/p} = \Big(\sum \limits_{i=1}^{\infty}|x_i|^{p}\Big)^{1/p} = ||x||_{p} < \infty, 
				$$
				portanto $f_{x}$ é integrável (a menos de um conjunto de medida nula), com isso $f_{x} \in L_{p}(X)$, como desejávamos. Além disso, vejamos que $\forall x \in \ell^{p}$ temos $||x||_{p} = ||f_{x}||_{p}$ pelas equações anteriores, e como essa ultima é uma norma, então temos que $||.||_{p}: \ell^{p} \to \mathbb{R}$ também é uma norma. Vamos demonstrar agora que dado $\lambda \in \mathbb{R}$ e $x, y \in \ell^{p}$ então $x+\lambda y \in \ell^{p}$. Assim, como é uma norma:
				$$
				||x+\lambda y||_{p} \leq ||x||_{p} + |\lambda |||y||_{p} < \infty,
				$$
				pois $||x||_{p} < \infty$ e $||y||_{p} < \infty$, portanto $x+\lambda y \in \ell^{p}$.
				
				Vejamos que $x \mapsto f_{x}$ é um operador linear. Seja $x + \lambda y \in \ell^{p}$ com $\lambda \in \mathbb{R}$ temos
				$$
				f_{x + \lambda y }(t) := \left\{
				\begin{array}{cc}
				0, & t \notin [i, i+1)\\
				x_{i} +\lambda y_{i} = f_{x}(t) + \lambda f_{y}(t), & t \in [i, i+1) \\
				\end{array}
				\right. \Rightarrow				
				$$
				$$
				\Rightarrow f_{x + \lambda y }(t) := f_{x}(t) + \lambda f_{y}(t),
				$$
				ou seja, $f_{x + \lambda y } = f_{x} + \lambda f_{y } \in f(\ell^{p}) \subset L_{p}(X)$ que mostra que é um operador linear, além do mais $f(\ell^{p})$ é um subespaço vetorial de $L_{p}(X)$ pois $0 \in f(\ell^{p})$ e $\forall \lambda \in \mathbb{R}$ e $f_{x}, f_{y} \in \ell^{p}$ temos $f_{x} + \lambda f_{y } \in f(\ell^{p})$, como vimos anteriormente. É um operador injetivo pois tomando duas sequências $x, y \in \ell^{p}$ teremos $f_{x} = f_{y} \Rightarrow f_{x}(t) = f_{y}(t) \iff x_{i} = y_{i} \forall i \in \mathbb{N}, t \in \mathbb{R}$, isto é, $x=y$. Por fim, $f: \ell^{p} \to f(\ell^{p}) \subset L_{p}(X)$ é sobrejetora. Conclusão: $f$ é linear, injetora e sobrejetora sobre a imagem $f(\ell^{p})$, portanto um isomorfismo. Como, $\ell^{p}$ é isomorfo a um espaço vetorial, então é um espaço vetorial também, como desejávamos.
				
				
				
				
		\end{enumerate}
		
		
		\item A função $f:\mathbb{R} \to \mathbb{C}$ é dita ser periódica com período $T \neq 0$ se $f(x+T) = f(x), \forall x \in \mathbb{R} $. O conjunto de todas as funções periódicas de período $2\ell$ e de classe $C^{k}$ é denotado por $\mathcal{C}^{k}_{l}:=C^{k}([-\ell, \ell]; \mathbb{C})$. Para cada $k \in \mathbb{Z}$ e $f \in C^{0}([-\pi, \pi]; \mathbb{C})$, defina
		$$
		\hat{f}(k) = \frac{1}{2\pi}\int_{-\pi}^{\pi} f(x)e^{-ikx}dx
		$$
		
		\begin{enumerate}
			\item Mostre que $(\mathcal{C}^{k}_{l}; ||.||_{\infty})$ (definido no enunciado) é um espaço de Banach, onde $||f||_{\infty} = \sup_{x \in \mathbb{R}}|f(x)|$.
			
			\textbf{Solução:} Primeiro vamos mostrar que é um espaço vetorial. Sabemos que o espaço da funções contínuas $C^{0}(\mathbb{R})$ é um espaço vetorial e como $\mathcal{C}^{k}_{l} \subset C^{0}(\mathbb{R})$, então resta-nos mostrar que é um subespaço vetorial, portanto, será um espaço vetorial.
			
			Note que o elemento neutro, que é a função identicamente nula, é sempre periódica, portanto $0 \in \mathcal{C}^{k}_{l}$. Sejam $f, g \in \mathcal{C}^{k}_{l}$ e $\lambda \in \mathbb{R}$, então $(f+\lambda g)(x +\ell) = f(x +\ell)+\lambda g(x +\ell) = f(x)+\lambda g(x)=(f+\lambda g)(x) \Rightarrow f+\lambda g \in  \mathcal{C}^{k}_{l}$. Portanto, $\mathcal{C}^{k}_{l}$ é um subespaço vetorial, isto é, um espaço vetorial. Já sabemos que $||.||_{\infty}:\mathcal{C}^{k}_{l} \to \mathbb{R}$ é uma norma, portanto $(\mathcal{C}^{k}_{l}; ||.||_{\infty})$ é um espaço normado. Resta-nos mostrar que é completo nessa norma, isto é, toda sequência de convergente tem seu limite contido nele. Primeiramente, mostramos na lista 1 que a convergencia na norma infinita é equivalente a convergencia uniforme, isto é, se uma sequência converge, então essa convergência é uniforme. Tome $(f_{n})_{n \in \mathbb{N}} \subset \mathcal{C}^{k}_{l}$ uma sequência convergente tal que $f_{n} \to f$. Sabemos que essa convergência é uniforme, e como é uma sequência de funções contínuas, então o limite é uma função contínua. Vamos mostrar que esse limite é uma função periódica. Pelos argumentos anteriores vale o limite 
			$$
			f(x + \ell) = \lim\limits_{n\to \infty} f_{n}(x + \ell) = \lim\limits_{n\to \infty} f_{n}(x ) = f(x),   
			$$
			portanto $f \in \mathcal{C}^{k}_{l}$, o que mostra que esse espaço é completo, como desejávamos.
			
			\item Mostre que
			$$
			||\hat{f}|| \leq \frac{1}{2\pi}||f||_{L^{1}}\leq ||f||_{L^{\infty}},
			$$
			e conclua que a sequência $\hat{f} := (\hat{f}(k))_{k \in \mathbb{Z}}$ é um elemento de $\ell^{\infty}(\mathbb{Z})$.
			
			\textbf{Solução:} Vamos definir $||\hat{f}||:= \sup_{k \in \mathbb{Z}}|\hat{f}(k)|$, então:
			
			$$
			|\hat{f}(k)| = \frac{1}{2\pi} \Big |\int_{-\pi}^{\pi} f(x)e^{-ikx}dx \Big | \leq \frac{1}{2\pi} \int_{-\pi}^{\pi} |f(x)|dx = \frac{1}{2\pi}||f||_{L^{1}} \Rightarrow  
			$$
			$$
			\Rightarrow ||\hat{f}||:= \sup_{k \in \mathbb{Z}}|\hat{f}(k)| \leq \frac{1}{2\pi}||f||_{L^{1}}.
			$$
			Contudo, temos uma segunda majoração
			
			$$
			\frac{1}{2\pi}||f||_{L^{1}}= \frac{1}{2\pi} \int_{-\pi}^{\pi} |f(x)|dx \leq \frac{1}{2\pi} \int_{-\pi}^{\pi} \sup_{y \in \mathbb{R}}|f(y)|dx = 
			$$
			$$
			= \frac{\sup_{y \in \mathbb{R}}|f(y)|}{2\pi} \int_{-\pi}^{\pi} dx = ||f||_{L^{\infty}}.
			$$
			Por fim, temos a sequência de desigualdes desejada
			
			$$
			||\hat{f}|| \leq \frac{1}{2\pi}||f||_{L^{1}}\leq ||f||_{L^{\infty}} < \infty.
			$$
			Sabemos que $\ell^{\infty}(\mathbb{Z})$ é o espaço das sequências indexadas pelos inteiros tais que, se $x = (x_{k})_{k \in \mathbb{Z}} \in \ell^{\infty}(\mathbb{Z})$ então $||x|| = \sup_{k \in \mathbb{Z}}|x_{k}| < \infty$. Mas observe que por definição $\hat{f} = (\hat{f}(k))_{k \in \mathbb{Z}}$ é uma sequência indexada por inteiros tal que $||\hat{f}|| <\infty$, portanto $\hat{f} \in \ell^{\infty}(\mathbb{Z})$, como desejávamos.
			
			\item Mostre que $\mathcal{F}: \mathcal{C}^{0}_{\pi} \to \ell^{\infty}(\mathbb{Z})$ definida por $\mathcal{F}(f) := \hat{f}$ é um operador linear limitado, chamado \textit{transformada de Fourier} da função $f$.
			
			\textbf{Solução:} Primeiramente, vamos mostrar a linearidade. Sejam $\lambda \in \mathbb{R}$ e $f, g \in \mathcal{C}^{0}_{\pi}$, então:
			$$
			\mathcal{F}(f+\lambda g)(k) = \widehat{f+\lambda g}(k) =  \frac{1}{2\pi}\int_{-\pi}^{\pi} (f(x)+\lambda g(x))e^{-ikx}dx = 
			$$
			$$
			= \frac{1}{2\pi}\int_{-\pi}^{\pi} f(x)e^{-ikx}dx + \lambda \frac{1}{2\pi}\int_{-\pi}^{\pi} g(x)e^{-ikx}dx = 
			$$
			$$
			= \hat{f}(k) +\lambda \hat{g}(k) = \mathcal{F}(f)(k) + \lambda \mathcal{F}(g)(k),
			$$
			então $\mathcal{F}(f+\lambda g) = \mathcal{F}(f) + \lambda \mathcal{F}(g)$ e $\mathcal{F}$ é um operador linear.
		\end{enumerate}
		
	\end{enumerate}
		
\end{document}