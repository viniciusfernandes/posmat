\documentclass{article}
\usepackage{graphicx}
\usepackage{indentfirst}
\usepackage[utf8]{inputenc}
\usepackage{amssymb}
\usepackage{enumitem}
\usepackage{color}
\usepackage[fleqn]{amsmath}
\usepackage[a4paper, margin=0.5in]{geometry}
\begin{document}
	
	\title{Lista 7 para Entregar}
	\author{Vinicius Fernandes}
	
	\maketitle
	
	\begin{enumerate}
		
		\item \textbf{Solução:} Vamos usar os seguintes fatos: se $g \in \mathcal{S}(\mathbb{R}^{n})$ teremos que $x^{\alpha} D^{|\beta|}g \in \mathcal{S}(\mathbb{R}^{n})$, de modo que valem os seguintes limites $\lim_{|x| \to \infty} g(x) = 0$ e $\lim_{|x| \to \infty} x^{\alpha} D^{|\beta|}g(x) = 0$ pois ambas as funções possuem suporte compacto. Esse fato será usado no caso particular abaixo em que teremos $\alpha = 0$, isto é, $\lim_{|x| \to \infty} D^{|\beta|}g(x) = 0$.
		
		Estudemos o caso em que $n = 1$, ou seja, a reta real. Sejam $f, g \in \mathcal{S}(\mathbb{R})$ e tomemos $k \in \mathbb{N}$:
		$$
		\begin{aligned}
		\int_{\mathbb{R}} f^{(k)}(x)g(x) = & \int_{\mathbb{R}} (f^{(k-1)}(x)g(x))' - f^{(k-1)}(x)g'(x)
		\\ 
		= & \underbrace{ f^{(k-1)}(x)g(x)'\Big|^{\infty}_{-\infty} }_{\to 0}- \int_{\mathbb{R}}  f^{(k-1)}(x)g'(x)
		\\ 
		= & -\int_{\mathbb{R}}  f^{(k-1)}(x)g'(x) \;
		\\
		 \therefore &
		 \\
		\langle f^{(k)},g \rangle = & - \langle f^{(k-1)},g^{(1)} \rangle
		\end{aligned}
		$$
		Realizando novamente esse procedimento para o termo $\langle f^{(k-1)},g^{(1)} \rangle$:
		$$
		\begin{aligned}
		\int_{\mathbb{R}} f^{(k-1)}(x)g^{(1)}(x) = & \int_{\mathbb{R}} (f^{(k-2)}(x)g^{(1)}(x))' - f^{(k-2)}(x)g^{(2)}(x)
		\\ 
		= & \underbrace{ f^{(k-2)}(x)g^{(1)}(x)'\Big|^{\infty}_{-\infty} }_{\to 0}- \int_{\mathbb{R}}  f^{(k-2)}(x)g^{(2)} (x)
		\\ 
		= & -\int_{\mathbb{R}}  f^{(k-2)}(x)g^{(2)} (x) \;  
		\\
		\therefore &
		\\
		\langle f^{(k-1)},g^{(1)} \rangle = & - \langle f^{(k-2)},g^{(2)} \rangle.
		\end{aligned}
		$$
		E substituindo esse resultado na relação anterior $\langle f^{(k)},g \rangle = - \langle f^{(k-1)},g^{(1)} \rangle = (-1)^{2} \langle f^{(k-2)},g^{(2)} \rangle$. Vamos aplicar a hipótese de indução para a demonstração final. Suponha que valha $\langle f^{(k)},g \rangle = (-1)^{j} \langle f^{(k-j)},g^{(j)} \rangle$ para $0 \leq j \leq k$, vamos mostrar que vale para $j+1$. Com efeito, ja vimos anteriormente que vale para $j=1$, então:
		$$
		\begin{aligned}
		\int_{\mathbb{R}} f^{(k-j)}(x)g^{(j)}(x) = & \int_{\mathbb{R}} (f^{(k-(j+1))}(x)g^{(j)}(x))' - f^{(k-(j+1))}(x)g^{(j+1)}(x)
		\\ 
		= & \underbrace{ f^{(k-(j+1))}(x)g^{(j)}(x)\Big|^{\infty}_{-\infty} }_{\to 0}- \int_{\mathbb{R}}  f^{(k-(j+1))}(x)g^{(j+1)}(x)
		\\ 
		= & -\int_{\mathbb{R}}  f^{(k-(j+1))}(x)g^{(j+1)}(x) \; 
		\\
		\therefore &
		\\
		\langle f^{(k-j)},g^{(j)} \rangle = & - \langle f^{(k-(j+1))},g^{(j+1)} \rangle.
		\end{aligned}
		$$
		E pela hipótese de indução temos $\langle f^{(k)},g \rangle = (-1)^{j} \langle f^{(k-j)},g^{(j)} \rangle = (-1)^{j+1}\langle f^{(k-(j+1))},g^{(j+1)} \rangle$, portanto $\langle f^{(k)},g \rangle = (-1)^{j+1}\langle f^{(k-(j+1))},g^{(j+1)} \rangle$ o que completa a demostração por indução. 
		
		Temos a relação: 
		$$
		\langle f^{(k)},g \rangle = (-1)^{j}\langle f^{(k-j)},g^{(j)} \rangle \; \text{onde} \; 0 \leq j \leq k,
		$$
		que aplicando-a para o caso particular em que $j = k$ teremos $\langle f^{(k)},g \rangle = (-1)^{k}\langle f,g^{(k)} \rangle$.
		
		Façamos agora o caso em que $f, g \in \mathcal{S}(\mathbb{R}^{n})$ e tomando um multi-índice $a \in \mathbb{N}^{n}$. Sabemos que $D^{|a|}f = \partial^{a_{1}}_{1} \dots \partial^{a_{n}}_{n} f$, onde $\partial^{a_{i}}_{i}  = \partial^{a_{i}}/\partial x^{a_{i}}_{i}$ com $0\leq i \leq n$. Definindo $F = \partial^{a_{1}}_{1}...\partial^{a_{i-1}}_{i-1}\partial^{a_{i+1}}_{i+1}...\partial^{a_{n}}_{n}f = D^{|a|-a_{i}}f$ então $\partial^{a_{i}}_{i}F = D^{|a|}f$, consequentemente teremos $\langle D^{|a|}f, g \rangle =\langle \partial^{a_{i}}_{i}F, g \rangle$. Como no caso unidimensional, faremos a demonstração por indução. Afirmo que $\langle \partial^{a_{i}}_{i}F, g \rangle = (-1)^{j} \langle \partial^{a_{i}-j}_{i}F, \partial^{j}_{i}g \rangle$ para $0 \leq j \leq a_{i}$
		$$
		\begin{aligned}
		\int_{\mathbb{R}^{n}}\partial^{a_{i}}_{i}F(x)g(x) = & \int_{\mathbb{R}^{n}} \partial_{i}(\partial^{a_{i}-1}_{i}F(x)g(x)) -  \partial^{a_{i}-1}_{i}F(x) \partial_{i}g(x)
		\\
		= & \int_{\mathbb{R}^{n-1}} \underbrace{ \lim_{x_{i} \to \infty}\partial^{a_{i}-1}_{i}F(x)g(x) \Big|^{x_{i}}_{-x_{i}} }_{\to 0} -  \int_{\mathbb{R}^{n}} \partial^{a_{i}-1}_{i}F(x) \partial_{i}g(x)
		\\
		= & - \int_{\mathbb{R}^{n}}\partial^{a_{i}-1}_{i}F(x) \partial_{i}g(x)
		\\
		\therefore
		\\
		\langle \partial^{a_{i}}_{i}F, g \rangle = & (-1)\langle \partial^{a_{i}-1}_{i}F, \partial_{i}g \rangle,
		\end{aligned}
		$$
		logo, verificamos que a afirmação vale para $j=1$, agora nos resta mostrar que vale para $j+1$. Com efeito, suponha que valha $\langle \partial^{a_{i}}_{i}F, g \rangle = (-1)^{j} \langle \partial^{a_{i}-j}_{i}F, \partial^{j}_{i}g \rangle$, então verifiquemos essa relação para $j+1$:
		$$
		\begin{aligned}
		\int_{\mathbb{R}^{n}}\partial^{a_{i}-j}_{i}F(x) \partial^{j}_{i}g(x) = & \int_{\mathbb{R}^{n}} \partial_{i}(\partial^{a_{i}-(j+1)}_{i}F(x)\partial^{j}_{i}g(x)) -  \partial^{a_{i}-(j+1)}_{i}F(x) \partial^{j+1}_{i}g(x)
		\\
		= & \int_{\mathbb{R}^{n-1}} \underbrace{ \lim_{x_{i} \to \infty} \partial^{a_{i}-(j+1)}_{i}F(x)\partial^{j}_{i}g(x) \Big|^{x_{i}}_{-x_{i}} }_{\to 0} -  \int_{\mathbb{R}^{n}} \partial^{a_{i}-(j+1)}_{i}F(x) \partial^{j+1}_{i}g(x)
		\\
		= & - \int_{\mathbb{R}^{n}} \partial^{a_{i}-(j+1)}_{i}F(x) \partial^{j+1}_{i}g(x)
		\\
		\therefore
		\\
		\langle \partial^{a_{i}-j}_{i}F, g \rangle = & (-1)\langle \partial^{a_{i}-(j+1)}_{i}F, \partial^{j+1}_{i}g \rangle.
		\end{aligned}
		$$
		E pela hipótese de indução temos $\langle \partial^{a_{i}}_{i}F, g \rangle = (-1)^{j} \langle \partial^{a_{i}-j}_{i}F, \partial^{j}_{i}g \rangle = (-1)^{j+1}\langle \partial^{a_{i}-(j+1)}_{i}F, \partial^{j+1}_{i}g \rangle$, o que completa a demonstração por indução.
		
		Temos a relação:
		$$
		\langle \partial^{a_{i}}_{i}F, g \rangle = (-1)^{j}\langle \partial^{a_{i}-j}_{i}F, \partial^{j}_{i}g \rangle \; \text{onde} \; 0 \leq j \leq a_{i},
		$$
		que aplicando-a para o caso particular em que $j = a_{i}$ teremos: 
		$$
		\begin{aligned}
		\langle \partial^{a_{i}}_{i}F, g \rangle = & (-1)^{a_{i}}\langle F, \partial^{a_{i}}_{i}g \rangle
		\\
		\langle D^{|a|}f, g \rangle = & (-1)^{a_{i}}\langle  D^{|a|-a_{i}}f, \partial^{a_{i}}_{i}g \rangle.
		\end{aligned}
		$$
		Note que realizamos esse procedimento para a i-ésima coordenada $x_{i}$, contudo, devemos repetir o procedimento para o termo  $\langle  D^{|a|-a_{i}}f, \partial^{a_{i}}_{i}g \rangle$ em todas as variáveis de modo a concluir a demonstração.
		
		\item \textbf{Solução:} Por simplicidade, vamos definir a notação $C^{0} = C^{0}(\mathbb{R}^{n})$, e tomaremos $f, g \in C^{0}$ e $\lambda \in \mathbb{R}$.
			\begin{enumerate}
				\item Sabemos que $C^{0}$ é um espaço vetorial, portanto $f + \lambda g \in C^{0}$. Vamos mostrar que a linearidade do operador $\tau_{a}$.
				$$
				\begin{aligned}
				(\tau_{a}(f+\lambda g))(x) = &  (f+\lambda g)(x+a) \\
				= & f(x+a) +(\lambda g)(x+a)
				\\
				= & f(x+a) +\lambda g(x+a)
				\\
				= & (\tau_{a}f)(x) +\lambda (\tau_{a}g)(x)
				\\
				\therefore &
				\\
				\tau_{a}(f+\lambda g) = & \tau_{a}(f)+\lambda \tau_{a}(g),
				\end{aligned}
				$$
				ou seja, $\tau_{a}$ é um operador linear.
				
				\item Notemos que $(\tau_{a} \circ \tau_{-a})(f)(x) = (\tau_{a})f(x-a) = f(x- a+a) = f(x)$, e também $(\tau_{-a} \circ \tau_{a})(f)(x) = (\tau_{a})f(x+a) = f(x+a-a) = f(x)$, portanto $\tau_{a} \circ \tau_{-a} = \tau_{-a} \circ \tau_{a} = Id \Rightarrow \tau_{-a} = \tau^{-1}_{a}$.
				
				\item Seja $\varphi \in \mathcal{S}(\mathbb{R}^{n})$, então:
				$$
				\begin{aligned}
				\langle \tau_{a}f, \varphi \rangle = & \int_{\mathbb{R}} (\tau_{a}f)(x)\varphi(x)
				= \int_{\mathbb{R}} f(\underbrace{ x+a }_{y = x+a}) \varphi(x)
				\\
				= & \int_{\mathbb{R}} f(y)\varphi(y-a)
				=  \int_{\mathbb{R}} f(y) (\tau_{-a}\varphi)(y) 
				\\
				= & \langle f, \tau_{-a}\varphi \rangle.
				\end{aligned}
				$$
				
				\item Primeiro, vejamos que e o operador $T$ é invertível. É injetor pois tomando $a, b \in \mathbb{R}^{n}$ teremos $Ta- Tb = Aa - Ab = A(a-b) = 0 \iff a=b$, pois $A$ é injetora. Além disso, para qualquer $y \in \mathbb{R}^{n}$ existe $x \in \mathbb{R}^{n}$ tal que tomando $x = A^{-1}(y -b)$ teremos $Tx = AA^{-1}(y -b) + b = y$, portanto é sobrejetora. Fazendo que $T^{-1}(x) = A^{-1}(x - b)$ teremos $TT^{-1}(x) = AA^{-1}(x - b) + b = x$, e também $T^{-1}T(x) = T^{-1}(Ax + b) = A^{-1}(Ax+b - b) = x$, portanto $TT^{-1} = T^{-1}T = Id$ e $T^{-1}$ é a inversa de $T$.
				
				Usaremos o seguinte resultado: tomando o diferencial do operator $T$ teremos $D_{x}(T): \mathbb{R}^{n} \to \mathbb{R}^{n}$ tal que $D_{x}(T)(v) = D_{x}(Ax+b)(v) = Av$, portanto $D_{x}(T) = A$ e $det(D_{x}(T)) = det(A) \neq 0 \; \forall x \in \mathbb{R}^{n}$. Além disso, realizando a mudança de coordenadas $y = Ax+b$ teremos $dy(x) = |det(D_{x}(T))|dx$ ou ainda $|det(D_{x}(T))|^{-1} dy(x) = dx$. Calculando:
				$$
				\begin{aligned}
				\langle f \circ T, \varphi \rangle = & \int_{\mathbb{R}} (f \circ T)(x)\varphi(x)
				= \int_{\mathbb{R}} f(\underbrace{ Ax+b }_{y = Ax+b})\varphi(x)
				\\
				= & \int_{\mathbb{R}} f(y)\varphi(\underbrace{ A^{-1}(y - b)}_{T^{-1}y} ) \Big| det(D_{x}(T)) \Big|^{-1}
				\\
				= & \int_{\mathbb{R}} f(y)\varphi(T^{-1}y)  |det(A)|^{-1}
				\\
				= & \frac{1}{|det(A)|} \int_{\mathbb{R}} f(y)\varphi(T^{-1}y)
				\\
				= & \frac{1}{|det(A)|} \langle f, \varphi \circ T^{-1}\rangle.
				\end{aligned}
				$$
				
				Supondo que $\varphi = \delta$ teremos
				$$
				\begin{aligned}
				\langle f \circ T, \delta \rangle = & \int_{\mathbb{R}} (f \circ T)(x)\delta(x)
				= (f \circ T)(0) = f(b).
				\\
				\end{aligned}
				$$
			
			\end{enumerate}
		
		\item \textbf{Solução:} Definiremos $K = supp(\varphi)$ para todos os itens.
			\begin{enumerate}
				\item Supondo que $\varphi \in \mathcal{S}(\mathbb{R}^{n})$, então para mostrar que $\varphi \in L^{2}(\mathbb{R}^{n})$ basta mostrarmos que $\int_{\mathbb{R}^{n}} |\varphi(x)|^{2} < \infty$. Com efeito: note que $\varphi(x)^{2} \neq 0 \iff \varphi(x) \neq 0 $, o que implica em $supp(\varphi^{2}) = supp(\varphi)$, como esse último é um compacto então $supp(\varphi^{2})$ é um compacto, consequentemente:
				$$
				\int_{\mathbb{R}^{n}} |\varphi(x)|^{2} = \int_{\mathbb{R}^{n}} \varphi(x)^{2} = \int_{ K} \underbrace{ \varphi(x)^{2} }_{<\infty} < \infty,
				$$
				sendo que a limitação vem do fato que de $\varphi^{2}$ é limitada no suporte compacto $K$, logo $(\int_{\mathbb{R}^{n}} |\varphi(x)|^{2})^{1/2} < \infty$. Portanto $\varphi \in L^{2}(\mathbb{R}^{n})$.
				
				\item Vamos demonstrar que, em certas condições a distribuição $Tf: \mathcal{S}(\mathbb{R}^{n}) \to \mathbb{R}$ dada por $\langle Tf, \varphi \rangle$ conicidirá com o produto interno em $L^{2}(\mathbb{R}^{n})$ dado por $\langle .,.\rangle_{2} : L^{2}(\mathbb{R}^{n}) \to L^{2}(\mathbb{R}^{n})$ dado por $\langle f, g \rangle_{2} = \int_{\mathbb{R}^{n}} f(x)g(x)$. Por definição temos:
				$$
				\begin{aligned}
				\langle Tf, \varphi \rangle = & \int_{\mathbb{R}^{n}} (Tf)(x)\varphi(x) = \int_{K} (Tf)(x)\varphi(x)
				\\
				= & \int_{\mathbb{R}^{n}} \underbrace{ (Tf)(x) }_{\in L^{2}} \underbrace{ \varphi(x) }_{<\infty} \leq ||\varphi||_{\infty} \int_{\mathbb{R}^{n}} \underbrace{ (Tf)(x) }_{\infty}
				\\
				< & \infty,
				\end{aligned}
				$$
				portanto  a distribuição $\langle Tf, \varphi \rangle < \infty$ esta bem-definida. Com isso, pela própria construção anterior, podemos escrever:
				$$
				\langle Tf, \varphi \rangle = \int_{K} (Tf)(x)\varphi(x) = \int_{\mathbb{R}^{n}} (Tf)(x)\varphi(x)  = \langle Tf, \varphi \rangle_{2}.
				$$
				Vejamos agora que o operador $T$ definido anteriormente é limitado, pois 
				$$
				||T|| = \sup_{f \in L^{2}}||Tf||_{2} < \infty,
				$$
				já que $Tf \in L^{2}(\mathbb{R}^{n})$, isto é $Tf$ é de quadrado integrável para qualquer $f \in L^{2}(\mathbb{R}^{n})$, portanto $T$ é limitado.  Como $T$ é um operador linear (pela hipótese) e limitado, então existe um único operador $T^{*} : L^{2}(\mathbb{R}^{n}) \to L^{2}(\mathbb{R}^{n})$ linear e limitado tal que $\langle Tf, \varphi \rangle_{2} = \langle f, T^{*}\varphi \rangle_{2}$. Por fim, vejamos que:
				$$
				\begin{aligned}
				\langle f, T^{*}\varphi \rangle = & \int_{\mathbb{R}^{n}} f(x)(T^{*}\varphi)(x) = \int_{K} \underbrace{ f(x) }_{\in L^{2}_{loc}} (T^{*}\varphi)(x)
				\\
				\leq & \sup_{x \in K}|f(x)| \int_{K} (T^{*}\varphi)(x) 	\leq \sup_{x \in K}|f(x)| \int_{K} \sup_{\varphi \in L^{2}} ||T^{*}\varphi||
				\\
				\leq & \sup_{x \in K}|f(x)| \sup_{\varphi \in L^{2}} ||T^{*}\varphi|| \int_{K} 1 
				\\
				= & \underbrace{ ||f||_{\infty} }_{<\infty} \underbrace{ ||T^{*}|| }_{< \infty} \int_{K} 1 
				\\
				< & \infty,
				\end{aligned}
				$$
				portanto o operador $T^{*}$ esta bem-definido e podemos escrever $\langle Tf, \varphi \rangle = \langle f, T^{*}\varphi \rangle $, como desejávamos.
			\end{enumerate}		
		\item \textbf{Solução:} Definamos a distribuição $\ln|x| \in \mathcal{S}'(\mathbb{R})$ por:
		$$
		f(x) = 
		\left\{
			\begin{array}{cc}
			\ln x & , x\neq 0 \\
			0 & , x =0.
			\end{array}		
		\right.
		$$
		Dados duas distribuições $h, g \in \mathcal{S}'(\mathbb{R})$, dizemos que $h = g$ se $\forall \phi \in \mathcal{S}(\mathbb{R})$ tem-se $\int_{\mathbb{R}}h(x)\phi(x) = \int_{\mathbb{R}}g(x)\phi(x)$. Vamos mostrar que $f = VP(\frac{1}{x})$. Primeiramente, podemos reescrever o valor principal como:
		$$
		\begin{aligned}
			VP(\frac{1}{x})(\phi) = & \lim_{\epsilon \to 0^{+}} \int_{|x|>\epsilon} \frac{1}{x}\phi(x) = \lim_{\epsilon \to 0} \Big( \int_{(-\infty, \epsilon)} \frac{1}{x}\phi(x) + \int_{(\epsilon, \infty)}  \frac{1}{x}\phi(x) \Big)
			\\
			= & \lim_{\epsilon \to 0} \Big( \int_{(\epsilon, \infty)} \underbrace{ \frac{1}{-y}\phi(-y) }_{x = -y} + \int_{(\epsilon, \infty)} \frac{1}{x}\phi(x) \Big)
			\\
			= & \lim_{\epsilon \to 0} \int_{(\epsilon, \infty)}  \frac{\phi(y)-\phi(-y)}{y}
		\end{aligned}
	 	$$
	 	Analogamente, definindo $Dln(x) \in \mathcal{S}'(\mathbb{R})$ tal que $Dln(x) = \frac{d}{dx}ln|x|$:
		$$
		\begin{aligned}
			Dln(x)(\phi) = & \lim_{\epsilon \to 0^{+}} \int_{|x|>\epsilon} \frac{d}{dx}\ln|x| \phi(x) = \lim_{\epsilon \to 0} \Big( \int_{(-\infty, \epsilon)} \frac{d}{dx}\ln|x| \phi(x) + \int_{(\epsilon, \infty)} \frac{d}{dx}\ln|x| \phi(x) \Big)
			\\
			= & \lim_{\epsilon \to 0} \Big( \int_{(-\infty, \epsilon)} \frac{d}{dx}\ln(-x) \phi(x) + \int_{(\epsilon, \infty)} \frac{d}{dx}\ln (x) \phi(x) \Big)
			\\
			= & \lim_{\epsilon \to 0} \Big( \int_{(-\infty, \epsilon)} \frac{1}{x} \phi(x) + \int_{(\epsilon, \infty)} \frac{1}{x} \phi(x) \Big)
			\\
			= & \lim_{\epsilon \to 0} \Big( \int_{(\epsilon, \infty)} \underbrace{ \frac{1}{-y} \phi(-y) }_{x=-y} + \int_{(\epsilon, \infty)} \frac{1}{x} \phi(x) \Big)
			\\
			= & \lim_{\epsilon \to 0} \int_{(\epsilon, \infty)}  \frac{\phi(y)-\phi(-y)}{y}
			\\
			= & VP(\frac{1}{x})(\phi),
		\end{aligned}
		$$
		logo, como $\phi$ é arbitrária temos $Dln(x) = VP(\frac{1}{x})$ (no sentido distribucional).  
		
		Vamos agora mostrar que $u \in \mathcal{S}'(\mathbb{R})$ tal que $u(x) = c_{1} + c_{2}\theta(x) + \ln|x|$ é uma solução fraca para $xu'(x) = 1$. Com efeito, tomando $\phi \in \mathcal{S}(\mathbb{R})$, mas antes provemos que $\theta' = \delta \in \mathcal{S}'(\mathbb{R})$. Pois bem, sabemos do primeiro exercício que $\langle \theta',g \rangle = - \langle \theta,g' \rangle, \forall g \in \mathcal{S}(\mathbb{R})$, então:
		$$
		\begin{aligned}
		\langle \theta',g \rangle = & - \langle \theta,g' \rangle = -\int_{\mathbb{R}} \theta(x)g'(x)
		\\
		= & -\int_{x \geq 0} \underbrace{ \theta(x) }_{=1} g'(x)= -\int_{x \geq 0} g'(x)
		\\
		= & -g(x) \Big|^{\infty}_{0} = \underbrace{ \lim_{x \to \infty}-g(x) }_{supp. compac.} + g(0)
		\\
		= & g(0)
		\\
		= & \int_{\mathbb{R}} \delta(x)g(x),
		\end{aligned}
		$$
		logo, $\langle \theta',g \rangle = \int_{\mathbb{R}} \theta'(x)g(x) = \int_{\mathbb{R}} \delta(x)g(x)$, portanto $\theta' = \delta$ (no sentido distribucional).
		
		Partindo para a demostração de que $xu'(x) = 1$:
		$$
		\begin{aligned}
		\int_{\mathbb{R}^{n}} xu'(x)\phi(x) = & \int_{\mathbb{R}^{n}} x(c_{1} + c_{2}\theta(x) + \ln|x|)'\phi(x)
		\\
		= & \int_{\mathbb{R}^{n}} x(0 + c_{2} \underbrace{ \theta'(x) }_{\delta(x)} + \underbrace{ (\ln|x|)' }_{VP(\frac{1}{x}))} ) \phi(x)
		\\
		= & \int_{\mathbb{R}^{n}} x \delta(x)\phi(x) + \int_{\mathbb{R}^{n}}  xVP(\frac{1}{x}))\phi(x)
		\\
		= & \underbrace{ x \phi(x)\Big|_{x=0} }_{=0} + \int_{\mathbb{R}^{n}}  x\frac{1}{x}\phi(x)
		\\
		= & \int_{\mathbb{R}^{n}} \phi(x),
		\end{aligned}
		$$
		logo, $xu'(x) = 1$ (no sentido distribucional), como desejávamos.
			
	\end{enumerate}
	
	
\end{document}