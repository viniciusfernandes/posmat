\documentclass{article}
\usepackage{graphicx}
\usepackage{indentfirst}
\usepackage[utf8]{inputenc}
\usepackage{amssymb}
\usepackage{enumitem}
\usepackage{color}
\usepackage[fleqn]{amsmath}
\usepackage[a4paper, margin=0.5in]{geometry}
\begin{document}
	
	\title{Lista 7 para Entregar}
	\author{Vinicius Fernandes}
	
	\maketitle
	
	\begin{enumerate}
		
		\item \textbf{Solução:} Vamos usar os seguintes fatos: se $g \in \mathcal{S}(\mathbb{R}^{n})$ teremos que $x^{\alpha} D^{|\beta|}g \in \mathcal{S}(\mathbb{R}^{n})$, de modo que valem os seguintes limites $\lim_{|x| \to \infty} g(x) = 0$ e $\lim_{|x| \to \infty} x^{\alpha} D^{|\beta|}g(x) = 0$.
		
		Estudemos o caso em que $n = 1$, ou seja, a reta real, então sejam $f, g \in \mathcal{S}(\mathbb{R})$:
		$$
		\begin{aligned}
		(fg)'(x) = & f'(x)g(x) + f'(x)g'(x) 
		\\ 
		\int_{\mathbb{R}} (fg)'(x) = & \int_{\mathbb{R}} f'(x)g(x) + \int_{\mathbb{R}} f(x)g'(x) 
		\\ 
		\underbrace{ (fg)(y)\Big|^{\infty}_{-\infty} }_{\to 0} = & \underbrace{ \int_{\mathbb{R}} f'(x)g(x) }_{\langle f',g \rangle} + \underbrace{ \int_{\mathbb{R}} f(x)g'(x) }_{\langle f,g' \rangle}
		\\
		\langle f',g \rangle = & - \langle f,g' \rangle.
		\end{aligned}
		$$
		Realizando novamente esse procedimento:
		$$
		\begin{aligned}
		(fg')'(x) = & f'(x)g'(x) + f(x)g''(x) 
		\\ 
		\int_{\mathbb{R}} (fg')'(x) = & \int_{\mathbb{R}} f'(x)g'(x) + \int_{\mathbb{R}} f(x)g''(x) 
		\\ 
		\underbrace{ (fg')(y)\Big|^{\infty}_{-\infty} }_{\to 0} = & \underbrace{ \int_{\mathbb{R}} f'(x)g'(x) }_{\langle f',g' \rangle} + \underbrace{ \int_{\mathbb{R}} f(x)g''(x) }_{\langle f,g'' \rangle}
		\\
		\langle f',g \rangle = & - \langle f,g' \rangle.
		\end{aligned}
		$$
		
		\item \textbf{Solução:} Por simplicidade, vamos definir a notação $C^{0} = C^{0}(\mathbb{R}^{n})$, e tomaremos $f, g \in C^{0}$ e $\lambda \in \mathbb{R}$.
			\begin{enumerate}
				\item Sabemos que $C^{0}$ é um espaço vetorial, portanto $f + \lambda g \in C^{0}$. Vamos mostrar que a linearidade do operador $\tau_{a}$.
				$$
				\begin{aligned}
				(\tau_{a}(f+\lambda g))(x) = &  (f+\lambda g)(x+a) \\
				= & f(x+a) +(\lambda g)(x+a)
				\\
				= & f(x+a) +\lambda g(x+a)
				\\
				= & (\tau_{a}f)(x) +\lambda (\tau_{a}g)(x)
				\\
				\therefore & \tau_{a}(f+\lambda g) = \tau_{a}(f)+\lambda \tau_{a}(g),
				\end{aligned}
				$$
				ou seja, $\tau_{a}$ é um operador linear.
				
				\item Notemos que $(\tau_{a} \circ \tau_{-a})(f)(x) = (\tau_{a})f(x-a) = f(x- a+a) = f(x)$, e também $(\tau_{-a} \circ \tau_{a})(f)(x) = (\tau_{a})f(x+a) = f(x+a-a) = f(x)$, portanto $\tau_{a} \circ \tau_{-a} = \tau_{-a} \circ \tau_{a} = Id \Rightarrow \tau_{-a} = \tau^{-1}_{a}$.
				
				\item Seja $\varphi \in \mathcal{S}(\mathbb{R}^{n})$, então:
				$$
				\begin{aligned}
				\langle \tau_{a}f, \varphi \rangle = & \int_{\mathbb{R}} (\tau_{a}f)(x)\varphi(x)
				= \int_{\mathbb{R}} f(\underbrace{ x+a }_{y = x+a}) \varphi(x)
				\\
				= & \int_{\mathbb{R}} f(y)\varphi(y-a)
				=  \int_{\mathbb{R}} f(y) (\tau_{-a}\varphi)(y) 
				\\
				= & \langle f, \tau_{-a}\varphi \rangle.
				\end{aligned}
				$$
				
				\item Primeiro, vejamos que e o operador $T$ é invertível. É injetor pois tomando $a, b \in \mathbb{R}^{n}$ teremos $Ta- Tb = Aa - Ab = A(a-b) = 0 \iff a=b$, pois $A$ é injetora. Além disso, para qualquer $y \in \mathbb{R}^{n}$ existe $x \in \mathbb{R}^{n}$ tal que tomando $x = A^{-1}(y -b)$ teremos $Tx = AA^{-1}(y -b) + b = y$, portanto é sobrejetora. Fazendo que $T^{-1}(x) = A^{-1}(x - b)$ teremos $TT^{-1}(x) = AA^{-1}(x - b) + b = x$, e também $T^{-1}T(x) = T^{-1}(Ax + b) = A^{-1}(Ax+b - b) = x$, portanto $TT^{-1} = T^{-1}T = Id$ e $T^{-1}$ é a inversa de $T$.
				
				Usaremos o seguinte resultado: tomando o diferencial do operator $T$ teremos $D_{x}(T): \mathbb{R}^{n} \to \mathbb{R}^{n}$ tal que $D_{x}(T)(v) = D_{x}(Ax+b)(v) = Av$, portanto $D_{x}(T) = A$ e $det(D_{x}(T)) = det(A) \neq 0 \; \forall x \in \mathbb{R}^{n}$. Além disso, realizando a mudança de coordenadas $y = Ax+b$ teremos $dy(x) = |det(D_{x}(T))|dx$ ou ainda $|det(D_{x}(T))|^{-1} dy(x) = dx$. Calculando:
				$$
				\begin{aligned}
				\langle f \circ T, \varphi \rangle = & \int_{\mathbb{R}} (f \circ T)(x)\varphi(x)
				= \int_{\mathbb{R}} f(\underbrace{ Ax+b }_{y = Ax+b})\varphi(x)
				\\
				= & \int_{\mathbb{R}} f(y)\varphi(\underbrace{ A^{-1}(y - b)}_{T^{-1}y} ) \Big| det(D_{x}(T)) \Big|^{-1}
				\\
				= & \int_{\mathbb{R}} f(y)\varphi(T^{-1}y)  |det(A)|^{-1}
				\\
				= & \frac{1}{|det(A)|} \int_{\mathbb{R}} f(y)\varphi(T^{-1}y)
				\\
				= & \frac{1}{|det(A)|} \langle f, \varphi \circ T^{-1}\rangle.
				\end{aligned}
				$$
				
				Supondo que $\varphi = \delta$ teremos
				$$
				\begin{aligned}
				\langle f \circ T, \delta \rangle = & \int_{\mathbb{R}} (f \circ T)(x)\delta(x)
				= (f \circ T)(0) = f(b).
				\\
				\end{aligned}
				$$
				
			\end{enumerate}
		
		\item \textbf{Solução:} Definiremos $K = supp(\varphi)$ para todos os itens.
			\begin{enumerate}
				\item Supondo que $\varphi \in \mathcal{S}(\mathbb{R}^{n})$, então para mostrar que $\varphi \in L^{2}(\mathbb{R}^{n})$ basta mostrarmos que $\int_{\mathbb{R}^{n}} |\varphi(x)|^{2} < \infty$. Com efeito: note que $\varphi(x)^{2} \neq 0 \iff \varphi(x) \neq 0 $, o que implica em $supp(\varphi^{2}) = supp(\varphi)$, como esse último é um compacto então $supp(\varphi^{2})$ é um compacto, consequentemente:
				$$
				\int_{\mathbb{R}^{n}} |\varphi(x)|^{2} = \int_{\mathbb{R}^{n}} \varphi(x)^{2} = \int_{ K} \underbrace{ \varphi(x)^{2} }_{<\infty} < \infty,
				$$
				sendo que a limitação vem do fato que de $\varphi^{2}$ é limitada no suporte compacto $K$, logo $(\int_{\mathbb{R}^{n}} |\varphi(x)|^{2})^{1/2} < \infty$. Portanto $\varphi \in L^{2}(\mathbb{R}^{n})$.
				
				\item Vamos demonstrar que, em certas condições a distribuição $Tf: \mathcal{S}(\mathbb{R}^{n}) \to \mathbb{R}$ dada por $\langle Tf, \varphi \rangle$ conicidirá com o produto interno em $L^{2}(\mathbb{R}^{n})$ dado por $\langle .,.\rangle_{2} : L^{2}(\mathbb{R}^{n}) \to L^{2}(\mathbb{R}^{n})$ dado por $\langle f, g \rangle_{2} = \int_{\mathbb{R}^{n}} f(x)g(x)$. Por definição temos:
				$$
				\begin{aligned}
				\langle Tf, \varphi \rangle = & \int_{\mathbb{R}^{n}} (Tf)(x)\varphi(x) = \int_{K} (Tf)(x)\varphi(x)
				\\
				= & \int_{\mathbb{R}^{n}} \underbrace{ (Tf)(x) }_{\in L^{2}} \underbrace{ \varphi(x) }_{<\infty} \leq ||\varphi||_{\infty} \int_{\mathbb{R}^{n}} \underbrace{ (Tf)(x) }_{\infty}
				\\
				< & \infty,
				\end{aligned}
				$$
				portanto  a distribuição $\langle Tf, \varphi \rangle < \infty$ esta bem-definida. Com isso, pela própria construção anterior, podemos escrever:
				$$
				\langle Tf, \varphi \rangle = \int_{K} (Tf)(x)\varphi(x) = \int_{\mathbb{R}^{n}} (Tf)(x)\varphi(x)  = \langle Tf, \varphi \rangle_{2}.
				$$
				Vejamos agora que o operador $T$ definido anteriormente é limitado, pois 
				$$
				||T|| = \sup_{f \in L^{2}}||Tf||_{2} < \infty,
				$$
				já que $Tf \in L^{2}(\mathbb{R}^{n})$, isto é $Tf$ é de quadrado integrável para qualquer $f \in L^{2}(\mathbb{R}^{n})$, portanto $T$ é limitado.  Como $T$ é um operador linear (pela hipótese) e limitado, então existe um único operador $T^{*} : L^{2}(\mathbb{R}^{n}) \to L^{2}(\mathbb{R}^{n})$ linear e limitado tal que $\langle Tf, \varphi \rangle_{2} = \langle f, T^{*}\varphi \rangle_{2}$. Por fim, vejamos que:
				$$
				\begin{aligned}
				\langle f, T^{*}\varphi \rangle = & \int_{\mathbb{R}^{n}} f(x)(T^{*}\varphi)(x) = \int_{K} \underbrace{ f(x) }_{\in L^{2}_{loc}} (T^{*}\varphi)(x)
				\\
				\leq & \sup_{x \in K}|f(x)| \int_{K} (T^{*}\varphi)(x) 	\leq \sup_{x \in K}|f(x)| \int_{K} \sup_{\varphi \in L^{2}} ||T^{*}\varphi||
				\\
				\leq & \sup_{x \in K}|f(x)| \sup_{\varphi \in L^{2}} ||T^{*}\varphi|| \int_{K} 1 
				\\
				= & \underbrace{ ||f||_{\infty} }_{<\infty} \underbrace{ ||T^{*}|| }_{< \infty} \int_{K} 1 
				\\
				< & \infty,
				\end{aligned}
				$$
				portanto o operador $T^{*}$ esta bem-definido e podemos escrever $\langle Tf, \varphi \rangle = \langle f, T^{*}\varphi \rangle $, como desejávamos.
			\end{enumerate}		
		\item \textbf{Solução:} Definamos a distribuição $f \in \mathcal{S}'(\mathbb{R})$ por:
		$$
		f(x) = 
		\left\{
			\begin{array}{cc}
			1/|x| & , x\neq 0 \\
			0 & , x =0.
			\end{array}		
		\right.
		$$
		Dados duas distribuições $f, g \in \mathcal{S}'(\mathbb{R})$, dizemos que $f = g$ se $\forall \phi \in \mathcal{S}(\mathbb{R})$ tem-se $\int_{\mathbb{R}}f(x)\phi(x) = \int_{\mathbb{R}}g(x)\phi(x)$. Vamos mostrar que $f = VP(\frac{1}{x})$. Sejam $f$ definida anteriormente e $\phi \in \mathcal{S}(\mathbb{R})$, então calculando:
		$$
		\begin{aligned}
		\int_{\mathbb{R}} f(x)\phi(x) = \int_{\mathbb{R}} \frac{1}{|x|}\phi(x) = \int_{\mathbb{R}} VP\Big( \frac{1}{x} \Big)\phi(x), 
		\end{aligned}
		$$
		como $\phi$ é arbitrária, então temos que $f = VP(\frac{1}{x})$. Contudo, podemos ver que, no sentido clássico, temos a relação $f(x) = \frac{d}{dx}\ln|x| \; \forall x \in \mathbb{R}\backslash\{0\}$, assim 
		$$
		\begin{aligned}
			\int_{\mathbb{R}} f(x)\phi(x) = \int_{\mathbb{R}} \frac{1}{|x|}\phi(x) = \int_{\mathbb{R}\backslash\{0\}} \frac{1}{|x|}\phi(x) = \int_{\mathbb{R}\backslash\{0\}} \frac{d}{dx}\ln|x|\phi(x) , 
		\end{aligned}
		$$
		note que podemos escrever $f = \frac{d}{dx}\ln|x|$ pois a ultima igualdade difere apenas um ponto de medida nula $x=0$ e $f(0)$, o que implica $\frac{d}{dx}\ln|x| = VP(\frac{1}{x})$.
		
	\end{enumerate}
	
	
\end{document}