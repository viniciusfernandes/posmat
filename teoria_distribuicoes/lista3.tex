\documentclass{article}
\usepackage{graphicx}
\usepackage{indentfirst}
\usepackage[utf8]{inputenc}
\usepackage{amssymb}
\usepackage{enumitem}
\usepackage{color}
\usepackage[a4paper, margin=0.5in]{geometry}
\begin{document}
	
	\title{Lista 2 para Entregar}
	\author{Vinicius Fernades}
	
	\maketitle
	
	\begin{enumerate}
		\item[3.] \textbf{Solução:}
		\begin{enumerate}
			\item Devemos mostrar que $D\circ I = Id$. Pois bem: segundo o teorema fundamental do cálculo, para toda função $f  \in C^{0}$ temos que $\exists F \in C^{0}; $ tal que $ F(x) = \int_{[0,x]}f(t)$ além disso $F'(x) = f(x) \; \forall x \in \mathbb{R}$ e $F$ é chamada de primitiva de $f$. Assim podemos escrever $I(f)(x) = F(x)$, com isso $(D\circ I)(f)(x) = D(I(f))(x) = D(F)(x) = F'(x) = f(x) \Rightarrow D\circ I = Id$.
			
			\item Devemos mostrar que $I \circ D \neq Id$. Pois bem, tomando $f \in C^{\infty}$ e pela definição temos $(I \circ D) (f)(x) = I(f')(x) = \int_{[0,x]}f'(t) = f(x)-f(0) \neq Id(f)(x)$.
			
			\item Determinando agora:
				\begin{enumerate}
					\item O núcleo do operador $D$ é dado por $N(D)=\{f \in C^{\infty}: D(f)=0 \}$, assim $\forall x  \in \mathbb{R}$ temos que $D(f)(x) = f'(x) =0 \Rightarrow f = f_{0} \in C^{\infty}$, pois essa é uma caracterização de função constante. Portanto $N(D)$ é o conjunto das funções constantes.
					
					\item O núcleo do operador $I$ é dado por $N(I)=\{f \in C^{\infty}: I(f)=0 \}$, assim $\forall x  \in \mathbb{R}$ temos que $I(f)(x) = \int_{[0,x]}f(t) =0$ e como essa condição deve valer para todos $x \in \mathbb{R}$, então $f = 0 \in C^{\infty}$. Portanto $N(I) = \{0\}$.
					
					\item O núcleo do operador $I$ é dado por $N(I\circ D)=\{f \in C^{\infty}: (I\circ D)(f)=0 \}$, assim $\forall x  \in \mathbb{R}$ temos que $ (I\circ D)(f)(x) = \int_{[0,x]}f'(t) = f(x) - f(0) = 0 \Rightarrow f(x) = f(0) \in \mathbb{R}$. Como essa condição deve valer para todos $x \in \mathbb{R}$, então $f$ função constante. Portanto $N(D)$ é o conjunto das funções constantes.
				\end{enumerate}
		\end{enumerate}
		
		\item[4.] \textbf{Solução:} Observação: Aqui vamos utilizar a notação $f$ em vez de $[f] \in W^{1,p}$ para as classes de equivalência, e também omitiremos a notação do domínio $\Omega$, por simplicidade. Com isso:
			\begin{enumerate}
				\item Devemos mostrar que $||.||_{W^{1,p}}$ satisfaz os axiomas de norma. Sejam $f, g \in W^{1,p}$, então:
				\begin{enumerate}
					\item Sabemos que  $||f||_{L^{p}}, ||f'||_{L^{p}} \geq 0$ pois $||.||_{L^{p}}$ é uma norma, como já foi demonstrado em exercícios anteiores, então $||f||_{W^{1,p}} = ||f||_{L^{p}} + ||f'||_{L^{p}} \geq 0$.
					
					\item Sabemos que $||f||_{L^{p}}, ||f'||_{L^{p}} = 0 \iff f=0 \;\ e \; f'=0$, pois $||.||_{L^{p}}$ é positiva definida, como já foi demonstrado em exercícios anteiores ,então $||f||_{W^{1,p}} = 0 \iff f = 0 e f' =0$, consequentemente $f$ deve ser constante e identicamente nula.
					
					\item Verificando a desigualdade triangular: 
					$$
					||f + g||_{W^{1,p}} =  ||f+g||_{L^{p}} + ||(f + g)'||_{L^{p}} \leq
					$$
					$$ \leq ||f||_{L^{p}}+||g||_{L^{p}}+||f'||_{L^{p}}+||g'||_{L^{p}} =  
					$$
					$$
					=\underbrace{||f||_{L^{p}}+||f'||_{L^{p}}} + \underbrace{||g||_{L^{p}}+||g'||_{L^{p}}}
					$$
					$$
					||f ||_{W^{1,p}}+|| g||_{W^{1,p}}.
					$$ 
					Assim, $||f + g||_{W^{1,p}} \leq ||f ||_{W^{1,p}}+|| g||_{W^{1,p}}$
					
				\end{enumerate}
				Portanto, $||. ||_{W^{1,p}}: W^{1,p} \to \mathbb{R}$ é uma norma.
				
				\item Sabemos que uma norma é proveniente de um produto interno se, e somente se, satisfaz a lei do paralelogramo, consequentemente, a fórmula de polarização definirá produto interno nesse espaço. Com efeito: como a lei do paralelogramo deve valer para quaisquer classes $f, g \in W^{1,p}$, então deve valer para o caso particular em que $f = \chi_{[a,b]}$ e $g = \chi_{[c,d]}$ como sendo as classes das funções características com suporte nos intervalos definidos, e tomemos $[a,b] \cap [c,d] = \emptyset$. Calculando os seguintes valores para serem aplicados na lei do paralelogramo: 
				$$
				||f||_{L^{p}}^{2} = \Big( \int_{\mathbb{R}} |\chi_{[a,b]}(x)|^{p} \Big)^{2/p} = (b-a)^{2},
				$$
				$$  
				||g||_{L^{p}}^{2} = \Big( \int_{\mathbb{R}} |\chi_{[c,d]}(x)|^{p} \Big)^{2/p} = (d-c)^{2},
				$$
				$$
				||f +g||_{L^{p}}^{2} = \Big (\int_{\mathbb{R}}|f(x)+g(x)|^{p} \Big)^{2/p} =((b-a)^{p}+(d-c)^{p})^{2/p},
				$$
				$$
				||f -g||_{L^{p}}^{2} = \Big (\int_{\mathbb{R}}|f(x)-g(x)|^{p} \Big)^{2/p} =((b-a)^{p}-(d-c)^{p})^{2/p},
				$$
				
				que aplicadas na lei do paralelogramo, teremos
				$$
				||f +g||_{L^{p}}^{2}+||f -g||_{L^{p}}^{2} =2(||f||_{L^{p}}^{2}+||g||_{L^{p}}^{2}),
				$$
				o que se segue
				$$
				((b-a)^{p}+(d-c)^{p})^{2/p}+((b-a)^{p}-(d-c)^{p})^{2/p} = 2((b-a)^{2}+(d-c)^{2}).
				$$
				Note que no caso em que $p=2$, teremos a condição de igualdade satisfeita, ou seja, realizaremos a lei do paralelogramo. Portanto, $||. ||_{W^{1,2}} : W^{1,2} \to \mathbb{R}$ é uma norma proveniente de um produto interno quando $p=2$. Por fim, poderíamos ter outros valores para $p$, mas como essa última igualdade deve valer para quaisquer intervalos suporte $[a, b]$ e $[c,d]$ da função característica, então teremos que $p=2$ é o único valor realizando essa condição.
				
				Finalmente, o resultado anterior nos permite afirmar que a fómula da polaridade define um produto interno em $W^{1,2}$, assim teremos o espaço com produto interno $(W^{1,2}, <.,.>)$, onde:
				$$
				<f,g> := T(f,g) = (||f +g||_{L^{2}}^{2} - ||f -g||_{L^{2}}^{2})/4.
				$$
			\end{enumerate}
		
		\item[5.] \textbf{Solução:}
			\begin{enumerate}
				\item Devemos mostrar que vale os axiomas de produto interno para $T: \mathcal{P}_{2}(\mathbb{R}) \times \mathcal{P}_{2}(\mathbb{R}) \to \mathbb{R}$, para isso tomemos em diante $f,g,h \in \mathcal{P}_{2}(\mathbb{R}), \lambda \in \mathbb{R}$:
					\begin{enumerate}
						\item \textit{(Comutatividade)} Assim: 
						$$
						T(f, g):= f(-1)g(-1) + f(0)g(0) + f(1)g(1) = g(-1)f(-1) + g(0)f(0) + g(1)f(1) = T(g, f).
						$$
						
						\item \textit{(Bilinearidade)} Temos a linearidade no primeiro parametro: 
						$$
						T(f+g, h) := (f+g)(-1)h(-1) +(f+g)(0)h(0)+(f+g)(1)h(1) =  
						$$
						$$
						= f(-1)h(-1)+g(-1)h(-1) +f(0)h(0)+g(0)h(0)+f(1)h(1)+g(1)h(1)=
						$$ 
						$$
						= \underbrace{f(-1)h(-1)f(0)h(0)+f(1)h(1)}+\underbrace{g(-1)h(-1) +g(0)h(0)+g(1)h(1)}=
						$$
						$$
						= T(f, h) + T(g,h).
						$$
						Analogamente, temos a linearidade no segundo parametro 
						$$
						T(f, h+g) = T(h+g, f) = T(h, f) + T(g, f) = T(h, f) + T(g, f).
						$$
						
						E por fim:
						$$
						T(f, \lambda g):= f(-1)(\lambda g)(-1) + f(0)(\lambda g)(0) + (\lambda g)(1)(\lambda g)(1) =
						$$
						$$
						=\lambda (f(-1)g(-1) + f(0)g(0) + g(1)g(1)) = \lambda T(f, g).
						$$
						E analogamente, 
						$$
						T(\lambda f, g) = T(g, \lambda f) = \lambda T(g,f) = \lambda T(f,g).
						$$
						
						\item \textit{(Positividade)} Assim:
						$$
						T(f,f) := (f(-1))^{2} + (f(0))^{2}+ (f(1))^{2}>0.
						$$
						Portanto $T: \mathcal{P}_{2}(\mathbb{R}) \times \mathcal{P}_{2}(\mathbb{R}) \to \mathbb{R}$ é um produto interno.
					\end{enumerate}
					
					\item Assumindo que $M = Span\{x^{2}+a\}$ com $a \in \mathbb{R}$, vamos determinar os elementos de $\mathcal{P}_{2}(\mathbb{R})$ ortogonais a $M$, assim, tomando $p \in \mathcal{P}_{2}(\mathbb{R})$ devemos ter $T(p, x^{2}+a)=0$. Note que temos $\mathcal{P}_{2}(\mathbb{R}) = Span\{1, x, x^{2}\}$, então basta determinar quais desses elementos da base são ortogonais a $M$:
					$$
					T(1, x^{2}+a) = (1+a)+(a)+(1+a)= 3a+2 \neq 0,
					$$
					$$
					T(x, x^{2}+a) = -(1+a)+0+(1+a)=0,
					$$
					$$
					T(x^{2}, x^{2}+a) = (1+a)+0+(1+a) \neq 0,
					$$
					logo, temos que o espaço gerado pelo elemento $x$ da base é ortogonal a $M$, isto é, $Span\{x\}\perp M$. Portanto, se $T(p, x^{2}+a) = 0$, então $p \in Span\{x\}$.
				
					
			\end{enumerate}
		
		
	\end{enumerate}
		
\end{document}