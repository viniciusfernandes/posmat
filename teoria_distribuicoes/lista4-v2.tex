\documentclass{article}
\usepackage{graphicx}
\usepackage{indentfirst}
\usepackage[utf8]{inputenc}
\usepackage{amssymb}
\usepackage{enumitem}
\usepackage{color}
\usepackage[fleqn]{amsmath}
\usepackage[a4paper, margin=0.5in]{geometry}
\begin{document}
	
	\title{Lista 4 para Entregar}
	\author{Vinicius Fernandes}
	
	\maketitle
	
	\begin{enumerate}
		
		\item[3.] \textbf{Solução:} Por brevidade, vamos omitir a notação do conjunto suporte $\Omega$, então escreveresmo $C^{\infty} := C^{\infty}(\Omega)$
		\begin{enumerate}
			\item Devemos mostrar que $(C^{\infty}, m)$ é uma álgebra comutativa e associativa, para isso deve satisfazer os axiomas de uma álgebra. Para isso, definamos $f, g, h \in C^{\infty}$ e $\lambda \in \mathbb{R}$:
			
			\begin{enumerate}
				\item \textit{(Comutatividade)} $m(f, g)(x) = (f.g)(x) = f(x)g(x) = g(x)f(x) = m(g,f)(x)$, portanto $m(f,g) = m(g,f)$ mostrando que é uma operação comutativa. 
				
				\item \textit{(Associatividade)} $m(m(f,g), h)(x) = (m(f, g).h)(x) = (f.g)(x)h(x) = f(x)g(x)h(x) = f(x).m(g,h)(x) = m(f, m(g,h))(x)$, portanto $m(m(f,g), h)=m(f, m(g,h))$, o que mostra a associatividade.
				
				\item \textit{(Bilinearidade)} $m(f, \lambda(g + h))(x) = (f.( \lambda(g + h)))(x)  = f(x).( \lambda(g + h)))(x) = f(x) \lambda(g(x) + h(x)) = \lambda f(x) g(x) + \lambda  f(x)h(x) = \lambda m(f, g)(x) + \lambda m(f,h)(x)$, assim $m(f, \lambda(g + h)) = \lambda m(f,g) + \lambda m(f,h)$, portanto é linear na segunda entrada. Como é comutativa, então $m(\lambda(g + h), f) = m(f, \lambda(g + h)) = \lambda m(f,g) + \lambda m(f,h) = \lambda m(g,f) + \lambda m(h,f)$, portanto também é linear no primeiro parâmetro e assim mostramos a bilinearidade.
				
			\end{enumerate}
			Portanto, $(C^{\infty}, m)$ é uma álgebra comutativa e associativa. 
		
			\item Sabemos da lista 1 que $C^{0}$ é um espaço vetorial, e como $C^{\infty}_{0} \subset C^{0}$, vamos mostrar que $C^{\infty}_{0}$ é um subespaço vetorial, portanto é um espaço vetorial. Devemos mostrar os axiomas de subespaço:
			
			\begin{enumerate}
				\item Claro que a função identicamente nula é diferenciável, portanto $0 \in C^{\infty}$. Agora, note que $supp (0) = \emptyset$, e como o conjunto vazio é um fechado e compacto, então $0 \in C^{\infty}_{0}$.
				
				\item Sabemos que o produto de uma função diferenciável por uma constante resulta em uma função diferenciável, isto é, $\forall g \in C^{\infty}$ temos que $\lambda g \in C^{\infty}$, além disso, temos que $supp(\lambda g) = \overline{\{x \in \Omega: \lambda g(x) \neq 0\}} = \overline{\{x \in \Omega: g(x) \neq 0\}} = supp(g)$, portanto $\lambda g \in C^{\infty}_{0}$
				
				\item Sabemos que a soma de duas funções diferenciáveis também é uma função diferenciável, isto é, $\forall f, g \in C^{\infty}$ temos $f + \lambda g \in C^{\infty}$. Agora vamos mostrar que $f + \lambda g \in C^{\infty}_{0}$.
				
				Definindo $I := \{x \in \Omega: f(x) = - \lambda g(x)\}$, veremos que:
				$$
				\begin{aligned}
				supp(f + \lambda g ) 
				= & \overline{\{x \in \Omega: f(x) \neq -\lambda g(x)\} } \\
				= & \overline{( \{x \in \Omega: f(x) \neq 0 \} \cup \{x \in \Omega: \lambda g(x) \neq 0\}) \backslash I } \\
				= & \underbrace{\overline{\{x \in \Omega: f(x) \neq 0 \}\backslash I}}_{\subseteq suppf(f)} \cup \underbrace{\overline{\{x \in \Omega: \lambda g(x) \neq 0\}\backslash I }}_{\subseteq supp(g)}, 
				\end{aligned}
				$$
				que é a união finita de conjuntos fechados, portanto é fechado. Como são subconjuntos de conjuntos limitados, então é limitado, portanto, é fechado e limitado em $\mathbb{R}^{n}$ (ou $\mathbb{C}^{n}$), então é um compacto.
			\end{enumerate}
			Conclusão, $C^{\infty}_{0}$ é um subespaço vetorial de $C^{0}$, portanto é um espaço vetorial.
			
			\item Devemos mostrar que $\forall f \in C^{\infty}$ e $g \in C^{\infty}_{0}$ teremos $m(g,f) \in C^{\infty}_{0}$. Por definição $m(g, f) = g.f$, assim vejamos que 
			$$
			\begin{aligned}
			supp(g.f) 
			= & \overline{ \{x \in \Omega: f(x) \neq 0 \; e \; g(x) \neq 0 \} } \\
			= & \underbrace{supp(f) \cap supp(g)}_{\subseteq supp(g)},
			\end{aligned}
			$$
			mostrando que é um subconjunto de um conjunto limitado, além disso, é a intersecção finita de dois conjuntos fechados, portanto é fechado e limitado, então é um compacto. Conclusão, o produto $m(g, f) = g.f \in C^{\infty}_{0}$. Mostramos então que $C^{\infty}_{0}$ é um ideal à direita, mas como $m$ é comutativa, então $C^{\infty}_{0}$ é um ideal à esquerda.
		\end{enumerate}
		
		\item[6.] \textbf{Solução:} Novamente vamos adotar uma notação compactada, por simplicidade, onde convencionaremos $L^{p} := L^{p}(\mathbb{R})$, onde $1 \leq p \leq \infty$ e também a identificação $f = [f] \in L^{p}$.
			\begin{enumerate}
				\item CANCELADO
				\item Definindo $f, g:\mathbb{R} \to \mathbb{R} $ tal que 
				$$
				f(x) = (1+x^{2})^{-1/2} \; e \;
				g(x) := \left\{
				\begin{array}{cc}
				2, & |x| \leq 1 \\
				1/|x|, & |x| > 1 \\
				\end{array}
				\right.
				$$
				E consequentemente, temos a estimativa $g(x) > f(x) \; \forall x \in \mathbb{R}$. Agora, mostrando que se a integral $\int_{\mathbb{R}}g(x)$ não existir, então a integral $\int_{\mathbb{R}}f(x)$ também não existirá, pois
				$$
				\int_{\mathbb{R}}g(x) > \int_{\mathbb{R}}f(x),
				$$
				com isso $f \notin L^{1}$. Com efeito:
				$$
				\begin{aligned}
				\int_{\mathbb{R}}g(x) 
				= & \int_{|x| \leq 1} 2 dx + \int_{|x| > 1} \frac{1}{|x|} dx \\
				= & \int_{[0,1]} 2 dx + \int_{[-1,0]} 2 dx + \int_{(-\infty, 1]}\frac{-1}{x}dx + \int_{[1, \infty)}\frac{1}{x}dx \\
				= & 2\int_{[0,1]} 2 dx +  2\int_{[1, \infty)}\frac{1}{x}dx \\
				= & 4 + 2 \lim_{x \to \infty} \ln(x) = +\infty,  
				\end{aligned}
				$$
				logo 
				$\int_{\mathbb{R}}g(x)$ não existe, e como $\int_{\mathbb{R}}g(x)> \int_{\mathbb{R}}f(x)$, consequentemente $f$ não é integrável, ou seja, $f \notin L^{1}$.
				
				Agora, resta-nos mostrar que $f \in L^{2}$, isto é $\int_{\mathbb{R}}f(x)^{2} <\infty$. Com efeito, definindo $y^{2} = 1+x^{2}$ teremos:
				$$
				\int_{\mathbb{R}}f(x) 
				= \int_{\mathbb{R}}\frac{1}{1+x^{2}}dx = \int_{\mathbb{R}}\frac{1}{y^{2}}dy = \frac{ -1}{y} \Big|^{\infty}_{-\infty} = 0,
				$$
				portanto $f^{2}$ é integrável e é de quadrado integrável, ou seja, $f^{2} \in L^{2}$.
				
				\item Primeiro devemos mostrar que $V= L^{1} \cap L^{2} \neq \emptyset$. De fato, tomando $\chi_{[a, b]}$ é claro que $\int_{\mathbb{R}} \chi_{[a,b]}(x) = 1 < \infty$, portanto $\chi_{[a, b]} \in L^{1}$. Por outro lado, fazendo $\int_{\mathbb{R}} (\chi_{[a,b]}(x))^{2} = \int_{\mathbb{R}} \chi_{[a,b]}(x) = 1 < \infty$, portanto $\chi_{[a, b]} \in L^{2}$, e consequentemente $f \in L^{1} \cap L^{2} = V$ e $V \neq \emptyset$.
				
				Já foi demonstrado em um exercício anterior da lista 2 que os espaços $L^{p}$ são espaços vetoriais para $1 \leq p \leq \infty$, assim vamos mostrar que $V \subset L^{2}$ é um subespaço vetorial, para isso basta mostrar os axiomas de subespaço:
					\begin{enumerate}
						\item Claro que $0 \in L^{1}$ e $0 \in L^{2}$, pois a classe da função identicamente nula é integrável e tem quadrado integrável, portanto $0 \in L^{1} \cap L^{2}$.
						
						\item Sejam as classes $f, g \in L^{1} \cap L^{2}$ e $\lambda \in \mathbb{R}$, então $f+\lambda g \in L^{1}$ e também $f+\lambda g \in L^{2}$ pois ambos são espaços vetoriais, portanto $f+\lambda g \in L^{1} \cap L^{2} = V$.
					\end{enumerate}
				Logo, $V = L^{1} \cap L^{2} \subset L^{2}$ é um subespaço vetorial, portanto é um espaço vetorial.
				
				\item Seja $\mathcal{F}:V \to V$, e vimos no item anterior que $V \subset L^{2}$, e como $L^{2}$ é um espaço de Hilbert, então vamos definir um produto interno em $L^{1}$ como sendo $<.,.>_{1} := <.,.>_{2}|_{L^{1}}$, portanto $(V, <.,.>_{1})$ é um espaço vetorial com produto interno, consequentemente, é um espaço normado. Para que o operador $\mathcal{F}:V \to V$ tenha um adjunto é necessário que ele seja limitado. Agora vamos mostrar isso, pois bem 
				$\forall f \in V$ temos:
				$$
				\sup_{\theta \in \mathbb{R}} |\mathcal{F}(f)(\theta)|
				= \frac{1}{\sqrt{2 \pi}}\sup_{\theta \in \mathbb{R}} \Big| \int_{\mathbb{R}}f(x)e^{-ix\theta} dx \Big| \leq \frac{1}{\sqrt{2 \pi}} \int_{\mathbb{R}}|f(x)| dx <\infty, 
				$$
				pois a função $f$ é integrável, logo,
				$$
				||\mathcal{F}|| 
				:= \sup_{||f||=1}||\mathcal{F}(f)|| = \sup_{||f||=1} \Big( \sup_{\theta \in \mathbb{R}} |\mathcal{F}(f)(\theta)| \Big) < \infty,
				$$
				pois já tinhamos a limitação pela primeira equação, portanto $\mathcal{F}$ é limitado. Determinando agora o adjunto $\mathcal{F}^{*} :  V \to V$ de modo que devemos satisfazer a relação $<\mathcal{F}(f), g>_{1} = <f, \mathcal{F}^{*}(g)>_{1}\; \forall f, g \in V$. Então:
				
				$$
				\begin{aligned}
				<\mathcal{F}(f), g>_{1} 
				= & \int_{\mathbb{R}} \mathcal{F}(f)(y)g(y)dy = \int_{\mathbb{R}} \int_{\mathbb{R}} f(x)e^{-iyx} dx g(y) dy  \\
				= & \int_{\mathbb{R}} \int_{\mathbb{R}}g(y)e^{-iyx}  dy f(x) dx = \int_{\mathbb{R}} \mathcal{F}(g)(x) f(x) dx \\
				= & <\mathcal{F}(g), f>_{1} \\
				= & <f, \mathcal{F}(g)>_{1},
				\end{aligned}
				$$
				portanto $\mathcal{F}^{*} = \mathcal{F}$ e $\mathcal{F}$ é um operador auto-adjunto dado por:
				$$
				\mathcal{F}^{*}(f)(\theta) = \frac{1}{\sqrt{2 \pi}} \int_{\mathbb{R}}f(x)e^{-ix\theta} dx
				$$
				\end{enumerate}
		
		\item[7.] \textbf{Solução:} 
			\begin{enumerate}
				\item 
				\begin{enumerate}
					\item Calculando os suportes:
					$$
					\begin{aligned}
					supp(f_{1}) = & \overline{ \{x \in \mathbb{R}: e^{-|x|} \neq 0\} } = \mathbb{R} \\ 
					supp(f_{2}) = & \overline{ \{x \in \mathbb{R}: \sin{|x|} \neq 0\} } = \overline{\mathbb{R} \backslash \pi \mathbb{Z} } = \mathbb{R}\\ 
					supp(f_{3}) = & \overline{ \{x \in \mathbb{R}: x \neq 0\} } = \overline{ \mathbb{R}\backslash \{0\} } = \mathbb{R} \\
					supp(k) = & \overline{ \{(x,y) \in \mathbb{R}^{2}: e^{-|x-y|} \neq 0\} } = \mathbb{R}^{2}
					\end{aligned}
					$$
					
					\item Calculando as integrais:
					\begin{enumerate}
						\item Calculando $I_{1}(x).$ 
						
						Caso $x \geq 0$ teremos:
						$$
						\begin{aligned}
						I_{1}(x) 
						= & \int_{\mathbb{R}}e^{-|y|}e^{-|x-y|} dy \\
						= & \int_{y \leq x }e^{-|y|}e^{-|x-y|} dy + \int_{y \geq x }e^{-|y|}e^{-|x-y|} dy \\
						= & \underbrace{ \int_{y \leq 0 }e^{-|y|}e^{(y-x)} dy + \int_{0 \leq y \leq x }e^{-|y|}e^{(y-x)} dy }_{ y \leq x} + \int_{y \geq x }e^{-|y|}e^{-(y-x)} dy \\
						= & e^{-x} \int_{y \leq 0 }e^{-|y|}e^{y} dy + e^{-x} \int_{0 \leq y \leq x }e^{-|y|}e^{y} dy + e^{x} \int_{y \geq x }e^{-|y|}e^{-y} dy \\
						= & e^{-x}\int_{y \leq 0}e^{2y} dy + e^{-x} \int_{0 \leq y \leq x } 1 dy  + e^{x} \int_{y \geq x}e^{-2y} dy \\
						= & e^{-x} \frac{e^{2y}}{2} \Big|^{0}_{-\infty} + e^{-x} y \Big|^{x}_{0} + e^{x} \frac{e^{-2y}}{-2} \Big|^{\infty}_{x} \\
						= & e^{-x} \Big(\frac{1}{2} + x \Big) + \frac{e^{-x}}{2} \\
						= & e^{-x} \Big( x + 1\Big) 
						\end{aligned}
						$$
						Caso $x < 0$ teremos:
						$$
						\begin{aligned}
						I_{1}(x) 
						= & \int_{\mathbb{R}}e^{-|y|}e^{-|x-y|} dy \\
						= & \int_{y \leq x }e^{-|y|}e^{-|x-y|} dy + \int_{y \geq x }e^{-|y|}e^{-|x-y|} dy \\
						= & \int_{y \leq x }e^{-|y|}e^{(y+x)} dy + \int_{x \leq y \leq 0 }e^{-|y|}e^{(y+x)} dy  + \int_{y \geq 0 }e^{-|y|}e^{-y + x} dy \\
						= & \int_{y \leq 0 }e^{-|y|}e^{(y+x)} dy + \int_{y \geq 0 }e^{-|y|}e^{-y +x} dy \\
						= & e^{x} \int_{y \leq 0 }e^{-|y|}e^{y} dy + e^{x}\int_{y \geq 0 }e^{-|y|}e^{-y} dy \\
						= & e^{x} \int_{y \leq 0 }e^{2y} dy + e^{x}\int_{y \geq 0 }e^{-2y} dy \\
						= & e^{x} \; 2 \int_{y \geq 0 }e^{-2y} dy \\
						= & e^{x} \; 2 \frac{e^{-2y}}{-2} \Big|^{\infty}_{0} \\
						= & e^{x}.
						\end{aligned}
						$$
						Portanto:
						$$
						I_{1}(x) = \left\{
						\begin{array}{cc}
						e^{-x} \Big( x + 1\Big) &, x \geq 0 \\
						e^{x}  &, x < 0
						\end{array}
						\right.
						$$
						
					\item Calculando $I_{2}(x).$
							
					Caso $x \geq 0:$
					$$
					\begin{aligned}
					I_{2}(x) 
					= & \int_{\mathbb{R}}\sin(|y|)e^{-|x-y|} dy \\
					= & \int_{y \leq x}\sin(|y|)e^{-|x-y|} dy + \int_{y \geq x}\sin(|y|)e^{-|x-y|} dy\\
					= & \underbrace{ \int_{y \leq 0 }\sin(|y|)e^{(y-x)} dy + \int_{0 \leq y \leq x }\sin(|y|)e^{(y-x)} dy }_{ y \leq x} + \int_{y \geq x }\sin(|y|)e^{-(y-x)} dy \\
					= & \int_{y \leq 0 } -\sin(y)e^{(y-x)} dy + \int_{0 \leq y \leq x }\sin(y)e^{(y-x)} dy + \int_{y \geq x }\sin(y)e^{-(y-x)} dy \\
					= & \Re \Big( \int_{y \leq 0 } \underbrace{ -\frac{e^{iy} - e^{-iy}}{2i} }_{-\sin(y)} e^{(y-x)}  dy + \int_{0 \leq y \leq x } \underbrace{ \frac{e^{iy} - e^{-iy}}{2i} }_{\sin(y)}e^{(y-x)} dy + \int_{y \geq x } \frac{e^{iy} - e^{-iy}}{2i} e^{-(y-x)} dy \Big)\\
					= & \Re \Big( \frac{e^{-x}}{2i} \Big( \int_{y \leq 0 } -e^{(1+i)y} + e^{(1-i)y}dy + \int_{0 \leq y \leq x } e^{(1+i)y} - e^{(1-i)y} dy \Big) + \frac{e^{x}}{2i} \int_{y \geq x } e^{(i- 1)y} - e^{-(i+1)y} dy \Big)\\
					= & \Re \Big( \frac{e^{-x}}{2i} \Big(  -\frac{e^{(1+i)y}}{1+i}\Big|^{0}_{-\infty} + \frac{e^{(1-i)y}}{1-i} \Big|^{0}_{-\infty} + \frac{e^{(1+i)y}}{1+i}\Big|^{x}_{0}  - \frac{e^{(1-i)y}}{1-i} \Big|^{x}_{0}  \Big) + \frac{e^{x}}{2i} \Big(  \frac{e^{(i-1)y}}{i-1}\Big|^{\infty}_{x}  - \frac{e^{-(1+i)y}}{-(1+i)} \Big|^{\infty}_{x} \Big) \Big)\\
					= & \Re \Big( \frac{e^{-x}}{2i} \Big( -\frac{2}{1+i} + \frac{2}{1-i} + \frac{e^{(1+i)x}}{1+i} - \frac{e^{(1-i)x}}{1-i} \Big) + \frac{e^{x}}{2i} \Big(  -\frac{e^{(i-1)x}}{i-1} - \frac{e^{-(1+i)x}}{(1+i)} \Big) \Big)\\
					= & \Re \Big( \frac{e^{-x}}{2i} \Big( 2i + i e^{x}(\sin(x) - \cos(x)) \Big) + \frac{e^{x}}{2i} \Big( -ie^{-x} (\sin(x)+\cos(x))
					\Big) \Big)\\
					= & \Re \Big( e^{-x} + \frac{1}{2}(\sin(x) - \cos(x)) - \frac{1}{2} (\sin(x)+\cos(x)) \Big)\\
					= & e^{-x} - \cos(x).
					\end{aligned}
					$$
					
					Caso $x <0:$
					$$
					\begin{aligned}
					I_{2}(x) 
					= & \int_{\mathbb{R}}\sin(|y|)e^{-|x-y|} dy \\
					= & \int_{y \leq x}\sin(|y|)e^{-|x-y|} dy + \int_{y \geq x}\sin(|y|)e^{-|x-y|} dy\\
					= & \int_{y \leq x}\sin(|y|)e^{(y+x)} dy + \int_{x \leq y \leq 0}\sin(|y|)e^{(y+x)} dy + \int_{y \geq 0 }\sin(|y|)e^{-y+x} dy \\
					= & \int_{y \leq 0}\sin(|y|)e^{(y+x)} dy + \int_{y \geq 0 }\sin(|y|)e^{-y+x} dy \\
					= & \int_{y \leq 0} -\sin(y) e^{(y+x)} dy + \int_{y \geq 0 }\sin(y)e^{-y+x} dy \\
					= & \Re \Big( \int_{y \leq 0} -\frac{e^{iy} - e^{-iy}}{2i} e^{(y+x)} dy + \int_{y \geq 0 } \frac{e^{iy} - e^{-iy}}{2i} e^{-y+x} dy \Big)\\
					= & \Re \Big( \frac{e^{x}}{2i}\int_{y \leq 0} -e^{(1+i)y} + e^{(1-i)y} dy + \frac{e^{x}}{2i}\int_{y \geq 0 } e^{(i-1)y} - e^{-(1+i)y} dy \Big) \\
					= & \Re \Big( \frac{e^{x}}{2i} \Big( \frac{-e^{(1+i)y}}{1+i} + \frac{e^{(1-i)y}}{1-i} \Big) \Big|^{0}_{-\infty} + \frac{e^{x}}{2i} \Big( \frac{e^{(i-1)y}}{i-1} + \frac{e^{-(1+i)y}}{1+i} \Big) \Big|^{\infty}_{0} \Big) \\
					= & \Re \Big( \frac{e^{x}}{2i} \Big( \frac{-1}{1+i} + \frac{1}{1-i} \Big) + \frac{e^{x}}{2i} \Big( \frac{-1}{i-1} + \frac{-1}{1+i} \Big) \Big) \\
					= & \Re \Big( \frac{e^{x}}{2i} \Big( \frac{-2}{1+i} \Big) \Big) \\
					= & \Re \Big( -\frac{e^{x}}{2} \Big( \frac{2}{i-1} \frac{i+1}{i+1}  \Big) = -\frac{e^{x}}{2} \Big( \frac{2(i+1)}{-2} \Big) \Big) \\
					 = & \Re \Big( \frac{e^{x}}{2} (i+1) \Big) \\
					 = & \frac{e^{x}}{2}. \\
					\end{aligned}
					$$
					
					Portanto:
					$$
					I_{2}(x) = \left\{
					\begin{array}{cc}
					e^{-x} -\cos(x) &, x \geq 0 \\
					e^{x}/2  &, x < 0
					\end{array}
					\right.
					$$
					\item Calculando $I_{3}(x).$
					
					Caso $x \geq 0:$
					$$
					\begin{aligned}
					I_{3}(x) 
					= & \int_{\mathbb{R}} y e^{-|x-y|} dy \\
					= & \int_{y \leq x} y e^{-|x-y|} dy + \int_{y \geq x} y e^{-|x-y|} dy \\
					= & \int_{y \leq 0} y e^{-x+y} dy + \int_{0 \leq y \leq x} y e^{-x+y} dy + \int_{y \geq x} y e^{-y + x} dy \\
					= & \int_{y \leq x} y e^{-x+y} dy + \int_{y \geq x} y e^{-y + x} dy \\		
					= & e^{-x}\int_{y \leq x} y e^{y} dy + e^{x}\int_{y \geq x} y e^{-y } dy \\	
					= & e^{-x}\int_{y \leq x} y e^{y} dy + e^{x}\int_{y \geq x} y e^{-y } dy \\	
					= & e^{-x} \frac{ye^{y}}{ln(ye)} \Big|^{x}_{-\infty}+ e^{x} \frac{-ye^{-y}}{ln(ye)} \Big|^{\infty}_{x} \\
					= & e^{-x} \frac{ye^{y}}{ln(ye)} \Big|^{x}_{-\infty}+ e^{x} \frac{-ye^{-y}}{ln(ye)} \Big|^{\infty}_{x} \\
					\end{aligned}
					$$
					\end{enumerate}
				
					$$
					\begin{aligned}
				I_{3}(x)
						= & \int_{\mathbb{R}} y e^{-|x-y|} dy = \int_{y \leq 0 }ye^{-(x-y)} dy + \int_{y \geq 0 }ye^{-(x+y)} dy \\
						= & \int_{y \geq 0 }-ye^{-(x+y)} dy + \int_{y \geq 0 }ye^{-(x+y)} dy \\
						= & 0.
					\end{aligned}
					$$
				\end{enumerate}
			\end{enumerate}
	\end{enumerate}
		
\end{document}