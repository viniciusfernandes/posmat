\documentclass{article}
\usepackage{graphicx}
\usepackage{indentfirst}
\usepackage[utf8]{inputenc}
\usepackage{amssymb}
\usepackage{enumitem}
\usepackage{color}
\usepackage[fleqn]{amsmath}
\usepackage[a4paper, margin=0.5in]{geometry}
\begin{document}
	
	\title{Lista 8 para Entregar}
	\author{Vinicius Fernandes}
	
	\maketitle
	
	\begin{enumerate}
		
		
		\item \textbf{Solução:}
		
		\item \textbf{Solução:} Seja $f \in \mathcal{S}(\mathbb{R}^{n})$, e por definição temos $\mathcal{F}(f)(k) = \int_{\mathbb{R}^{n}} f(x)e^{-ik.x}$, então:
			\begin{enumerate}
				\item 
				$$
				\begin{aligned}
				-i\mathcal{F}(\partial_{i}f)(k) = & \frac{-i}{(2\pi)^{n/2}}\int_{\mathbb{R}^{n}} \partial_{i}f(x)e^{-ik.x}
				\\
				= & \frac{-i}{(2\pi)^{n/2}}\int_{\mathbb{R}^{n}} \partial_{i}( f(x)e^{-ik.x} ) - f(x)\partial_{i}(e^{-ik.x})
				\\
				= & \frac{-i}{(2\pi)^{n/2}}\int_{\mathbb{R}^{n-1}} \lim_{x_{i} \to \infty} f(x)e^{-ik.x} \Big|^{x_{i}}_{-x_{i}} - \frac{-i}{(2\pi)^{n/2}}\int_{\mathbb{R}^{n}}  f(x) (-ik_{j}e^{-ik.x})
				\\
				= & \frac{-i^{2}k_{j}}{(2\pi)^{n/2}}\int_{\mathbb{R}^{n}}  f(x) e^{-ik.x}
				\\
				= & k_{j}\hat{f}(k).	
				\end{aligned}
				$$
				
				\item
				$$
				\begin{aligned}
				-i\mathcal{F}(x_{i}f)(k) = & \frac{-i}{(2\pi)^{n/2}}\int_{\mathbb{R}^{n}} x_{i}f(x)e^{-ik.x}
				\\
				= & \frac{-i}{(2\pi)^{n/2}}\int_{\mathbb{R}^{n}} i\frac{\partial}{\partial k_{i}}( f(x)e^{-ik.x} )
				\\
				= & \frac{-i^{2}}{(2\pi)^{n/2}} \frac{\partial}{\partial k_{i}}\int_{\mathbb{R}^{n}} f(x)e^{-ik.x}
				\\
				= & \frac{\partial}{\partial k_{i}}\hat{f}(k)
				\end{aligned}
				$$
			\end{enumerate}
		
		\item \textbf{Solução:} Primeiro, vejamos o resultado da integral $I = \int_{a}^{b} e^{-x^{2}}$, e fazendo 
		$$
		\begin{aligned}
		I^{2} = & \int_{a}^{b} e^{-x^{2}}\int_{a}^{b} e^{-y^{2}} = \int_{a}^{b} \int_{a}^{b} e^{-(x^{2}+y^{2})} 
		\\
		= & \int_{r_{a}}^{r_{b}} \int_{0}^{\pi} \underbrace{ e^{-r^{2}}r dr d\theta }_{coord. polar} = 2\pi \int_{r_{a}}^{r_{b}} e^{-r^{2}}r dr
		\\
		= & \pi (e^{-r_{a}^{2}} - e^{-r_{b}^{2}}),
		\\
		\therefore I(r_{a}, r_{b}) = & \sqrt{\pi (e^{-r_{a}^{2}} - e^{-r_{b}^{2}})},
		\end{aligned}
		$$
		onde $r_{a}$ e $r_{b}$ serão dados quando $x=y=a$ e $x=y=b$ em  $r^{2} = x^{2}+y^{2}$, respectivamente, isto é, $r_{a} = \sqrt{2}a, r_{b} = \sqrt{2}b$. Vamos utilizar esse resultado nos cáculos adiante.
			\begin{enumerate}
				\item
					$$
					\begin{aligned}
					(\phi_{n}*f)(y) = &\int_{\mathbb{R}} \phi_{n}(x)f(x-y)
					\\
					= &\int_{\mathbb{R}} \frac{n}{\sqrt{\pi}}e^{-(nx)^{2}} e^{-|x-y|}
					\\
					= & \frac{n}{\sqrt{\pi}}\int_{x \leq y} e^{-(nx)^{2}} e^{-(y-x)} + \frac{n}{\sqrt{\pi}}\int_{x \geq y} e^{-(nx)^{2}} e^{-(x-y)}
					\\
					= & \frac{ne^{-y}}{\sqrt{\pi}}\int_{x \leq y} e^{-(nx)^{2}} e^{x} + \frac{ne^{y}}{\sqrt{\pi}}\int_{x \geq y} e^{-(nx)^{2}} e^{-x}
					\\
					= & \frac{ne^{-y}}{\sqrt{\pi}}\int_{x \leq y} e^{-((nx)^{2} -x)}  + \frac{ne^{y}}{\sqrt{\pi}}\int_{x \geq y} e^{-((nx)^{2}+x)}
					\\
					= & \frac{ne^{-y}}{\sqrt{\pi}}\int_{x \leq y} e^{-(nx - 1/2n)^{2} + (1/2n)^{2}} + \frac{ne^{y}}{\sqrt{\pi}}\int_{x \geq y} e^{-(nx + 1/2n)^{2} + (1/2n)^{2}}
					\\
					= & \frac{ne^{-y}e^{ (1/2n)^{2}}}{\sqrt{\pi}}\int_{x \leq y} e^{-(nx - 1/2n)^{2}} + \frac{ne^{y}e^{ (1/2n)^{2}}}{\sqrt{\pi}}\int_{x \geq y} e^{-(nx + 1/2n)^{2}}
					\\
					= & \frac{ne^{-y}e^{(1/2n)^{2}}}{\sqrt{\pi}} \int_{-\infty}^{ny-1/2n} e^{-z^{2}}\frac{1}{n} + \frac{ne^{y}e^{ (1/2n)^{2}}}{\sqrt{\pi}}\int_{ny+1/2n}^{\infty} e^{-z^{2}}\frac{1}{n}
					\\
					= & \frac{e^{-y}e^{(1/2n)^{2}}}{\sqrt{\pi}} \sqrt{\pi e^{-r_{1}^{2}}} + \frac{e^{y}e^{ - (1/2n)^{2}}}{\sqrt{\pi}}\sqrt{\pi e^{-r_{2}^{2}}}
					\\
					= & e^{-y}e^{(1/2n)^{2}} e^{-r_{1}^{2}/2}+ e^{y}e^{(1/2n)^{2}} e^{-r_{2}^{2}/2}
					\\
					= & e^{-y}e^{(1/2n)^{2}} e^{-(ny-1/2n)^{2}}+ e^{y}e^{ (1/2n)^{2}} e^{-(ny+1/2n)^{2}}
					\\
					= & e^{-y}e^{(1/2n)^{2}} e^{-(ny)^{2}+y - (1/2n)^{2}}+ e^{y}e^{ (1/2n)^{2}}e^{-(ny)^{2}-y - (1/2n)^{2}}
					\\
					= &  e^{-(ny)^{2}}+ e^{-(ny)^{2}}
					\\
					= & 2e^{-(ny)^{2}}
					\\
					= & \psi_{n}(y),
					\end{aligned}
					$$
					logo $\lim_{n \to \infty} \psi_{n}(x) = 0$.
					
					\item
						$$
						\begin{aligned}
						\mathcal{F}(\phi_{n})(k) = & \frac{1}{(2\pi)^{1/2}}\int_{\mathbb{R}} \phi_{n}(x)e^{-ikx}
						\\
						 = & \frac{1}{(2\pi)^{1/2}}\int_{\mathbb{R}} \frac{n}{\sqrt{\pi}}e^{-(nx)^{2}} e^{-ikx}
						 \\
						 = & \frac{1}{(2\pi)^{1/2}}\int_{\mathbb{R}} \frac{n}{\sqrt{\pi}}e^{-(nx + ik/2n)^{2} + (ik/2n)^{2}}
						 \\
						 = & \frac{e^{-(ik/2n)^{2}}}{(2\pi)^{1/2}} \frac{n}{\sqrt{\pi}} \int_{\mathbb{R}} e^{-(nx + ik/2n)^{2}}
						 \\
						 = & \frac{e^{(ik/2n)^{2}}}{(2\pi)^{1/2}} \frac{n}{\sqrt{\pi}} \int_{\mathbb{R}} e^{-z^{2}} \frac{1}{n}
						 \\
						 = & \frac{e^{-(k/2n)^{2}}}{(2\pi)^{1/2}} \frac{1}{\sqrt{\pi}} \sqrt{\pi}
						 \\
						 = & \frac{e^{-(k/2n)^{2}}}{(2\pi)^{1/2}}
						 \\
						 = & \hat{\phi}_{n}(k).
						\end{aligned}
						$$
					\item 
					$$
					\begin{aligned}
					\mathcal{F}(\psi_{n})(k) = & \frac{1}{(2\pi)^{1/2}}\int_{\mathbb{R}} \psi_{n}(x)e^{-ikx}
					\\
					= & \frac{1}{(2\pi)^{1/2}}\int_{\mathbb{R}}  2 e^{-(nx)^{2}}e^{-ikx}
					\\
					= & \frac{2}{(2\pi)^{1/2}}\int_{\mathbb{R}} e^{-(nx)^{2} -ikx }
					\\
					= & \frac{2}{(2\pi)^{1/2}}\int_{\mathbb{R}} e^{-(nx + ik/2n)^{2} + (ik/2n)^{2}}
					\\
					= & \frac{2e^{(ik/2n)^{2}} }{(2\pi)^{1/2}}\int_{\mathbb{R}} e^{-(nx + ik/2n)^{2} }
					\\
					= & \frac{2e^{-(k/2n)^{2}} }{(2\pi)^{1/2}}\int_{\mathbb{R}} e^{-z^{2} }
					\\
					= & \frac{2 e^{-(k/2n)^{2}} }{(2\pi)^{1/2}} \sqrt{\pi}
					\\
					= & \sqrt{2} e^{-(k/2n)^{2}}
					\\
					= & \hat{\psi}_{n}(k). 
					\end{aligned}
					$$
			\end{enumerate}
	\end{enumerate}
	
	
\end{document}