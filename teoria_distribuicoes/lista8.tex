\documentclass{article}
\usepackage{graphicx}
\usepackage{indentfirst}
\usepackage[utf8]{inputenc}
\usepackage{amssymb}
\usepackage{enumitem}
\usepackage{color}
\usepackage[fleqn]{amsmath}
\usepackage[a4paper, margin=0.5in]{geometry}
\begin{document}
	
	\title{Lista 8 para Entregar}
	\author{Vinicius Fernandes}
	
	\maketitle
	
	\begin{enumerate}
		
		\item \textbf{Solução:}
			\begin{enumerate}
				\item 
				Vejamos primeiro que o produto de convolução $* : L^{1}(\mathbb{R}^{n}) \times L^{1}(\mathbb{R}^{n}) \to L^{1}(\mathbb{R}^{n})$ esta bem-definida:
				$$
				\begin{aligned}
				||f*g||_{1}=& \sup_{y \in \mathbb{R}^{n}}|(f*g)(y)|
				\\
				= &\sup_{y \in \mathbb{R}^{n}}|\int_{\mathbb{R}^{n} }f(x)g(y-x)| 
				\\
				\leq & \sup_{y \in \mathbb{R}^{n}}\int_{\mathbb{R}^{n} }|f(x)||g(y-x)|
				\\
				\leq & \int_{\mathbb{R}^{n} }|f(x)| \underbrace{ \sup_{y \in \mathbb{R}^{n}}|g(y-x)| }_{g \; limitada}
				\\
				= & \underbrace{ \int_{\mathbb{R}^{n} }|f(x)| }_{f \in L^{1}}||g||_{\infty}
				\\
				= & ||f||_{1}||g||_{\infty}
				\\
				\leq & \underbrace{ ||f||_{1} }_{<\infty} \underbrace{ ||g||_{1} }_{<\infty} < \infty,
				\end{aligned} 
				$$
				portanto $||f*g||_{1} < \infty \Rightarrow f*g \in L^{1}(\mathbb{R}^{n})$ mostrando que operação de convolução esta bem-definida, e temos a desigualdade 
				$$
				||f*g||_{1}\leq ||f||_{1}||g||_{1}.
				$$
				
					\begin{enumerate}
						\item 
						
						$$
						\begin{aligned}
						(f*g)(y)
						= & \int_{\mathbb{R}^{n}}f(x)f(y-x)dx
						\\
						= & \underbrace{ \int_{-\infty}^{\infty}...\int_{-\infty}^{\infty} }_{n-vezes} f(x)g(\underbrace{ y-x }_{z=y-x})dx, \; \text{onde} \; \lim_{x \to \pm \infty} z(x) = \mp \infty
						\\
						=&  \int_{\infty}^{-\infty}...\int_{\infty}^{-\infty} f(x-z)g(z) \underbrace{ (-1)^{n}dz }_{dx}
						\\
						=&  \underbrace{(-1)^{n}  \int_{-\infty}^{\infty}...\int_{-\infty}^{\infty} }_{invert. \; integr.}f(x-z)g(z) (-1)^{n}dz
						\\
						= & \underbrace{ (-1)^{2n} }_{=1} \int_{\mathbb{R}^{n}}g(z)f(x-z)dz
						\\
						= & (g*f)(y)
						\\
						\therefore f*g = & f*g
						\end{aligned}
						$$
						
						\item
						$$
						\begin{aligned}
						\lambda(\varphi*\psi)(y) = & \lambda \int_{\mathbb{R}^{n}}\varphi(x)\psi(y-x)dx = \int_{\mathbb{R}^{n}}\lambda\varphi(x)\psi(y-x)dx
						\\
						=& \int_{\mathbb{R}^{n}}(\lambda\varphi)(x)\psi(y-x)dx
						\\
						= & ((\lambda \varphi)*\psi)(y)
						\\
						\therefore \lambda( \varphi*\psi) = & (\lambda \varphi)*\psi
						\end{aligned}
						$$
						E também
						$$
						\begin{aligned}
						\lambda(\varphi*\psi)(y) = & \lambda \int_{\mathbb{R}^{n}}\varphi(x)\psi(y-x)dx = \int_{\mathbb{R}^{n}}\varphi(x)\lambda\psi(y-x)dx
						\\
						=& \int_{\mathbb{R}^{n}}\varphi(x)(\lambda\psi)(y-x)dx
						\\
						= & (\varphi*(\lambda\psi))(y)
						\\
						\therefore \lambda( \varphi*\psi) = & \varphi*(\lambda \psi),
						\end{aligned}
						$$
						logo $\lambda( \varphi*\psi)=(\lambda \varphi)*\psi=\varphi*(\lambda \psi)$.
						
						\item 
							$$
							\begin{aligned}
							((f*g)*h)(y) = & \int_{\mathbb{R}^{n}}(f*g)(x)h(y-x)dx =  \int_{\mathbb{R}^{n}}\int_{\mathbb{R}^{n}}f(z)g(x-z)dzh(y-x)dx
							\\
							= & \int_{\mathbb{R}^{n}}\int_{\mathbb{R}^{n}}f(z)g(x-z)h(y-x)dzdx
							\\
							= & \int_{\mathbb{R}^{n}}f(z)\int_{\mathbb{R}^{n}}g(u)h(y-z-u)dudz
							\\
							= & \int_{\mathbb{R}^{n}}f(z)(g*h)(y-z)dz
							\\
							= & 
							(f*(g*h))(y)
							\\
							\therefore ((f*g)*h) = & (f*(g*h)).
							\end{aligned}
							$$
							
							\item
							$$
							\begin{aligned}
							((f+g)*h)(y) = & \int_{\mathbb{R}^{n}}(f+g)(x)h(y-x)dx 
							\\
							= & \int_{\mathbb{R}^{n}}(f(x)+g(x))h(y-x)dx
							\\
							= & \int_{\mathbb{R}^{n}}f(x)h(y-x) + g(x)h(y-x)dx
							\\
							= & \int_{\mathbb{R}^{n}}f(x)h(y-x) + \int_{\mathbb{R}^{n}}g(x)h(y-x)dx
							\\
							= & (f*h)(y) + (g*h)(y)
							\\
							\therefore (f+g)*h = & f*h + g*h.
							\end{aligned}
							$$
							
					Enfim, vimos que a operação de convolução $* : L^{1}(\mathbb{R}^{n}) \times L^{1}(\mathbb{R}^{n}) \to L^{1}(\mathbb{R}^{n})$ esta bem-definida, é comutativa e associativa, portanto $(L^{1}(\mathbb{R}^{n}), *)$ forma uma álgebra comutativa e associativa. Além disso, $||f*g||_{1} \leq ||f||_{1}||g||_{1}$, o que implica que $(L^{1}(\mathbb{R}^{n}), *)$ é uma álgebra de Banach.
					\end{enumerate}
				\item 
				$$
				\begin{aligned}
				\tau_{a}(\varphi*\psi)(y)= & (\varphi*\psi)(y+a) 
				\\
				= & \int_{\mathbb{R}^{n}}\varphi(x)\psi(y+a-x)dx
				\\
				= & \int_{\mathbb{R}^{n}}\varphi(x)(\tau_{a}\psi)(y-x)dx
				\\
				= & (\varphi*(\tau_{a}\psi))(y)
				\\
				\therefore \tau_{a}(\varphi*\psi) = & \varphi*(\tau_{a}\psi).
				\end{aligned}
				$$
				E também
				$$
				\begin{aligned}
				\tau_{a}(\varphi*\psi)(y)= & (\varphi*\psi)(y+a) 
				\\
				= & \int_{\mathbb{R}^{n}}\varphi(x)\psi(y+a-x)dx
				\\
				= & \int_{\mathbb{R}^{n}}\varphi(x)\psi(y-(x-a))dx
				\\
				= & \int_{\mathbb{R}^{n}}\varphi(a+z)\psi(y-z)dz
				\\
				= & \int_{\mathbb{R}^{n}} (\tau_{a}\varphi)(z)\psi(y-z)dz
				\\
				= & ((\tau_{a}\varphi)*\psi))(y)
				\\
				\therefore \tau_{a}(\varphi*\psi) = & (\tau_{a}\varphi)*\psi,
				\end{aligned}
				$$
				logo $\tau_{a}(\varphi*\psi) = (\tau_{a}\varphi)*\psi = \varphi*(\tau_{a}\psi)$.
				
				\item
				$$
				\begin{aligned}
				\int_{\mathbb{R}^{n}} (R\tau_{x}f)(y)g(y)dy = & \int_{\mathbb{R}^{n}} (Rf)(y+x)g(y)dy
				\\ = & \int_{\mathbb{R}^{n}} f(-(y+x))g(y)dy
				\\ = & \int_{\mathbb{R}^{n}} f(z)g(x-z)(-1)dz
				\\ = & \int_{\mathbb{R}^{n}} f(z)g(x-z)dz
				\\ = & (f*g)(x).
				\end{aligned}
				$$
			\end{enumerate}
		\item \textbf{Solução:} Seja $f \in \mathcal{S}(\mathbb{R}^{n})$ e por definição temos $\mathcal{F}(f)(k) = \int_{\mathbb{R}^{n}} f(x)e^{-ik.x}$, então escrevendo $\partial_{i} = \partial/\partial x_{i}$ teremos:
			\begin{enumerate}
				\item 
				$$
				\begin{aligned}
				-i\mathcal{F}(\partial_{i}f)(k) = & \frac{-i}{(2\pi)^{n/2}}\int_{\mathbb{R}^{n}} \partial_{i}f(x)e^{-ik.x}
				\\
				= & \frac{-i}{(2\pi)^{n/2}}\int_{\mathbb{R}^{n}} \partial_{i}( f(x)e^{-ik.x} ) - f(x)\partial_{i}(e^{-ik.x})
				\\
				= & \frac{-i}{(2\pi)^{n/2}}\int_{\mathbb{R}^{n-1}} \underbrace{ \lim_{x_{i} \to \infty} f(x)e^{-ik.x} \Big|^{x_{i}}_{-x_{i}} }_{f \; supp. \; compact.} - \frac{-i}{(2\pi)^{n/2}}\int_{\mathbb{R}^{n}}  f(x) (-ik_{j}e^{-ik.x})
				\\
				= & \frac{-i^{2}k_{j}}{(2\pi)^{n/2}}\int_{\mathbb{R}^{n}}  f(x) e^{-ik.x}
				\\
				= & k_{j}\hat{f}(k).	
				\end{aligned}
				$$
				
				\item
				$$
				\begin{aligned}
				-i\mathcal{F}(x_{i}f)(k) = & \frac{-i}{(2\pi)^{n/2}}\int_{\mathbb{R}^{n}} x_{i}f(x)e^{-ik.x}
				\\
				= & \frac{-i}{(2\pi)^{n/2}}\int_{\mathbb{R}^{n}} i\frac{\partial}{\partial k_{i}}( f(x)e^{-ik.x} )
				\\
				= & \frac{-i^{2}}{(2\pi)^{n/2}} \frac{\partial}{\partial k_{i}}\int_{\mathbb{R}^{n}} f(x)e^{-ik.x}
				\\
				= & \frac{\partial}{\partial k_{i}}\hat{f}(k)
				\end{aligned}
				$$
			\end{enumerate}
		
		\item \textbf{Solução:} Vejamos o resultado da integral $I = \int_{a}^{b} e^{-x^{2}}$ calculando primeiro:
		$$
		\begin{aligned}
		I^{2} = & \int_{a}^{b} e^{-x^{2}}\int_{a}^{b} e^{-y^{2}} = \int_{a}^{b} \int_{a}^{b} e^{-(x^{2}+y^{2})} 
		\\
		= & \int_{r_{a}}^{r_{b}} \int_{0}^{\pi} \underbrace{ e^{-r^{2}}r dr d\theta }_{coord. polar} = 2\pi \int_{r_{a}}^{r_{b}} e^{-r^{2}}r dr
		\\
		= & \pi (e^{-r_{a}^{2}} - e^{-r_{b}^{2}}),
		\\
		\therefore I(r_{a}, r_{b}) = & \sqrt{\pi (e^{-r_{a}^{2}} - e^{-r_{b}^{2}})},
		\\
		I(r_{a}) = & \lim_{r_{b} \to \infty} I(r_{a}, r_{b}) = \sqrt{\pi} e^{-r_{a}^{2}/2}
		\end{aligned}
		$$
		onde $r_{a}$ e $r_{b}$ serão dados quando $x=y=a$ e $x=y=b$ em  $r^{2} = x^{2}+y^{2}$, respectivamente, isto é, $r_{a} = \sqrt{2}a, r_{b} = \sqrt{2}b$, portanto $I(r_{a}) = \sqrt{\pi}e^{-a^{2}}$. Vamos utilizar esse resultado nos cáculos adiante.
			\begin{enumerate}
				\item
					$$
					\begin{aligned}
					(\phi_{n}*f)(y) = &\int_{\mathbb{R}} \phi_{n}(x)f(x-y)
					\\
					= &\int_{\mathbb{R}} \frac{n}{\sqrt{\pi}}e^{-(nx)^{2}} e^{-|x-y|}
					\\
					= & \frac{n}{\sqrt{\pi}}\int_{x \leq y} e^{-(nx)^{2}} e^{-(y-x)} + \frac{n}{\sqrt{\pi}}\int_{x \geq y} e^{-(nx)^{2}} e^{-(x-y)}
					\\
					= & \frac{ne^{-y}}{\sqrt{\pi}}\int_{x \leq y} e^{-(nx)^{2}} e^{x} + \frac{ne^{y}}{\sqrt{\pi}}\int_{x \geq y} e^{-(nx)^{2}} e^{-x}
					\\
					= & \frac{ne^{-y}}{\sqrt{\pi}}\int_{x \leq y} e^{-((nx)^{2} -x)}  + \frac{ne^{y}}{\sqrt{\pi}}\int_{x \geq y} e^{-((nx)^{2}+x)}
					\\
					= & \frac{ne^{-y}}{\sqrt{\pi}}\int_{x \leq y} \underbrace{ e^{-(nx - 1/2n)^{2} + (1/2n)^{2} }}_{complet. \; quadrado} + \frac{ne^{y}}{\sqrt{\pi}}\int_{x \geq y} \underbrace{ e^{-(nx + 1/2n)^{2} + (1/2n)^{2}} }_{complet.\;quadrado}
					\\
					= & \frac{ne^{-y}e^{ (1/2n)^{2}}}{\sqrt{\pi}}\int_{x \leq y} e^{-(nx - 1/2n)^{2}} + \frac{ne^{y}e^{ (1/2n)^{2}}}{\sqrt{\pi}}\int_{x \geq y} e^{-(nx + 1/2n)^{2}}
					\\
					= & \frac{ne^{-y}e^{(1/2n)^{2}}}{\sqrt{\pi}} \underbrace{ \int_{-\infty}^{ny-1/2n} e^{-z^{2}} }_{I(r_{1})} \frac{1}{n} + \frac{ne^{y}e^{ (1/2n)^{2}}}{\sqrt{\pi}}
					\underbrace{ \int_{ny+1/2n}^{\infty} e^{-z^{2}} }_{I(r_{2})} \frac{1}{n}
					\\
					\text{onde} \;& r_{1}=\sqrt{2}(ny-1/2n) \; \text{e} \; r_{2}=\sqrt{2}(ny+1/2n)
					\\
					= & \frac{e^{-y}e^{(1/2n)^{2}}}{\sqrt{\pi}} I(\sqrt{2}(ny-1/2n)) + \frac{e^{y}e^{ - (1/2n)^{2}}}{\sqrt{\pi}}I( \sqrt{2}(ny+1/2n))
					\\
					= & \frac{e^{-y}e^{(1/2n)^{2}}}{\sqrt{\pi}} \sqrt{\pi}e^{-(ny-1/2n)^{2}} + \frac{e^{y}e^{ - (1/2n)^{2}}}{\sqrt{\pi}} \sqrt{\pi} e^{-(ny+1/2n)^{2}}
					\\
					= & e^{-y}e^{(1/2n)^{2}} e^{-(ny)^{2}+y - (1/2n)^{2}}+ e^{y}e^{ (1/2n)^{2}}e^{-(ny)^{2}-y - (1/2n)^{2}}
					\\
					= &  e^{-(ny)^{2}}+ e^{-(ny)^{2}}
					\\
					= & 2e^{-(ny)^{2}}
					\\
					= & \psi_{n}(y),
					\end{aligned}
					$$
					logo $\lim_{n \to \infty} \psi_{n}(x) = 0$.
					
					\item
						$$
						\begin{aligned}
						\mathcal{F}(\phi_{n})(k) = & \frac{1}{(2\pi)^{1/2}}\int_{\mathbb{R}} \phi_{n}(x)e^{-ikx}
						\\
						 = & \frac{1}{(2\pi)^{1/2}}\int_{\mathbb{R}} \frac{n}{\sqrt{\pi}}e^{-(nx)^{2}} e^{-ikx}
						 \\
						 = & \frac{1}{(2\pi)^{1/2}}\int_{\mathbb{R}} \frac{n}{\sqrt{\pi}} \underbrace{ e^{-(nx + ik/2n)^{2} + (ik/2n)^{2}} }_{complet.\; quadrado}
						 \\
						 = & \frac{e^{(ik/2n)^{2}}}{(2\pi)^{1/2}} \frac{n}{\sqrt{\pi}} \int_{\mathbb{R}} e^{-(nx + ik/2n)^{2}}
						 \\
						 = & \frac{e^{(ik/2n)^{2}}}{(2\pi)^{1/2}} \frac{n}{\sqrt{\pi}} \underbrace{ \int_{\mathbb{R}} e^{-z^{2}} }_{\sqrt{\pi}} \frac{1}{n}
						 \\
						 = & \frac{e^{-(k/2n)^{2}}}{(2\pi)^{1/2}} \frac{1}{\sqrt{\pi}} \sqrt{\pi}
						 \\
						 = & \frac{e^{-(k/2n)^{2}}}{(2\pi)^{1/2}}
						 \\
						 = & \hat{\phi}_{n}(k).
						\end{aligned}
						$$
					\item 
					$$
					\begin{aligned}
					\mathcal{F}(\psi_{n})(k) = & \frac{1}{(2\pi)^{1/2}}\int_{\mathbb{R}} \psi_{n}(x)e^{-ikx}
					\\
					= & \frac{1}{(2\pi)^{1/2}}\int_{\mathbb{R}}  2 e^{-(nx)^{2}}e^{-ikx}
					\\
					= & \frac{2}{(2\pi)^{1/2}}\int_{\mathbb{R}} e^{-(nx)^{2} -ikx }
					\\
					= & \frac{2}{(2\pi)^{1/2}}\int_{\mathbb{R}} e^{-(nx + ik/2n)^{2} + (ik/2n)^{2}}
					\\
					= & \frac{2e^{(ik/2n)^{2}} }{(2\pi)^{1/2}}\int_{\mathbb{R}} e^{-(nx + ik/2n)^{2} }
					\\
					= & \frac{2e^{-(k/2n)^{2}} }{(2\pi)^{1/2}} \underbrace{ \int_{\mathbb{R}} e^{-z^{2} } }_{\sqrt{\pi}}
					\\
					= & \frac{2 e^{-(k/2n)^{2}} }{(2\pi)^{1/2}} \sqrt{\pi}
					\\
					= & \sqrt{2} e^{-(k/2n)^{2}}
					\\
					= & \hat{\psi}_{n}(k). 
					\end{aligned}
					$$
			\end{enumerate}
	\end{enumerate}
	
	
\end{document}