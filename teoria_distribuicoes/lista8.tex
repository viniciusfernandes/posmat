\documentclass{article}
\usepackage{graphicx}
\usepackage{indentfirst}
\usepackage[utf8]{inputenc}
\usepackage{amssymb}
\usepackage{enumitem}
\usepackage{color}
\usepackage[fleqn]{amsmath}
\usepackage[a4paper, margin=0.5in]{geometry}
\begin{document}
	
	\title{Lista 8 para Entregar}
	\author{Vinicius Fernandes}
	
	\maketitle
	
	\begin{enumerate}
		
		
		\item \textbf{Solução:}
		
		\item \textbf{Solução:} Seja $f \in \mathcal{S}(\mathbb{R}^{n})$, e por definição temos $\mathcal{F}(f)(k) = \int_{\mathbb{R}^{n}} f(x)e^{-ik.x}$, então:
			\begin{enumerate}
				\item 
				$$
				\begin{aligned}
				-i\mathcal{F}(\partial_{i}f)(k) = & \frac{-i}{(2\pi)^{n/2}}\int_{\mathbb{R}^{n}} \partial_{i}f(x)e^{-ik.x}
				\\
				= & \frac{-i}{(2\pi)^{n/2}}\int_{\mathbb{R}^{n}} \partial_{i}( f(x)e^{-ik.x} ) - f(x)\partial_{i}(e^{-ik.x})
				\\
				= & \frac{-i}{(2\pi)^{n/2}}\int_{\mathbb{R}^{n-1}} \lim_{x_{i} \to \infty} f(x)e^{-ik.x} \Big|^{x_{i}}_{-x_{i}} - \frac{-i}{(2\pi)^{n/2}}\int_{\mathbb{R}^{n}}  f(x) (-ik_{j}e^{-ik.x})
				\\
				= & \frac{-i^{2}k_{j}}{(2\pi)^{n/2}}\int_{\mathbb{R}^{n}}  f(x) e^{-ik.x}
				\\
				= & k_{j}\hat{f}(k).	
				\end{aligned}
				$$
				
				\item
				$$
				\begin{aligned}
				-i\mathcal{F}(x_{i}f)(k) = & \frac{-i}{(2\pi)^{n/2}}\int_{\mathbb{R}^{n}} x_{i}f(x)e^{-ik.x}
				\\
				= & \frac{-i}{(2\pi)^{n/2}}\int_{\mathbb{R}^{n}} i\frac{\partial}{\partial k_{i}}( f(x)e^{-ik.x} )
				\\
				= & \frac{-i^{2}}{(2\pi)^{n/2}} \frac{\partial}{\partial k_{i}}\int_{\mathbb{R}^{n}} f(x)e^{-ik.x}
				\\
				= & \frac{\partial}{\partial k_{i}}\hat{f}(k)
				\end{aligned}
				$$
			\end{enumerate}
		
		\item \textbf{Solução:} Primeiro, vejamos o resultado da integral $I = \int_{a}^{b} e^{-x^{2}}$, e fazendo 
		$$
		\begin{aligned}
		I^{2} = & \int_{a}^{b} e^{-x^{2}}\int_{a}^{b} e^{-y^{2}} = \int_{a}^{b} \int_{a}^{b} e^{-(x^{2}+y^{2})} 
		\\
		= & \int_{r_{a}}^{r_{b}} \int_{0}^{\pi} \underbrace{ e^{-r^{2}}r dr d\theta }_{coord. polar} = 2\pi \int_{r_{a}}^{r_{b}} e^{-r^{2}}r dr
		\\
		= & \pi (e^{-r_{a}^{2}} - e^{-r_{b}^{2}}),
		\\
		\therefore I(r_{a}, r_{b}) = & \sqrt{\pi (e^{-r_{a}^{2}} - e^{-r_{b}^{2}})}.
		\end{aligned}
		$$
		Vamos utilizar esse resultado nos cáculos adiante.
			\begin{enumerate}
				\item
					$$
					\begin{aligned}
					(\phi_{n}*f)(x) = &\int_{\mathbb{R}} \phi_{n}(x)f(x-y)
					\\
					= &\int_{\mathbb{R}} \frac{n}{\sqrt{\pi}}e^{-(nx)^{2}} e^{-|x-y|}
					\\
					= & \frac{n}{\sqrt{\pi}}\int_{x \leq y} e^{-(nx)^{2}} e^{-(y-x)} + \frac{n}{\sqrt{\pi}}\int_{x \geq y} e^{-(nx)^{2}} e^{-(x-y)}
					\\
					= & \frac{ne^{-y}}{\sqrt{\pi}}\int_{x \leq y} e^{-(nx)^{2}} e^{x} + \frac{ne^{y}}{\sqrt{\pi}}\int_{x \geq y} e^{-(nx)^{2}} e^{-x}
					\\
					= & \frac{ne^{-y}}{\sqrt{\pi}}\int_{x \leq y} e^{-((nx)^{2} -x)}  + \frac{ne^{y}}{\sqrt{\pi}}\int_{x \geq y} e^{-((nx)^{2}+x)}
					\\
					= & \frac{ne^{-y}}{\sqrt{\pi}}\int_{x \leq y} e^{-(nx - 1/2n)^{2} - (1/2n)^{2}} + \frac{ne^{y}}{\sqrt{\pi}}\int_{x \geq y} e^{-(nx + 1/2n)^{2} - (1/2n)^{2}}
					\\
					= & \frac{ne^{-y}e^{- (1/2n)^{2}}}{\sqrt{\pi}}\int_{x \leq y} e^{-(nx - 1/2n)^{2}} + \frac{ne^{y}e^{ - (1/2n)^{2}}}{\sqrt{\pi}}\int_{x \geq y} e^{-(nx + 1/2n)^{2}}
					\\
					= & \frac{ne^{-y}e^{- (1/2n)^{2}}}{\sqrt{\pi}} \int_{-\infty}^{ny-1/2n} e^{-z^{2}} + \frac{ne^{y}e^{ - (1/2n)^{2}}}{\sqrt{\pi}}\int_{ny+1/2n}^{\infty} e^{-z^{2}}
					\\
					= & \frac{ne^{-y}e^{- (1/2n)^{2}}}{\sqrt{\pi}} \sqrt{\pi e^{-r_{a}^{2}}} + \frac{ne^{y}e^{ - (1/2n)^{2}}}{\sqrt{\pi}}\int_{ny+1/2n}^{\infty} e^{-z^{2}}	
					\end{aligned}
					$$
			\end{enumerate}
	\end{enumerate}
	
	
\end{document}